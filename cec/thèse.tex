\documentclass[12pt,oneside,a4paper]{book}
\usepackage[french]{babel}
\newcommand\fr{\selectlanguage{french}}
%\usepackage{uv}
\usepackage{epigraph}
\usepackage{longtable}
%\usepackage{epipart}
\usepackage{makeidx}
\usepackage{graphicx}
\usepackage{multind}

\makeindex{gnl}
\makeindex{cec}

\addtolength{\hoffset}{-0.5cm}
\addtolength{\textwidth}{1cm}

\pagestyle{plain}

\usepackage[T1]{fontenc}
%\usepackage[latin1]{inputenc}
\newcommand{\textsubring}[1]{#1}
\newcommand{\danda}{|}
\newcommand{\ddanda}{||}
\newcommand{\tdanda}{|||}
\renewcommand{\includegraphics}[1]{}
\linespread{1.3}

\begin{document}
\frontmatter
\thispagestyle{empty}
\begin{center}
{\Large \textsc{\textbf{Université Sorbonne Nouvelle - Paris 3}}}

{\large ED 268 Langages et langues: description, théorisation, transmission}

{\large UFR Langues, littératures, cultures et sociétés étrangères}

{\large UMR 7528 Mondes iranien et indien}

\vspace*{1,5cm}

{\LARGE \textbf{\textsc{C\=\i k\=a\b li: hymnes, héros, histoire.}\\ \large Rayonnement d'un lieu saint shiva\"ite\\
au Pays Tamoul}}

\vspace*{1,5cm}

Thèse de doctorat d'études indiennes présentée par\\
Uthaya \textsc{Veluppillai}

\vspace*{1cm}

Sous la direction de Mme Nalini \textsc{Balbir}

\vspace*{1.5cm}

Soutenue le vendredi 19 avril 2013

\end{center}

\vspace*{2.5cm}


\noindent
\small
Jury:\\
\begin{tabular}{lll}
Mme Nalini \textsc{Balbir}& Professeur d'Université, Paris 3&Directeur de thèse\\
M. Jean-Luc \textsc{Chevillard}& Chargé de Recherche, CNRS&\\
M. Nicolas \textsc{Dejenne}& Maître de Conférences, Paris 3&Président\\
M. Dominic \textsc{Goodall}& Directeur d'\'Etudes, EFEO&\\
Mme Leslie \textsc{C. Orr}& Professeur à l'Université Concordia,& \hspace*{-0.8cm} Montréal, Québec, Canada\\
Mme Charlotte \textsc{Schmid}& Maître de Conférences habilité, EFEO&Rapporteur\\
\end{tabular}

\normalsize
\newpage
\thispagestyle{empty}
\section*{Résumé}

C\=\i k\=a\b li est le site le plus c\'el\'ebr\'e dans le \textit{T\=ev\=aram}, corpus de poèmes de la \textit{bhakti} shivaïte composés en tamoul dans la seconde moitié du premier millénaire: soixante-et-onze hymnes lui sont dédi\'es. Lieu de naissance de Campantar, un des trois auteurs du \textit{T\=ev\=aram}, C\=\i k\=a\b li aurait \'et\'e chant\'e, selon la tradition, sous douze toponymes diff\'erents.

Notre travail de type monographique porte sur l'histoire religieuse du site de C\=\i k\=a\b li qui n'a jamais été étudié alors qu'il représente un haut lieu de la tradition des textes de \textit{bhakti} shivaïte tamoule. Nos sources sont constituées de trois corpus textuels appartenant à trois genres différents de diverses périodes qui permettent de rendre compte du rayonnement continu de ce site:
le corpus du \textit{T\=ev\=aram} sur C\=\i k\=a\b li (partie \textsc{i}), généralement daté des \textsc{vii}\up{e}-\textsc{ix}\up{e} siècles,
le corpus des hagiographies sur Campantar (partie \textsc{ii}) attribuées à des poètes des \textsc{xi}\up{e}-\textsc{xii}\up{e} siècles,
et le corpus des inscriptions du temple de C\=\i k\=a\b li (partie \textsc{iii}) qui forme une documentation inédite du \textsc{xii}\up{e} au \textsc{xvi}\up{e} siècle.

\`A travers une approche \og archéologique\fg\ de ces sources qui permettent de reconstituer, de manière générale, l'histoire du site de C\=\i k\=a\b li, nous proposons une étude historique des textes du \textit{T\=ev\=aram} sur C\=\i k\=a\b li, nous retraçons l'histoire de la légende de l'enfant Campantar et nous éditons le corpus épigraphique de ce temple au rayonnement local.

\vspace*{1cm}
\noindent
Mots-clés: C\=\i k\=a\b li, Campantar, Pays Tamoul, \textit{T\=ev\=aram}, temple, histoire.

\newpage
\thispagestyle{empty}

\begin{center}
\Large{\textbf{{\textsc{C\=\i k\=a\b li: hymns, heroes, history.}\\
Spread of a Shaiva sacred place in Tamilnad}}}
\end{center}

\section*{Abstract}

C\=\i k\=a\b li is the most celebrated temple in the \textit{T\=ev\=aram}, a corpus of Shaiva \textit{bhakti} poems composed in Tamil in the second half of the first millennium: 71 hymns are dedicated to it. The birth place of Campantar, one of the three authors of the \textit{T\=ev\=aram}, C\=\i k\=a\b li has been praised, according to tradition, under 12 names.

Our monographic study deals with the religious history of the C\=\i k\=a\b li temple which has never been studied althought it is a highly traditional place for Tamil \textit{bhakti} texts. Our sources are three corpuses of different genres and periods which highlight the continuous spread of this site:
the \textit{T\=ev\=aram} corpus on C\=\i k\=a\b li (part \textsc{i}), which can be dated in the \textsc{vii}\up{th}-\textsc{ix}\up{th} centuries,
the hagiographical corpus on Campantar (part \textsc{ii}) attributed to poets of the \textsc{xi}\up{th}-\textsc{xii}\up{th} centuries,
and the unpublished epigraphical corpus of the C\=\i k\=a\b li temple (part \textsc{iii}) from the \textsc{xii}\up{th} to the \textsc{xvi}\up{th} century.

On the basis of our archaeological approach of these sources, we reconstruct the history of the C\=\i k\=a\b li temple. Further, we propose a historical study of the \textit{T\=ev\=aram} on C\=\i k\=a\b li, we investigate the history of the child Campantar's legend and we edit the epigraphical corpus of this localy spread site.

\vspace*{1cm}
\noindent
Keywords: C\=\i k\=a\b li, Campantar, Tamilnad, \textit{T\=ev\=aram}, temple, history.

\newpage
\thispagestyle{empty}
\vspace*{3cm}

\begin{flushright}
\`A la mémoire de T. V. Gopal Iyer (1926-2007).
\end{flushright}

\newpage
\thispagestyle{empty}
\section*{Remerciements}

Cette thèse est le fruit d'un travail de recherche effectué avec l'aide et le soutien de plusieurs institutions et de nombreuses personnes que nous tenons à remercier.

L'équipe de recherche Mondes Iranien et Indien (anciennement LACMI) a financé à plusieurs reprises nos déplacements en Inde du Sud. L'\'Ecole française d'Extrême-Orient nous a octroyé trois bourses de voyage successives pour mener nos recherches dans les meilleures conditions au centre de Pondichéry. Nous remercions vivement ces deux institutions.

Nous remercions chaleureusement Nalini \textsc{Balbir}, qui toutes ces années durant, a dirigé notre thèse avec intérêt en nous témoignant une confiance sans faille. Ses encouragements, surtout dans les dernières étapes, ont été d'un grand réconfort.

Nous n'aurions pas pu mener notre thèse à bien sans les lumières du regretté T. V. \textsc{Gopal Iyer}. Son enseignement et son amour du tamoul nous ont profondément marquée. Nous aurions tant voulu qu'il voit l'achèvement de ce travail. Nous espérons qu'il lui fera honneur.
Nous sommes aussi redevable à G. \textsc{Vijayavenugopal} qui nous a transmis sa passion des inscriptions et qui a éclairé, avec enthousiasme, nos textes épigraphiques.

Nous exprimons notre sincère gratitude à tous ceux qui ont lu et commenté une partie ou la totalité de notre étude. Nous voulons témoigner ici de notre vive reconnaissance à Charlotte \textsc{Schmid} dont les lectures attentives et les conseils précieux ont beaucoup amélioré notre travail au fil des années. Nous pensons aussi à Jean-Luc \textsc{Chevillard}, à Emmanuel \textsc{Francis}, à Valérie \textsc{Gillet}, à Dominic \textsc{Goddall}, à Arlo \textsc{Griffiths}, à Karine \textsc{Ladrech}, à Leslie \textsc{Orr}, à Elisabeth \textsc{Sethupathy}, à Dominique \textsc{Soutif} et à Eva \textsc{Wilden}.

Nous remercions profondément les employés du temple de C\=\i k\=a\b li, particulièrement MM. \textsc{Tirugnanam} et \textsc{Senthilkumar}, qui nous ont toujours bien accueillie et dont la coopération a été fondamentale à notre recherche. Nous n'oublions pas le chef du monastère de Tarumapuram qui nous a ouvert toutes les portes des temples qu'il contrôle.

Nous témoignons toute notre gratitude au personnel des bibliothèques: Shanti \textsc{Rayapouillé} (EFEO), Anurupa \textsc{Naik} (IFP), R. \textsc{Narenthiran} (IFP) ainsi qu'à la secrétaire du centre EFEO de Pondichéry, Prerana \textsc{Patel}. Nous n'oublions pas de remercier N. \textsc{Ramaswamy} (Babu), G. \textsc{Ravindran} (Ravi) et Sharif pour leur assistance de lors de nos déplacements sur le terrain.

Enfin, plus personnel, nous pensons à notre famille. Kirupa, Shanti et Tharani ont gardé les enfants pour que nous puissions avancer dans notre étude. Maman et Theepa étaient toujours présentes. Cega était fidèlement à l'écoute. Aadavan, Ilanko et Elilan ont soutenu quotidennement, chacun à leur manière, notre travail.

Merci à tous.

\newpage
%\thispagestyle{empty}
\subsubsection*{Abréviations}

\begin{tabular}{ll}
\textit{APA} &\textit{\=A\d lu\d taiyapi\d l\d laiy\=ar\index{gnl}{Campantar!Alutaiyapillaiyar@\=A\d lu\d taiyapi\d l\d laiy\=ar} tiruvant\=ati}\\
\textit{APCV} &\textit{\=A\d lu\d taiyapi\d l\d laiy\=ar\index{gnl}{Campantar!Alutaiyapillaiyar@\=A\d lu\d taiyapi\d l\d laiy\=ar} tiruca\d npai viruttam}\\
\textit{APK} &\textit{\=A\d lu\d taiyapi\d l\d laiy\=ar\index{gnl}{Campantar!Alutaiyapillaiyar@\=A\d lu\d taiyapi\d l\d laiy\=ar} tirukkalampakam}\\
\textit{APMK} &\textit{\=A\d lu\d taiyapi\d l\d laiy\=ar\index{gnl}{Campantar!Alutaiyapillaiyar@\=A\d lu\d taiyapi\d l\d laiy\=ar} tirumumma\d nikk\=ovai}\\
\textit{APT} &\textit{\=A\d lu\d taiyapi\d l\d laiy\=ar\index{gnl}{Campantar!Alutaiyapillaiyar@\=A\d lu\d taiyapi\d l\d laiy\=ar} tiruttokai}\\
\textit{APUM} &\textit{\=A\d lu\d taiyapi\d l\d laiy\=ar\index{gnl}{Campantar!Alutaiyapillaiyar@\=A\d lu\d taiyapi\d l\d laiy\=ar} tiruvul\=am\=alai}\\
\textit{ARE} &\textit{Annual Report of Epigraphy}\\
\textit{CEC} &\textit{Corpus épigraphique de C\=\i k\=a\b li}\\
\textit{Dar.} &\textit{Darasuram}\\
\textit{EI} & \textit{Epigraphica Indica}\\
\textit{IPS}& \textit{Inscriptions of Pudukottai State}\\
\textit{PI} & \textit{Pondicherry Inscriptions}\\
\textit{PP} & \textit{Periyapur\=a\d nam}\\
\textit{SII} & \textit{South Indian Inscriptions}\\
\textit{SITI} & \textit{South Indian Temple Inscriptions}\\
\textit{TL} & \textit{Tami\b l Lexicon}\\
\textit{TTA} &\textit{Tirutto\d n\d tar tiruvant\=ati}\\
\end{tabular}

\subsubsection*{Note préliminaire}
Nous avons choisi d'adopter une translittération stricte pour tous les toponymes du Pays Tamoul.


Les traductions tamoules données sont les nôtres, sauf mention contraire. Bien qu'elles résultent d'un travail en collaboration avec T. V. \textsc{Gopal Iyer} pour les textes littéraires et avec G. \textsc{Vijayavenugopal} pour les textes épigraphiques, nous demeurons entièrement responsable des erreurs.

Nous avons travaillé avec \LaTeX, un logiciel de composition typographique, pour présenter notre thèse. Nous remercions A. \textsc{Griffiths}, K. \textsc{Harimoto} et J.-J. \textsc{Dh\'enin} qui nous ont aidée dans l'installation et dans l'utilisation de ce logiciel.

\tableofcontents


\mainmatter
\addcontentsline{toc}{chapter}{Introduction}
\chapter*{Introduction}


\scriptsize
\begin{verse}
\textit{k\=atal \=aki, kacintu, ka\d n\d n\=\i r malki,\\
\=otuv\=artamai na\b nne\b rikku uyppatu;\\
v\=etamn\=a\b nki\b num meypporu\d l \=avatu ---\\
n\=ata\b n n\=amam namacciv\=ayav\=e.\\}
(\textit{T\=ev\=aram} III 49.1)
\end{verse}

\normalsize
\begin{verse}
Il mène sur la bonne voie ceux qui chantent\\
Avec amour, fondant en larmes,\\
Il est la vérité essentielle des quatre \textit{Veda},\\
C'est le nom du Seigneur; hommage à \'Siva!\\
(\textit{T\=ev\=aram} III 49.1)
\end{verse}

Le Seigneur invoqué dans ce quatrain est \'Siva\index{gnl}{Siva@\'Siva}. Dieu suprême, il est l'Essence des \textit{Veda}\index{gnl}{Veda@\textit{Veda}}, textes \og révélés\fg\ qui forment les livres canoniques les plus anciens en sanskrit\footnote{En bref, de 1500 à 500 avant J.-C., dans les \textit{Veda}\index{gnl}{Veda@\textit{Veda}}, Rudra\index{gnl}{Rudra}, assimilé à \'Siva\index{gnl}{Siva@\'Siva} par la suite, est un dieu secondaire, terrible et protecteur, qui obtient, entre autres, dès le \textit{\textsubring{R}gveda} l'épithète de \og bienveillant\fg\ (\textit{\'siva}). Ensuite, dans les \textit{Br\=ahma\d na}\index{gnl}{Brahmana@\textit{Br\=ahma\d na}} et les \textit{Upani\d sad}\index{gnl}{Upanisad@\textit{Upani\d sad}}, Rudra\index{gnl}{Rudra} devient une divinité majeure dont les traits essentiels sont développés et figés par les épopées\index{gnl}{epopees@épopées}, les \textit{Pur\=a\d na}\index{gnl}{Purana@\textit{Pur\=a\d na}} et les \textit{\=Agama}\index{gnl}{Agama@\textit{\=Agama}} (voir \textsc{Kramrisch} 1988 et le premier chapitre sur les débuts du shivaïsme dans \textsc{Bhatt} 2000). Le foisonnement et l'enchev\^etrement des récits mythologiques sur \'Siva issus de différentes traditions font écho à des représentations iconographiques variées et complexes (voir, entre autres, \textsc{Rao} (*1997 [1914]), \textsc{Filliozat} 1961, les articles de Marguerite E. \textsc{Adicéam}, \textsc{Sivaramamurthi} (*1994 [1974]), \textsc{Gillet} 2010 et \textsc{Ladrech} 2010).}. La présence de \'Siva dans les \textit{Veda}\index{gnl}{Veda@\textit{Veda}} confère à ce dieu une autorité éternelle, incontestable et panindienne.
%Dans le panthéon hindou, \'Siva\index{gnl}{Siva@\'Siva} occupe une place capitale. Destructeur dans la Triade qu'il forme avec Brahm\=a\index{gnl}{Brahma@Brahm\=a}, le créateur, et Vi\d s\d nu\index{gnl}{Visnu@Vi\d s\d nu}, le \og mainteneur\fg. En contexte sectaire\index{gnl}{sectaire}, il est le dieu absolu. Il est responsable de la création, du maintien et de la destruction.
Ici, \'Siva est aussi un dieu\index{gnl}{dieu} proche, aimé et tamoul. Chanter le nom de \'Siva, corps et âme, est salutaire. La langue de communication est le tamoul. C'est la religion de la \textit{bhakti}\index{gnl}{bhakti@\textit{bhakti}}, une religion du \og c\oe ur\fg\ basée sur la dévotion\index{gnl}{devotion@dévotion} personnelle envers une divinité\index{gnl}{divinité} d'élection et exprimée à travers des textes composés dans un contexte régional, en langues vernaculaires et par des auteurs de différentes couches sociales.
La strophe citée ci-dessus appartient au \textit{T\=ev\=aram}, corpus de poèmes de la \textit{bhakti} shivaïte\index{gnl}{shivaïte} tamoule composés dans la seconde moitié du premier millénaire au Pays Tamoul\index{gnl}{Pays Tamoul}. Elle est attribuée à Campantar\index{gnl}{Campantar}, un des trois auteurs du \textit{T\=ev\=aram}\index{gnl}{Tevaram@\textit{T\=ev\=aram}}.
\og Ceux qui chantent\fg\footnote{Le terme \textit{\=otuv\=ar}\index{gnl}{otuvar@\textit{\=otuv\=ar}} renvoit aujourd'hui au chanteur, non brahmane, de poèmes tamouls shivaïtes qui intervient dans le rituel du temple\index{gnl}{temple} après la \textit{p\=uj\=a}\index{gnl}{puja@\textit{p\=uj\=a}} \=agamique conduite par l'officiant brahmane. Sur cette communauté de chanteurs voir \textsc{Barnoud-Sethupathy} 1994.} sont les dévots shivaïtes qui récitent, encore aujourd'hui, cette strophe et d'autres du \textit{T\=ev\=aram} lors des cultes\index{gnl}{culte} domestiques ou de temples.

Dans les textes tamouls, \'Siva\index{gnl}{Siva@\'Siva} apparaît sporadiquement dès la littérature profane du \textit{Ca\.nkam}\index{gnl}{Cankam@\textit{Ca\.nkam}}\footnote{Cette litt\'erature comporte huit anthologies, dix poème\index{gnl}{poeme@poème}s et une grammaire compos\'es, selon la tradition, par un total de quatre cent soixante-treize auteurs. Elle est une po\'esie profane conventionnelle class\'ee en deux th\`emes (\textit{akam}\index{gnl}{akam@\textit{akam}} et \textit{pu\b ram}\index{gnl}{puram@\textit{pu\b ram}}). L'\textit{akam}\index{gnl}{akam@\textit{akam}} s'organise autour de cinq paysages d\'efinis (\textit{ti\d nai}), symboles chacun d'un \'etat de la relation amoureuse. \`A ces \oe uvres s'ajoutent par la suite les deux \'epopées que sont le \textit{Cilappatik\=aram}\index{gnl}{Cilappatikaram@\textit{Cilappatik\=aram}} et le \textit{Ma\d nim\=ekalai}\index{gnl}{Manimekalai@\textit{Ma\d nim\=ekalai}} ainsi que les \textit{Pati\b ne\d nki\b lkka\d nakku}\index{gnl}{Patinenkilkkanakku@\textit{Pati\b ne\d nki\b lkka\d nakku}} (ensemble de dix-huit textes) dont le célèbre \textit{Tirukku\b ral}\index{gnl}{Tirukkural@\textit{Tirukku\b ra\d l}}.}. De la fin de la période dite du \textit{Ca\.nkam} (\textsc{vi-vii}\up{e} siècles?) jusqu'au \textsc{xii}\up{e} si\`ecle, se d\'eveloppe, s'ordonne puis se cristallise un corpus\index{gnl}{corpus} plut\^ot h\'et\'eroclite dont la compilation\index{gnl}{compilation} et l'agencement r\'esultent d'heureux hasards l\'egendaires et historiques: le \textit{Tirumu\b rai}\index{gnl}{Tirumurai@\textit{Tirumu\b rai}}, \og Canon\index{gnl}{canon} sacr\'e\fg\footnote{Le \textit{Tamil Lexicon} donne seize d\'efinitions distinctes pour le terme \textit{mu\b rai}. \textsc{Rangaswamy} (*1990 [1958]: 1) traduit \textit{Tirumu\b rai}\index{gnl}{Tirumurai@\textit{Tirumu\b rai}} par \og Sacred book\fg, \textsc{Gros} (1984: v) par \og ouvrages sacr\'es\fg\ et \textsc{Zvelebil} (1995: \textit{s.v.}) par \og holy order\fg. Notre traduction du mot \textit{mu\b rai} souhaite conserver un terme singulier qui d\'esigne un ensemble de livres sacr\'es.}. Ce \og canon\index{gnl}{canon}\fg\ est compos\'e de douze\index{gnl}{douze} livres dont les sept premiers regroupent, sous le titre de \textit{T\=ev\=aram}\index{gnl}{Tevaram@\textit{T\=ev\=aram}}, les hymne\index{gnl}{hymne}s attribu\'es \`a trois poète\index{gnl}{poete@poète}s, les \textit{m\=uvar}\index{gnl}{muvar@\textit{m\=uvar}}, qui sont Campantar\index{gnl}{Campantar} (\textit{Tirumu\b rai}\index{gnl}{Tirumurai@\textit{Tirumu\b rai}} \textsc{i} \`a \textsc{iii}), Appar\index{gnl}{Appar} (\textsc{iv} \`a \textsc{vi}) et Cuntarar\index{gnl}{Cuntarar} (\textsc{vii}). Dans chacun de leurs poèmes\index{gnl}{poeme@poème} ces \textit{m\=uvar}\index{gnl}{muvar@\textit{m\=uvar}} ont chant\'e la gloire d'un \'Siva\index{gnl}{Siva@\'Siva} particulier, localis\'e et associ\'e \`a un site concret\footnote{Sur les sept cent quatre-vingt-dix-huit hymne\index{gnl}{hymne}s du \textit{T\=ev\=aram}\index{gnl}{Tevaram@\textit{T\=ev\=aram}} seuls quarante-huit sont \`a caract\`ere g\'en\'eral (\textit{potu}) et ne renvoient \`a aucun site pr\'ecis.}. La concentration de ces lieux saints dans l'ancien Pays des C\=o\b la\index{gnl}{Cola@C\=o\b la}, le delta fertile de la K\=av\=eri\index{gnl}{Kaveri@K\=av\=eri}, a favorisé ensuite la construction d'une g\'eographie sacr\'ee de pèlerinage\index{gnl}{pelerinage@pèlerinage} tamoul\footnote{Cf., entre autres, \textsc{Ve\d l\d laiv\=ara\d na\b n} (*1994 [1962 et 1969]: 883-960), \textsc{Spencer} 1970, \textsc{Veluthat} 1979, \textsc{Peterson} 1982 et \textsc{Chevillard} 2000.}.

Les hymnes du \textit{T\=ev\=aram} sont nés aux \textsc{vii}\up{e}-\textsc{ix}\up{e} siècles dans un contexte de temple\index{gnl}{temple}\footnote{Le temple est un lieu de culte à une divinité présente. Si l'élément divin indispensable est présent tout lieu peut être considéré comme un temple et comme favorable à l'élaboration d'un culte. Aussi, la construction architecturale étant facultative, un arbre ou/et une pierre, parce qu'habités par un dieu, peuvent être des temples.}, demeure de \'Siva, qui leur offre, dès la fin du premier millénaire, un cadre cultuel dans lequel le chant de ces poèmes est institutionnalisé dans le service\index{gnl}{service} divin. Le temple accueille par la suite les images des poètes --- à qui sont attribués ces chants --- qui y sont installées pour être honorées (voir 1.3)\footnote{Les hymnes ne sont pas chantés dans tous les temples célébrés dans le \textit{T\=ev\=aram}. Les temples qui ont intégré le chant des hymnes dans leurs cultes ne sont pas tous célébrés dans le \textit{T\=ev\=aram} (voir \textsc{Orr} 2007).}. Le temple peut aussi servir de lieu de sauvegarde pour ces hymnes\index{gnl}{hymne} du \textit{T\=ev\=aram} grâce à la préservation des manuscrits\index{gnl}{manuscrit} sur lesquels les hymnes sont gravés (voir CEC 26 dans 7.2). Le temple joue donc un rôle primordial dans l'élaboration, la pratique, la transmission et la conservation des hymnes du \textit{T\=ev\=aram} à date ancienne.

Notre étude porte sur un de ces temples shivaïtes du Pays Tamoul qui a été célébré dans le \textit{T\=ev\=aram}, qui a intégré le chant des hymnes du \textit{T\=ev\=aram} dans le culte et qui a incorporé dans son enceinte une chapelle dédiée à Campantar\index{gnl}{Campantar} où étaient préservés à un moment donné les manuscrits d'un corpus compilé d'hymnes shivaïtes: le temple de Brahm\=apur\=\i \'svara à C\=\i k\=a\b li\footnote{Le toponyme de C\=\i k\=a\b li possède plusieurs orthographes. \og Shiyali\fg\ est l'appelation rencontrée dans les relevés de l'ASI en 1896 et en 1918, dans \textsc{Mahalingam} (1992) et dans \textsc{Nagaswamy} (2005). \og C\=\i rk\=a\b li\fg\ correspond au nom actuel de la ville. D'après les données archéologiques disponibles, \og C\=\i k\=a\b li\fg\ semble être la première orthographe attestée de ce toponyme (voir SII 4 133 l. 1 datant de 1116 et \textsc{Karashima, Subbarayalu, Matsui} 1978: 738). Dans notre thèse, nous avons choisi d'écrire ce mot selon cette dernière variante.}.

La ville de C\=\i k\=a\b li\index{gnl}{Cikali@C\=\i k\=a\b li} est plac\'ee au c\oe ur du delta fertile de la K\=av\=eri\index{gnl}{Kaveri@K\=av\=eri}, \`a vingt kilom\`etres au sud de Citamparam\index{gnl}{Citamparam}, et \`a deux cent cinquante kilom\`etres au sud de Ce\b n\b nai\index{gnl}{Cennai@Ce\b n\b nai}. Il s'agit du chef-lieu administratif du taluk du m\^eme nom dans le district de N\=akappa\d t\d ti\d nam\index{gnl}{Nakappattinam@N\=akappa\d t\d ti\d nam} (ant\'erieurement, elle appartenait au district de Ta\~nc\=av\=ur\index{gnl}{Tancavur@Ta\~nc\=av\=ur}). Travers\'ee par l'actuelle route nationale 45 qui relie Ce\b n\b nai \`a N\=akappa\d t\d ti\d nam, la ville est situ\'ee dans une zone de circulation tr\`es importante. Son emplacement sur une terre\index{gnl}{terre} productive et au contact d'autres r\'egions a grandement particip\'e \`a son d\'eveloppement social, \'economique et religieux.

\begin{figure}[!h]
  \centering
 \includegraphics[width=11cm]{docthese/cartethese.JPG}
  \caption{Schéma du Delta de la K\=av\=eri.}
\end{figure}

C\=\i k\=a\b li\index{gnl}{Cikali@C\=\i k\=a\b li} est un des deux cent soixante-seize sites honor\'es dans le \textit{T\=ev\=aram}\index{gnl}{Tevaram@\textit{T\=ev\=aram}}. Il est le plus c\'el\'ebr\'e de tous: soixante-et-onze hymne\index{gnl}{hymne}s lui sont dédi\'es. Lieu de naissance\index{gnl}{naissance} de Campantar\index{gnl}{Campantar}, C\=\i k\=a\b li aurait \'et\'e chant\'e, selon la tradition\index{gnl}{tradition}, sous douze\index{gnl}{douze} toponymes diff\'erents. Campantar\index{gnl}{Campantar}, par son origine\index{gnl}{origine} et ses \oe uvres qui représentent soixante-sept des soixante-et-onze poème\index{gnl}{poeme@poème}s dédiés à C\=\i k\=a\b li (\textsc{vii}\up{e}-\textsc{ix}\up{e} siècles), puis d'autres poète\index{gnl}{poete@poète}s int\'egr\'es au \textit{Tirumu\b rai}\index{gnl}{Tirumurai@\textit{Tirumu\b rai}} (\textsc{x}\up{e}-\textsc{xii}\up{e} siècles) ont particip\'e \`a l'essor l\'egendaire du temple\index{gnl}{temple}. Quant à l'histoire du site, des données épigraphiques gravées sur les murs du temple fournissent une documentation substantielle et inédite pour l'analyser (\textsc{xii}\up{e}-\textsc{xvi}\up{e} siècles).
Autour du temple de C\=\i k\=a\b li gravitent donc trois corpus textuels appartenant à trois genres différents de diverses périodes qui permettent de rendre compte du rayonnement de ce site à date ancienne: le corpus du \textit{T\=ev\=aram} sur C\=\i k\=a\b li (partie \textsc{i}), le corpus des hagiographies sur Campantar (partie \textsc{ii}) et le corpus des inscriptions du temple de C\=\i k\=a\b li (partie \textsc{iii}).

Le temple de C\=\i k\=a\b li n'a jamais été étudié. Nous cherchons donc à reconstituer, à l'aide de ces trois corpus, l'histoire du site de C\=\i k\=a\b li\index{gnl}{Cikali@C\=\i k\=a\b li} et de son poète Campantar.
Des études multidisciplinaires et monographiques ont déjà porté sur quelques grands temples\index{gnl}{temple} shivaïtes du Pays Tamoul\index{gnl}{Pays Tamoul}\footnote{Un projet collectif des institutions de Putucc\=eri\index{gnl}{Putucc\=eri} a abouti aux cinq volumes présentant les inscriptions (\textsc{Srinivasan} \&\ \textsc{Reiniche} 1990), l'archéologie (\textsc{L'Hernault}, \textsc{Pichard} \&\ \textsc{Deloche} 1990), les rites\index{gnl}{rite} et fêtes\index{gnl}{fete@fête} (\textsc{L'Hernault} \&\ \textsc{Reiniche} 1999), la configuration sociologique (\textsc{Reiniche} 1989) et la ville (\textsc{Guilmoto}, \textsc{Reiniche} \&\ \textsc{Pichard} 1990) du site de Tiruva\d n\d n\=amalai\index{gnl}{Tiruvannamalai@Tiruva\d n\d n\=amalai}.}. Certains temples\index{gnl}{temple} royaux de la dynastie \textit{c\=o\b la}\index{gnl}{cola@\textit{c\=o\b la}} (du milieu du \textsc{ix}\up{e} à la fin du \textsc{xiii}\up{e} siècle), remarquables par leurs dimensions, leurs sculptures et leur épigraphie ont été l'objet d'analyses fouillées\footnote{Le temple\index{gnl}{temple} de R\=ajar\=aje\'svara à Ta\~nc\=av\=ur\index{gnl}{Tancavur@Ta\~nc\=av\=ur} fondé par R\=ajar\=aja I\index{gnl}{Rajaraja I@R\=ajar\=aja I} (985-1014) a bénéficié d'une étude architecturale (\textsc{Pichard} 1995). En plus d'un travail détaillé sur l'architecture, la monographie sur le temple\index{gnl}{temple} érigé par R\=ajendra I\index{gnl}{Rajendra I@R\=ajendra I} (1012-1044) à Ka\.nkaiko\d n\d tac\=o\b lapuram\index{gnl}{Kankaikontacolapuram@Ka\.nkaiko\d n\d tac\=o\b lapuram} présente en annexes des études iconographique et épigraphique (\textsc{Pichard} 1994). Et, \textsc{L'Hernault} (1987) propose une étude épigraphique, architecturale et iconographique du temple\index{gnl}{temple} de T\=ar\=acuram\index{gnl}{Taracuram@T\=ar\=acuram} fondé par R\=ajar\=aja II\index{gnl}{Rajaraja II@R\=ajar\=aja II} (1146-1173).}.
Cependant, un accès indirect aux sources primaires, particulièrement aux textes de la littérature tamoule, limite la portée et la pertinence des recherches et peut conduire parfois à des erreurs\footnote{Nous pensons par exemple à l'étude de la frise narrative\index{gnl}{frise narrative} de T\=ar\=acuram\index{gnl}{Taracuram@T\=ar\=acuram} (cf. 4.3.3).}.
La connaissance\index{gnl}{connaissance} du tamoul ne suffit pas, non plus, à présenter un travail rigoureux, précis et détaillé. La plupart des monographies de temples\index{gnl}{temple} shivaïtes du Pays Tamoul\index{gnl}{Pays Tamoul} peinent à se détacher de la tradition\index{gnl}{tradition} qui fige l'histoire de la littérature tamoule\footnote{Cf. par exemple \textsc{Srinivasan} (1979: 11-12), \textsc{Devakunjari} (1979: 88, 99-100) et \textsc{Mookka Reddy} (1986: 18).}. Rechercher la vérité historique armé de poème\index{gnl}{poeme@poème}s dévotionnels\index{gnl}{devotionnel@dévotionnel} (célébrant la grandeur d'un dieu en un site particulier), sans l'aide d'autres sources, nous paraît être une démarche incomplète.

Comment expliquer le peu d'intérêt manifesté par la littérature secondaire pour le temple de C\=\i k\=a\b li?
Le sanctuaire que nous observons aujourd'hui appartient à la période dite \og \textit{c\=o\b la}\index{gnl}{cola@\textit{c\=o\b la}} tardif\fg. Aucun élément de la structure actuelle ne laisse envisager une datation antérieure au \textsc{xii}\up{e} siècle (voir 8.1). Il ne s'agit pas d'une fondation\index{gnl}{fondation} royale et aucun des cinquante-cinq textes épigraphiques du site ne mentionne de dons offerts par la famille royale (voir chapitre 7). Pourtant, le temple de C\=\i k\=a\b li est un \og grand temple\fg\ (\textit{periya k\=oyil}), tel que l'appellent les habitants, et jouit d'une importante notori\'et\'e religieuse. La grandeur de ce temple en activité n'est pas uniquement définie par sa taille ou par la fréquentation des fidèles locaux et des pèlerins mais par la place capitale que ce site occupe dans la tradition\index{gnl}{tradition} des textes tamouls de \textit{bhakti}.
Dans ce travail de type monographique nous avons adopté une approche \og archéologique\fg\ de nos trois corpus textuels qui sont présentés dans l'ordre chronologique. Pour étudier le temple de C\=\i k\=a\b li et son poète Campantar il nous paraît fondamental de scruter les textes tamouls de \textit{bhakti}, du \textsc{vii}\up{e} au \textsc{xii}\up{e} siècle, qui les célèbrent ainsi que leur contexte. Ceci nous conduit parfois à heurter la tradition\index{gnl}{tradition} sur des questions d'interpolation (2.3, 3.3), de chronologie (5.1), de transmission (5.1) et de compilation (6.1). En l'absence d'édition critique, la lecture des poèmes puis un travail de comparaison entre eux permettent d'esquisser une étude historique de ces textes\footnote{Le pr\'esent travail est bas\'e sur l'\'edition \'etablie par T. V. \textsc{Gopal Iyer} pour le \textit{T\=ev\=aram}\index{gnl}{Tevaram@\textit{T\=ev\=aram}}, sur celle de Ci. K\=e. \textsc{Cuppirama\d niya Mutaliy\=ar} pour le \textit{Periyapur\=a\d nam}\index{gnl}{Periyapuranam@\textit{Periyapur\=a\d nam}} et enfin, sur celle du monast\`ere\index{gnl}{monastère} de Tarumapuram\index{gnl}{Tarumapuram} pour les autres volumes du \textit{Tirumu\b rai}\index{gnl}{Tirumurai@\textit{Tirumu\b rai}}.}. Nous étudions le premier corpus de manière indépendante pour éviter de mêler les hymnes aux légendes\index{gnl}{legende@légende} qui leur sont liées par la suite, amalgame fréquent dans la littérature secondaire. La confrontation du deuxième corpus sur les textes légendaires avec les données iconographiques et épigraphiques nous semble cruciale pour l'étude de la formation de la légende de Campantar. Le dernier corpus sur les inscriptions disponibles est le résultat d'un travail d'édition inédit. L'accès direct aux textes épigraphiques du site nous apparaît comme indispensable pour éviter de répéter certaines erreurs véhiculées dans la littérature secondaire et surtout, pour comprendre le rayonnement\index{gnl}{rayonnement} local de ce site.

Cette étude nous laisse entrevoir, entre autres, que le corpus\index{gnl}{corpus} sacré et figé du \textit{T\=ev\=aram}\index{gnl}{Tevaram@\textit{T\=ev\=aram}} comporte des éléments probablement interpolés (partie \textsc{i}), que l'origine\index{gnl}{origine} de la légende\index{gnl}{legende@légende} de l'enfant\index{gnl}{enfant} Campantar\index{gnl}{Campantar} est palpable (partie \textsc{ii}) et que l'éclat d'un site aussi sublimé dans la littérature contraste avec le réel tel qu'il fut gravé sur la pierre (partie \textsc{iii}). C\=\i k\=a\b li\index{gnl}{Cikali@C\=\i k\=a\b li} ou la destinée d'un temple\index{gnl}{temple} vivant.

\begin{figure}[!h]
  \centering
 \includegraphics[height=8cm]{docthese/photoCIIKAALI/sept020}
  \caption{Gopura est, temple de C\=\i k\=a\b li (cliché U. \textsc{Veluppillai}, 2006).}
\end{figure}

\part{Hymnes}

La \textit{bhakti}\index{gnl}{bhakti@\textit{bhakti}} est n\'ee de la volont\'e de rompre le cycle des r\'eincarnations et ainsi, d'ouvrir l'acc\`es \`a la Lib\'eration, th\'eoriquement \`a tous. La notion de \textit{bhakti}\index{gnl}{bhakti@\textit{bhakti}} a suscit\'e diverses \'etudes ces derniers si\`ecles. Le premier chapitre de \textsc{Prentiss} (1999), intitul\'e \og images of bhakti\fg, rend compte de l'histoire de l'\'etude scientifique de cette notion\footnote{Notion qu'elle propose, \`a son tour, de cerner, pour la \textit{bhakti}\index{gnl}{bhakti@\textit{bhakti}} shiva\"ite\index{gnl}{shiva\"ite} en Pays Tamoul\index{gnl}{Pays Tamoul}, dans un contexte d'\'evolution \`a travers ses diverses \og incarnations\fg\ du \textsc{vii}\up{e} et \textsc{xiv}\up{e} si\`ecle.}.

En Pays Tamoul\index{gnl}{Pays Tamoul}, la \textit{bhakti}\index{gnl}{bhakti@\textit{bhakti}} bourgeonne d\`es les derni\`eres couches de la litt\'erature du \textit{Ca\.nkam}\index{gnl}{Cankam@\textit{Ca\.nkam}} (\textsc{Gros} 1968, \textsc{Filliozat} 1973 et \textsc{Zvelebil} 1977). Mais il faut attendre l'expression des mouvements sectaire\index{gnl}{sectaire}s pour voir sa litt\'erature s'\'epanouir compl\`etement. Des hymne\index{gnl}{hymne}s \`a la gloire d'un dieu unique, puissant et parfaitement ancr\'e sur le sol tamoul naissent et composent ce qui sera appel\'e, postérieurement, le \textit{N\=al\=ayirattiviyappirapantam}\index{gnl}{Nalayirattiviya@\textit{N\=al\=ayirattiviyappirapantam}}\footnote{Pour un expos\'e de la \textit{bhakti}\index{gnl}{bhakti@\textit{bhakti}} \'emotionnelle, pour ne pas dire de la s\'eparation amoureuse, en milieu vishnouite\index{gnl}{vishnouite}, cf. \textsc{Hardy} *2001 [1983].} ou le \textit{T\=ev\=aram}\index{gnl}{Tevaram@\textit{T\=ev\=aram}}.



Dans cette partie consacr\'ee aux hymne\index{gnl}{hymne}s sur C\=\i k\=a\b li\index{gnl}{Cikali@C\=\i k\=a\b li} dans le \textit{T\=ev\=aram}\index{gnl}{Tevaram@\textit{T\=ev\=aram}}, nous pr\'esentons de fa\c con g\'en\'erale ce corpus\index{gnl}{corpus} (chapitre 1), qui est notre source principale, pour \'etudier plus particuli\`erement la figure du poète\index{gnl}{poete@poète} Campantar\index{gnl}{Campantar} (chapitre 2) et le site de C\=\i k\=a\b li\index{gnl}{Cikali@C\=\i k\=a\b li}, son lieu de naissance\index{gnl}{naissance} aux douze\index{gnl}{douze} noms (chapitre 3).

\chapter{Le \textit{T\=ev\=aram}}

La br\`eve pr\'esentation qui suit n\'ecessite une lecture pr\'ealable des travaux sur le \textit{T\=ev\=aram}\index{gnl}{Tevaram@\textit{T\=ev\=aram}} qui ont permis de b\^atir une base relativement solide \`a l'\'edification de son \'etude. Nous pensons, de mani\`ere s\'elective, en premier lieu aux \'ecrits sp\'ecialis\'es et \og traditionnels\fg\ de \textsc{Rangaswamy} (*1990 [1958]) et de \textsc{Ve\d l\d laiv\=ara\d na\b n} (*1994 [1962 et 1969]). \textsc{Zvelebil} (1975: 130-151) offre une approche g\'en\'erale de l'histoire de la litt\'erature tamoule de \textit{bhakti}\index{gnl}{bhakti@\textit{bhakti}} shiva\"ite\index{gnl}{shiva\"ite}. \textsc{Gros} (1984) donne une introduction historique au texte tout en consid\'erant les cadres litt\'eraires et l\'egendaires qui le mettent en forme. \textsc{Gopal Iyer} (1991) rassemble quantit\'e de donn\'ees sur le texte: une introduction \'enum\'erant la place du \textit{T\=ev\=aram}\index{gnl}{Tevaram@\textit{T\=ev\=aram}} dans de nombreux textes, des commentaires d'hymne\index{gnl}{hymne}s, une \'etude des sites, un relev\'e des informations mythologiques et litt\'eraires contenues dans le corpus\index{gnl}{corpus} et, enfin, un index des mots rares.

\textsc{Kingsbury \& Phillips} (*2000 [1921]) sont les premiers \`a traduire quelques textes des \textit{n\=alvar}\index{gnl}{nalvar@\textit{n\=alvar}}\footnote{Les \textit{n\=alvar}\index{gnl}{nalvar@\textit{n\=alvar}} renvoient au quatuor form\'e des trois auteurs du \textit{T\=ev\=aram}\index{gnl}{Tevaram@\textit{T\=ev\=aram}} et du poète\index{gnl}{poete@poète} M\=a\d nikkav\=acakar\index{gnl}{Manikkavacakar@M\=a\d nikkav\=acakar}.}. Il faut attendre ensuite \textsc{Peterson} (*1991 [1989]) pour lire de belles traductions de nombreux passages choisis et organis\'es selon de fines analyses th\'ematiques. \textsc{Shulman} (1990), plus exclusif, se consacre aux poème\index{gnl}{poeme@poème}s de Cuntarar\index{gnl}{Cuntarar}. Il offre une traduction compl\`ete\footnote{Signalons toutefois qu'il n'inclut pas dans l'\oe uvre de Cuntarar\index{gnl}{Cuntarar} le cent uni\`eme hymne\index{gnl}{hymne} que pr\'esente l'\'edition de T. V. \textsc{Gopal Iyer}, cf. \textsc{Gros} (2001: 20, n. 2).}, fid\`ele et annot\'ee de l'\oe uvre du poète\index{gnl}{poete@poète} suivie de courts commentaires qui contextualisent chaque hymne\index{gnl}{hymne} en accord avec les donn\'ees du \textit{Periyapur\=a\d nam}\index{gnl}{Periyapuranam@\textit{Periyapur\=a\d nam}}. Enfin, la traduction glos\'ee de \textsc{V. M. Subramanya Aiyar} de l'int\'egralit\'e du corpus\index{gnl}{corpus} est disponible sous forme \'electronique\footnote{\textsc{Subramanya Aiyar, Chevillard, Sarma} 2007 est un outil indispensable pour \'etudier le \textit{T\=ev\=aram}\index{gnl}{Tevaram@\textit{T\=ev\=aram}} aujourd'hui. Sur l'\'elaboration de ce travail, cf. \textsc{Chevillard} 2000.}.

D'autres chercheurs ont exploit\'e le corpus\index{gnl}{corpus} du \textit{T\=ev\=aram} avec une approche sp\'ecifique. \textsc{Barnoud-Sethupathy} (1994) pr\'esente la tradition\index{gnl}{tradition} vivante du chant\index{gnl}{chant} du \textit{T\=ev\=aram}\index{gnl}{Tevaram@\textit{T\=ev\=aram}} effectu\'e par les \textit{\=otuv\=ar}\index{gnl}{otuvar@\textit{\=otuv\=ar}} dans les temples du Pays Tamoul\index{gnl}{Pays Tamoul}. \textsc{Peterson} (1982) s'est int\'eress\'ee \`a l'\'elaboration de l'identit\'e\index{gnl}{identit\'e} tamoule shiva\"ite\index{gnl}{shiva\"ite} \`a travers le syst\`eme de p\`elerinage\index{gnl}{pelerinage@pèlerinage} d\'epeint dans ce texte. Les attaques prof\'er\'ees contre les asc\`etes ja\"in\index{gnl}{jain@ja\"in}s et bouddhiste\index{gnl}{bouddhiste}s qui abondent dans le corpus\index{gnl}{corpus} ont \'et\'e soulign\'ees par \textsc{Peterson} (*1999 [1998]) et \textsc{Davis} (*1999 [1998]). Pour clore cette \'enum\'eration, certes non exhaustive, il faut mentionner \textsc{Swamy} (1972), que la critique n'a pas \'epargn\'e, mais qui offre comme souvent dans ses publications des r\'ef\'erences \'epigraphiques nombreuses et fiables. %En lisant ses pages regorgeant de precieuses informations sur la vie du texte \`a date ancienne, nous ne pouvons bl\^amer ses conclusions parfois h\^atives et/ou provocantes.

Nous ne pouvons fournir dans le cadre du pr\'esent travail de recherche une \'etude compl\`ete du corpus\index{gnl}{corpus} du \textit{T\=ev\=aram}\index{gnl}{Tevaram@\textit{T\=ev\=aram}}. Soulignons seulement ici quelques caract\'eristiques, parfois jamais remarqu\'ees, qui illustrent la richesse du corpus\index{gnl}{corpus} mais aussi, malgr\'e une bibliographie abondante, certains points qui restent encore dans l'obscurit\'e. Ainsi, nous pr\'esentons la forme et le contenu des poème\index{gnl}{poeme@poème}s puis l'\'evolution s\'emantique du terme \textit{T\=ev\=aram}\index{gnl}{Tevaram@\textit{T\=ev\=aram}} et, enfin, quelques attestations \'epigraphiques du chant\index{gnl}{chant} des hymne\index{gnl}{hymne}s dans les temples.

\section{Le corpus}
Le \textit{T\=ev\=aram}\index{gnl}{Tevaram@\textit{T\=ev\=aram}} constitue aujourd'hui les sept premiers volumes du \textit{Tirumu\b rai}\index{gnl}{Tirumurai@\textit{Tirumu\b rai}} contenant les hymne\index{gnl}{hymne}s des poète\index{gnl}{poete@poète}s Tiru\~n\=a\b nacampantar ou Campantar\index{gnl}{Campantar} (\textsc{i}-\textsc{iii}), Tiru-n\=avukkaracar\index{gnl}{Appar!Tirunavukkaracar@Tirun\=avukkaracar} ou Appar\index{gnl}{Appar} (\textsc{iv}-\textsc{vi}) et Nampi \=Ar\=urar ou Cuntarar\index{gnl}{Cuntarar} (\textsc{vii}) qui forment un \og Trio\fg, shiva\"ite\index{gnl}{shiva\"ite}, les \textit{m\=uvar}\index{gnl}{muvar@\textit{m\=uvar}}\footnote{Sur le probl\`eme de leur identit\'e\index{gnl}{identit\'e} historique, voir 2.2 et 2.3.}. Il comporte trois cent quatre-vingt-cinq poème\index{gnl}{poeme@poème}s attribu\'es \`a Campantar\index{gnl}{Campantar}\footnote{Sont inclus dans le d\'ecompte les deux hymne\index{gnl}{hymne}s d\'ecouverts dans une inscription de Tiruvi\d taiv\=acal (ARE 1918 8) et dans un manuscrit\index{gnl}{manuscrit} sur \^ole en 1932 (\textsc{Gros} 1984: xxx-xxxi).}, trois cent douze\index{gnl}{douze} \`a Appar\index{gnl}{Appar} et cent un \`a Cuntarar\index{gnl}{Cuntarar} soit un total de sept cent quatre-vingt-dix-huit. La r\'epartition des \textit{Tirumu\b rai}\index{gnl}{Tirumurai@\textit{Tirumu\b rai}} par auteur et ces chiffres ne sont pas tr\`es loin de ce qui a été assembl\'e par Nampi\index{gnl}{Nampi \=A\d n\d t\=ar Nampi} dans la pièce\index{gnl}{piece@pièce} de Citamparam\index{gnl}{Citamparam} ou plut\^ot du corpus\index{gnl}{corpus} en circulation \`a partir du \textsc{xiv}\up{e} si\`ecle. En effet, le \textit{Tirumu\b raika\d n\d tapur\=a\d nam}\index{gnl}{Tirumuraikantapuranam@\textit{Tirumu\b raika\d n\d tapur\=a\d nam}} (voir 4.1.1) mentionne trois cent quatre-vingt-quatre hymne\index{gnl}{hymne}s formant trois \textit{tirumu\b rai}\index{gnl}{Tirumurai@\textit{Tirumu\b rai}} pour Campantar\index{gnl}{Campantar}, trois cent sept pour Appar\index{gnl}{Appar} (trois \textit{tirumu\b rai}\index{gnl}{Tirumurai@\textit{Tirumu\b rai}}) et cent pour Cuntarar\index{gnl}{Cuntarar} (un \textit{tirumu\b rai}\index{gnl}{Tirumurai@\textit{Tirumu\b rai}})\footnote{\textit{Tirumu\b raika\d n\d tapur\=a\d nam}\index{gnl}{Tirumuraikantapuranam@\textit{Tirumu\b raika\d n\d tapur\=a\d nam}} (st. 25):
\begin{verse}
\textit{pa\d npu\b r\b ra tiru\~n\=a\b na campantar\index{gnl}{Campantar} patika\textbf{mun n\=u\b r\\
\b re\d npatti \b n\=a\b nki\b n\=a l}ila\.nkutiru mu\b raim\=u\b n\b ru\\
na\d npu\b r\b ra n\=avaracar \textbf{munn\=u\b r\b r\=e\b l} m\=u\b n\b ri\b n\=al\\
va\d npe\b r\b ra mu\b rai; yo\b n\b ru \textbf{n\=u\b r\b ri\b n\=al} va\b n\b ro\d n\d tar} (25)\\
\end{verse}}.

Chaque hymne\index{gnl}{hymne} est la c\'el\'ebration d'un \'Siva\index{gnl}{Siva@\'Siva} particulier ancr\'e g\'en\'eralement sur un site r\'eel du sol tamoul. \`A l'exception des quarante-huit poème\index{gnl}{poeme@poème}s \`a caract\`ere g\'en\'eral (\textit{potu}) ou louant des temples situ\'es au Srilanka\index{gnl}{Srilanka} (deux sites), en Pays Kanna\d da (un), Tu\d lu (un) et dans les r\'egions \og du Nord\fg\ (cinq) --- englobant aussi bien des provinces r\'eelles que la mythique montagne du Kail\=asa\index{gnl}{Kailasa@Kail\=asa} ---, tous les autres hymne\index{gnl}{hymne}s sont localis\'es dans le Pays Tamoul\index{gnl}{Pays Tamoul}: C\=o\b lan\=a\d tu (cent quatre-vingt-onze sites), P\=a\d n\d dyan\=a\d tu (quatorze), Malain\=a\d tu (un), Ko\.nku\index{gnl}{Pays Konku@Pays Ko\.nku}n\=a\d tu (sept), Na\d tun\=a\d tu (vingt-deux) et To\d n\d tain\=a\d tu (trente-trois). Le nombre total de temples chant\'es, \textit{p\=a\d tal pe\b r\b ra talam}, s'\'el\`eve \`a deux cent soixante-seize (\textsc{Gopal Iyer} 1991: 188-203.). %C\=\i k\=a\b li\index{gnl}{Cikali@C\=\i k\=a\b li} est un de ces sites. Lieu de naissance\index{gnl}{naissance} de Campantar\index{gnl}{Campantar}, situ\'e dans le delta de la K\=av\=eri\index{gnl}{Kaveri@K\=av\=eri}, il est le plus c\'el\'ebr\'e: soixante et onze hymne\index{gnl}{hymne}s lui sont assign\'es.

Il existe deux classements des hymne\index{gnl}{hymne}s. Le premier, plus fr\'equent, suit l'ordonnance\index{gnl}{ordonnance} selon les modes musicaux (\textit{pa\d nmu\b rai}) que la tradition\index{gnl}{tradition} attribue \`a Nampi\index{gnl}{Nampi \=A\d n\d t\=ar Nampi} \=A\d n\d t\=ar Nampi. Le second, plus tardif, présente les poème\index{gnl}{poeme@poème}s selon les sites (\textit{talamu\b rai}), et aurait \'et\'e inaugur\'ee par un Um\=apati\index{gnl}{Umapati@Um\=apati}. Ce dernier agencement classe les hymne\index{gnl}{hymne}s dans un ordre\index{gnl}{ordre} g\'eographique pr\'ecis \`a l'int\'erieur duquel ils sont rang\'es selon les \textit{pa\d n}.

Les poème\index{gnl}{poeme@poème}s, \textit{patikam}\index{gnl}{patikam@\textit{patikam}} ou \textit{patiyam} (sk. \textit{padya}), sont constitu\'es en g\'en\'eral de dix strophes (\textit{p\=a\d t\d tu}) de quatre vers. Chez Campantar\index{gnl}{Campantar} s'ajoute r\'eguli\`erement l'envoi\index{gnl}{envoi} qui est l'une de ses caract\'eristiques\footnote{Nous d\'eveloppons ce point en 2.2.1.}. Quand un manque survient il est d'usage de consid\'erer qu'un quatrain s'est perdu\footnote{C'est le cas dans les travaux de \textsc{V. M. Subramanya Ayyar} et dans les \'editions des monast\`ere\index{gnl}{monastère}s de Tarumapuram\index{gnl}{Tarumapuram} et de Tiruv\=ava\d tutu\b rai\index{gnl}{Avatuturai@\=Ava\d tutu\b rai!Tiruv\=ava\d tutu\b rai}.} mais le raisonnement de \textsc{Gros} (1984: xxxiii), fond\'e sur l'introduction de \textsc{T. V. Gopal Iyer}, en faveur d'\og une certaine \'elasticit\'e\fg\ de l'unit\'e de dix strophes chez les hymnistes vishnouite\index{gnl}{vishnouite}s et shiva\"ite\index{gnl}{shiva\"ite}s, nous semble pr\'ef\'erable.

\textsc{Peterson} (*1991 [1989]) et \textsc{Gopal Iyer} (1991) donnent une excellente vision des ressources du \textit{T\=ev\=aram}\index{gnl}{Tevaram@\textit{T\=ev\=aram}}. Nous pr\'esentons ici quelques points pr\'ecis qui illustrent les richesses iconographiques, religieuses et lexicales du corpus\index{gnl}{corpus}. Il faut cependant garder \`a l'esprit qu'il n'existe aucun travail critique qui aborde de fa\c con scientifique les questions, par exemple, de la paternit\'e des hymne\index{gnl}{hymne}s, de leurs \'eventuelles strates de composition et de leur homog\'en\'eisation. Notre \'etude repose sur un corpus\index{gnl}{corpus} \'etabli et fig\'e par la tradition\index{gnl}{tradition}.

Il semble, de prime abord, que les premi\`eres apparitions de \'Siva\index{gnl}{Siva@\'Siva} dans la litt\'erature tamoule soient dispers\'ees dans les textes du \textit{Ca\.nkam}\index{gnl}{Cankam@\textit{Ca\.nkam}}. Elles se trouvent, en particulier, dans les stances d'invocation\index{gnl}{invocation} de certains poème\index{gnl}{poeme@poème}s\footnote{Nous pensons par exemple au poème\index{gnl}{poeme@poème} d'invocation\index{gnl}{invocation} de l'\textit{Ai\.nku\b run\=u\b ru}\index{gnl}{Ainkurunuru@\textit{Ai\.nku\b run\=u\b ru}}, \og Les Cinq Centuries de pièce\index{gnl}{piece@pièce}s br\`eves\fg, attribu\'e \`a Perunt\=eva\b n\=ar.}. Or, ces derni\`eres sont probablement des ajouts post\'erieurs\footnote{Information communiqu\'ee par E. \textsc{Wilden}.}. Toutefois, les deux lignes ouvrant un poème\index{gnl}{poeme@poème} \og authentique\fg\ du \textit{Pu\b ran\=a\b n\=u\b ru}\index{gnl}{Purananuru@\textit{Pu\b ran\=a\b n\=u\b ru}}, \og Les Quatre Cents poème\index{gnl}{poeme@poème}s de guerre\fg, d\'ecrivent \'Siva\index{gnl}{Siva@\'Siva} parmi un groupe de quatre divinit\'es\footnote{\textit{Pu\b ran\=a\b n\=u\b ru}\index{gnl}{Purananuru@\textit{Pu\b ran\=a\b n\=u\b ru}} 56 1-2:
\begin{verse}
\textit{\=e\b r\b ruvala \b nuyariya verimaru \d lavirca\d tai\\
m\=a\b r\b raru\.n ka\d nicci ma\d nimi\d ta\b r \b r\=o\b num}\\
\end{verse}
\og There is the god whose neck is the color of sapphire, on whose banner the bull spells out victory, whose matted hair spreads like fire, whose ax is irresistible\fg\ (traduction de \textsc{Hart} \& \textsc{Heifetz} *2002 [1999]).}. Mais c'est dans les hymne\index{gnl}{hymne}s du \textit{Tirumu\b rai}\index{gnl}{Tirumurai@\textit{Tirumu\b rai}} qu'il est pour la premi\`ere fois pleinement honor\'e. L'\oe uvre novatrice de K\=araikk\=alammaiy\=ar\index{gnl}{Karaikkalammaiyar@K\=araikk\=alammaiy\=ar} serait la plus ancienne dans le \textit{Tirumu\b rai}. Les hymne\index{gnl}{hymne}s de cette po\'etesse du \textsc{vi}\up{e} si\`ecle (\textsc{Zvelebil} 1975: 136-7) s'ins\`erent dans le livre \textsc{xi} du canon\index{gnl}{canon} dont l'ordonnance\index{gnl}{ordonnance} pr\'etendue chronologique se trouve ainsi boulevers\'ee\footnote{cf. \textsc{Gros} (1982: 97).}. Ensuite, les hymne\index{gnl}{hymne}s du \textit{T\=ev\=aram}\index{gnl}{Tevaram@\textit{T\=ev\=aram}}, enti\`erement consacr\'es \`a la c\'el\'ebration de \'Siva\index{gnl}{Siva@\'Siva}, foisonnent de descriptions iconographiques vari\'ees et complexes d'un dieu supr\^eme, \`a la fois panindien et local\index{gnl}{local}. \textsc{Gopal Iyer} (1991: 357-360) rel\`eve quarante formes \og sanskrites\fg\ de \'Siva\index{gnl}{Siva@\'Siva}\footnote{Nous avons \'evoqu\'e la densit\'e des manifestations de la divinit\'e au cours du DEA o\`u l'\'etude d'un simple \'echantillon de cinq poème\index{gnl}{poeme@poème}s de Campantar\index{gnl}{Campantar} dressait le tableau de vingt et une formes, cf. \textsc{Veluppillai} (2003a: 72).}. Par ailleurs, les hymne\index{gnl}{hymne}s peignent parfois des images ou des r\'ecits mythologiques qui semblent propres \`a une tradition\index{gnl}{tradition} tamoule. \'Siva\index{gnl}{Siva@\'Siva} T\=o\d niyappar\index{gnl}{Toniyappar@T\=o\d niyappar}\footnote{Cette manifestation li\'ee au site de C\=\i k\=a\b li\index{gnl}{Cikali@C\=\i k\=a\b li} est \'etudi\'ee en d\'etail dans le chapitre 3.} et \'Siva\index{gnl}{Siva@\'Siva} \'ecrasant R\=ava\d na\index{gnl}{Ravana@R\=ava\d na} en forment de parfaites illustrations. Ce dernier a été analys\'e par \textsc{Gillet} (2007) au cours de l'identification de certains panneaux narratifs \textit{pallava}\index{gnl}{pallava@\textit{pallava}} repr\'esentant le démon\index{gnl}{demon@démon} R\=ava\d na jouant d'un instrument \`a corde singulier. En effet, ce dernier joue de la musique\index{gnl}{musique} avec les tendons qu'il a extraits de son bras. Alors qu'il semble qu'aucun texte sanskrit n'explique l'\'episode de R\=ava\d na musicien, le \textit{T\=ev\=aram}\index{gnl}{Tevaram@\textit{T\=ev\=aram}} propose une variante mythologique vraisemblablement tamoule: R\=ava\d na soul\`eve le mont Kail\=asa\index{gnl}{Kailasa@Kail\=asa}. \'Siva\index{gnl}{Siva@\'Siva} l'\'ecrase de son orteil. Puis le démon\index{gnl}{demon@démon} devenu d\'evot chante la louange du dieu en jouant de la musique\index{gnl}{musique} avec les tendons de son bras.

Ensuite, les informations sur les pratiques shiva\"ite\index{gnl}{shiva\"ite}s \`a l'\'epoque de la composition des hymne\index{gnl}{hymne}s, bien qu'\'eparses et allusives, sont pr\'ecieuses et permettent de mieux cerner la connaissance\index{gnl}{connaissance} qu'en avaient les poète\index{gnl}{poete@poète}s et leur \'eventuelle participation aux diff\'erents mouvements sectaire\index{gnl}{sectaire}s. Par exemple, l'interrogation premi\`ere de \textsc{T\"orzs\"ok} (2004) sur la probable existence d'un culte\index{gnl}{culte} de la forme de \'Siva\index{gnl}{Siva@\'Siva} le fou peint dans le \textit{T\=ev\=aram}\index{gnl}{Tevaram@\textit{T\=ev\=aram}}, qui serait li\'e \`a celui de Bhairava\index{gnl}{Bhairava} le fou de l'Inde septentrionale, la conduit ainsi \`a \'etudier, entre autres, la notion de la folie chez \'Siva\index{gnl}{Siva@\'Siva} et ses d\'evots, ainsi que sa relation avec \'Siva\index{gnl}{Siva@\'Siva} Pa\'supati\index{gnl}{Pasupati@Pa\'supati}. Ailleurs, les hymne\index{gnl}{hymne}s pr\'esentent un \'Siva\index{gnl}{Siva@\'Siva} porteur d'un \textit{pa\~ncava\d ti}\index{gnl}{pancavati@\textit{pa\~ncava\d ti}}, cordon\index{gnl}{cordon} sacrificiel fait de cheveux. Nous avons montr\'e que ce cordon\index{gnl}{cordon} \og qui semble \^etre une parure des renon\c cants asc\'etiques shiva\"ite\index{gnl}{shiva\"ite}s terribles --- li\'es au bois cr\'ematoire, porteurs de cr\^anes, couverts d'os et frictionn\'es de cendre\index{gnl}{cendre} --- d'apr\`es le t\'emoignage du \textit{T\=ev\=aram}\index{gnl}{Tevaram@\textit{T\=ev\=aram}}, devient tr\`es clairement une sp\'ecificit\'e des \textit{mah\=avratin}\index{gnl}{mahavratin@\textit{mah\=avratin}} d'apr\`es la \textit{Ni\'sv\=asatattvasa\d mhit\=a}\index{gnl}{Nisvasatattva@\textit{Ni\'sv\=asatattvasa\d mhit\=a}}, K\d semar\=aja\index{gnl}{Ksemaraja@K\d semar\=aja}, Nirmalama\d ni\index{gnl}{Nirmalama\d ni} et C\=ekki\b l\=ar\index{gnl}{Cekkilar@C\=ekki\b l\=ar}\fg\ (\textsc{Veluppillai} 2003b: 107-108, n.~28). Ainsi, le \textit{T\=ev\=aram}\index{gnl}{Tevaram@\textit{T\=ev\=aram}} apporte une documentation non n\'egligeable sur les diverses sectes connues des auteurs du corpus\index{gnl}{corpus}.

Enfin, nous souhaitons attirer l'attention sur un point d'ordre\index{gnl}{ordre} lexical. \textsc{Chevillard} (2000) nous convainc de l'utilit\'e d'une concordance \`a travers l'exemple de deux termes d'origine\index{gnl}{origine} sanskrite (\textit{gopura}\index{gnl}{gopura@\textit{gopura}} et \textit{\=agama}\index{gnl}{Agama@\textit{\=Agama}}) et souligne que le lex\`eme \textit{k\=opuram}\index{gnl}{gopura@\textit{gopura}} qui conna\^it quinze occurrences dans le corpus\index{gnl}{corpus} du \textit{T\=ev\=aram}\index{gnl}{Tevaram@\textit{T\=ev\=aram}} appara\^it en fait pour la premi\`ere fois dans la litt\'erature tamoule, mais ne semble pas d\'esigner comme aujourd'hui le pavillon d'entr\'ee des temples. Nous avons propos\'e une autre illustration fond\'ee sur le verbe compos\'e \textit{amutu cey}- qui renforce cette id\'ee d'une langue du \textit{T\=ev\=aram}\index{gnl}{Tevaram@\textit{T\=ev\=aram}} comme maillon distinct entre celle de l'\'epoque du \textit{Ca\.nkam}\index{gnl}{Cankam@\textit{Ca\.nkam}} et celle post\'erieure au corpus\index{gnl}{corpus} (\textsc{Veluppillai} 2013): le verbe \textit{amutu\index{gnl}{amutu@\textit{amutu}} cey}- (\og ambroisie\index{gnl}{ambroisie}\fg\ + \og faire\fg) signifie \og manger\fg\ dans la litt\'erature post-\textit{T\=ev\=aram}\index{gnl}{Tevaram@\textit{T\=ev\=aram}} et dans l'\'epigraphie; son sujet est g\'en\'eralement une divinit\'e ou une personne honorable. Ce compos\'e semble appara\^itre pour la premi\`ere fois dans le \textit{T\=ev\=aram}\index{gnl}{Tevaram@\textit{T\=ev\=aram}}. Nous y trouvons treize occurrences attach\'ees au mythe\index{gnl}{mythe} o\`u \'Siva\index{gnl}{Siva@\'Siva} contient dans sa gorge le poison issu du barattage de l'oc\'ean de lait\index{gnl}{lait}. Cependant, son emploi dans le corpus\index{gnl}{corpus} ne correspond pas au sens attest\'e dans les textes post-\textit{T\=ev\=aram}\index{gnl}{Tevaram@\textit{T\=ev\=aram}} tels que le \textit{Periyapur\=a\d nam}\index{gnl}{Periyapuranam@\textit{Periyapur\=a\d nam}} et les inscriptions médiévales du Pays Tamoul. En effet, douze\index{gnl}{douze} des treize occurrences du \textit{T\=ev\=aram} ont toujours le m\^eme sujet, \'Siva\index{gnl}{Siva@\'Siva}, et le m\^eme objet, le poison (\textit{na\~ncu}\index{gnl}{nancu@\textit{na\~ncu}}, \textit{\=alam}\index{gnl}{alam@\textit{\=alam}})\footnote{I 62 5 \textit{na\~ncu\index{gnl}{nancu@\textit{na\~ncu}} amutuceytu-aru\d lum nampi}; II 33 5 \textit{na\~ncu\index{gnl}{nancu@\textit{na\~ncu}} amutuceytava\b n}; 97 5 \textit{na\~ncu amutucey}; III 49 10 \textit{amutucey\index{gnl}{amutucey@\textit{amutucey}} na\~ncu u\d l ka\d n\d ta\b n}; 71 6 \textit{na\~ncu amutuceyta ma\d nika\d n\d ta\b n}; 88 6 \textit{vi\d tam amutucey ka\b rai a\d ni mi\d ta\b ri\b nar}; VI 26 5 \textit{na\~ncu amutuceytu}; 27 8 \textit{na\~ncai amutuceyta ka\b rpakattai}; 50 5 \textit{na\~ncu amutuceyt\=a\b nai}; 84 9 \textit{\=al\=alam amutuceyta kariyatu oru ka\d n\d tattu}; VII 61 1 \textit{\=alamt\=a\b n ukantu amutuceyt\=a\b nai}. Dans VII 65 2 l'objet n'est pas mentionn\'e, \textit{amutuceyta amutam}.}: il semble bien qu'il s'agisse d'une formule. Deux interpr\'etations sont possibles: \'Siva\index{gnl}{Siva@\'Siva} mange le poison ou, litt\'eralement, \'Siva\index{gnl}{Siva@\'Siva} fait du poison de l'ambroisie\index{gnl}{ambroisie} --- qui est sa nourriture par excellence. D'autres passages exprimant la puissante action de \'Siva\index{gnl}{Siva@\'Siva}, qui avale le poison tel de l'ambroisie\index{gnl}{ambroisie}, viennent à l'appui de cette seconde interpr\'etation\footnote{II 37 4 \textit{ka\d talna\~ncu amut\=a-atu u\d n\d ta karutt\=e} \og O objet de d\'esir qui a mang\'e le poison de la mer\index{gnl}{mer} comme de l'ambroisie\index{gnl}{ambroisie}\fg; 77 4 \textit{na\~ncu amutu-\=aka u\d n\d tu} \og ayant mang\'e le poison comme de l'ambroisie\index{gnl}{ambroisie}\fg; 118 3 \textit{na\~ncam amutu-\=aka u\d n\d ta ka\d tavu\d l} \og le dieu\index{gnl}{dieu} qui a mang\'e le poison comme de l'ambroisie\index{gnl}{ambroisie}\fg; et III 62 3, 78 2, 105 11, 121 4, IV 70 5 (Appar\index{gnl}{Appar}, sujet, renvoie \`a l'\'episode de son empoisonnement par les ja\"in\index{gnl}{jain@ja\"in}s et pr\'esente en 70 7 un parall\`ele entre sa capacit\'e, \`a travers la d\'evotion\index{gnl}{devotion@dévotion}, de transformer le poison en ambroisie\index{gnl}{ambroisie} et celle de \'Siva\index{gnl}{Siva@\'Siva}), 89 1, V 73 4, VI 16 8, 20 3, 40 1, 51 8, 86 2, 96 7, VII 9 10.}. Ainsi, \textit{k\=opuram} et \textit{amutu\index{gnl}{amutu@\textit{amutu}} cey}-, termes apparus vraisemblablement dans le \textit{T\=ev\=aram}\index{gnl}{Tevaram@\textit{T\=ev\=aram}} et connaissant une \'evolution s\'emantique par la suite, t\'emoignent des originalit\'es lexicales du corpus\index{gnl}{corpus}.

\section{Le terme}
Le terme \textit{T\=ev\=aram}\index{gnl}{Tevaram@\textit{T\=ev\=aram}} ne semble appara\^itre dans aucun texte du \textit{Tirumu\b rai}\index{gnl}{Tirumurai@\textit{Tirumu\b rai}}. Son \'etymologie et son sens ont \'et\'e discut\'es dans \textsc{Ve\d l\d laiv\=ara\d na\b n} (*1994 [1962 et 1969]: 35-43), \textsc{Rangaswamy} (*1990 [1958]: 27-35) et \textsc{Gros} (1984: vii) sans aboutir \`a une conclusion d\'efinitive.

Il existe deux interpr\'etations, dites modernes et `tamoules', sur la formation du terme. La premi\`ere repose sur un compos\'e de \textit{t\=e} \og dieu\index{gnl}{dieu}\fg\ et d'\textit{\=aram} \og guirlande\index{gnl}{guirlande}\fg\ dont l'emploi m\'etaphorique chez Campantar\index{gnl}{Campantar}, Appar\index{gnl}{Appar} (un seul) et Cuntarar\index{gnl}{Cuntarar} d\'esigne l'hymne\index{gnl}{hymne}; la semi-voyelle \textit{v} r\'esulte de la liaison. En effet, les envois\index{gnl}{envoi} rappellent souvent que les dix quatrains compos\'es \`a la gloire de \'Siva\index{gnl}{Siva@\'Siva} sont une offrande de guirlandes\footnote{Cette image est aussi tr\`es courante dans le corpus\index{gnl}{corpus} vishnouite\index{gnl}{vishnouite} du \textit{N\=al\=ayirattiviyappirapantam}\index{gnl}{Nalayirattiviya@\textit{N\=al\=ayirattiviyappirapantam}}, en particulier, dans les envois\index{gnl}{envoi} des poète\index{gnl}{poete@poète}s suivants: Periy\=a\b lv\=ar\index{gnl}{Periyalvar@Periy\=a\b lv\=ar} ( st. 401, 432), \=A\d n\d t\=a\d l (st. 503, 513, 626), Namm\=a\b lv\=ar (st. 2577, 2920, 2975, 3087, 3142, 3241, 3318, 3406, 3472, 3582, 3681, 3769, 3813, 3934, 3956) et, surtout, Tiruma\.nkaiy\=a\b lv\=ar\index{gnl}{Tirumankaiyalvar@Tiruma\.nkaiy\=a\b lv\=ar} (st. 977, 1007, 1047, 1067, 1077, 1087, 1127, 1137, 1187, 1197, 1207, 1217, 1227, 1237, 1247, 1287, 1317, 1337, 1387, 1397, 1407, 1427, 1437, 1447, 1467, 1477, 1497, 1517, 1547, 1557, 1567, 1577, 1587, 1597, 1617, 1637, 1657, 1677, 1687, 1727, 1737, 1747, 1757, 1767, 1777, 1807, 1847, 1877, 1907, 1921, 1931, 1951, 1981, 2011, 2021, 2051, 2081). Nous remercions Charlotte \textsc{Schmid} qui nous a signal\'e ce parall\'elisme.}. Toutefois, le terme employ\'e pour signifier \og poème\index{gnl}{poeme@poème}-guirlande\index{gnl}{guirlande}\fg\ dans les hymne\index{gnl}{hymne}s, aussi bien shiva\"ite\index{gnl}{shiva\"ite}s que vishnouite\index{gnl}{vishnouite}s, est g\'en\'eralement \textit{m\=alai}\index{gnl}{malai@\textit{m\=alai}}. Si nous trouvons parfois \textit{to\d tai}\footnote{Le terme \textit{to\d tai} est utilis\'e par Campantar\index{gnl}{Campantar} (I 100 11) et Namm\=a\b lv\=ar (st. 2577, 2920, 2975, 3241 et 3934).}, \textit{\=aram} semble absent. Il nous est donc difficile d'adh\'erer \`a cette premi\`ere interpr\'etation reposant sur \textit{\=aram}. La seconde interpr\'etation est fond\'ee sur l'association de \textit{t\=e} et de \textit{v\=aram} \og chant\fg, synonyme de \textit{t\=evap\=a\d ni}, signifiant un chant adress\'e \`a un dieu\index{gnl}{dieu}\footnote{Cette alternative est propos\'ee par \textsc{Ve\d l\d laiv\=ara\d na\b n} (*1994 [1962 et 1969]: 37-41) qui se base sur le commentaire du \textit{Cilappatik\=aram}\index{gnl}{Cilappatikaram@\textit{Cilappatik\=aram}}.}.

Or, les attestations \'epigraphiques nous permettent d'\'elaborer pour ce terme une \'etymologie m\'edi\'evale et sanskrite.
Dans les inscriptions, le terme \textit{T\=ev\=aram}\index{gnl}{Tevaram@\textit{T\=ev\=aram}} relev\'e et analys\'e par \textsc{Rangaswamy} est vu tout le long de son argumentation, qu'il veut chronologique, \`a travers la premi\`ere interpr\'etation qu'il en donne, \textit{i.e.} de culte\index{gnl}{culte} priv\'e. En effet, selon cet auteur, le mot \textit{T\=ev\=aram}\index{gnl}{Tevaram@\textit{T\=ev\=aram}} signifiait au d\'epart un culte\index{gnl}{culte} priv\'e et par extension d\'esignait la divinit\'e d'\'election de ce culte\index{gnl}{culte}. Par la suite, il conna\^it une \'evolution s\'emantique et prend ainsi, au fil du temps, les sens d'adoration puis d'hymne\index{gnl}{hymne} pour finalement renvoyer au corpus\index{gnl}{corpus} de poème\index{gnl}{poeme@poème}s des \textit{m\=uvar}\index{gnl}{muvar@\textit{m\=uvar}}.

Cependant, il nous para\^it possible de sugg\'erer un autre glissement s\'emantique en reprenant les exemples de \textsc{Rangaswamy} et en apportant de nouveaux \'el\'ements.
\`A notre connaissance\index{gnl}{connaissance}, la premi\`ere inscription brahmanique mentionnant le terme \textit{T\=ev\=aram}\index{gnl}{Tevaram@\textit{T\=ev\=aram}}, que \textsc{Rangaswamy} ne semble pas conna\^itre, est une \'epigraphe tamoule du temple d'O\b r\b riy\=ur (EI 27 47) datant de la vingti\`eme ann\'ee de r\`egne du roi\index{gnl}{roi} \textit{r\=a\d s\d trak\=u\d ta} K\textsubring{r}\d s\d na III\index{gnl}{Krsna@K\textsubring{r}\d s\d na III} (939-967), c'est-\`a-dire de 959. Elle enregistre une donation de pièce\index{gnl}{piece@pièce}s d'or par un chef\index{gnl}{chef} de monast\`ere\index{gnl}{monastère} pour assurer \'eternellement, les jours de son ast\'erisme de naissance\index{gnl}{naissance}, la c\'er\'emonie au \'Siva\index{gnl}{Siva@\'Siva} du temple. Elle \'evoque les diff\'erentes offrandes et \'enum\`ere les employ\'es du temple parmi lesquels figurent, en t\^ete, trois \textit{d\=ev\=aram\=a\d ni} (l. 20). \textsc{V. Raghavan}, \'editeur de l'inscription, pense que ce terme renvoie aux chanteurs du \textit{T\=ev\=aram}\index{gnl}{Tevaram@\textit{T\=ev\=aram}} actuel
%\footnote{Il \'ecrit p.~301: \og \textit{D\=ev\=aram\=a\d nis} (l.20) are reciters of the \textit{D\=ev\=aram} hymns. M\=a\d ni is either a student or Brahmach\=ar\=\i\ (\textit{M\=a\d navaka}, \textit{M\=a\d n\=akka}), and refers perhaps to the class of temple singers solely devoted to the recital of \textit{D\=ev\=aram} hymns, at service time\fg.}
alors que \textsc{Subbarayalu} (2003: 340) pense plut\^ot aux jeunes brahmane\index{gnl}{brahmane}s c\'elibataires qui s'occupent des images de culte\index{gnl}{culte}s; la d\'efinition du premier est anachronique et celle du second est probablement fond\'ee sur l'interpr\'etation de \textsc{Rangaswamy}. Il nous para\^it opportun de rappeler ici la signification de \og lieu de culte\index{gnl}{culte}\fg\ que rev\^et \textit{T\=ev\=aram}\index{gnl}{Tevaram@\textit{T\=ev\=aram}} dans deux inscriptions ja\"in\index{gnl}{jain@ja\"in}es donn\'ees par \textsc{Gros} (1984)\footnote{Les deux \'epigraphes, ARE 1936-37 251 publi\'ee dans EI 29 28 et ARE 1972-73 B273, ne sont pas dat\'ees et mentionnent la mise en place de sites naturels form\'es de rocs sculpt\'es. Leur donn\'ee pal\'eographique supposerait le \textsc{ix-x}\up{e} si\`ecle.} et l'\'etymologie solide que propose P. B. \textsc{Desai}, l'\'editeur de EI 29 28, p.~201, n.~2\footnote{Cette \'etymologie est reprise par \textsc{Gros} (1984: vii) et \textsc{Nagaswamy} (1989: 219-220), sur lequel nous reviendrons.}:

\scriptsize
\begin{quote}
\og it may not be unreasonable to connect it with the Sanskrit \textit{d\=ev\=ag\=ara}, in which case it would mean `a shrine'. Use of the word \textit{d\=eh\=ara} in the sense of `a shrine' is found in an 11\up{th} century Kanna\d da inscription in the Bellary District; \textit{SII}, vol. IX, part i, No. 115.\fg % qui r\'esume aussi tr\`es succintement les propos de \textsc{Rangaswamy}: \og Mais les inscriptions m\'edi\'evales nous ram\`enent aux deux notions d'adoration et m\^eme plus pr\'ecis\'ement de culte\index{gnl}{culte} priv\'e ou d'un corps de b\^atiment particulier, li\'e aussi \`a un temple et \`a l'exercice du culte\index{gnl}{culte}, ce qui nous conduit aux \'equivalents s\'emantiques sanskrits du type \textit{deva}-\textit{\=ag\=ara}, \textit{deva}-\textit{\=arha}, qui d\'esignent un lieu de culte\index{gnl}{culte} et pourraient \^etre \`a l'origine\index{gnl}{origine} du terme tamoul\fg.}
\end{quote}
\normalsize
\noindent Ainsi, nous supposons que \textit{t\=ev\=aram\=a\d ni}, dans EI 27 47, peut d\'esigner l'\og officiant\fg\ (\textit{m\=a\d ni}), c\'elibataire ou apprenti, du \og temple\fg\ (\textit{t\=ev\=aram}\index{gnl}{Tevaram@\textit{T\=ev\=aram}}).

\textsc{Rangaswamy} d\'ebute son argumentation avec SII 2 38 qui est ant\'erieure \`a la vingt-neuvi\`eme ann\'ee de R\=ajar\=aja I\index{gnl}{Rajaraja I@R\=ajar\=aja I} (1014) et qui mentionne un \og dieu\index{gnl}{dieu} install\'e en tant que \textit{dev\=aradevar} pour le grand seigneur (qu'est le roi\index{gnl}{roi})\fg\footnote{l. 36: \textit{periya perum\=a\d lukku dev\=aradevar\=aka e\b luntaru\d livitta devar}.} \`a Ta\~nc\=av\=ur\index{gnl}{Tancavur@Ta\~nc\=av\=ur}. Il pense qu'il s'agit de la divinit\'e favorite, \textit{i\d s\d tadevat\=a}\index{gnl}{istadevata@\textit{i\d s\d tadevat\=a}}, du culte\index{gnl}{culte} priv\'e du roi\index{gnl}{roi}. Or, cette inscription enregistre l'installation\index{gnl}{installation d'une image} de sept images par un officier\index{gnl}{officier} du roi\index{gnl}{roi}: Cuntarar\index{gnl}{Cuntarar} et son \'epouse Paravai\index{gnl}{Paravai}, Appar\index{gnl}{Appar}, Campantar\index{gnl}{Campantar}, le roi\index{gnl}{roi}, sa reine\index{gnl}{reine} et enfin, Candra\'sekhara\index{gnl}{Candra\'sekhara} (forme de \'Siva\index{gnl}{Siva@\'Siva} portant la lune pour aigrette) en tant que \textit{t\=ev\=arat\=evar} du roi\index{gnl}{roi}. Bien qu'il soit possible de consid\'erer avec \textsc{Rangaswamy} que Candra\'sekhara est l'\textit{i\d s\d tadevat\=a}\index{gnl}{istadevata@\textit{i\d s\d tadevat\=a}} du culte\index{gnl}{culte} priv\'e du roi\index{gnl}{roi}, les dimensions des images offertes remettent quelque peu en question le statut privil\'egi\'e de Candra\'sekhara\index{gnl}{Candra\'sekhara}. Ce dernier est trois \`a quatre fois plus petit que les \textit{m\=uvar}\index{gnl}{muvar@\textit{m\=uvar}} et cinq fois plus petit que le roi\index{gnl}{roi}. Est-il d'usage d'installer l'image d'un d\'evot, f\^ut-il roi\index{gnl}{roi}, cinq fois plus grande que celle de sa divinit\'e d'\'election\string? De plus, cette image a \'et\'e install\'ee avec les \textit{m\=uvar}\index{gnl}{muvar@\textit{m\=uvar}} dont les \oe uvres formeront ce qui sera plus tard appelé le \textit{T\=ev\=aram}\index{gnl}{Tevaram@\textit{T\=ev\=aram}}. S'agit-il d'une simple co\"incidence ou faut-il percevoir un rapport entre la fonction du \textit{t\=ev\=arat\=evar} et les \textit{m\=uvar}\index{gnl}{muvar@\textit{m\=uvar}} ou leurs hymne\index{gnl}{hymne}s\string? L'interpr\'etation de \textsc{Rangaswamy} devient alors peu convaincante. Ainsi, nous comprenons \textit{t\=ev\=arat\=evar} en tant que \og divinit\'e de l'espace de culte\index{gnl}{culte}\fg, li\'e au roi\index{gnl}{roi}.
De plus, la deuxi\`eme inscription (SII 2 20), que \textsc{Rangaswamy} pr\'esente du m\^eme r\`egne, mais qui est en fait un texte datant de la dix-neuvi\`eme ann\'ee de R\=ajendra I\index{gnl}{Rajendra I@R\=ajendra I} (1031), accorde un don\index{gnl}{don} aux \textit{\=ac\=arya} du temple de R\=ajar\=aje\'svara \`a Ta\~nc\=av\=ur\index{gnl}{Tancavur@Ta\~nc\=av\=ur}. Le roi\index{gnl}{roi} prononce l'acte depuis le \textit{dev\=arattuc cu\b r\b ruk kall\=uri}\footnote{\textsc{Rangaswamy} comprend \`a la lumi\`ere de sa premi\`ere interpr\'etation de culte\index{gnl}{culte} priv\'e qu'il s'agit de \og the place of king's private worship, where T\=ev\=aram meant only private individual worship or \=a\b nm\=artta p\=uj\=a\index{gnl}{puja@\textit{p\=uj\=a}}\fg.} qui se trouve au nord du \textit{tirum\=a\d likai} de Mu\d tiko\d n\d taco\b la\b n \`a l'int\'erieur du \textit{koyil} de Ka\.nkaiko\d n\d tac\=o\b lapuram\index{gnl}{Kankaikontacolapuram@Ka\.nkaiko\d n\d tac\=o\b lapuram} (l. 12-13). \textit{Kall\=uri} semble d\'esigner la galerie-verandah qui entoure un temple (\textsc{Subramaniam} 1957, s.v.); cette interp\'etation est soutenue par la description de SII 2 20: \og \textit{kall\=uri} qui entoure (\textit{cu\b r\b ru}) le \textit{dev\=aram}\fg\ dont le r\'ef\'erent serait le terme \textit{tirum\=a\d likai} qui semble renvoyer au palais ou \`a un b\^atiment du palais\footnote{Dans \textsc{Pichard} (1994: 179), L. \textsc{Thyagarajan} consid\`ere que \textit{koyil} se r\'ef\`ere au \og palace area\fg\ et que \textit{tirum\=a\d likai} est le palais m\^eme. Il ajoute que le nom de ce palais, Mu\d tiko\d n\d taco\b la\b n, d\'enote qu'il fut construit par R\=ajendra I\index{gnl}{Rajendra I@R\=ajendra I} et qu'il fut nomm\'e ainsi en son honneur. Deux autres \'epigraphes du \textsc{xi}\up{e} si\`ecle (SII 5 978 et ARE 1931-32 74 qui a été publi\'ee en part. II p.~50) pr\'ecisent que Ka\.nkaiko\d n\d tac\=o\b lapuram\index{gnl}{Kankaikontacolapuram@Ka\.nkaiko\d n\d tac\=o\b lapuram} est la maison du roi\index{gnl}{roi}: \textit{ka\.nkaiko\d n\d taco\b lapurattu nam v\=\i \d t\d tinu\d l\d l\=al}, \og \`a l'int\'erieur de notre maison de Ka\.nkaiko\d n\d tac\=o\b lapuram\index{gnl}{Kankaikontacolapuram@Ka\.nkaiko\d n\d tac\=o\b lapuram}\fg.}. Ainsi, \textit{T\=ev\=aram}\index{gnl}{Tevaram@\textit{T\=ev\=aram}} nous para\^it \^etre simplement ici un b\^atiment du palais royal. %SII 8 222 l.~69 m\=a\d likaiyin melaikkall\=uriyil e\b luntaruli///// Thyagarajan pense que le malikai mitikontacola est le meme que kankaikontacola du XIe s. et que p.181 ce malikai fut tr\`es grand et pourvu d'\'etage et used frequemment par le roi\index{gnl}{roi} pour d\'elivrer des ordre\index{gnl}{ordre}s.
Ces deux textes (SII 2 38 et 20) permettent \`a \textsc{Rangaswamy} de conclure que le \textit{t\=ev\=aran\=ayakam} de ARE 1931-32 97\footnote{Le texte date de la cinqui\`eme ann\'ee de R\=ajendra I\index{gnl}{Rajendra I@R\=ajendra I}, identifi\'e gr\^ace \`a l'éloge\index{gnl}{eloge@éloge} royal. Il pr\'ecise que le donateur est \og Ma\b raikk\=a\d da\b n Pata\~njali Bha\d t\=ara of N\=a\.ng\=ur, who was doing the \textit{D\=ev\=aran\=ayakam} of R\=a\d l\=endra-Ch\=o\b lad\=eva, \textit{i.e.} the king\fg.}, \og responsable du \textit{T\=ev\=aram}\index{gnl}{Tevaram@\textit{T\=ev\=aram}}\fg\ --- lu avec anachronisme par \textsc{Nilakanta Sastri} (*2000 [1955]: 638) et \textsc{Zvelebil} (1975: 150) --- est celui qui est charg\'e de pr\'eparer le culte\index{gnl}{culte} priv\'e du roi\index{gnl}{roi}. Cependant, la notion de culte\index{gnl}{culte} priv\'e n'appara\^it ni dans ce texte de 1017 ni dans SII 24 262 (\textsc{xiii}\up{e} s.) %, r\'ef\'erence ajout\'ee par \textsc{Gros} 1984,
 qui \'evoque une donation au temple vishnouite\index{gnl}{vishnouite} de \'Sr\=\i ra\.nkam\index{gnl}{Srirankam@\'Sr\=\i ra\.nkam} par un certain I\b rai A\d n\d na\b n appartenant au \textit{T\=ev\=aram}\index{gnl}{Tevaram@\textit{T\=ev\=aram}} de la reine\index{gnl}{reine} (\string?) Somalateviy\=ar (l. 1: \textit{somalateviy\=ar tev\=arattu i\b rai a\d n\d na\b ne\b n}, \og moi I\b rai A\d n\d na\b n du \textit{T\=ev\=aram}\index{gnl}{Tevaram@\textit{T\=ev\=aram}} de Somalateviy\=ar\fg).
Il nous appara\^it qu'\`a travers ces \'epigraphes le terme \textit{T\=ev\=aram}\index{gnl}{Tevaram@\textit{T\=ev\=aram}} peut prendre le sens d'espace de culte\index{gnl}{culte} et/ou du palais en contexte royal.

Ensuite, l'argumentation de \textsc{Rangaswamy} offre deux inscriptions qu'il dit du \textsc{xi}\up{e} siècle cherchant \`a marquer un changement chronologique. Or, celles trait\'ees plus haut sont de ce m\^eme si\`ecle : SII 8 260 et 675. La premi\`ere de celles qu'évoque \textsc{Rangaswamy}, compl\`ete, ne pr\'ecise ni roi\index{gnl}{roi} ni date mais contient des termes appartenant au champ lexical de l'ordre\index{gnl}{ordre royal} royal (l. 3, la premi\`ere personne \textit{nam} et l.~7-8, la formule \textit{tiruv\=aymo\b lintaru\d li\b na tirumukappa\d ti}, \og selon l'ordre\index{gnl}{ordre royal} prononc\'e par la bouche sacr\'ee [du roi\index{gnl}{roi}]\fg). Bien que l'ARE interpr\`ete un ordre\index{gnl}{ordre} divin en l'absence de la mention royale, il semble plus pragmatique de consid\'erer ici un ordre\index{gnl}{ordre royal} royal adress\'e aux employ\'es du temple. Le texte enregistre un don\index{gnl}{don} de droits\index{gnl}{droits (\textit{k\=a\d ni})} (\textit{k\=a\d ni}) dans le temple (\textit{m\=ahe\'svarakka\d n k\=a\d ni} \og droit sur la surveillance \textit{mahe\'svara}\fg\ et \textit{tiruppatiyak k\=a\d ni} \og droit sur la [récitation\index{gnl}{recitation@récitation}] du chant\fg) \`a un certain Periya\b n Maraite\d tumporu\d l alias Aka\d la\.nkapiriya\b n qui chantait des \textit{tiruppatiyam} dans le \textit{T\=ev\=aram}\index{gnl}{Tevaram@\textit{T\=ev\=aram}} du roi\index{gnl}{roi}, du moins de celui qui donne l'ordre\index{gnl}{ordre}.
La seconde, datant de 1055, est clairement un ordre\index{gnl}{ordre royal} royal de R\=aj\=adhir\=aja I\index{gnl}{Rajadhiraja I@R\=aj\=adhir\=aja I} qui assigne un chanteur nomm\'e Ampalatt\=a\d ti Tirun\=avukkaraiya\b n au \textit{T\=ev\=aram}\index{gnl}{Tevaram@\textit{T\=ev\=aram}} du monast\`ere\index{gnl}{monastère} Mah\=adevar St\=a\b nama\d tam. \textsc{Rangaswamy} en conclut que le terme signifie le chant des hymne\index{gnl}{hymne}s au moment du culte\index{gnl}{culte} priv\'e dans le palais ou au monast\`ere\index{gnl}{monastère}. Par ailleurs, \textsc{Nagaswamy} (1989: 219-220) prend appui sur cette inscription pour d\'emontrer que le terme \textit{t\=ev\=aram} renvoie \`a un espace sacr\'e de culte\index{gnl}{culte} et non aux hymne\index{gnl}{hymne}s. Ensuite, il suppose que ce mot d\'erive du sk. \textit{dev\=ag\=ara} (\textit{deva} \og dieu\index{gnl}{dieu}\fg\ et \textit{ag\=ara} \og demeure\fg) et qu'il a \'et\'e prakritis\'e en \textit{dev\=aram}, \`a l'instar de \textit{bh\=a\d n\d d\=ag\=ara} devenu \textit{bh\=a\d n\d d\=aram} et de \textit{ko\d s\d th\=ag\=ara} devenu \textit{ko\d t\d t\=aram}, signifiant tous deux \og trésorerie\index{gnl}{tresorerie@trésorerie}, magasin\fg. Cependant, il conclut, sans transition ni explication, que l'image du \textit{dev\=aradevar} de SII 2 38 est celle du culte\index{gnl}{culte} priv\'e du roi\index{gnl}{roi}\footnote{Nous remercions Dominic \textsc{Goodall} d'avoir attir\'e notre attention sur cette \'etude.}!
%\og the singing of the hymns at the time of the private worship in the palace or at the mutt\fg.
Nous ajoutons une troisi\`eme inscription que \textsc{Rangaswamy} ne cite pas, SII 8 769, qui reproduit un ordre\index{gnl}{ordre royal} royal de Kulottu\.nga II\index{gnl}{Kulottu\.nga II} en 1135. Ce dernier remplace le groupe de chanteurs aveugles du temple de Tiruv\=amatt\=ur\index{gnl}{Tiruvamattur@Tiruv\=amatt\=ur} par un groupe pr\'esid\'e par le chanteur de son \textit{T\=ev\=aram}\index{gnl}{Tevaram@\textit{T\=ev\=aram}}\footnote{l. 7-8: \textit{nam tev\=arattu tiruppatiyam p\=a\d tum poyy\=ataceva\d ti tevaka\d n\=atan\=ana ir\=ajar\=ajap picca\b nukkum}, \og et pour Poyy\=ataceva\d ti Tevaka\d n\=atan alias Ir\=ajar\=ajappicca\b n de notre \textit{T\=ev\=aram}\index{gnl}{Tevaram@\textit{T\=ev\=aram}} qui chante les hymne\index{gnl}{hymne}s sacr\'es\fg.}.
Ainsi, il nous semble que le terme \textit{T\=ev\=aram}\index{gnl}{Tevaram@\textit{T\=ev\=aram}} peut parfaitement renvoyer, dans le discours royal du \textsc{xi}\up{e} et \textsc{xii}\up{e} si\`ecle, \`a un espace de culte\index{gnl}{culte}, du palais royal ou du monast\`ere\index{gnl}{monastère}, o\`u sont chant\'es les \textit{tiruppatiyam}\index{gnl}{tiruppatiyam@\textit{tiruppatiyam}}. Serait-ce la raison pour laquelle R\=ajar\=aja I\index{gnl}{Rajaraja I@R\=ajar\=aja I} installe les images des \textit{m\=uvar}\index{gnl}{muvar@\textit{m\=uvar}} avec son \og dev\=aradevar\fg, la divinit\'e de son \textit{T\=ev\=aram}\index{gnl}{Tevaram@\textit{T\=ev\=aram}}\string?

Puis, \textsc{Rangaswamy} pr\'esente une \'epigraphe du \textsc{xii}\up{e} si\`ecle (ARE 1921 39 et part II, para. 33, datant de 1110 sous Kulottu\.nga I\index{gnl}{Kulottu\.nga I}) o\`u le terme coupl\'e avec le verbe \textit{cey}- \og faire\fg\ d\'enoterait, toujours sur la base de l'interpr\'etation de culte\index{gnl}{culte} priv\'e, un culte\index{gnl}{culte} individuel donn\'e au dieu\index{gnl}{dieu} par le roi\index{gnl}{roi} par opposition au culte\index{gnl}{culte} public des temples. Le texte non publi\'e mentionnerait \textit{tiruv\=u\b ral perum\=a\b nait t\=ev\=aram ceytu}\footnote{Passage donn\'e par \textsc{Rangaswamy} (*1990 [1958]: 29).} dont le sujet serait le roi\index{gnl}{roi} et l'objet le seigneur (\`a l'accusatif) de Tiruv\=u\b ral\index{gnl}{Tiruvural@Tiruv\=u\b ral}. Or, compte tenu du rapport \'etabli ci-dessus entre le \textit{T\=ev\=aram}\index{gnl}{Tevaram@\textit{T\=ev\=aram}}, le culte\index{gnl}{culte} et le chant, il nous semble que le verbe compos\'e transitif \textit{t\=ev\=aram cey}- pourrait prendre le sens d'\og adorer (une divinit\'e)\fg, voire de (la) \og chanter\fg\footnote{Par ailleurs, selon \textsc{Rangaswamy}, ce sens figure dans l'\textit{\=Ekamparan\=atar Ul\=a} des Ira\d t\d taiyar au \textsc{xiv}\up{e} si\`ecle.}.

\textsc{Rangaswamy} poursuit son argumentation en l'illustrant avec une inscription du \textsc{xiii}\up{e} si\`ecle de K\=opperu\~nci\.nka I\index{gnl}{Kopperuncinka I@K\=opperu\~nci\.nka I} (EI 23 27) qui pr\'ecise l.~6: \textit{vi\d la\.nku cempo\b ni\b nampalakk\=uttu n[\=\i]y virumpiya tev\=aram}\footnote{Lecture personnelle \'etablie \`a partir de l'examen du fac-simil\'e de l'estampage pr\'esent\'e dans la publication.}, \og le \textit{k\=uttu} de la brillante salle d'or pur est le \textit{tev\=aram} que tu aimes\fg. Il d\'efend avec obstination que \textit{T\=ev\=aram}\index{gnl}{Tevaram@\textit{T\=ev\=aram}}, qui se r\'ef\`ere ici \`a la danse\index{gnl}{danse} (\textit{k\=uttu}) du \'Siva\index{gnl}{Siva@\'Siva} de Citamparam\index{gnl}{Citamparam}, est la divinit\'e priv\'ee du roi\index{gnl}{roi}. Or, il est tout aussi possible de consid\'erer, selon nous, que le terme renvoie \`a la danse\index{gnl}{danse} de \'Siva\index{gnl}{Siva@\'Siva} qui est le refuge\index{gnl}{refuge} du roi\index{gnl}{roi}, son temple ou de supposer qu'il se r\'ef\`ere \`a la repr\'esentation d'une \oe uvre intitul\'ee ainsi, jou\'ee, chant\'ee ou dansé\index{gnl}{danser}e devant le roi\index{gnl}{roi}.

Enfin, \textsc{Rangaswamy} termine sa d\'emonstration du domaine \'epigraphique avec ARE 1911 158 (relev\'ee aussi dans ARE 1902 608 et publi\'ee dans SII 8 205), sous R\=ajar\=aja III\index{gnl}{Rajaraja III@R\=ajar\=aja III} en 1244, qui mentionne un monast\`ere\index{gnl}{monastère} nomm\'e Tirumu\b rait-t\=ev\=arac-celva\b n\index{gnl}{Tirumuraittevaraccelvan@Tirumu\b raitt\=ev\=araccelva\b n}. Il soutient que le nom propre T\=ev\=araccelva\b n\index{gnl}{Tirumuraittevaraccelvan@Tirumu\b raitt\=ev\=araccelva\b n} renvoie soit \`a l'image d'un culte\index{gnl}{culte} priv\'e soit \`a la personne qui g\`ere le monast\`ere\index{gnl}{monastère}, s'occupe du culte\index{gnl}{culte} priv\'e et dont le titre serait \textit{tirumu\b rai}\index{gnl}{Tirumurai@\textit{Tirumu\b rai}}. Or cette \'epigraphe traite d'un monast\`ere\index{gnl}{monastère} qui se trouve au nord du temple de T\=o\d nipuram\index{gnl}{Tonipuram@T\=o\d nipuram} \`a Ka\b lumalam\index{gnl}{Kalumalam@Ka\b lumalam}, \textit{i.e.} C\=\i k\=ali --- ARE 1918 10 l.~1 sous R\=ajendra III\index{gnl}{Rajendra III@R\=ajendra III} (1250) vient confirmer ce fait --- et \textit{tirumu\b rai}\index{gnl}{Tirumurai@\textit{Tirumu\b rai}} d\'esigne d\`es 1136 un corpus\index{gnl}{corpus} d'hymne\index{gnl}{hymne}s chant\'es (voir CEC 26). Une inscription plus tardive de C\=\i k\=ali, CEC 17 de 1391, nous informe de l'existence d'un monast\`ere\index{gnl}{monastère} de Campantar\index{gnl}{Campantar} au nord du temple (l. 2). Il n'est pas n\'ecessaire de rappeler que C\=\i k\=ali est le lieu de naissance\index{gnl}{naissance} de Campantar\index{gnl}{Campantar}, que celui-ci a un temple sp\'ecifique sur ce site et que grand nombre de monast\`ere\index{gnl}{monastère}s sont nomm\'es d'apr\`es les \textit{m\=uvar}\index{gnl}{muvar@\textit{m\=uvar}} (\textsc{Swamy} 1972: 113-118). Ainsi, il nous para\^it vraisemblable que T\=ev\=araccelva\b n\index{gnl}{Tirumuraittevaraccelvan@Tirumu\b raitt\=ev\=araccelva\b n} soit un des qualificatifs de Campantar\index{gnl}{Campantar} et que le \textit{tirumu\b rai}\index{gnl}{Tirumurai@\textit{Tirumu\b rai}} qui le pr\'ec\`ede souligne sa contribution \`a ce corpus\index{gnl}{corpus}. Si le terme \textit{T\=ev\=aram}\index{gnl}{Tevaram@\textit{T\=ev\=aram}} peut prendre le sens d'hymne\index{gnl}{hymne} ou de chant au \textsc{xiii}\up{e} si\`ecle, Campantar\index{gnl}{Campantar} serait le \og Fortun\'e des hymne\index{gnl}{hymne}s du \textit{Tirumu\b rai}\index{gnl}{Tirumurai@\textit{Tirumu\b rai}}\fg\footnote{Il n'est plus \'etonnant que le donateur U\d taiyan\=ayaka\b n alias Tev\=arama\b lakiy\=a\b n V\=a\d nar\=aja\b n, mentionn\'e dans ARE 1928-29 228 sous R\=ajar\=aja III\index{gnl}{Rajaraja III@R\=ajar\=aja III} (1226), qui installe les images des poète\index{gnl}{poete@poète}s \=A\d tko\d n\d tan\=ayaka\b n (Cuntarar\index{gnl}{Cuntarar}) et Tiruv\=atav\=ur Perum\=a\d l (M\=a\d nikkav\=acakar\index{gnl}{Manikkavacakar@M\=a\d nikkav\=acakar}), porte dans son titre un qualificatif de Campantar\index{gnl}{Campantar}: Tev\=arama\b lakiy\=a\b n, le \og Beau des hymne\index{gnl}{hymne}s\fg.}.
Une autre inscription soutient cette hypoth\`ese. ARE 1924 208, qui date de la neuvi\`eme ann\'ee de Ja\d t\=avarman V\=\i rap\=a\d n\d dyadeva\index{gnl}{Jatavarman V@Ja\d t\=avarman V\=\i rap\=a\d n\d dyadeva} alias Tribhuvanacakravartin V\=\i rap\=a\d n\d dyadeva, probablement du \textsc{xiii}\up{e} si\`ecle\footnote{\textsc{Mahalingam} (1991a: 334) propose, avec incertitude, la date du 22 f\'evrier 1198 alors que \textsc{Swamy} (1972: 103) pense \`a 1260.}, enregistre un accord entre les employ\'es du temple de Pir\=a\b nmalai\index{gnl}{Piranmalai@Pir\=a\b nmalai} (Ramanathapuram\index{gnl}{Ramanathapuram dt.} dt.) et un donateur qui stipule que le temple veillerait aux offrandes et culte\index{gnl}{culte} de l'image de Tiru\~n\=anam-pe\b r\b ra-pi\d l\d laiy\=ar (Campantar\index{gnl}{Campantar}) que ce donateur a install\'ee, ainsi qu'aux offrandes sp\'eciales, aux récitation\index{gnl}{recitation@récitation}s du \textit{T\=ev\=aram}\index{gnl}{Tevaram@\textit{T\=ev\=aram}}, aux lampe\index{gnl}{lampe}s perp\'etuelles et aux procession\index{gnl}{procession}s de la divinit\'e les jours de f\^ete\index{gnl}{fete@fête}. Si le r\'esum\'e de l'ARE est juste\footnote{Pour le moment cette inscription non publi\'ee ne peut faire autorit\'e parce que nous ne disposons pas de son texte.}, le terme \textit{T\=ev\=aram}\index{gnl}{Tevaram@\textit{T\=ev\=aram}} s'applique clairement \`a des hymne\index{gnl}{hymne}s chant\'es et li\'es \`a Campantar\index{gnl}{Campantar} d\`es le \textsc{xiii}\up{e} si\`ecle.

Les t\'emoignages litt\'eraires post\'erieurs (\textsc{xiv-xv}\up{e} si\`ecles) propos\'es et interpr\'et\'es sur la base de culte\index{gnl}{culte} priv\'e par \textsc{Rangaswamy} sont parfaitement adaptables et compr\'ehensibles si le terme \textit{T\=ev\=aram}\index{gnl}{Tevaram@\textit{T\=ev\=aram}} a le sens d'hymne\index{gnl}{hymne} chant\'e.

Ainsi, la notion initiale de culte\index{gnl}{culte} priv\'e formul\'ee par \textsc{Rangaswamy} remise en cause, nous sugg\'erons pour le terme \textit{T\=ev\=aram}\index{gnl}{Tevaram@\textit{T\=ev\=aram}}, dans les inscriptions, une autre \'evolution s\'emantique\footnote{Toutefois, il ne faut pas oublier que toutes les donn\'ees \'epigraphiques ne sont pas exploit\'ees et que notre analyse n'est fond\'ee que sur les quelques \'el\'ements disponibles.}: un espace de culte\index{gnl}{culte} (\textsc{x}\up{e} si\`ecle) li\'e \`a la royaut\'e (\textsc{x}\up{e} et \textsc{xi}\up{e}) et au chant des \textit{tiruppatiyam}\index{gnl}{tiruppatiyam@\textit{tiruppatiyam}} (\textsc{xi}\up{e} et \textsc{xii}\up{e}); son sens se restreindrait ensuite aux hymne\index{gnl}{hymne}s m\^emes (\textsc{xiii}\up{e}) pour d\'esigner, finalement, de fa\c con sectaire\index{gnl}{sectaire}\footnote{Il appara\^it que ce sectarisme est un fait tardif parce que Kulottu\.nga I\index{gnl}{Kulottu\.nga I} aurait chant\'e un Vi\d s\d nu\index{gnl}{Visnu@Vi\d s\d nu} (ARE 1921 39), parce que le commentaire du \textit{Tiruv\=aymo\b li} utilise ce terme (\textsc{Rangaswamy} *1990 [1958]: 30), et parce que I\b rai A\d n\d na\b n fait un don\index{gnl}{don} au temple de \'Sr\=\i ra\.nkam\index{gnl}{Srirankam@\'Sr\=\i ra\.nkam} (SII 24 262).}, l'ensemble des hymne\index{gnl}{hymne}s des \textit{m\=uvar}\index{gnl}{muvar@\textit{m\=uvar}} (\textsc{xvi}\up{e}).

\section{L'histoire}

Les dates de l'histoire de la litt\'erature tamoule sont encore des sujets controvers\'es\footnote{Par exemple, pour un r\'esum\'e pertinent des diff\'erentes th\`eses sur la datation des poète\index{gnl}{poete@poète}s vishnouite\index{gnl}{vishnouite}s tamouls, les \textit{\=a\b lv\=ar}, et leur limite; voir les critiques de \textsc{Hardy} (*2001 [1983]: 262, n. 69).}. Celles des auteurs et des textes du \textit{T\=ev\=aram}\index{gnl}{Tevaram@\textit{T\=ev\=aram}} en sont de belles illustrations\footnote{Nous distinguons les dates des textes du corpus\index{gnl}{corpus} de celles des trois auteurs qui correspondraient \`a des figures litt\'eraires plut\^ot qu'\`a des personnes historiques (\textsc{Shulman} *2001 [1993]). Cependant, sous le poids de la tradition\index{gnl}{tradition} et pour des raisons pratiques, nous continuons \`a \'ecrire que les auteurs du corpus\index{gnl}{corpus} sont les \textit{m\=uvar}\index{gnl}{muvar@\textit{m\=uvar}}.}. \textsc{Gros} (1984), dans sa dense introduction, rappelle les diverses hypoth\`eses et conclusions formul\'ees sur ce sujet avant de replacer le corpus\index{gnl}{corpus} dans son contexte litt\'eraire, historique et religieux pour finalement suivre les datations propos\'ees par la tradition\index{gnl}{tradition}, qui situent Campantar\index{gnl}{Campantar} et Appar\index{gnl}{Appar} au \textsc{vii}\up{e} si\`ecle puis Cuntarar\index{gnl}{Cuntarar} au \textsc{viii-ix}\up{e} si\`ecle. Apr\`es lui, \textsc{Peterson} (*1991 [1989]: 19), entre autres, a affin\'e la fourchette\footnote{\og It appears most likely that the lives of Appar\index{gnl}{Appar} and Campantar\index{gnl}{Campantar} overlapped, and that they lived between \textsc{a.d.} 570 and 670. The most plausible date for Cuntarar\index{gnl}{Cuntarar} is the end of the seventh and the beginning of the eighth centuries.\fg}, de mani\`ere toutefois peu persuasive car fond\'ee sur l'acceptation de l'identification de personnages contemporains des auteurs du \textit{T\=ev\=aram}\index{gnl}{Tevaram@\textit{T\=ev\=aram}} dans le \textit{Periyapur\=a\d nam}\index{gnl}{Periyapuranam@\textit{Periyapur\=a\d nam}}, comme des acteurs historiques. Mais \textsc{Tieken} (2001) relance la pol\'emique en pla\c cant les d\'ebuts de la po\'esie de \textit{bhakti}\index{gnl}{bhakti@\textit{bhakti}} \`a la fin du \textsc{ix}\up{e} et au d\'ebut du \textsc{x}\up{e} si\`ecle. Il se fonde pour cela essentiellement sur la nouvelle datation de la p\'eriode du \textit{Ca\.nkam}\index{gnl}{Cankam@\textit{Ca\.nkam}} qu'il a proposée\footnote{Pour une critique de son ouvrage, surtout des chapitres portant sur la litt\'erature du \textit{Ca\.nkam}\index{gnl}{Cankam@\textit{Ca\.nkam}}, cf. \textsc{Wilden} 2002; et pour des \'el\'ements de r\'eponse, cf., notamment, \textsc{Tieken} 2004.}. S'il n'est pas encore possible de donner une datation exacte, il nous para\^it toutefois probable que les \textit{m\=uvar}\index{gnl}{muvar@\textit{m\=uvar}}, ou du moins la plupart des poème\index{gnl}{poeme@poème}s qui leur sont attribu\'es, aient été composés avant les premi\`eres r\'ef\'erences \'epigraphiques des \textit{tiruppatiyam}\index{gnl}{tiruppatiyam@\textit{tiruppatiyam}} chant\'es dans les temples shiva\"ite\index{gnl}{shiva\"ite}s au \textsc{ix}\up{e} si\`ecle car ceux-ci, même s'ils ne sont pas précisément identifi\'es sont susceptibles de renvoyer aux hymne\index{gnl}{hymne}s qui ont form\'e le \textit{T\=ev\=aram}\index{gnl}{Tevaram@\textit{T\=ev\=aram}}.
Nous proposons donc de dater les hymne\index{gnl}{hymne}s attribués à Cuntarar\index{gnl}{Cuntarar} dans la seconde moiti\'e du \textsc{ix}\up{e} si\`ecle, au plus tard, et de dater, avec une relative ant\'eriorit\'e par rapport à ceux-ci, les poèmes attribués à Campantar\index{gnl}{Campantar} et à Appar\index{gnl}{Appar}.
Le caractère postérieur des poème\index{gnl}{poeme@poème}s attribu\'es \`a Cuntarar\index{gnl}{Cuntarar} se d\'eduit par leurs allusions aux autres auteurs comme dans le c\'el\`ebre \og Recueil des saints serviteurs\fg, le \textit{Tirutto\d n\d tattokai}\index{gnl}{Tiruttontattokai@\textit{Tirutto\d n\d tattokai}} (VII 39 4a et 5ab)\footnote{Ce poème\index{gnl}{poeme@poème} en onze quatrains \'evoque les soixante-trois \og ma\^itres\fg\ (\textit{n\=aya\b nm\=ar}\index{gnl}{nayanmar@\textit{n\=aya\b nm\=ar}}) les d\'evots shiva\"ite\index{gnl}{shiva\"ite}s exemplaires, ainsi que neuf groupes de d\'evots. Pour trois traductions int\'egrales de cet hymne\index{gnl}{hymne}, cf. \textsc{Marr} (1979: 271-272), \textsc{Peterson} (*1991 [1989]: 331-336) et \textsc{Shulman} (1990: 239-248).}, ainsi que, par une th\'ematique novatrice qui souligne une \'evolution socio-religieuse: la revendication d'une identit\'e\index{gnl}{identit\'e} shiva\"ite\index{gnl}{shiva\"ite} tamoule exprim\'ee, entre autres, par une critique virulente des hérétique\index{gnl}{heretique@hérétique}s dans les hymne\index{gnl}{hymne}s attribu\'es \`a Appar\index{gnl}{Appar} et \`a Campantar\index{gnl}{Campantar} s'efface devant la constitution d'une soci\'et\'e shiva\"ite\index{gnl}{shiva\"ite} d\'efinie par des listes de sites sacr\'es et de d\'evots exemplaires (\textsc{Gros} 2001: 21).\\

Les t\'emoignages \'epigraphiques sur l'institution des hymne\index{gnl}{hymne}s du \textit{T\=ev\=aram}\index{gnl}{Tevaram@\textit{T\=ev\=aram}} et d'autres poème\index{gnl}{poeme@poème}s, dans la vie cultuelle des temples sont multiples\footnote{Notre point de d\'epart est l'article de \textsc{Swamy} (1972) qui offre une base de donn\'ees riche et diversifi\'ee sur les chants\index{gnl}{chant}, les images, les culte\index{gnl}{culte}s et les monast\`ere\index{gnl}{monastère}s des \textit{n\=alvar}\index{gnl}{nalvar@\textit{n\=alvar}} dans les inscriptions.}. Ils semblent appara\^itre d\`es le \textsc{ix}\up{e} si\`ecle sous les Pallava\index{gnl}{Pallava}\footnote{L'inscription qui d\'etiendrait aujourd'hui le record d'anciennet\'e dans ce domaine est celle du temple de Tiruvallam\index{gnl}{Tiruvallam} (Ce\.nka\b rpa\d t\d tu\index{gnl}{Ce\.nka\b rpa\d t\d tu dt.} dt.) qui enregistre une donation de l'assemblée\index{gnl}{assemblée} villageoise pour payer les diff\'erents employ\'es du temple dont ceux qui chantent les \textit{tiruppatiyam}\index{gnl}{tiruppatiyam@\textit{tiruppatiyam}} (SII 3 43 l.~32-33). Elle date de la dix-septi\`eme ann\'ee de r\`egne de Vijayanantivikkiramapanmar identifi\'e comme Nandivarman III par \textsc{Gros} (1984: viii), soit de 863. Cependant, cette \'epigraphe est une copie d'un original qui a \'et\'e d\'etruit au moment de la r\'enovation du \textit{ma\d n\d dapa} (l.~1-2). L'authenticit\'e des informations qu'elle contient, surtout en ce qui concerne la datation, reste donc contestable.}, se d\'eveloppent et se propagent avec le r\`egne \textit{c\=o\b la}\index{gnl}{cola@\textit{c\=o\b la}} pour atteindre leur apog\'ee au \textsc{xiii}\up{e} dans tout le royaume tamoul. La grande majorit\'e des hymne\index{gnl}{hymne}s est appel\'ee \textit{tiruppatiyam}\index{gnl}{tiruppatiyam@\textit{tiruppatiyam}} sans distinction de textes, d'auteurs et m\^eme de sectes. Ainsi, les poème\index{gnl}{poeme@poème}s vishnouite\index{gnl}{vishnouite}s sont aussi commun\'ement d\'esign\'es par ce terme. Bien qu'il soit difficile d'identifier avec pr\'ecision la nature de ces \textit{tiruppatiyam}\index{gnl}{tiruppatiyam@\textit{tiruppatiyam}} qui peuvent finalement renvoyer \`a n'importe quel hymne\index{gnl}{hymne} d\'evotionnel\index{gnl}{devotionnel@dévotionnel} chant\'e dans un temple\index{gnl}{temple}, il est d'usage de consid\'erer qu'en contexte shiva\"ite\index{gnl}{shiva\"ite}, il s'agit des \oe uvres des \textit{m\=uvar}\index{gnl}{muvar@\textit{m\=uvar}} et autres figures sanctifi\'ees dont l'intronisation s'observe \`a partir du \textsc{xi}\up{e} si\`ecle\footnote{Quelques inscriptions laissent envisager la possibilit\'e qu'il existait des poète\index{gnl}{poete@poète}s \`a la renomm\'ee perdue et dont les chants\index{gnl}{chant} furent r\'ecit\'es dans les temples\index{gnl}{temple}. En effet, SII 22 333 l.~9-11 enregistre sous R\=ajar\=aja I\index{gnl}{Rajaraja I@R\=ajar\=aja I}, en 994-95, un don\index{gnl}{don} pour que soit chant\'e lors d'une f\^ete\index{gnl}{fete@fête} un \textit{tiruppatiyam}\index{gnl}{tiruppatiyam@\textit{tiruppatiyam}} d\'edi\'e au Vi\d s\d nu\index{gnl}{Visnu@Vi\d s\d nu} du temple\index{gnl}{temple} et compos\'e par le p\`ere du donateur (cf. \textsc{Rangaswamy} *1990 [1958]: 18). Cependant, nous envisageons aussi l'hypoth\`ese, l\'egitime et s\'eduisante mais extr\^emement difficile \`a v\'erifier, que des textes d'auteurs inconnus aient \'et\'e \textit{a posteriori} arbitrairement distribu\'es aux poète\index{gnl}{poete@poète}s, \textit{personae} ou r\'eels, connus.}. Quelques textes sont d\'esign\'es par les noms sous lesquels ils sont connus actuellement, comme, le \textit{Tirutto\d n\d tattokai}\index{gnl}{Tiruttontattokai@\textit{Tirutto\d n\d tattokai}} attribu\'e \`a Cuntarar\index{gnl}{Cuntarar} (SII 4 223, sous R\=ajendra I\index{gnl}{Rajendra I@R\=ajendra I}), le \textit{Tirutt\=a\d n\d takam}\index{gnl}{Tiruttantakam@\textit{Tirutt\=a\d n\d takam}} d'Appar\index{gnl}{Appar} (ARE 1917 219) ou encore le \textit{Tiruvemp\=avai}\index{gnl}{Tiruvempavai@\textit{Tiruvemp\=avai}} de M\=a\d nikkav\=acakar\index{gnl}{Manikkavacakar@M\=a\d nikkav\=acakar} (ARE 1912 421).

Le vocabulaire qui r\'ef\`ere au chant et/ou \`a la récitation\index{gnl}{recitation@récitation} des \textit{tiruppatiyam}\index{gnl}{tiruppatiyam@\textit{tiruppatiyam}} est assez peu vari\'e: \textit{p\=a\d tu}-, \og chanter\fg, semble le mot le plus r\'epandu\footnote{Par exemple dans SII 3 43, 139, 151A; SII 8 260, 687; SII 19 69; SII 13 14, 50, 51, 74 et 141.}; \textit{vi\d n\d nappam cey}-, \og r\'eciter\fg, se rencontre aussi, dans SII 2 65 et 8 675 par exemple. Le terme \textit{\=otu}-, attest\'e relativement t\^ot pour les récitation\index{gnl}{recitation@récitation}s de textes sanskrits (SII 14 81 date de 954), ne semble \^etre employ\'e que tardivement en contexte tamoul, notamment lors du chant du \textit{Tirumu\b rai}\index{gnl}{Tirumurai@\textit{Tirumu\b rai}} (ARE 1908 454 et 1918 10 datent du \textsc{xiii}\up{e} si\`ecle). Il est \`a l'origine\index{gnl}{origine} de la d\'esignation des chanteurs professionnels de textes shiva\"ite\index{gnl}{shiva\"ite}s dans les temples\index{gnl}{temple}, les \textit{\=otuv\=ar}\index{gnl}{otuvar@\textit{\=otuv\=ar}}.
Il appara\^it qu'\`a date ancienne ce service\index{gnl}{service} du chant\index{gnl}{chant} \'etait assum\'e par diverses personnes:
des d\'evots (\textit{a\d tika\d lm\=ar}, SII 8 687, 13 51 et 141),
des asc\`etes\footnote{\textsc{Swamy} (1972: 105) souligne leur association au chant d'un texte particulier nomm\'e \textit{Tiru\~n\=a\b nam} en contexte monastique et ce, dans la r\'egion de Tirunelv\=eli au \textsc{xiii}\up{e} si\`ecle.}
et des groupes (SII 8 749\footnote{Cette inscription enregistre un ordre\index{gnl}{ordre royal} royal de Kulottu\.nga II\index{gnl}{Kulottu\.nga II} qui remplace le groupe de chanteurs aveugles du temple\index{gnl}{temple} par un groupe de seize personnes pr\'esid\'e par un chanteur de sa cour.}).
Une place substantielle \'etait accord\'ee aux femmes. Elles semblent entrer en sc\`ene d\`es le \textsc{x}\up{e} si\`ecle dans une inscription mentionnant que trois femmes ont \'et\'e offertes au temple\index{gnl}{temple} pour effectuer divers service\index{gnl}{service}s dont le chant des \textit{tiruppatiyam}\index{gnl}{tiruppatiyam@\textit{tiruppatiyam}} (ARE 1936-37 149). Plus tard, au \textsc{xiii}\up{e} si\`ecle, des danseuse\index{gnl}{danseuse}s d'un temple\index{gnl}{temple} de Nall\=ur\index{gnl}{Nallur@Nall\=ur} (Te\b n\b n\=a\b rk\=a\d tu dt.\index{gnl}{Te\b n\b n\=a\b rk\=a\d tu dt.}) ach\`etent le droit de chanter diff\'erents passages du \textit{Tiruvemp\=avai}\index{gnl}{Tiruvempavai@\textit{Tiruvemp\=avai}} et d'accompagner en dansant la procession\index{gnl}{procession} de la divinit\'e\footnote{Cf. \textsc{Swamy} (1972: 102) et \textsc{Orr} 2007, conf\'erence non publi\'ee.}.
Les chants\index{gnl}{chant} sont accompagn\'es g\'en\'eralement de musique\index{gnl}{musique} instrumentale, souvent \`a percussion\footnote{Les plus communs sont les cymbales \textit{t\=a\d lam} (SII 13 51), le tambour-sablier \textit{u\d tukkai} (SII 2 65, 13 51) et un gros tambour \textit{ko\d t\d timatta\d lam} (SII 2 65).}.
Le nombre de chanteurs variait de un (SII 19 69), deux (SII 13 74), trois (SII 8 687), quatre (SII 13 14) et seize (SII 8 749) \`a quarante-huit dans la grandiose inscription de R\=ajar\=aja I\index{gnl}{Rajaraja I@R\=ajar\=aja I} (985-1014) \`a Ta\~nc\=av\=ur\index{gnl}{Tancavur@Ta\~nc\=av\=ur} (SII 2 65)\footnote{Les noms des chanteurs, \textsc{Hultzsch} le souligne en introduction, sont souvent d\'eriv\'es de ceux des \textit{m\=uvar}\index{gnl}{muvar@\textit{m\=uvar}} ou d'autres d\'evots shiva\"ite\index{gnl}{shiva\"ite}s. Ils portent tous probablement un titre d'initiation d\'enot\'e par le suffixe -\textit{\'siva\b n}.}.
Ils \'etaient r\'emun\'er\'es avec des terre\index{gnl}{terre}s (SII 3 139, 8 687, 19 69, 13 50 , 51 et 74 et 141) ou du riz\index{gnl}{riz} non-d\'ecortiqu\'e \textit{nellu} (SII 2 65, 3 151A, 13 14).
Les donateurs qui instauraient ou promouvaient ce service\index{gnl}{service} appartenaient \`a des classes tr\`es vari\'ees. Les dons\index{gnl}{don} de la famille royale (SII 2 65, 13 14) ne sont pas majoritaires. Souvent, c'est l'importance de la localit\'e qui est soulign\'ee dans ces transactions. Les chef\index{gnl}{chef}s locaux\index{gnl}{local} (SII 13 50, 51, 74, 141, SII 19 69, ARE 1927-28 93), les assemblées\index{gnl}{assemblée} villageoises (ARE 1908 423, 1922 224) et les autorit\'es du temple\index{gnl}{temple} (ARE 1940-41 143, 161) paraissent \^etre les donateurs les plus fr\'equents.
Les chants\index{gnl}{chant} s'effectuaient \`a l'occasion de culte\index{gnl}{culte}s quotidiens (ARE 1922 224, SII 13 141) mais aussi de f\^etes\index{gnl}{fete@fête} (ARE 1912 421, SII 4 223).
Certaines inscriptions mentionnent une pièce\index{gnl}{piece@pièce} particuli\`ere o\`u les \textit{tiruppatiyam}\index{gnl}{tiruppatiyam@\textit{tiruppatiyam}} \'etaient chant\'es. Le \textit{tirukkaikk\=o\d t\d ti}\index{gnl}{tirukkaikkotti@\textit{tirukkaikk\=o\d t\d ti}}\footnote{Voir CEC 26.} (ARE 1908 203, 414, 454, ARE 1928-29 350), et le \textit{tiruppa\d l\d liya\b rai}\index{gnl}{tiruppalliyarai@\textit{tiruppa\d l\d liya\b rai}} (ARE 1918 10) sont g\'en\'eralement ce lieu. Notons qu'un \og service\index{gnl}{service}\fg\ distinctif \'etait attach\'e \`a cette pièce\index{gnl}{piece@pièce}, \textit{tirukkaikk\=o\d t\d tipa\d ni}, dans le temple\index{gnl}{temple} de Tiruvallam\index{gnl}{Tiruvallam} (SII 4 309, Va\d t\=a\b rk\=a\d tu dt.\index{gnl}{Va\d t\=a\b rk\=a\d tu dt.}).

Pour clore cette br\`eve pr\'esentation de l'histoire des \textit{tiruppatiyam}\index{gnl}{tiruppatiyam@\textit{tiruppatiyam}} \`a travers des donn\'ees \'epigraphiques, il semble important de souligner l'une des conclusions et surtout, les interrogations formul\'ees par \textsc{Orr} (2007) concernant la répartition g\'eographique de ces chants\index{gnl}{chant}. Dans son \'etude d\'epouillant cent quatre-vingt inscriptions \textit{c\=o\b la}\index{gnl}{cola@\textit{c\=o\b la}} de temples\index{gnl}{temple} shiva\"ite\index{gnl}{shiva\"ite}s et vishnouite\index{gnl}{vishnouite}s, Leslie \textsc{Orr} constate qu'il existe un lien tr\`es faible entre les sites dont l'\'epigraphie mentionne le chant des \textit{tiruppatiyam}\index{gnl}{tiruppatiyam@\textit{tiruppatiyam}} et ceux connus pour avoir \'et\'e c\'el\'ebr\'es dans le corpus\index{gnl}{corpus} du \textit{T\=ev\=aram}\index{gnl}{Tevaram@\textit{T\=ev\=aram}}, les fameux \textit{p\=a\d tal pe\b r\b ra talam}:

\scriptsize
\begin{quote}
Quite contrary to what I expected to find, the places that are sung in the hymns --- the 276 \textit{p\=a\d tal pe\b r\b ra talam} and the 108 \textit{divyade\'sas} --- are not in general places where the hymns were sung. And vice versa. Were people in medieval Tamilnadu even aware that their local\index{gnl}{local} temple\index{gnl}{temple} had a poem dedicated to it\string? In the apparent absence of such knowledge --- or absence of regard for such knowledge --- and in the absence of professional hymn-singers, how in fact were these hymns transmitted\string? Were the poems of \textit{T\=ev\=aram}\index{gnl}{Tevaram@\textit{T\=ev\=aram}}, \textit{Tiruv\=acakam}\index{gnl}{Tiruvacakam@\textit{Tiruv\=acakam}} and \textit{Divyaprabandham} largely known in literary rather than performance circles\string? Were there just a few hymns that gained popularity and were taken up for performance at temples\index{gnl}{temple}\string?
\end{quote}

\normalsize
\begin{center}
*
\end{center}

Le \textit{T\=ev\=aram}\index{gnl}{Tevaram@\textit{T\=ev\=aram}} est notre source principale. Ce texte fondateur de l'identit\'e\index{gnl}{identit\'e} shiva\"ite\index{gnl}{shiva\"ite} tamoule pr\'esente aujourd'hui une forme parfaitement organis\'ee et tellement fig\'ee par la tradition\index{gnl}{tradition} qu'il n'en n'existe pas d'\'edition critique. Malgr\'e une bibliographie abondante, de nombreuses zones d'ombre demeurent, par exemple, sur la datation des auteurs, l'attribution des hymne\index{gnl}{hymne}s ou sur l'histoire de la mise en forme. Notre \'etude se concentre dorénavant sur la figure du poète\index{gnl}{poete@poète} Campantar\index{gnl}{Campantar} \`a travers ce corpus\index{gnl}{corpus} bien plus complexe que son apparence homog\`ene ne pourrait le faire croire.

\chapter{Campantar le poète}


Les trois premiers livres du corpus\index{gnl}{corpus} actuel du \textit{T\=ev\=aram}\index{gnl}{Tevaram@\textit{T\=ev\=aram}} sont attribu\'es \`a Campantar\index{gnl}{Campantar}. Le premier livre contient cent trente-six hymne\index{gnl}{hymne}s, le deuxi\`eme cent vingt-deux et le troisi\`eme cent vingt-sept, soit un total de trois cent quatre-vingt-cinq. La pr\'esentation des particularit\'es des poème\index{gnl}{poeme@poème}s de Campantar\index{gnl}{Campantar} permettra de nous concentrer ensuite sur l'une d'elles, \`a savoir les envois\index{gnl}{envoi} qui fournissent une quantit\'e substantielle d'\'el\'ements biographiques\index{gnl}{biographie!biographique}, et enfin, de consid\'erer les allusions dites autobiographique\index{gnl}{autobiographique}s qui pars\`ement l'\oe uvre de Campantar.

\section{Le \textit{T\=ev\=aram} de Campantar}

Il est d'usage de penser que les hymne\index{gnl}{hymne}s attribu\'es \`a Campantar\index{gnl}{Campantar} se caract\'erisent par un style simple, formulaire et d\'epourvu de lyrisme. Nous constatons, en effet, que les r\'ep\'etitions, les refrains et les sch\'emas syntaxiques fixes abondent dans son corpus\index{gnl}{corpus}. Ne s'agit-il pas pr\'ecis\'ement d'\'el\'ements qui favorisent la m\'emorisation et la compr\'ehension des textes bhaktiques? Ces poème\index{gnl}{poeme@poème}s louant un \'Siva\index{gnl}{Siva@\'Siva} pr\'esent \`a tel ou tel endroit dans un langage abordable sont ainsi rendus accessibles \`a un grand nombre: le chant de \'Siva\index{gnl}{Siva@\'Siva} est ancr\'e dans l'ensemble du sol tamoul pour qu'il soit entendu et compris de tous. Toutefois, le corpus\index{gnl}{corpus}, t\'emoignant, \c c\`a et l\`a, de la nostalgie des conventions d'antan et usant d'une rh\'etorique nouvelle, présente une certaine h\'et\'erogénéité. Pour souligner la diversit\'e, voire parfois, la dissonance entre les hymne\index{gnl}{hymne}s attribu\'es \`a Campantar\index{gnl}{Campantar}, nous nous attarderons d'abord sur leur structure, puis sur l'influence des th\`emes amoureux du \textit{Ca\.nkam}\index{gnl}{Cankam@\textit{Ca\.nkam}} qu'ils trahissent et enfin, sur les jeux litt\'eraires que nous y rencontrons.

\subsection{La structure}

Chaque hymne\index{gnl}{hymne} de Campantar\index{gnl}{Campantar} est g\'en\'eralement compos\'e selon une structure d\'efinie. De fa\c con simplifi\'ee et sch\'ematique, un poème\index{gnl}{poeme@poème} contient onze quatrains dont les quatre derniers ont syst\'ematiquement la m\^eme fonction: ainsi, la huiti\`eme strophe est consacr\'ee au mythe\index{gnl}{mythe} du démon\index{gnl}{demon@démon} R\=ava\d na\index{gnl}{Ravana@R\=ava\d na} qui soul\`eve le Kail\=asa\index{gnl}{Kailasa@Kail\=asa}, qui se fait \'ecraser par \'Siva\index{gnl}{Siva@\'Siva} et qui finalement le chante et lui joue m\^eme de la musique\index{gnl}{musique} sur les tendons de ses bras.
La neuvi\`eme raconte le mythe\index{gnl}{mythe} du Li\.ngodbhava o\`u Vi\d s\d nu\index{gnl}{Visnu@Vi\d s\d nu} sous l'apparence d'un sanglier et Brahm\=a\index{gnl}{Brahma@Brahm\=a} sous celle d'un oiseau\index{gnl}{oiseau} \textit{hamsa} cherchent en vain, respectivement, les pieds et la t\^ete de \'Siva\index{gnl}{Siva@\'Siva}, qui a pris la forme d'une colonne de feu\index{gnl}{feu} et qui, signifie ainsi, sa sup\'eriorit\'e absolue.
La dixi\`eme strophe est une critique vive des asc\`etes ja\"in\index{gnl}{jain@ja\"in}s et bouddhiste\index{gnl}{bouddhiste}s; cette vitup\'eration porte aussi bien sur leur doctrine que sur leur coutume\footnote{Reprenons notre analyse de cette dixi\`eme strophe \`a partir de cinq poème\index{gnl}{poeme@poème}s de Campantar\index{gnl}{Campantar} d\'edi\'es \`a C\=\i k\=a\b li\index{gnl}{Cikali@C\=\i k\=a\b li}: \og les renon\c cants ja\"in\index{gnl}{jain@ja\"in}s sont appel\'es \textit{ama\d nar} (I-74, 10) `ceux qui sont nus' et \textit{cama\d nar} (I-24,10) du sanskrit \textit{\'srama\d na} qui dans un contexte non-tamoul d\'esigne une personne qui accomplit des aust\'erit\'es, un moine mendiant aussi bien bouddhique que ja\"in\index{gnl}{jain@ja\"in}. Les moines bouddhiste\index{gnl}{bouddhiste}s sont, ici, les \textit{t\^erar} (I-9,9) qui appartiennent au Therav\=ada, la branche du petit v\'ehicule. Ils sont aussi les \textit{c\^akkiyar} (I-24,10) du nom du clan de Buddha \'S\^akyamuni. Ils n'honorent pas (\textit{va\d na\.nk\^amai}~/ I-9,9) les pieds de \'Siva\index{gnl}{Siva@\'Siva} qui est d\'ecrit comme leur ennemi : il ignore leur dogme (\textit{uraiyai vi\d t\d t\^ar}~/ I-24,10), les condamne en le d\'etruisant (\textit{k\^olum mo\b lika\d l o\b liya}~/ I-74,10) et r\'eduit au silence (\textit{v\^ay ma\d tiya}~/ III-100,6) ces moines hérétique\index{gnl}{heretique@hérétique}s. Leur doctrine fausse et obscure (\textit{karakkum urai}~/ I-24,10) ne doit pas \^etre suivie, voire prise en consid\'eration (\textit{collum antara\~n\^anamell\^am avai \^or poru\d l e\b n\b n\^el}~/ I-104,10). Les renon\c cants ja\"in\index{gnl}{jain@ja\"in}s sont nus, ils ne portent pas de cache-sexe (\textit{viri k\^ova\d nam n\^itt\^ar}~/ I-104,10), vivent d'aum\^one (\textit{k\^ocaram}~/ III-100,6) et portent comme attributs la cruche \`a eau\index{gnl}{eau} des asc\`etes (\textit{ku\d n\d tikai}~/ III-100,6), une plume de paon (\textit{p\^ili}~/ I-74,10 et III-100,6) pour balayer en douceur les chemins qu'ils empruntent, et une natte (\textit{ta\d t\d tu}~/ III-100,6) sur laquelle ils s'assoient. Les bouddhiste\index{gnl}{bouddhiste}s sont v\^etus de leur habit monastique qui donne l'impression que leur corps est couleur safran (\textit{ven tuvar m\^e\b niyin\^ar}~/ I-104,10). Ils vivent aussi d'aum\^one d'eau\index{gnl}{eau} de riz\index{gnl}{riz} bouilli (\textit{ka\~nci}~/ III-100,6) qu'ils re\c coivent avec contentement (\textit{ma\b nam ko\d l}~/ III-100,6) dans un bol en forme de cr\^ane (\textit{ma\d n\d tai}~/ III-100,6). Ainsi, ils sont critiqu\'es non seulement pour leur doctrine mais aussi pour leur coutume. Les deux groupes sont vivement injuri\'es et d\'enigr\'es : ils sont sans intelligence (\textit{a\b rivu il}~/ I-74,10), leurs corps sont sales (\textit{m\^acu \^eriya u\d tal\^ar}~/ I-9,9), ils sont insignifiants, petits (\textit{ci\b ru}~/ I-74,10) et mauvais (\textit{ko\d l\d liyar}~/ III-100,6).\fg\ (\textsc{Veluppillai} 2003: 65).}.
Et enfin, la onzi\`eme strophe est l'envoi\index{gnl}{envoi}, le \textit{tirukka\d taikk\=appu}\index{gnl}{tirukkataikkappu@\textit{tirukka\d taikk\=appu}} \og protection finale\fg, dans lequel Campantar\index{gnl}{Campantar} est pr\'esent\'e \`a la troisi\`eme personne (voir 2.2.1). Nous traduisons, ci-dessous, en guise d'illustration, les quatre derni\`eres strophes de l'hymne\index{gnl}{hymne} inaugural du corpus\index{gnl}{corpus} effectif du \textit{T\=ev\=aram}\index{gnl}{Tevaram@\textit{T\=ev\=aram}}, compos\'e \`a la gloire de Piramapuram\index{gnl}{Piramapuram} (un des douze\index{gnl}{douze} noms de C\=\i k\=a\b li\index{gnl}{Cikali@C\=\i k\=a\b li}):

\scriptsize
\begin{verse}
\textit{viyar ila\.nku varai untiya t\=o\d lka\d lai v\=\i ram vi\d laivitta\\
uyar ila\.nkai araiya\b n vali ce\b r\b ru, e\b natu u\d l\d lam kavar ka\d lva\b n ---\\
tuyar ila\.nkum(m) ulakil pala \=u\b lika\d l too\b n\b rumpo\b lutu ell\=am\\
peyar ila\.nku piram\=apuram m\=eviya pemm\=a\b n --- iva\b n a\b n\b r\=e!} (I~1.8)\\
\textit{t\=a\d l nutal ceytu, i\b rai k\=a\d niya, m\=alo\d tu ta\d nt\=amaraiy\=a\b num,\\
n\=\i \d nutal ceytu o\b liya(n) nimirnt\=a\b n, e\b natu u\d l\d lam kavar ka\d lva\b n ---\\
v\=a\d lnutal cey maka\d l\=\i r mutal\=akiya vaiyattavar \=etta,\\
p\=e\d nutalcey piram\=apuram m\=eviya pemm\=a\b n --- iva\b n a\b n\b r\=e!} (I~1.9)\\
\textit{puttar\=o\d tu po\b ri il cama\d num pu\b ramk\=u\b ra, ne\b ri nill\=a\\
otta colla, ulakam pali t\=erntu, e\b natu u\d l\d lam kavar ka\d lva\b n ---\\
\og mattay\=a\b nai ma\b ruka(v), uri p\=orttatu orm\=ayam(m)itu!\fg\ e\b n\b na,\\
pittarp\=olum, piram\=apuram m\=eviya pemm\=a\b n --- iva\b n a\b n\b r\=e!} (I~1.10)\\
\textit{arune\b riya ma\b rai valla mu\b ni aka\b n poykai alar m\=eya,\\
peru ne\b riya, piram\=apuram m\=eviya pemm\=a\b n iva\b nta\b n\b nai,\\
oru ne\b riya ma\b namvaittu u\d nar \~n\=a\b nacampanta\b n(\b n) uraiceyta\\
tiru ne\b riya tami\b l vallavar tolvi\b nai t\=\i rtal e\d litu \=am\=e.} (I~1.11)\\
\end{verse}

\normalsize
\begin{verse}
Le voleur qui ravit mon for int\'erieur\\
A d\'etruit la force du roi\index{gnl}{roi} de la haute Ila\.nkai\index{gnl}{Srilanka!Ilankai@Ila\.nkai}\\
Qui a port\'e la montagne fameuse par sa grandeur\\
Et dont l'h\'ero\"isme fait croître ses épaule\index{gnl}{epaule@épaule}s;\\
N'est-ce pas lui,\\
Le seigneur qui vit \`a Piramapuram\index{gnl}{Piramapuram}\footnote{L'allongement, dans l'hymne\index{gnl}{hymne}, de la troisi\`eme syllabe du toponyme Piramapuram\index{gnl}{Piramapuram} r\'esulte de la m\'etrique; information de \textsc{T. V.~Gopal Iyer}.} dont la renomm\'ee brille,\\
Dans ce monde de souffrance,\\
Toutes les fois qu'apparaissent de multiples déluge\index{gnl}{deluge@déluge}s? (I~1.8)\\
\end{verse}

\begin{verse}
Le voleur qui ravit mon for int\'erieur\\
S'est dress\'e si bien que M\=al\index{gnl}{Visnu@Vi\d s\d nu!Mal@M\=al} et Celui du lotus frais,\\
Ayant explor\'e la base et le sommet\footnote{Litt\'eralement \og ayant fait les pieds et le front\fg\ (\textit{t\=a\d l nutal ceytu}).} pour voir le seigneur\\
Et ayant parcouru une longue distance,\\
Cessent [de le chercher];\\
N'est-ce pas lui,\\
Le seigneur qui vit \`a Piramapuram\index{gnl}{Piramapuram},\\
Offrant son amour, sous la louange des habitants de la terre\index{gnl}{terre},\\
\`A commencer par les femmes au front \'eclatant? (I~1.9)\\
\end{verse}

\begin{verse}
Le voleur qui ravit mon for int\'erieur\\
Chercha l'aum\^one dans le monde\\
Alors que les ja\"in\index{gnl}{jain@ja\"in}s sans intelligence et les bouddhiste\index{gnl}{bouddhiste}s m\'edisaient\\
Et pr\^echaient un comportement d\'eroutant;\\
N'est-ce pas lui,\\
Le seigneur qui vit \`a Piramapuram\index{gnl}{Piramapuram}\\
Semblable \`a un fou dont on dit:\\
\og Pour d\'econcerter l'\'el\'ephant en rut,\\
Quel \'etonnement de se couvrir de sa d\'epouille!\fg? (I~1.10)\\
\end{verse}

\begin{verse}
Tandis qu'Aka\b n\index{gnl}{Brahma@Brahm\=a!Aka\b n}, le sage\index{gnl}{sage} fort dans les \textit{Veda}\index{gnl}{Veda@\textit{Veda}} \`a l'acc\`es difficile,\\
R\'eside sur la fleur de l'\'etang,\\
Pour ceux capables [de chanter ces strophes] tamoules salvatrices,\\
\`A propos du seigneur qui vit \`a Piramapuram\index{gnl}{Piramapuram} au grand chemin,\\
R\'ecit\'ees par le sensible \~N\=a\b nacampanta\b n\index{gnl}{Campantar!N\=a\b nacampanta\b n@\~N\=a\b nacampanta\b n},\\
Qui a pos\'e son esprit sur la voie de l'unique,\\
\lbrack Pour eux] la fin des souffrances sera facile. (I~1.11)\\
\end{verse}

\noindent Il existe \'evidemment des exceptions \`a cette structure qui repr\'esentent environ un dixi\`eme du corpus\index{gnl}{corpus} disponible. Sur les trois cent quatre-vingt-cinq hymne\index{gnl}{hymne}s attribu\'es \`a Campantar\index{gnl}{Campantar}, quarante-deux comportent dix strophes: trente-deux suivent la structure typique des quatre derniers quatrains que nous venons de d\'ecrire\footnote{I~5, 6, 9, 18, 55, 66, 68, 89, 102, 103, 113, 114, 116, 133; II~1, 11, 17, 23, 36, 58, 64, 83, 89, 95, 97, 108, 122; III~23, 32, 91, 122 et 123.}, sept sont d\'epourvus d'un \'el\'ement de cette organisation\footnote{Il manque la strophe sur le mythe\index{gnl}{mythe} de R\=ava\d na\index{gnl}{Ravana@R\=ava\d na} en III~55, sur celui du Li\.ngodbhava en III~10 et 37, sur la vitup\'eration des asc\`etes hérétique\index{gnl}{heretique@hérétique}s en II~45, III~76 et 121, et enfin, l'envoi\index{gnl}{envoi} fait d\'efaut en II~81.} et trois ne respectent pas ce sch\'ema (I~105, III~63 et 94). Treize poème\index{gnl}{poeme@poème}s contiennent douze\index{gnl}{douze} strophes, dont dix hymnes sont en l'honneur des douze\index{gnl}{douze} noms de C\=\i k\=a\b li\index{gnl}{Cikali@C\=\i k\=a\b li} (I~45, II~6 et III~54 et voir tableau 2.1). Mis \`a part l'hymne\index{gnl}{hymne} III~124 compos\'e de six quatrains qui ne fournissent ni la structure ni l'envoi\index{gnl}{envoi}, les autres poème\index{gnl}{poeme@poème}s de sept (I~81 et III~100), huit (III~50 et 99) et neuf (I~106, III~33 et 36) strophes ob\'eissent \`a la r\`egle.

\noindent La structure de la strophe est aussi, fr\'equemment, caract\'eris\'ee par un sch\'ema fixe ou un refrain qui se r\'ep\`ete dans tout le poème\index{gnl}{poeme@poème}, \`a l'exception de l'envoi\index{gnl}{envoi}. Souvent, le nom du site, repris en fin de strophe, est qualifi\'e aux vers qui pr\'ec\`edent par des descriptions des paysages et la pr\'esence de \'Siva\index{gnl}{Siva@\'Siva}\footnote{Dans le \textit{T\=ev\=aram}\index{gnl}{Tevaram@\textit{T\=ev\=aram}}, les hymne\index{gnl}{hymne}s attribu\'es \`a Campantar\index{gnl}{Campantar} sont ceux qui d\'ecrivent le plus abondamment les sites chant\'es (leurs paysages, leurs \'edifices et leurs habitants). Les poème\index{gnl}{poeme@poème}s attribu\'es \`a Appar\index{gnl}{Appar} pr\'ef\`erent c\'el\'ebrer la nature de \'Siva\index{gnl}{Siva@\'Siva} et ceux de Cuntarar\index{gnl}{Cuntarar} rapportent souvent ses probl\`emes priv\'es (\textsc{Orr} 2009).}. C'est ainsi que fonctionne par exemple I~9.1:

\scriptsize
\begin{verse}
\textit{va\d n\d tu \=ar ku\b lal arivaiyo\d tu piriy\=a vakai p\=akam\\
pe\d nt\=a\b n mika \=a\b n\=a\b n, pi\b raic ce\b n\b nip perum\=a\b n, \textbf{\=ur} ---\\
ta\d nt\=amaraimalar\=a\d l u\b rai tava\d la(n) ne\d tum\=a\d tam\\
vi\d n t\=a\.nkuva p\=olum(m) miku --- \textbf{V\=e\d nupuram\index{gnl}{Venupuram@V\=e\d nupuram} atuv\=e}}.
\end{verse}

\normalsize
\begin{verse}
La demeure du Seigneur couronn\'e du croissant de lune,\\
Devenu femme par la moiti\'e ins\'eparable\\
Avec la jeune dame \`a la chevelure habit\'ee par les abeilles,\\
Est bien V\=e\d nupuram\index{gnl}{Venupuram@V\=e\d nupuram}, o\`u r\'eside Celle \`a la fra\^iche fleur de lotus,\\
O\`u les maisons sont blanches\\
Et si hautes qu'elles semblent soutenir le ciel.\\
\end{verse}


\noindent Le terme \textit{\=ur} (I~9.1b), qui d\'esigne un lieu, est le sujet principal de la strophe. Il est pr\'ec\'ed\'e par un compl\'ement de nom \textit{perum\=a\b n} (\'Siva\index{gnl}{Siva@\'Siva}) qui, lui-m\^eme, est qualifi\'e par \textit{\=a\b n\=a\b n}. L'attribut du sujet est le toponyme V\=e\d nupuram\index{gnl}{Venupuram@V\=e\d nupuram} (I~9.1d), un des douze\index{gnl}{douze} noms de C\=\i k\=a\b li\index{gnl}{Cikali@C\=\i k\=a\b li}, dont la prosp\'erit\'e est d\'ecrite aux deux derniers vers\footnote{Nous retrouvons ce sch\'ema ailleurs: le temple\index{gnl}{temple}, \textit{k\=oyil} (v.~2), a pour attribut le toponyme C\=aykk\=a\d tu (v.~4) dans II 38; le lieu, \textit{i\d tam} (v.~2), est donn\'e en fin de vers 4 dans II 71, 72, 116; III 103 et 104.}.
\noindent De nombreux poème\index{gnl}{poeme@poème}s sont pourvus de refrains en fin de strophe qui mettent en valeur la localit\'e o\`u \'Siva\index{gnl}{Siva@\'Siva} habite\footnote{I 40, 49, 113; II 31, 32, 42, 45, 88, 95, 101; III 23, 25, 57, 62, 82, 90, 101 et 120. Remarquons que certains hymne\index{gnl}{hymne}s qui se succ\`edent dans le corpus\index{gnl}{corpus} \'etabli fonctionnent suivant une m\^eme structure: le sujet est la localit\'e prosp\`ere (\textit{va\d la nakar}) de \'Siva\index{gnl}{Siva@\'Siva} qui a accompli tel ou tel exploit (I 109, 110, 111).} et qu'il aime (I 103 et III 61).
Une autre cat\'egorie de refrains r\'ep\`ete le nom de \'Siva\index{gnl}{Siva@\'Siva} en fin de quatrain. \'Siva\index{gnl}{Siva@\'Siva} est celui du site (I 43 et II 26), celui qui y r\'eside (I 50, 51, 52; II 18, 22, 65, 89, 93, 94; III 39 et 58). Il est le seigneur du lieu (I 45, 62, 87, 123; II 6, 50, 80, 87 et III 8) qui y demeure (I 1, 22; III 59, 92, 108 et 121) r\'ejoui (I 75; II 67 et III 64), avec sa par\`edre (I 74 et III 24). Ces deux dispositions de refrains, pr\'esentant \'Siva\index{gnl}{Siva@\'Siva} et sa demeure, mettent l'accent sur la pr\'esence de \'Siva\index{gnl}{Siva@\'Siva} et son ancrage dans ces sites qui \'etablissent la g\'eographie sacr\'ee du Pays Tamoul\index{gnl}{Pays Tamoul}\footnote{Le poète\index{gnl}{poete@poète} peut chanter plusieurs sites comme dans les hymne\index{gnl}{hymne}s construits sur le proc\'ed\'e d'interrogation \textit{vi\b n\=a urai} (voir \textit{infra}) et dans III 109 qui c\'el\`ebre quatre sites.}.
Quelques refrains \og imp\'eratifs\fg\ invitent le d\'evot \`a chanter (I 8), \`a louer (I 59, 118; II 86 et III 2) et, surtout, \`a visiter les temples\index{gnl}{temple} de \'Siva\index{gnl}{Siva@\'Siva} (I 12, 28; II 97, 99 et 100). D'autres, \og interrogatifs\fg, questionnent \'Siva\index{gnl}{Siva@\'Siva} sur sa nature (I 78 et III 112), ses actes divins (II 1, 2, 3, 4, et 36) et amoureux (I 63 et 76), ainsi que sur le choix de sa demeure (I 4, 6 et 7). Ajoutons enfin les refrains qui scandent \`a la fin de chaque strophe les bienfait\index{gnl}{bienfait}s qu'on obtient en honorant \'Siva\index{gnl}{Siva@\'Siva} ou son temple\index{gnl}{temple} (I 79, 88, 124; II 79, 82, 85 et III 119), la gr\^ace accessible (II 51, 53, 90; III 4 et 55) et les propri\'et\'es mantriques de \'Siva\index{gnl}{Siva@\'Siva} (la cendre\index{gnl}{cendre} sacr\'ee en II 66 et les cinq syllabes en III 22 et 49).
\noindent La structure fig\'ee des hymne\index{gnl}{hymne}s, les sch\'emas fixes des strophes et les nombreux refrains traduisent le style simple et formulaire des poème\index{gnl}{poeme@poème}s attribu\'es \`a Campantar\index{gnl}{Campantar} qui peuvent parfois exceller en lyrisme\footnote{\textsc{Hardy} (*2001 [1983]: 271-275) d\'efinit six types de phrases po\'etiques qui comblent la structure des hymne\index{gnl}{hymne}s vishnouite\index{gnl}{vishnouite}s. Cette classification peut, parfaitement, \^etre appliqu\'ee aux poème\index{gnl}{poeme@poème}s attribu\'es \`a Campantar\index{gnl}{Campantar}. Ses strophes contiennent des expressions st\'er\'eotyp\'ees d\'epeignant la nature, des \'epith\`etes de \'Siva\index{gnl}{Siva@\'Siva}, avec des r\'ef\'erences mythologiques et th\'eologiques, puis son attachement \`a un site, lui-m\^eme d\'ecrit.}.

\subsection{Influence du \textit{Ca\.nkam}}

La disposition de l'hymne\index{gnl}{hymne} en onze quatrains marqu\'es par une structure interne n'est pas novatrice. Il nous semble \'evident que de nombreux poème\index{gnl}{poeme@poème}s attribu\'es \`a Campantar\index{gnl}{Campantar} s'inscrivent dans un contexte litt\'eraire pr\'ecis pr\'e-existant d'expression sanskrite et tamoule. S'il n'est pas possible d'affirmer que les deux \textit{patikam}\index{gnl}{patikam@\textit{patikam}} de K\=araikk\=alammaiy\=ar\index{gnl}{Karaikkalammaiyar@K\=araikk\=alammaiy\=ar}, ainsi que les dizains de l'\textit{Ai\.nku\b run\=u\b ru}\index{gnl}{Ainkurunuru@\textit{Ai\.nku\b run\=u\b ru}} et du \textit{Pati\b r\b ruppattu}\index{gnl}{Patirruppattu@\textit{Pati\b r\b ruppattu}}, aient servi de mod\`ele structural aux d\'ecades du \textit{T\=ev\=aram}\index{gnl}{Tevaram@\textit{T\=ev\=aram}} (\textsc{Gros} 1982: 103), nous observons, toutefois, que les poème\index{gnl}{poeme@poème}s de ce dernier corpus\index{gnl}{corpus} t\'emoignent, souvent, de l'h\'eritage des deux registres, int\'erieur (\textit{akam}\index{gnl}{akam@\textit{akam}}) et ext\'erieur (\textit{pu\b ram}\index{gnl}{puram@\textit{pu\b ram}}), de la litt\'erature codifi\'ee du \textit{Ca\.nkam}\index{gnl}{Cankam@\textit{Ca\.nkam}}\footnote{Nous ne cherchons pas \`a \'etablir une \'etude comparative rigoureuse entre la litt\'erature du \textit{Ca\.nkam}\index{gnl}{Cankam@\textit{Ca\.nkam}} et les hymne\index{gnl}{hymne}s attribu\'es \`a Campantar\index{gnl}{Campantar}. Nous soulignons simplement une certaine influence qui permettra de mieux appr\'ecier, au chapitre 6, la mécanique de la construction hagiographique\index{gnl}{hagiographie!hagiographique}.}. D'une part, dans la continuit\'e de la po\'esie h\'ero\"ique (\textit{pu\b ram}\index{gnl}{puram@\textit{pu\b ram}}), \'Siva\index{gnl}{Siva@\'Siva} devient le roi\index{gnl}{roi} vaillant: lou\'e, entour\'e d'une arm\'ee de gnomes, il \'ecrase ses adversaires, les d\'emons, au combat. Il est aussi le g\'en\'ereux qui leur pardonne ou qui donne\footnote{Nous pensons par exemple \`a l'hymne\index{gnl}{hymne} I~92, d\'edi\'e \`a V\=\i \b limi\b lalai\index{gnl}{Vilimilalai@V\=\i \b limi\b lalai}, dans lequel le poète\index{gnl}{poete@poète} demande des pièce\index{gnl}{piece@pièce}s (\textit{k\=acu} I 92, 1a) sur le m\^eme plan que des faveurs (\textit{p\=e\b ru} I 92, 4b), la protection (\textit{c\=emam} I 92, 5b) et le m\'erite (\textit{paya\b n} I 92, 9b). Cette requ\^ete fait \'echo \`a celle des bardes devant le roi\index{gnl}{roi} dans les poème\index{gnl}{poeme@poème}s du \textit{pu\b ram}\index{gnl}{puram@\textit{pu\b ram}}, comme \textit{Pu\b ran\=a\b n\=u\b ru}\index{gnl}{Purananuru@\textit{Pu\b ran\=a\b n\=u\b ru}} 315.}. D'autre part, dans la continuit\'e de la po\'esie amoureuse (\textit{akam}\index{gnl}{akam@\textit{akam}}), \'Siva\index{gnl}{Siva@\'Siva} est l'\^etre aim\'e, l'amant, attendu ou recherch\'e.
Ainsi, nous retrouvons le genre du poème\index{gnl}{poeme@poème}-messager de la litt\'erature profane en contexte bhaktique\footnote{Pour une mise au point r\'ecente de ce genre, originaire du sanskrit, dans la litt\'erature tamoule, voir \textsc{Dubyanskiy} 2005.}. Le poète\index{gnl}{poete@poète} met en scène une jeune femme qui envoie des oiseaux\index{gnl}{oiseau} \`a \'Siva\index{gnl}{Siva@\'Siva} pour signaler que la séparation a provoqué une grave maladie d'amour. Chaque strophe du poème\index{gnl}{poeme@poème} I~60, \`a la gloire de T\=o\d nipuram\index{gnl}{Tonipuram@T\=o\d nipuram} (C\=\i k\=a\b li\index{gnl}{Cikali@C\=\i k\=a\b li}), narre la plainte de l'amante qui s'adresse \`a divers oiseaux (échassier, caille, perroquet, \dots)\footnote{La première strophe de l'hymne I 60 présente une exception car la jeune femme ne fait pas appel à un oiseau mais au roi des abeilles (\textit{a\d liyarac\=e}).} en leur demandant d'aller dire son mal au \'Siva\index{gnl}{Siva@\'Siva} r\'esidant \`a T\=o\d nipuram\index{gnl}{Tonipuram@T\=o\d nipuram}\footnote{Un poème\index{gnl}{poeme@poème} tr\`es semblable est attribu\'e \`a Cuntarar\index{gnl}{Cuntarar} qui chante le \'Siva\index{gnl}{Siva@\'Siva} d'\=Ar\=ur\index{gnl}{Ar\=ur@\=Ar\=ur} (VII~37).}. Nous lisons, par exemple en I~60.8:

\scriptsize
\begin{verse}
\textit{p\=al n\=a\b rum malarc c\=utap pallava\.nka\d l avai k\=oti,\\
\=e\b n\=orkkum i\b nitu \=aka mo\b liyum e\b lil i\d la\.nkuyil\=e!\\
t\=e\b n \=arum po\b lil pu\d tai c\=u\b l tirut t\=o\d nipurattu amarar-\\
k\=o\b n\=arai e\b n\b ni\d taikk\=e vara oru k\=al k\=uv\=ay\=e!} (I~60.8)\\
\end{verse}

\normalsize
\begin{verse}
O jeune et joli coucou \\
Qui dit, b\'equetant les feuilles tendres du manguier \\
Aux fleurs parfum\'ees de lait\index{gnl}{lait},\\
Des mots agr\'eables \`a tous! \\
Ne diras-tu pas, une fois, \\
Au roi\index{gnl}{roi} des dieux du saint T\=o\d nipuram\index{gnl}{Tonipuram@T\=o\d nipuram}, \\
Entour\'e de jardins regorgeant de miel, \\
De venir \`a moi! (I~60.8)\\
\end{verse}

\normalsize
\noindent
Dans un autre poème\index{gnl}{poeme@poème} en l'honneur de Ce\.nk\=a\d t\d ta\.nku\d ti\index{gnl}{Cenkattankuti@Ce\.nk\=a\d t\d ta\.nku\d ti}, un humble serviteur envoie divers oiseau\index{gnl}{oiseau}x au seigneur, par pure d\'evotion\index{gnl}{devotion@dévotion}, pour demander s'il obtiendra, un jour, la gr\^ace\footnote{III~63, 8:
\begin{verse}
\textit{k\=ur \=aral irai c\=erntu, ku\d lam ulavi, vayal v\=a\b lum\\
t\=ar\=av\=e! ma\d tan\=ar\=ay! tamiy\=e\b rku o\b n\b ru uraiy\=\i r\=e!\\
c\=\i r\=a\d la\b n, ci\b rutto\d n\d ta\b n Ce\.nk\=a\d t\d ta\.nku\d ti\index{gnl}{Cenkattankuti@Ce\.nk\=a\d t\d ta\.nku\d ti} m\=eya\\
p\=er\=a\d la\b n, perum\=a\b nta\b n aru\d l oru n\=a\d l pe\b ral \=am\=e?}\\
\end{verse}

\begin{verse}
\og O h\'eron qui atteignant sa proie d'abondants poissons \textit{\=aral},\\
Flanant dans les \'etangs, vit dans les rizi\`eres,\\
O bel \'echassier!\\
Ne me diras-tu pas un mot \`a moi le solitaire?\\
Est-il possible de recevoir un jour la gr\^ace du Glorieux,\\
Du grand ma\^itre qui vit dans Ce\.nk\=a\d t\d ta\.nku\d ti\index{gnl}{Cenkattankuti@Ce\.nk\=a\d t\d ta\.nku\d ti} de l'humble serviteur,\\
Du Seigneur?\fg \\
\end{verse}}.
Le genre du poème\index{gnl}{poeme@poème}-messager est transpos\'e ici en contexte strictement bhaktique.

\noindent Ailleurs, \'Siva\index{gnl}{Siva@\'Siva} devient le coeur du poète\index{gnl}{poete@poète} qui le loue: chaque strophe de l'hymne\index{gnl}{hymne} III~89 c\'el\`ebre le \'Siva\index{gnl}{Siva@\'Siva} de Koccai\index{gnl}{Koccai}vayam (C\=\i k\=a\b li\index{gnl}{Cikali@C\=\i k\=a\b li}) qui est interpell\'e par le terme affectueux \textit{ne\~ncam}, \og coeur\fg%\footnote{VOIR LES TRAD DE EVA donner le txt cankam XXXXX na\b r 16.4/52.8/95.10/98.12/118.11/154.12/178.10/312.1/348.10; akan 9.26/19.2/21.9/40.17/63.19/79.11/135.14/138.20/198.17/372.14/390.17 et Ku\b ru 4/11.4/63.4/187.5/202}
. Puis, dans chaque quatrain de l'hymne\index{gnl}{hymne} III~100, d\'edi\'e \`a T\=o\d nipuram\index{gnl}{Tonipuram@T\=o\d nipuram}, \`a l'exception de l'envoi\index{gnl}{envoi}, \'Siva\index{gnl}{Siva@\'Siva} est pr\'esent\'e comme le dieu\index{gnl}{dieu} aim\'e qui vient \^oter la f\'eminit\'e de la narratrice touch\'ee par les sympt\^omes classiques de l'\'etat amoureux\footnote{III~100.1c: \textit{perum pakal\=e vantu, e\b n pe\d nmai ko\d n\d tu, p\=erntavar c\=ernta i\d tam}; \og la demeure inh\'erente au ravisseur venu en plein jour \^oter ma f\'eminit\'e\fg. III~100.2c: \textit{ca\.nku iyal ve\d lva\d lai c\=ora vantu, e\b n c\=ayal ko\d n\d t\=artamatu \=ur}; \og la demeure de celui qui, venu, faisant tomber mes bracelets de conques, \^ota ma beaut\'e\fg. III~100.6c: \textit{e\b n e\b lil kavarnt\=ar i\d tam\=am}; \og la demeure du charmeur de ma jeunesse\fg.}. L'\'episode ne s'inscrit pas dans une r\'egion, une situation et un \'etat psychologique particuliers propres \`a la po\'esie d'\textit{akam}\index{gnl}{akam@\textit{akam}}\footnote{Cf. la postface d'A. K.~\textsc{Ramanujan} incluant un tableau r\'ecapitulatif de ces \'el\'ements (\textsc{Daniels-Ramanujan} 2004: 97-115).}. Mais les formules de la s\'eparation du registre d'\textit{akam} sont utilisées dans un contexte inversé, celui de la rencontre. L'amante qui prend la parole, \'emaci\'ee et p\^ale, perd sa beaut\'e, sa jeunesse et ses bracelets, non \`a cause de la s\'eparation avec l'aim\'e, mais \`a la vue de \'Siva\index{gnl}{Siva@\'Siva}. Nous constatons donc ici que le cadre conventionnel litt\'eraire est rompu mais que les formules sont reprises (voir aussi \textsc{Gros} 1984: xvi).
Parfois, c'est le narrateur qui exprime la d\'etresse de la jeune femme, son mal d'amour. Ainsi, dans l'hymne\index{gnl}{hymne} I~44, d\'edi\'e \`a P\=accil\=accir\=amam\index{gnl}{Paccilacciramam@P\=accil\=accir\=amam}, il s'interroge, \`a chaque strophe, sur la nature du dieu\index{gnl}{dieu} qui fl\'etrit une jeune femme\footnote{I~44.1d et 4d: \textit{ma\.nkaiyai v\=a\d ta mayal ceyvat\=o ivar m\=a\d np\=e?}, \og est-ce sa grandeur de troubler la jeune femme pour qu'elle se fane (ou alors qu'elle se fane)?\fg; I~44.2d: \textit{\=e\b laiyai v\=a\d ta i\d tar ceyvat\=o ivar \=\i \d t\=e?}, \og est-ce sa force d'affliger la femme pour qu'elle se fane?\fg; I~44.3d: \textit{painto\d ti v\=a\d tac citaiceyvat\=o ivar c\=\i r\=e?}, \og est-ce sa gloire de d\'etruire celle au bracelet d'or pour qu'elle se fane?\fg; etc.}, et dans le poème\index{gnl}{poeme@poème} II~18, c\'el\'ebrant Marukal\index{gnl}{Marukal}, il interpelle \'Siva\index{gnl}{Siva@\'Siva} et lui demande s'il est convenable de faire languir une jeune fille, marqu\'ee physiquement par les signes amoureux\footnote{II~18.1d: \textit{takum\=o, iva\d l u\d l meliv\=e?}, \og est-ce convenable que son coeur s'affaiblisse?\fg; II~18.1d: \textit{takum\=o, iva\d l \=eca\b rav\=e?}, \og sa langueur est-elle convenable?\fg; \textit{iva\d lai i\b rai \=ar va\d lai ko\d n\d tu, e\b lil vavvi\b naiy\=e?}, \og ayant pris ses bracelets de poignet, as-tu \^ot\'e sa beaut\'e?\fg; etc.}. Notons qu'\`a la strophe 6 le poète\index{gnl}{poete@poète} reprend une image particuli\`ere du registre amoureux de l'\textit{akam}\index{gnl}{akam@\textit{akam}} qui d\'epeint le comm\'erage dans le village \`a propos de l'amante qui a perdu le sommeil\footnote{II~18.6cd: \textit{pularum ta\b naiyum tuyil\=a\d l, pu\d tai p\=ontu} / \textit{alarum pa\d tum\=o, a\d tiy\=a\d l iva\d l\=e?}, \og elle ne dort pas jusqu'\`a l'aube, est-ce convenable que les voisins viennent comm\'erer sur elle, la dévot\index{gnl}{devot(e)@dévot(e)}e?\fg.}.
Afin de clore cette \'enum\'eration non exhaustive de poème\index{gnl}{poeme@poème}s mentionnant le personnage de l'amante, signalons l'hymne\index{gnl}{hymne} II~47, dit de P\=ump\=avai\index{gnl}{Pumpavai@P\=ump\=avai}, qui, \`a chaque fin de strophe r\'ep\`ete l'interrogation \textit{k\=a\d n\=at\=e p\=otiy\=o p\=ump\=av\=ay}, \og \^O belle jeune fille, pars-tu sans regarder?\fg. Ici, le poète\index{gnl}{poete@poète} retient ou rappelle la jeune femme qui part sans assister aux diff\'erentes f\^etes\index{gnl}{fete@fête} du temple\index{gnl}{temple} de Kap\=al\=\i \'svara \`a Mayil\=apuri\index{gnl}{Mayil\=apuri}\footnote{Nous pr\'ef\'erons lire litt\'eralement \textit{P\=ump\=avai\index{gnl}{Pumpavai@P\=ump\=avai}} comme un compos\'e plut\^ot que comme un nom propre. P\=ump\=avai\index{gnl}{Pumpavai@P\=ump\=avai} n'est pas un nom familier de la litt\'erature du \textit{Ca\.nkam}\index{gnl}{Cankam@\textit{Ca\.nkam}}. Nous le trouvons en compos\'e d\`es le \textit{Cilappatik\=aram}\index{gnl}{Cilappatikaram@\textit{Cilappatik\=aram}} pour d\'ecrire une jeune femme (chapitre 21 l.~23 et 34).}.
\noindent Ces divers exemples t\'emoignent donc de l'influence de la litt\'erature du \textit{Ca\.nkam}\index{gnl}{Cankam@\textit{Ca\.nkam}} dans quelques hymne\index{gnl}{hymne}s attribu\'es \`a Campantar\index{gnl}{Campantar} qui, par ailleurs, se distingue dans le corpus\index{gnl}{corpus} par ses prouesses rh\'etoriques.

\subsection{Les proc\'ed\'es litt\'eraires}

Un des \'el\'ements remarquables des trois premiers \textit{Tirumu\b rai}\index{gnl}{Tirumurai@\textit{Tirumu\b rai}} est que la ville natale de Campantar\index{gnl}{Campantar}, C\=\i k\=a\b li\index{gnl}{Cikali@C\=\i k\=a\b li}, est grandement chant\'ee. Rappelons que soixante-sept des soixante-et-onze hymne\index{gnl}{hymne}s c\'el\'ebrant ce site, sous douze\index{gnl}{douze} appellations distinctes, sont attribu\'es \`a Campantar. Soulignons ensuite que onze de ces poème\index{gnl}{poeme@poème}s chantent les douze\index{gnl}{douze} noms, en douze\index{gnl}{douze} strophes\footnote{Except\'e I~128 compos\'e en prose, les dix autres hymne\index{gnl}{hymne}s contiennent chacun douze\index{gnl}{douze} strophes.}, et ce, dans un ordre\index{gnl}{ordre} parfaitement d\'efini. En effet, \`a l'exception de II~73 et 74, chaque quatrain des poèmes concernés consacre un nom selon la succession suivante: Piramapuram\index{gnl}{Piramapuram}, V\=e\d nupuram\index{gnl}{Venupuram@V\=e\d nupuram}, Pukali\index{gnl}{Pukali}, Ve\.nkuru\index{gnl}{Venkuru@Ve\.nkuru}, T\=o\d nipuram\index{gnl}{Tonipuram@T\=o\d nipuram}, Tar\=ay\index{gnl}{Taray@Tar\=ay}, Cirapuram\index{gnl}{Cirapuram}, Pu\b ravam\index{gnl}{Puravam@Pu\b ravam}, Ca\d npai\index{gnl}{Canpai@Ca\d npai}, K\=a\b li\index{gnl}{Kali@K\=a\b li}, Koccai\index{gnl}{Koccai} et Ka\b lumalam\index{gnl}{Kalumalam@Ka\b lumalam}. Notons que dix chants\index{gnl}{chant} portent le titre, apocryphe probablement, de Piramapuram\index{gnl}{Piramapuram} et un de Ka\b lumalam\index{gnl}{Kalumalam@Ka\b lumalam}. Ajoutons enfin que ces onze hymne\index{gnl}{hymne}s, \`a l'instar de treize autres sur C\=\i k\=a\b li\index{gnl}{Cikali@C\=\i k\=a\b li}, sont tous compos\'es selon des procédé\index{gnl}{procédé littéraire}s litt\'eraires particuliers\footnote{Il s'agit de procédé\index{gnl}{procédé littéraire}s qui organisent l'hymne\index{gnl}{hymne} entier. Nous ne discutons pas des figures stylistiques telles que l'assonance, l'allit\'eration, l'anaphore, le chiasme (ligne 4 de chaque strophe de III~46), etc. qui abondent dans pratiquement tous les poème\index{gnl}{poeme@poème}s.}. Le tableau 2.1, ci-dessous, pr\'esente la r\'epartition des soixante-sept poème\index{gnl}{poeme@poème}s attribu\'es \`a Campantar\index{gnl}{Campantar} dans le corpus\index{gnl}{corpus} et selon le toponyme. Nous soulignons les hymne\index{gnl}{hymne}s \`a procédé\index{gnl}{procédé littéraire} po\'etique et mettons en gras ceux c\'el\'ebrant les douze\index{gnl}{douze} noms ensemble. Dans l'\oe uvre attribu\'ee \`a Campantar\index{gnl}{Campantar}, nous relevons un total de dix-huit procédé\index{gnl}{procédé littéraire}s\footnote{Bien qu'ils ne correspondent pas pr\'ecis\'ement \`a une figure po\'etique, nous incluons le \textit{palpeyarpattu} (I~63), \og dix [strophes] sur plusieurs noms\fg, et le \textit{t\=a\d laccati} (I~126), \og agreement of time in music and dancing\fg\ (TL, \textit{s.v.} \textit{cati}).} utilis\'es pour c\'el\'ebrer C\=\i k\=a\b li\index{gnl}{Cikali@C\=\i k\=a\b li} et d'autres temples\index{gnl}{temple}\footnote{Les d\'efinitions des procédé\index{gnl}{procédé littéraire}s de composition que nous pr\'esentons suivent principalement celles enseign\'ees par T. V.~\textsc{Gopal Iyer} lors de nos s\'eances de lecture de 2004 \`a 2006.}. Notons que neuf d'entre eux ne sont employ\'es qu'\`a la gloire de C\=\i k\=a\b li\index{gnl}{Cikali@C\=\i k\=a\b li}\footnote{Le \textit{cakkaram\=a\b r\b ru}\index{gnl}{cakkaramarru@\textit{cakkaram\=a\b r\b ru}} (II 70 et 73), le \textit{mo\b lim\=a\b r\b ru}\index{gnl}{molimarru@\textit{mo\b lim\=a\b r\b ru}} (I 117), le \textit{k\=om\=uttiri\index{gnl}{komuttiri@\textit{k\=om\=uttiri}} ant\=ati}\index{gnl}{antati@\textit{ant\=ati}} (II 74), l'\textit{\=ekap\=atam}\index{gnl}{ekapatam@\textit{\=ekap\=atam}} (I 127), l'\textit{e\b luk\=u\b r\b ru}\index{gnl}{elukurru@\textit{e\b luk\=u\b r\b ru}} (I 128), le \textit{m\=alaim\=a\b r\b ru}\index{gnl}{malaimarru@\textit{m\=alaim\=a\b r\b ru}} (III 117), le \textit{va\b limo\b lit tiruvir\=akam} (III 67), le \textit{palpeyarpattu} (I 63) et le \textit{t\=a\d laccati} (I 126). Nous d\'etaillerons ces procédé\index{gnl}{procédé littéraire}s.}.


\begin{center}
\scriptsize
\begin{longtable}{|l|c|c|c|}
\caption{Les douze toponymes}\endfirsthead
\hline
\ Toponymes & \textit{Tirumu\b rai}\index{gnl}{Tirumurai@\textit{Tirumu\b rai}} I~& \textit{Tirumu\b rai} II~& \textit{Tirumu\b rai} III\endhead
\hline
\ Toponymes & \textit{Tirumu\b rai}\index{gnl}{Tirumurai@\textit{Tirumu\b rai}} I~& \textit{Tirumu\b rai} II~& \textit{Tirumu\b rai} III\\
\hline\hline
Piramapuram\index{gnl}{Piramapuram} & 1, \underline{\textbf{63}}, \underline{\textbf{90}}, \underline{\textbf{117}}, \underline{\textbf{127}}, \underline{\textbf{128}} & 40, 65, \underline{\textbf{70}}, \underline{\textbf{73}}, \underline{\textbf{74}} & 37, 56, \underline{\textbf{67}}, \underline{\textbf{110}}\\
\hline
V\=e\d nupuram\index{gnl}{Venupuram@V\=e\d nupuram} & 9 & 17, 81 & \\
\hline
Pukali\index{gnl}{Pukali} & \underline{4}, 30, 104 & 25, \underline{29}, 54, 122 & \underline{3}, 7\\
\hline
Ve\.nkuru\index{gnl}{Venkuru@Ve\.nkuru} & 75 & & \underline{94}\\
\hline
T\=o\d nipuram\index{gnl}{Tonipuram@T\=o\d nipuram} & 60 & & \underline{81}, 100\\
\hline
Tar\=ay\index{gnl}{Taray@Tar\=ay} & & \underline{1} & 2, \underline{5}, 13\\
\hline
Cirapuram\index{gnl}{Cirapuram} & 47, 109 & 102 & \\
\hline
Pu\b ravam\index{gnl}{Puravam@Pu\b ravam} & 74, 97 & & \underline{84}\\
\hline
Ca\d npai\index{gnl}{Canpai@Ca\d npai} & 66 & & \underline{75}\\
\hline
K\=a\b li\index{gnl}{Kali@K\=a\b li} & 24, 34, 81, 102 & 11, 49, 59, 75, 96, \underline{97}, & 43, \underline{117}\\
 & & 113 & \\
\hline
Koccai\index{gnl}{Koccai} & & 83, 89 & 89\\
\hline
Ka\b lumalam\index{gnl}{Kalumalam@Ka\b lumalam} & \underline{19}, 79, \underline{126}, 129 & & 24, \underline{\textbf{113}}, 118\\
\hline
\end{longtable}
\end{center}


\normalsize
Ce r\'epertoire compte des jeux sur le m\`etre, la forme et les mots. Le \textit{tirumukk\=al}\index{gnl}{tirumukkal@\textit{tirumukk\=al}} \og trois-quart\fg\ (III~94-99), l'\textit{\=\i ra\d ti}\index{gnl}{irati@\textit{\=\i ra\d ti}} \og deux pieds\fg\ (III~110-112), le \textit{n\=ala\d tim\=el vaippu}\index{gnl}{nalatimel@\textit{n\=ala\d tim\=el vaippu}} (III~3, 4 et 108), l'\textit{\=\i ra\d tim\=el vaippu}\index{gnl}{iratimel@\textit{\=\i ra\d tim\=el vaippu}} (III~5 et 6) et le \textit{y\=a\b lm\=uri}\index{gnl}{yalmuri@\textit{y\=a\b lm\=uri}} \og brisure du \textit{y\=a\b l}\fg\ (I 136) sont des figures reposant sur la m\'etrique\footnote{Sur ces notions, et en particulier, sur \textit{n\=ala\d tim\=el vaippu}\index{gnl}{nalatimel@\textit{n\=ala\d tim\=el vaippu}} et \textit{\=\i ra\d tim\=el vaippu}\index{gnl}{iratimel@\textit{\=\i ra\d tim\=el vaippu}}, voir \textsc{Gopal Iyer} (1991: 90).}.
%r\'ep\'etition dans le figure po\'etique comme tirumukk\=al (entre ligne 2 et premi\`ere h\'emistiche de ligne 3, III~94-99) ou comme n\=ala\d tim\=el vaippu (o\`u distique ajout\'ee au quatrain sert de refrain \`a chaque strophe, III~108 et III~4; alors que pas comme ça pour III~3)
S'ajoutent dans cette cat\'egorie l'\textit{irukkukku\b ra\d l}\index{gnl}{irukkukkural@\textit{irukkukku\b ra\d l}} (I~90-96 et III~40-41), litt\'eralement \og distique \textsubring{r}gv\'edique\fg, et le \textit{tiruvir\=akam}\index{gnl}{tiruvirakam@\textit{tiruvir\=akam}}, figure de rythme caract\'eris\'ee par l'emploi presque unique de mots \`a syllabes br\`eves pour obtenir un mouvement rapide\footnote{Son usage est fr\'equent. Quarante-quatre hymne\index{gnl}{hymne}s, dont sept en l'honneur de C\=\i k\=a\b li\index{gnl}{Cikali@C\=\i k\=a\b li}, illustrent ce procédé\index{gnl}{procédé littéraire}: I~19-22, 120-125; II~29-34, 97, 98, 100, 101 et III~52, 53, 67-88.}. Une vari\'et\'e de ce dernier procédé\index{gnl}{procédé littéraire} est le \textit{va\b limo\b lit tiruvir\=akam} qui est employ\'e, selon \textsc{T. V. Gopal Iyer}, pour la premi\`ere fois dans la litt\'erature tamoule avec III 67. Cette figure consiste \`a reprendre en \textit{etukai}\index{gnl}{etukai@\textit{etukai}} (rime de la deuxi\`eme syllabe du vers ou du pied) la deuxi\`eme syllabe de l'un des douze\index{gnl}{douze} noms de C\=\i k\=a\b li\index{gnl}{Cikali@C\=\i k\=a\b li} à chaque strophe.
Ensuite, le \textit{vi\b n\=avurai}\index{gnl}{vinavurai@\textit{vi\b n\=avurai}}, constitu\'e d'interrogations (\textit{vi\b n\=a}), joue sur la forme (I~4, 6, 7, II~1-4, 36 et III~38).
Enfin, les jeux de mots sont nombreux:
\begin{enumerate}
\item
Le \textit{cakkaram\=a\b r\b ru}\index{gnl}{cakkaramarru@\textit{cakkaram\=a\b r\b ru}}, \og \'echange circulaire\fg, semble \^etre un procédé\index{gnl}{procédé littéraire} propre \`a deux hymne\index{gnl}{hymne}s sur C\=\i k\=a\b li\index{gnl}{Cikali@C\=\i k\=a\b li} (II~70 et 73) dans lesquels les douze\index{gnl}{douze} noms du site apparaissent \`a chaque quatrain, et le dernier toponyme mentionn\'e dans une strophe d\'ebute la suivante.
\item
Le \textit{mo\b lim\=a\b r\b ru}\index{gnl}{molimarru@\textit{mo\b lim\=a\b r\b ru}}, \og \'echange de mots\fg, organise l'hymne\index{gnl}{hymne} I~117 consacrant C\=\i k\=a\b li\index{gnl}{Cikali@C\=\i k\=a\b li}, unique exemple du corpus\index{gnl}{corpus}: les strophes sont construites de telle sorte que certains mots doivent \^etre déplacés pour les comprendre. Ainsi, la strophe initiale,

\scriptsize
\begin{verse}
\textit{k\=a\d tuatu, a\d nikalam k\=ar aravam, \textbf{pati}; k\=al ata\b nil,-\\
t\=o\d tuatu a\d nikuvar cuntarak k\=ati\b nil,-\textbf{t\=uuc cilampar};\\
v\=e\d tuatu a\d nivar, vicaya\b rku, \textbf{uruvam}, \textbf{villum ko\d tuppar};---\\
p\=\i \d tuatu a\d ni ma\d ni m\=a\d tap piramapurattu arar\=e.}\\
\end{verse}
\normalsize
\noindent
doit \^etre lue:

\noindent
\normalsize
\begin{tabular}{ll}
\textit{k\=a\d tu atu \textbf{pati}} & La demeure est le bois (cr\'ematoire), \\
\textit{a\d nikalam k\=ar aravam} & L'ornement le serpent\index{gnl}{serpent} noir;\\
\textit{k\=al ata\b nil \textbf{t\=uc cilampar}} & Celui aux anneaux purs aux pieds\\
\textit{t\=o\d tu atu a\d nikuvar cuntarak k\=ati\b nil} & Porte une boucle \`a la belle oreille,\\
\textit{v\=e\d tu atu \textbf{uruvam} a\d nivar } & Porte la forme du chasseur \\
\textit{vicaya\b rku \textbf{villum ko\d tuppar}} & Et donne l'arc \`a Vijaya;\\
\textit{piramapurattu arar\=e} & \^O Hara\index{gnl}{Hara} de Piramapuram\index{gnl}{Piramapuram}\\
\textit{p\=\i \d tu atu a\d ni ma\d ni m\=a\d tap} & Aux maisons gemm\'ees\\
& pourvues de grandeur!\\
\end{tabular}\\


\noindent
\item
Le \textit{k\=om\=uttiri\index{gnl}{komuttiri@\textit{k\=om\=uttiri}} ant\=ati}\index{gnl}{antati@\textit{ant\=ati}} (sk. \textit{gom\=utrik\=a}), \og \textit{ant\=ati}\index{gnl}{antati@\textit{ant\=ati}} en urine de vache\fg, est une figure dont la lecture s'effectue en zigzag, comme l'indique son nom. Chaque quatrain y fonctionne par paire de vers. La premi\`ere syllabe du premier vers, suivie de la deuxi\`eme du deuxi\`eme vers, puis de la troisi\`eme du premier et de la quatri\`eme du deuxi\`eme, et ainsi de suite, lues ensemble forment le premier vers. De m\^eme, la premi\`ere syllabe du deuxi\`eme vers est suivie de la deuxi\`eme du premier vers, puis de la troisi\`eme du deuxi\`eme et de la quatri\`eme du premier, et ainsi de suite. Ces syllabes lues ensemble forment le deuxi\`eme vers. L'hymne\index{gnl}{hymne} II~74, d\'edi\'e \`a C\=\i k\=a\b li\index{gnl}{Cikali@C\=\i k\=a\b li}, est le seul exemple de cette figure dans le corpus\index{gnl}{corpus}, mais cette r\`egle n'y est respect\'ee que, partiellement, en d\'ebut de vers.
\item
L'\textit{\=ekap\=atam}\index{gnl}{ekapatam@\textit{\=ekap\=atam}}, \og pied unique\fg, se caract\'erise par une strophe de quatre vers identiques phon\'etiquement mais diff\'erents s\'emantiquement\footnote{\textsc{Gopal Iyer} (1991: 99-176) pr\'esente les commentaires anciens et modernes des poème\index{gnl}{poeme@poème}s compos\'es en \textit{\=ekap\=atam}\index{gnl}{ekapatam@\textit{\=ekap\=atam}} (I~127), \textit{e\b luk\=u\b r\b ru}\index{gnl}{elukurru@\textit{e\b luk\=u\b r\b ru}} (I~128), \textit{iyamakam}\index{gnl}{iyamakam@\textit{iyamakam}} (III~113-116) et en \textit{m\=alaim\=a\b r\b ru}\index{gnl}{malaimarru@\textit{m\=alaim\=a\b r\b ru}} (III~117), que nous d\'etaillons.}. I~127, c\'el\'ebrant C\=\i k\=a\b li\index{gnl}{Cikali@C\=\i k\=a\b li}, est l'unique poème\index{gnl}{poeme@poème} construit selon cette figure.
\item
L'\textit{e\b luk\=u\b r\b ru}\index{gnl}{elukurru@\textit{e\b luk\=u\b r\b ru}}, \og mot croissant\fg, est un jeu de mots num\'erique dans lequel les chiffres de un \`a sept apparaissent dans l'ordre\index{gnl}{ordre} croissant puis d\'ecroissant selon le sch\'ema suivant, du moins, en ce qui concerne l'hymne\index{gnl}{hymne} I~128, unique exemple du genre, \'etabli en prose:\\

\scriptsize
\begin{tabular}{ccccccccccccccccl}
&& & & & 1 & 2 & 1 & & & & && &&&l. 1-2\\
& && & 1 & 2 & 3 & 2 & 1 & &&&& &&&l. 3-5\\
&&& 1& 2 & 3 & 4 & 3 & 2 & 1&&&& &&&l. 6-9\\
& &1 & 2 & 3 & 4 & 5 & 4 & 3 & 2 & 1& && &&&l. 10-13\\
&1& 2 &3 &4& 5& 6& 5& 4 &3 &2& 1&& &&&l. 14-18\\
1 &2 &3& 4 &5& 6 &7& 6& 5& 4 &3 &2& 1& &&&l. 19-31\\
1& 2 &3& 4& 5& 6 &7 &6 &5& 4 &3 &2& 1& &&&l. 32-42\\
\end{tabular}\\

\normalsize
\noindent
Prenons l'exemple des lignes 10 \`a 13:

\scriptsize
\begin{tabular}{ll}
\textit{\textbf{oru}t\=a\d l \textbf{\=\i r} ayil \textbf{m\=u} ilaicc\=ulam,} &10\\
\textit{\textbf{n\=al}k\=al m\=a\b nma\b ri, \textbf{ain}talai aravam} & \\
\textit{\=enti\b nai; k\=aynta \textbf{n\=al} v\=ay \textbf{mu}mmatattu} & \\
\textit{\textbf{iru}k\=o\d t\d tu \textbf{oru}kari \=\i \d tu a\b littu uritta\b nai;} & 13\\
\end{tabular}\\

\normalsize
\begin{tabular}{ll}
\textit{\textbf{oru}t\=a\d l} & Un b\^aton,\\
\textit{\textbf{\=\i r}} \textit{ayil \textbf{m\=u} ilaicc\=ulam} & une pique \`a trois [pointes en forme de]\\
 & feuilles grandes et aiguis\'ees,\\
\textit{\textbf{n\=al}k\=al m\=a\b nma\b ri} & une jeune gazelle \`a quatre pattes,\\
\textit{\textbf{ain}talai aravam} & un serpent\index{gnl}{serpent} \`a cinq t\^etes,\\
\textit{\=enti\b nai.} & tu brandis.\\

\textit{k\=aynta} & En col\`ere,\\
\textit{\textbf{n\=al}} \textit{v\=ay} & la trompe pendante,\\
\textit{\textbf{mu}mmatattu} & \`a trois \textit{matam},\\
\textit{\textbf{iru}k\=o\d t\d tu} & \`a deux d\'efenses,\\
\textit{\textbf{oru}kari} & un \'el\'ephant,\\
\textit{\=\i \d tu a\b littu uritta\b nai.} & d\'etruisant sa force, tu [le] d\'epouillas.\\
\end{tabular}\\
\normalsize
\noindent
Le terme \textit{\=\i r} (l. 10), repr\'esentant le chiffre deux, prend le sens de \og grand\fg\ ici et le terme \textit{n\=al} (l. 12), repr\'esentant le chiffre quatre, est ici un verbe signifiant \og pendre\fg.

\noindent
\item
L'\textit{iyamakam}\index{gnl}{iyamakam@\textit{iyamakam}} (sk. \textit{yamaka})\footnote{Cf. \textsc{Sohnen} 1995.}, \og double\fg, d\'esigne les hymne\index{gnl}{hymne}s (III~113-116) dans lesquels se r\'ep\`etent des suites de phon\`emes semblables mais dont le sens diff\`ere, notamment \`a cause des coupes. Ce jeu des homophonies est proche de la paronomase. Par exemple:

\scriptsize
\begin{verse}
\textit{cati mika vanta \textbf{calantara\b n\=e} ta\d ti ciram n\=er ko\d l \textbf{calam tara\b n\=e}!\\
atir o\d li c\=er tikirip\textbf{pa\d taiy\=al} amarnta\b nar umpar, tutip\textbf{pu a\d taiy\=al};\\
mati tava\b l ve\b rpuatu \textbf{kaic cilaiy\=e}; maru vi\d tam \=e\b rpatu \textbf{kaiccilaiy\=e}---\\
vitiyi\b nil i\d t\d tu a\textbf{virum para\b n\=e}! v\=e\d nupurattai \textbf{virumpu ara\b n\=e}!} (III~113.2)\\
\end{verse}

\normalsize
\begin{verse}
Jalandhara qui avan\c cait tr\`es rapidement,\\
\^O Porteur de la belle eau\index{gnl}{eau}, tu le d\'ecapitas, \\
Avec l'arme circulaire o\`u brille la peur,\\
R\'ealisant [ainsi] le souhait de ceux qui r\'esident dans le ciel; \\
L'arc dans ta main est la montagne o\`u rampe la lune;\\
Accepter le poison apparu n'est pas une amertume; \\
\^O rayonnant seigneur qui pla\c ca [le monde] dans l'ordre\index{gnl}{ordre}! \\
\^O Hara\index{gnl}{Hara} qui aime V\=e\d nupuram\index{gnl}{Venupuram@V\=e\d nupuram}! (III~113.2)\\
\end{verse}

\normalsize
\noindent
Au premier vers, le nom du démon\index{gnl}{demon@démon} Calantara\b n\index{gnl}{Calantara\b n (sk. Jalandhara)} (sk. Jalandhara) est r\'ep\'et\'e et coup\'e en deux termes pour signifier le porteur (\textit{tara\b n}, sk. \textit{dhara}) d'eau\index{gnl}{eau} (\textit{calam}, sk. \textit{jala}), appellation de \'Siva\index{gnl}{Siva@\'Siva} portant la Ga\.ng\=a\index{gnl}{Ganga@Ga\.ng\=a} dans sa chevelure. Au deuxi\`eme vers, le terme \textit{pa\d tai} \og arme\fg\ \`a l'instrumental est repris avec la derni\`ere syllabe de \textit{tutippu} \og souhait\fg\ et le nom verbal d'\textit{a\d tai} \og atteindre, r\'ealiser\fg. Au troisi\`eme vers, le compos\'e \textit{kai} \og main\fg\ et \textit{cilai} \og arc\fg\, dont le sens est \og arc \`a la main\fg, revient mais la c\'esure est plac\'ee entre \textit{kaiccu} (d\'eriv\'e de \textit{kaittu}) \og amertume\fg\ et la n\'egation \textit{ilai}. Au quatri\`eme vers, les deux derni\`eres syllabes d'\textit{avirum} \og qui brille\fg\ et le nom appellatif \textit{para\b n} \og seigneur\fg\ sont reproduits par le verbe \textit{virumpu} \og aimer\fg\ et le nom de \'Siva\index{gnl}{Siva@\'Siva} le Destructeur, \textit{ara\b n} (sk. Hara\index{gnl}{Hara}). Ainsi, \`a chaque vers, la fin des h\'emistiches est semblable phon\'etiquement mais diff\'erente lexicalement.
\item
Le \textit{m\=alaim\=a\b r\b ru}\index{gnl}{malaimarru@\textit{m\=alaim\=a\b r\b ru}}, \og \'echange de poème\index{gnl}{poeme@poème}s-guirlandes\fg, est une figure qui fonctionne par paire de vers. Les syllabes, qui forment le premier vers, lues dans le sens inverse, constituent le second vers. Le seul exemple du corpus\index{gnl}{corpus} est le poème\index{gnl}{poeme@poème} III~117. Voici le premier distique, accompagn\'e de la c\'esure et de la ponctuation propos\'ees par \textsc{T. V.~Gopal Iyer}:

\scriptsize
\begin{verse}
\textit{y\=am\=am\=an\=\i\ y\=am\=am\=a y\=a\b l\=\i k\=am\=a k\=a\d n\=ak\=a\\
k\=a\d n\=ak\=am\=a k\=a\b l\=\i y\=a m\=am\=ay\=an\=\i\ m\=am\=ay\=a.\\
y\=am \=am\=a? n\=\i\ \=am \=am; m\=ay\=a\b l\=\i! k\=am\=a! k\=a\d n n\=ak\=a!\\
k\=a\d n\=a k\=am\=a! k\=a\b l\=\i y\=a! m\=a m\=ay\=a! n\=\i, m\=a m\=ay\=a!\\}
\end{verse}

\normalsize
\begin{tabular}{ll}
\textit{y\=am \=am\=a?} & Sommes-nous [absolus]?\\
\textit{n\=\i\ \=am \=am} & Tu es bien [l'Absolu].\\
\textit{m\=ay\=a\b l\=\i!} & Toi au grand \textit{y\=a\b l}!\\
\textit{k\=am\=a!} & Toi l'aim\'e [de tous]!\\
\textit{k\=a\d n n\=ak\=a!} & Toi au serpent\index{gnl}{serpent} visible!\\
\textit{k\=a\d n\=a k\=am\=a!} & [Tu as fait que] K\=ama\index{gnl}{Kama@K\=ama} ne soit plus vu!\\
\textit{k\=a\b l\=\i y\=a!} &Toi de (C\=\i)k\=a\b li!\\
\textit{m\=a m\=ay\=a!} &Toi qui est M\=a (Vi\d s\d nu\index{gnl}{Visnu@Vi\d s\d nu}) de M\=a (Lak\d sm\=\i\index{gnl}{Lak\d sm\=\i})!\\
\textit{n\=\i, m\=a m\=ay\=a!} &Annihile l'illusion noire!\\
\end{tabular}\\

\end{enumerate}
Dans le corpus\index{gnl}{corpus} du \textit{T\=ev\=aram}\index{gnl}{Tevaram@\textit{T\=ev\=aram}} seuls des hymne\index{gnl}{hymne}s attribu\'es \`a Campantar\index{gnl}{Campantar} sont compos\'es selon des procédé\index{gnl}{procédé littéraire}s litt\'eraires qui structurent l'ensemble du poème. L'examen de ces derniers montre que chacun d'entre eux a \'et\'e employ\'e pour c\'el\'ebrer le site de C\=\i k\=a\b li\index{gnl}{Cikali@C\=\i k\=a\b li}, et ce, parfois, de mani\`ere exclusive (cf. I 117, 126, 127, 128; II 70, 73, 74 et III 117). Ajoutons que tous les hymne\index{gnl}{hymne}s louant C\=\i k\=a\b li\index{gnl}{Cikali@C\=\i k\=a\b li} sous ces douze\index{gnl}{douze} appellations sont con\c cus selon ces procédé\index{gnl}{procédé littéraire}s (cf. I 63, 90, 117, 127, 128; II 70, 73, 74; III 67, 110 et 113).\\

Ainsi, ces hymne\index{gnl}{hymne}s de \textit{bhakti}\index{gnl}{bhakti@\textit{bhakti}} attribu\'es \`a Campantar\index{gnl}{Campantar} illustrent syst\'ematiquement la supr\'ematie de \'Siva\index{gnl}{Siva@\'Siva} sur le démon\index{gnl}{demon@démon} R\=ava\d na\index{gnl}{Ravana@R\=ava\d na} et sur les dieux Vi\d s\d nu\index{gnl}{Visnu@Vi\d s\d nu} et Brahm\=a\index{gnl}{Brahma@Brahm\=a}. Ils expriment invariablement un m\'epris et une haine profonde envers les bouddhiste\index{gnl}{bouddhiste}s et les ja\"in\index{gnl}{jain@ja\"in}s, qui, selon Campantar, ne correspondent pas \`a l'identit\'e\index{gnl}{identit\'e} du Pays Tamoul\index{gnl}{Pays Tamoul}. Ils rappellent la litt\'erature amoureuse du \textit{Ca\.nkam}\index{gnl}{Cankam@\textit{Ca\.nkam}} et donnent g\'en\'ereusement la parole aux femmes en mal d'amour qui languissent pour \'Siva\index{gnl}{Siva@\'Siva}. Notons que l'inimiti\'e contre les hérétique\index{gnl}{heretique@hérétique}s et les voix f\'eminines sont des sujets repris et incorpor\'es dans l'hagiographie de Campantar\index{gnl}{Campantar}.

Du point de vue de la forme, nous observons une volont\'e ind\'eniable de structurer les hymne\index{gnl}{hymne}s et leur strophe, parfois gr\^ace \`a un usage prolifique de refrains et r\'ep\'etitions. Cependant, les figures de style d\'eploy\'ees dans la composition des vingt-quatre hymne\index{gnl}{hymne}s c\'el\'ebrant C\=\i k\=a\b li\index{gnl}{Cikali@C\=\i k\=a\b li}, en particulier les onze \`a la gloire des douze\index{gnl}{douze} toponymes, nous laissent, souvent, l'impression d'un artifice qui ne correspond pas \`a l'esprit des poème\index{gnl}{poeme@poème}s du corpus\index{gnl}{corpus} attribu\'es \`a Appar\index{gnl}{Appar} et Cuntarar\index{gnl}{Cuntarar}. En effet, le \textit{T\=ev\=aram}\index{gnl}{Tevaram@\textit{T\=ev\=aram}}, expression par excellence du mouvement de \textit{bhakti}\index{gnl}{bhakti@\textit{bhakti}} shiva\"ite\index{gnl}{shiva\"ite} tamoul, est censé relever de la tradition\index{gnl}{tradition} orale et se caract\'eriser par une langue accessible au plus grand nombre. Les hymne\index{gnl}{hymne}s bas\'es sur des procédé\index{gnl}{procédé littéraire}s tels que l'\textit{\=ekap\=atam}\index{gnl}{ekapatam@\textit{\=ekap\=atam}}, le \textit{m\=alaim\=a\b r\b ru}\index{gnl}{malaimarru@\textit{m\=alaim\=a\b r\b ru}}, etc. ont un sens tellement obscur que leur compr\'ehension n\'ecessite les ex\'eg\`eses tardives\footnote{Voir \textsc{Schulman} (2004: 158 et suiv.) qui date ce type de litt\'erature aux \textsc{xiii-xv}\up{e} si\`ecles.}. De tels poème\index{gnl}{poeme@poème}s apparaissent tels des intrus dans un ensemble destin\'e \`a \^etre \`a la port\'ee de tous. Les onze hymne\index{gnl}{hymne}s c\'el\'ebrant les douze\index{gnl}{douze} noms de C\=\i k\=a\b li\index{gnl}{Cikali@C\=\i k\=a\b li}, dans un ordre\index{gnl}{ordre} parfaitement d\'efini, et format\'es selon ces figures soul\`event, \`a notre avis, des interrogations substantielles quant \`a leur appartenance à un corpus\index{gnl}{corpus} \og premier\fg.
\noindent
Un autre \'el\'ement des hymne\index{gnl}{hymne}s attribu\'es \`a Campantar\index{gnl}{Campantar} sugg\`ere l'interpolation\index{gnl}{interpolation}. L'envoi\index{gnl}{envoi}, le \textit{tirukka\d taikk\=appu}\index{gnl}{tirukkataikkappu@\textit{tirukka\d taikk\=appu}}, est la strophe de \og protection finale\fg\ dans laquelle l'auteur signe et annonce les bienfait\index{gnl}{bienfait}s de la récitation\index{gnl}{recitation@récitation} de ses compositions.

\section{Campantar par lui-m\^eme}

L'envoi\index{gnl}{envoi} semble na\^itre dans les deux dizains de K\=araikk\=alammaiy\=ar\index{gnl}{Karaikkalammaiyar@K\=araikk\=alammaiy\=ar}, les \textit{Tiruv\=ala\.nk\=a\d t\d tu m\=uttatiruppatikam} et \textit{tiruppatikam}, qui r\'ev\`elent la griffe de l'auteur et signalent les bienfait\index{gnl}{bienfait}s obtenus en chantant ses strophes (\textsc{Gros} 1982: 103)\footnote{\textit{Tiruv\=ala\.nk\=a\d t\d tu m\=uttatiruppatikam} 11cd: \textit{appa\b nai ya\d nitiru \=ala\.nk\=a\d t\d tu\d l a\d tika\d laic ce\d titalaik k\=araikk\=a\b rp\=ey~/ ceppiya centami\b l pattumvall\=ar civakati c\=ernti\b npa meytuv\=ar\=e}, \og Les forts [capables de chanter] la d\'ecade en pur tamoul r\'ecit\'ee par la d\'emone de K\=araikk\=al aux cheveux \'ebouriff\'es sur le P\`ere, sur les Pieds orn\'es [de \'Siva\index{gnl}{Siva@\'Siva} se trouvant] \`a \=Ala\.nk\=a\d tu\index{gnl}{Alankatu@\=Ala\.nk\=a\d tu}, joindront le monde de \'Siva\index{gnl}{Siva@\'Siva} et atteindront le bonheur\fg; \textit{Tiruv\=ala\.nk\=a\d t\d tu Tiruppatikam} 11cd: \textit{k\=a\d tu malinta ka\b nalv\=a yeyi\b r\b ruk k\=araik k\=a\b rp\=eyta\b n~/ p\=a\d tal pattum p\=a\d ti y\=a\d tap p\=ava n\=acam\=e}, \og En chantant et dansant les dix strophes de la d\'emone de K\=araikk\=al, aux dents d'une bouche de feu\index{gnl}{feu} et habitant les bois, les péché\index{gnl}{peche@péché}s sont d\'etruits\fg.}. La po\'etesse se pr\'esente là par son surnom de d\'emone (\textit{p\=ey}) et son origine\index{gnl}{origine} g\'eographique (K\=araikk\=al). La récitation\index{gnl}{recitation@récitation} de ses poème\index{gnl}{poeme@poème}s, d\'esign\'es par un terme qui signifie groupement de dix (strophes) compos\'ees en tamoul (\textit{patikam}\index{gnl}{patikam@\textit{patikam}}), m\`ene au monde de \'Siva\index{gnl}{Siva@\'Siva}, \`a la lib\'eration. Les \textit{tirukka\d taikk\=appu}\index{gnl}{tirukkataikkappu@\textit{tirukka\d taikk\=appu}} attribu\'es \`a Campantar\index{gnl}{Campantar}, bien que versés dans un moule similaire, connaissent des variations diverses. Ainsi, nous analyserons les caract\'eristiques de ce quatrain chez ce poète\index{gnl}{poete@poète} avant de nous concentrer sur le portrait que Campantar y dresserait de lui-m\^eme, tout en soulignant les probl\`emes qui en d\'ecoulent\footnote{Nous regrettons de ne pas avoir eu l'opportunit\'e de consulter la th\`ese non publi\'ee de M. A.~\textsc{Kandiah} (\textit{A critical study of early Tamil \'saiva bhakti literature with special reference to T\=ev\=aram}, University of London, 1973) qui consacre un chapitre de son \'etude \`a cette strophe finale. \textsc{Gros} (1982: 103 et 1984: xvi) s'y r\'ef\`ere sans préciser les conclusions de ce travail.}.

\subsection{Caract\'eristiques des envois}

L'ajout d'un envoi\index{gnl}{envoi} \`a l'unit\'e des dix strophes que forme un hymne\index{gnl}{hymne} est une caract\'eristique de Campantar\index{gnl}{Campantar}. Ce ph\'enom\`ene se retrouve dans les poème\index{gnl}{poeme@poème}s attribu\'es \`a Cuntarar\index{gnl}{Cuntarar} mais, dans ce dernier cas, l'envoi\index{gnl}{envoi} est souvent inclu dans la d\'ecade\footnote{Dans soixante-quatre des cent un hymne\index{gnl}{hymne}s attribu\'es \`a Cuntarar\index{gnl}{Cuntarar}, l'envoi\index{gnl}{envoi} est inclu dans la d\'ecade. Il se positionne \`a la onzi\`eme strophe dans trente-et-un poème\index{gnl}{poeme@poème}s et \`a la douzi\`eme dans trois. Un hymne\index{gnl}{hymne} de huit quatrains pr\'esente le \textit{tirukka\d taikk\=appu}\index{gnl}{tirukkataikkappu@\textit{tirukka\d taikk\=appu}} dans sa derni\`ere strophe (VII~11). Notons enfin, que l'envoi\index{gnl}{envoi} est incomplet dans VII~63 et qu'il est absent dans VII~65 et 66, pourvus, respectivement, de sept et de cinq quatrains.}. Rappelons, n\'eanmoins, que la structure des hymne\index{gnl}{hymne}s attribu\'es \`a Campantar\index{gnl}{Campantar} connaît des exceptions (voir \textit{supra}). Sur les trois cent quatre-vingt-cinq poème\index{gnl}{poeme@poème}s des trois premiers \textit{Tirumu\b rai}\index{gnl}{Tirumurai@\textit{Tirumu\b rai}}, seulement deux hymne\index{gnl}{hymne}s n'ont pas d'envoi\index{gnl}{envoi} (II~81 et III~94\footnote{Ce poème\index{gnl}{poeme@poème} se d\'emarque par sa singularit\'e: il comporte dix strophes et la structure typique de l'auteur --- qui place aux quatre derniers quatrains les r\'ef\'erences aux mythe\index{gnl}{mythe}s de R\=ava\d na\index{gnl}{Ravana@R\=ava\d na} et de Li\.ngodbhava, ainsi que les critiques contre les asc\`etes ja\"in\index{gnl}{jain@ja\"in}s et bouddhiste\index{gnl}{bouddhiste}s --- est absente. Soulignons que cet hymne\index{gnl}{hymne} est compos\'e \`a la gloire de Ve\.nkuru\index{gnl}{Venkuru@Ve\.nkuru} (C\=\i k\=a\b li\index{gnl}{Cikali@C\=\i k\=a\b li}), selon le procédé\index{gnl}{procédé littéraire} m\'etrique nomm\'e \textit{tirumukk\=al}\index{gnl}{tirumukkal@\textit{tirumukk\=al}}, et que chaque quatrain annonce les bienfait\index{gnl}{bienfait}s de la récitation\index{gnl}{recitation@récitation} ou du culte\index{gnl}{culte} de \'Siva\index{gnl}{Siva@\'Siva}. Tous ces \'el\'ements font de ce poème\index{gnl}{poeme@poème} une exception, et, partant, peut-\^etre, une interpolation\index{gnl}{interpolation}.}) et pour quatre autres ils sont incomplets (I~ 53, 115, II~9 et 96). Pour ce qui concerne II 9 les composants essentiels d'un envoi\index{gnl}{envoi} sont toutefois lisibles.

Chacun des trois cent quatre-vingt \textit{tirukka\d taikk\=appu}\index{gnl}{tirukkataikkappu@\textit{tirukka\d taikk\=appu}} restants contient le nom du poète\index{gnl}{poete@poète} dans un style formulaire et annonce les bienfait\index{gnl}{bienfait}s que le dévot\index{gnl}{devot(e)@dévot(e)} peut acqu\'erir. L'envoi\index{gnl}{envoi} se distingue de l'ensemble de l'hymne\index{gnl}{hymne} par sa position finale mais aussi par le fait qu'il ne respecte pas la structure de la d\'ecade. Les refrains ou les sch\'emas syntaxiques des strophes pr\'ec\'edentes n'y apparaissent pas. Le sujet n'est plus \'Siva\index{gnl}{Siva@\'Siva} ou sa demeure mais ceux qui sont capables de chanter, de r\'eciter, de r\'epandre un hymne\index{gnl}{hymne} compos\'e par Campantar\index{gnl}{Campantar} en l'honneur du \'Siva\index{gnl}{Siva@\'Siva} r\'esidant \`a tel ou tel endroit, et donc, ceux qui obtiendront les effets b\'en\'efiques de leur action. Examinons d\`es lors le portrait de ce \og sujet\fg, les m\'etaphores de l'hymne\index{gnl}{hymne} et la vari\'et\'e des bienfait\index{gnl}{bienfait}s annonc\'es.

Le sujet conna\^it une diversit\'e d\'enominative remarquable. Principalement, il prend la forme du nom appellatif \textit{vall\=ar} ou \textit{vallavar} \og ceux capables, ceux forts en\fg\ avec deux cent dix-sept occurrences. Il est, syst\'ematiquement, mis en relation avec l'hymne\index{gnl}{hymne}. Ainsi, le dévot\index{gnl}{devot(e)@dévot(e)} sujet est dou\'e en musique\index{gnl}{musique} (\textit{icai vall\=ar} I~9, 11; II~106, 114; III~7 et 69), et, souvent, dans la [récitation\index{gnl}{recitation@récitation} de la] d\'ecade\footnote{I~1, 4, 5, 7, 10, 14, 22, 23, 26, 30, 34, 35, 36, 39, 40, 41, 42, 45, 48, 49, 50, 51, 60, 61, 63, 64, 67, 69, 70, 71, 76, 81, 89, 98, 100, 101, 102, 103, 106, 107, 108, 109, 110, 116, 129, 133, 134;
II~1, 3, 6, 8, 9, 14, 18, 22, 23, 25, 28, 30, 32, 36, 37, 40, 42, 44, 47, 48, 50, 52,54, 55, 56, 58, 59, 60, 61, 65, 66, 67, 71, 76, 82, 89, 93, 94, 97, 99, 102, 105, 109, 119;
III~3, 4, 13, 15, 17, 22, 23, 25, 30, 32, 33, 44, 45, 48, 50, 54, 56, 59, 60, 61, 62, 64, 65, 71, 74, 80, 89, 100, 101, 108, 110, 111, 115, 116, 121, 122, 125, 126 et 127. Voir aussi \textit{supra} sur les diff\'erentes d\'esignations de la d\'ecade.}. Il est capable d'\'ecouter (\textit{k\=e\d tal vall\=ar} I~105 et 117), de dire\footnote{\textit{urai vall\=ar} (I~93 et II~38), \textit{k\=ura vall\=ar} (I~113), \textit{colla vall\=ar} (I~6, 78, 82, 112, II~43, 69, 79, 95, 120, III~46, 103 et 112), \textit{n\=avi\b n\=al vall\=ar} (III~42) et \textit{ceppa vall\=ar} (II~63, 122 et III~51).}, de r\'eciter (\textit{\=ota vall\=ar} I~104), de chanter\footnote{\textit{p\=a\d ta vall\=ar} (I~3, 32, 33, 43, 52, 66, 80, 132; II~13, 16, 26, 53, 64, 104, 117; III~26, 95, 114 et 120).}, de louer\footnote{\textit{\=etta vall\=ar} (I~79, 97, 114, II~10, 46, 92, 118, III~2, 6, 10, 11, 49, 55, 57, 58, 66 et 106), \textit{to\b lutu \=etta vall\=ar} (II~35) et \textit{v\=a\b ltta vall\=ar} (II~21).}, de m\'editer (\textit{ni\b naiya vallavar} I~128; \textit{cintaiceya vall\=ar} III~78; \textit{cintaiyu\d l c\=erkka vall\=ar} II~91), de pratiquer (\textit{payila vall\=ar} I~75, 122 et II~45) et de r\'epandre (\textit{parava vallavar} I~24, 121 et II~110) les dix strophes qui pr\'ec\`edent l'envoi\index{gnl}{envoi}.
Le sujet peut aussi recevoir les noms appellatifs suivants: \og ceux dont la conduite\fg, \textit{n\=\i tiyar} (III~84), \og ceux dont la pens\'ee\fg, \textit{cintaiyi\b n\=ar} (III~107), et \og ceux dont le coeur\fg, \textit{ma\b nattavar} (I~90). Relevons, ensuite, des participes qui ont pour objet la d\'ecade et signifiant
\og ceux qui disent\fg\footnote{\textit{colluv\=ar} (II~116 et III~85), \textit{k\=uruv\=ar} (II~17 et 103), \textit{pakarv\=ar} (II~19), \textit{mo\b lipavar} (I~125; II~11, 49, 70, 72, 86; III~76, 77, 83, 86 et 93), \textit{uraipp\=ar} (I~72, 83; II~57; III~9, 38, 39, 63, 70, 88 et 123), \textit{navilpavar} (I~21; III~31, 118 et 119) et \textit{ceppumavar} (III~75).};
\og ceux qui chantent\fg, \textit{p\=a\d tuv\=ar} (I~58, 84, 91, 131, II~33, 75, 83, III~14, 28, 34, 36, 90 et 99);
\og ceux qui danse\index{gnl}{danser}nt\fg, \textit{\=a\d tuv\=ar} (III~90);
\og ceux qui louent\fg, \textit{\=ettuv\=ar} (I~86, 130, II~15, 29, 34, 41 et 78) et \textit{puka\b lv\=ar} (III~82 et 94);
\og ceux qui se l\`event en louange\fg, \textit{to\b lutu e\b luv\=ar} (II~111);
\og ceux qui pensent\fg, \textit{ni\b naiv\=ar} (I~37, I~46, II~4, 80 et 87) et \textit{cintaiceypavar} (III~18 et 40);
\og ceux qui sentent\fg, \textit{u\d narnt\=ar} (I~38, II~20 et III~72);
\og ceux qui consid\`erent\fg, \textit{e\d n\d nuv\=ar} (III~52);
\og ceux qui portent\fg, \textit{taritt\=ar} (II~73 et 121);
\og ceux qui aiment\fg, \textit{a\b npu ceyv\=ar} (I~73), \textit{virumpuvar} (I~54 et III~24), \textit{malkuv\=ar} (III~96) et \textit{p\=e\d nutal uriy\=ar} (I~136);
\og ceux qui ont le coeur\fg, \textit{u\d l\d lam u\d taiy\=ar} (II~7) et \textit{ma\b nam u\d taiyavar} (I~19);
\og ceux qui sont attach\'es\fg, \textit{ke\b luvi\b n\=ar} (II~77);
\og ceux qui apprennent\fg, \textit{ka\b rpavar} (I~59, II~31, 74, III~16 et 53);
\og ceux qui pratiquent\fg, \textit{payilpavar} (I~20, 126, II~68, III~76 et 102);
\og ceux qui connaissent\fg, \textit{a\b ripavar} (III~87);
\og ceux qui \'ecoutent\fg, \textit{k\=e\d t\d t\=ar} (I~59);
et \og ceux qui s'assemblent\fg, \textit{k\=u\d tuv\=ar} (I~8 et III~91). Parfois, il s'agit simplement d'\^etres humains, \textit{m\=antar} (I~2, 65, II~90 et III~37), de dévot\index{gnl}{devot(e)@dévot(e)}s, \textit{pattar} (I~47, II~88, 107 et III~79) ou, plus sp\'ecifiquement, de serviteurs, \textit{a\d tiy\=ar} (I~12, 13, 68, 77; II~39, II~62, 85 et III~81) et \textit{to\d n\d tar} (II~101 et III~73). Nous trouvons quelques pr\'ecisions quant \`a la mani\`ere d'utiliser ces strophes: nuit et jour (\textit{iravum pakalum} II~80), de fa\c con plaisante \`a l'oreille (\textit{cevikku i\b nitu \=aka} I~31), avec sinc\'erit\'e (\textit{u\d nmaiyi\b n\=al} I~79), avec habilet\'e (\textit{vittakatt\=al} II~72), selon la m\'elodie (\textit{pa\d n\d ni\b n\=al} III~34), etc. La danse\index{gnl}{danse} des dévot\index{gnl}{devot(e)@dévot(e)}s accompagne fr\'equemment ces hymne\index{gnl}{hymne}s (I~8, 74, 75; II~62; III~12, 90, 107, etc) qui b\'en\'eficient eux aussi de d\'esignations vari\'ees.

Les images employ\'ees pour d\'esigner le poème\index{gnl}{poeme@poème} nous replacent en Pays Tamoul\index{gnl}{Pays Tamoul}, dans le contexte du temple\index{gnl}{temple}. L'hymne\index{gnl}{hymne} est une guirlande\index{gnl}{guirlande} (\textit{m\=alai}\index{gnl}{malai@\textit{m\=alai}} I~30, 36, 51, 68, 86; II~76, 110, 118; III~5, 6, 48, 52, 81, 82, 83 et 101), de mots (\textit{pa\b nuvalm\=alai} I~52; \textit{colm\=alai} II~80 et 85) et de vers (\textit{p\=am\=alai} II~107 et III~119). Le plus souvent, il est appel\'e, m\'etonymiquement, par sa langue d'expression, le tamoul\footnote{\textit{tami\b l} I~1, 10, 13, 14, 18, 41, 44, 57, 60, 61, 62, 77, 81, 95, 113, 117, 135, 136; II~7, 8, 11, 17, 22, 30, 49, 61, 73, 74, 99, 102, 112, 115, 117; III~8, 10, 20, 24, 29, 30, 33, 38, 43, 44, 46, 50, 63, 64, 65, 85, 90, 95, 96, 97, 99, 105, 109, 110, 112, 125 et 126}. Parfois, c'est une guirlande\index{gnl}{guirlande} tamoule (\textit{tami\b lm\=alai} I~2, 31, 46, 74, 80, 84, 104; II~16, 63, 67, 83, 89, 103, 106, 108, 111; III~2, 4, 16, 19, 78, 106 et 118). Le chiffre dix, \textit{pattu}, nombre de strophes th\'eorique dans un poème\index{gnl}{poeme@poème} de type \textit{patikam}\index{gnl}{patikam@\textit{patikam}}, devient aussi une m\'etonymie d\'esignant l'hymne\index{gnl}{hymne}\footnote{\textit{ivai pattum}, \og ces dix\fg, I~9, I~15, 19, 20, 21, 25, 27, 29, 32, 38, 42, 49, 50, 52, 59, 64, 69, 70, 73, 76, 99, 100, 101, 105, 106, 108, 122, 123, 124, 130, 132, 133; II~9, 13, 23, 28, 38, 40, 47, 50, 51, 56, 65, 66, 68, 69, 71, 82, 90, 119, 122; III~1, 25, 32, 39, 51, 53, 54, 56, 57, 58, 84, 87, 100, 102, 103, 114, 121, 122, 127; \textit{aintu\b no\d tu aintu}, \og cinq plus cinq\fg, I~129; \textit{\=\i r-aintu}, \og deux fois cinq\fg, I~97 et II~25; \textit{p\=a\d talka\d l pattum}, \og dix chants\index{gnl}{chant}-strophes\fg, I~7 et II~3; \textit{v\=aymo\b lipattum}, \og dix veda\fg, I~75; et, \textit{mo\b lika\d lpattum}, \og dix mots-strophes I~90.}, et ce, m\^eme s'il n'y a pas dix strophes pr\'ec\'edant l'envoi\index{gnl}{envoi}\footnote{Cf. les envois\index{gnl}{envoi} des hymne\index{gnl}{hymne}s suivants: I~5, 9, 103, 105, 106, 116, 133; II~1, 6, 23, 122; III~32, 100 et 123.}. Ainsi, le poème\index{gnl}{poeme@poème} pourrait être form\'e de dix guirlandes ou d'un dizain en guirlande\index{gnl}{guirlande}\footnote{Interpr\'etation de Charlotte \textsc{Schmid} que nous remercions.} (\textit{m\=alai\index{gnl}{malai@\textit{m\=alai}} pattum} I~5, 67, 72, 79, 118; II~1, 24, 52, 86; III~104; \textit{m\=alai\index{gnl}{malai@\textit{m\=alai}} \=\i r-aintu} \og deux fois cinq guirlandes\fg\ II~37, 39; III~22 et 34 ), de dix guirlandes de mots (\textit{colm\=alaipattum} I~37) et de guirlandes au pluriel (\textit{m\=alaika\d l} III~37 et 89). Il contient dix [strophes] tamoules\footnote{\textit{tami\b l pattum} I~3, 11, 22, 33, 109, 110, 116; II~6, 29, 31, 32, 36, 41, 62, 88, 91, 92, 93, 95; III~3, 7, 9, 11, 15, 17, 18, 59, 66, 70, 72, 74, 111, 115, 116; \textit{tami\b livai}, \og ces [strophes] tamoules I~111, 112, 120, 121; II~94; III~88, 98, 120; et, \textit{tami\b lka\d l}, \og les [strophes] tamoules I~114; II~20, 59, 114; III~73 et 75}. Ces guirlandes tamoules, m\'etaphores des hymne\index{gnl}{hymne}s et des strophes, sont \`a l'image de celles offertes au dieu\index{gnl}{dieu} dans le temple\index{gnl}{temple}\footnote{Rappelons que les hymne\index{gnl}{hymne}s vishnouite\index{gnl}{vishnouite}s des \=A\b lv\=ar embrassent \'egalement cette phras\'eologie (voir n.~17 du chapitre 1).}.

Si le sujet du \textit{tirukka\d taikk\=appu}\index{gnl}{tirukkataikkappu@\textit{tirukka\d taikk\=appu}} est le dévot\index{gnl}{devot(e)@dévot(e)}, l'objet est le bienfait\index{gnl}{bienfait} qu'il obtiendra; ce dernier peut \^etre pluriel. Les actions du fid\`ele, d\'ecrites plus haut, sont elles-m\^emes une forme de p\'enitence (\textit{tavam} I~16, 118, 130; II~51, 73, 111; III~3, 49 et 50), de r\'ecompense (\textit{varam} II~108). Elles rendent le dévot\index{gnl}{devot(e)@dévot(e)} heureux (\textit{i\b npam} I~91, 111; II~97; III~21, 106 et 110), bon (\textit{nalam} I~19, 21, 30, 67, 80, 82; II~22, 55, 66, 74 et III~112), beau (\textit{e\b lil} I~22) et riche (I~123; II~40, 86; III~51 et 98). Elles suppriment le d\'em\'erite\footnote{\textit{vi\b nai}, litt\'eralement \og acte\fg, connote, en particulier dans le \textit{T\=ev\=aram}\index{gnl}{Tevaram@\textit{T\=ev\=aram}}, les mauvais actes, I~1, 6, 17, 23, 44, 46, 54, 55, 77, 107, 121, 122, 125; II~1, 24, 25, 31, 61, 71, 76, 78, 80, 84, 89, 90, 111, 113, 121; III~2, 4, 5, 15, 46, 53, 55, 60, 61, 62, 64, 68, 72, 73, 88, 92, 93, 101, 102, 103 et 121.}, le bl\^ame (\textit{pa\b li} I~101, 102; II~33, 72, 117; III~47, 90, 95, 99 et 125), la souffrance\footnote{\textit{tuyar} I~14, 26, 35, 36, 40, 70, 73, 78, 85, 97, 100, 104, 105, 136; II~5, 28, 41, 56, 67, 69, 77, 79, 106, 112; III~10, 39, 42, 45, 82, 85, 86, 87, 96, 104, 105, 107, 113, 118, 125 et 127.}, les fautes (\textit{ku\b r\b ram} II~98, 103; III~28 et 126), et lui \'epargnent le malheur\footnote{\textit{p\=avam} I~12, 29, 39, 52, 58, 71, 99; II~12, 13, 16, 19, 42, 45, 68, 93, 99, 110, 115, 8, 12; III~20, 23, 25, 26, 27, 29, 30, 32, 34, 35, 36, 48 et 125.} et la peur (\textit{ca\.nkai} III~74). Le serviteur peut mener une vie religieuse: il sera un dévot\index{gnl}{devot(e)@dévot(e)} (\textit{a\d tiyavar} I~124 et \textit{pattar} III~111), dans le bon chemin (\textit{na\b n ne\b ri} II~69, 78, 94, 107; III~33, 83 et 85), avec du m\'erite (\textit{p\=akkiyavar} III~108), qui honorera \'Siva\index{gnl}{Siva@\'Siva} (II~102) et, tous les lieux qu'il atteindra seront des \textit{t\=\i rtha} (I~45).
Il obtiendra la gloire (\textit{puka\b l} I~5, 18, 25, 86, 109, 110, 120; II~18, 49, 75, 109 et III~41), r\'egnera sur terre\index{gnl}{terre} (I~42, I~131; II~4 et III~100), et aussi dans le ciel (I~4, 20, 48, 132; II~48 et III~84).
Au final, il atteindra la gr\^ace (\textit{aru\d l} III~11 et 81), les pieds de \'Siva\index{gnl}{Siva@\'Siva}\footnote{I~2, 10, 41, 87, 113, 114; II~8, 32, 39, 63, 64, 83, 118; III~9 et 16.}, le monde de \'Siva\index{gnl}{Siva@\'Siva}\footnote{I~9, 15, 50, 60, 62, 66, 112, 129; II~53, 57, 104, 105, 119, 122; III~3, 13, 17, 18, 31, 75, 80 et 103.}, le ciel\footnote{I~3, 8, 11, 13, 24, 32, 34, 37, 43, 51, 61, 64, 74, 83, 84, 89, 106, 108, 126; II~6, 7, 11, 14, 26, 29, 34, 46, 50, 52, 54, 58, 62, 87, 92; III~44, 56, 57, 59, 65, 68, 69, 70, 71, 76, 77, 79, 89, 91, 102, 118, 119 et 123.}, la lib\'eration (I~31, 33, 72, 76, 81, 134; II~30, 36, 43, 114; III~40, 57, 59 et 104); il deviendra un dieu\index{gnl}{dieu} (III~22 et 52), \'epousera \'Sr\=\i\ (I~129); mais encore, il m\`enera une vie parmi les c\'elestes\footnote{I~49, 59, 65, 68, 98, 103, 116, 117, 133; II~3, 21, 35, 47, 59, 60, 61, 65, 70, 89, 91; III~1, 4, 6, 38, 66 et 122.}, dans laquelle il r\`egnera sur eux\footnote{I~63, 75; II~10, 15, 38, 45, 85, 88, 101, 120; III~24, 54, 58 et 78.}, couvert de leur louange (I~7, 69, 79; II~23, 37, 95, 122; III~7 et 120).

Ajoutons, enfin, qu'il existe des envois\index{gnl}{envoi} sans sujet exprimé\footnote{I~16, 18, 17, 28, 29, 44, 55, 57, 95, 111, 118, 119; II~5, 12, 24, 51, 112, 113, 115; III~8, 20, 27, 29, 41, 43, 47, 68, 92, 98, 104 et 105.}. Le simple fait de chanter, de danse\index{gnl}{danser}r, de louer, etc. procure les bienfait\index{gnl}{bienfait}s mentionn\'es ci-dessus. Signalons aussi que l'hymne\index{gnl}{hymne} lui-m\^eme peut apporter ces avantages (I~94, 123 et II~98). Notons enfin que, parfois, les formules se r\'ep\`etent d'un hymne\index{gnl}{hymne} \`a l'autre dans la succession du corpus\index{gnl}{corpus} \'etabli. Est-ce un simple hasard de la compilation\index{gnl}{compilation} effectu\'ee selon les m\`etres ou est-ce un moyen de masquer une intrusion?\footnote{\'Siva\index{gnl}{Siva@\'Siva} est appel\'e \textit{m\=eyava\b n\=e}, \og celui qui demeure\fg, \`a chaque fin de strophe, dans I 50, 51 et 52. Les constructions avec l'imp\'eratif de questionnement \textit{col\=\i r}, \og dites\fg\ (II 1, 2, 4), plac\'e au vers 3 de chaque quatrain, dans les \textit{vi\b n\=avurai}\index{gnl}{vinavurai@\textit{vi\b n\=avurai}} et le nom \textit{i\d tamp\=olum}, \og le lieu\fg\ (II 71, 72 et III 103, 104), au vers 2, refl\`etent aussi cette r\'ep\'etition. La fin des envois\index{gnl}{envoi} de II 84 et 85 s'ach\`eve par la locution \textit{\=a\d nai namat\=e}, \og [ceci est] notre ordre\index{gnl}{ordre}\fg. Nous observons des r\'ep\'etitions dans l'emploi des m\'etonymies des poème\index{gnl}{poeme@poème}s (\textit{m\=alai}\index{gnl}{malai@\textit{m\=alai}}), \og guirlande\index{gnl}{guirlande}\fg\ (III 81, 82, 83) et des strophes: \textit{pattum val\=ar}, \og dou\'e dans les dix [strophes]\fg\ (I 108, 109, 110); \textit{ivai pattum}, \og ces dix [strophes]\fg\ (III 51, 53, 54, 56, 57, 58); \textit{tami\b l pattum}, \og dix [strophes] tamoules\fg\ (II 91, 92, 93); \textit{tami\b l ivai}, \og ces [strophes] tamoules\fg\ (I 111, 112 et I 120, 121); \textit{tami\b l} (III 95, 96, 97). Les bienfait\index{gnl}{bienfait}s annonc\'es dans l'envoi\index{gnl}{envoi} reviennent aussi: la suppression des \og actes\fg, \textit{vi\b nai} (III 60, 61, 62 et III 101, 102, 103); de la \og souffrance\fg, \textit{tuyar} (III 85, 86, 87); du \og mal\fg, \textit{p\=avam} (III 23, 25, 26, 27, 29, 30, 32, 34, 35, 36); etc.}. Apr\`es cette br\`eve pr\'esentation des sujets et objets de la strophe finale, \'etudions son \'el\'ement fondamental, le portrait du poète\index{gnl}{poete@poète} Campantar\index{gnl}{Campantar}.

\subsection{Le portrait de Campantar}

La griffe figurant dans l'envoi\index{gnl}{envoi} sert de certificat d'authenticit\'e permettant de partager les hymne\index{gnl}{hymne}s entre les poète\index{gnl}{poete@poète}s. Par exemple l'hymne\index{gnl}{hymne} --- retrouv\'e grav\'e sur un mur de temple\index{gnl}{temple} (ARE 1918 8), in\'edit jusqu'alors dans les diverses \'editions du \textit{T\=ev\=aram}\index{gnl}{Tevaram@\textit{T\=ev\=aram}} et r\'ef\'erenc\'e III~126 dans le corpus\index{gnl}{corpus} \'edit\'e par \textsc{T. V.~Gopal Iyer} --- est attribu\'e \`a Campantar\index{gnl}{Campantar}, parce qu'il est pr\'ecis\'e dans l'envoi\index{gnl}{envoi} qu'il en est l'auteur\footnote{Cette inscription grav\'ee sur un mur d'un temple\index{gnl}{temple} \`a Tiruvi\d taiv\=acal daterait du \textsc{xii}\up{e} si\`ecle selon la pal\'eographie. Cf. \textsc{Gros} (1984: xxx) pour une bibliographie autour de ce texte.}. Si nous pouvons douter de la v\'eracit\'e de l'information donn\'ee, il est difficile de prouver qu'il s'agit d'un ajout. L'\'etude des \textit{tirukka\d taikk\=appu}\index{gnl}{tirukkataikkappu@\textit{tirukka\d taikk\=appu}} de Campantar\index{gnl}{Campantar} nous permet, cependant, de souligner certaines anomalies. Pour ce faire, nous distinguons les deux éléments constitutifs du poète\index{gnl}{poete@poète} dans ces strophes finales, l'une nominale et l'autre géographique.

Le poète\index{gnl}{poete@poète} signe \`a la troisi\`eme personne\footnote{L'emploi de la premi\`ere personne est attest\'e une fois dans le \textit{tirukka\d taikk\=appu}\index{gnl}{tirukkataikkappu@\textit{tirukka\d taikk\=appu}} de III~115: \textit{n\=a\b n uraitta\b na centami\b l pattum\=e}, \og les dix [strophes] en tamoul pur que j'ai dites\fg. I~99 pourrait contenir une exception, mais la premi\`ere personne est au pluriel et exprime un collectif incluant les dévot\index{gnl}{devot(e)@dévot(e)}s: \textit{p\=a\d tal pattum p\=a\d ta nam p\=avam pa\b raiyum\=e} (I~99), \og en chantant les dix strophes nos malheurs dispara\^itront\fg.}. Il est, parfois, d\'esign\'e uniquement par son nom: Panta\b n\index{gnl}{Campantar!Panta\b n} (I~9; II~29, 52 et III~13), Campanta\b n\index{gnl}{Campantar!Campanta\b n} (I~84, 87, 91, 93, 94, 95; II~40, 102 et III~56) et, surtout, \~N\=a\b nacampanta\b n\index{gnl}{Campantar!N\=a\b nacampanta\b n@\~N\=a\b nacampanta\b n}\footnote{I~117, 129; II~2, 11, 17, 18, 20, 49, 65, 74, 75, 97, 101, 111, 112, 119; III~3, 4, 5, 7, 37, 43, 46, 47, 51, 95, 96, 97, 99, 101, 109, 110, 112, 118 et 127.}. Il n'est appel\'e qu'une seule fois Tiru\~n\=a\b nacampanta\b n\index{gnl}{Campantar!Tiru\~n\=a\b nacampanta\b n} (III~81) --- nom sanctifi\'e par le pr\'efixe de majesté \textit{tiru}, qui se retrouve dans les inscriptions mentionnant son intronisation ou dans le \textit{Periyapur\=a\d nam}\index{gnl}{Periyapuranam@\textit{Periyapur\=a\d nam}} --- et ce, dans un hymne\index{gnl}{hymne} c\'el\'ebrant T\=o\d nipuram\index{gnl}{Tonipuram@T\=o\d nipuram} (C\=\i k\=a\b li\index{gnl}{Cikali@C\=\i k\=a\b li}). Ajoutons aussi les jeux de mots \'etablis entre \textit{\~n\=a\b nam}, \og connaissance\index{gnl}{connaissance}\fg, et l'anthroponyme \~N\=a\b na-cam-panta\b n (I~14, 18, 48, 82, 104; II~34, 83, 88 et III~78), avec la r\'ep\'etition de \textit{\~n\=a\b nam}, en anaphore, \`a chaque vers de l'envoi\index{gnl}{envoi} (II~2, 18 et 20).
Le poète\index{gnl}{poete@poète}, ou plutôt, son nom est accompagné de qualificatifs, souvent mélioratifs. Campantar est tamoul (I~37, 113; II~12, 47, 76; III~23, 24, 49, 75 et 104) et ma\^itrise langue (I~58) et litt\'erature (III~2) tamoules. Quelquefois, sa virtuosit\'e en po\'esie tamoule suffit \`a l'identifier, il devient l'expert tamoul (\textit{tami\b lviraki\b na\b n} I~19, 74; II~73, 113; III~19, 67, 84 et 113) ou incarne la source du tamoul (\textit{tami\b l \=akara\b n} III~117). Parall\`element, il souligne sa connaissance\index{gnl}{connaissance} des textes sacr\'es v\'ediques: il est vers\'e dans les \textit{Veda}\index{gnl}{Veda@\textit{Veda}} (I~4, 26, 47, 79, 109; II~70, 71, 89; III~64 et 89) et les \textit{A\.nga} (III~100).
Brahmane\index{gnl}{brahmane} et fervent dévot\index{gnl}{devot(e)@dévot(e)} de \'Siva\index{gnl}{Siva@\'Siva} (I~1, 56, I~111; II~59 et III~31), il appartient au \textit{ kau\d n\d dinya\index{gnl}{kaundinya@\textit{kau\d n\d dinya}} gotra} (II~122). Il est beau (I~66), c\'el\`ebre (I~24; II~92 et III~28), bon (I~34, 88, 110; II~19 et 25), fortun\'e (I~135), par\'e (I~60; II~54, 115 et 117), sans d\'efaut (I~73), sans obscurantisme (I~102 et I~114) et sans \'egal (I~81). La perfection de ses strophes est souvent soulign\'ee. Il est l'auteur (\textit{n\=ula\b n} I~75), par excellence, qui chante (I~28).
%Ces \'epith\`etes m\'elioratifs expliquent un proverbe tamoul attribu\'e \`a \'Siva lui-m\^eme qui peint les \textit{m\=uvar} ainsi: \og Appar\index{gnl}{Appar} a chant\'e ma personne, Campantar\index{gnl}{Campantar} la sienne et Cuntarar\index{gnl}{Cuntarar} les femmes\fg\footnote{texte tamoul XXXXX}.

Les cent dix-huit envois\index{gnl}{envoi} cit\'es ci-dessus, qui pr\'esentent le poète\index{gnl}{poete@poète} sans appartenance g\'eographique, constituent moins du tiers de la totalit\'e. C'est surtout en relation avec C\=\i k\=a\b li\index{gnl}{Cikali@C\=\i k\=a\b li} que Campantar\index{gnl}{Campantar} est g\'en\'eralement pr\'esent\'e.

Si nous constatons que dans les envois\index{gnl}{envoi} des soixante-sept poème\index{gnl}{poeme@poème}s attribu\'es \`a Campantar\index{gnl}{Campantar}, louant le site de C\=\i k\=a\b li\index{gnl}{Cikali@C\=\i k\=a\b li} sous douze\index{gnl}{douze} noms diff\'erents, le poète\index{gnl}{poete@poète} signe, le plus souvent, avec cinquante-et-une occurrences, sans mention de sa ville d'origine\index{gnl}{origine}\footnote{I~1, 4, 9, 19, 24, 34, 47, 60, 66, 74, 75, 79, 81, 102, 104, 109; II~11, 25, 29, 40, 49, 54, 59, 65, 70, 73, 74, 75, 83, 89, 97, 102, 113, 122; III~2, 3, 5, 7, 13, 24, 37, 43, 56, 67, 75, 81, 89, 100, 110, 113 et 118.}, ailleurs, dans les autres hymne\index{gnl}{hymne}s du corpus\index{gnl}{corpus} qui lui sont attribu\'es, pour d\'efinir l'origine\index{gnl}{origine} du poète\index{gnl}{poete@poète}, le toponyme K\=a\b li\index{gnl}{Kali@K\=a\b li} est le plus fr\'equemment employ\'e, avec cent cinquante-trois occurrences. L'association entre [C\=\i ]k\=a\b li et Campantar\index{gnl}{Campantar} est si \'evidente que le nom du poète\index{gnl}{poete@poète} n'est plus n\'ecessaire dans certains \textit{tirukka\d taikk\=appu}\index{gnl}{tirukkataikkappu@\textit{tirukka\d taikk\=appu}}\footnote{Le poète\index{gnl}{poete@poète} est \og celui de K\=a\b li\index{gnl}{Kali@K\=a\b li}\fg, \textit{k\=a\b liy\=a\b n} II~114; \og le roi\index{gnl}{roi} de K\=a\b li\index{gnl}{Kali@K\=a\b li}, ma\^itre de la lign\'ee des brahmane\index{gnl}{brahmane}s \textit{kavu\d ni}\index{gnl}{kaundinya@\textit{kau\d n\d dinya}!\textit{kavu\d ni}}\fg, \textit{kavu\d niyarkulapati k\=a\b liyark\=o\b n} I~112; \og le brahmane\index{gnl}{brahmane} \textit{kavu\d ni}\index{gnl}{kaundinya@\textit{kau\d n\d dinya}!\textit{kavu\d ni}} de K\=a\b li\index{gnl}{Kali@K\=a\b li}\fg, \textit{k\=a\b lik kavu\d niya\b n} II~9; \og le sage\index{gnl}{sage} de la connaissance\index{gnl}{connaissance} dont la ville est \dots\ K\=a\b li\index{gnl}{Kali@K\=a\b li}\fg, \textit{k\=a\b li \dots\ pati \=a\b na \~n\=a\b namu\b niva\b n} II~84; \og le chef\index{gnl}{chef} des habitants de K\=a\b li\index{gnl}{Kali@K\=a\b li}\fg, \textit{k\=a\b liyar tam talaiva\b n} II~23; \og l'expert du tamoul pur \dots\ roi\index{gnl}{roi} des habitants de K\=a\b li\index{gnl}{Kali@K\=a\b li}\fg, \textit{k\=a\b liyark\=o\b n \dots\ centami\b li\b n viraka\b n} II~24. Nous observons le m\^eme fonctionnement avec le toponyme Ka\b lumalam\index{gnl}{Kalumalam@Ka\b lumalam}: \textit{ka\b lumala mutupatik kavu\d niya\b n}, \og le brahmane\index{gnl}{brahmane} \textit{kavu\d ni}\index{gnl}{kaundinya@\textit{kau\d n\d dinya}!\textit{kavu\d ni}} de l'ancienne ville de Ka\b lumalam\index{gnl}{Kalumalam@Ka\b lumalam}\fg\ (I~127 et 128); \textit{ka\b lumalanakar i\b rai tami\b lviraka\b n}, \og seigneur de la ville de Ka\b lumalam\index{gnl}{Kalumalam@Ka\b lumalam}, l'expert tamoul\fg\ (I~22 et 123).}. Souvent, Campantar est simplement attach\'e au site\footnote{\~N\=a\b nacampanta\b n\index{gnl}{Campantar!N\=a\b nacampanta\b n@\~N\=a\b nacampanta\b n} de K\=a\b li\index{gnl}{Kali@K\=a\b li} I~3, 6, 10, 23, 31, 32, 33, 35, 38, 46, 49, 55, 59, 64, 65, 69, 80, 82, 86, 134; II~3, 4, 8, 15, 27, 28, 35, 46, 55, 58, 60, 63, 68, 69, 79, 90, 91, 93, 100, 103, 108, 109, 121; III~6, 9, 12, 16, 21, 30, 34, 35, 38, 45, 48, 53, 78, 93, 105, 106, 108, 116, 120, 122, 125 et 126~/ Campanta\b n\index{gnl}{Campantar!Campanta\b n} de K\=a\b li\index{gnl}{Kali@K\=a\b li} I~62, 71, 89, 92; II~41 et 64~/ Panta\b n\index{gnl}{Campantar!Panta\b n} le brillant de K\=a\b li\index{gnl}{Kali@K\=a\b li} I~119~/ Panta\b n\index{gnl}{Campantar!Panta\b n} de K\=a\b li\index{gnl}{Kali@K\=a\b li} II~33, 87, 88 et III~40~/ \~N\=a\b napanta\b n\index{gnl}{Campantar!N\=a\b napanta\b n@\~N\=a\b napanta\b n} de K\=a\b li\index{gnl}{Kali@K\=a\b li} II~86.}. Nous retrouvons, par ailleurs, toutes les qualit\'es d\'ecrites plus haut: Campantar\index{gnl}{Campantar} est un brahmane\index{gnl}{brahmane} de K\=a\b li\index{gnl}{Kali@K\=a\b li} (I~7 et 17) du \textit{kau\d n\d dinya\index{gnl}{kaundinya@\textit{kau\d n\d dinya}} gotra} (I~8, II~43, 51, 72 et III~76). Il est bon (II~95) et brillant (I~43). Il clame son identit\'e\index{gnl}{identit\'e} tamoule (I~15, 39, 44, 61, 99; II~45, 67, 82, 94, 118; 11, 25, 42, 58, 62, 65, 66, 77 et 79) et sa connaissance\index{gnl}{connaissance} des \textit{Veda}\index{gnl}{Veda@\textit{Veda}} (I~130; II~7, 53, 106; III~1, 8, 14, 20, 36, 44, 70 et 91). Parfois, il souligne simultan\'ement cette double culture (I~40; II~116 et III~22). Il reste un pieux serviteur de \'Siva\index{gnl}{Siva@\'Siva} (I~77 et II~120), de Celui localis\'e particuli\`erement \`a K\=a\b li\index{gnl}{Kali@K\=a\b li} (I~29, 78 et III~63). Il se proclame m\^eme protecteur de cette ville (I~5, 27, 36; II~6, 10, 38, 107; III~72 et 123). Enfin, artiste (I~2, 11, 16; II~37, 61 et III~115), il conna\^it la gloire (I~68; II~14, 42; III~26 et 55).

Les autres toponymes les plus r\'ecurrents sont Pukali\index{gnl}{Pukali} (quarante-quatre occurrences), Ka\b lumalam\index{gnl}{Kalumalam@Ka\b lumalam} (vingt-et-une) et Ca\d npai\index{gnl}{Canpai@Ca\d npai} (seize). Les noms restants sont beaucoup moins pr\'esents dans les \textit{tirukka\d taikk\=appu}\index{gnl}{tirukkataikkappu@\textit{tirukka\d taikk\=appu}}: sept occurrences pour Cirapuram\index{gnl}{Cirapuram} (I~20, 21, 51, 52, 132; II~110 et III~71) et Koccai\index{gnl}{Koccai} (I~85, 106; II~39, 44, 50, 57 et III~41), six pour T\=o\d nipuram\index{gnl}{Tonipuram@T\=o\d nipuram} (I~48; II~5; III~50, 82, 83 et III~119), trois pour V\=e\d nupuram\index{gnl}{Venupuram@V\=e\d nupuram} (I~67, 70 et 83) et Tar\=ay\index{gnl}{Taray@Tar\=ay} (I~96; II~13 et 104), deux pour Piramapuram\index{gnl}{Piramapuram} (I~53 et II~85) et Ve\.nkuru\index{gnl}{Venkuru@Ve\.nkuru} (III~59 et 80) et, enfin, une seule pour Pu\b ravam\index{gnl}{Puravam@Pu\b ravam} (III~29). Nous constatons donc un traitement tr\`es in\'egal des douze\index{gnl}{douze} toponymes dans les envois\index{gnl}{envoi}. Serait-ce r\'ev\'elateur d'un artifice form\'e autour de cette unit\'e de douze\index{gnl}{douze} noms?

Par ailleurs, Campantar\index{gnl}{Campantar} signe souvent en exprimant sa souverainet\'e sur la ville de C\=\i k\=a\b li\index{gnl}{Cikali@C\=\i k\=a\b li}. Il est le roi\index{gnl}{roi}\footnote{\textit{k\=o\b n} (I 5, 51, 52, 97, 101, 112, 126; II 24, 38; III 62 et 72), \textit{ma\b n} (I 27, 100, 106; II 105; III 33 et 39), \textit{ma\b n\b na\b n} (I 36, 99, 108; II 104; III 29 et 22), \textit{ma\b n\b nava\b n} (II 32) et \textit{v\=enta\b n} (I 50; II 57 et III 41)}, le chef\index{gnl}{chef}\footnote{\textit{talaiva\b n} (I 14, 48, 53; II 23, 44; III 57, 60, 61, 69 et 71), \textit{talaimaka\b n} (I 107), \textit{atipati} (I 77) et \textit{atipa\b n} (II 82)}, le protecteur (\textit{k\=avala\b n} II 6, 10 et 56) et le seigneur\footnote{\textit{i\b rai} (II 5, 20, 21, 22, 30, 50, 107, 123 et III 50), \textit{i\b raiva\b n} (II 39) et \textit{perum\=a\b n} (III 123)} de ce site. Ces termes, habituellement, employ\'es pour d\'esigner \'Siva\index{gnl}{Siva@\'Siva} dans les hymne\index{gnl}{hymne}s\footnote{\'Siva\index{gnl}{Siva@\'Siva} est \og le roi\index{gnl}{roi} des habitants de K\=a\b li\index{gnl}{Kali@K\=a\b li}\fg\ (II 16.11: \textit{k\=a\b liyark\=o\b n}), \og le seigneur r\'esidant \`a Ka\b lumalam\index{gnl}{Kalumalam@Ka\b lumalam}\fg\ (III 113.12: \textit{Ka\b lumalam\index{gnl}{Kalumalam@Ka\b lumalam} amar i\b rai}), \og le protecteur demeurant \`a Piramapuram\index{gnl}{Piramapuram}\fg\ (II 40.11: \textit{piramapurattu uraiyum k\=avala\b nai}), \og le chef\index{gnl}{chef} habitant avec plaisir \`a Ka\b lumalam\index{gnl}{Kalumalam@Ka\b lumalam}\fg\ (I 19.5: \textit{Ka\b lumalam\index{gnl}{Kalumalam@Ka\b lumalam} i\b nitu amar talaiva\b n\=e}) et \og le roi\index{gnl}{roi} de Koccai\index{gnl}{Koccai}\fg\ (I 90.11: \textit{Koccai\index{gnl}{Koccai} ma\b n}).} glorifient le poète\index{gnl}{poete@poète} qui est ainsi plac\'e sur le m\^eme plan que son dieu\index{gnl}{dieu}. Nous relevons, par exemple, un vocabulaire commun dans l'envoi\index{gnl}{envoi} de l'hymne\index{gnl}{hymne} I 123, d\'edi\'e \`a Valivalam\index{gnl}{Valivalam}, qui sugg\`ere une identit\'e\index{gnl}{identit\'e} entre \'Siva\index{gnl}{Siva@\'Siva} et Campantar\index{gnl}{Campantar}:

\scriptsize
\begin{verse}
\textit{ma\b n\b niya \textbf{valivalanakar} u\b rai \textbf{i\b raiva\b nai},\\
i\b n iyal \textbf{ka\b lumalanakar i\b rai} --- e\b lil ma\b rai\\
ta\b n iyal kalai vala tami\b lviraka\b natu --- urai\\
u\b n\b niya orupatum uyarporu\d l tarum\=e.} (I 123.11)\\
\end{verse}

\normalsize
\begin{verse}
La dizaine [de strophes] qui consid\`erent les mots\\
Du seigneur de la ville, naturellement belle, de Ka\b lumalam\index{gnl}{Kalumalam@Ka\b lumalam},\\
De l'expert tamoul fort dans les arts propres aux beaux \textit{Veda}\index{gnl}{Veda@\textit{Veda}},\\
Sur le seigneur demeurant dans la ville de Valivalam\index{gnl}{Valivalam},\\
Procure la lib\'eration supr\^eme. (I 123.11)\\
\end{verse}


\noindent La description des poète\index{gnl}{poete@poète}s comme souverains dans les envois\index{gnl}{envoi} est r\'epandue\footnote{Periy\=a\b lv\=ar\index{gnl}{Periyalvar@Periy\=a\b lv\=ar} et Tiruma\.nkaiy\=a\b lv\=ar\index{gnl}{Tirumankaiyalvar@Tiruma\.nkaiy\=a\b lv\=ar} appr\'ecient aussi ces images, ce qui conduit \textsc{Hardy} (*2001 [1983]: 253-255) \`a supposer que ces deux poète\index{gnl}{poete@poète}s vishnouite\index{gnl}{vishnouite}s \'etaient des souverains ou chef\index{gnl}{chef}s locaux\index{gnl}{local}. Cependant, nous ne pouvons en faire autant avec Campantar\index{gnl}{Campantar} qui, contrairement aux poète\index{gnl}{poete@poète}s vishnouite\index{gnl}{vishnouite}s, n'est jamais pourvu d'armes, ou d'autres attributs, mais qui souligne, contamment, sa caste brahmane\index{gnl}{brahmane} et son \'erudition sanskrite et tamoule.}. Dans les \textit{tirukka\d taikk\=appu}\index{gnl}{tirukkataikkappu@\textit{tirukka\d taikk\=appu}}, Cuntarar\index{gnl}{Cuntarar} se pla\^it \`a dire qu'il porte le nom du seigneur d'\=Ar\=ur\index{gnl}{Ar\=ur@\=Ar\=ur}, \=Ar\=ura\b n\index{gnl}{Cuntarar!\=Ar\=ura\b n} (VII 59) ou \`a jouer avec ce nom, permettant ainsi une assimilation du dévot\index{gnl}{devot(e)@dévot(e)} au dieu\index{gnl}{dieu}. Par exemple, il est \og celui qui est aux pieds d'\=Ar\=ura\b n\index{gnl}{Cuntarar!\=Ar\=ura\b n}, le serviteur aux pieds, \=Ar\=ura\b n\index{gnl}{Cuntarar!\=Ar\=ura\b n}\fg\ (VII 21: \textit{\=ar\=ura\b n a\d tiy\=a\b n, a\d tito\d n\d ta\b n, \=ar\=ura\b n}). Par son origine\index{gnl}{origine}, il se proclame roi\index{gnl}{roi} de N\=aval\=ur\index{gnl}{Navalur@N\=aval\=ur} (\textit{k\=o\b n} VII 3, 4 18, 23, 24, 28, 39, 42, 83, 84, 101 / \textit{v\=enta\b n} 57 / \textit{\=a\d li} 64 / \textit{k\=oma\b n} 82) et d\'esigne \'Siva\index{gnl}{Siva@\'Siva} pareillement dans un hymne\index{gnl}{hymne} c\'el\'ebrant Ma\b raikk\=a\d tu\index{gnl}{Maraikkatu@Ma\b raikk\=a\d tu} (VII 71.11: \textit{n\=avalarv\=enta\b n}). Cependant, il n'est jamais le seigneur, \textit{i\b rai} ou \textit{perum\=a\b n}.

Ainsi, la d\'eification de Campantar\index{gnl}{Campantar} dans certains envois\index{gnl}{envoi} soul\`eve ici encore, \`a notre avis, la question des ajouts tardifs.

Bien que nous doutions de l'authenticit\'e de certains envois\index{gnl}{envoi}, ces derniers, dans l'ensemble, sont une source substantielle d'informations identitaires. Nous apprenons que Campantar\index{gnl}{Campantar} est un brahmane\index{gnl}{brahmane} du \textit{kau\d n\d dinya\index{gnl}{kaundinya@\textit{kau\d n\d dinya}} gotra}, un poète\index{gnl}{poete@poète} tamoul ma\^itrisant les \textit{Veda}\index{gnl}{Veda@\textit{Veda}}, un dévot\index{gnl}{devot(e)@dévot(e)} de \'Siva\index{gnl}{Siva@\'Siva}, et qu'il est originaire de C\=\i k\=a\b li\index{gnl}{Cikali@C\=\i k\=a\b li}. Examinons maintenant les donn\'ees internes du corpus\index{gnl}{corpus} qui ont \'egalement servi \`a construire son hagiographie\index{gnl}{hagiographie}.

\section{Campantar dans le \textit{T\=ev\=aram}}

Nous pouvons penser qu'un poète peut employer la première personne pour évoquer des moments de sa vie personnelle.
Dans les hymne\index{gnl}{hymne}s attribu\'es \`a Campantar, l'emploi de la premi\`ere personne se limite \`a exprimer la ferveur religieuse du poète\index{gnl}{poete@poète} dévot\index{gnl}{devot(e)@dévot(e)}. Elle appara\^it, par exemple, dans les adresses \`a \'Siva\index{gnl}{Siva@\'Siva}, dans ses descriptions\footnote{Notons des expressions telles que \og mon seigneur est de l'ambroisie\index{gnl}{ambroisie} pour moi\fg\ (\textit{em pir\=a\b n e\b nakku amutam} II 40, 1), \og celui qui me gouverne\fg\ (\textit{e\b nai \=a\d n\d tava\b n} III 24, 2), \og mon seigneur\fg\ (\textit{em i\b rai} I 4; I 104; II 25, 7; III 5, 4; etc.).} ou encore, dans les poème\index{gnl}{poeme@poème}s mettant en sc\`ene une dévot\index{gnl}{devot(e)@dévot(e)}e languissante. Purement po\'etique, elle ne renvoie \`a aucune r\'ealit\'e personnelle relative \`a l'auteur. De la m\^eme fa\c con, les imp\'eratifs, qui permettent un \'echange direct avec le lecteur ou l'auditeur, rel\`event strictement de la phras\'eologie po\'etique\footnote{Dans les envois\index{gnl}{envoi}, le poète\index{gnl}{poete@poète} invite son public \`a chanter et \`a r\'epandre ses d\'ecades: \og vivez en r\'epandant les dix strophes\fg, \textit{p\=a\d tal pattum paravi v\=a\b lmi\b n\=e} I 27; \og r\'epandez en chantant les dix strophes en tamoul plaisant\fg, \textit{i\b n tami\b l pattum p\=a\d tip paravum\=e} I 56; \og vivez en ayant atteint, en chant, [le temple\index{gnl}{temple} d'] A\b n\b niy\=ur du roi\index{gnl}{roi}\fg, \textit{p\=a\d tal\=al v\=enta\b n a\b n\b niy\=ur c\=erntu v\=a\b lmi\b n\=e} I 96; \og vivez en r\'ecitant les chants\index{gnl}{chant}\fg, \textit{p\=a\d tal ko\d n\d tu \=oti v\=a\b lmi\b n\=e} II 27; \og louez\fg, \textit{\=ettumi\b n} II 100; \og O dévot\index{gnl}{devot(e)@dévot(e)}s, chantez\fg, \textit{p\=a\d tumi\b n, pattark\=a\d l!} II 107; \og dites\fg, \textit{mo\b liyumi\b n} II 108; \og vivez en louant! vos actes li\'es \`a l'attachement seront d\'etruits\fg, \textit{\=etti v\=a\b lum! num pantam \=ar vi\b nai p\=a\b ri\d tum\=e} III 5; \og dites l'[hymne\index{gnl}{hymne}] tamoul de \~N\=a\b nacampanta\b n\index{gnl}{Campantar!N\=a\b nacampanta\b n@\~N\=a\b nacampanta\b n}\fg, \textit{\~n\=a\b nacampanta\b n tami\b l collum\=e} III 109; \og vous qui r\'ecitez, obtenez la guirlande\index{gnl}{guirlande} faite en tamoul par le brillant expert tamoul\fg, \textit{\=otuv\=\i r! ko\d nmi\b n --- tami\b l ke\b lu viraki\b na\b n tami\b lceym\=alaiy\=e!} III 19.

Dans les refrains, Campantar l'exhorte \`a atteindre certains sites (\og atteignons Mutuku\b n\b ru\fg, \textit{mutuku\b n\b ru a\d taiv\=om\=e} I 12; \og allons et atteignons C\=o\b r\b ruttu\b rai\fg, \textit{c\=o\b r\b ruttu\b rai ce\b n\b ru a\d taiv\=om\=e} I 28; \og rejoignez K\=a\b li\index{gnl}{Kali@K\=a\b li}\fg, \textit{k\=a\b li c\=ermi\b n\=e} II 97, etc.) et \`a louer \'Siva\index{gnl}{Siva@\'Siva} (\og honorez le T\=u\.nk\=a\b naim\=a\d tam du grand temple\index{gnl}{temple} de Ka\d tantai\fg, \textit{ka\d tantait ta\d ta\.nk\=oyil c\=er t\=u\.nk\=a\b naim\=a\d tam to\b lumi\b nka\d l\=e} I 59, etc.).}.

Campantar\index{gnl}{Campantar} intervient peu, \`a la premi\`ere personne, pour faire allusion aux \'ev\`enements de sa vie. Sa biographie est constituée à partir d'autres passages du \textit{T\=ev\=aram} qui renverraient à sa vie personnelle.

Nous pr\'esentons ici les quelques r\'ef\'erences de type biographique\index{gnl}{biographie!biographique}, habituellement cit\'ees dans la littérature secondaire, figurant dans les poème\index{gnl}{poeme@poème}s attribu\'es \`a Campantar\index{gnl}{Campantar} et, parall\`element, nous discutons leur fiabilit\'e avant d'analyser les descriptions de Campantar\index{gnl}{Campantar} faites par les deux autres \textit{m\=uvar}\index{gnl}{muvar@\textit{m\=uvar}}.

\subsection{Les allusions dites \og autobiographiques\fg}

Les quelques r\'ef\'erences suivantes sont relev\'ees dans l'\oe uvre attribu\'ee \`a Campantar\index{gnl}{Campantar} et sont présentées dans l'ordre chronologique de la biographie de Campantar établie par le \textit{Periyapur\=a\d nam}, sur laquelle nous reviendrons dans le chapitre 6. Ces références \'evoqueraient, pour certains auteurs, des \'el\'ements autobiographique\index{gnl}{autobiographique}s\footnote{Signalons que \textsc{Rangaswamy} (*1990 [1958]: 977-984), \textsc{Soundra} (1979: 31-45) et \textsc{Somasundaram} (1986: 28-45) pr\'esentent ces allusions. Cependant, ils expliquent souvent les passages en accord avec le \textit{Periyapur\=a\d nam}\index{gnl}{Periyapuranam@\textit{Periyapur\=a\d nam}} sans examiner seuls les hymne\index{gnl}{hymne}s du \textit{T\=ev\=aram}\index{gnl}{Tevaram@\textit{T\=ev\=aram}}. \textsc{Soundra} reprend, un \`a un, tous les hymne\index{gnl}{hymne}s attach\'es aux miracle\index{gnl}{miracle}s de Campantar\index{gnl}{Campantar} et donn\'es dans l'hagiographie\index{gnl}{hagiographie}, \`a l'exception de ceux li\'es aux \'ev\`enements de Maturai\index{gnl}{Maturai}, puis les analyse. Compte tenu du fait que cet auteur, d\`es le d\'ebut de son \'etude et sans analyse, est convaincu de l'authenticit\'e des allusions biographiques\index{gnl}{biographie!biographique} dans tous les passages que nous citons, nous l'ignorons dans notre pr\'esentation. \textsc{Somasundaram} fait un travail semblable. Nous \'ecartons aussi ses interpr\'etations qui, acceptant syst\'ematiquement toutes ces strophes dites \og autobiographique\index{gnl}{autobiographique}s\fg\ comme des allusions biographiques\index{gnl}{biographie!biographique}, pour la seule raison qu'elles sont cit\'ees dans le \textit{Periyapur\=a\d nam}\index{gnl}{Periyapuranam@\textit{Periyapur\=a\d nam}}, cherchent des indices \`a tout prix.}.

La deuxi\`eme strophe de l'hymne\index{gnl}{hymne} III 24, d\'edi\'e \`a Ka\b lumalam\index{gnl}{Kalumalam@Ka\b lumalam} ferait allusion au miracle\index{gnl}{miracle} du don\index{gnl}{don} de lait\index{gnl}{lait}\footnote{Nous mettons en gras les passages des poèmes du \textit{T\=ev\=aram} qui ont été utilisés pour justifier la biographie de Campantar.}:

\scriptsize
\begin{verse}
\textit{p\=otai \=ar \textbf{po\b n ki\d n\d nattu a\d ticil poll\=atu e\b nat\\
t\=ataiy\=ar mu\b nivu u\b ra}, t\=a\b n e\b nai \=a\d n\d tava\b n;\\
k\=atai \=ar ku\b laiyi\b na\b n; ka\b lumala va\d la nakar,\\
p\=etaiy\=a\d l ava\d lo\d tum peruntakai iruntat\=e!} (III 24.2)\\
\end{verse}

\normalsize
\begin{verse}
Quand le père\index{gnl}{pere@père} en col\`ere dit que\\
La nourriture de la coupe d'or\\
Telle une fleur, est mauvaise,\\
Il me guida,\\
Celui \`a la boucle sur l'oreille, \\
Le grand qui demeure avec la jeune femme\\
Dans la ville prosp\`ere de Ka\b lumalam\index{gnl}{Kalumalam@Ka\b lumalam}. (III 24.2)\\
\end{verse}

\normalsize
\noindent Campantar\index{gnl}{Campantar}, s'exprimant \`a la premi\`ere personne (\`a l'accusatif, \textit{e\b nai}), mentionne simplement que son père critique une forme de nourriture et que \'Siva fut alors son guide. Cette stance a souvent été mise en correspondance avec l'\'episode narr\'e dans le \textit{Periyapur\=a\d nam}\index{gnl}{Periyapuranam@\textit{Periyapur\=a\d nam}} o\`u le père\index{gnl}{pere@père} de Campantar, ne sachant pas qui lui a donn\'e du lait\index{gnl}{lait}, l\`eve la main pour le r\'eprimander. Cependant, dans cette légende\index{gnl}{legende@légende}, selon le \textit{Periyapur\=a\d nam}\index{gnl}{Periyapuranam@\textit{Periyapur\=a\d nam}}, le père\index{gnl}{pere@père} ne dit pas que la nourriture, \`a savoir le lait\index{gnl}{lait}, est mauvaise; il demande seulement, craignant une \'eventuelle pollution, qui le lui a donn\'e\footnote{Il en est de m\^eme dans la version attribu\'ee \`a Pa\d t\d ti\b nattuppi\d l\d lai\index{gnl}{Pattinattu Pillai@Pa\d t\d ti\b nattuppi\d l\d lai} (voir 5.1.1).}.



Dans le \textit{tirukka\d taikk\=appu}\index{gnl}{tirukkataikkappu@\textit{tirukka\d taikk\=appu}} de l'hymne\index{gnl}{hymne} II 84, \`a la gloire de Na\b nipa\d l\d li\index{gnl}{Nanipalli@Na\b nipa\d l\d li}, le poète\index{gnl}{poete@poète} se pr\'esente comme le sage\index{gnl}{sage} de la connaissance\index{gnl}{connaissance}, \textit{\~n\=a\b namu\b ni}, qui a compos\'e la d\'ecade en \'etant assis sur les épaule\index{gnl}{epaule@épaule}s de son père\index{gnl}{pere@père}:

\scriptsize
\begin{verse}
\textit{ka\d tal varai \=otam malku ka\b li k\=a\b nal p\=a\b nal kama\b l k\=a\b li e\b n\b ru karuta,\\
pa\d tu poru\d l \=a\b rum n\=alum u\d latu \=aka vaitta pati \=a\b na \~n\=a\b namu\b niva\b n,\\
i\d tu pa\b rai o\b n\b ra \textbf{attar piyalm\=el iruntu} i\b n icaiy\=al uraitta pa\b nuval,\\
na\d tu iru\d l \=a\d tum entai na\b nipa\d l\d li u\d lka, vi\b nai ke\d tutal \=a\d nai namat\=e.} (II 84.11)\\
\end{verse}

\normalsize
\begin{verse}
Les actes p\'eriront \`a la pens\'ee du chant prononc\'e,\\
En plaisante musique\index{gnl}{musique}, sur Na\b nipa\d l\d li\index{gnl}{Nanipalli@Na\b nipa\d l\d li}\\
\`A propos de notre seigneur qui, au son des tambours,\\
danse\index{gnl}{danser} en pleine nuit,\\
Par le sage\index{gnl}{sage} de la connaissance\index{gnl}{connaissance}, assis sur la nuque du père\index{gnl}{pere@père},\\
Qui a pour ville K\=a\b li\index{gnl}{Kali@K\=a\b li} que parfument les n\'elombos odorants\\
Des lagunes o\`u abondent les montagnes de vague\index{gnl}{vague}s maritimes,\\
Et o\`u demeurent les six [\textit{A\.nga}] et les quatre [\textit{Veda}\index{gnl}{Veda@\textit{Veda}}] de la lib\'eration;\\
Ceci est notre ordre\index{gnl}{ordre}. (II 84.11)\\
\end{verse}

\normalsize
\noindent Cette description fait \'echo \`a l'image parfaitement fig\'ee dans la dévotion populaire de l'enfant\index{gnl}{enfant} qui se promène assis sur les épaule\index{gnl}{epaule@épaule}s de son père\index{gnl}{pere@père}. Des représentations modernes figurent \'Siva portant l'enfant Skanda sur ses épaules (fig. 2.1)\footnote{\textsc{L'Hernault} (1978: 51) mentionne rapidement cette représentation qui lui rappelle un vers de la \textit{Vi\'svagu\d n\=adar\'sacamp\=u} évoquant \'Siva portant Skanda sur son giron, puis ce chercheur décrit que sur cette image en bois \'Siva porte Skanda \og assis au creux de son bras\fg. L'enfant Skanda est souvent représenté assis sur le giron de sa mère dans la manifestation du Som\=askanda. Nous supposons que sur cette image de Mailam le sculpteur n'a pas voulu représenter Skanda assis sur le giron ou au creux du bras de \'Siva mais plutôt sur ses épaules et qu'il a dû probablement surmonter la difficulté de positionner Skanda sur les deux épaules gauches d'un \'Siva à quatre bras.}. D'après le \textit{Periyapur\=a\d nam}, l'enfant Campantar se déplace sur les épaules de son père dans ses premiers pèlerinages (fig. 2.2).

\begin{figure}[!h]
  \begin{minipage}[c]{0.45\textwidth}
  \centering
  \includegraphics[height=4.5cm]{docthese/skandasurepaule.JPG}
  \caption{Skanda assis sur les épaules de son père \'Siva, panneau de bois du char du temple de Skanda à Mailam dans le taluk de Ti\d n\d tiva\b nam, Te\b n\b n\=a\b rk\=a\d tu dt. (cliché IFP/EFEO 6889-7 dans \textsc{L'Hernault} 1978: Ph. 4).}
  \end{minipage}\hspace{1cm}
  \begin{minipage}[c]{0.45\textwidth}
  \centering
  \includegraphics[height=4.5cm]{docthese/tncsurepaules.JPG}
  \caption{Campantar assis sur les épaules de son père lors des premiers pèlerinages, peinture du mur sud de la petite chapelle de Campantar dans le temple principal de \'Siva (A14), C\=\i k\=a\b li (cliché U. \textsc{Veluppillai}, 2003).}
  \end{minipage}
\end{figure}

\noindent Notons que ce d\'etail formul\'e dans l'envoi\index{gnl}{envoi} est fort inhabituel car Campantar\index{gnl}{Campantar} ne mentionne jamais ses parents pour se pr\'esenter. Nommer sa parent\'e est une caract\'eristique des envois\index{gnl}{envoi} attribu\'es \`a Cuntarar\index{gnl}{Cuntarar} qui se pr\'esente comme le fils de Ca\d taiya\b n et d'Icai\~n\=a\b ni, et comme le père\index{gnl}{pere@père} de Ci\.nka\d ti et de Va\b nappakai (VII 7, 12, 16, 29, 30, 33, 34, 37, 38, 39, 42, 47, 57, 58, 70, 87 et 98). D'autre part cet envoi\index{gnl}{envoi} est particulier parce qu'il contient l'expression \og ceci est notre ordre\index{gnl}{ordre}\fg\ (\textit{\=a\d nai namatu}), dont il y a peu d'occurrences dans le \textit{T\=ev\=aram}\index{gnl}{Tevaram@\textit{T\=ev\=aram}} (II 85, III 78 et 118) mais qui est reprise dans les textes post\'erieurs attribu\'es \`a Nampi\index{gnl}{Nampi \=A\d n\d t\=ar Nampi} (\textit{APT} 45) et \`a C\=ekki\b l\=ar\index{gnl}{Cekkilar@C\=ekki\b l\=ar} (\textit{Periyapur\=a\d nam}\index{gnl}{Periyapuranam@\textit{Periyapur\=a\d nam}} st. 2013) et dans une inscription (\textit{SII} 8 442 l. 24).

Puis, l'hymne\index{gnl}{hymne} I 92 construit selon le procédé\index{gnl}{procédé littéraire} du \og distique \textsubring{r}gv\'edique\fg, c\'el\'ebrant V\=\i \b limi\b lalai\index{gnl}{Vilimilalai@V\=\i \b limi\b lalai}, ferait allusion \`a la demande de pièce\index{gnl}{piece@pièce}s d'or pour combattre la famine dans la légende de Campantar. Rappelons que cette requ\^ete du poète\index{gnl}{poete@poète} n'est qu'une parmi d'autres et qu'elle s'apparente beaucoup \`a celle des bardes de la litt\'erature du \textit{Ca\.nkam}\index{gnl}{Cankam@\textit{Ca\.nkam}} (voir 2.1.2). De plus, les passages mentionnant les pièce\index{gnl}{piece@pièce}s ne sont pas clairs. Le premier vers, \textit{v\=aci t\=\i rav\=e, k\=acu nalkuv\=\i r}, que nous pouvons traduire ainsi: \og donnez des pièce\index{gnl}{piece@pièce}s pour d\'etruire \textit{v\=aci}\fg, nous est incompr\'ehensible. La d\'efinition du terme \textit{v\=aci} est probl\'ematique. Celle propos\'ee par le \textit{Tamil Lexicon} (\textit{i.e.} \og discount, in changing money\fg), sur la base des commentaires qui l'expliquent uniquement en accord avec l'\'episode narr\'e dans le \textit{Periyapur\=a\d nam}\index{gnl}{Periyapuranam@\textit{Periyapur\=a\d nam}}, est contestable. Signalons toutefois que ce terme appara\^it dans les textes \'epigraphiques suivi du participe n\'egatif \textit{pa\d t\=a} signifiant ainsi \og sans manque, sans d\'efaut\fg\ (voir par exemple SII 12 190 l.9 donn\'ee par \textsc{Subbarayalu} (2003: 547) et SII 5 1409 l.49) et qu'il est associ\'e \`a l'argent. Contrairement \`a \textsc{Rangaswamy}, nous n'y percevons en tout cas aucune allusion biographique\index{gnl}{biographie!biographique}.

Campantar\index{gnl}{Campantar}, employant la premi\`ere personne dans la strophe inaugurale du poème\index{gnl}{poeme@poème} II 37, d\'edi\'e \`a Ma\b raikk\=a\d tu\index{gnl}{Maraikkatu@Ma\b raikk\=a\d tu}, demande \`a Siva de lui rendre gr\^ace (\textit{aru\d l ceyka e\b nakku}) par le fait de fermer ses portes (\textit{u\b n katavam tirukk\=appuk ko\d l\d lum karutt\=al\=e}). Cependant, la structure et le th\`eme changent dans les quatrains suivants o\`u \'Siva\index{gnl}{Siva@\'Siva} est interrog\'e sur ses diff\'erents exploits\footnote{Le poète\index{gnl}{poete@poète} questionne le dieu\index{gnl}{dieu} sur le fait de porter la Ga\.ng\=a\index{gnl}{Ganga@Ga\.ng\=a} dans ses cheveux (st. 2), de r\'eunir le serpent\index{gnl}{serpent} et la lune dans son chignon (st. 3), de boire le poison (st. 4), de prendre la forme du chasseur (st. 5), de mendier dans un cr\^ane (st. 6), de consumer K\=ama\index{gnl}{Kama@K\=ama} (st. 7), d'\'ecraser puis de gracier R\=ava\d na\index{gnl}{Ravana@R\=ava\d na} (st. 8), de ne pas \^etre vu par Vi\d s\d nu\index{gnl}{Visnu@Vi\d s\d nu} et Brahm\=a\index{gnl}{Brahma@Brahm\=a} (st. 9) et sur les raisons pour lesquelles les ja\"in\index{gnl}{jain@ja\"in}s et boudhistes le diffament (st. 10).}, et ce, selon une phras\'eologie identique. Parce que la premi\`ere strophe se distingue du reste de la d\'ecade par sa forme et son fond, \textit{contra} \textsc{Rangaswamy} qui voit là un élément biographique ancien, nous suspectons une interpolation\index{gnl}{interpolation}.

Plusieurs hymne\index{gnl}{hymne}s, dont cinq d\'edi\'es \`a \=Alav\=ay\index{gnl}{Maturai!Alavay@\=Alav\=ay}\footnote{\=Alav\=ay et K\=u\d tal sont des noms anciens de Maturai.}, feraient r\'ef\'erence aux p\'erip\'eties de Maturai\index{gnl}{Maturai} décrites dans le \textit{Periyapur\=a\d nam} (III 120.1 et 2, III 51, II 66.11, III 115.6, III 39.1, III 87.11, III 54.11 et III 113.12). \textsc{Rangaswamy} trouve partout des allusions biographiques\index{gnl}{biographie!biographique}.

\noindent Dans l'hymne\index{gnl}{hymne} III 120, en l'honneur d'\=Alav\=ay\index{gnl}{Maturai!Alavay@\=Alav\=ay}, Campantar\index{gnl}{Campantar} chante un ministre\index{gnl}{ministre} incarnant \og la protection de la lign\'ee\fg\ (\textit{kulacci\b rai}) et une grande reine\index{gnl}{reine} \textit{p\=a\d n\d dya}\index{gnl}{pandya@\textit{p\=a\d n\d dya}} (\textit{p\=a\d n\d tim\=at\=evi}) qu'il qualifie de \textit{ma\.nkaiyarkku araci}, \og reine\index{gnl}{reine} des femmes\fg, et de \textit{ma\d tam\=a\b ni}, \og joyau des femmes\fg\footnote{III 120.1ab: \textit{ma\.nkaiyarkku araci --- va\d lavark\=o\b n p\=avai, vari va\d laik kaim ma\d tam\=a\b ni, / pa\.nkayaccelvi, p\=a\d n\d tim\=at\=evi --- pa\d ni ceytu n\=a\d lto\b rum parava}; \og alors que la reine\index{gnl}{reine} des femmes --- la fille du roi\index{gnl}{roi} \textit{va\d lavar} (\textit{c\=o\b la}\index{gnl}{cola@\textit{c\=o\b la}}), belle dame aux mains pourvues de bracelets ray\'es, fortun\'ee du lotus, grande reine\index{gnl}{reine} \textit{p\=a\d n\d dya}\index{gnl}{pandya@\textit{p\=a\d n\d dya}} --- r\'epandait [sa gloire] quotidiennement en effectuant des service\index{gnl}{service}s\fg. III 120.2ab: \textit{ve\b r\b rav\=e a\d tiy\=ar a\d timicai v\=\i \b lum viruppi\b na\b n, ve\d l\d lain\=\i \b ru a\d niyum --- / ko\b r\b rava\b nta\b nakku mantiri \=aya --- kulacci\b rai kul\=avi ni\b n\b ru \=ettum}; \og [\'Siva\index{gnl}{Siva@\'Siva}] que loue r\'ejoui et debout celui qui, pour briser l'ignorance, aime tomber aux pieds des dévot\index{gnl}{devot(e)@dévot(e)}s, ministre\index{gnl}{ministre} du roi\index{gnl}{roi} (\textit{p\=a\d n\d dya}\index{gnl}{pandya@\textit{p\=a\d n\d dya}}) qui porte la cendre\index{gnl}{cendre} blanche, gardien de la lign\'ee\fg.}. Si nous trouvons l'appelation de Kulacci\b rai\index{gnl}{kulaccirai@Kulacci\b rai} d\`es le \textit{Tirutto\d n\d tattokai}\index{gnl}{Tiruttontattokai@\textit{Tirutto\d n\d tattokai}} de Cuntarar\index{gnl}{Cuntarar} (VII 39.4), celle de Ma\.nkaiyarkkaraci\index{gnl}{Mankaiyarkkaraci@Ma\.nkaiyarkkaraci} n'appara\^it ailleurs que dans le \textit{Periyapur\=a\d nam}\index{gnl}{Periyapuranam@\textit{Periyapur\=a\d nam}}. Cette reine\index{gnl}{reine} \textit{p\=a\d n\d dya}\index{gnl}{pandya@\textit{p\=a\d n\d dya}} n'est d\'esign\'ee dans le \textit{Tirutto\d n\d tattokai}\index{gnl}{Tiruttontattokai@\textit{Tirutto\d n\d tattokai}} et les textes attribu\'es \`a Nampi\index{gnl}{Nampi \=A\d n\d t\=ar Nampi} \=A\d n\d t\=ar Nampi que par le terme \textit{m\=a\b ni}, en composition ici avec \textit{ma\d ta}. L'emploi du nom Ma\.nkaiyarkkaraci\index{gnl}{Mankaiyarkkaraci@Ma\.nkaiyarkkaraci} dans ce poème\index{gnl}{poeme@poème} du \textit{T\=ev\=aram}\index{gnl}{Tevaram@\textit{T\=ev\=aram}} ne pourrait-il pas refléter un certain anachronisme? L'utilisation du terme \textit{ko\b r\b rava\b n}\index{gnl}{korravan@\textit{ko\b r\b rava\b n}}, litt\'eralement le \og victorieux\fg, pour d\'esigner le roi\index{gnl}{roi} (\textit{p\=a\d n\d dya}\index{gnl}{pandya@\textit{p\=a\d n\d dya}}) est en effet également surprenant dans cette strophe du \textit{T\=ev\=aram}\index{gnl}{Tevaram@\textit{T\=ev\=aram}}
\footnote{Dans l'\'etat actuel de nos connaissance\index{gnl}{connaissance}s, le terme \textit{ko\b r\b rava\b n}\index{gnl}{korravan@\textit{ko\b r\b rava\b n}} semble devenir une d\'esignation du roi\index{gnl}{roi} \textit{p\=a\d n\d dya}\index{gnl}{pandya@\textit{p\=a\d n\d dya}} uniquement dans les textes contemporains ou post\'erieurs au \textit{Periyapur\=a\d nam}\index{gnl}{Periyapuranam@\textit{Periyapur\=a\d nam}}. En effet, un survol des occurrences de ce nom dans les textes du \textit{pu\b ram}\index{gnl}{puram@\textit{pu\b ram}}, des \textit{meykk\=\i rtti}\index{gnl}{meykkirtti@\textit{meykk\=\i rtti}} et du \textit{Cilappatik\=aram}\index{gnl}{Cilappatikaram@\textit{Cilappatik\=aram}} nous conduit \`a supposer que \textit{ko\b r\b rava\b n}\index{gnl}{korravan@\textit{ko\b r\b rava\b n}} d\'esigne un roi\index{gnl}{roi} mais pas sp\'ecifiquement un roi\index{gnl}{roi} \textit{p\=a\d n\d dya}\index{gnl}{pandya@\textit{p\=a\d n\d dya}}. Ensuite, dans le \textit{T\=ev\=aram}\index{gnl}{Tevaram@\textit{T\=ev\=aram}} ce terme est appliqu\'e, presque exclusivement, \`a \'Siva\index{gnl}{Siva@\'Siva} victorieux dans ses exploits. Puis, dans les textes attribu\'es \`a Nampi\index{gnl}{Nampi \=A\d n\d t\=ar Nampi} \=A\d n\d t\=ar Nampi il d\'esigne Campantar\index{gnl}{Campantar} en tant que souverain de la ville de C\=\i k\=a\b li\index{gnl}{Cikali@C\=\i k\=a\b li}. Et enfin, dans le \textit{Periyapur\=a\d nam}\index{gnl}{Periyapuranam@\textit{Periyapur\=a\d nam}} il est synonyme de roi\index{gnl}{roi}, renvoie aussi \`a \'Siva\index{gnl}{Siva@\'Siva} et ne signifie le roi\index{gnl}{roi} \textit{p\=a\d n\d dya}\index{gnl}{pandya@\textit{p\=a\d n\d dya}} que dans l'\'episode de Maturai\index{gnl}{Maturai} du \textit{Tiru\~n\=a\b nacampantapur\=a\d nam}.}.
Parmi les huit occurrences de ce terme relev\'ees dans le corpus\index{gnl}{corpus}, six (I 117.11, V 63.10, VI 69.8, VI 76.9, VII 19.1 et VII 61.11) renvoient tr\`es clairement \`a \'Siva\index{gnl}{Siva@\'Siva} victorieux sur les trois citadelles, sur Dak\d sa, etc. Les deux restantes (III 87.11 et 120.2) se trouvent dans des strophes contenant des allusions dites autobiographique\index{gnl}{autobiographique}s de Campantar\index{gnl}{Campantar} associ\'ees \`a l'\'episode de Maturai\index{gnl}{Maturai} \`a cause du \textit{Periyapur\=a\d nam}\index{gnl}{Periyapuranam@\textit{Periyapur\=a\d nam}}. Dans III 87.11, les strophes inscrites sur \^oles sont jet\'ees dans les flamme\index{gnl}{flamme}s devant \textit{ko\b r\b rava\b n}\index{gnl}{korravan@\textit{ko\b r\b rava\b n}}. Compte tenu du refrain des quatrains pr\'ec\'edents --- \og les noms du [\'Siva\index{gnl}{Siva@\'Siva}] de Na\d l\d l\=a\b ru\index{gnl}{Nallaru@Na\d l\d l\=a\b ru} \dots, m\^eme plac\'es dans le feu\index{gnl}{feu}, sont sans d\'efaut et vrais\fg\ (\textit{na\d l\d l\=a\b rartam n\=amam\=e \dots\ eriyi\b nil i\d til, ivai pa\b lutu ilai; meymmaiy\=e!}) --- ce \textit{ko\b r\b rava\b n}\index{gnl}{korravan@\textit{ko\b r\b rava\b n}} peut aussi tr\`es bien \^etre \'Siva\index{gnl}{Siva@\'Siva}. Dans III 120.2, celui qui est le \og gardien de la lign\'ee\fg\ occupe le poste de ministre\index{gnl}{ministre} du \textit{ko\b r\b rava\b n}\index{gnl}{korravan@\textit{ko\b r\b rava\b n}} qui est, sans ambigu\"it\'e, le roi\index{gnl}{roi} \textit{p\=a\d n\d dya}\index{gnl}{pandya@\textit{p\=a\d n\d dya}} puisqu'il y est aussi question de la reine\index{gnl}{reine} de la m\^eme dynastie. Ainsi, consid\'erant notre d\'eveloppement sur le nom de Ma\.nkaiyarkkaraci\index{gnl}{Mankaiyarkkaraci@Ma\.nkaiyarkkaraci} et notre analyse du terme \textit{ko\b r\b rava\b n}\index{gnl}{korravan@\textit{ko\b r\b rava\b n}}, encore \`a l'\'etat d'\'ebauche, nous posons l'hypoth\`ese, sous toutes r\'eserves, que la strophe III 120.2 contenant le terme \textit{ko\b r\b rava\b n}\index{gnl}{korravan@\textit{ko\b r\b rava\b n}} signifiant le roi\index{gnl}{roi} \textit{p\=a\d n\d dya}\index{gnl}{pandya@\textit{p\=a\d n\d dya}}, en dissonance avec le reste du \textit{T\=ev\=aram}\index{gnl}{Tevaram@\textit{T\=ev\=aram}}, est un ajout tardif influenc\'e par le \textit{Periyapur\=a\d nam}\index{gnl}{Periyapuranam@\textit{Periyapur\=a\d nam}}.

\noindent La premi\`ere strophe de l'hymne\index{gnl}{hymne} III 39, c\'el\'ebrant ce m\^eme lieu, pr\'esente une adresse présentée \`a la reine\index{gnl}{reine} par le poète\index{gnl}{poete@poète}:

\scriptsize
\begin{verse}
\textit{m\=a\b ni\b n n\=er vi\b li m\=atar\=ay! va\b lutikku m\=a perunt\=evi! k\=e\d l:\\
``p\=al nal v\=ay oru p\=ala\b n \=\i \.nku iva\b n'' e\b n\b ru n\=\i\ parivu eyti\d t\=el!\\
\=a\b naim\=amalai \=ati \=aya i\d ta\.nka\d lil pala allal c\=er\\
\=\i \b narka\d tku e\d liy\=e\b n al\=e\b n --- tiru \=alav\=ay ara\b n ni\b rkav\=e.} (III 39.1)\\
\end{verse}

\normalsize
\begin{verse}
O femme aux yeux de biches!\\
Tr\`es grande reine\index{gnl}{reine} du Va\b luti\index{gnl}{Valuti@Va\b luti}\footnote{Terme d\'esignant le royaume \textit{p\=a\d n\d dya}\index{gnl}{pandya@\textit{p\=a\d n\d dya}}.}! Ecoute,\\
Ne prends pas piti\'e en disant\\
``Il est un jeune \`a la belle bouche de lait\index{gnl}{lait}'',\\
En pr\'esence du Hara\index{gnl}{Hara} d'\=Alav\=ay\index{gnl}{Maturai!Alavay@\=Alav\=ay}, je ne suis pas faible\\
Pour les inf\^ames dot\'es de plusieurs maux (qui vivent)\\
Dans les endroits \`a commencer par la grande montagne \=A\b nai! (III 39.1)\\
\end{verse}

\noindent et la derni\`ere pr\'ecise que cette \og d\'ecade fut dite \dots\ par \~N\=a\b nacampanta\b n\index{gnl}{Campantar!N\=a\b nacampanta\b n@\~N\=a\b nacampanta\b n} \dots\ en face de celui du Sud \`a la couronne \'eclatante\fg\ (\textit{tu\d la\.nkum mu\d tit te\b n\b na\b n mu\b n, ivai \dots\ \~n\=a\b nacampanta\b n \dots\ uraiceyta pattum}). L'image de l'enfant\index{gnl}{enfant} donn\'ee en st. 1 ne concorde pas avec celle du poète\index{gnl}{poete@poète} signant l'envoi\index{gnl}{envoi}, \og \~N\=a\b nacampanta\b n\index{gnl}{Campantar!N\=a\b nacampanta\b n@\~N\=a\b nacampanta\b n}, roi\index{gnl}{roi} de Pukali\index{gnl}{Pukali} et seigneur tamoul\fg\ (III 39.11: \textit{c\=\i rp pukalikku ma\b n --- tami\b l n\=ata\b n, \~n\=a\b nacampanta\b n}). Notons aussi que l'int\'egralit\'e du poème\index{gnl}{poeme@poème} est vou\'e \`a une ardente critique des ja\"in\index{gnl}{jain@ja\"in}s\footnote{\`A chaque strophe, les ja\"in\index{gnl}{jain@ja\"in}s sont attaqu\'es: sur leur sanskrit et prakrit, leur mani\`ere de manger debout (st. 2), leur enseignement, leur nudit\'e (st. 3), leurs noms, leur habitude d'errer comme des singes femelles, et leur ignorance du tamoul pur (st. 4). Ce sont des voleurs, sans piti\'e, qui ont introduit le m\`etre du perroquet et de la souris (st. 5). Leurs noms (st. 6), leur pouvoir myst\'erieux (st. 7), leurs attributs comme la cruche, la plume de paon, la natte, et leur fausse p\'enitence (st. 8) sont vivement critiqu\'es. Ils s'arrachent les cheveux, se recouvrent d'une poudre, ont la bouche boueuse (st. 9), se mettent \`a dos les lettr\'es (st. 10) et sont vils (st. 11).}.

\noindent Dans un autre poème\index{gnl}{poeme@poème}, \`a la gloire de ce site, III 51, Campantar\index{gnl}{Campantar} implore, \`a chaque strophe, la gr\^ace de \'Siva\index{gnl}{Siva@\'Siva} (\textit{aru\d lcey e\d nai}, \og fais moi gr\^ace\fg) et envoie \og un feu\index{gnl}{feu} allum\'e par les ja\"in\index{gnl}{jain@ja\"in}s\fg\ (\textit{ama\d nar ko\d luvum cu\d tar} ou \textit{ama\d nkaiyar i\d tum cu\d tar}, st. 7) sur \og celui du Sud\fg\ (\textit{te\b n\b na\b n}), le roi\index{gnl}{roi} \textit{p\=a\d n\d dya}\index{gnl}{pandya@\textit{p\=a\d n\d dya}}. Ces refrains rappellent l'\'episode du \textit{Periyapur\=a\d nam}\index{gnl}{Periyapuranam@\textit{Periyapur\=a\d nam}} (st. 2601) o\`u les ja\"in\index{gnl}{jain@ja\"in}s, avec l'accord du roi\index{gnl}{roi}, mettent le feu\index{gnl}{feu} au monast\`ere\index{gnl}{monastère} de Campantar\index{gnl}{Campantar}. Ce dernier en chantant parvient \`a transf\'erer le feu\index{gnl}{feu} dans le corps du roi\index{gnl}{roi} \textit{p\=a\d n\d dya}\index{gnl}{pandya@\textit{p\=a\d n\d dya}}.

\noindent L'envoi\index{gnl}{envoi} de l'hymne\index{gnl}{hymne} des cendre\index{gnl}{cendre}s, \textit{tirun\=\i \b r\b ruppatikam}, II 66, d\'edi\'e au m\^eme site, fait allusion \`a la guérison\index{gnl}{guerison@guérison} du roi\index{gnl}{roi} \textit{p\=a\d n\d dya}\index{gnl}{pandya@\textit{p\=a\d n\d dya}} d\'evor\'e par un feu\index{gnl}{feu} interne. Campantar\index{gnl}{Campantar} pr\'etend que les \og dix strophes sont offertes pour d\'etruire la fièvre\index{gnl}{fievre@fièvre} ressentie par le corps de celui du Sud\fg\ (\textit{te\b n\b na\b n u\d tal u\b r\b ra t\=\i ppi\d ni \=ayi\b na t\=\i rac c\=a\b r\b riya p\=a\d talka\d lpattum}). Dans le \textit{Periyapur\=a\d nam}\index{gnl}{Periyapuranam@\textit{Periyapur\=a\d nam}} (st. 2662) Campantar\index{gnl}{Campantar} chante cet hymne\index{gnl}{hymne} pour gu\'erir la fièvre\index{gnl}{fievre@fièvre} du roi\index{gnl}{roi} \textit{p\=a\d n\d dya}\index{gnl}{pandya@\textit{p\=a\d n\d dya}}. Mis \`a part l'envoi\index{gnl}{envoi} qui nous renvoie \`a ce miracle\index{gnl}{miracle}, le reste du poème\index{gnl}{poeme@poème} est une c\'el\'ebration des bienfait\index{gnl}{bienfait}s des cendre\index{gnl}{cendre}s.

\noindent Dans un autre poème\index{gnl}{poeme@poème} construit selon le jeu litt\'eraire \textit{iyamakam}\index{gnl}{iyamakam@\textit{iyamakam}}, toujours en l'honneur d'\=Alav\=ay\index{gnl}{Maturai!Alavay@\=Alav\=ay}, nous lisons que \'Siva\index{gnl}{Siva@\'Siva} est \og l'ornement des serviteurs qui a donn\'e délicatement l'ornement \`a l'illustre reine\index{gnl}{reine} de celui du Sud\fg\ (III 115, 6c: \textit{mikka te\b n\b nava\b nt\=evikku a\d niyaiy\=e mella nalkiya to\d n\d tarkku a\d niyaiy\=e}). Ce serait une allusion au fait que \'Siva\index{gnl}{Siva@\'Siva} a rendu le bijou marital \`a la reine\index{gnl}{reine} \textit{p\=a\d n\d dya}\index{gnl}{pandya@\textit{p\=a\d n\d dya}}, \textit{i.e.} qu'il n'a pas tu\'e son \'epoux.

\noindent Ajoutons que parmi les dix hymne\index{gnl}{hymne}s c\'el\'ebrant le site d'\=Alav\=ay\index{gnl}{Maturai!Alavay@\=Alav\=ay} dans les trois premiers \textit{Tirumu\b rai}\index{gnl}{Tirumurai@\textit{Tirumu\b rai}} cinq font r\'ef\'erence \`a la biographie\index{gnl}{biographie} de Campantar\index{gnl}{Campantar} (voir \textit{supra}) et cinq sont compos\'es selon des procédé\index{gnl}{procédé littéraire}s stylistiques (I 7 et 94; III 52, 108 et 115). Par ailleurs, quatre d'entre eux se distinguent du lot par leur singularit\'e. En effet, un poème\index{gnl}{poeme@poème} est enti\`erement vou\'e \`a la glorification des cendre\index{gnl}{cendre}s (II 66) et trois autres sont exclusivement consacr\'es \`a la vitup\'eration contre les ja\"in\index{gnl}{jain@ja\"in}s (III 39, 47 et 108). Autant de particularit\'es dans une poign\'ee d'hymne\index{gnl}{hymne}s d\'edi\'es \`a un seul site nous recommandent de les lire avec la plus grande prudence.

\noindent L'épreuve\index{gnl}{epreuve@épreuve} du feu\index{gnl}{feu}, consistant \`a faire sortir du feu\index{gnl}{feu} les \^oles indemnes, serait mentionn\'ee dans un hymne\index{gnl}{hymne} c\'el\'ebrant Na\d l\d l\=a\b ru\index{gnl}{Nallaru@Na\d l\d l\=a\b ru}, III 87. Le refrain, en fin de strophe, r\'ep\`ete que \og les noms du [\'Siva\index{gnl}{Siva@\'Siva}] de Na\d l\d l\=a\b ru\index{gnl}{Nallaru@Na\d l\d l\=a\b ru} \dots, m\^eme plac\'es dans le feu\index{gnl}{feu}, sont sans d\'efaut et vrais\fg\ (\textit{na\d l\d l\=a\b rartam n\=amam\=e \dots\ eriyi\b nil i\d til, ivai pa\b lutu ilai; meymmaiy\=e!}). Ces r\'ep\'etitions soulignent, simplement, le caract\`ere permanent et imp\'erissable de \'Siva\index{gnl}{Siva@\'Siva} et de ses attributs. Seul l'envoi\index{gnl}{envoi} vient pr\'eciser que ces strophes ont \'et\'e \og jet\'ees dans les flamme\index{gnl}{flamme}s devant \textit{ko\b r\b rava\b n}\index{gnl}{korravan@\textit{ko\b r\b rava\b n}}\fg\footnote{Cf. note 78 p. 59} (\textit{ko\b r\b rava\b n etir i\d tai eriyi\b nil i\d ta}) et replace, ce faisant, l'hymne\index{gnl}{hymne} dans une narration hagiographique\index{gnl}{hagiographie!hagiographique}. Par ailleurs l'hymne\index{gnl}{hymne} qui aurait \'et\'e jet\'e dans le feu\index{gnl}{feu} devant les ja\"in\index{gnl}{jain@ja\"in}s et le roi\index{gnl}{roi} \textit{p\=a\d n\d dya}\index{gnl}{pandya@\textit{p\=a\d n\d dya}} commence par \textit{p\=okam \=artta po\b n mulaiy\=a\d l} d'apr\`es le \textit{Periyapur\=a\d nam}\index{gnl}{Periyapuranam@\textit{Periyapur\=a\d nam}} (st. 2680). Il s'agit du poème\index{gnl}{poeme@poème} I 49 dans le corpus\index{gnl}{corpus} actuel du \textit{T\=ev\=aram}\index{gnl}{Tevaram@\textit{T\=ev\=aram}} qui a acquis par la suite le surnom de \textit{paccai patikam} (voir chapitre 6).

\noindent Quant \`a l'épreuve\index{gnl}{epreuve@épreuve} de l'eau\index{gnl}{eau}, deux strophes appartenant \`a des poème\index{gnl}{poeme@poème}s particuliers relatent ce miracle\index{gnl}{miracle} qui consiste à faire remonter les \^oles \`a contre-courant. III 54 n'est pas associ\'e \`a un site sp\'ecifique, il est dit g\'en\'eral, \textit{potu}. Titr\'e \textit{tirupp\=acuram}, \og chant sacr\'e\fg, il comporte douze\index{gnl}{douze} quatrains dont le onzi\`eme d\'ecrit le prodige\index{gnl}{prodige} \`a Maturai\index{gnl}{Maturai}:

\scriptsize
\begin{verse}
\textit{a\b r\b ru a\b n\b ri am ta\d n maturait tokai \=akki\b n\=a\b num,\\
te\b r\b ru e\b n\b ra teyvam te\d liy\=ar karaikku \=olai te\d nn\=\i rp\\
pa\b r\b ru i\b n\b rip p\=a\.nku etirvi\b n \=uravum, pa\d npu n\=okkil,\\
pe\b r\b ro\b n\b ru uyartta perum\=a\b n perum\=a\b num a\b n\b r\=e!} (III 54.11)\\
\end{verse}

\normalsize
\begin{verse}
Mais encore, il cr\'ea l'assemblée\index{gnl}{assemblée} de la belle et fra\^iche Maturai\index{gnl}{Maturai},\\
Ceux qui ne clarifient pas qu'il est le dieu\index{gnl}{dieu} dense,\\
Consid\'erant le fait que les \^oles sans attache flottaient\index{gnl}{flotter}\\
\`A contre-courant dans l'eau\index{gnl}{eau} limpide jusqu'\`a la rive,\\
N'est-il pas un grand dieu\index{gnl}{dieu}\\
Le seigneur mont\'e sur un taureau\index{gnl}{taureau}? (III 54.11)\\
\end{verse}

\noindent Le \textit{tirukka\d taikk\=appu}\index{gnl}{tirukkataikkappu@\textit{tirukka\d taikk\=appu}} de l'hymne\index{gnl}{hymne} III 113, en l'honneur des douze\index{gnl}{douze} noms de C\=\i k\=a\b li\index{gnl}{Cikali@C\=\i k\=a\b li} et construit selon l'\textit{iyamakam}\index{gnl}{iyamakam@\textit{iyamakam}}, narre aussi ce miracle\index{gnl}{miracle}:

\scriptsize
\begin{verse}
\textit{paru matil maturai ma\b n avai etir\=e patikam atu e\b lutu ilai avai etir\=e\\
varu nati i\d tai micai varu kara\b n\=e! vacaiyo\d tum alar ke\d ta aruku ara\b n\=e!} (III 113.12ab)\\
\end{verse}

\normalsize
\begin{verse}
En face d'eux et du roi\index{gnl}{roi} de Maturai\index{gnl}{Maturai} aux grandes fortifications,\\
\^O celui \`a l'acte de faire venir \`a contre[-courant]\\
Les feuilles \'ecrites de d\'ecades, dans le fleuve qui coule!\\
\^O celui qui d\'etruit les jains pour an\'eantir bl\^ame et bassesse! (III 113.12ab)\\
\end{verse}

\noindent La nature de ces deux hymne\index{gnl}{hymne}s, la position des strophes qui nous occupent (l'avant-derni\`ere strophe consacr\'ee habituellement \`a la vitup\'eration des hérétique\index{gnl}{heretique@hérétique}s pour III 54 et l'envoi\index{gnl}{envoi} pour III 113), ainsi que la r\'ef\'erence \`a Maturai\index{gnl}{Maturai}, nomm\'ee en g\'en\'eral \=Alav\=ay\index{gnl}{Maturai!Alavay@\=Alav\=ay} dans les hymne\index{gnl}{hymne}s attribu\'es \`a Campantar\index{gnl}{Campantar}, nous conduisent \`a douter de leur authenticit\'e.

\noindent Dans l'envoi\index{gnl}{envoi} de III 32, \`a la gloire du site d'\=E\d takam\index{gnl}{Etakam@\=E\d takam}, le poète\index{gnl}{poete@poète} stipule que les \^oles, naviguant dans le fleuve Vaikai\index{gnl}{Vaikai}, se sont arr\^et\'es en ce site: \textit{vaikain\=\i r \=e\d tu ce\b n\b ru a\d naitarum \=e\d takattu oruva\b nai}, \og l'unique d'\=E\d takam\index{gnl}{Etakam@\=E\d takam} que viennent embrasser les \^oles des flots\index{gnl}{flots} de Vaikai\index{gnl}{Vaikai}\fg. Or, le terme \textit{\=e\d tu} peut aussi signifier p\'etale ou fleur, comme dans I 1.1, % III 17.5, 95.11, 75.5, 126.2, 104.9, 59.10, etc.
et renvoyer \`a un des \'el\'ements que le fleuve charrie naturellement.

L'hymne\index{gnl}{hymne} III 6, construit selon le procédé\index{gnl}{procédé littéraire} m\'etrique \textit{\=\i ra\d tim\=elvaippu}, c\'el\`ebre le site de Ko\d l\d lamp\=ut\=ur\index{gnl}{Kollamputur@Ko\d l\d lamp\=ut\=ur}. La sixi\`eme strophe ferait allusion \`a la barque\index{gnl}{barque} qui a conduit Campantar\index{gnl}{Campantar} sur la rive oppos\'ee: \textit{\=o\d tam vantu a\d naiyum ko\d l\d lamp\=ut\=ur}; \og Ko\d l\d lamp\=ut\=ur\index{gnl}{Kollamputur@Ko\d l\d lamp\=ut\=ur} que viennent joindre les barque\index{gnl}{barque}s\fg. Cependant, il n'est pas \'etonnant qu'une barque\index{gnl}{barque} et un cours d'eau\index{gnl}{eau} (st. 7, selon une phras\'eologie identique, \textit{\=a\b ru vantu a\d naiyum ko\d l\d lamp\=ut\=ur}) soient d\'ecrits dans un poème\index{gnl}{poeme@poème} louant un site plac\'e sur le littoral d'un bras de la K\=av\=eri\index{gnl}{Kaveri@K\=av\=eri}. Par ailleurs, la barque\index{gnl}{barque} est un \'el\'ement souvent associ\'e au fleuve comme le bateau l'est \`a la mer\index{gnl}{mer}\footnote{Le clich\'e litt\'eraire de la mer\index{gnl}{mer} pourvue de bateaux est fr\'equent: \textit{va\.nkam ka\d tal} (I 66.1; II 29.6, 37.2, etc.), \textit{kalam ka\d tal} (I 30.6, 34.11, 66.5, 84.8; II 17.10, 24. 11, 37.8, etc.).}. Elle se trouve par exemple sur la Ga\.ng\=a\index{gnl}{Ganga@Ga\.ng\=a} que \'Siva\index{gnl}{Siva@\'Siva} porte dans sa chevelure (II 55.4: \textit{o\d tam c\=u\b l ka\.nkaiyum ucci vaitt\=\i r}; \og tu as plac\'e sur le sommet [de la t\^ete] la Ga\.ng\=a\index{gnl}{Ganga@Ga\.ng\=a} o\`u tourne la barque\index{gnl}{barque}\fg). Ainsi, nous ne voyons aucun \'el\'ement biographique\index{gnl}{biographie!biographique} dans cet hymne\index{gnl}{hymne}, d'ailleurs, que \textsc{Rangaswamy} ne mentionne pas.

Le \textit{tirukka\d taikk\=appu}\index{gnl}{tirukkataikkappu@\textit{tirukka\d taikk\=appu}} de l'hymne\index{gnl}{hymne} I 54 contiendrait une r\'ef\'erence au miracle\index{gnl}{miracle} des palmier\index{gnl}{palmier}s mais nous lisons simplement: \og \=Ott\=ur aux grappes issues de jeunes palmier\index{gnl}{palmier}s m\^ales\fg\ (\textit{kurumpai \=a\d npa\b nai \=\i \b n kulai \=ott\=ur}). Les palmier\index{gnl}{palmier}s ont effectivement un genre et seules les femelles donnent des fruits. Cependant, le palmier\index{gnl}{palmier} m\^ale est aussi pourvu de petites grappes, sans fleur (information communiqu\'ee par \textsc{A.~Markkanthu}). Comme \textsc{Rangaswamy}, nous ne relevons aucune allusion biographique\index{gnl}{biographie!biographique} dans ce quatrain.

Certains passages ont \'et\'e utilis\'es pour d\'emontrer que Campantar\index{gnl}{Campantar} \'etait contemporain de quelques \textit{n\=aya\b nm\=ar}\index{gnl}{nayanmar@\textit{n\=aya\b nm\=ar}}. Nous avons déjà \'evoqu\'e le couple royal \textit{p\=a\d n\d dya}\index{gnl}{pandya@\textit{p\=a\d n\d dya}} et leur ministre\index{gnl}{ministre}. Dans l'envoi\index{gnl}{envoi} de l'hymne\index{gnl}{hymne} III 58, il est pr\'ecis\'e que la d\'ecade c\'el\`ebre la ville de C\=attama\.nkai\index{gnl}{Cattamankai@C\=attama\.nkai} d'o\`u est originaire le serviteur N\=\i lanakka\b n\index{gnl}{Nilanakkan@N\=\i lanakka\b n}. Ensuite, nous avons vu que le \textit{ci\b rutto\d n\d tar}\index{gnl}{ciruttontar@\textit{ci\b rutto\d n\d tar}} de l'hymne\index{gnl}{hymne} III 63 n'est pas un individu particulier mais la repr\'esentation du dévot\index{gnl}{devot(e)@dévot(e)} humble. Il en est de m\^eme pour les autres occurrences (I 45.5; I 61.10; I 99.5; I 103.6 et III 46.3). Enfin, le nom Muruka\b n\index{gnl}{Murukan@Muruka\b n}, mentionn\'e \`a la troisi\`eme strophe du poème\index{gnl}{poeme@poème} II 92, sans allusion biographique\index{gnl}{biographie!biographique}, peut parfaitement renvoyer \`a la divinit\'e Skanda\index{gnl}{Skanda}.

Ainsi, nous constatons que la plupart des r\'ef\'erences dites \og autobiographique\index{gnl}{autobiographique}s\fg\ donn\'ees se trouvent dans les envois\index{gnl}{envoi} (I 54; II 66, 84; III 32, 87, 113), dans des poème\index{gnl}{poeme@poème}s \`a la gloire de C\=\i k\=a\b li\index{gnl}{Cikali@C\=\i k\=a\b li} (III 24 et 113) et dans des hymne\index{gnl}{hymne}s dont la composition est bien particuli\`ere (I 92; III 6, 54, 58, 113, 115). Nous gardons une certaine r\'eserve quant \`a l'appartenance de ces passages au corpus\index{gnl}{corpus} \og initial\fg\ du \textit{T\=ev\=aram}. Par ailleurs, dans de nombreux cas, nous manquons s\'erieusement d'\'el\'ements pour \'etablir un lien direct entre le fait racont\'e et Campantar\index{gnl}{Campantar} (I 54, 92; II 66, 92; III 6, 32, 54, 87, 113, 115, 120), ce qui laisse supposer qu'il peut aussi s'agir de prodige\index{gnl}{prodige}s locaux\index{gnl}{local} li\'es au site.

Examinons le t\'emoignage des textes attribu\'es aux deux autres \textit{m\=uvar}\index{gnl}{muvar@\textit{m\=uvar}}.


\subsection{Poète chez Appar et Cuntarar}

Selon \textsc{Rangaswamy} (*1990 [1958]: 977-984), trois strophes attribu\'ees \`a Appar\index{gnl}{Appar} feraient allusion \`a Campantar\index{gnl}{Campantar}: IV 56.1, VI 58.1 et V 50.8. Le nom de Campantar\index{gnl}{Campantar} n'y est cependant jamais donn\'e. Seul un quatrain se r\'ef\`ere clairement \`a un \'el\'ement identitaire du poète\index{gnl}{poete@poète}, son origine\index{gnl}{origine} g\'eographique, et \`a un fait pr\'ecis connu de la légende\index{gnl}{legende@légende} \'etablie dans le \textit{Periyapur\=a\d nam}, celui du don\index{gnl}{don} des pièce\index{gnl}{piece@pièce}s d'or: \og \^O Celui d'\=Ava\d tutu\b rai\index{gnl}{Avatuturai@\=Ava\d tutu\b rai}, comme Celui qui donne mille belles pièce\index{gnl}{piece@pièce}s d'or \`a l'habitant de Ka\b lumalam\index{gnl}{Kalumalam@Ka\b lumalam}\fg\ (IV 56.1: \textit{ka\b lumala \=urarkku am po\b n \=ayiram ko\d tupparp\=olum --- \=ava\d tutu\b raiya\b n\=ar\=e}). Ailleurs, le prodige\index{gnl}{prodige} de la fermeture des portes serait mentionn\'e non sans ambigu\"it\'e:

\scriptsize
\begin{verse}
\textit{ti\b rakkap p\=a\d tiya e\b n\b ni\b num centami\b l\\
u\b raippup p\=a\d ti a\d taippitt\=ar un ni\b n\b r\=ar;\\
ma\b raikka vallar\=o, tamait tiru v\=aym\=urp\\
pi\b raik ko\d l ce\~nca\d taiy\=ar? ivar pittar\=e!} (V 50.8)\\
\end{verse}

\normalsize
\begin{verse}
Mieux que moi\\
Qui ai chant\'e pour ouvrir,\\
Celui qui \'etait l\`a,\\
Chantant en pur tamoul,\\
A fait fermer;\\
Est-il capable de se cacher,\\
Celui aux m\`eches rouges pourvues du croissant\\
De Tiruv\=aym\=ur\index{gnl}{Tiruv\=aym\=ur}?\\
Il est fou! (V 50.8)\\
\end{verse}


\noindent Si \textit{ni\b n\b r\=ar} \'etait un verbe conjugu\'e appartenant \`a une phrase ind\'ependante, il pourrait renvoyer \`a Campantar\index{gnl}{Campantar}. Mais compte tenu de l'unit\'e, g\'en\'eralement, syntaxique d'une strophe, nous pouvons aussi l'analyser comme un verbe appellatif subordonn\'e \`a la principale \textit{ivar pittar}. Ainsi, c'est \'Siva\index{gnl}{Siva@\'Siva} qui serait d\'esign\'e par \textit{pittar}, \textit{ca\d taiy\=ar}, \textit{vallar}, \textit{ni\b n\b r\=ar} et \textit{a\d taippitt\=ar} et qui donc ferait fermer les portes. Enfin, la strophe inaugurale de VI 58 \'evoquerait le talent po\'etique de Campantar\index{gnl}{Campantar}. Cependant, aucune donn\'ee ne vient confirmer cette interpr\'etation, nous y lisons simplement: \og celui (ou ceux) aux mots dot\'es de m\'elodie\fg\ (\textit{pa\d n malinta mo\b liyavar}). Rappelons, par ailleurs, que \textit{mo\b liy\=ar} est souvent un nom appellatif renvoyant \`a une femme (I 72.7; II 51.11, 81.5, etc.), l'hypothèse nous paraît peu fondée.

Chez Cuntarar\index{gnl}{Cuntarar}, les r\'ef\'erences explicites \`a Campantar\index{gnl}{Campantar} sont plus nombreuses. Campantar\index{gnl}{Campantar} appara\^it principalement comme un bon poète\index{gnl}{poete@poète} tamoul. Il figure dans les listes des serviteurs de \'Siva\index{gnl}{Siva@\'Siva} comme le \textit{Tirutto\d n\d tattokai}\index{gnl}{Tiruttontattokai@\textit{Tirutto\d n\d tattokai}} (VII 39.5: \textit{ko\b n\b raiy\=a\b n a\d ti al\=al p\=e\d n\=a empir\=a\b n campanta\b n}, \og mon seigneur Campanta\b n\index{gnl}{Campantar!Campanta\b n} qui ne chante que les pieds de celui aux fleurs de cassier\fg) et la quatri\`eme strophe de l'hymne\index{gnl}{hymne} VII 55 qui \'enum\`ere divers dévot\index{gnl}{devot(e)@dévot(e)}s dont le premier est \og \~N\=a\b nacampanta\b n\index{gnl}{Campantar!N\=a\b nacampanta\b n@\~N\=a\b nacampanta\b n} dou\'e en bon tamoul\fg\ (\textit{nal tami\b l \~n\=a\b nacampanta\b n\index{gnl}{Campantar!N\=a\b nacampanta\b n@\~N\=a\b nacampanta\b n}}). Il est coupl\'e \`a N\=avukkaracar\index{gnl}{Appar!Navukkaracar@N\=avukkaracar} (\og roi\index{gnl}{roi} de la langue\fg), \textit{i.e.} Appar\index{gnl}{Appar}, dans le poème\index{gnl}{poeme@poème} VII 67.5 (\textit{nal icai \~n\=a\b nacampanta\b num n\=avukku aracarum p\=a\d tiya nal tami\b l m\=alai}, \og les belles guirlandes tamoules chant\'ees par le bon musicien \~N\=a\b nacampanta\b n\index{gnl}{Campantar!N\=a\b nacampanta\b n@\~N\=a\b nacampanta\b n} et le roi\index{gnl}{roi} de la langue\fg) et dans l'envoi\index{gnl}{envoi} de l'hymne\index{gnl}{hymne} VII 78 o\`u Cuntarar\index{gnl}{Cuntarar} se pr\'esente comme \og le serviteur du roi\index{gnl}{roi} de la langue, de \~N\=a\b nacampanta\b n\index{gnl}{Campantar!N\=a\b nacampanta\b n@\~N\=a\b nacampanta\b n} le Tamoul et de tous les dévot\index{gnl}{devot(e)@dévot(e)}s de \'Siva\index{gnl}{Siva@\'Siva}\fg\ (\textit{n\=avi\b n micai araiya(\b n)\b n\=o\d tu, tami\b l \~n\=a\b nacampanta\b n, y\=avar civa\b n a\d tiy\=arka\d lukku, a\d tiy\=a\b n}). Ensuite, Campantar\index{gnl}{Campantar} est mentionn\'e seul dans deux hymne\index{gnl}{hymne}s. Le premier d\'edi\'e \`a K\=olakk\=a\index{gnl}{Kolakka@K\=olakk\=a} ferait allusion au don\index{gnl}{don} de cymbale\index{gnl}{cymbale}s\footnote{VII 62.8: \textit{n\=a\d lum i\b n icaiy\=al tami\b l parappum \~n\=a\b nacampanta\b nukku ulakavar mu\b n t\=a\d lam \=\i ntu, ava\b n p\=a\d talukku ira\.nkum ta\b nmaiy\=a\d la\b nai}; \og celui qui a donn\'e des cymbale\index{gnl}{cymbale}s devant les habitants du monde \`a \~N\=a\b nacampanta\b n\index{gnl}{Campantar!N\=a\b nacampanta\b n@\~N\=a\b nacampanta\b n}, qui r\'epand quotidiennement le tamoul par une musique\index{gnl}{musique} plaisante, a la nature de s'\'emouvoir \`a ses chants\index{gnl}{chant}\fg. Signalons toutefois que le terme \textit{t\=a\d lam} signifie aussi bien le rythme ou le battement que l'instrument, les cymbale\index{gnl}{cymbale}s, qui sert \`a le marquer.}; le second, en l'honneur de Na\b nipa\d l\d li\index{gnl}{Nanipalli@Na\b nipa\d l\d li}, rappelle le don\index{gnl}{don} de la connaissance\index{gnl}{connaissance}\footnote{VII 97.9: \textit{\=u\b nam il k\=a\b lita\b n\b nu\d l(\d l) uyar \~n\=a\b nacampanta\b rku a\b n\b ru \~n\=a\b nam aru\d l purint\=a\b n}; \og celui qui fit gr\^ace de la connaissance\index{gnl}{connaissance}, jadis, au grand \~N\=a\b nacampanta\b n\index{gnl}{Campantar!N\=a\b nacampanta\b n@\~N\=a\b nacampanta\b n} dans K\=a\b li\index{gnl}{Kali@K\=a\b li} sans d\'efaut\fg.}. Enfin, une strophe d'un poème\index{gnl}{poeme@poème} c\'el\'ebrant V\=\i \b limi\b lalai\index{gnl}{Vilimilalai@V\=\i \b limi\b lalai} (VII 88.8) ferait r\'ef\'erence au don\index{gnl}{don} de pièce\index{gnl}{piece@pièce}s d'or gr\^ace au chant. Cependant, Campantar\index{gnl}{Campantar} n'est pas \'evoqu\'e dans cette dernière. Nous sugg\'erons que cet \'episode, d\'ecrit chez Campantar\index{gnl}{Campantar} en I 92.1 et apparaissant uniquement dans des hymne\index{gnl}{hymne}s \`a la gloire de V\=\i \b limi\b lalai\index{gnl}{Vilimilalai@V\=\i \b limi\b lalai}, peut aussi traduire une légende\index{gnl}{legende@légende} ou un pouvoir propre au site.

Les t\'emoignages des poème\index{gnl}{poeme@poème}s attribu\'es \`a Appar\index{gnl}{Appar} et à Cuntarar\index{gnl}{Cuntarar} nous confirment donc que Campantar\index{gnl}{Campantar} est un poète\index{gnl}{poete@poète} tamoul originaire de (C\=\i)k\=a\b li. Ils \'evoquent aussi des faits qui sont devenus des miracle\index{gnl}{miracle}s dans l'hagiographie\index{gnl}{hagiographie}: l'obtention de la connaissance\index{gnl}{connaissance} \`a C\=\i k\=a\b li\index{gnl}{Cikali@C\=\i k\=a\b li}, de cymbale\index{gnl}{cymbale}s \`a K\=olakk\=a\index{gnl}{Kolakka@K\=olakk\=a}, de pièce\index{gnl}{piece@pièce}s d'or \`a \=Ava\d tutu\b rai\index{gnl}{Avatuturai@\=Ava\d tutu\b rai} et \`a V\=\i \b limi\b lalai\index{gnl}{Vilimilalai@V\=\i \b limi\b lalai} ainsi que le prodige\index{gnl}{prodige} de Ma\b raik\=a\d tu\index{gnl}{maraikatu@Ma\b raik\=a\d tu}. Soulignons que ces deux poète\index{gnl}{poete@poète}s chantent le site de Campantar\index{gnl}{Campantar} sous les toponymes de Ka\b lumalam\index{gnl}{Kalumalam@Ka\b lumalam} (IV 82, 83 et VII~58) et de T\=o\d nipuram\index{gnl}{Tonipuram@T\=o\d nipuram} (V 45), et qu'ils ne font nullement r\'ef\'erence aux neuf autres noms, ni \`a l'unit\'e des douze\index{gnl}{douze} noms.

\subsection{Le \textit{Tiruv\=acakam} de M\=a\d nikkav\=acakar}

Le \textit{Tiruv\=acakam}\index{gnl}{Tiruvacakam@\textit{Tiruv\=acakam}} et le \textit{Tirukk\=ovaiy\=ar}\index{gnl}{Tirukkovaiyar@\textit{Tirukk\=ovaiy\=ar}} constituent l'\oe uvre attribu\'ee au poète M\=a\d nikkav\=acakar\index{gnl}{Manikkavacakar@M\=a\d nikkav\=acakar}. Ils forment le livre \textsc{viii} du \textit{Tirumu\b rai}\index{gnl}{Tirumurai@\textit{Tirumu\b rai}}. Nous proposons ici un rapide survol descriptif de la forme du \textit{Tiruv\=acakam}\index{gnl}{Tiruvacakam@\textit{Tiruv\=acakam}}\footnote{Les informations sur Campantar\index{gnl}{Campantar} et C\=\i k\=a\b li\index{gnl}{Cikali@C\=\i k\=a\b li} sont quasi-inexistantes chez M\=a\d nikkav\=acakar\index{gnl}{Manikkavacakar@M\=a\d nikkav\=acakar}. Nous pr\'esentons toutefois son \oe uvre car elle suit de pr\`es chronologiquement celle attribu\'ee \`a Campantar\index{gnl}{Campantar}.}.

Le \textit{Tiruv\=acakam}\index{gnl}{Tiruvacakam@\textit{Tiruv\=acakam}}, \og Paroles sacr\'ees\fg, constitue un ensemble de cinquante et un hymne\index{gnl}{hymne}s. La longueur des poème\index{gnl}{poeme@poème}s est variable: les quatre premiers contiennent quatre-vingt-quinze vers et plus. Les deux suivants ont respectivement cent et cinquante strophes. Les textes 7 \`a 14 en comportent vingt et les autres, moins longs, en poss\`edent souvent dix (17-29, 31, 33-38, 40-43 et 45).
Leur agencement ne concorde pas avec le d\'eroulement des diff\'erents \'ev\'enements narr\'es dans l'hagiographie\index{gnl}{hagiographie} de M\=a\d nikkav\=acakar\index{gnl}{Manikkavacakar@M\=a\d nikkav\=acakar}. Tous les hymne\index{gnl}{hymne}s sont dits \^etre li\'es \`a un site particulier: vingt-cinq pour Tillai\index{gnl}{Citamparam!Tillai} (Citamparam\index{gnl}{Citamparam}), vingt pour Peruntu\b rai\index{gnl}{Peruntu\b rai}, deux pour Tiruva\d n\d n\=amalai\index{gnl}{Tiruvannamalai@Tiruva\d n\d n\=amalai} et un pour Uttarak\=ocama\.nkai\index{gnl}{Uttarak\=ocama\.nkai}, Tirukka\b lukku\b n\b ram\index{gnl}{Tirukkalukunram@Tirukka\b lukku\b n\b ram}, Tirutt\=o\d nipuram\index{gnl}{Tonipuram@T\=o\d nipuram!Tirutt\=o\d nipuram} (C\=\i k\=a\b li\index{gnl}{Cikali@C\=\i k\=a\b li}), et Tiruv\=ar\=ur\index{gnl}{Ar\=ur@\=Ar\=ur!Tiruv\=ar\=ur}. De nombreux poème\index{gnl}{poeme@poème}s exaltent la dévotion\index{gnl}{devotion@dévotion} ardente envers \'Siva\index{gnl}{Siva@\'Siva} et sa puissance; certains (7 \`a 19) sont plac\'es dans la bouche de femmes vaquant \`a des occupations domestiques, ludiques ou autres. Ainsi, les hymne\index{gnl}{hymne}s de M\=a\d nikkav\=acakar\index{gnl}{Manikkavacakar@M\=a\d nikkav\=acakar} les plus chant\'es dans les temples\index{gnl}{temple}, et peut-\^etre les plus connus, sont le \textit{Tiruvemp\=avai}\index{gnl}{Tiruvempavai@\textit{Tiruvemp\=avai}} (septi\`eme) qui met en sc\`ene le chant des femme\index{gnl}{femme}s prenant leur bain matinal, ou le \textit{Tiruc\=a\b lal}\index{gnl}{Tirucalal@\textit{Tiruc\=a\b lal}} (douzi\`eme) qui est un jeu de questions-r\'eponses, entre jeunes filles, sur les formes de \'Siva\index{gnl}{Siva@\'Siva}. La consonnance philosophique de certains passages a \'et\'e considérée comme les racines de la doctrine \'Saiva Siddh\=anta\index{gnl}{Saiva@\'Saiva Siddh\=anta}\footnote{Cf. \textsc{Yocum} (1982) pour une \'etude r\'ecente de ce texte et \textsc{Ramachandran} (2001) pour une bibliographie exub\'erante mais non s\'elective.}.\\

%Notre r\'esum\'e de l'hagiographie\index{gnl}{hagiographie} de M\=a\d nikkav\=acakar\index{gnl}{Manikkavacakar@M\=a\d nikkav\=acakar} reprend celui de \textsc{Yocum} (1982) qui est bas\'e sur le \textit{Tiruv\=atav\=urpur\=a\d nam} attribu\'e \`a un certain Ka\d tavu\d lm\=amu\b nivar (\textsc{xv}\up{e} si\`ecle\string?)\footnote{\`A propos de la formation l\'egendaire autour de M\=a\d nikkav\=acakar\index{gnl}{Manikkavacakar@M\=a\d nikkav\=acakar}, avec toutefois des r\'eserves concernant la chronologie des textes qu'il pr\'esente, cf. \textsc{Swamy} (1972: 126): \og The process of incorporation is long drawn out and has been accomplished step by step and bit by bit as in Tiruv\=alav\=ayu\d daiyar's \textit{Tiruvilaiy\=a\d dalpur\=a\b nam}, \textit{Tiruvuttarak\=o\'samangaippur\=a\b nam}, \textit{Tirupperunturaippur\=a\b nam}, \textit{Kadambavanappur\=anam}, Paranjotiminiver's \textit{Tiruvilaiy\=a\d dalpur\=a\b nam}, etc., until a separate narrative of the entire life of V\=adav\=ur was composed in \textit{Tiruv\=adav\=ura\d diga\d lpur\=a\b nam}\fg.}.

%M\=a\d nikkav\=acakar\index{gnl}{Manikkavacakar@M\=a\d nikkav\=acakar}, n\'e \`a V\=atav\=ur\footnote{Il existe un temple shiva\"ite\index{gnl}{shiva\"ite} de ce nom \`a douze\index{gnl}{douze} kilom\`etres \`a l'est de Maturai\index{gnl}{Maturai} qui appartient au temple de M\=\i n\=ak\d s\=\i\ et dont l'image de procession\index{gnl}{procession} de M\=a\d nikkav\=acakar\index{gnl}{Manikkavacakar@M\=a\d nikkav\=acakar} est apport\'ee \`a Maturai\index{gnl}{Maturai} pendant la f\^ete\index{gnl}{fete@fête} d'\textit{\=Ava\d ni m\=ulam} (\textsc{Fuller} 1984: 119, 197 et 204).} dans une famille brahmane\index{gnl}{brahmane}, fait preuve d'une \'erudition remarquable \`a seize ans, \`a tel point que le roi\index{gnl}{roi} \textit{p\=a\d n\d dya}\index{gnl}{pandya@\textit{p\=a\d n\d dya}} le recrute en tant que premier ministre\index{gnl}{ministre} et lui octroie le titre de Te\b n\b nava\b n Piramar\=aya\b n. Pourtant le jeune homme, dont la d\'evotion envers \'Siva s'intensifie, est \`a la recherche d'un ma\^itre spirituel. Quand le roi\index{gnl}{roi} lui ordonne d'aller \`a Peruntu\b rai\index{gnl}{Peruntu\b rai} acheter des chevaux venus de l'\'etranger, il saisit l'occasion du voyage pour y trouver son guide. Il rencontre, \`a l'ombre d'un arbre, \'Siva d\'eguis\'e en ma\^itre qui le prend sous sa tutelle. Le jeune dévot\index{gnl}{devot(e)@dévot(e)} est initi\'e, puis il renonce et d\'epense tout ce qu'il poss\`ede, m\^eme l'argent du roi\index{gnl}{roi}. Ce dernier le convoque et l'emprisonne \`a Maturai\index{gnl}{Maturai}. Mais \'Siva incarn\'e en maquignon pr\'esente au roi\index{gnl}{roi} de magnifiques chevaux, des chacals transform\'es, et fait lib\'erer le ministre\index{gnl}{ministre} pour quelque temps seulement, car la nuit tomb\'ee les chevaux retrouvent leur nature origine\index{gnl}{origine}lle et M\=a\d nikkav\=acakar\index{gnl}{Manikkavacakar@M\=a\d nikkav\=acakar} la prison. Ensuite, au cours de travaux de barrage sur le fleuve Vaikai\index{gnl}{Vaikai} le roi\index{gnl}{roi} prend conscience de la grandeur de \'Siva et lib\`ere d\'efinitivement le jeune dévot\index{gnl}{devot(e)@dévot(e)} qui rejoint son guide \`a Peruntu\b rai\index{gnl}{Peruntu\b rai}. Ce dernier lui demande d'aller \`a Citamparam\index{gnl}{Citamparam}, en visitant au passage divers temples, o\`u il lui pr\'edit son union avec \'Siva apr\`es un d\'ebat victorieux contre les bouddhiste\index{gnl}{bouddhiste}s. En route pour Citamparam\index{gnl}{Citamparam}, M\=a\d nikkav\=acakar\index{gnl}{Manikkavacakar@M\=a\d nikkav\=acakar} compose des hymne\index{gnl}{hymne}s \`a la gloire de diff\'erents temples et arrive en extase sur le site du \'Siva dansant. S'y organise un d\'ebat contre des bouddhiste\index{gnl}{bouddhiste}s venus de Srilanka\index{gnl}{Srilanka} en pr\'esence du roi\index{gnl}{roi} \textit{c\=o\b la}\index{gnl}{cola@\textit{c\=o\b la}} et du roi\index{gnl}{roi} cingalais venu pour gu\'erir sa fille sourde-muette. M\=a\d nikkav\=acakar\index{gnl}{Manikkavacakar@M\=a\d nikkav\=acakar} implore Sarasvat\=\i, r\'eduit au silence les bouddhiste\index{gnl}{bouddhiste}s, donne parole \`a la princesse et ce faisant, convertit le roi\index{gnl}{roi} cingalais au shiva\"isme\index{gnl}{shivaisme@shiva\"isme}. Puis, \'Siva, sous la forme d'un gourou, demande \`a mettre par \'ecrit tous les hymne\index{gnl}{hymne}s que M\=a\d nikkav\=acakar\index{gnl}{Manikkavacakar@M\=a\d nikkav\=acakar} a chant\'es et les rend publics. Enfin, le poète\index{gnl}{poete@poète} s'unit avec \'Siva lorsqu'il explique aux dévot\index{gnl}{devot(e)@dévot(e)}s que \'Siva est le sens de ces poème\index{gnl}{poeme@poème}s\footnote{\textsc{Smith} (1998: 230-232) pr\'esente un r\'esum\'e similaire en se basant sur le \textit{Ku\~ncit\=a\.nghri-stava} attribu\'e \`a un Um\=apati\index{gnl}{Umapati@Um\=apati} Civ\=ac\=ariyar de Citamparam\index{gnl}{Citamparam}.}.\\

M\=a\d nikkav\=acakar\index{gnl}{Manikkavacakar@M\=a\d nikkav\=acakar} appartient aujourd'hui au groupe des \og ma\^itres de la religion\fg\ shiva\"ite\index{gnl}{shiva\"ite}, les \textit{camay\=ac\=ariyar}, ou du Quatuor, \textit{n\=alvar}\index{gnl}{nalvar@\textit{n\=alvar}}, qu'il forme avec les trois auteurs du \textit{T\=ev\=aram}\index{gnl}{Tevaram@\textit{T\=ev\=aram}}. Sa datation a été l'objet de controverse; aujourd'hui, le \textsc{ix}\up{e} si\`ecle est g\'en\'eralement accept\'e (\textsc{Zvelebil} 1975: 144).
M\=a\d nikkav\=acakar\index{gnl}{Manikkavacakar@M\=a\d nikkav\=acakar} serait \og le poète\index{gnl}{poete@poète} le plus important du mouvement shiva\"ite\index{gnl}{shiva\"ite} et le plus repr\'esentatif de l'\^ame tamoule\fg\ selon \textsc{Filliozat} (1994: 329). Cependant, il existe tr\`es peu d'\'etudes scientifiques sur cet auteur et les textes qui lui sont attribu\'es. \`A notre connaissance\index{gnl}{connaissance}, \textsc{Yocum} (1982) est le seul ouvrage r\'ecent qui propose une monographie sur le poète\index{gnl}{poete@poète}. Bien que \textsc{Yocum} ait donn\'e priorit\'e \`a l'analyse litt\'eraire des textes, il \'enum\`ere (\textsc{Yocum} 1982: 46-50) six arguments qui permettraient de placer M\=a\d nikkav\=acakar\index{gnl}{Manikkavacakar@M\=a\d nikkav\=acakar} dans le \textsc{ix}\up{e} si\`ecle\footnote{
Pour suivre le raisonnement de Glenn E. \textsc{Yocum}, il faut tout d'abord admettre que tous les hymne\index{gnl}{hymne}s rassemblés dans les deux textes qui constituent le livre \textsc{viii} du \textit{Tirumu\b rai}\index{gnl}{Tirumurai@\textit{Tirumu\b rai}}, depuis le \textit{Tirumu\b raika\d n\d tapur\=a\d nam}\index{gnl}{Tirumuraikantapuranam@\textit{Tirumu\b raika\d n\d tapur\=a\d nam}} au plus tard (voir 4.1), ont \'et\'e compos\'es par un unique auteur, nomm\'e M\=a\d nikkav\=acakar\index{gnl}{Manikkavacakar@M\=a\d nikkav\=acakar}. Nous r\'esumons ici ses propos.
Son premier argument, soutenu par de nombreux chercheurs, est l'absence de M\=a\d nikkav\=acakar\index{gnl}{Manikkavacakar@M\=a\d nikkav\=acakar} dans le \og Recueil des saints serviteurs\fg, le \textit{Tirutto\d n\d tattokai}\index{gnl}{Tiruttontattokai@\textit{Tirutto\d n\d tattokai}} (VII 39), qui formera la liste immuable des soixante-trois \textit{n\=aya\b nm\=ar}\index{gnl}{nayanmar@\textit{n\=aya\b nm\=ar}}, attribu\'e \`a Cuntarar\index{gnl}{Cuntarar} que la tradition\index{gnl}{tradition} place au \textsc{viii-ix}\up{e} si\`ecle. M\=a\d nikkav\=acakar\index{gnl}{Manikkavacakar@M\=a\d nikkav\=acakar} lui serait donc post\'erieur. Par ailleurs, son absence dans les \'etoffements hagiographique\index{gnl}{hagiographie!hagiographique}s de Nampi\index{gnl}{Nampi \=A\d n\d t\=ar Nampi} \=A\d n\d t\=ar Nampi et de C\=ekki\b l\=ar\index{gnl}{Cekkilar@C\=ekki\b l\=ar} n'impliquerait pas une post\'eriorit\'e \`a ces auteurs mais reposerait sur la fid\'elit\'e de ces derniers qui ont suivi Cuntarar\index{gnl}{Cuntarar}.
Ensuite, \textsc{Yocum} s'appuie sur les r\'ef\'erences au terme \textit{m\=ay\=av\=ada} dans l'\oe uvre du poète\index{gnl}{poete@poète} qui illustreraient sa connaissance\index{gnl}{connaissance} de la philosophie de \'Sa\.nkara dont le d\'ec\`es est plac\'e en 820. Ainsi, M\=a\d nikkav\=acakar\index{gnl}{Manikkavacakar@M\=a\d nikkav\=acakar} serait post\'erieur ou contemporain de celui-ci. Pour un compte rendu des \'etudes sur la datation de \'Sa\.nkara cf. \textsc{Harimoto} (2006) qui propose une nouvelle datation du \textit{Brahmas\=utra\'s\=a\.nkarabh\=a\d sya} et qui souligne la confusion dans laquelle est n\'ee la datation dite `traditionnelle' du philosophe, 788-820.
Le troisi\`eme argument s'appuie sur le fait que le \textit{Tirukk\=ovaiy\=ar}\index{gnl}{Tirukkovaiyar@\textit{Tirukk\=ovaiy\=ar}} mentionne un roi\index{gnl}{roi} \textit{p\=a\d n\d dya}\index{gnl}{pandya@\textit{p\=a\d n\d dya}} nomm\'e Varagu\d na. Deux rois de cette dynastie portent ce nom au \textsc{ix}\up{e} si\`ecle. Les historiens que suit \textsc{Yocum} s'accordent \`a consid\'erer notre poète\index{gnl}{poete@poète} comme contemporain de Varagu\d na II alias Varagu\d navarman (862-885\string?).
Le quatri\`eme argument de \textsc{Yocum} est fond\'e sur le fait que M\=a\d nikkav\=acakar\index{gnl}{Manikkavacakar@M\=a\d nikkav\=acakar} aurait eu connaissance\index{gnl}{connaissance} des \textit{m\=uvar}\index{gnl}{muvar@\textit{m\=uvar}}: il chante les sites de Campantar\index{gnl}{Campantar} (Ka\b lumalam\index{gnl}{Kalumalam@Ka\b lumalam}, \textit{i.e.} C\=\i k\=a\b li\index{gnl}{Cikali@C\=\i k\=a\b li}) et de Cuntarar\index{gnl}{Cuntarar} (Tiruv\=ar\=ur\index{gnl}{Ar\=ur@\=Ar\=ur!Tiruv\=ar\=ur}) et il reprendrait un vers d'Appar\index{gnl}{Appar}. \textsc{Yocum} (1982: 47): \og \textit{Tiruv\=acakam}\index{gnl}{Tiruvacakam@\textit{Tiruv\=acakam}} 5: 30, where M\=a\d nikkav\=acakar\index{gnl}{Manikkavacakar@M\=a\d nikkav\=acakar} says, ``\textit{y\=am \=arkkum ku\d ti all\=om y\=atum a\~nc\=om}", appears to rely on Appar\index{gnl}{Appar}'s \textit{T\=ev\=aram}\index{gnl}{Tevaram@\textit{T\=ev\=aram}}: ``\textit{n\=am\=arkum ku\d ti all\=om nama\b nai a\~nc\=om}"\fg. Le vers d'Appar\index{gnl}{Appar} se trouve en VI 98 1. Cependant, le poète\index{gnl}{poete@poète} ne mentionne jamais les \textit{m\=uvar}\index{gnl}{muvar@\textit{m\=uvar}} (cf. \textsc{Prentiss} 1999: 79).
Ensuite, la ressemblance entre le \textit{Tiruvemp\=avai}\index{gnl}{Tiruvempavai@\textit{Tiruvemp\=avai}} (hymne\index{gnl}{hymne} 7 du \textit{Tiruv\=acakam}\index{gnl}{Tiruvacakam@\textit{Tiruv\=acakam}}) et le \textit{Tirupp\=avai} d'\=A\d n\d t\=a\d l, po\'etesse vishnouite\index{gnl}{vishnouite} qui aurait v\'ecu au \textsc{ix}\up{e} si\`ecle, est un argument suppl\'ementaire pour dater l'auteur de ce m\^eme si\`ecle. \textsc{Filliozat} (1972: xiii) écrit: \og La similitude de composition du \textit{Tirupp\=avai} et du \textit{Tiruvemp\=avai}\index{gnl}{Tiruvempavai@\textit{Tiruvemp\=avai}}, tous deux de forme exceptionnelle dans la litt\'erature tamoule est aussi en faveur d'un rapprochement des \'epoques des deux poète\index{gnl}{poete@poète}s. La connaissance\index{gnl}{connaissance} chez l'un de l'\oe uvre de l'autre semble bien avoir inspir\'e au premier l'id\'ee de donner la r\'eplique au second. Mais, faute d'une chronologie pr\'ecise, nous ne pouvons d\'ecider de la priorit\'e de l'un ou de l'autre\fg.
Enfin, le dernier argument de \textsc{Yocum} repose sur l'identification d'un roi\index{gnl}{roi} cingalais bouddhiste\index{gnl}{bouddhiste} converti au shiva\"isme\index{gnl}{shivaisme@shiva\"isme} apr\`es la guérison\index{gnl}{guerison@guérison} de sa fille par M\=a\d nikkav\=acakar\index{gnl}{Manikkavacakar@M\=a\d nikkav\=acakar} \`a Citamparam\index{gnl}{Citamparam}.}.
%\footnote{\textsc{Yocum} soul\`eve deux objections \`a cette datation qu'il \'ecarte vite. La premi\`ere est fond\'ee sur l'importance accord\'ee au bouddhisme au \textsc{ix}\up{e} si\`ecle alors que ce mouvement aurait \'et\'e en d\'eclin dans le Sud et sur la totale absence de la mention des ja\"in\index{gnl}{jain@ja\"in}s qui \'etaient plus nombreux. \textsc{Yocum} pense que les bouddhiste\index{gnl}{bouddhiste}s rencontr\'es par M\=a\d nikkav\=acakar\index{gnl}{Manikkavacakar@M\=a\d nikkav\=acakar} sont venus du Srilanka\index{gnl}{Srilanka}. La seconde objection porte sur un r\'ecit l\'egendaire o\`u \'Siva transforme des chacals en chevaux. Cette légende\index{gnl}{legende@légende}, \'el\'ement caract\'eristique de l'hagiographie\index{gnl}{hagiographie} de M\=a\d nikkav\=acakar\index{gnl}{Manikkavacakar@M\=a\d nikkav\=acakar}, fut \'evoqu\'ee bri\`evement par Appar\index{gnl}{Appar} (IV 4 2: \textit{nariyaik kutirai ceyv\=a\b num} \og et Celui qui transforme les renards/chacals en chevaux\fg). \textsc{Yocum} plaide qu'il s'agit d'un ancien r\'ecit connu par de nombreux poète\index{gnl}{poete@poète}s comme Appar\index{gnl}{Appar} mais que l'usage fr\'equent dont en fait M\=a\d nikkav\=acakar\index{gnl}{Manikkavacakar@M\=a\d nikkav\=acakar} a conduit \`a l'associer et \`a l'int\'egrer dans la légende\index{gnl}{legende@légende}.}

%Nous regrettons la faiblesse argumentative de \textsc{Yocum} qui s'appuie principalement sur les \'etudes pr\'ec\'edentes. La suppos\'ee connaissance\index{gnl}{connaissance} des \textit{m\=uvar} et de \'Sa\.nkara par M\=a\d nikkav\=acakar\index{gnl}{Manikkavacakar@M\=a\d nikkav\=acakar} implique que notre poète\index{gnl}{poete@poète} leur est post\'erieur certes, mais non qu'il a v\'ecu au \textsc{ix}\up{e} si\`ecle. De plus, la ressemblance litt\'eraire entre le \textit{Tiruvemp\=avai}\index{gnl}{Tiruvempavai@\textit{Tiruvemp\=avai}} et le \textit{Tirupp\=avai} d'\=A\d n\d t\=a\d l, dont la date n'est pas certaine, n'induit pas n\'ecessairement que les poète\index{gnl}{poete@poète}s sont contemporains. Ensuite, l'identification du roi\index{gnl}{roi} cingalais bouddhiste\index{gnl}{bouddhiste}, bas\'ee sur l'autorit\'e d'une chronique cingalaise du \textsc{xv}\up{e} si\`ecle et plus tard du \textit{Tiruv\=atav\=urarpur\=a\d nam}, hagiographie\index{gnl}{hagiographie} du poète\index{gnl}{poete@poète}\footnote{Nous rappelons que la conversion au shiva\"isme\index{gnl}{shivaisme@shiva\"isme} d'un roi\index{gnl}{roi} h\'et\'erodoxe est un th\`eme r\'epandu dans les hagiographie\index{gnl}{hagiographie}s: Campantar\index{gnl}{Campantar} convertit un roi\index{gnl}{roi} \textit{p\=a\d n\d dya}\index{gnl}{pandya@\textit{p\=a\d n\d dya}} ja\"in\index{gnl}{jain@ja\"in} apr\`es la guérison\index{gnl}{guerison@guérison} de ce dernier et Appar\index{gnl}{Appar} un roi\index{gnl}{roi} \textit{pallava}\index{gnl}{pallava@\textit{pallava}} ja\"in\index{gnl}{jain@ja\"in}.}, n'est absolument pas convaincante. Enfin, l'identification du roi\index{gnl}{roi} Varaku\d na\b n, mentionn\'e \`a deux reprise dans le \textit{Tirukk\=ovaiy\=ar}\index{gnl}{Tirukkovaiyar@\textit{Tirukk\=ovaiy\=ar}} (st. 306 et 327), comme Varagu\d na II (862-885) est discutable. Aucun \'el\'ement ne permet cette identification dans le poème\index{gnl}{poeme@poème} o\`u il est question d'un roi\index{gnl}{roi} \textit{p\=a\d n\d dya}\index{gnl}{pandya@\textit{p\=a\d n\d dya}}, nomm\'e ou titr\'e Varaku\d na\b n, li\'e \`a Citamparam\index{gnl}{Citamparam}\footnote{Notons que, plus tard, dans le \textit{Tiruvi\d laiy\=a\d ta\b rpur\=a\d nam}, \og légende\index{gnl}{legende@légende} des jeux sacr\'es (de \'Siva)\fg, aux jeux 39 \`a 44, le roi\index{gnl}{roi} de Maturai\index{gnl}{Maturai}, fervent dévot\index{gnl}{devot(e)@dévot(e)} shiva\"ite\index{gnl}{shiva\"ite}, est XXXXX; cf. \textit{Tiruvi\d laiy\=a\d ta\b rpur\=a\d nam} st. XXXXX et \textsc{Dessigane}, \textsc{Pattabiramin}, \textsc{Filliozat} (1960: 59-67).}. Ainsi, il nous semble que la datation du \textsc{ix}\up{e} si\`ecle pour M\=a\d nikkav\=acakar\index{gnl}{Manikkavacakar@M\=a\d nikkav\=acakar} ou les \oe uvres qu'il aurait compos\'ees, fond\'ee sur des sources litt\'eraires, reste contestable.

\`A notre connaissance\index{gnl}{connaissance}, ce n'est qu'au \textsc{xii}\up{e} si\`ecle\footnote{L'affirmation de \textsc{Swamy} (1972: 97) que ARE 1940-41 157 (Nall\=ur\index{gnl}{Nallur@Nall\=ur}, Te\b n\b n\=a\b rk\=a\d tu dt.\index{gnl}{Te\b n\b n\=a\b rk\=a\d tu dt.}) est la premi\`ere inscription \`a mentionner le \textit{Tiruc\=a\b lal}\index{gnl}{Tirucalal@\textit{Tiruc\=a\b lal}} (douxi\`eme hymne\index{gnl}{hymne} de ce qui forme le \textit{Tiruv\=acakam}\index{gnl}{Tiruvacakam@\textit{Tiruv\=acakam}}) et qu'elle date du r\`egne de V\=\i rar\=ajendrac\=o\b la, soit de 1069, est discutable car cet ARE p.~243 et \textsc{Mahalingam} (1988: 496) identifient ce roi\index{gnl}{roi} comme Kulottu\.nga III\index{gnl}{Kulottu\.nga III} et datent le texte de 1184. L'agencement des autres \'epigraphes sur les murs du temple\index{gnl}{temple} et leur datation tardive donnent plut\^ot raison \`a ces derniers. Une v\'erification \textit{in situ} est indispensable pour trancher la question.} que des t\'emoignages \'epigraphiques pr\'ecis sur les images du poète\index{gnl}{poete@poète} et le chant de deux hymne\index{gnl}{hymne}s du corpus\index{gnl}{corpus} du \textit{Tiruv\=acakam}\index{gnl}{Tiruvacakam@\textit{Tiruv\=acakam}} semblent appara\^itre, bien que, \'etrangement, les noms de M\=a\d nikkav\=acakar\index{gnl}{Manikkavacakar@M\=a\d nikkav\=acakar} et du \textit{Tiruv\=acakam}\index{gnl}{Tiruvacakam@\textit{Tiruv\=acakam}} n'y figurent pas\footnote{Une inscription du temple\index{gnl}{temple} de N\=age\'svara \`a Kumpak\=o\d nam (ARE 1911 258), qui contient l'éloge\index{gnl}{eloge@éloge} royal de R\=ajar\=aja III\index{gnl}{Rajaraja III@R\=ajar\=aja III} (1216-1279) \textit{c\=\i r ma\b n\b ni irun\=a\b nku ticai}, mentionne un donateur nomm\'e Tiru\~n\=a\b nacampantar M\=a\d nikkav\=acaka\b n. Cependant, nous n'avons pas rencontr\'e de textes \'epigraphiques nommant ainsi une image du poète\index{gnl}{poete@poète}.}. Une \'epigraphe du r\`egne de R\=ajar\=aja II\index{gnl}{Rajaraja II@R\=ajar\=aja II}, dat\'ee de 1158, d\'ecrit l'installation\index{gnl}{installation d'une image} par deux danseuse\index{gnl}{danseuse}s du temple\index{gnl}{temple} de trois images: Appar\index{gnl}{Appar}, Tiruv\=atav\=ur\=a\d lika\d l\index{gnl}{Manikkavacakar@M\=a\d nikkav\=acakar!Tiruv\=atav\=ur\=a\d lika\d l} et Ka\d n\d nappar\index{gnl}{Kannappar@Ka\d n\d nappar}\footnote{SII 8 228 l.~9-10: \textit{e\b lunta[ru\d luvitta] tirun\=a[vu]kkarai[cu]teva\b r[kum] tiruv\=atav\=ur\=a\d lika\d lukkum [tiruk]ka\d n\d nappatevarkkum}, \og pour Tirun\=avukkaraicutevar, Tiruv\=atav\=ur\=a\d lika\d l\index{gnl}{Manikkavacakar@M\=a\d nikkav\=acakar!Tiruv\=atav\=ur\=a\d lika\d l} et Tirukka\d n\d nappatevar qui ont \'et\'e \'erig\'es\fg.}. Tiruv\=atav\=ur\=a\d lika\d l\index{gnl}{Manikkavacakar@M\=a\d nikkav\=acakar!Tiruv\=atav\=ur\=a\d lika\d l} est identifi\'e comme M\=a\d nikkav\=acakar\index{gnl}{Manikkavacakar@M\=a\d nikkav\=acakar} parce que V\=atav\=ur\index{gnl}{Vatavur@V\=atav\=ur} est son lieu de naissance\index{gnl}{naissance}, parce que des inscriptions lient cette figure avec le chant du \textit{Tiruvemp\=avai}\index{gnl}{Tiruvempavai@\textit{Tiruvemp\=avai}} (ARE 1912 421) et, enfin, parce que Nampi\index{gnl}{Nampi \=A\d n\d t\=ar Nampi} \=A\d n\d t\=ar Nampi, jouant sur le terme \textit{v\=acakam}, semble se r\'ef\'erer \`a lui quand il parle d'un dévot\index{gnl}{devot(e)@dévot(e)} shiva\"ite\index{gnl}{shiva\"ite} de V\=atav\=ur\index{gnl}{Vatavur@V\=atav\=ur} qui a compos\'e un \textit{Tirukk\=ovai}\footnote{\textit{K\=oyil tiruppa\d n\d niyar viruttam}:
\begin{verse}
\textit{varu\textbf{v\=a cakatti\b nil}, mu\b r\b ru\d nart t\=o\b nai, va\d ntillaima\b n\b nait\\
\textbf{tiruv\=ata v\=ur}c\textbf{civa p\=attiya\b n} ceytiruc ci\b r\b rampalap\\
poru\d l\=ar taru\textbf{tiruk k\=ovai}ka\d n \d t\=eyuma\b r \b rapporu\d lait\\
teru\d l\=ata vu\d l\d lat tavarkavi p\=a\d tic cirippippar\=e.} (58)\\
\end{verse}}. Ainsi, M\=a\d nikkav\=acakar\index{gnl}{Manikkavacakar@M\=a\d nikkav\=acakar}, sur le m\^eme plan que les \og v\'eritables\fg\ \textit{n\=aya\b nm\=ar}\index{gnl}{nayanmar@\textit{n\=aya\b nm\=ar}}, est sanctifi\'e dans l'enceinte des temples\index{gnl}{temple}. Ailleurs, il est aussi appel\'e \og V\=adav\=ur-N\=aya\b n\=ar\fg\ (ARE 1912 420).
Bien que nous envisagions la possibilit\'e que les hymne\index{gnl}{hymne}s de M\=a\d nikkav\=acakar\index{gnl}{Manikkavacakar@M\=a\d nikkav\=acakar} aient \'et\'e r\'epertori\'es dans les inscriptions sous la d\'esignation g\'en\'erale de \textit{tiruppatiyam}\index{gnl}{tiruppatiyam@\textit{tiruppatiyam}}, une \'etude plus ample est n\'ecessaire pour la soutenir. Ne sont donc abord\'es ici que certains textes \'evoquant les chants\index{gnl}{chant} du \textit{Tiruc\=a\b lal}\index{gnl}{Tirucalal@\textit{Tiruc\=a\b lal}} et du \textit{Tiruvemp\=avai}\index{gnl}{Tiruvempavai@\textit{Tiruvemp\=avai}}, textes qui appartiennent, nous le rappelons, au \textit{Tiruv\=acakam}\index{gnl}{Tiruvacakam@\textit{Tiruv\=acakam}}.

Une inscription datant de la dix-septi\`eme ann\'ee de Vikramac\=o\b la\index{gnl}{Vikramacola@Vikramac\=o\b la}, SII 22 165 l.~2, enregistre un don\index{gnl}{don} pour qu'entre autres la d\'eesse\index{gnl}{deesse@déesse} parte en procession\index{gnl}{procession} tous les dimanches accompagn\'ee du chant du \textit{Tiruc\=a\b lal}\index{gnl}{Tirucalal@\textit{Tiruc\=a\b lal}}\footnote{Il est int\'eressant de noter que le rapport de l'ARE 1940-41 157 met aussi en relation cet hymne\index{gnl}{hymne}, qui c\'el\`ebre exclusivement \'Siva\index{gnl}{Siva@\'Siva}, avec la d\'eesse\index{gnl}{deesse@déesse}; en effet, il \'evoque un don\index{gnl}{don} de terre\index{gnl}{terre}s pour assurer le chant du \textit{Tiruc\=a\b lal}\index{gnl}{Tirucalal@\textit{Tiruc\=a\b lal}} et diverses offrandes \`a la d\'eesse\index{gnl}{deesse@déesse}.}.
De plus, ARE 1912 421 de Va\b luv\=ur\index{gnl}{Valuvur@Va\b luv\=ur} (M\=ayavaram tk.\index{gnl}{Mayavaram@M\=ayavaram tk.}), datant du premier juillet 1167 (\textsc{Mahalingam} 1992: 349), stipule un don\index{gnl}{don} pour que soit r\'ecit\'e le \textit{Tiruvemp\=avai}\index{gnl}{Tiruvempavai@\textit{Tiruvemp\=avai}} devant l'image de \og V\=adav\=ur\=a\d l-N\=aya\b n\=ar\fg\ pendant la f\^ete\index{gnl}{fete@fête} du mois de \textit{m\=arka\b li} (d\'ecembre-janvier)\footnote{Le rapport de l'ARE 1943-44 192 de Maturai\index{gnl}{Maturai}, d'une inscription grav\'ee sous Sundara P\=a\d n\d dya III, datant de 1219, mentionne le chant du \textit{Tiruvemp\=avai}\index{gnl}{Tiruvempavai@\textit{Tiruvemp\=avai}} par des asc\`etes le mois de \textit{m\=arka\b li}.}. Enfin, quatre inscriptions de Nall\=ur\index{gnl}{Nallur@Nall\=ur} (Te\b n\b n\=a\b rk\=a\d tu dt.\index{gnl}{Te\b n\b n\=a\b rk\=a\d tu dt.}) du r\`egne de Kulottu\.nga III\index{gnl}{Kulottu\.nga III}, entre 1198 et 1202, enregistrent des dons\index{gnl}{don} pour que soient chant\'es diff\'erents morceaux du \textit{Tiruvemp\=avai}\index{gnl}{Tiruvempavai@\textit{Tiruvemp\=avai}} par les danseuse\index{gnl}{danseuse}s du temple\index{gnl}{temple}. En effet, ARE 1940-41 143 \'evoque le \textit{mutalp\=a\d t\d tu} \og premi\`ere strophe\fg\ de l'hymne\index{gnl}{hymne}, 161 l'\textit{ira\d n\d t\=amp\=a\d t\d tu} \og deuxi\`eme strophe\fg\ et enfin, 149 et 160 le \textit{ka\d taik\=appu} \og protection finale\fg. Il semble que ces chants\index{gnl}{chant} \'etaient accompagn\'es de danse\index{gnl}{danse}. Nous rappelons que le \textit{Tiruvemp\=avai}\index{gnl}{Tiruvempavai@\textit{Tiruvemp\=avai}} \'etait chant\'e principalement par des renon\c cants et des femme\index{gnl}{femme}s et que cette mise en sc\`ene particuli\`ere souligne le statut asc\'etique traditionnel de M\=a\d nikkav\=acakar\index{gnl}{Manikkavacakar@M\=a\d nikkav\=acakar} et le contenu de son hymne\index{gnl}{hymne} qui n'est autre que le chant des femme\index{gnl}{femme}s prenant leur bain matinal\footnote{Cette pr\'esentation sommaire des donn\'ees \'epigraphiques m\'erite d'\^etre d\'evelopp\'ee et compl\'et\'ee pour chercher des \'el\'ements de r\'eponse aux interrogations l\'egitimes et aux conclusions h\^atives de \textsc{Swamy} (1972: 118-128).}.


Mais les textes attribu\'es \`a M\=a\d nikkav\=acakar\index{gnl}{Manikkavacakar@M\=a\d nikkav\=acakar} ne mentionnent pas notre poète\index{gnl}{poete@poète} Campantar\index{gnl}{Campantar} et la seule et br\`eve r\'ef\'erence \`a C\=\i k\=a\b li\index{gnl}{Cikali@C\=\i k\=a\b li} se trouve dans une liste de lieux saints shiva\"ite\index{gnl}{shiva\"ite}s du poème\index{gnl}{poeme@poème} intitul\'e \textit{K\=\i rttittiruvakaval}: \og et ayant fait apparition \`a Ka\b lumalam\index{gnl}{Kalumalam@Ka\b lumalam}\fg\ (\textit{ka\b lumala mata\b ni\b r k\=a\d tci ko\d tuttum}, l.~88). Enfin la tradition\index{gnl}{tradition} rattache le poème\index{gnl}{poeme@poème} intitul\'e \textit{Pi\d titta pattu}, attribu\'e \`a ce m\^eme auteur, \`a T\=o\d nipuram\index{gnl}{Tonipuram@T\=o\d nipuram} parce que, au d\'ebut de la strophe 3, \'Siva\index{gnl}{Siva@\'Siva} est appel\'e \textit{ammaiy\=e app\=a}, \og m\`ere, père\index{gnl}{pere@père}\fg. Si Ammaiyappa\b n est bien le nom actuel de l'image de \'Siva\index{gnl}{Siva@\'Siva} se trouvant dans le temple\index{gnl}{temple} \`a \'etage du complexe de C\=\i k\=a\b li\index{gnl}{Cikali@C\=\i k\=a\b li} (voir 8.1), \'Siva\index{gnl}{Siva@\'Siva} ne porte ce nom ni dans les textes du \textit{Tirumu\b rai}\index{gnl}{Tirumurai@\textit{Tirumu\b rai}}, ni dans les inscriptions. Par ailleurs, il n'y a aucune autre r\'ef\'erence au site dans le poème\index{gnl}{poeme@poème}. Nous pensons donc que les commentateurs du \textit{Pi\d titta pattu} ont associ\'e ce texte au site de C\=\i k\=a\b li\index{gnl}{Cikali@C\=\i k\=a\b li} de fa\c con anachronique et que, cet hymne\index{gnl}{hymne}, \`a caract\`ere g\'en\'eral, ne c\'el\`ebre aucun temple\index{gnl}{temple} en particulier.

\begin{center}
*
\end{center}

%C'est dans quelques textes du onzi\`eme \textit{Tirumu\b rai} que la légende\index{gnl}{legende@légende} de Campantar\index{gnl}{Campantar} se retrouve v\'eritablement repr\'esent\'ee, apparaissant par petites touches.

L'\'etude intrins\`eque de l'\oe uvre attribu\'ee \`a Campantar\index{gnl}{Campantar} nous place devant des probl\`emes d'interpolation\index{gnl}{interpolation}s.
Les hymne\index{gnl}{hymne}s du \textit{T\=ev\=aram}\index{gnl}{Tevaram@\textit{T\=ev\=aram}} attribu\'es \`a Campantar\index{gnl}{Campantar} se caract\'erisent par une structure fixe, une griffe personnalis\'ee dans l'envoi\index{gnl}{envoi} et l'emploi fr\'equent de procédé\index{gnl}{procédé littéraire}s litt\'eraires. Or, nous suspectons qu'un grand nombre d'envois\index{gnl}{envoi} et d'hymne\index{gnl}{hymne}s compos\'es selon des procédé\index{gnl}{procédé littéraire}s stylistiques seraient des ajouts post\'erieurs. De plus, certaines r\'ef\'erences biographiques\index{gnl}{biographie!biographique} de Campantar\index{gnl}{Campantar} sont douteuses, et d'autres se révèlent \^etre clairement des ajouts.
Enfin, les allusions \`a Campantar\index{gnl}{Campantar} dans les hymne\index{gnl}{hymne}s des autres \textit{m\=uvar}\index{gnl}{muvar@\textit{m\=uvar}} n'infirment pas nos doutes.

 Ainsi, sur la base des donn\'ees internes, nous proposons l'hypoth\`ese que le poète Campantar\index{gnl}{Campantar} n'est pas l'auteur unique des trois cent quatre-vingt-cinq hymne\index{gnl}{hymne}s du \textit{T\=ev\=aram} qui semblent avoir \'et\'e r\'eunis au moment d'une compilation\index{gnl}{compilation} ou, peut-\^etre, pour justifier, en partie, les \'ecrits des hagiographes. Nous avons le sentiment d'\^etre confront\'ee \`a un corpus\index{gnl}{corpus} h\'et\'erog\`ene, incluant des strophes et des hymne\index{gnl}{hymne}s de poète\index{gnl}{poete@poète}s de dates vari\'ees, qui est présenté comme l'\oe uvre d'un auteur unique appel\'e sous le nom collectif de Campantar\index{gnl}{Campantar}\footnote{Cf. \textsc{Shulman} (1990: xxxviii-xl) pour une interpr\'etation similaire de la figure de Cuntarar\index{gnl}{Cuntarar} et \textsc{Hawley} 1988 pour une étude sur les auteurs des poèmes de \textit{bhakti}\index{gnl}{bhakti@\textit{bhakti}} des \textsc{xv-xvii}\up{e} siècles de l'Inde du Nord.}.
De plus, nous avons aussi \'emis des doutes quant \`a l'attribution des douze\index{gnl}{douze} toponymes au seul site de C\=\i k\=a\b li\index{gnl}{Cikali@C\=\i k\=a\b li}. Ne faudrait-il pas aussi consid\'erer C\=\i k\=a\b li\index{gnl}{Cikali@C\=\i k\=a\b li} comme un toponyme sous lequel auraient été rassemblés douze\index{gnl}{douze} sites distincts?


\chapter{C\=\i k\=a\b li aux douze noms}

Un \emph{talapur\=a\d nam}\index{gnl}{Purana@\textit{Pur\=a\d na}!\emph{talapur\=a\d nam}}, \og histoire d'un site\fg\ \footnote{Le terme est d\'eriv\'e du sanskrit \emph{sthalapur\=a\d na}\index{gnl}{Purana@\textit{Pur\=a\d na}!\emph{sthalapur\=a\d na}}. D'ailleurs, chaque texte tamoul se r\'ef\`ere \`a un anc\^etre sanskrit, souvent introuvable et douteux. La plupart des textes se disent \^etre des traductions de divers chapitres de \emph{pur\=a\d na}\index{gnl}{Purana@\textit{Pur\=a\d na}} sanskrits dont le plus fr\'equent est le \emph{Skandapur\=a\d na}.}, est un type de texte, g\'en\'eralement compos\'e en vers, racontant les mythe\index{gnl}{mythe}s fondateurs qui ont apport\'e ou r\'ev\'el\'e la saintet\'e d'un lieu, souvent d'un temple\index{gnl}{temple}.
Le \textit{C\=\i k\=a\b littalapur\=a\d nam}\index{gnl}{Cikalittalapuranam@\textit{C\=\i k\=a\b littalapur\=a\d nam}}, \og histoire du site de C\=\i k\=a\b li\index{gnl}{Cikali@C\=\i k\=a\b li}\fg, a \'et\'e compos\'e au milieu du \textsc{xviii}\up{e} si\`ecle par Aru\d n\=acalakkavir\=ayar\index{gnl}{Arunacalakkavirayar@Aru\d n\=acalakkavir\=ayar} (1712-1779), originaire de Tillaiy\=a\d ti, dont les talents ont \'et\'e grandement r\'ecompens\'es \`a la cour du Mah\=ar\=aja de Ta\~nc\=av\=ur\index{gnl}{Tancavur@Ta\~nc\=av\=ur}\footnote{Ce texte a été commandit\'e par Citamparan\=atamu\b ni, disciple renon\c cant responsable du monast\`ere\index{gnl}{monastère} de C\=\i k\=a\b li\index{gnl}{Cikali@C\=\i k\=a\b li}, une des annexes de Tarumapuram\index{gnl}{Tarumapuram} \`a l'\'epoque. Il a été publi\'e en 1887 par Cap\=an\=ayakamutaliy\=ar, un puissant local\index{gnl}{local}, puis r\'eimprim\'e, en 1937, sous la direction de son fils, Ca. Cat\=acivamutaliy\=ar. Cf. \og Autour des \emph{talapur\=a\d nam}\index{gnl}{Purana@\textit{Pur\=a\d na}!\emph{talapur\=a\d nam}} au Pays Tamoul\index{gnl}{Pays Tamoul}\fg, notre pr\'esentation, faite le 20 mars 2006, dans le cadre de la premi\`ere Journ\'ee Monde Indien organis\'ee par l'UMR 7528 Mondes iranien et indien.}.
%Dans cette r\'eimpression sont ajout\'es: une introduction, une biographie\index{gnl}{biographie} de l'auteur (par S. Ir\=amaccantira\b n), les f\'elicitations d'un homme politique local\index{gnl}{local} (le repr\'esentant de l'\'education du district, \emph{jill\=a kalvi atik\=ari}, Pa.Mu. C\=omacuntara\b n), de nombreux textes c\'el\'ebrant le site et des r\'esum\'es complets. Le texte n'a pas \'et\'e r\'eimprim\'e depuis.\\
Ce texte contient mille cinq cent cinquante-trois quatrains et serait une traduction condens\'ee d'une version sanskrite\footnote{Voici les d\'etails de cette version sanskrite en quarante chapitres que nous lisons dans l'introduction de l'\'edition du \textit{C\=\i k\=a\b littalapur\=a\d nam}\index{gnl}{Cikalittalapuranam@\textit{C\=\i k\=a\b littalapur\=a\d nam}} (p. xiv): dix-huit du \emph{Pavu\d tika} (sk. \textit{Bhavi\d sya}), un du \emph{Piram\=a\d n\d ta} (sk. \textit{Brahm\=a\index{gnl}{Brahma@Brahm\=a}\d n\d da}) et vingt-et-un du \emph{Kantapur\=a\d nam} (dont neuf du \textit{Ca\.nkaraca\.nkitai}, un du \textit{Ca\b na\b rkum\=araca\.nkitai} et onze du troisi\`eme \textit{Pariccetam}). Nous n'avons pas retrouv\'e ces diff\'erents chapitres sanskrits.} en quarante chapitres. Le texte s'organise en une introduction\footnote{L'introduction comprend une \og protection de Vin\=ayakar\fg\ (\emph{vin\=ayakar k\=appu}, deux st.), un \og hommage aux dieux\fg\ (\emph{ka\d tavu\d l v\=a\b lttu}, trente-et-une st.), une \og c\'el\'ebration du pays\fg\ (\emph{tirun\=a\d t\d tucci\b rappu}, soixante st.), une \og c\'el\'ebration de la ville\fg\ (\emph{tirunakaracci\b rappu}, quatre-vingt-dix st.) et une \og histoire du \textit{pur\=a\d na}\fg\ (\emph{pur\=a\d na varal\=a\b ru}, quarante-sept st.).} et trente-et-un chapitres (\emph{attiy\=ayam}), dont douze\index{gnl}{douze} sur les mythe\index{gnl}{mythe}s fondateurs du site (chapitres 2, 3, 6, 7, 9, 10, 12, 13, 15, 16, 17 et 18), un sur Campantar\index{gnl}{Campantar} (chapitre 23) et deux sur Ca\d t\d tain\=atar\index{gnl}{Cattainatar@Ca\d t\d tain\=atar} (chapitres 20 et 25)\footnote{Nous \'etudions cette figure dans la dernière partie.}:

\begin{center}
\scriptsize
\begin{longtable} {|c|c|c|}
\caption{Les douze chapitres des mythes fondateurs}\endfirsthead
\hline
Chapitre& Toponyme& Nombre de strophes\endhead
\hline
Chapitre& Toponyme& Nombre de strophes\\
\hline
\hline
 2 &T\=o\d nipuram\index{gnl}{Tonipuram@T\=o\d nipuram} &41\\
 \hline
3& Piramapuram\index{gnl}{Piramapuram}& 50 \\
\hline
6& \'Sr\=\i k\=a\d lipuram &61 \\
\hline
7& Ve\.nkuru\index{gnl}{Venkuru@Ve\.nkuru} &39 \\
\hline
9& Pukali\index{gnl}{Pukali} &39 \\
\hline
10& Cirapuram\index{gnl}{Cirapuram} &24 \\
\hline
12& Ca\d npai\index{gnl}{Canpai@Ca\d npai} &34 \\
\hline
13& Koccai\index{gnl}{Koccai} &38 \\
\hline
15& V\=e\d nupuram\index{gnl}{Venupuram@V\=e\d nupuram} &47 \\
\hline
16& Ka\b lumalam\index{gnl}{Kalumalam@Ka\b lumalam} &19 \\
\hline
17& Pu\b ravam\index{gnl}{Puravam@Pu\b ravam} &39 \\
\hline
 18 &Tar\=ay\index{gnl}{Taray@Tar\=ay} &19 \\
 \hline
\end{longtable}
\end{center}


\normalsize

Les douze\index{gnl}{douze} noms de C\=\i k\=a\b li\index{gnl}{Cikali@C\=\i k\=a\b li} sont expliqu\'es dans le \emph{talapur\=a\d nam}\index{gnl}{Purana@\textit{Pur\=a\d na}!\emph{talapur\=a\d nam}} par des mythe\index{gnl}{mythe}s fondateurs qui sont, souvent, des versions \og tamoulis\'ees\fg\ et localis\'ees de r\'ecits mythologiques panindiens connus à travers des textes fameux du corpus sanskrit. Ainsi, des histoires emprunt\'ees aux \textit{pur\=a\d na}\index{gnl}{Purana@\textit{Pur\=a\d na}} et aux \'epop\'ees\index{gnl}{epopees@épopées} sanskrits sont relocalis\'ees \`a C\=\i k\=a\b li\index{gnl}{Cikali@C\=\i k\=a\b li}. Nous commençons par r\'esumer ici les douze\index{gnl}{douze} chapitres dans l'ordre\index{gnl}{ordre} de pr\'esentation du \textit{talapur\=a\d na} qui n'est pas, au passage, celui que nous avons relev\'e dans les \textit{Tirumu\b rai}\index{gnl}{Tirumurai@\textit{Tirumu\b rai}} (cf. 2.1.3):
\begin{enumerate}

\item Le site est appel\'e T\=o\d nipuram\index{gnl}{Tonipuram@T\=o\d nipuram}, \og ville du radeau\fg, parce que \'Siva\index{gnl}{Siva@\'Siva} et sa par\`edre y sont venus se poser pendant le déluge\index{gnl}{deluge@déluge} sur leur barque\index{gnl}{barque}. Ce lieu devient le centre cosmique \`a partir duquel la cr\'eation peut recommencer. Si le mythe\index{gnl}{mythe} du déluge\index{gnl}{deluge@déluge} est universel celui de \'Siva\index{gnl}{Siva@\'Siva} y naviguant sur une barque\index{gnl}{barque} para\^it appartenir \`a la tradition\index{gnl}{tradition} tamoule
(\textsc{Shulman} 1980: 55-63).

\item C\=\i k\=a\b li\index{gnl}{Cikali@C\=\i k\=a\b li} obtient le nom de Piramapuram\index{gnl}{Piramapuram}, \og ville de Brahm\=a\index{gnl}{Brahma@Brahm\=a}\fg, parce que ce dernier y a honor\'e \'Siva\index{gnl}{Siva@\'Siva} pour que son \oe uvre de cr\'eation se d\'eroule correctement.
De nombreux temples\index{gnl}{temple} ou \textit{li\.nga}\index{gnl}{linga@\textit{li\.nga}} portent ce nom au Pays Tamoul\index{gnl}{Pays Tamoul} comme \`a Pu\d l\d lma\.nkai\index{gnl}{Pullamankai@Pu\d l\d lma\.nkai} par exemple (voir \textsc{Schmid} (2005) pour une étude monographique de ce temple).

\item Le toponyme K\=a\b li\index{gnl}{Kali@K\=a\b li} est expliqu\'e par deux mythe\index{gnl}{mythe}s dans le \emph{talapur\=a\d nam}\index{gnl}{Purana@\textit{Pur\=a\d na}!\emph{talapur\=a\d nam}}. Dans le premier, la d\'eesse\index{gnl}{deesse@déesse} K\=al\=\i\index{gnl}{Kali@K\=al\=\i\ la déesse} est venue là faire p\'enitence apr\`es sa d\'efaite lors de la comp\'etition de danse\index{gnl}{danse} contre \'Siva\index{gnl}{Siva@\'Siva} \`a Tillai\index{gnl}{Citamparam!Tillai}. Dans le second, le serpent\index{gnl}{serpent} K\=aliya\index{gnl}{Kaliya@K\=aliya}, vaincu par K\textsubring{r}\d s\d na qui a dansé\index{gnl}{danser} sur sa t\^ete et suivant son ordre\index{gnl}{ordre}, y est venu expier sa faute. Notons que les deux anthroponymes, K\=al\=\i\index{gnl}{Kali@K\=al\=\i\ la déesse} (tam. K\=a\d li) et K\=aliya\index{gnl}{Kaliya@K\=aliya}, ne poss\`edent pas l'alv\'eolaire du toponyme K\=a\b li\index{gnl}{Kali@K\=a\b li}. % et que le titre de ce chapitre dans le \emph{talapur\=a\d nam}\index{gnl}{Purana@\textit{Pur\=a\d na}!\emph{talapur\=a\d nam}} est orthographi\'e ainsi: \og \'Sr\=\i k\=a\d lipuram\fg.
La d\'efaite de K\=al\=\i\index{gnl}{Kali@K\=al\=\i\ la déesse} lors de la comp\'etition de danse\index{gnl}{danse} \`a Citamparam\index{gnl}{Citamparam} n'est pas mentionn\'ee dans les \emph{talapur\=a\d nam}\index{gnl}{Purana@\textit{Pur\=a\d na}!\emph{talapur\=a\d nam}} principaux du site que sont le \textit{Cidambaram\=ah\=atmya}\index{gnl}{Cidambara@\textit{Cidambaram\=ah\=atmya}} (skt.) et le \textit{K\=oyi\b rpur\=a\d nam}\index{gnl}{Koyil@\textit{K\=oyi\b rpur\=a\d nam}} (tam.) mais dans une version sanskrite mineure, le \textit{Vy\=aghrapuram\=ah\=atmya}\index{gnl}{Vyaghrapuram@\textit{Vy\=aghrapuram\=ah\=atmya}} (\textsc{Smith} 1998: 143-145).
Dans le \textit{Hariva\d m\'sa}\index{gnl}{Harivamsa@\textit{Hariva\d m\'sa}} (chapitres 55 et 56), K\textsubring{r}\d s\d na\index{gnl}{Krsna@K\textsubring{r}\d s\d na}, apr\`es avoir dompt\'e le serpent\index{gnl}{serpent} K\=aliya\index{gnl}{Kaliya@K\=aliya}, le cong\'edie expier sa faute dans l'oc\'ean (\textsc{Couture} 1991: 218-226).

\item L'appellation de Ve\.nkuru\index{gnl}{Venkuru@Ve\.nkuru}, \og ma\^itre cruel\fg, r\'esulte de deux mythe\index{gnl}{mythe}s. Dans le premier, Ve\.nkuru\index{gnl}{Venkuru@Ve\.nkuru}, identifi\'e comme \'Sukr\=ac\=arya\index{gnl}{Sukracarya@\'Sukr\=ac\=arya}, attrist\'e par le manque de respect que les dieux lui portent parce qu'il est le ma\^itre des d\'emons, fait p\'enitence \`a C\=\i k\=a\b li\index{gnl}{Cikali@C\=\i k\=a\b li}. Dans le second, Ve\.nkuru\index{gnl}{Venkuru@Ve\.nkuru} est Yama\index{gnl}{Yama}. Il d\'ecide d'honorer \'Siva\index{gnl}{Siva@\'Siva} \`a C\=\i k\=a\b li\index{gnl}{Cikali@C\=\i k\=a\b li} pour que ce dernier accorde aux damn\'es la facult\'e de se souvenir de leurs bons et mauvais actes ant\'erieurs afin de comprendre leur sort aux enfers.

\item Pukali\index{gnl}{Pukali} est le \og refuge\index{gnl}{refuge}\fg\ des dieux qui y sont venus honorer \'Siva\index{gnl}{Siva@\'Siva} pour se prot\'eger du démon\index{gnl}{demon@démon} \'S\=urapadma\index{gnl}{Surapadma@\'S\=urapadma}. Dans le \textit{Kantapur\=a\d nam}\index{gnl}{Kantapuranam@\textit{Kantapur\=a\d nam}} (II.24 st.19-22 et III.30 st.1-9) Indra\index{gnl}{Indra} y est venu faire p\'enitence.

\item Le site obtient le nom de Cirapuram\index{gnl}{Cirapuram}, \og ville de la t\^ete\fg, parce que la t\^ete de R\=ahu\index{gnl}{Rahu@R\=ahu}, coup\'ee par le Soleil pour le punir d'avoir bu l'ambroisie\index{gnl}{ambroisie} du barattage destin\'ee aux dieux, est tomb\'ee en ce lieu.
%kampa r\=am\=ayanam, yuddha k\=anda
%bh\=agavata astama skandha

\item C\=\i k\=a\b li\index{gnl}{Cikali@C\=\i k\=a\b li} porte le nom de Ca\d npai\index{gnl}{Canpai@Ca\d npai}, d\'eriv\'e de \textit{ca\d npu} désignant une plante herbac\'ee. Selon le \emph{talapur\=a\d nam}\index{gnl}{Purana@\textit{Pur\=a\d na}!\emph{talapur\=a\d nam}}, le clan des Y\=adava\index{gnl}{Yadava@Y\=adava} voulant mettre \`a l'épreuve\index{gnl}{epreuve@épreuve} le sage\index{gnl}{sage} Kapilar\index{gnl}{Kapilar} lui pr\'esente un homme d\'eguis\'e en une femme\index{gnl}{femme} enceinte. Le sage\index{gnl}{sage}, en col\`ere, maudit le clan et fait na\^itre de l'homme un pilon. Les Y\=adava\index{gnl}{Yadava@Y\=adava} r\'eduisent l'objet en poudre. De chaque poussi\`ere pousse une plante \`a feuilles coupantes que les Y\=adava\index{gnl}{Yadava@Y\=adava} utilisent comme arme dans leur querelle intestine et s'entretuent. Cet épisode est narré dans le \textit{Mah\=abh\=arata}\index{gnl}{Mahabharata@\textit{Mah\=abh\=arata}}, livre 16 (\textsc{Vettam} *2002 [1975]: 890). K\textsubring{r}\d s\d na\index{gnl}{Krsna@K\textsubring{r}\d s\d na}, qui appartient \`a ce clan, veut \'echapper \`a la mal\'ediction et se rend \`a C\=\i k\=a\b li\index{gnl}{Cikali@C\=\i k\=a\b li} pour faire p\'enitence.


\item Le toponyme Koccai\index{gnl}{Koccai}, \og bassesse\fg, a pour origine\index{gnl}{origine} selon le \emph{talapur\=a\d nam}\index{gnl}{Purana@\textit{Pur\=a\d na}!\emph{talapur\=a\d nam}} l'humiliation subie par Par\=a\'sara\index{gnl}{Parasara@Par\=a\'sara} devant les autres sages \`a cause de son union avec Matsyagandh\=a\index{gnl}{Matsyagandh\=a}, \og celle \`a l'odeur de poisson\fg; cf. \textit{Mah\=abh\=arata}\index{gnl}{Mahabharata@\textit{Mah\=abh\=arata}}, livre 1, chapitre 57 (\textsc{Buitenen} 1971: 132-134). Par\=a\'sara\index{gnl}{Parasara@Par\=a\'sara} se rend \`a C\=\i k\=a\b li\index{gnl}{Cikali@C\=\i k\=a\b li} pour se purifier.
%d\=evi bh\=agavata skandha 2


\item Selon le \emph{talapur\=a\d nam}\index{gnl}{Purana@\textit{Pur\=a\d na}!\emph{talapur\=a\d nam}} le site est nomm\'e V\=e\d nupuram\index{gnl}{Venupuram@V\=e\d nupuram}, \og ville du bambou\index{gnl}{bambou}\fg, pour deux raisons. D'abord, parce que \'Siva\index{gnl}{Siva@\'Siva}, sous la forme d'un bambou\index{gnl}{bambou}, a accord\'e la force requise par le démon\index{gnl}{demon@démon} \'S\=urapadma\index{gnl}{Surapadma@\'S\=urapadma} pour combattre Indra\index{gnl}{Indra}. Ensuite, parce qu'Indra\index{gnl}{Indra} descend \`a C\=\i k\=a\b li\index{gnl}{Cikali@C\=\i k\=a\b li} en prenant l'apparence d'un bambou\index{gnl}{bambou} pour expier ses fautes, à savoir qu'il a n\'eglig\'e puis tu\'e ses ma\^itres.
Dans le \textit{Kantapur\=a\d nam}\index{gnl}{Kantapuranam@\textit{Kantapur\=a\d nam}}, Indra\index{gnl}{Indra} et sa femme\index{gnl}{femme} se transforment en bambou\index{gnl}{bambou} pour \'echapper au démon\index{gnl}{demon@démon} \'S\=urapadma\index{gnl}{Surapadma@\'S\=urapadma} (II.21 st.44-45). Ce texte d\'ecrit aussi le sacrifice conduit par \'S\=urapadma\index{gnl}{Surapadma@\'S\=urapadma} pour obtenir un don\index{gnl}{don} de \'Siva\index{gnl}{Siva@\'Siva} (II.8 et 9).

\item C\=\i k\=a\b li\index{gnl}{Cikali@C\=\i k\=a\b li} obtient le toponyme de Ka\b lumalam\index{gnl}{Kalumalam@Ka\b lumalam}, \og dont les péché\index{gnl}{peche@péché}s sont lav\'es\fg, parce que le sage\index{gnl}{sage} Roma\'sa\index{gnl}{Roma\'sa}, d\'esireux d'acqu\'erir la connaissance\index{gnl}{connaissance} de \'Siva\index{gnl}{Siva@\'Siva}, s'y rend et qu'il y jouit de l'enseignement de \'Siva\index{gnl}{Siva@\'Siva} qui lave ses péché\index{gnl}{peche@péché}s.

\item Pu\b ravam\index{gnl}{Puravam@Pu\b ravam}, \og pigeon\fg, vient de la mise \`a l'épreuve\index{gnl}{epreuve@épreuve} du roi\index{gnl}{roi} \'Sibi\index{gnl}{Sibi@\'Sibi} qui aurait eu lieu \`a C\=\i k\=a\b li\index{gnl}{Cikali@C\=\i k\=a\b li}. La g\'en\'erosit\'e du roi\index{gnl}{roi} est \'eprouv\'ee par Indra\index{gnl}{Indra} et Agni\index{gnl}{Agni} sur la demande de \'Siva\index{gnl}{Siva@\'Siva}. Indra\index{gnl}{Indra} sous la forme d'un aigle poursuit Agni\index{gnl}{Agni} qui a pris l'apparence d'un pigeon. Ce dernier prend refuge\index{gnl}{refuge} aupr\`es du roi\index{gnl}{roi} qui donne son propre corps \`a l'aigle pour sauver le pigeon\footnote{Ce mythe\index{gnl}{mythe} célèbre est conté, très brièvement, dans le \textit{Mah\=abh\=arata}\index{gnl}{Mahabharata@\textit{Mah\=abh\=arata}}, livre 3, chapitre 199 (\textsc{Buitenen} 1975: 623) et, en détail dans les récits des vies antérieures du Bouddha (\textsc{Cowell} *1999 [1990]). Les rois \textit{c\=o\b la}\index{gnl}{cola@\textit{c\=o\b la}} revendiquent leur descendance de ce roi\index{gnl}{roi} \'Sibi\index{gnl}{Sibi@\'Sibi} (voir les généalogies décrites dans les plaquettes de cuivre de Leiden\index{gnl}{Leiden} (EI 22 34 v. 4) et de Tiruv\=ala\.nk\=a\d tu\index{gnl}{Alankatu@\=Ala\.nk\=a\d tu!Tiruvalankatu@Tiruv\=ala\.nk\=a\d tu} (SII 3 205 v. 27) par exemple).}.
%(XXXXXBROCQUET)

\item Tar\=ay\index{gnl}{Taray@Tar\=ay}, \og mont(?)\fg, est selon le \emph{talapur\=a\d nam}\index{gnl}{Purana@\textit{Pur\=a\d na}!\emph{talapur\=a\d nam}} la ville o\`u Var\=aha\index{gnl}{Varaha@Var\=aha}, destructeur du démon\index{gnl}{demon@démon} Hira\d ny\=ak\d sa\index{gnl}{Hiranyaksa@Hira\d ny\=ak\d sa}, a honor\'e \'Siva\index{gnl}{Siva@\'Siva} apr\`es avoir terrifi\'e la terre\index{gnl}{terre} en la soulevant avec sa d\'efense.
%bh\=agavata skandha 3
%devii bh\=agavata skandha 8 et 9
%Agni\index{gnl}{Agni} pur\=ana chap 5
%padma pur\=ana bh\=umikanda chap 91
\end{enumerate}

Ce sont aujourd'hui ces versions des légende\index{gnl}{legende@légende}s qui font autorit\'e et qui sont retenues par les fid\`eles gr\^ace aux brochures par exemple\footnote{Ces brochures, publi\'ees par le \emph{t\=evast\=a\b nam} du temple\index{gnl}{temple}, sous le patronnage du monast\`ere\index{gnl}{monastère} de Tarumapuram\index{gnl}{Tarumapuram}, vendues une dizaine de roupies, sont accessibles \`a un grand nombre. Elles servent de guide aux fid\`eles et aux touristes. %Elles mettent visuellement en valeur le monast\`ere\index{gnl}{monastère} de Dharmapuram en pr\'esentant sur la premi\`ere page la photo du chef\index{gnl}{chef} religieux, une introduction r\'edig\'ee par un disciple renon\c cant (\emph{tampir\=a\b n}) et les éloge\index{gnl}{eloge@éloge}s sans cesse faits au monast\`ere\index{gnl}{monastère} pour ses actions g\'en\'ereuses.
La treizi\`eme r\'eimpression de la brochure du temple\index{gnl}{temple} de C\=\i k\=a\b li\index{gnl}{Cikali@C\=\i k\=a\b li}, \emph{C\=\i k\=a\b littalavaral\=a\b ru}, date de 2004. Elle comporte une pr\'esentation g\'en\'erale du temple\index{gnl}{temple} (son am\'enagement, ses divinit\'es majeures, ses inscriptions et ses rites\index{gnl}{rite}). On y retrouve aussi quelques mots sur le fonctionnement du temple\index{gnl}{temple} (l'administration, les propri\'et\'es et les templions des environs). Des r\'esum\'es des mythe\index{gnl}{mythe}s sont pr\'esents (les diff\'erents noms du site, les mythe\index{gnl}{mythe}s, les \emph{t\=\i rtha}, l'hagiographie\index{gnl}{hagiographie} du poète\index{gnl}{poete@poète} Campantar\index{gnl}{Campantar}). Sont inclus quelques extraits de poème\index{gnl}{poeme@poème}s attach\'es au temple\index{gnl}{temple}, ainsi que de nombreuses pages sur les hauts faits du monast\`ere\index{gnl}{monastère}.}. Mais nous constatons que dans la plupart des textes du \textit{Tirumu\b rai}\index{gnl}{Tirumurai@\textit{Tirumu\b rai}}, \`a l'exception des hymne\index{gnl}{hymne}s attribu\'es \`a Campantar\index{gnl}{Campantar}, ces versions sont inexistantes.

La traduction des hymne\index{gnl}{hymne}s \`a douze\index{gnl}{douze} noms attribu\'es \`a Campantar\index{gnl}{Campantar} nous permettra d'examiner le traitement particulier des légende\index{gnl}{legende@légende}s dans le \textit{T\=ev\=aram}\index{gnl}{Tevaram@\textit{T\=ev\=aram}} afin de souligner les probl\`emes rencontr\'es.

\section{Les hymnes aux douze noms}

Dans le \textit{T\=ev\=aram}\index{gnl}{Tevaram@\textit{T\=ev\=aram}}, le mention des douze\index{gnl}{douze} toponymes de C\=\i k\=a\b li\index{gnl}{Cikali@C\=\i k\=a\b li} se trouve uniquement dans les hymne\index{gnl}{hymne}s attribu\'es \`a Campantar\index{gnl}{Campantar}. Nous avons signal\'e au chapitre pr\'ec\'edent que ces douze\index{gnl}{douze} noms sont pr\'esent\'es dans un ordre\index{gnl}{ordre} d\'efini et qu'ils figurent toujours dans des poème\index{gnl}{poeme@poème}s \`a douze\index{gnl}{douze} strophes\footnote{\`A l'exception de I 128 qui est compos\'e en prose mais qui respecte l'ordre\index{gnl}{ordre} de pr\'esentation des toponymes.} qui sont tous compos\'es selon des procédé\index{gnl}{procédé littéraire}s litt\'eraires. Une traduction de ces onze hymne\index{gnl}{hymne}s nous paraît nécessaire pour étudier les douze noms.

Ce travail de traduction est le fruit des s\'eances de lecture effectu\'ees avec \textsc{T. V. Gopal Iyer} en 2004 et 2005. Il repose sur l'\'edition tamoule \'etablie par ce m\^eme pandit. Nous ne pr\'etendons pas rendre en fran\c cais tous les \'el\'ements lyriques et rh\'etoriques de ces poème\index{gnl}{poeme@poème}s mais cherchons \`a pr\'esenter, le plus fid\`element possible, leurs id\'ees.


\subsection{Hymne I 63}

Cet hymne\index{gnl}{hymne} est appel\'e \textit{palpeyarppattu}, \og dizain \`a plusieurs noms\fg, parce qu'il est ainsi d\'esign\'e dans l'envoi\index{gnl}{envoi} et parce qu'il renferme les douze\index{gnl}{douze} noms de C\=\i k\=a\b li\index{gnl}{Cikali@C\=\i k\=a\b li} accompagn\'es d'une allusion \`a leurs légende\index{gnl}{legende@légende}s respectives (sauf pour Ka\b lumalam\index{gnl}{Kalumalam@Ka\b lumalam} qui renvoie dans l'envoi\index{gnl}{envoi} \`a la ville d'origine\index{gnl}{origine} du poète\index{gnl}{poete@poète}). Ce poème\index{gnl}{poeme@poème} ne suit pas strictement la structure typique de Campantar\index{gnl}{Campantar} dans laquelle les quatre derniers quatrains ont une fonction propre (voir 2.1.1): le mythe\index{gnl}{mythe} de R\=ava\d na\index{gnl}{Ravana@R\=ava\d na} est absent et la critique des hérétique\index{gnl}{heretique@hérétique}s (st. 9) est plac\'ee avant la strophe consacr\'ee \`a la manifestation du \textit{li\.nga}\index{gnl}{linga@\textit{li\.nga}} de feu\index{gnl}{feu} (st. 11).

\scriptsize
\begin{verse}
\textit{eri \=ar ma\b lu o\b n\b ru \=enti, a\.nkai i\d tutalaiy\=e kala\b n\=a,\\
vari \=ar va\d laiy\=ar aiyam vavv\=ay, m\=a nalam vavvutiy\=e?---\\
cariy\=a n\=avi\b n v\=etak\=\i ta\b n, t\=amarai n\=a\b nmukatta\b n,\\
periy\=a\b n, pirama\b n p\=e\d ni \=a\d n\d ta piramapuratt\=a\b n\=e!} (I 63.1)\\
\end{verse}
\normalsize
\begin{verse}
Brandissant une hachette enflamm\'ee,\\
Ayant pour bol un cr\^ane plac\'e dans la paume,\\
Tu ne prends pas l'aum\^one de celles aux bracelets pleins de lignes,\\
Tu as pris leur grande vertu,\\
\^O Celui de Piramapuram\index{gnl}{Piramapuram} o\`u a r\'egn\'e avec plaisir Brahm\=a\index{gnl}{Brahma@Brahm\=a},\\
Celui au chant des \textit{Veda}\index{gnl}{Veda@\textit{Veda}} sur la langue qui ne faillit pas,\\
Celui \`a quatre t\^etes sur le lotus, le grand. (I 63.1)\\
\end{verse}


\scriptsize
\begin{verse}
\textit{piyal \=ar ca\d taikku \=or ti\.nka\d l c\=u\d ti, pey palikku e\b n\b ru, ayal\=e\\
kayal \=ar ta\d ta\.nka\d n am col nall\=ar ka\d n tuyil vavvutiy\=e?---\\
iyal\=al na\d t\=avi, i\b npam eyti, intira\b n \=a\d l ma\d nm\=el\\
viyal \=ar muracam \=o\.nku cemmai v\=e\d nupuratt\=a\b n\=e!} (I 63.2)\\
\end{verse}
\normalsize
\begin{verse}
Couronn\'e de la lune dans les m\`eches gorg\'ees d'eau\index{gnl}{eau}\footnote{Le terme \textit{piyal} qui signifie \og nuque, épaule\index{gnl}{epaule@épaule}\fg\ pose probl\`eme. \textsc{T. V. Gopal Iyer}, fid\`ele \`a son \'edition, propose de lire \og les m\`eches abondantes sur la nuque\fg. Cette image est inhabituelle dans le \textit{T\=ev\=aram}\index{gnl}{Tevaram@\textit{T\=ev\=aram}}. \textsc{V. M. Subramanya Iyar} corrige le terme et lit \textit{peyal}, \og nuage, pluie, eau\index{gnl}{eau}\fg. Nous pr\'ef\'erons cette seconde lecture qui est plus appropri\'ee \`a la description de la chevelure de \'Siva\index{gnl}{Siva@\'Siva} qui porte la Ga\.ng\=a\index{gnl}{Ganga@Ga\.ng\=a}.},\\
\lbrack Allant\rbrack\ dans le voisinage pour l'aum\^one qu'on sert,\\
Tu as pris le sommeil des yeux des vertueuses\\
Aux mots beaux et aux yeux longs tels des poissons \textit{kayal}, \\
Ayant gouvern\'e convenablement et ayant atteint le bonheur,\\
\^O Celui de la belle V\=e\d nupuram\index{gnl}{Venupuram@V\=e\d nupuram}\\
O\`u s'\'el\`eve le son des tambours pleins de grandeur\\
Sur la terre\index{gnl}{terre} o\`u r\`egne Indra\index{gnl}{Indra}. (I 63.2)\\
\end{verse}


\scriptsize
\begin{verse}
\textit{nakal\=artalaiyum ve\d npi\b raiyum na\d lirca\d taim\=a\d t\d tu, ayal\=e\\
pakal\=ap pali t\=erntu, aiyam vavv\=ay, p\=ay kalai vavvutiy\=e?---\\
akal\=atu u\b raiyum m\=a nilattil ayal i\b nmaiy\=al, amarar\\
pukal\=al malinta p\=um pukali\index{gnl}{Pukali} m\=eviya pu\d n\d niya\b n\=e!} (I 63.3)\\
\end{verse}
\normalsize
\begin{verse}
Ayant fix\'e dans les m\`eches mouill\'ees\\
Le cr\^ane rieur et le croissant blanc,\\
\lbrack Allant\rbrack\ dans le voisinage chercher l'aum\^one le jour,\\
Tu n'as pas pris l'aum\^one,\\
Tu as pris les v\^etements qui couvrent \lbrack les corps\rbrack, \\
\^O Vertueux qui r\'eside dans la belle Pukali\index{gnl}{Pukali}\\
Qui se d\'eveloppa par le refuge\index{gnl}{refuge} des immortels \\
Du fait qu'il n'y a pas d'autre place sur ce grand sol\\
Qui demeure sans fin. (I 63.3)\\
\end{verse}

\scriptsize
\begin{verse}
\textit{ca\.nk\=o\d tu ila\.nkat t\=o\d tu peytu, k\=atil or t\=a\b lku\b laiya\b n,\\
am k\=olva\d laiy\=ar aiyam vavv\=ay, \=aynalam vavvutiy\=e?---\\
ce\.nk\=ol na\d t\=avip paluyirkkum cey vi\b nai mey teriya,\\
ve\d n k\=ot taruma\b n m\=evi \=a\d n\d ta ve\.nkuru\index{gnl}{Venkuru@Ve\.nkuru} m\=eyava\b n\=e!} (I 63.4)\\
\end{verse}
\normalsize
\begin{verse}
Ayant plac\'e une boucle pour qu'elle brille avec la conque,\\
Ô Celui \`a la boucle qui pend sur une oreille,\\
Tu n'as pas pris l'aum\^one de celles aux bracelets larges et beaux,\\
Tu as pris leur belle vertu,\\
Ayant r\'egn\'e avec un sceptre juste\\
Pour que les nombreux \^etres connaissent\\
La v\'erit\'e de leur actions accomplies\\
\^O Celui qui r\'eside \`a Ve\.nkuru\index{gnl}{Venkuru@Ve\.nkuru}\\
O\`u a r\'egn\'e avec plaisir Dharma\index{gnl}{Dharma} au sceptre cruel. (I 63.4)\footnote{Nous ne comprenons pas si le sujet de l'absolutif \textit{na\d t\=avi} et de l'infinitif \textit{teriya} est \textit{taruma\b n} (Dharma\index{gnl}{Dharma}) ou \textit{m\=eyava\b n} (\'Siva\index{gnl}{Siva@\'Siva}).} \\
\end{verse}

\scriptsize
\begin{verse}
\textit{ta\d ni n\=\i r matiyam c\=u\d ti, n\=\i \d tu t\=a\.nkiya t\=a\b lca\d taiya\b n,\\
pi\d ni n\=\i r ma\d tav\=ar aiyam vavv\=ay, pey kalai vavvutiy\=e?-\\
a\d ni n\=\i r ulakam \=aki e\.nkum \=a\b lka\d tal\=al a\b lu\.nka,\\
tu\d ni n\=\i r pa\d niya, t\=a\b n mitanta t\=o\d nipuratt\=a\b n\=e!} (I 63.5)\\
\end{verse}
\normalsize
\begin{verse}
Couronn\'e de la lune et de l'eau\index{gnl}{eau} apais\'ee,\\
Ô Celui aux m\`eches pendantes portées en permanence,\\
Tu n'as pas pris l'aum\^one des femme\index{gnl}{femme}s aux humeurs amoureuses,\\
Tu as pris leurs v\^etements orn\'es,\\
Quand le monde s'orna d'eau\index{gnl}{eau}\\
Et souffrit partout \`a cause de la mer\index{gnl}{mer} profonde\\
\^O Celui de T\=o\d nipuram\index{gnl}{Tonipuram@T\=o\d nipuram}\\
Qui \'emergea quand l'eau\index{gnl}{eau} pure d\'ecrut. (I 63.5)\\
\end{verse}

\scriptsize
\begin{verse}
\textit{kavar p\=umpu\b nalum ta\d nmatiyum kama\b l ca\d taim\=a\d t\d tu, ayal\=e\\
avar p\=um paliy\=o\d tu aiyam vavv\=ay, \=aynalam vavvutiy\=e?---\\
avar p\=u\d n araiyarkku \=ati \=aya a\d tal ma\b n\b na\b n \=a\d l ma\d nm\=el\\
tavar p\=um patika\d l e\.nkum \=o\.nkum ta\.nku tar\=ayava\b n\=e!} (I 63.6)\\
\end{verse}
\normalsize
\begin{verse}
Ayant fix\'e dans les m\`eches parfum\'ees\\
La fra\^iche lune et l'eau\index{gnl}{eau} parfum\'ee qui charme,\\
\lbrack Allant\rbrack\ dans le voisinage,\\
Tu n'as pas pris l'aum\^one avec leur offrande de fleur,\\
Tu as pris leur belle vertu,\\
Sur la terre\index{gnl}{terre} o\`u a r\'egn\'e le roi\index{gnl}{roi} victorieux,\\
Le premier des rois par\'es de leur ornement \lbrack respectif\rbrack,\\
\^O Celui de l'\'eternelle Tar\=ay\index{gnl}{Taray@Tar\=ay} \\
O\`u s'\'el\`event partout les beaux temples\index{gnl}{temple} des asc\`etes. (I 63.6)\\
\end{verse}

\scriptsize
\begin{verse}
\textit{mulaiy\=a\b l ke\b luma, montai ko\d t\d ta, mu\b nka\d taim\=a\d t\d tu ayal\=e,\\
nilaiy\=ap pali t\=erntu, aiyam vavv\=ay, n\=\i nalam vavvutiy\=e?---\\
talai \=ayk ki\d tantu iv vaiyamell\=am ta\b natu \=or \=a\d nai na\d t\=ay,\\
cilaiy\=al malinta c\=\i rc cilampa\b n\index{gnl}{Cilampan@Cilampa\b n} cirapuram\index{gnl}{Cirapuram} m\=eyava\b n\=e!} (I 63.7)\\
\end{verse}
\normalsize
\begin{verse}
Quand le \textit{y\=a\b l} de poitrine jouait,\\
Quand le tambour (\`a une face) frappait,\\
\lbrack Allant\rbrack\ dans le voisinage,\\
Chercher offrande \`a l'entr\'ee, debout,\\
Tu n'as pas pris l'aum\^one \\
Tu as pris, toi, leur vertu,\\
\^O Celui qui r\'eside \`a Cirapuram\index{gnl}{Cirapuram} \\
De Cilampa\b n\index{gnl}{Cilampan@Cilampa\b n} \`a la gloire \'etendue par [son] arc,\\
Qui \'etant une t\^ete pla\c ca sous son autorit\'e tout ce monde. (I 63.7)\\
\end{verse}

\scriptsize
\begin{verse}
\textit{"erut\=e ko\d narka!" e\b n\b ru \=e\b ri, a\.nkai i\d tu talaiy\=e kala\b n\=a,\\
karutu \=er ma\d tav\=ar aiyam vavv\=ay, ka\d n tuyil vavvutiy\=e?---\\
oru t\=er ka\d t\=avi \=ar amaru\d l orupatut\=er tolaiyap\\
poru t\=er valava\b n m\=evi \=a\d n\d ta pu\b ravu amar pu\d n\d niya\b n\=e!} (I 63.8)\\
\end{verse}
\normalsize
\begin{verse}
Ayant mont\'e [le taureau\index{gnl}{taureau}] en disant : " \^o taureau\index{gnl}{taureau} avance!\\
Et ayant pour bol un cr\^ane plac\'e dans la paume,\\
Tu n'as pas pris l'aum\^one des belles femme\index{gnl}{femme}s d\'esireuses,\\
Tu as pris le sommeil de leurs yeux,\\
\^O Vertueux qui r\'eside \`a Pu\b ravam\index{gnl}{Puravam@Pu\b ravam}\\
O\`u a r\'egn\'e avec plaisir le vaillant au char de combat\\
Qui conduisant un char\\
A d\'etruit une dizaine de chars dans la guerre cruelle. (I 63.8)\\
\end{verse}

\scriptsize
\begin{verse}
\textit{tuvar c\=er kali\.nkapp\=orvaiy\=arum, t\=uymai il\=ac cama\d num,\\
kavarceytu u\b lavak ka\d n\d ta va\d n\d nam, k\=arikai v\=arku\b lal\=ar-\\
avar p\=um paliy\=o\d tu aiyam vavv\=ay, \=aynalam vavvutiy\=e?---\\
tavarcey ne\d tuv\=el ca\d n\d ta\b n \=a\d lac ca\d npai\index{gnl}{Canpai@Ca\d npai} amarntava\b n\=e!} (I 63.9)\\
\end{verse}
\normalsize
\begin{verse}
Ceux couverts de v\^etements jaunes (les bouddhiste\index{gnl}{bouddhiste}s)\\
Et les ja\"in\index{gnl}{jain@ja\"in}s sans puret\'e\\
Dans la mesure o\`u [tu les] as vu errer commettant des fautes,\\
Tu n'as pas pris l'aum\^one et les offrandes de fleurs\\
Des femme\index{gnl}{femme}s aux longs cheveux,\\
Tu as pris leur belle vertu, \\
\^O Celui qui s'est install\'e \`a Ca\d npai\index{gnl}{Canpai@Ca\d npai} \\
Quand r\'egnait Ca\d n\d ta\b n\index{gnl}{Cantan@Ca\d n\d ta\b n} \\
\`A la longue lance faite par des asc\`etes. (I 63.9)\\
\end{verse}

\scriptsize
\begin{verse}
\textit{ni\b lal\=al malinta ko\b n\b rai c\=u\d ti, n\=\i \b ru mey p\=uci, nalla\\
ku\b lal \=ar ma\d tav\=ar aiyam vavv\=ay, k\=olva\d lai vavvutiy\=e?---\\
a\b lal\=ay ulakam kavvai t\=\i ra, aintalai n\=\i \d l mu\d tiya\\
ka\b lal n\=aka (a)raiya\b n k\=aval \=akak k\=a\b li amarnta\b n\=e!} (I 63.10)\\
\end{verse}
\normalsize
\begin{verse}
Couronn\'e de la fleur de cassier pleine de brillance,\\
Ayant enduit le corps de cendre\index{gnl}{cendre}s,\\
Tu n'as pas pris l'aum\^one des femme\index{gnl}{femme}s \`a la belle chevelure,\\
Tu as pris leur larges bracelets,\\
\^O Celui qui s'est install\'e \`a K\=a\b li\index{gnl}{Kali@K\=a\b li}\\
Quand, pour d\'etruire la calamit\'e du monde en feu\index{gnl}{feu},\\
Devint gardien le roi\index{gnl}{roi} des serpent\index{gnl}{serpent}s aux anneaux\\
Et au capuchon haut de cinq t\^etes. (I 63.10)\\
\end{verse}

\scriptsize
\begin{verse}
\textit{ka\d t\d tu \=ar tu\b l\=aya\b n, t\=amaraiy\=a\b n, e\b n\b ru ivar k\=a\d npu ariya\\
ci\d t\d t\=ar pali t\=erntu, aiyam vavv\=ay, cey kalai vavvutiy\=e?---\\
na\d t\d t\=ar na\d tuv\=e nanta\b n \=a\d la, nalvi\b naiy\=al uyarnta\\
ko\d t\d t\=a\b ru u\d tutta ta\d nvayal c\=u\b l koccai\index{gnl}{Koccai} amarntava\b n\=e!} (I 63.11)\\
\end{verse}
\normalsize
\begin{verse}
Celui à la guirlande\index{gnl}{guirlande} pourvue de basilic [Vi\d s\d nu] et Celui du lotus [Brahm\=a],\\
De mani\`ere \`a ce qu'ils sachent, sans voir;\\
Ayant cherch\'e des offrandes pleines de grandeur,\\
Tu n'as pas pris l'aum\^one,\\
Tu as pris leur v\^etement port\'e,\\
\^O Celui qui s'est install\'e \`a Koccai\index{gnl}{Koccai},\\
Entour\'ee de rizi\`eres fra\^iches,\\
Habill\'ee de la rivi\`ere Ko\d t\d tu qui cro\^it par ses bons actes,\\
Quand Nanta\b n\index{gnl}{Nantan@Nanta\b n} r\'egnait au milieu d'amis. (I 63.11)\\
\end{verse}

\scriptsize
\begin{verse}
\textit{ka\d tai \=ar ko\d ti nal m\=a\d ta v\=\i tik ka\b lumala \=urk kavu\d ni---\\
na\d tai \=ar pa\b nuvalm\=alai \=aka ñ\=a\b nacampanta\b n---nalla\\
pa\d tai \=ar ma\b luva\b nm\=el mo\b linta palpeyarppattum vall\=arkku\\
a\d taiy\=a, vi\b naika\d l ulakil n\=a\d lum; amarulaku \=a\d lpavar\=e.} (I 63.12)\\
\end{verse}
\normalsize
\begin{verse}
Pour ceux qui sont forts dans le dizain aux diff\'erents noms\\
Dit sur Celui \`a la hache de combat,\\
En tant que guirlande\index{gnl}{guirlande} de m\`etres au bon style,\\
Par \~N\=a\b nacampanta\b n\index{gnl}{Campantar!N\=a\b nacampanta\b n@\~N\=a\b nacampanta\b n},\\
Le \textit{kavu\d ni}\index{gnl}{kaundinya@\textit{kau\d n\d dinya}!\textit{kavu\d ni}} de la ville de Ka\b lumalam\index{gnl}{Kalumalam@Ka\b lumalam}\\
Aux rues pourvues de belles maisons avec des drapeaux \`a l'entr\'ee, \\
La [reconduite] des actions ne les atteindra jamais dans ce monde,\\
Ils r\'egneront sur le monde les immortels. (I 63.12)\\
\end{verse}


\subsection{Hymne I 90}

Le poème\index{gnl}{poeme@poème} est compos\'e selon le procédé\index{gnl}{procédé littéraire} litt\'eraire de l'\textit{irukkukku\b ra\d l}\index{gnl}{irukkukkural@\textit{irukkukku\b ra\d l}}, \og distique \textsubring{r}gv\'edique\fg\ (voir 2.1.3). Il n'y a pas de r\'ef\'erence aux légende\index{gnl}{legende@légende}s. Seuls quatre toponymes sont pr\'esent\'es selon leur \'etymologie: Piramapuram\index{gnl}{Piramapuram} (st. 1), V\=e\d nupuram\index{gnl}{Venupuram@V\=e\d nupuram} (st. 2), Pukali\index{gnl}{Pukali} (st. 3) et T\=o\d nipuram\index{gnl}{Tonipuram@T\=o\d nipuram} (st. 5).

\scriptsize
\begin{verse}
\textit{ara\b nai u\d lkuv\=\i r! pirama\b n\=uru\d l em\\
para\b naiy\=e ma\b nam paravi, uymmi\b n\=e!} (I 90.1)\\
\end{verse}
\normalsize
\begin{verse}
\^O vous qui méditez sur Hara\index{gnl}{Hara}!\\
Dans la ville de Brahm\=a\index{gnl}{Brahma@Brahm\=a} (Piramapuram\index{gnl}{Piramapuram}),\\
N'honorant de [tout] c\oe ur que notre Sup\'erieur, \\
Lib\'erez-vous! (I 90.1)\\
\end{verse}

\scriptsize
\begin{verse}
\textit{k\=a\d na u\d lkuv\=\i r! v\=e\d nunalpurat\\
t\=a\d nuvi\b n ka\b lal p\=e\d ni, uymmi\b n\=e!} (I 90.2)\\
\end{verse}
\normalsize
\begin{verse}
\^O vous qui méditez pour [le] voir!\\
Ayant aim\'e les [Pieds aux] anneaux de cheville du Stable\\
De la bonne ville de bambou\index{gnl}{bambou} (V\=e\d nupuram\index{gnl}{Venupuram@V\=e\d nupuram}), \\
Lib\'erez-vous! (I 90.2)\\
\end{verse}

\scriptsize
\begin{verse}
\textit{n\=ata\b n e\b npirk\=a\d l! k\=atal o\d n pukal\\
\=atip\=atam\=e \=oti, uymmi\b n\=e!} (I 90.3)
\end{verse}
\normalsize
\begin{verse}
\^O vous qui dites \og seigneur\fg!\\
Ayant chant\'e les Pieds premiers\\
Du brillant refuge\index{gnl}{refuge} (Pukali\index{gnl}{Pukali}) d'amour,\\
Lib\'erez-vous ! (I 90.3)\\
\end{verse}

\scriptsize
\begin{verse}
\textit{a\.nkam m\=atu c\=er pa\.nkam\=ayava\b n,\\
ve\.nkuru\index{gnl}{Venkuru@Ve\.nkuru} ma\b n\b num e\.nka\d l \=\i ca\b n\=e.} (I 90.4)\\
\end{verse}
\normalsize
\begin{verse}
Celui devenu moiti\'e qui rejoint la femme\index{gnl}{femme} sur le corps\\
Est notre Seigneur qui r\'eside \`a Ve\.nkuru\index{gnl}{Venkuru@Ve\.nkuru}. (I 90.4)\\
\end{verse}

\scriptsize
\begin{verse}
\textit{v\=a\d lnil\=ac ca\d tait t\=o\d niva\d npurattu\\
\=a\d nina\b npo\b naik k\=a\d numi\b nka\d l\=e!} (I 90.5)\\
\end{verse}
\normalsize
\begin{verse}
Voyez le bel étalon d'or (\'Siva\index{gnl}{Siva@\'Siva})\\
Aux m\`eches [orn\'ees] de la lune \'eclatante\\
De la ville fertile du radeau\index{gnl}{radeau} (T\=o\d nipuram\index{gnl}{Tonipuram@T\=o\d nipuram}). (I 90.5)\\
\end{verse}

\scriptsize
\begin{verse}
\textit{\og p\=anta\d l \=ar ca\d taip p\=untar\=ay\index{gnl}{Taray@Tar\=ay} ma\b n\b num,\\
\=entu ko\.nkaiy\=a\d l v\=enta\b n\fg\ e\b npar\=e.} (I 90.6)\\
\end{verse}
\normalsize
\begin{verse}
\og Le roi\index{gnl}{roi}, aux m\`eches pleines de serpent\index{gnl}{serpent}s,\\
De Celle \`a la poitrine abondante\\
R\'eside dans la belle Tar\=ay\index{gnl}{Taray@Tar\=ay}\fg\ dit-on. (I 90.6)\\
\end{verse}

\scriptsize
\begin{verse}
\textit{kariya ka\d n\d ta\b nai, cirapurattu\d l em\\
aracai, n\=a\d lto\b rum paravi, uymmi\b n\=e!} (I 90.7)\\
\end{verse}
\normalsize
\begin{verse}
Ayant honor\'e tous les jours\\
Celui \`a la gorge tach\'ee,\\
Notre Seigneur dans Cirapuram\index{gnl}{Cirapuram},\\
Lib\'erez-vous! (I 90.7)\\
\end{verse}

\scriptsize
\begin{verse}
\textit{na\b ravam \=ar po\b lil pu\b ravam\index{gnl}{Puravam@Pu\b ravam} nal pati\\
i\b raiva\b n n\=amam\=e ma\b raval, neñcam\=e!} (I 90.8)\\
\end{verse}
\normalsize
\begin{verse}
\^O c\oe ur! N'oublie pas le nom du Seigneur\\
De la belle ville de Pu\b ravam\index{gnl}{Puravam@Pu\b ravam}\\
Aux jardins pleins de miel. (I 90.8)\\
\end{verse}

\scriptsize
\begin{verse}
\textit{te\b n\b ril arakka\b naik ku\b n\b ril ca\d npai\index{gnl}{Canpai@Ca\d npai} ma\b n\\
a\b n\b ru nerittav\=a, ni\b n\b ru ni\b naimi\b n\=e!} (I 90.9)\\
\end{verse}
\normalsize
\begin{verse}
Pensez, debout, au fait que\\
Le Seigneur de Ca\d npai\index{gnl}{Canpai@Ca\d npai}, sur la montagne,\\
A \'ecras\'e, jadis,\\
Le démon\index{gnl}{demon@démon} du Sud (R\=ava\d na\index{gnl}{Ravana@R\=ava\d na}). (I 90.9)\\
\end{verse}

\scriptsize
\begin{verse}
\textit{aya\b num m\=alum\=ay muyalum k\=a\b liy\=a\b n\\
peyalvai eyti ni\b n\b ru iyalum, u\d l\d lam\=e.} (I 90.10)\\
\end{verse}
\normalsize
\begin{verse}
\^O for int\'erieur qui demeure, restant,\\
\lbrack Bien qu'\rbrack ayant obtenu la pluie [de gr\^ace]\\
De Celui de K\=a\b li\index{gnl}{Kali@K\=a\b li} \\
Qu'Aya\b n\index{gnl}{Brahma@Brahm\=a!Aya\b n} et M\=al\index{gnl}{Visnu@Vi\d s\d nu!Mal@M\=al} pers\'ev\`erent [\`a trouver]. (I 90.10)\\
\end{verse}

\scriptsize
\begin{verse}
\textit{t\=erar ama\d naraic c\=ervu il koccai\index{gnl}{Koccai} ma\b n\\
n\=er il ka\b lal ni\b naintu \=orum, u\d l\d lam\=e.} (I 90.11)\\
\end{verse}
\normalsize
\begin{verse}
\^O for int\'erieur qui s'unit\\
En pensant aux [Pieds aux] anneaux de cheville,\\
Sans \'egal, du seigneur de Koccai\index{gnl}{Koccai}\\
Sans lien avec les bouddhiste\index{gnl}{bouddhiste}s et les ja\"in\index{gnl}{jain@ja\"in}s. (I 90.11)\\
\end{verse}

\scriptsize
\begin{verse}
\textit{to\b lu ma\b nattavar, ka\b lumalattu u\b rai\\
pa\b lutu il campanta\b n mo\b lika\d lpattum\=e.} (I 90.12)\\
\end{verse}
\normalsize
\begin{verse}
Ceci est le dizain de mots\\
De Campanta\b n\index{gnl}{Campantar!Campanta\b n}, sans d\'efaut, \\
Qui chanta Ka\b lumalam\index{gnl}{Kalumalam@Ka\b lumalam} \\
De ceux \`a l'esprit qui honore. (I 90.12)\\
\end{verse}

\subsection{Hymne I 117}

Les strophes de ce poème\index{gnl}{poeme@poème}, \`a l'exception de l'envoi\index{gnl}{envoi}, sont compos\'ees selon le procédé\index{gnl}{procédé littéraire} litt\'eraire du \textit{mo\b lim\=a\b r\b ru}\index{gnl}{molimarru@\textit{mo\b lim\=a\b r\b ru}}, \og \'echange de mots\fg\ (voir 2.1.3). Il n'y aucune allusion aux légende\index{gnl}{legende@légende}s qui justifient les douze\index{gnl}{douze} noms de la ville.

\scriptsize
\begin{verse}
\textit{k\=a\d tu atu, a\d nikalam k\=ar aravam, pati; k\=al ata\b nil,-\\
t\=o\d tu atu a\d nikuvar cuntarak k\=ati\b nil, t\=uc cilampar;\\
v\=e\d tu atu a\d nivar, vicaya\b rku, uruvam, villum ko\d tuppar;-\\
p\=\i \d tu atu a\d ni ma\d ni m\=a\d tap piramapurattu arar\=e.} (I 117.1)\footnote{Cf. notre explication du mode de fonctionnement de cette strophe initiale dans le deuxi\`eme chapitre (2.1.3).}\\
\end{verse}
\normalsize
\begin{verse}
La demeure est le bois (cr\'ematoire), \\
L'ornement le serpent\index{gnl}{serpent} noir;\\
Celui aux anneaux purs aux pieds\\
Porte une boucle \`a la belle oreille,\\
Porte la forme du chasseur \\
Et donne l'arc \`a Vijaya (Arjuna\index{gnl}{Arjuna});\\
\^O Hara\index{gnl}{Hara} de Piramapuram\index{gnl}{Piramapuram}\\
Aux maisons gemm\'ees pourvues de grandeur! (I 117.1) \\
\end{verse}

\scriptsize
\begin{verse}
\textit{ka\b r\b raicca\d taiyatu, ka\.nka\d nam mu\b nkaiyil-ti\.nka\d l ka\.nkai;\\
pa\b r\b rittu, muppuram, p\=ar pa\d taitt\=o\b n talai, cu\d t\d tatu pa\d n\d tu;\\
e\b r\b rittu, p\=ampai a\d nintatu, k\=u\b r\b rai;-e\b lil vi\d la\.nkum\\
ve\b r\b ric cilaimatil v\=e\d nupurattu e\.nka\d l v\=etiyar\=e.} (I 117.2)\\
\end{verse}
\scriptsize
\begin{verse}
\textit{ti\.nka\d l ka\.nkai ka\b r\b raicca\d taiyatu,\\
ka\.nka\d nam mu\b nkaiyil,\\
p\=ar pa\d taitt\=o\b n talai pa\b r\b rittu,\\
muppuram pa\d n\d tu cu\d t\d tatu,\\
p\=ampai a\d nintatu,\\
k\=u\b r\b rai e\b r\b rittu,\\
e\b lil vi\d la\.nkum ve\b r\b ric cilaimatil\\
v\=e\d nupurattu e\.nka\d l v\=etiyar\=e.} (I 117.2)\\
\end{verse}
\normalsize
\begin{verse}
Les m\`eches sont regroup\'ees [avec] la lune et Ga\.ng\=a\index{gnl}{Ganga@Ga\.ng\=a},\\
Un bracelet sur l'avant-bras,\\
Ayant pris la t\^ete de Celui qui cr\'ea la terre\index{gnl}{terre} (Brahm\=a),\\
Ayant br\^ul\'e jadis les trois citadelles,\\
Portant le serpent\index{gnl}{serpent},\\
Il frappa Yama\index{gnl}{Yama};\\
\^O notre V\'edisant de V\=e\d nupuram\index{gnl}{Venupuram@V\=e\d nupuram}\\
Aux fortifications de pierre\\
\`A la victoire \'eclatante de beaut\'e. (I 117.2)\\
\end{verse}

\scriptsize
\begin{verse}
\textit{k\=uvi\d lam, kaiyatu p\=eri, ca\d taimu\d tik k\=u\d t\d tattatu;\\
t\=u vi\d la\.nkum po\d ti, p\=u\d n\d tatu, p\=uci\b r\b ru, tuttin\=akam;\\
\=e vi\d la\.nkum nutal, \=a\b naiyum, p\=akam, uritta\b nar;-i\b n\\
p\=u i\d lañ c\=olaip pukaliyu\d l m\=evi pu\d n\d niyar\=e.} (I 117.3)\\
\end{verse}
\scriptsize
\begin{verse}
\textit{k\=uvi\d lam ca\d taimu\d tik k\=u\d t\d tattatu,\\
kaiyatu p\=eri,\\
t\=u vi\d la\.nkum po\d ti p\=uci\b r\b ru,\\
tuttin\=akam p\=u\d n\d tatu,\\
\=e vi\d la\.nkum nutal p\=akam,\\
\=a\b naiyum uritta\b nar,\\
i\b n p\=u i\d lañ c\=olaip pukaliyu\d l m\=evi pu\d n\d niyar\=e.} (I 117.3)\\
\end{verse}
\normalsize
\begin{verse}
Les [feuilles de] \textit{k\=uvi\d lam} en groupe dans les cheveux en m\`eches,\\
Un tambour \`a la main,\\
Enduit de la cendre\index{gnl}{cendre} qui brille de puret\'e,\\
Portant le serpent\index{gnl}{serpent} \`a capuchon,\\
Moiti\'e de Celle au front courb\'e comme un arc,\\
Il d\'epouilla m\^eme l'\'el\'ephant; \\
\^O le Vertueux qui habita dans Pukali\index{gnl}{Pukali} \\
Aux jardins de jeunes arbres aux fleurs miellées. (I 117.3)\\
\end{verse}

\scriptsize
\begin{verse}
\textit{urittatu, p\=ampai u\d talmicai i\d t\d tatu, \=or o\d n ka\d li\b r\b rai;\\
erittatu, or \=amaiyai i\b np\=u\b rap p\=u\d n\d tatu, muppurattai;\\
ceruttatu, c\=ulattai \=enti\b r\b ru, takka\b nai v\=e\d lvi;-pal-n\=ul\\
virittavar v\=a\b ltaru ve\.nkuruvil v\=\i \b r\b riruntavar\=e.} (I 117.4)\\
\end{verse}
\scriptsize
\begin{verse}
\textit{\=or o\d n ka\d li\b r\b rai urittatu,\\
p\=ampai u\d talmicai i\d t\d tatu,\\
muppurattai erittatu,\\
or \=amaiyai i\b np\=u\b rap p\=u\d n\d tatu,\\
takka\b nai v\=e\d lvi ceruttatu,\\
c\=ulattai \=enti\b r\b ru,\\
pal n\=ul virittavar v\=a\b ltaru ve\.nkuruvil v\=\i \b r\b riruntavar\=e.} (I 117.4)\\
\end{verse}
\normalsize
\begin{verse}
Un \'el\'ephant excellent a \'et\'e d\'epouill\'e;\\
Le serpent\index{gnl}{serpent} est port\'e sur le corps;\\
Les trois citadelles ont \'et\'e consum\'ees;\\
Une tortue est port\'ee avec plaisir;\\
Dak\d sa est d\'etruit dans le sacrifice;\\
Brandissant le trident,\\
\^O Celui qui est \'eminent \`a Ve\.nkuru\index{gnl}{Venkuru@Ve\.nkuru}\\
O\`u vivent ceux qui ont expos\'e divers textes. (I 117.4)\\
\end{verse}

\scriptsize
\begin{verse}
\textit{ko\d t\d tuvar, akku arai \=arppatu, takkai; ku\b runt\=a\d la\b n\\
i\d t\d tuvar p\=utam, kalappu ilar, i\b npuka\b l, e\b npu; ulavi\b n\\
ma\d t\d tu varum ta\b lal, c\=u\d tuvar mattamum, \=entuvar;-v\=a\b n\\
to\d t\d tu varum ko\d tit t\=o\d nipurattu u\b rai cuntarar\=e.} (I 117.5)\\
\end{verse}
\scriptsize
\begin{verse}
\textit{takkai ko\d t\d tuvar,\\
akku arai \=arppatu,\\
ku\b runt\=a\d la\b n p\=utam i\d t\d tuvar,\\
i\b npuka\b l kalappu ilar,\\
e\b npu mattamum c\=u\d tuvar,\\
ulavi\b n ma\d t\d tu varum ta\b lal \=entuvar,\\
v\=a\b n to\d t\d tu varum ko\d tit t\=o\d nipurattu u\b rai cuntarar\=e.} (I 117.5)\\
\end{verse}
\normalsize
\begin{verse}
Il frappe le tambour,\\
Il porte des graines \`a la taille,\\
Il a des gnomes \`a petites jambes,\\
Il est sans alt\'eration dans la belle gloire,\\
Il se couronne d'os et de la fleur de datura,\\
Quand il marche il brandit le feu\index{gnl}{feu} parfum\'e (?),\\
\^O le Magnifique qui vit \`a T\=o\d nipuram\index{gnl}{Tonipuram@T\=o\d nipuram}\\
Aux drapeaux qui viennent touchant le ciel. (I 117.5)\\
\end{verse}

\scriptsize
\begin{verse}
\textit{c\=attuvar, p\=acam ta\d takkaiyil \=entuvar, k\=ova\d nam; tam\\
k\=uttu, avar, kaccuk kulavi ni\b n\b ru, \=a\d tuvar; kokku i\b rakum,\\
p\=erttavar palpa\d tai p\=eyavai,c\=u\d tuvar; p\=er e\b lil\=ar;-\\
p\=uttavar kaito\b lu p\=untar\=ay\index{gnl}{Taray@Tar\=ay} m\=eviya pu\d n\d niyar\=e.} (I 117.6)\\
\end{verse}
\scriptsize
\begin{verse}
\textit{k\=ova\d nam c\=attuvar,\\
p\=acam ta\d takkaiyil \=entuvar,\\
kaccuk kulavi ni\b n\b ru tam k\=uttu avar \=a\d tuvar,\\
kokku i\b rakum c\=u\d tuvar,\\
palpa\d tai p\=eyavai p\=erttavar,\\
p\=er e\b lil\=ar,\\
p\=uttavar kaito\b lu p\=untar\=ay\index{gnl}{Taray@Tar\=ay} m\=eviya pu\d n\d niyar\=e.} (I 117.6)\\
\end{verse}
\normalsize
\begin{verse}
Il porte un cache-sexe,\\
Il brandit dans [sa] large main le lasso,\\
Portant une ceinture il ex\'ecute sa danse\index{gnl}{danse},\\
Il se couronne m\^eme de la plume de la grue\footnote{Autre lecture possible: \og Il se couronne m\^eme de la fleur \textit{kokki\b raku}\fg.},\\
Il dirige plusieurs arm\'ees de fant\^omes,\\
Il est d'une grande beaut\'e,\\
\^O le Vertueux qui vit dans la belle Tar\=ay\index{gnl}{Taray@Tar\=ay}\\
Que v\'en\`erent des mains les habitants de la terre\index{gnl}{terre}. (I 117.6)\\
\end{verse}

\scriptsize
\begin{verse}
\textit{k\=alatu, ka\.nkai ka\b r\b raicca\d taiyu\d l\d l\=al, ka\b lal cilampu;\\
m\=alatu, \=ental ma\b luatu, p\=akam; va\d lar ko\b lu\d n k\=o\d t\d tu\\
\=al atu, \=urvar a\d tal \=e\b r\b ru, iruppar;-a\d ni ma\d nin\=\i rc\\
c\=el atuka\d n\d ni orpa\.nkar cirapuram\index{gnl}{Cirapuram} m\=eyavar\=e.} (I 117.7)\\
\end{verse}
\scriptsize
\begin{verse}
\textit{ka\b lal cilampu k\=alatu,\\
ka\.nkai ka\b r\b raicca\d taiyu\d l\d l\=al,\\
p\=akam m\=alatu,\\
ma\b luatu \=ental,\\
va\d lar ko\b lu\d n k\=o\d t\d tu \=al atu iruppar,\\
a\d tal \=e\b r\b ru \=urvar,\\
a\d ni ma\d nin\=\i rc c\=el atuka\d n\d ni orpa\.nkar cirapuram\index{gnl}{Cirapuram} m\=eyavar\=e.} (I 117.7)\\
\end{verse}
\normalsize
\begin{verse}
Il porte aux pieds \textit{ka\b lal} et \textit{cilampu}\footnote{\textit{ka\b lal} et \textit{cilampu} sont des anneaux de cheville port\'es, respectivement, par des hommes et des femme\index{gnl}{femme}s.},\\
Il a dans [ses] m\`eches en touffe Ga\.ng\=a\index{gnl}{Ganga@Ga\.ng\=a},\\
Sa moiti\'e est M\=al\index{gnl}{Visnu@Vi\d s\d nu!Mal@M\=al},\\
Il porte la hache,\\
Il est dans le banyan aux branches fertiles et croissantes,\\
Il monte le taureau\index{gnl}{taureau} puissant,\\
Celui qui vit \`a Cirapuram\index{gnl}{Cirapuram} est la moiti\'e\\
De Celle aux yeux tels les poissons \textit{c\=el}\\
Des eaux\index{gnl}{eau} [couleur] d'un beau saphir. (I 117.7)\\
\end{verse}

%atu (7c) locatif, atu (7d) outil de comparaison.

\scriptsize
\begin{verse}
\textit{neruppu uru, ve\d lvi\d tai, m\=e\b niyar,\=e\b ruvar; ne\b r\b riyi\b nka\d n,\\
maruppu uruva\b n, ka\d n\d nar, t\=ataiyaik k\=a\d t\d tuvar; m\=a muruka\b n\index{gnl}{Murukan@Muruka\b n}\\
viruppu u\b ru, p\=ampukku mey, tantaiy\=ar;-vi\b ral m\=a tavar v\=a\b l\\
poruppu u\b ru m\=a\d likait te\b npu\b ravattu a\d ni pu\d n\d niyar\=e.} (I 117.8)\\
\end{verse}
\scriptsize
\begin{verse}
\textit{neruppu uru m\=e\b niyar,\\
ve\d lvi\d tai \=e\b ruvar,\\
ne\b r\b riyi\b nka\d n ka\d n\d nar,\\
maruppu uruva\b n t\=ataiyai,\\
m\=a muruka\b n\index{gnl}{Murukan@Muruka\b n} viruppu u\b ru tantaiy\=ar,\\
p\=ampukku mey k\=a\d t\d tuvar,\\
vi\b ral m\=a tavar v\=a\b l poruppu u\b ru m\=a\d likait te\b npu\b ravattu a\d ni pu\d n\d niyar\=e.} (I 117.8)\\
\end{verse}
\normalsize
\begin{verse}
Son corps a la couleur du feu\index{gnl}{feu},\\
Il monte le taureau\index{gnl}{taureau} blanc,\\
Il a un oeil sur le front,\\
Il est le père\index{gnl}{pere@père} de Celui \`a la forme de l'\'el\'ephant,\\
Il est le père\index{gnl}{pere@père} aimant du grand Muruka\b n\index{gnl}{Murukan@Muruka\b n},\\
Il prête [son] corps au serpent\index{gnl}{serpent},\\
\^O Celui aux beaux m\'erites de la belle Pu\b ravam\index{gnl}{Puravam@Pu\b ravam}\\
Aux maisons telles des montagnes\\
O\`u vivent les asc\`etes supr\^emes. (I 117.8)\\
\end{verse}

\scriptsize
\begin{verse}
\textit{ila\.nkait talaiva\b nai, \=enti\b r\b ru, i\b ruttatu, iralai; il-n\=a\d l,\\
kala\.nkiya k\=u\b r\b ru\index{gnl}{Kurru@K\=u\b r\b ru}, uyir pe\b r\b ratu m\=a\d ni, kumaipe\b r\b ratu;\\
kalam ki\d lar montaiyi\b n, \=a\d tuvar, ko\d t\d tuvar, k\=a\d t\d tu akattu;-\\
calam ki\d lar v\=a\b l vayal ca\d npaiyu\d l m\=eviya tattuvar\=e.} (I 117.9)\\
\end{verse}
\scriptsize
\begin{verse}
\textit{ila\.nkait talaiva\b nai i\b ruttatu,\\
iralai \=enti\b r\b ru,\\
il n\=a\d l m\=a\d ni uyir pe\b r\b ratu,\\
kala\.nkiya k\=u\b r\b ru\index{gnl}{Kurru@K\=u\b r\b ru} kumaipe\b r\b ratu,\\
kalam ki\d lar montaiyi\b n ko\d t\d tuvar,
k\=a\d t\d tu akattu \=a\d tuvar,\\
calam ki\d lar v\=a\b l vayal ca\d npaiyu\d l m\=eviya tattuvar\=e.} (I 117.9)\\
\end{verse}
\normalsize
\begin{verse}
Il \'ecrase le chef\index{gnl}{chef} de Ila\.nkai\index{gnl}{Srilanka!Ila\.nkai},\\
Il porte l'antilope,\\
M\=a\d ni, sans avenir, obtint la vie\\
\lbrack Et\rbrack\ K\=u\b r\b ru\index{gnl}{Kurru@K\=u\b r\b ru}, troubl\'e, obtint la destruction,\footnote{Il s'agit du mythe\index{gnl}{mythe} du jeune M\=arka\d n\d deya qui est sauv\'e par \'Siva\index{gnl}{Siva@\'Siva} des griffes de Yama\index{gnl}{Yama}, dieu\index{gnl}{dieu} de la mort.}\\
Il frappe le tambour qui brille comme un bijou,\\
Il danse\index{gnl}{danser} dans le bois [cr\'ematoire],\\
\^O l'Absolu qui vit dans Ca\d npai\index{gnl}{Canpai@Ca\d npai}\\
Aux rizi\`eres fertiles o\`u coule l'eau\index{gnl}{eau}. (I 117.9)\\
\end{verse}

\scriptsize
\begin{verse}
\textit{a\d t\=\i \d nai ka\d n\d tila\b n, t\=amaraiy\=o\b n, m\=al, mu\d ti ka\d n\d tila\b n;\\
ko\d ti a\d niyum, puli, \=e\b ru, ukantu \=e\b ruvar, t\=ol u\d tuppar;\\
pi\d ti a\d niyum na\d taiy\=a\d l, ve\b rpu iruppatu; \=ork\=u\b ru u\d taiyar;-\\
ka\d ti a\d niyum po\b lil k\=a\b liyu\d l m\=eya ka\b raikka\d n\d tar\=e.} (I 117.10)\\
\end{verse}
\scriptsize
\begin{verse}
\textit{m\=al a\d t\=\i \d nai ka\d n\d tila\b n,\\
t\=amaraiy\=o\b n mu\d ti ka\d n\d tila\b n,\\
ko\d ti a\d niyum \=e\b ru ukantu \=e\b ruvar,\\
puli t\=ol u\d tuppar,\\
pi\d ti a\d niyum na\d taiy\=a\d l \=ork\=u\b ru u\d taiyar,\\
ve\b rpu iruppatu,\\
ka\d ti a\d niyum po\b lil k\=a\b liyu\d l m\=eya ka\b raikka\d n\d tar\=e.} (I 117.10)\\
\end{verse}
\normalsize
\begin{verse}
M\=al\index{gnl}{Visnu@Vi\d s\d nu!Mal@M\=al} n'a pas vu la paire de pieds,\\
Celui du lotus n'a pas vu la t\^ete,\\
Il monte avec joie le taureau\index{gnl}{taureau} qui porte la banni\`ere,\\
Il porte une peau de tigre,\\
Il poss\`ede une partie de Celle \`a la d\'emarche de l'\'el\'ephante,\\
Il est sur la montagne,\\
\^O Celui \`a la gorge tach\'ee\\
Qui vit dans K\=a\b li\index{gnl}{Kali@K\=a\b li} aux jardins parfum\'es. (I 117.10)\\
\end{verse}

\scriptsize
\begin{verse}
\textit{kaiyatu, ve\.nku\b lai k\=atatu, c\=ulam; ama\d narputtar,\\
eytuvar, tammai, a\d tiyavar, eyt\=ar; \=or \=e\b nakkompu,\\
mey tika\b l k\=ova\d nam, p\=u\d npatu, u\d tuppatu;-m\=etakaiya\\
koytu alar p\=umpo\b lil koccai\index{gnl}{Koccai}yu\d l m\=eviya ko\b r\b ravar\=e.} (I 117.11)\\
\end{verse}
\scriptsize
\begin{verse}
\textit{c\=ulam kaiyatu,\\
ve\.nku\b lai k\=atatu,\\
ama\d narputtar eyt\=ar,\\
a\d tiyavar tammai eytuvar,\\
\=or \=e\b nakkompu p\=u\d npatu,\\
mey tika\b l k\=ova\d nam u\d tuppatu,\\
m\=etakaiya koytu alar p\=umpo\b lil koccai\index{gnl}{Koccai}yu\d l m\=eviya ko\b r\b ravar\=e.} (I 117.11)\\
\end{verse}
\normalsize
\begin{verse}
Le trident \`a la main,\\
La boucle blanche \`a l'oreille;\\
Il n'approche pas les bouddhiste\index{gnl}{bouddhiste}s et les ja\"in\index{gnl}{jain@ja\"in}s,\\
Il approche les dévot\index{gnl}{devot(e)@dévot(e)}s;\\
Une d\'efense de sanglier pour ornement, \\
Un cache-sexe pour v\^etement sur son corps \'eclatant;\\
\^O le Victorieux qui vit dans Koccai\index{gnl}{Koccai} \\
Aux excellents jardins\\
Fleuris [m\^eme quand les fleurs ont \'et\'e] cueillies. (I 117.11)\\
\end{verse}

\scriptsize
\begin{verse}
\textit{kal uyariñcik ka\b lumalam\index{gnl}{Kalumalam@Ka\b lumalam} m\=eya ka\d tavu\d lta\b n\b nai\\
nalurai \~n\=a\b nacampanta\b n \~n\=a\b nattami\b l na\b nku u\d narac\\
colli\d tal k\=e\d t\d tal vall\=or, tollaiv\=a\b navarta\.nka\d lo\d tum\\
celkuvar; c\=\i r aru\d l\=al pe\b ral\=am cival\=okamat\=e.} (I 117.12)\\
\end{verse}
\normalsize
\begin{verse}
Ceux qui sont forts dans l'\'ecoute et la récitation\index{gnl}{recitation@récitation},\\
De fa\c con \`a bien ressentir le tamoul de la connaissance\index{gnl}{connaissance}\\
Que \~N\=a\b nacampanta\b n\index{gnl}{Campantar!N\=a\b nacampanta\b n@\~N\=a\b nacampanta\b n} a bien prononcé sur le dieu\index{gnl}{dieu} qui vit \`a Ka\b lumalam\index{gnl}{Kalumalam@Ka\b lumalam}\\
Aux hautes fortifications de pierres,\\
\lbrack Ceux-l\`a\rbrack\ iront avec les anciens c\'elestes;\\
Par la grande gr\^ace le monde de \'Siva\index{gnl}{Siva@\'Siva} est accessible. (I 117.12)\\
\end{verse}

\subsection{Hymne I 127}

Ce poème\index{gnl}{poeme@poème} est compos\'e selon le procédé\index{gnl}{procédé littéraire} litt\'eraire de l'\textit{\=ekap\=atam}\index{gnl}{ekapatam@\textit{\=ekap\=atam}} (voir 2.1.3). Dans l'\'etat actuel de notre connaissance\index{gnl}{connaissance} nous ne pouvons proposer de traduction \`a cet hymne\index{gnl}{hymne} particulier. Nous sugg\'erons de consulter les commentaires et traductions \'etablis par \textsc{T. V. Gopal Iyer} (1991: 101-112) et par \textsc{V. M. Subramanya Aiyar} (voir \textsc{Chevillard} 2007).

\subsection{Hymne I 128}

L'hymne\index{gnl}{hymne} I 128 est \'elabor\'e selon la figure de l'\textit{e\b luk\=u\b r\b rirukkai} (voir 2.1.3). T\=o\d nipuram\index{gnl}{Tonipuram@T\=o\d nipuram} (l. 28-29) est le seul toponyme dont la légende\index{gnl}{legende@légende} soit ici pr\'esent\'ee.

\scriptsize
\begin{verse}
\textit{\=or uru \=ayi\b nai; m\=a\b n \=a\.nk\=arattu\\
\=\i r iyalpu\=ay, oru vi\d n mutal p\=utalam\\
o\b n\b riya irucu\d tar umparka\d l pi\b ravum\\
pa\d taittu, a\d littu, a\b lippa, mumm\=urttika\d l \=ayi\b nai;\\
iruvar\=o\d tu oruva\b n \=aki ni\b n\b ra\b nai;} 5\\
\textit{\=or \=aln\=\i \b lal, o\.nka\b lalira\d n\d tum\\
muppo\b lutu \=ettiya n\=alvarkku o\d line\b ri\\
k\=a\d t\d ti\b nai; n\=a\d t\d tam m\=u\b n\b ru \=akak k\=o\d t\d ti\b nai;\\
irunati aravam\=o\d tu orumati c\=u\d ti\b nai;\\
orut\=a\d l \=\i r ayil m\=u ilaicc\=ulam,} 10\\
\textit{n\=alk\=al m\=a\b nma\b ri, aintalai aravam\\
\=enti\b nai; k\=aynta n\=al v\=ay mummatattu\\
iruk\=o\d t\d tu orukari \=\i \d tu a\b littu uritta\b nai;\\
oruta\b nu iruk\=al va\d laiya v\=a\.nki,\\
mumpuratt\=o\d tu n\=a\b nilam añca,} 15\\
\textit{ko\b n\b ru talattu u\b ra avu\d narai a\b rutta\b nai;\\
aimpula\b n, n\=al \=am antakkara\d nam,\\
mukku\d nam, iruva\d li, oru\.nkiya v\=a\b n\=or\\
\=etta ni\b n\b ra\b nai; oru\.nkiya ma\b natt\=o\d tu\\
irupi\b rappu \=orntu, muppo\b lutu ku\b rai mu\d tittu,} 20\\
\textit{n\=alma\b rai \=oti, aivakai v\=e\d lvi\\
amaittu, \=a\b ru a\.nkam mutal e\b luttu \=oti,\\
varal mu\b rai payi\b n\b ru, e\b lu v\=a\b nta\b nai va\d larkkum\\
piramapuram\index{gnl}{Piramapuram} p\=e\d ni\b nai;\\
a\b rupatam muralum v\=e\d nupuram\index{gnl}{Venupuram@V\=e\d nupuram} virumpi\b nai;} 25\\
\textit{ikali amaintu u\d nar pukali\index{gnl}{Pukali} amarnta\b nai;\\
po\.nku n\=alka\d tal c\=u\b l ve\.nkuru\index{gnl}{Venkuru@Ve\.nkuru} vi\d la\.nki\b nai;\\
p\=a\d ni m\=uulakum putaiya, m\=el mitanta\\
t\=o\d nipurattu u\b rainta\b nai; tolaiy\=a iruniti\\
v\=aynta p\=untar\=ay\index{gnl}{Taray@Tar\=ay} \=eynta\b nai;} 30\\
\textit{vara puram o\b n\b ru u\d nar cirapurattu u\b rainta\b nai;\\
orumalai e\d tutta iruti\b ral arakka\b n\\
vi\b ral ke\d tuttu aru\d li\b nai; pu\b ravam\index{gnl}{Puravam@Pu\b ravam} purinta\b nai;\\
munn\=\i rt tuyi\b n\b r\=o\b n, n\=a\b nmuka\b n, a\b riy\=ap\\
pa\d npo\d tu ni\b n\b ra\b nai; ca\d npai\index{gnl}{Canpai@Ca\d npai} amarnta\b nai;} 35\\
\textit{aiyu\b rum ama\d narum a\b ruvakait t\=erarum\\
\=u\b liyum u\d nar\=ak k\=a\b li amarnta\b nai;\\
ecca\b n \=e\b licaiy\=o\b n koccai\index{gnl}{Koccai}yai mecci\b nai;\\
\=a\b rupatamum, aintu amar kalviyum,\\
ma\b rai mutal n\=a\b nkum,} 40\\
\textit{m\=u\b n\b ruk\=alamum, t\=o\b n\b ra ni\b n\b ra\b nai;\\
irumaiyi\b n orumaiyum, orumaiyi\b n perumaiyum,\\
ma\b ru il\=a ma\b raiy\=or\\
ka\b lumala mutu patik kavu\d niya\b n ka\d t\d turai\\
ka\b lumala mutupatik ka u\d niya\b n a\b riyum;} 45\\
\textit{a\b naiya ta\b nmaiyai \=atali\b n, ni\b n\b nai\\
ni\b naiya vallavar illai, n\=\i \d l nilatt\=e.}\\
\end{verse}

\normalsize
\begin{verse}
%\=or uru \=ayi\b nai
Tu es devenu une forme;\\
%m\=a\b n \=a\.nk\=arattu \=\i r iyalpu\=ay
Tu es devenu la nature double de \'Siva\index{gnl}{Siva@\'Siva} (\textit{\=a\.nkaram}) et de \'Sakti\index{gnl}{Sakti@\'Sakti} (\textit{m\=a\b n});\\
%oru vi\d n mutal p\=utalam o\b n\b riya irucu\d tar umparka\d l pi\b ravum pa\d taittu, a\d littu, a\b lippa, mumm\=urttika\d l \=ayi\b nai
Le ciel unique jusqu'\`a la terre\index{gnl}{terre}, les deux luminaires unis, les êtres c\'elestes et tous les autres, [les] ayant cr\'e\'es, [les] ayant maintenus, pour [les] d\'etruire, tu es devenu la triple manifestation;\\
%iruvar\=o\d tu oruva\b n \=aki ni\b n\b ra\b nai
Tu te tiens devenu un avec les deux (Brahm\=a\index{gnl}{Brahma@Brahm\=a} et Vi\d s\d nu\index{gnl}{Visnu@Vi\d s\d nu});\\
%\=or \=aln\=\i \b lal, o\.nka\b lalira\d n\d tum muppo\b lutu \=ettiya n\=alvarkku o\d line\b ri k\=a\d t\d ti\b nai
\`A l'ombre d'un banyan, aux quatre qui ont honor\'e trois fois [tes] deux [pieds] aux anneaux de chevilles brillants, tu as montr\'e le chemin lumineux;\\
%n\=a\d t\d tam m\=u\b n\b ru \=akak k\=o\d t\d ti\b nai
Tu dessinas [sur le front] pour que les yeux deviennent trois;\\
%irunati aravam\=o\d tu orumati c\=u\d ti\b nai
Tu t'es couronn\'e de la grande rivi\`ere, de serpent\index{gnl}{serpent}s et d'une lune unique;\\
%orut\=a\d l \=\i r ayil m\=u ilaicc\=ulam, n\=alk\=al m\=a\b nma\b ri, aintalai aravam \=enti\b nai
Tu as tenu la pique \`a trois [pointes en forme de] feuilles grandes et pointue \`a manche, la jeune gazelle \`a quatre pattes et le serpent\index{gnl}{serpent} \`a cinq t\^etes;\\
%k\=aynta n\=al v\=ay mummatattu iruk\=o\d t\d tu orukari \=\i \d tu a\b littu uritta\b nai
Tu d\'epouillas, ayant d\'etruit sa force, un \'el\'ephant qui s'\'etait mis en col\`ere, \`a la bouche pendante (trompe), \`a trois \textit{matam}\footnote{Le terme \textit{matam} renvoie \`a la p\'eriode de rut de l'\'el\'ephant pendant laquelle une s\'ecr\'etion, appel\'ee aussi \textit{matam}, coule de trois endroits: de la trompe, des yeux et des tempes, selon la tradition que nous donnons ici d'après des propos recueillis auprès de \textsc{T. V. Gopal Iyer}.} et \`a deux d\'efenses;\\
%oruta\b nu iruk\=al va\d laiya v\=a\.nki
Ayant raccord\'e faisant courber les deux extrémit\'es d'un arc;
%mumpuratt\=o\d tu n\=a\b nilam añca
alors que les trois citadelles avaient peur avec les quatre r\'egions,
%ko\b n\b ru talattu u\b ra avu\d narai a\b rutta\b nai
ayant tu\'es les démons, [leur] faisant sentir le sol, tu [les] brisas;\\
%aimpula\b n, n\=al \=am antakkara\d nam, mukku\d nam, iruva\d li, oru\.nkiya v\=a\b n\=or \=etta ni\b n\b ra\b nai
Tu t'es tenu alors que louaient les êtres c\'elestes unis aux deux souffles, aux trois caract\`eres, aux quatre beaux \textit{kara\d na} %\footnote{Selon T. V. Gopal Iyerces quatres kara\d na sont (ma\b nam, buddhi, cittam, \=a\.nk\=aram)}
et aux cinq sens;\\
%oru\.nkiya ma\b natt\=o\d tu irupi\b rappu \=orntu
Ayant compris les deux vies avec un esprit unifi\'e (ferme),
%muppo\b lutu ku\b rai mu\d tittu
ayant accompli les t\^aches (c\'er\'emonies) trois fois,
%n\=alma\b rai \=oti
ayant chant\'e les quatre \textit{Veda}\index{gnl}{Veda@\textit{Veda}},
%aivakai v\=e\d lvi amaittu ayant effectu\'e
ayant accompli les sacrifices de cinq sortes,
%\=a\b ru a\.nkam mutal e\b luttu \=oti
ayant chant\'e la premi\`ere syllabe des six \textit{a\.nga},
%varal mu\b rai payi\b n\b ru
s'\'etant exerc\'e selon les convenances,
%e\b lu v\=a\b nta\b nai va\d larkkum Piramapuram\index{gnl}{Piramapuram} p\=e\d ni\b nai
tu r\'esidas \`a Piramapuram\index{gnl}{Piramapuram} qui fait cro\^itre les nuages qui montent;\\
%a\b rupatam muralum V\=e\d nupuram\index{gnl}{Venupuram@V\=e\d nupuram} virumpi\b nai
Tu aimas V\=e\d nupuram\index{gnl}{Venupuram@V\=e\d nupuram} o\`u bourdonnent les [abeilles \`a] six pattes;\\
%ikali amaintu u\d nar Pukali\index{gnl}{Pukali} amarnta\b nai
Tu r\'esidas \`a Pukali\index{gnl}{Pukali} qui est estim\'ee [par ceux qui] ayant de l'hostilit\'e se sont \'etablis [l\`a];\\
%po\.nku n\=alka\d tal c\=u\b l Ve\.nkuru\index{gnl}{Venkuru@Ve\.nkuru} vi\d la\.nki\b nai
Tu r\'esidas \`a Ve\.nkuru\index{gnl}{Venkuru@Ve\.nkuru} entour\'e des quatre mers agit\'ees;\\
%p\=a\d ni m\=uulakum putaiya, m\=el mitanta t\=o\d nipurattu u\b rainta\b nai
Alors que les trois mondes furent cach\'es par les eaux\index{gnl}{eau}, tu r\'esidas \`a T\=o\d nipuram\index{gnl}{Tonipuram@T\=o\d nipuram} qui \'emergeait au-dessus [d'elles];\\
%tolaiy\=a iruniti v\=aynta p\=unTar\=ay\index{gnl}{Taray@Tar\=ay} \=eynta\b nai
Tu as atteint la belle Tar\=ay\index{gnl}{Taray@Tar\=ay} o\`u les deux richesses abondent sans dispara\^itre;\\
%vara puram o\b n\b ru u\d nar cirapurattu u\b rainta\b nai
Tu r\'esidas \`a Cirapuram\index{gnl}{Cirapuram} qui est estim\'e [comme] ville sup\'erieure ;\\
%orumalai e\d tutta iruti\b ral arakka\b n vi\b ral ke\d tuttu aru\d li\b nai
Tu accordas la gr\^ace ayant d\'etruit la puissance du démon\index{gnl}{demon@démon} \`a la grande force qui a pris la montagne;\\
Tu d\'esiras Pu\b ravam\index{gnl}{Puravam@Pu\b ravam};\\
%munn\=\i rt tuyi\b n\b r\=o\b n, n\=a\b nmuka\b n, a\b riy\=ap pa\d npo\d tu ni\b n\b ra\b nai
Tu t'es tenu avec la qualit\'e qui n'est pas connue de Celui qui dort sur le triple oc\'ean et de Celui aux quatre visages;\\
%Ca\d npai\index{gnl}{Canpai@Ca\d npai} amarnta\b nai;
Tu r\'esidas \`a Ca\d npai\index{gnl}{Canpai@Ca\d npai};\\
%aiyu\b rum ama\d narum a\b ruvakait t\=erarum \=u\b liyum u\d nar\=ak k\=a\b li amarnta\b nai
Tu r\'esidas \`a K\=a\b li\index{gnl}{Kali@K\=a\b li} qui ne peut \^etre sentie m\^eme [au moment] de la dissolution par les ja\"in\index{gnl}{jain@ja\"in}s qui doutent [des \textit{Veda}\index{gnl}{Veda@\textit{Veda}}] et par les bouddhiste\index{gnl}{bouddhiste}s de six sortes;\\
%ecca\b n \=e\b licaiy\=o\b n Koccai\index{gnl}{Koccai}yai mecci\b nai
\^O Celui des sept notes, \^O Celui du sacrifice, tu r\'esidas \`a Koccai\index{gnl}{Koccai};\\
%\=a\b rupatamum, aintu amar kalviyum, ma\b rai mutal n\=a\b nkum, m\=u\b n\b ruk\=alamum, t\=o\b n\b ra ni\b n\b ra\b nai
Tu t'es tenu pour faire appara\^itre les six pas [du yoga], l'apprentissage qui r\'eside par les cinq [sens], les quatre premiers \textit{Veda}\index{gnl}{Veda@\textit{Veda}} et les trois temps;\\
%ma\b ru il\=a ma\b raiy\=or ka\b lumala mutu patik kavu\d niya\b n ka\d t\d turai ka\b lumala mutupatik ka u\d niya\b n a\b riyum
Celui de l'ancienne ville de Ka\b lumalam\index{gnl}{Kalumalam@Ka\b lumalam}, qui porte le cr\^ane [pour bol], conna\^it le texte du \textit{kavu\d ni}\index{gnl}{kaundinya@\textit{kau\d n\d dinya}!\textit{kavu\d ni}} de l'ancienne ville de Ka\b lumalam\index{gnl}{Kalumalam@Ka\b lumalam} des [brahmane\index{gnl}{brahmane}s] v\'ediques qui sont sans faute;\\
%irumaiyi\b n orumaiyum, orumaiyi\b n perumaiyum
Un (Ardhan\=ar\=\i \'svara) des deux (\'Siva\index{gnl}{Siva@\'Siva} et P\=arvat\=\i\index{gnl}{Parvati@P\=arvat\=\i}), grandeur de l'union,\\
%a\b naiya ta\b nmaiyai \=atali\b n, ni\b n\b nai ni\b naiya vallavar illai, n\=\i \d l nilatt\=e
\`A cause d'une telle nature, il n'y a [personne] capable de te sentir dans ce grand monde.\\
\end{verse}

\subsection{Hymne II 70}

Le poème\index{gnl}{poeme@poème} est organis\'e selon la figure de style du \textit{cakkaram\=a\b r\b ru}\index{gnl}{cakkaramarru@\textit{cakkaram\=a\b r\b ru}}, \og \'echange circulaire\fg\ (voir 2.1.3).
Le \textit{Tamil Lexicon} d\'efinit ce procédé\index{gnl}{procédé littéraire} ainsi:

\scriptsize
\begin{quote}
\og a poem on Shiyali by Saint Campantar\index{gnl}{Campantar}, wherein each stanza mentions all the names of that sacred shrine and the last mentioned name in a stanza begins the next stanza.\fg
\end{quote}
\normalsize
\noindent
Mais ce poème\index{gnl}{poeme@poème} ne respecte pas cette d\'efinition. En effet, aucune strophe ne d\'ebute avec le dernier toponyme mentionné dans la pr\'ec\'edente. Chaque quatrain d\'ebute par les noms de C\=\i k\=a\b li selon leur ordre\index{gnl}{ordre}: Piramapuram\index{gnl}{Piramapuram} ouvre la premi\`ere strophe, V\=e\d nupuram\index{gnl}{Venupuram@V\=e\d nupuram} la deuxi\`eme, Pukali\index{gnl}{Pukali} la troisi\`eme, et ainsi de suite. Trois appellations seulement sont pr\'esent\'ees avec leurs légende\index{gnl}{legende@légende}s d'origine\index{gnl}{origine}: Piramapuram\index{gnl}{Piramapuram} (st. 1, 2, 5, 6, 7, 8, 10 et 11), T\=o\d nipuram\index{gnl}{Tonipuram@T\=o\d nipuram} (st. 1, 2, 3, 4 et 5) et Cirapuram\index{gnl}{Cirapuram} (st. 4, 5 et 6). Concernant la structure de l'hymne\index{gnl}{hymne}, il n'y a pas de strophe d\'edi\'ee \`a R\=ava\d na\index{gnl}{Ravana@R\=ava\d na}. Seules la manifestation du \textit{li\.nga}\index{gnl}{linga@\textit{li\.nga}} et la critique des hérétique\index{gnl}{heretique@hérétique}s figurent dans les strophes 7 et 9, respectivement, ce qui ne correspond pas \`a leur place usuelle.

\scriptsize
\begin{verse}
\textit{pirama\b n \=ur, v\=e\d nupuram\index{gnl}{Venupuram@V\=e\d nupuram}, pukali\index{gnl}{Pukali}, ve\.nkuru\index{gnl}{Venkuru@Ve\.nkuru}, perun\=\i rt t\=o\d ni-\\
puram, ma\b n\b nu p\=untar\=ay\index{gnl}{Taray@Tar\=ay}, po\b n am cirapuram\index{gnl}{Cirapuram}, pu\b ravam\index{gnl}{Puravam@Pu\b ravam}, ca\d npai\index{gnl}{Canpai@Ca\d npai},\\
ara\b n ma\b n\b nu ta\d n k\=a\b li, koccai\index{gnl}{Koccai}vayam, u\d l\d li\d t\d tu a\.nku \=ati \=aya\\
parama\b n \=ur pa\b n\b nira\d n\d tu \=ay ni\b n\b ra tiruk ka\b lumalam\index{gnl}{Kalumalam@Ka\b lumalam}---n\=am paravum \=ur\=e.} (II 70.1)\\
\end{verse}
\normalsize
\begin{verse}
La ville de Brahm\=a\index{gnl}{Brahma@Brahm\=a}, V\=enupuram, Pukali\index{gnl}{Pukali}, Ve\.nkuru\index{gnl}{Venkuru@Ve\.nkuru},\\
T\=o\d nipuram\index{gnl}{Tonipuram@T\=o\d nipuram} des grandes eaux\index{gnl}{eau},\\
L'in\'ebranlable belle Tar\=ay\index{gnl}{Taray@Tar\=ay}, Cirapuram\index{gnl}{Cirapuram} la dor\'ee, Pu\b ravam\index{gnl}{Puravam@Pu\b ravam}, Ca\d npai\index{gnl}{Canpai@Ca\d npai},\\
La fra\^iche K\=a\b li\index{gnl}{Kali@K\=a\b li} o\`u r\'eside Hara\index{gnl}{Hara} [et] Koccai\index{gnl}{Koccai}vayam, en [les] incluant\\
La ville que nous louons est l'honorable Ka\b lumalam\index{gnl}{Kalumalam@Ka\b lumalam} qui demeure\\
\'Etant devenue les douze\index{gnl}{douze} villes du Seigneur\\
Qui fut l\`a le commencement. (II 70.1)\\
\end{verse}

\scriptsize
\begin{verse}
\textit{v\=e\d nupuram\index{gnl}{Venupuram@V\=e\d nupuram}, pirama\b n \=ur, pukali\index{gnl}{Pukali}, peru ve\.nkuru\index{gnl}{Venkuru@Ve\.nkuru}, ve\d l\d lattu \=o\.nkum\\
t\=o\d nipuram\index{gnl}{Tonipuram@T\=o\d nipuram}, p\=untar\=ay\index{gnl}{Taray@Tar\=ay}, t\=u n\=\i rc cirapuram\index{gnl}{Cirapuram}, pu\b ravam\index{gnl}{Puravam@Pu\b ravam}, k\=a\b li,\\
k\=o\d niya k\=o\d t\d t\=a\b r\b ruk koccai\index{gnl}{Koccai}vayam, ca\d npai\index{gnl}{Canpai@Ca\d npai}, k\=urum celvam\\
k\=a\d niya vaiyakatt\=ar \=ettum ka\b lumalam\index{gnl}{Kalumalam@Ka\b lumalam}---n\=am karutum \=ur\=e.}(II 70.2)\\
\end{verse}
\normalsize
\begin{verse}
V\=e\d nupuram\index{gnl}{Venupuram@V\=e\d nupuram}, la ville de Brahm\=a\index{gnl}{Brahma@Brahm\=a}, Pukali\index{gnl}{Pukali}, la grande Ve\.nkuru\index{gnl}{Venkuru@Ve\.nkuru},\\
T\=o\d nipuram\index{gnl}{Tonipuram@T\=o\d nipuram} qui s'\'el\`eve au déluge\index{gnl}{deluge@déluge},\\
La belle Tar\=ay\index{gnl}{Taray@Tar\=ay}, Cirapuram\index{gnl}{Cirapuram} aux eaux\index{gnl}{eau} pures, Pu\b ravam\index{gnl}{Puravam@Pu\b ravam}, K\=a\b li\index{gnl}{Kali@K\=a\b li},\\
Koccai\index{gnl}{Koccai}vayam \`a la rivi\`ere sinueuse de K\=o\d t\d tam\index{gnl}{Kottam@K\=o\d t\d tam} [et] Ca\d npai\index{gnl}{Canpai@Ca\d npai} [c'est]\\
Ka\b lumalam\index{gnl}{Kalumalam@Ka\b lumalam}, la ville que nous m\'editons\\
Et que les habitants de la terre\index{gnl}{terre} louent\\
Pour conna\^itre une fortune abondante. (II 70.2)\\
\end{verse}

\scriptsize
\begin{verse}
\textit{pukali\index{gnl}{Pukali}, cirapuram\index{gnl}{Cirapuram}, v\=e\d nupuram\index{gnl}{Venupuram@V\=e\d nupuram}, ca\d npai\index{gnl}{Canpai@Ca\d npai}, pu\b ravam\index{gnl}{Puravam@Pu\b ravam}, k\=a\b li,\\
nikar il piramapuram\index{gnl}{Piramapuram}, koccai\index{gnl}{Koccai}vayam, n\=\i rm\=el ni\b n\b ra m\=ut\=ur,\\
akaliya ve\.nkuruv\=o\d tu, am ta\d n tar\=ay\index{gnl}{Taray@Tar\=ay}, amararperum\=a\b rku i\b npam\\
pakarum nakar nalla ka\b lumalam\index{gnl}{Kalumalam@Ka\b lumalam}---n\=am kaito\b lutu p\=a\d tum \=ur\=e.} (II 70.3)\\
\end{verse}
\normalsize
\begin{verse}
Pukali\index{gnl}{Pukali}, Cirapuram\index{gnl}{Cirapuram}, V\=e\d nupuram\index{gnl}{Venupuram@V\=e\d nupuram}, Ca\d npai\index{gnl}{Canpai@Ca\d npai}, Pu\b ravam\index{gnl}{Puravam@Pu\b ravam}, K\=a\b li\index{gnl}{Kali@K\=a\b li},\\
Piramapuram\index{gnl}{Piramapuram} sans comparaison, Koccai\index{gnl}{Koccai}vayam,\\
L'antique ville qui restait sur l'eau\index{gnl}{eau},\\
La belle Ve\.nkuru\index{gnl}{Venkuru@Ve\.nkuru} [et] la belle et fra\^iche Tar\=ay\index{gnl}{Taray@Tar\=ay} [c'est]\\
La ville qui donne du bonheur au Seigneur des immortels,\\
La ville que nous chantons en v\'en\'erant les mains [jointes],\\
La bonne Ka\b lumalam\index{gnl}{Kalumalam@Ka\b lumalam}. (II 70.3)\\
\end{verse}

\scriptsize
\begin{verse}
\textit{ve\.nkuru\index{gnl}{Venkuru@Ve\.nkuru}, ta\d n pukali\index{gnl}{Pukali}, v\=e\d nupuram\index{gnl}{Venupuram@V\=e\d nupuram}, ca\d npai\index{gnl}{Canpai@Ca\d npai}, ve\d l\d lam ko\d l\d lat\\
to\.nkiya t\=o\d nipuram\index{gnl}{Tonipuram@T\=o\d nipuram}, p\=untar\=ay\index{gnl}{Taray@Tar\=ay}, toku piramapuram\index{gnl}{Piramapuram}, tol k\=a\b li,\\
ta\.nku po\b lil pu\b ravam\index{gnl}{Puravam@Pu\b ravam}, koccai\index{gnl}{Koccai}vayam, talai pa\d n\d tu \=a\d n\d ta m\=ut\=ur,\\
ka\.nkai ca\d taimu\d tim\=el \=e\b r\b r\=a\b n ka\b lumalam\index{gnl}{Kalumalam@Ka\b lumalam}-n\=am karutum \=ur\=e.} (II 70.4)\\
\end{verse}
\normalsize
\begin{verse}
Ve\.nkuru\index{gnl}{Venkuru@Ve\.nkuru}, la fra\^iche Pukali\index{gnl}{Pukali}, V\=e\d nupuram\index{gnl}{Venupuram@V\=e\d nupuram}, Ca\d npai\index{gnl}{Canpai@Ca\d npai},\\
T\=o\d nipuram\index{gnl}{Tonipuram@T\=o\d nipuram} qui demeura alors que le déluge\index{gnl}{deluge@déluge} s'abattait,\\
La belle Tar\=ay\index{gnl}{Taray@Tar\=ay}, l'estimable Piramapuram\index{gnl}{Piramapuram}, l'antique K\=a\b li\index{gnl}{Kali@K\=a\b li},\\
Pu\b ravam\index{gnl}{Puravam@Pu\b ravam} aux jardins permanents, Koccai\index{gnl}{Koccai}vayam\\
\lbrack Et] l'antique ville o\`u a r\'egn\'e jadis une t\^ete [c'est]\\
Ka\b lumalam\index{gnl}{Kalumalam@Ka\b lumalam}, la ville que nous m\'editons\\
De Celui qui \'eleva la Ga\.ng\=a\index{gnl}{Ganga@Ga\.ng\=a} dans son chignon de m\`eches. (II 70.4)\\
\end{verse}

\scriptsize
\begin{verse}
\textit{tol n\=\i ril t\=o\d nipuram\index{gnl}{Tonipuram@T\=o\d nipuram}, pukali\index{gnl}{Pukali}, ve\.nkuru\index{gnl}{Venkuru@Ve\.nkuru}, tuyar t\=\i r k\=a\b li,\\
i\b n n\=\i ra v\=e\d nupuram\index{gnl}{Venupuram@V\=e\d nupuram}, p\=untar\=ay\index{gnl}{Taray@Tar\=ay}, pirama\b n \=ur, e\b lil \=ar ca\d npai\index{gnl}{Canpai@Ca\d npai},\\
na\b nn\=\i ra p\=um pu\b ravam\index{gnl}{Puravam@Pu\b ravam}, koccai\index{gnl}{Koccai}vayam, cilampa\b nnakar, \=am nalla\\
po\b nn\=\i ra pu\b nca\d taiy\=a\b n p\=un ta\d n ka\b lumalam\index{gnl}{Kalumalam@Ka\b lumalam}---n\=am puka\b lum \=ur\=e.} (II 70.5)\\
\end{verse}
\normalsize
\begin{verse}
T\=o\d nipuram\index{gnl}{Tonipuram@T\=o\d nipuram} sur l'eau\index{gnl}{eau} ancienne, Pukali\index{gnl}{Pukali}, Ve\.nkuru\index{gnl}{Venkuru@Ve\.nkuru},\\
K\=a\b li\index{gnl}{Kali@K\=a\b li} qui gu\'erit des souffrances,\\
V\=e\d nupuram\index{gnl}{Venupuram@V\=e\d nupuram} aux eaux\index{gnl}{eau} miellées, la belle Tar\=ay\index{gnl}{Taray@Tar\=ay},\\
La ville de Brahm\=a\index{gnl}{Brahma@Brahm\=a}, Ca\d npai\index{gnl}{Canpai@Ca\d npai} la toute belle,\\
Pu\b ravam\index{gnl}{Puravam@Pu\b ravam} la belle aux bonnes eaux\index{gnl}{eau}, Koccai\index{gnl}{Koccai}vayam\\
\lbrack Et] la ville de Cilampa\b n\index{gnl}{Cilampan@Cilampa\b n} [c'est]\\
La fra\^iche et fleurie Ka\b lumalam\index{gnl}{Kalumalam@Ka\b lumalam}, la ville que nous louons\\
De Celui aux m\`eches sombres\\
Qui a la belle et bonne couleur de l'or. (II 70.5)\\
\end{verse}

\scriptsize
\begin{verse}
\textit{ta\d n am tar\=ay\index{gnl}{Taray@Tar\=ay}, pukali\index{gnl}{Pukali}, t\=amaraiy\=a\b n\=ur, ca\d npai\index{gnl}{Canpai@Ca\d npai}, talai mu\b n \=a\d n\d ta\\
a\d n\d nal nakar, koccai\index{gnl}{Koccai}vayam, ta\d n pu\b ravam\index{gnl}{Puravam@Pu\b ravam}, c\=\i r a\d ni \=ar k\=a\b li,\\
vi\d n iyal c\=\i r ve\.nkuru\index{gnl}{Venkuru@Ve\.nkuru}, nal v\=e\d nupuram\index{gnl}{Venupuram@V\=e\d nupuram}, t\=o\d nipuram\index{gnl}{Tonipuram@T\=o\d nipuram}, m\=el\=ar \=ettu\\
ka\d nnutal\=a\b n m\=eviya nal ka\b lumalam\index{gnl}{Kalumalam@Ka\b lumalam}---n\=am kaito\b lutu karutum \=ur\=e.} (II 70.6)\\
\end{verse}
\normalsize
\begin{verse}
La belle et fra\^iche Tar\=ay\index{gnl}{Taray@Tar\=ay}, Pukali\index{gnl}{Pukali}, la ville de Celui du lotus, Ca\d npai\index{gnl}{Canpai@Ca\d npai},\\
La ville excellente o\`u a jadis r\'egn\'e une t\^ete, Koccai\index{gnl}{Koccai}vayam,\\
La fra\^iche Pu\b ravam\index{gnl}{Puravam@Pu\b ravam}, K\=a\b li\index{gnl}{Kali@K\=a\b li} toute orn\'ee de gloire, \\
Ve\.nkuru\index{gnl}{Venkuru@Ve\.nkuru} \`a la gloire comparable au ciel,\\
La bonne V\=e\d nupuram\index{gnl}{Venupuram@V\=e\d nupuram} [et] T\=o\d nipuram\index{gnl}{Tonipuram@T\=o\d nipuram} [c'est]\\
La bonne Ka\b lumalam\index{gnl}{Kalumalam@Ka\b lumalam},\\
La ville que nous m\'editons en louant avec les mains [jointes]\\
O\`u r\'esidait avec plaisir Celui \`a l'oeil frontal\\
Que les c\'elestes v\'en\`erent. (II 70.6)\\
\end{verse}

\scriptsize
\begin{verse}
\textit{c\=\i r \=ar cirapuram\index{gnl}{Cirapuram}um, koccai\index{gnl}{Koccai}vayam, ca\d npaiyo\d tu, pu\b ravam\index{gnl}{Puravam@Pu\b ravam}, nalla\\
\=ar\=at tar\=ay\index{gnl}{Taray@Tar\=ay}, pirama\b n \=ur, pukali\index{gnl}{Pukali}, ve\.nkuruvo\d tu, am ta\d n k\=a\b li,\\
\=er \=ar ka\b lumalam\index{gnl}{Kalumalam@Ka\b lumalam}um, v\=e\d nupuram\index{gnl}{Venupuram@V\=e\d nupuram}, t\=o\d nipuram\index{gnl}{Tonipuram@T\=o\d nipuram}, e\b n\b rue\b n\b ru u\d lki,\\
p\=er\=al ne\d tiyava\b num n\=a\b nmuka\b num k\=a\d npu ariya perum\=a\b n \=ur\=e.} (II 70.7)\\
\end{verse}
\normalsize
\begin{verse}
Cirapuram\index{gnl}{Cirapuram} emplie de gloire, Koccai\index{gnl}{Koccai}vayam, ainsi que Ca\d npai\index{gnl}{Canpai@Ca\d npai},\\
Pu\b ravam\index{gnl}{Puravam@Pu\b ravam}, la bonne Tar\=ay\index{gnl}{Taray@Tar\=ay} sans pareille,\\
La ville de Brahm\=a\index{gnl}{Brahma@Brahm\=a}, Pukali\index{gnl}{Pukali}, avec Ve\.nkuru\index{gnl}{Venkuru@Ve\.nkuru}, la belle et fra\^iche K\=a\b li\index{gnl}{Kali@K\=a\b li},\\
Et Ka\b lumalam\index{gnl}{Kalumalam@Ka\b lumalam} la toute belle, V\=e\d nupuram\index{gnl}{Venupuram@V\=e\d nupuram} [et] T\=o\d nipuram\index{gnl}{Tonipuram@T\=o\d nipuram} [c'est]\\
La ville du Seigneur qui n'a pu \^etre vu\\
Par Celui aux quatre visages et par le Grand,\\
Alors qu'ils m\'editaient sans relâche\\
\lbrack Cette ville] par ses noms\footnote{\textsc{T. V. Gopal Iyer} propose de lire \textit{p\=er\=al ne\d tiyava\b n}: \og [Vi\d s\d nu\index{gnl}{Visnu@Vi\d s\d nu}] le grand par [ses] noms\fg.}. (II 70.7)\\
\end{verse}

\scriptsize
\begin{verse}
\textit{pu\b ravam\index{gnl}{Puravam@Pu\b ravam}, cirapuram\index{gnl}{Cirapuram}um, t\=o\d nipuram\index{gnl}{Tonipuram@T\=o\d nipuram}, ca\d npai\index{gnl}{Canpai@Ca\d npai}, miku pukali\index{gnl}{Pukali}, k\=a\b li,\\
na\b ravam miku c\=olaik koccai\index{gnl}{Koccai}vayam, tar\=ay\index{gnl}{Taray@Tar\=ay}, n\=a\b nmuka\b nta\b n \=ur,\\
vi\b ral \=aya ve\.nkuruvum, v\=e\d nupuram\index{gnl}{Venupuram@V\=e\d nupuram}, vicaya\b nm\=el ampu eytu\\
ti\b ral\=al arakka\b naic ce\b r\b r\=a\b nta\b n ka\b lumalam\index{gnl}{Kalumalam@Ka\b lumalam}---n\=am c\=erum \=ur\=e.} (II 70.8)\\
\end{verse}
\normalsize
\begin{verse}
Pu\b ravam\index{gnl}{Puravam@Pu\b ravam}, Cirapuram\index{gnl}{Cirapuram}, T\=o\d nipuram\index{gnl}{Tonipuram@T\=o\d nipuram}, Ca\d npai\index{gnl}{Canpai@Ca\d npai}, l'excellente Pukali\index{gnl}{Pukali},\\
K\=a\b li\index{gnl}{Kali@K\=a\b li}, Koccai\index{gnl}{Koccai}vayam aux jardins o\`u abonde le miel, Tar\=ay\index{gnl}{Taray@Tar\=ay},\\
La ville de Celui aux quatre visages,\\
Ve\.nkuru\index{gnl}{Venkuru@Ve\.nkuru} la victorieuse [et], V\=e\d nupuram\index{gnl}{Venupuram@V\=e\d nupuram}, [c'est]\\
Ka\b lumalam\index{gnl}{Kalumalam@Ka\b lumalam}, la ville que nous rejoignons\\
De Celui qui, ayant lanc\'e une fl\`eche sur Vijaya\b n\index{gnl}{Vijaya\b n} (Arjuna\index{gnl}{Arjuna}),\\
Tua le démon\index{gnl}{demon@démon} avec vaillance. (II 70.8)\\
\end{verse}

\scriptsize
\begin{verse}
\textit{ca\d npai\index{gnl}{Canpai@Ca\d npai}, piramapuram\index{gnl}{Piramapuram}, ta\d n pukali\index{gnl}{Pukali}, ve\.nkuru\index{gnl}{Venkuru@Ve\.nkuru}, nal k\=a\b li, c\=ay\=ap\\
pa\d npu \=ar cirapuram\index{gnl}{Cirapuram}um, koccai\index{gnl}{Koccai}vayam, tar\=ay\index{gnl}{Taray@Tar\=ay}, pu\b ravam\index{gnl}{Puravam@Pu\b ravam}, p\=arm\=el\\
na\d npu \=ar ka\b lumalam\index{gnl}{Kalumalam@Ka\b lumalam}, c\=\i r v\=e\d nupuram\index{gnl}{Venupuram@V\=e\d nupuram}, t\=o\d nipuram\index{gnl}{Tonipuram@T\=o\d nipuram}---n\=a\d n il\=ata\\
ve\d npal cama\d naro\d tu c\=akkiyarai viyappu a\b litta vimala\b n \=ur\=e.} (II 70.9)\\
\end{verse}
\normalsize
\begin{verse}
Ca\d npai\index{gnl}{Canpai@Ca\d npai}, Piramapuram\index{gnl}{Piramapuram}, la fra\^iche Pukali\index{gnl}{Pukali}, Ve\.nkuru\index{gnl}{Venkuru@Ve\.nkuru},\\
La bonne K\=a\b li\index{gnl}{Kali@K\=a\b li}, Cirapuram\index{gnl}{Cirapuram} pleine de qualit\'e qui ne faillit pas,\\
Koccai\index{gnl}{Koccai}vayam, Tar\=ay\index{gnl}{Taray@Tar\=ay}, Pu\b ravam\index{gnl}{Puravam@Pu\b ravam},\\
Ka\b lumalam\index{gnl}{Kalumalam@Ka\b lumalam} pleine d'amour sur la terre\index{gnl}{terre},\\
La glorieuse V\=e\d nupuram\index{gnl}{Venupuram@V\=e\d nupuram} [et] T\=o\d nipuram\index{gnl}{Tonipuram@T\=o\d nipuram} [c'est]\\
La ville du Pur qui a d\'etruit, en col\`ere,\\
Les bouddhiste\index{gnl}{bouddhiste}s et les ja\"in\index{gnl}{jain@ja\"in}s\\
Aux dents blanches, sans pudeur. (II 70.9)\\
\end{verse}

\scriptsize
\begin{verse}
\textit{ce\b lu maliya p\=u\.n k\=a\b li, pu\b ravam\index{gnl}{Puravam@Pu\b ravam}, cirapuram\index{gnl}{Cirapuram}, c\=\i rp pukali\index{gnl}{Pukali}, ceyya\\
ko\b lumalar\=a\b n na\b nnakaram, t\=o\d nipuram\index{gnl}{Tonipuram@T\=o\d nipuram}, koccai\index{gnl}{Koccai}vayam, ca\d npai\index{gnl}{Canpai@Ca\d npai}, \=aya\\
vi\b lumiya c\=\i r ve\.nkuruvo\d tu, \=o\.nku tar\=ay\index{gnl}{Taray@Tar\=ay}, v\=e\d nupuram\index{gnl}{Venupuram@V\=e\d nupuram}, miku nal m\=a\d tak\\
ka\b lumalam\index{gnl}{Kalumalam@Ka\b lumalam}, e\b n\b ru i\b n\b na peyarpa\b n\b nira\d n\d tum---ka\d nnutal\=a\b n karutum \=ur\=e.} (II 70.10)\\
\end{verse}
\normalsize
\begin{verse}
La belle K\=a\b li\index{gnl}{Kali@K\=a\b li} o\`u fleurit la beaut\'e, Pu\b ravam\index{gnl}{Puravam@Pu\b ravam},\\
Cirapuram\index{gnl}{Cirapuram}, la glorieuse Pukali\index{gnl}{Pukali},\\
La bonne ville de Celui de la pulpeuse fleur rouge,\\
T\=o\d nipuram\index{gnl}{Tonipuram@T\=o\d nipuram}, Koccai\index{gnl}{Koccai}vayam, Ca\d npai\index{gnl}{Canpai@Ca\d npai},\\
Avec Ve\.nkuru\index{gnl}{Venkuru@Ve\.nkuru} \`a la gloire excellente, la haute Tar\=ay\index{gnl}{Taray@Tar\=ay},\\
V\=e\d nupuram\index{gnl}{Venupuram@V\=e\d nupuram} [et] Ka\b lumalam\index{gnl}{Kalumalam@Ka\b lumalam} aux nombreuses belles maisons\\
Sont les douze\index{gnl}{douze} noms de la ville\\
Qu'estime Celui \`a l'oeil frontal. (II 70.10)\\
\end{verse}

\scriptsize
\begin{verse}
\textit{koccai\index{gnl}{Koccai}vayam, pirama\b n \=ur, pukali\index{gnl}{Pukali}, ve\.nkuru\index{gnl}{Venkuru@Ve\.nkuru}, pu\b ravam\index{gnl}{Puravam@Pu\b ravam}, k\=a\b li,\\
niccal vi\b lavu \=ov\=a n\=\i \d tu \=ar cirapuram\index{gnl}{Cirapuram}, n\=\i \d l ca\d npaim\=ut\=ur,\\
naccu i\b niya p\=untar\=ay\index{gnl}{Taray@Tar\=ay}, v\=e\d nupuram\index{gnl}{Venupuram@V\=e\d nupuram}, t\=o\d nipuram\index{gnl}{Tonipuram@T\=o\d nipuram}, \=aki namm\=el\\
acca\.nka\d l t\=\i rttu aru\d lum amm\=a\b n ka\b lumalam\index{gnl}{Kalumalam@Ka\b lumalam}---n\=am amarum \=ur\=e.} (II 70.11)\\
\end{verse}
\normalsize
\begin{verse}
Koccai\index{gnl}{Koccai}vayam, la ville de Brahm\=a\index{gnl}{Brahma@Brahm\=a}, Pukali\index{gnl}{Pukali}, Ve\.nkuru\index{gnl}{Venkuru@Ve\.nkuru}, Pu\b ravam\index{gnl}{Puravam@Pu\b ravam},\\
K\=a\b li\index{gnl}{Kali@K\=a\b li}, la haute Cirapuram\index{gnl}{Cirapuram} o\`u ne cessent jamais les f\^etes\index{gnl}{fete@fête},\\
L'antique ville de la haute Ca\d npai\index{gnl}{Canpai@Ca\d npai}, la douce et d\'esirable belle Tar\=ay\index{gnl}{Taray@Tar\=ay},\\
V\=e\d nupuram\index{gnl}{Venupuram@V\=e\d nupuram} [et] T\=o\d nipuram\index{gnl}{Tonipuram@T\=o\d nipuram} [c'est]\\
Ka\b lumalam\index{gnl}{Kalumalam@Ka\b lumalam}, la ville o\`u nous nous installons\\
Du P\`ere qui, rem\'ediant \`a nos peurs, accorde gr\^ace. (II 70.11)\\
\end{verse}

\scriptsize
\begin{verse}
\textit{k\=avimalar puraiyum ka\d n\d n\=ar ka\b lumalatti\b n peyarai n\=a\d lum\\
p\=aviya c\=\i rp pa\b n\b nira\d n\d tum na\b nn\=ul\=ap pattimaiy\=al pa\b nuvalm\=alai\\
n\=avi\b n nalam puka\b l c\=\i r n\=alma\b raiy\=a\b n ñ\=a\b nacampanta\b n co\b n\b na\\
m\=evi icai mo\b liv\=ar vi\d n\d navaril e\d n\d nutalai viruppu u\d l\=ar\=e.} (II 70.12)\\
\end{verse}
\normalsize
\begin{verse}
Les noms de Ka\b lumalam\index{gnl}{Kalumalam@Ka\b lumalam} [des femme\index{gnl}{femme}s]\\
Aux yeux semblables aux nélombos bleus,\\
Les douze\index{gnl}{douze} [noms] glorieux r\'epandues tous les jours, \\
La guirlande\index{gnl}{guirlande} de strophes [compos\'ee]\\
Avec la d\'evotion\index{gnl}{devotion@dévotion} des bons livres,\\
Que dit celui des quatre \textit{Veda}\index{gnl}{Veda@\textit{Veda}}\\
\`A la gloire r\'epandue par la bont\'e de sa langue,\\
\~N\=a\b nacampanta\b n\index{gnl}{Campantar!N\=a\b nacampanta\b n@\~N\=a\b nacampanta\b n},\\
Ceux qui disent [ces noms] avec plaisir en musique\index{gnl}{musique}\\
Auront le plaisir de compter parmi les c\'elestes. (II 70.12)\\
\end{verse}

\subsection{Hymne II 73}

L'hymne\index{gnl}{hymne} II 73, compos\'e aussi selon le procédé\index{gnl}{procédé littéraire} litt\'eraire du \textit{cakkaram\=a\b r\b ru}\index{gnl}{cakkaramarru@\textit{cakkaram\=a\b r\b ru}}, r\'epond ici \`a la d\'efinition donn\'ee par le \textit{Tamil Lexicon}. Le terme \textit{cakkaram} (sk. \textit{cakra}), \og cercle\fg, est d'ailleurs mentionn\'e dans l'envoi\index{gnl}{envoi}. Nous relevons des allusions aux légende\index{gnl}{legende@légende}s de Piramapuram\index{gnl}{Piramapuram} (toutes les strophes), V\=e\d nupuram\index{gnl}{Venupuram@V\=e\d nupuram} (st. 2, 3, 4, 5, 6, 7, 9, 10 et 11), Pukali\index{gnl}{Pukali} (st. 9), Ve\.nkuru\index{gnl}{Venkuru@Ve\.nkuru} (st. 12) et Cirapuram\index{gnl}{Cirapuram} (st. 2, 3, 4, 5, 8, 9, 10 et 12).

\scriptsize
\begin{verse}
\textit{vi\d la\.nkiya c\=\i rp pirama\b n \=ur, v\=e\d nupuram\index{gnl}{Venupuram@V\=e\d nupuram}, pukali\index{gnl}{Pukali}, ve\.nkuru\index{gnl}{Venkuru@Ve\.nkuru}, m\=el c\=olai\\
va\d lam kavarum t\=o\d nipuram\index{gnl}{Tonipuram@T\=o\d nipuram}, p\=untar\=ay\index{gnl}{Taray@Tar\=ay}, cirapuram\index{gnl}{Cirapuram}, va\d n pu\b ravam\index{gnl}{Puravam@Pu\b ravam}, ma\d nm\=el\\
ka\d la\.nkam il \=urca\d npai, kama\b l k\=a\b li, vayamkoccai\index{gnl}{Koccai}, ka\b lumalam\index{gnl}{Kalumalam@Ka\b lumalam}, e\b n\b ru i\b n\b na---\\
i\d la\.nkumara\b nta\b n\b naip pe\b r\b ru, imaiyavartam pakai e\b rivitta i\b raiva\b n \=ur\=e.} (II 73.1)\\
\end{verse}
\normalsize
\begin{verse}
La ville prosp\`ere et illustre de Brahm\=a\index{gnl}{Brahma@Brahm\=a}, V\=e\d nupuram\index{gnl}{Venupuram@V\=e\d nupuram}, Pukali\index{gnl}{Pukali},\\
Ve\.nkuru\index{gnl}{Venkuru@Ve\.nkuru}, T\=o\d nipuram\index{gnl}{Tonipuram@T\=o\d nipuram} qui captive la fertilit\'e des jardins excellents,\\
La belle Tar\=ay\index{gnl}{Taray@Tar\=ay}, Cirapuram\index{gnl}{Cirapuram}, Pu\b ravam\index{gnl}{Puravam@Pu\b ravam} la lib\'erale,\\
Ca\d npai\index{gnl}{Canpai@Ca\d npai} la ville sans trouble sur terre\index{gnl}{terre}, K\=a\b li\index{gnl}{Kali@K\=a\b li} la fleurie,\\
La terre\index{gnl}{terre} de Koccai\index{gnl}{Koccai} [et] Ka\b lumalam\index{gnl}{Kalumalam@Ka\b lumalam} [c'est] ainsi\\
La ville du seigneur qui, en obtenant le jeune Kumara,\\
Fit d\'etruire l'ennemi des c\'elestes. (II 73.1)\\
\end{verse}

\scriptsize
\begin{verse}
\textit{tiru va\d larum ka\b lumalam\index{gnl}{Kalumalam@Ka\b lumalam}\=e, koccai\index{gnl}{Koccai}, t\=ev\=entira\b n\=ur, aya\b n\=ur, teyvat-\\
taru va\d larum po\b lil pu\b ravam\index{gnl}{Puravam@Pu\b ravam}, cilampa\b n\index{gnl}{Cilampan@Cilampa\b n}\=ur, k\=a\b li, taku ca\d npai\index{gnl}{Canpai@Ca\d npai}, o\d n p\=a\\
uru va\d lar ve\.nkuru\index{gnl}{Venkuru@Ve\.nkuru}, pukali\index{gnl}{Pukali}, \=o\.nku tar\=ay\index{gnl}{Taray@Tar\=ay}, t\=o\d nipuram\index{gnl}{Tonipuram@T\=o\d nipuram}---uyarnta t\=evar\\
veruva, va\d lar ka\d talvi\d tamatu u\d n\d tu a\d ni ko\d l ka\d n\d tatt\=o\b n virumpum \=ur\=e.} (II 73.2)\\
\end{verse}
\normalsize
\begin{verse}
Ka\b lumalam\index{gnl}{Kalumalam@Ka\b lumalam} o\`u cro\^it la prosp\'erit\'e, Koccai\index{gnl}{Koccai},\\
La ville du roi\index{gnl}{roi} des dieux, la ville d'Aya\b n\index{gnl}{Brahma@Brahm\=a!Aya\b n},\\
Pu\b ravam\index{gnl}{Puravam@Pu\b ravam} aux jardins o\`u croissent des arbres divins,\\
La ville de Cilampa\b n\index{gnl}{Cilampan@Cilampa\b n}, K\=a\b li\index{gnl}{Kali@K\=a\b li}, Ca\d npai\index{gnl}{Canpai@Ca\d npai} l'excellente,\\
Ve\.nkuru\index{gnl}{Venkuru@Ve\.nkuru} o\`u cro\^it la musique\index{gnl}{musique} des chants\index{gnl}{chant}\index{gnl}{chant} beaux, Pukali\index{gnl}{Pukali},\\
Tar\=ay\index{gnl}{Taray@Tar\=ay} qui prosp\`ere [et] T\=o\d nipuram\index{gnl}{Tonipuram@T\=o\d nipuram} [c'est]\\
La ville qu'aime Celui \`a la gorge orn\'ee\\
Qui avala le poison de la mer\index{gnl}{mer} grandissante\\
Alors que les grands dieux avaient peur. (II 73.2)\\
\end{verse}

\scriptsize
\begin{verse}
\textit{v\=aynta puka\b l ma\b rai va\d larum t\=o\d nipuram\index{gnl}{Tonipuram@T\=o\d nipuram}, p\=untar\=ay\index{gnl}{Taray@Tar\=ay}, cilampa\b n\index{gnl}{Cilampan@Cilampa\b n} v\=a\b l \=ur,\\
\=eynta pu\b ravam\index{gnl}{Puravam@Pu\b ravam}, tika\b lum ca\d npai\index{gnl}{Canpai@Ca\d npai}, e\b lil k\=a\b li, i\b rai koccai\index{gnl}{Koccai}, am po\b n\\
v\=eynta matil ka\b lumalam\index{gnl}{Kalumalam@Ka\b lumalam}, vi\d n\d n\=or pa\d niya mikka(a)ya\b n\=ur, amarark\=o\b n\=ur,\\
\=aynta kalai \=ar pukali\index{gnl}{Pukali}, ve\.nkuruatu---ara\b n n\=a\d lum amarum \=ur\=e.} (II 73.3)\\
\end{verse}
\normalsize
\begin{verse}
T\=o\d nipuram\index{gnl}{Tonipuram@T\=o\d nipuram} o\`u croissent les \textit{Veda}\index{gnl}{Veda@\textit{Veda}} \`a la gloire pleine,\\
La belle Tar\=ay\index{gnl}{Taray@Tar\=ay}, la ville o\`u vit Cilampa\b n\index{gnl}{Cilampan@Cilampa\b n},\\
Pu\b ravam\index{gnl}{Puravam@Pu\b ravam} appropri\'ee [\`a \'Siva\index{gnl}{Siva@\'Siva}], Ca\d npai\index{gnl}{Canpai@Ca\d npai} qui brille,\\
La belle K\=a\b li\index{gnl}{Kali@K\=a\b li}, Koccai\index{gnl}{Koccai} du seigneur, \\
Ka\b lumalam\index{gnl}{Kalumalam@Ka\b lumalam} aux fortifications couvertes de bel or,\\
La ville du grand Aya\b n\index{gnl}{Brahma@Brahm\=a!Aya\b n} honor\'e par les c\'elestes,\\
La ville du roi\index{gnl}{roi} des immortels,\\
Pukali\index{gnl}{Pukali} pleine d'arts choisis [et] Ve\.nkuru\index{gnl}{Venkuru@Ve\.nkuru} [c'est]\\
La ville o\`u r\'eside tous les jours Hara\index{gnl}{Hara}. (II 73.3)\\
\end{verse}

\scriptsize
\begin{verse}
\textit{m\=amalaiy\=a\d lka\d nava\b n maki\b l ve\.nkuru\index{gnl}{Venkuru@Ve\.nkuru}, m\=ap pukali\index{gnl}{Pukali}, tar\=ay\index{gnl}{Taray@Tar\=ay}, t\=o\d nipuram\index{gnl}{Tonipuram@T\=o\d nipuram}, v\=a\b n\\
c\=ema matil pu\d tai tika\b lum ka\b lumalam\index{gnl}{Kalumalam@Ka\b lumalam}\=e, koccai\index{gnl}{Koccai}, t\=ev\=entira\b n\=ur, c\=\i rp\\
p\=umaka\b n\=ur, polivu u\d taiya pu\b ravam\index{gnl}{Puravam@Pu\b ravam}, vi\b ral cilampa\b n\index{gnl}{Cilampan@Cilampa\b n}\=ur, k\=a\b li, ca\d npai\index{gnl}{Canpai@Ca\d npai}---\\
p\=a maruvu kalaie\d t\d tu e\d t\d tu u\d narntu, ava\b r\b ri\b n paya\b n nukarv\=or paravum \=ur\=e.} (II 73.4)\\
\end{verse}
\normalsize
\begin{verse}
Ve\.nkuru\index{gnl}{Venkuru@Ve\.nkuru} dont se r\'ejouit l'\'epoux de Celle de la grande montagne,\\
La grande Pukali\index{gnl}{Pukali}, Tar\=ay\index{gnl}{Taray@Tar\=ay}, T\=o\d nipuram\index{gnl}{Tonipuram@T\=o\d nipuram}, Ka\b lumalam\index{gnl}{Kalumalam@Ka\b lumalam}\\
O\`u brille la place des grandes fortifications protectrices,\\
Koccai\index{gnl}{Koccai}, la ville du roi\index{gnl}{roi} des dieux, la ville du fils glorieux de la fleur,\\
Pu\b ravam\index{gnl}{Puravam@Pu\b ravam} qui poss\`ede la beaut\'e,\\
La ville de Cilampa\b n\index{gnl}{Cilampan@Cilampa\b n} le puissant, K\=a\b li\index{gnl}{Kali@K\=a\b li} [et] Ca\d npai\index{gnl}{Canpai@Ca\d npai} [c'est]\\
La ville que prient ceux qui jouissent du fruit d'avoir exp\'eriment\'e\\
Les huit fois huit (soixante-quatre) arts\\
D\'ecrits dans les chants\index{gnl}{chant}\index{gnl}{chant}. (II 73.4)\\
\end{verse}

\scriptsize
\begin{verse}
\textit{taraitt\=evar pa\d ni ca\d npai\index{gnl}{Canpai@Ca\d npai}, tami\b lk k\=a\b li, vayamkoccai\index{gnl}{Koccai}, taya\.nku p\=um\=el\\
viraic c\=erum ka\b lumalam\index{gnl}{Kalumalam@Ka\b lumalam}, mey u\d narnta(a)ya\b n\=ur, vi\d n\d navartamk\=o\b n\=ur, ve\b n\b rit\\
tiraic c\=erum pu\b nal pukali\index{gnl}{Pukali}, ve\.nkuru\index{gnl}{Venkuru@Ve\.nkuru}, celvam peruku t\=o\d nipuram\index{gnl}{Tonipuram@T\=o\d nipuram}, c\=\i r\\
uraic c\=er p\=untar\=ay\index{gnl}{Taray@Tar\=ay}, cilampa\b n\index{gnl}{Cilampan@Cilampa\b n}\=ur, pu\b ravam\index{gnl}{Puravam@Pu\b ravam}---ulakattil uyarnta \=ur\=e.} (II 73.5)\\
\end{verse}
\normalsize
\begin{verse}
Ca\d npai\index{gnl}{Canpai@Ca\d npai} o\`u servent les dieux terrestres (brahmane\index{gnl}{brahmane}s),\\
K\=a\b li\index{gnl}{Kali@K\=a\b li} la tamoule, la terre\index{gnl}{terre} de Koccai\index{gnl}{Koccai},\\
Ka\b lumalam\index{gnl}{Kalumalam@Ka\b lumalam} au parfum de fleurs brillantes,\\
La ville d'Aya\b n\index{gnl}{Brahma@Brahm\=a!Aya\b n} qui a exp\'eriment\'e la v\'erit\'e,\\
La ville du roi\index{gnl}{roi} des c\'elestes,\\
Pukali\index{gnl}{Pukali} la maritime que rejoignent les vague\index{gnl}{vague}s victorieuses,\\
Ve\.nkuru\index{gnl}{Venkuru@Ve\.nkuru}, T\=o\d nipuram\index{gnl}{Tonipuram@T\=o\d nipuram} o\`u cro\^it la richesse,\\
La belle Tar\=ay\index{gnl}{Taray@Tar\=ay} o\`u vient r\'esider la gloire,\\
La ville de Cilampa\b n\index{gnl}{Cilampan@Cilampa\b n} [et] Pu\b ravam\index{gnl}{Puravam@Pu\b ravam} [c'est]\\
La ville la plus haute du monde. (II 73.5)\\
\end{verse}

\scriptsize
\begin{verse}
\textit{pu\d n\d tarikattu \=ar vayal c\=u\b l pu\b ravam\index{gnl}{Puravam@Pu\b ravam}, miku cirapuram\index{gnl}{Cirapuram}, p\=u\.n k\=a\b li, ca\d npai\index{gnl}{Canpai@Ca\d npai},\\
e\d nticaiy\=or i\b raiñciya ve\.nkuru\index{gnl}{Venkuru@Ve\.nkuru}, pukali\index{gnl}{Pukali}, p\=untar\=ay\index{gnl}{Taray@Tar\=ay}, t\=o\d nipuram\index{gnl}{Tonipuram@T\=o\d nipuram}, c\=\i r\\
va\d n\d tu amarum po\b lil malku ka\b lumalam\index{gnl}{Kalumalam@Ka\b lumalam}, nal koccai\index{gnl}{Koccai}, v\=a\b navartamk\=o\b n\=ur,\\
a\d n\d tu aya\b n\=ur, ivai e\b npar---aru\.nk\=u\b r\b rai utaittu ukanta appa\b n ur\=e.} (II 73.6)\\
\end{verse}
\normalsize
\begin{verse}
Pu\b ravam\index{gnl}{Puravam@Pu\b ravam} entour\'ee de rizi\`eres pleines de lotus,\\
Cirapuram\index{gnl}{Cirapuram} l'excellente, K\=a\b li\index{gnl}{Kali@K\=a\b li} la fleurie, Ca\d npai\index{gnl}{Canpai@Ca\d npai},\\
Ve\.nkuru\index{gnl}{Venkuru@Ve\.nkuru} r\'ev\'er\'ee par ceux de toutes les directions,\\
Pukali\index{gnl}{Pukali}, la belle Tar\=ay\index{gnl}{Taray@Tar\=ay}, T\=o\d nipuram\index{gnl}{Tonipuram@T\=o\d nipuram}, Ka\b lumalam\index{gnl}{Kalumalam@Ka\b lumalam}\\
Emplie de jardins o\`u r\'esident les belles abeilles,\\
La bonne Koccai\index{gnl}{Koccai}, la ville du roi\index{gnl}{roi} des c\'elestes\\
\lbrack Et] la ville du Seigneur Aya\b n\index{gnl}{Brahma@Brahm\=a!Aya\b n},\\
On dit que c'est la ville du P\`ere qui se r\'ejouit\\
Ayant donn\'e un coup de pied au grand K\=u\b r\b ru\index{gnl}{Kurru@K\=u\b r\b ru}. (II 73.6)\\
\end{verse}

\scriptsize
\begin{verse}
\textit{va\d nmai va\d lar varattu aya\b n\=ur, v\=a\b navartam-k\=o\b n\=ur, va\d n pukali\index{gnl}{Pukali}, iñci\\
ve\d nmati c\=er ve\.nkuru\index{gnl}{Venkuru@Ve\.nkuru}, mikk\=or i\b raiñcu ca\d npai\index{gnl}{Canpai@Ca\d npai}, viya\b nk\=a\b li, koccai\index{gnl}{Koccai},\\
ka\d n maki\b lum ka\b lumalam\index{gnl}{Kalumalam@Ka\b lumalam}, ka\b r\b r\=or puka\b lum t\=o\d nipuram\index{gnl}{Tonipuram@T\=o\d nipuram}, p\=untar\=ay\index{gnl}{Taray@Tar\=ay}, c\=\i rp\\
pa\d n maliyum cirapuram\index{gnl}{Cirapuram}, p\=ar puka\b l pu\b ravam\index{gnl}{Puravam@Pu\b ravam}-p\=alva\d n\d na\b n payilum \=ur\=e.} (II 73.7)\\
\end{verse}
\normalsize
\begin{verse}
La ville d'Aya\b n\index{gnl}{Brahma@Brahm\=a!Aya\b n} au don\index{gnl}{don} qui fait cro\^itre la lib\'eralit\'e,\\
La ville du roi\index{gnl}{roi} des c\'elestes, la fertile Pukali\index{gnl}{Pukali},\\
Ve\.nkuru\index{gnl}{Venkuru@Ve\.nkuru} o\`u les fortifications joignent la lune blanche,\\
Ca\d npai\index{gnl}{Canpai@Ca\d npai} que r\'ev\`erent les grands,\\
K\=a\b li\index{gnl}{Kali@K\=a\b li} la vaste, Koccai\index{gnl}{Koccai},\\
Ka\b lumalam\index{gnl}{Kalumalam@Ka\b lumalam} dont se r\'ejouissent les yeux,\\
T\=o\d nipuram\index{gnl}{Tonipuram@T\=o\d nipuram} que louent les \'erudits, la belle Tar\=ay\index{gnl}{Taray@Tar\=ay},\\
Cirapuram\index{gnl}{Cirapuram} o\`u fleurit la musique\index{gnl}{musique} glorieuse\\
\lbrack Et] Pu\b ravam\index{gnl}{Puravam@Pu\b ravam} que loue le monde [c'est]\\
La ville o\`u demeure la couleur lait\index{gnl}{lait} [des cendre\index{gnl}{cendre}s]. (II 73.7)\\
\end{verse}

\scriptsize
\begin{verse}
\textit{m\=o\d ti pu\b ra\.nk\=akkum \=ur, pu\b ravam\index{gnl}{Puravam@Pu\b ravam} c\=\i rc cilampa\b n\=ur, k\=a\b lim\=ut\=ur,\\
n\=\i \d tu iyalum ca\d npai\index{gnl}{Canpai@Ca\d npai}, ka\b lumalam\index{gnl}{Kalumalam@Ka\b lumalam}, koccai\index{gnl}{Koccai}, v\=e\d nupuram\index{gnl}{Venupuram@V\=e\d nupuram}, kamalam n\=\i \d tu\\
k\=u\d tiya(a)ya\b n\=ur, va\d lar ve\.nkuru\index{gnl}{Venkuru@Ve\.nkuru}, pukali\index{gnl}{Pukali}, tar\=ay\index{gnl}{Taray@Tar\=ay}, t\=o\d nipuram\index{gnl}{Tonipuram@T\=o\d nipuram}-k\=u\d tap p\=or\\
t\=e\d ti u\b lal avu\d nar payil tiripura\.nka\d l ce\b r\b ra malaiccilaiya\b n \=ur\=e.} (II 73.8)\\
\end{verse}
\normalsize
\begin{verse}
La ville gard\'ee aux fronti\`eres par M\=o\d ti\index{gnl}{Moti@M\=o\d ti}, Pu\b ravam\index{gnl}{Puravam@Pu\b ravam},\\
La ville de Cilampa\b n\index{gnl}{Cilampan@Cilampa\b n} le glorieux, l'antique ville de K\=a\b li\index{gnl}{Kali@K\=a\b li},\\
Ca\d npai\index{gnl}{Canpai@Ca\d npai} qui demeure permanente, Ka\b lumalam\index{gnl}{Kalumalam@Ka\b lumalam}, Koccai\index{gnl}{Koccai},\\
V\=e\d nupuram\index{gnl}{Venupuram@V\=e\d nupuram}, la ville d'Aya\b n\index{gnl}{Brahma@Brahm\=a!Aya\b n} uni \'eternellement au lotus,\\
Ve\.nkuru\index{gnl}{Venkuru@Ve\.nkuru} qui cro\^it, Pukali\index{gnl}{Pukali}, Tar\=ay\index{gnl}{Taray@Tar\=ay} [et] T\=o\d nipuram\index{gnl}{Tonipuram@T\=o\d nipuram} [c'est]\\
La ville de Celui \`a l'arc de montagne qui d\'etruit\\
Les trois citadelles o\`u r\'esident les d\'emons\\
Qui errent cherchant \`a faire la guerre. (II 73.8)\\
\end{verse}

\scriptsize
\begin{verse}
\textit{irakka(m) u\d tai i\b raiyava\b n \=ur t\=o\d nipuram\index{gnl}{Tonipuram@T\=o\d nipuram}, p\=untar\=ay\index{gnl}{Taray@Tar\=ay}, cilampa\b nta\b n \=ur,\\
nirakka varupu\b nal pu\b ravam\index{gnl}{Puravam@Pu\b ravam}, ni\b n\b ra tavattu aya\b n\=ur, c\=\i rt t\=evark\=o\b n\=ur,\\
varak karav\=ap pukali\index{gnl}{Pukali}, ve\.nkuru\index{gnl}{Venkuru@Ve\.nkuru}, m\=acu il\=ac ca\d npai\index{gnl}{Canpai@Ca\d npai}, k\=a\b li, koccai\index{gnl}{Koccai},\\
arakka\b n vi\b ral a\b littuaru\d li ka\b lumalam\index{gnl}{Kalumalam@Ka\b lumalam}---anta\d nar v\=etam a\b r\=ata \=ur\=e.} (II 73.9)\\
\end{verse}
\normalsize
\begin{verse}
T\=o\d nipuram\index{gnl}{Tonipuram@T\=o\d nipuram} la ville du Seigneur qui a de la compassion,\\
La belle Tar\=ay\index{gnl}{Taray@Tar\=ay}, la ville de Cilampa\b n\index{gnl}{Cilampan@Cilampa\b n},\\
Pu\b ravam\index{gnl}{Puravam@Pu\b ravam} aux eaux\index{gnl}{eau} qui viennent continuellement,\\
La ville d'Aya\b n\index{gnl}{Brahma@Brahm\=a!Aya\b n} \`a l'asc\`ese permanente,\\
La ville du roi\index{gnl}{roi} des dieux glorieux,\\
Pukali\index{gnl}{Pukali} qui ne se cache pas quand on vient [s'y réfugier],\\
Ve\.nkuru\index{gnl}{Venkuru@Ve\.nkuru}, Ca\d npai\index{gnl}{Canpai@Ca\d npai} sans tache, K\=a\b li\index{gnl}{Kali@K\=a\b li}, Koccai\index{gnl}{Koccai} [et] Ka\b lumalam\index{gnl}{Kalumalam@Ka\b lumalam},\\
De Celui qui, d\'etruisant la force du démon\index{gnl}{demon@démon}, accorda gr\^ace, [c'est]\\
La ville o\`u ne cesse le [chant] des \textit{Veda}\index{gnl}{Veda@\textit{Veda}} des brahmane\index{gnl}{brahmane}s. (II 73.9)\\
\end{verse}

\scriptsize
\begin{verse}
\textit{m\=el \=otum ka\b lumalam\index{gnl}{Kalumalam@Ka\b lumalam}, meyttavam va\d larum Koccai\index{gnl}{koccai}, intira\b n\=ur, meymmai\\
n\=ul \=otum aya\b nta\b n\=ur, nu\d n a\b riv\=ar kuru, pukali\index{gnl}{Pukali}, tar\=ay\index{gnl}{Taray@Tar\=ay}, t\=u n\=\i rm\=el\\
c\=el \=o\d tu t\=o\d nipuram\index{gnl}{Tonipuram@T\=o\d nipuram}, tika\b l pu\b ravam\index{gnl}{Puravam@Pu\b ravam}, cilampa\b n\=ur, ceruc ceytu a\b n\b ru\\
m\=al\=o\d tum aya\b n a\b riy\=a\b n va\d n k\=a\b li, ca\d npai\index{gnl}{Canpai@Ca\d npai}---ma\d n\d n\=or v\=a\b lttum \=ur\=e.} (II 73.10)\\
\end{verse}
\normalsize
\begin{verse}
Ka\b lumalam\index{gnl}{Kalumalam@Ka\b lumalam} lou\'ee comme sup\'erieure,\\
Koccai\index{gnl}{Koccai} o\`u cro\^it la v\'eritable asc\`ese, la ville d'Indra\index{gnl}{Indra},\\
La ville d'Aya\b n\index{gnl}{Brahma@Brahm\=a!Aya\b n} qui loue le livre de la v\'erit\'e,\\
\lbrack Ve\.n]kuru de ceux \`a la connaissance\index{gnl}{connaissance} aiguis\'ee, Pukali\index{gnl}{Pukali}, Tar\=ay\index{gnl}{Taray@Tar\=ay},\\
T\=o\d nipuram\index{gnl}{Tonipuram@T\=o\d nipuram} o\`u courent les poissons \textit{c\=el} dans les eaux\index{gnl}{eau} pures,\\
Pu\b ravam\index{gnl}{Puravam@Pu\b ravam} la brillante, la ville de Cilampa\b n\index{gnl}{Cilampan@Cilampa\b n}, la fertile K\=a\b li\index{gnl}{Kali@K\=a\b li}\\
De Celui qui n'a pas \'et\'e vu par M\=al\index{gnl}{Visnu@Vi\d s\d nu!Mal@M\=al} et Aya\b n\index{gnl}{Brahma@Brahm\=a!Aya\b n}\\
Jadis en comp\'etition, [et] Ca\d npai\index{gnl}{Canpai@Ca\d npai} [c'est]\\
La ville que louent les habitants de la terre\index{gnl}{terre}. (II 73.10)\\
\end{verse}

\scriptsize
\begin{verse}
\textit{\=akku amar c\=\i r \=ur ca\d npai\index{gnl}{Canpai@Ca\d npai}, k\=a\b li, amar koccai\index{gnl}{Koccai}, ka\b lumalam\index{gnl}{Kalumalam@Ka\b lumalam}, a\b np\=a\b n \=ur\\
\=okkam(m) u\d tait t\=o\d nipuram\index{gnl}{Tonipuram@T\=o\d nipuram}, p\=untar\=ay\index{gnl}{Taray@Tar\=ay}, cirapuram\index{gnl}{Cirapuram}, o\d n pu\b ravam\index{gnl}{Puravam@Pu\b ravam}, na\d npu \=ar\\
p\=ukkamalatt\=o\b n maki\b l \=ur, purantara\b n\=ur, pukali\index{gnl}{Pukali}, ve\.nkuruvum, e\b npar-\\
c\=akkiyar\=o\d tu ama\.nkaiyart\=am a\b riy\=a vakai ni\b n\b r\=a\b n ta\.nkum \=ur\=e.} (II 73.11)\\
\end{verse}
\normalsize
\begin{verse}
Ca\d npai\index{gnl}{Canpai@Ca\d npai} la ville glorieuse o\`u demeure la cr\'eation,\\
K\=a\b li\index{gnl}{Kali@K\=a\b li}, Koccai\index{gnl}{Koccai} l'aim\'ee, Ka\b lumalam\index{gnl}{Kalumalam@Ka\b lumalam},\\
T\=o\d nipuram\index{gnl}{Tonipuram@T\=o\d nipuram} qui poss\`ede la grandeur, ville de Celui qui aime,\\
La belle Tar\=ay\index{gnl}{Taray@Tar\=ay}, Cirapuram\index{gnl}{Cirapuram}, Pu\b ravam\index{gnl}{Puravam@Pu\b ravam} la brillante,\\
La ville dont se r\'ejouit Celui de la fleur de lotus plein d'amour,\\
La ville de Purandara (Indra\index{gnl}{Indra}), Pukali\index{gnl}{Pukali} [et] Ve\.nkuru\index{gnl}{Venkuru@Ve\.nkuru}\\
On dit que c'est la ville o\`u r\'eside\\
Celui qui se tient de telle sorte\\
Que les bouddhiste\index{gnl}{bouddhiste}s et les n\'efastes (jaïns) ne le voient pas. (II 73.11)\\
\end{verse}

\scriptsize
\begin{verse}
\textit{akkaram c\=er taruma\b n\=ur, pukali\index{gnl}{Pukali}, tar\=ay\index{gnl}{Taray@Tar\=ay}, t\=o\d nipuram\index{gnl}{Tonipuram@T\=o\d nipuram}, a\d ni n\=\i rp poykaip\\
pukkaram c\=er pu\b ravam\index{gnl}{Puravam@Pu\b ravam}, c\=\i rc cilampa\b n\=ur, puka\b lk k\=a\b li, ca\d npai\index{gnl}{Canpai@Ca\d npai}, tol \=ur\\
mikkar am c\=\i rk ka\b lumalam\index{gnl}{Kalumalam@Ka\b lumalam}\=e, koccai\index{gnl}{Koccai}vayam, v\=e\d nupuram\index{gnl}{Venupuram@V\=e\d nupuram}, aya\b n\=ur, m\=el ic\\
cakkaram c\=\i rt tami\b lviraka\b nt\=a\b n co\b n\b na tami\b l taripp\=or tavam ceyt\=or\=e.} (II 73.12)\\
\end{verse}
\normalsize
\begin{verse}
La ville de Dharma\index{gnl}{Dharma} que rejoint la syllabe [primordiale],\\
Pukali\index{gnl}{Pukali}, Tar\=ay\index{gnl}{Taray@Tar\=ay}, T\=o\d nipuram\index{gnl}{Tonipuram@T\=o\d nipuram},\\
Pu\b ravam\index{gnl}{Puravam@Pu\b ravam} que rejoignent tout le temps les \'etangs aux belles eaux\index{gnl}{eau},\\
La ville de Cilampa\b n\index{gnl}{Cilampan@Cilampa\b n} le glorieux, K\=a\b li\index{gnl}{Kali@K\=a\b li} l'illustre,\\
Ca\d npai\index{gnl}{Canpai@Ca\d npai}, Ka\b lumalam\index{gnl}{Kalumalam@Ka\b lumalam}, ville antique et glorieuse pleine de grands,\\
La terre\index{gnl}{terre} de Koccai\index{gnl}{Koccai}, V\=e\d nupuram\index{gnl}{Venupuram@V\=e\d nupuram}, la ville d'Aya\b n\index{gnl}{Brahma@Brahm\=a!Aya\b n},\\
Ceux qui connaissent par coeur le tamoul\\
Dit par le glorieux expert tamoul\\
\`A propos du \textit{cakra} sur [les noms],\\
Ceux-l\`a ont fait p\'enitence. (II 73.12)\\
\end{verse}

\subsection{Hymne II 74}

Ce poème\index{gnl}{poeme@poème} est compos\'e selon le procédé\index{gnl}{procédé littéraire} litt\'eraire du \textit{k\=om\=uttiri\index{gnl}{komuttiri@\textit{k\=om\=uttiri}} ant\=ati}\index{gnl}{antati@\textit{ant\=ati}}, \og \textit{ant\=ati}\index{gnl}{antati@\textit{ant\=ati}} en urine de vache\fg\ (voir 2.1.3) qui est mentionn\'e dans l'envoi\index{gnl}{envoi}. Notons les r\'ef\'erences aux légende\index{gnl}{legende@légende}s de Piramapuram\index{gnl}{Piramapuram} (toutes les strophes), de V\=e\d nupuram\index{gnl}{Venupuram@V\=e\d nupuram} (st. 1, 2, 3, 4, 6, 8, 10, 11 et 12) et de Cirapuram\index{gnl}{Cirapuram} (st. 9 et 10).

\scriptsize
\begin{verse}
\textit{p\=umaka\b n\=ur, putt\=e\d lukkui\b raiva\b n\=ur, ku\b raivu il\=ap pukali\index{gnl}{Pukali}, p\=um\=el\\
m\=amaka\d l\=ur, ve\.nkuru\index{gnl}{Venkuru@Ve\.nkuru}, nal-t\=o\d nipuram\index{gnl}{Tonipuram@T\=o\d nipuram}, p\=untar\=ay\index{gnl}{Taray@Tar\=ay}, v\=aynta iñcic\\
c\=emam miku cirapuram\index{gnl}{Cirapuram}, c\=\i rp pu\b ravam\index{gnl}{Puravam@Pu\b ravam}, ni\b rai puka\b lc ca\d npai\index{gnl}{Canpai@Ca\d npai}, k\=a\b li, koccai\index{gnl}{Koccai},\\
k\=ama\b nai mu\b n k\=aynta nutalka\d n\d nava\b n \=ur ka\b lumalam\index{gnl}{Kalumalam@Ka\b lumalam}---n\=am karutum \=ur\=e.} (II 74.1)\\
\end{verse}
\normalsize
\begin{verse}
La ville du Fils de la fleur, la ville du Seigneur des dieux,\\
Pukali\index{gnl}{Pukali} sans manque, Ve\.nkuru\index{gnl}{Venkuru@Ve\.nkuru}, ville de la grande fille sur la fleur,\\
La bonne T\=o\d nipuram\index{gnl}{Tonipuram@T\=o\d nipuram}, la belle Tar\=ay\index{gnl}{Taray@Tar\=ay}, la grande Cirapuram\index{gnl}{Cirapuram}\\
\`A la protection des fortifications excellentes, \\
Pu\b ravam\index{gnl}{Puravam@Pu\b ravam} la glorieuse, Ca\d npai\index{gnl}{Canpai@Ca\d npai} \`a la gloire pleine,\\
K\=a\b li\index{gnl}{Kali@K\=a\b li}, Koccai\index{gnl}{Koccai} [et] Ka\b lumalam\index{gnl}{Kalumalam@Ka\b lumalam} [c'est]\\
Ville de Celui \`a l'\oe il frontal qui jadis consuma K\=ama\index{gnl}{Kama@K\=ama},\\
La ville que nous estimons. (II 74.1)\\
\end{verse}

\scriptsize
\begin{verse}
\textit{karuttu u\d taiya ma\b raiyavar c\=er ka\b lumalam\index{gnl}{Kalumalam@Ka\b lumalam}, meyt t\=o\d nipuram\index{gnl}{Tonipuram@T\=o\d nipuram}, ka\b naka m\=a\d ta\\
urut tika\b l ve\.nkuru\index{gnl}{Venkuru@Ve\.nkuru}, pukali\index{gnl}{Pukali}, \=o\.nku tar\=ay\index{gnl}{Taray@Tar\=ay}, ulaku \=arum koccai\index{gnl}{Koccai}, k\=a\b li,\\
tirut tika\b lum cirapuram\index{gnl}{Cirapuram}, t\=ev\=entira\b n\=ur, ce\.nkamalattu aya\b n\=ur, teyvat\\
tarut tika\b lum po\b lil pu\b ravam\index{gnl}{Puravam@Pu\b ravam}, ca\d npai\index{gnl}{Canpai@Ca\d npai}---ca\d taimu\d ti a\d n\d nal ta\.nkum \=ur\=e.} (II 74.2)\\
\end{verse}
\normalsize
\begin{verse}
Ka\b lumalam\index{gnl}{Kalumalam@Ka\b lumalam} que rejoignent ceux des \textit{Veda}\index{gnl}{Veda@\textit{Veda}} poss\'edant la doctrine,\\
La vraie T\=o\d nipuram\index{gnl}{Tonipuram@T\=o\d nipuram},\\
Ve\.nkuru\index{gnl}{Venkuru@Ve\.nkuru} o\`u brille le corps des maisons en or,\\
Pukali\index{gnl}{Pukali}, Tar\=ay\index{gnl}{Taray@Tar\=ay} qui grandit, Koccai\index{gnl}{Koccai} o\`u le monde se rassemble,\\
K\=a\b li\index{gnl}{Kali@K\=a\b li}, Cirapuram\index{gnl}{Cirapuram} o\`u brille la richesse, la ville du roi\index{gnl}{roi} des dieux,\\
La ville d'Aya\b n\index{gnl}{Brahma@Brahm\=a!Aya\b n} du lotus rouge,\\
Pu\b ravam\index{gnl}{Puravam@Pu\b ravam} aux jardins o\`u brillent les arbres divins [et] Ca\d npai\index{gnl}{Canpai@Ca\d npai} [c'est]\\
La ville o\`u r\'eside le Sup\'erieur aux cheveux en m\`eches. (II 74.2)\\
\end{verse}

\scriptsize
\begin{verse}
\textit{\=ur matiyaik katuva uyar matil ca\d npai\index{gnl}{Canpai@Ca\d npai}, o\d li maruvu k\=a\b li, koccai\index{gnl}{Koccai},\\
k\=ar maliyum po\b lil pu\d taic\=u\b l ka\b lumalam\index{gnl}{Kalumalam@Ka\b lumalam}, meyt t\=o\d nipuram\index{gnl}{Tonipuram@T\=o\d nipuram}, ka\b r\b r\=or \=ettum\\
c\=\i r maruvu p\=untar\=ay\index{gnl}{Taray@Tar\=ay}, cirapuram\index{gnl}{Cirapuram}, meyp pu\b ravam\index{gnl}{Puravam@Pu\b ravam}, aya\b n\=ur, p\=u\.n ka\b rpat\\
t\=ar maruvum intira\b n\=ur, pukali\index{gnl}{Pukali}, ve\.nkuru\index{gnl}{Venkuru@Ve\.nkuru}---ka\.nkai taritt\=o\b n \=ur\=e.} (II 74.3)\\
\end{verse}
\normalsize
\begin{verse}
Ca\d npai\index{gnl}{Canpai@Ca\d npai} aux hautes fortifications\\
Si bien qu'elles touchent la lune qui se meut,\\
K\=a\b li\index{gnl}{Kali@K\=a\b li} qu'embrasse la lumi\`ere, Koccai\index{gnl}{Koccai},\\
Ka\b lumalam\index{gnl}{Kalumalam@Ka\b lumalam} entour\'ee de jardins o\`u abondent les nuages,\\
La vraie T\=o\d nipuram\index{gnl}{Tonipuram@T\=o\d nipuram},\\
La belle Tar\=ay\index{gnl}{Taray@Tar\=ay} embrass\'ee par la gloire que louent les \'erudits,\\
Cirapuram\index{gnl}{Cirapuram}, la vraie Pu\b ravam\index{gnl}{Puravam@Pu\b ravam}, la ville d'Aya\b n\index{gnl}{Brahma@Brahm\=a!Aya\b n},\\
La ville d'Indra\index{gnl}{Indra} qu'embrasse une guirlande\index{gnl}{guirlande} de \textit{ka\b rpakam} fleurie,\\
Pukali\index{gnl}{Pukali} [et] Ve\.nkuru\index{gnl}{Venkuru@Ve\.nkuru} [c'est]\\
La ville de Celui qui porte la Ga\.ng\=a\index{gnl}{Ganga@Ga\.ng\=a}. (II 74.3)
\end{verse}

\scriptsize
\begin{verse}
\textit{taritta ma\b raiy\=a\d lar miku ve\.nkuru\index{gnl}{Venkuru@Ve\.nkuru}, c\=\i rt t\=o\d nipuram\index{gnl}{Tonipuram@T\=o\d nipuram}, tariy\=ar iñci\\
erittava\b n c\=er ka\b lumalam\index{gnl}{Kalumalam@Ka\b lumalam}\=e, koccai\index{gnl}{Koccai}, p\=untar\=ay\index{gnl}{Taray@Tar\=ay}, pukali\index{gnl}{Pukali}, imaiy\=ork\=o\d n\=ur,\\
teritta puka\b lc cirapuram\index{gnl}{Cirapuram}, c\=\i r tika\b l k\=a\b li, ca\d npai\index{gnl}{Canpai@Ca\d npai}, ce\b luma\b raika\d lell\=am\\
viritta puka\b lp pu\b ravam\index{gnl}{Puravam@Pu\b ravam}, viraikkamalatt\=o\b n\=ur---ulakil vi\d la\.nkum \=ur\=e.} (II 74.4)\\
\end{verse}
\normalsize
\begin{verse}
Ve\.nkuru\index{gnl}{Venkuru@Ve\.nkuru} o\`u abondent ceux des \textit{Veda}\index{gnl}{Veda@\textit{Veda}} impr\'egn\'es [dans l'esprit],\\
T\=o\d nipuram\index{gnl}{Tonipuram@T\=o\d nipuram} la glorieuse,\\
Ka\b lumalam\index{gnl}{Kalumalam@Ka\b lumalam} que rejoint Celui qui consuma\\
La fortifications des ennemis, Koccai\index{gnl}{Koccai},\\
La belle Tar\=ay\index{gnl}{Taray@Tar\=ay}, Pukali\index{gnl}{Pukali}, la ville du roi\index{gnl}{roi} des c\'elestes,\\
Cirapuram\index{gnl}{Cirapuram} \`a la renomm\'ee \'evidente, K\=a\b li\index{gnl}{Kali@K\=a\b li} qui brille de gloire,\\
Ca\d npai\index{gnl}{Canpai@Ca\d npai}, Pu\b ravam\index{gnl}{Puravam@Pu\b ravam} la renomm\'ee qui r\'epand tous les grands \textit{Veda}\index{gnl}{Veda@\textit{Veda}},\\
\lbrack Et] la ville de Celui au lotus \'epanoui [c'est]\\
La ville qui brille dans le monde. (II 74.4)\\
\end{verse}

\scriptsize
\begin{verse}
\textit{vi\d la\.nku aya\b n\=ur, p\=untar\=ay\index{gnl}{Taray@Tar\=ay}, miku ca\d npai\index{gnl}{Canpai@Ca\d npai}, v\=e\d nupuram\index{gnl}{Venupuram@V\=e\d nupuram}, m\=ekam \=eykkum\\
i\d la\.n kamukam po\b lil-t\=o\d nipuram\index{gnl}{Tonipuram@T\=o\d nipuram}, k\=a\b li, e\b lil pukali\index{gnl}{Pukali}, pu\b ravam\index{gnl}{Puravam@Pu\b ravam}, \=er \=ar\\
va\d lam kavarum vayal koccai\index{gnl}{Koccai}, ve\.nkuru\index{gnl}{Venkuru@Ve\.nkuru}, m\=ac cirapuram\index{gnl}{Cirapuram}, va\b nnañcam u\d n\d tu\\
ka\d la\.nkam mali ka\d lattava\b n c\=\i rk ka\b lumalam\index{gnl}{Kalumalam@Ka\b lumalam}---k\=ama\b n(\b n) u\d talam k\=aynt\=o\b n \=ur\=e.} (II 74.5)\\
\end{verse}
\normalsize
\begin{verse}
La ville d'Aya\b n\index{gnl}{Brahma@Brahm\=a!Aya\b n} qui brille, la belle Tar\=ay\index{gnl}{Taray@Tar\=ay},\\
Ca\d npai\index{gnl}{Canpai@Ca\d npai} l'excellente, V\=e\d nupuram\index{gnl}{Venupuram@V\=e\d nupuram},\\
T\=o\d nipuram\index{gnl}{Tonipuram@T\=o\d nipuram} aux jardins de jeunes ar\'equiers\\
Que rencontrent les nuages,\\
K\=a\b li\index{gnl}{Kali@K\=a\b li}, la belle Pukali\index{gnl}{Pukali}, Pu\b ravam\index{gnl}{Puravam@Pu\b ravam},\\
Koccai\index{gnl}{Koccai} aux rizi\`eres qui captivent la fertilit\'e pleine de beaut\'e,\\
Ve\.nkuru\index{gnl}{Venkuru@Ve\.nkuru}, la grande Cirapuram\index{gnl}{Cirapuram} [et]\\
Ka\b lumalam\index{gnl}{Kalumalam@Ka\b lumalam} la glorieuse de Celui au cou o\`u se r\'epand le noir,\\
Ayant consomm\'e le poison terrible, [c'est]\\
La ville de Celui qui consuma le corps de K\=ama\index{gnl}{Kama@K\=ama}. (II 74.5)\\
\end{verse}

\scriptsize
\begin{verse}
\textit{k\=ayntu varu k\=ala\b nai a\b n\b ru utaittava\b n \=ur ka\b lumalam\index{gnl}{Kalumalam@Ka\b lumalam}, m\=at t\=o\d nipuram\index{gnl}{Tonipuram@T\=o\d nipuram}, c\=\i r\\
\=eynta ve\.nkuru\index{gnl}{Venkuru@Ve\.nkuru}, pukali\index{gnl}{Pukali}, intira\b n\=ur, iru\.n kamalattu aya\b n\=ur, i\b npam\\
v\=aynta pu\b ravam\index{gnl}{Puravam@Pu\b ravam}, tika\b lum cirapuram\index{gnl}{Cirapuram}, p\=untar\=ay\index{gnl}{Taray@Tar\=ay}, koccai\index{gnl}{Koccai}, k\=a\b li, ca\d npai\index{gnl}{Canpai@Ca\d npai}---\\
c\=enta\b nai mu\b n payantu ulakil t\=evarka\d ltam pakai ke\d tutt\=o\b n tika\b lum \=ur\=e.} (II 74.6)\\
\end{verse}
\normalsize
\begin{verse}
Ka\b lumalam\index{gnl}{Kalumalam@Ka\b lumalam} la ville de Celui qui jadis\\
Donna un coup [de pied] \`a K\=ala qui venait en col\`ere,\\
La grande T\=o\d nipuram\index{gnl}{Tonipuram@T\=o\d nipuram}, Ve\.nkuru\index{gnl}{Venkuru@Ve\.nkuru} qui rencontre la gloire,\\
Pukali\index{gnl}{Pukali}, la ville d'Indra\index{gnl}{Indra}, la ville d'Aya\b n\index{gnl}{Brahma@Brahm\=a!Aya\b n} du lotus-si\`ege,\\
Pu\b ravam\index{gnl}{Puravam@Pu\b ravam} pleine de bonheur, Cirapuram\index{gnl}{Cirapuram} qui brille, la belle Tar\=ay\index{gnl}{Taray@Tar\=ay},\\
Koccai\index{gnl}{Koccai}, K\=a\b li\index{gnl}{Kali@K\=a\b li} [et] Ca\d npai\index{gnl}{Canpai@Ca\d npai} [c'est]\\
La ville o\`u brille Celui qui jadis a d\'etruit l'ennemi des dieux\\
En faisant na\^itre sur terre\index{gnl}{terre} C\=e\b nta\b n\index{gnl}{Centan@C\=e\b nta\b n} (Muruka\b n\index{gnl}{Murukan@Muruka\b n}). (II 74.6)\\
\end{verse}

\scriptsize
\begin{verse}
\textit{tika\b l m\=a\d tam mali ca\d npai\index{gnl}{Canpai@Ca\d npai}, p\=untar\=ay\index{gnl}{Taray@Tar\=ay}, pirama\b n\=ur, k\=a\b li, t\=ecu \=ar\\
miku t\=o\d nipuram\index{gnl}{Tonipuram@T\=o\d nipuram}, tika\b lum v\=e\d nupuram\index{gnl}{Venupuram@V\=e\d nupuram}, vayamkoccai\index{gnl}{Koccai}, pu\b ravam\index{gnl}{Puravam@Pu\b ravam}, vi\d n\d n\=or\\
puka\b l pukali\index{gnl}{Pukali}, ka\b lumalam\index{gnl}{Kalumalam@Ka\b lumalam}, c\=\i rc cirapuram\index{gnl}{Cirapuram}, ve\.nkuru\index{gnl}{Venkuru@Ve\.nkuru}---vemp\=or maki\d ta\b r ce\b r\b ru,\\
nika\b l n\=\i li, ni\b nmala\b nta\b n a\d t\=\i \d naika\d l pa\d nintu ulakil ni\b n\b ra \=ur\=e.} (II 74.7)\\
\end{verse}
\normalsize
\begin{verse}
Ca\d npai\index{gnl}{Canpai@Ca\d npai} o\`u se d\'eveloppent de brillantes maisons,\\
La belle Tar\=ay\index{gnl}{Taray@Tar\=ay}, la ville de Brahm\=a\index{gnl}{Brahma@Brahm\=a}, K\=a\b li\index{gnl}{Kali@K\=a\b li},\\
T\=o\d nipuram\index{gnl}{Tonipuram@T\=o\d nipuram} o\`u abonde l'asc\`ese,\\
V\=e\d nupuram\index{gnl}{Venupuram@V\=e\d nupuram} la brillante, la terre\index{gnl}{terre} de Koccai\index{gnl}{Koccai}, Pu\b ravam\index{gnl}{Puravam@Pu\b ravam},\\
Pukali\index{gnl}{Pukali} lou\'ee par les c\'elestes, Ka\b lumalam\index{gnl}{Kalumalam@Ka\b lumalam},\\
Cirapuram\index{gnl}{Cirapuram} la glorieuse [et] Ve\.nkuru\index{gnl}{Venkuru@Ve\.nkuru} [c'est]\\
La ville o\`u se tient sur terre\index{gnl}{terre} la brillante N\=\i li\index{gnl}{Nili@N\=\i li} qui,\\
Ayant d\'etruit Maki\d tar (Mahi\d sa\index{gnl}{Mahisa@Mahi\d sa}) dans une guerre cruelle,\\
Honora la paire de pieds du Pur. (II 74.7)\\
\end{verse}

\scriptsize
\begin{verse}
\textit{ni\b n\b ra matil c\=u\b ltaru ve\.nkuru\index{gnl}{Venkuru@Ve\.nkuru}, t\=o\d nipuram\index{gnl}{Tonipuram@T\=o\d nipuram}, nika\b lum v\=e\d nu, ma\b n\b ril\\
o\b n\b ru ka\b lumalam\index{gnl}{Kalumalam@Ka\b lumalam}, koccai\index{gnl}{Koccai}, uyar k\=a\b li, ca\d npai\index{gnl}{Canpai@Ca\d npai}, va\d lar pu\b ravam\index{gnl}{Puravam@Pu\b ravam}, m\=o\d ti\\
ce\b n\b ru pu\b ra\.nk\=akkum \=ur, cirapuram\index{gnl}{Cirapuram}, p\=untar\=ay\index{gnl}{Taray@Tar\=ay}, pukali\index{gnl}{Pukali}, t\=evark\=o\b n\=ur,\\
ve\b n\b ri mali piramapuram\index{gnl}{Piramapuram}---p\=uta\.nka\d lt\=am k\=akka mikka \=ur\=e.} (II 74.8)\\
\end{verse}
\normalsize
\begin{verse}
Ve\.nkuru\index{gnl}{Venkuru@Ve\.nkuru} entour\'ee de fortifications permanentes,\\
T\=o\d nipuram\index{gnl}{Tonipuram@T\=o\d nipuram}, V\=e\d nu qui brille,\\
Ka\b lumalam\index{gnl}{Kalumalam@Ka\b lumalam} est une des places publiques,\\
Koccai\index{gnl}{Koccai}, la haute K\=a\b li\index{gnl}{Kali@K\=a\b li}, Ca\d npai\index{gnl}{Canpai@Ca\d npai},\\
La croissante Pu\b ravam\index{gnl}{Puravam@Pu\b ravam},\\
La ville o\`u M\=o\d ti\index{gnl}{Moti@M\=o\d ti} vint garder les fronti\`eres,\\
Cirapuram\index{gnl}{Cirapuram}, la belle Tar\=ay\index{gnl}{Taray@Tar\=ay}, Pukali\index{gnl}{Pukali}, la ville du roi\index{gnl}{roi} des dieux [et]\\
La ville de Brahm\=a\index{gnl}{Brahma@Brahm\=a} qui se d\'eveloppe dans la victoire [c'est]\\
L'excellente ville que gardent les gnomes. (II 74.8)\\
\end{verse}

\scriptsize
\begin{verse}
\textit{mikka kamalattu aya\b n\=ur, vi\d la\.nku pu\b ravam\index{gnl}{Puravam@Pu\b ravam}, ca\d npai\index{gnl}{Canpai@Ca\d npai}, k\=a\b li, koccai\index{gnl}{Koccai}\\
tokka po\b lil ka\b lumalam\index{gnl}{Kalumalam@Ka\b lumalam}, t\=ut t\=o\d nipuram\index{gnl}{Tonipuram@T\=o\d nipuram}, p\=untar\=ay\index{gnl}{Taray@Tar\=ay}, cilampa\b n c\=er \=ur,\\
maik ko\d l po\b lil v\=e\d nupuram\index{gnl}{Venupuram@V\=e\d nupuram}, matil pukali\index{gnl}{Pukali}, ve\.nkuru\index{gnl}{Venkuru@Ve\.nkuru}---val arakka\b n ti\d nt\=o\d l\\
okka irupatum mu\d tika\d lorupatum \=\i \d tu a\b littu ukanta emm\=a\b n \=ur\=e.} (II 74.9)\\
\end{verse}
\normalsize
\begin{verse}
La ville d'Aya\b n\index{gnl}{Brahma@Brahm\=a!Aya\b n} du lotus excellent, Pu\b ravam\index{gnl}{Puravam@Pu\b ravam} qui brille,\\
Ca\d npai\index{gnl}{Canpai@Ca\d npai}, K\=a\b li\index{gnl}{Kali@K\=a\b li}, Koccai\index{gnl}{Koccai}, Ka\b lumalam\index{gnl}{Kalumalam@Ka\b lumalam} aux jardins denses,\\
T\=o\d nipuram\index{gnl}{Tonipuram@T\=o\d nipuram} la pure, la belle Tar\=ay\index{gnl}{Taray@Tar\=ay}, la ville que rejoint Cilampa\b n\index{gnl}{Cilampan@Cilampa\b n},\\
V\=e\d nupuram\index{gnl}{Venupuram@V\=e\d nupuram} aux jardins qui ont des nuages,\\
Pukali\index{gnl}{Pukali} aux fortifications [et] Ve\.nkuru\index{gnl}{Venkuru@Ve\.nkuru} [c'est]\\
La ville de notre Seigneur qui se r\'ejouit\\
Ayant d\'etruit le pouvoir des dix t\^etes\\
Et de l'ensemble des vingt épaule\index{gnl}{epaule@épaule}s robustes\\
Du puissant démon\index{gnl}{demon@démon} (R\=ava\d na). (II 74.9)\\
\end{verse}

\scriptsize
\begin{verse}
\textit{emm\=a\b n c\=er ve\.nkuru\index{gnl}{Venkuru@Ve\.nkuru}, c\=\i rc cilampa\b n\=ur, ka\b lumalam\index{gnl}{Kalumalam@Ka\b lumalam}, nal pukali\index{gnl}{Pukali}, e\b n\b rum\\
poymm\=a\d npu il\=or pu\b ravam\index{gnl}{Puravam@Pu\b ravam}, koccai\index{gnl}{Koccai}, purantara\b n\=ur, nal-t\=o\d nipuram\index{gnl}{Tonipuram@T\=o\d nipuram}, p\=ork\\
kaimm\=avai uriceyt\=o\b n k\=a\b li, aya\b n\=ur, tar\=ay\index{gnl}{Taray@Tar\=ay}, ca\d npai\index{gnl}{Canpai@Ca\d npai}---k\=ari\b n\\
meym m\=al, p\=u maka\b n, u\d nar\=a vakai ta\b lal\=ay vi\d la\.nkiya em i\b raiva\b n \=ur\=e.} (II 74.10)\\
\end{verse}
\normalsize
\begin{verse}
Ve\.nkuru\index{gnl}{Venkuru@Ve\.nkuru} que rejoint notre Seigneur,\\
La ville du glorieux Cilampa\b n\index{gnl}{Cilampan@Cilampa\b n},\\
Ka\b lumalam\index{gnl}{Kalumalam@Ka\b lumalam}, la bonne Pukali\index{gnl}{Pukali},\\
Pu\b ravam\index{gnl}{Puravam@Pu\b ravam} o\`u il n'y a jamais [de gens] \`a la fausse gloire,\\
Koccai\index{gnl}{Koccai}, la ville de Purandara (Indra\index{gnl}{Indra}), la bonne T\=o\d nipuram\index{gnl}{Tonipuram@T\=o\d nipuram},\\
K\=a\b li\index{gnl}{Kali@K\=a\b li} de Celui qui d\'epouilla l'\'el\'ephant belliqueux,\\
La ville d'Aya\b n\index{gnl}{Brahma@Brahm\=a!Aya\b n}, Tar\=ay\index{gnl}{Taray@Tar\=ay} [et] Ca\d npai\index{gnl}{Canpai@Ca\d npai} [c'est]\\
La ville de notre Seigneur qui brilla devenu feu\index{gnl}{feu} de telle sorte que\\
Le Fils de la fleur et M\=al\index{gnl}{Visnu@Vi\d s\d nu!Mal@M\=al} \`a la couleur du nuage (noir)\\
Ne le ressentent pas. (II 74.10)\\
\end{verse}

\scriptsize
\begin{verse}
\textit{i\b raiva\b n amar ca\d npai\index{gnl}{Canpai@Ca\d npai}, e\b lil pu\b ravam\index{gnl}{Puravam@Pu\b ravam}, aya\b n\=ur, imaiy\=orkkuatipa\b n c\=er \=ur,\\
ku\b raivu il puka\b lp pukali\index{gnl}{Pukali}, ve\.nkuru\index{gnl}{Venkuru@Ve\.nkuru}, t\=o\d nipuram\index{gnl}{Tonipuram@T\=o\d nipuram}, ku\d nam \=ar p\=untar\=ay\index{gnl}{Taray@Tar\=ay}, n\=\i rc\\
ci\b rai mali nal cirapuram\index{gnl}{Cirapuram}, c\=\i rk k\=a\b li, va\d lar koccai\index{gnl}{Koccai}, ka\b lumalam\index{gnl}{Kalumalam@Ka\b lumalam}---t\=ecu i\b n\b rip\\
pa\b ri talaiyo\d tu ama\.nkaiyar, c\=akkiyarka\d l, paricu a\b riy\=a amm\=a\b n \=ur\=e.} (II 74.11)\\
\end{verse}
\normalsize
\begin{verse}
Ca\d npai\index{gnl}{Canpai@Ca\d npai} o\`u si\`ege le Seigneur, la belle Pu\b ravam\index{gnl}{Puravam@Pu\b ravam},\\
La ville d'Aya\b n\index{gnl}{Brahma@Brahm\=a!Aya\b n}, la ville que rejoint le chef\index{gnl}{chef} des c\'elestes,\\
Pukali\index{gnl}{Pukali} la renomm\'ee sans manque, Ve\.nkuru\index{gnl}{Venkuru@Ve\.nkuru},\\
T\=o\d nipuram\index{gnl}{Tonipuram@T\=o\d nipuram}, la belle Tar\=ay\index{gnl}{Taray@Tar\=ay} pleine de qualit\'es,\\
La bonne Cirapuram\index{gnl}{Cirapuram} o\`u abondent les r\'eservoirs d'eau\index{gnl}{eau},\\
K\=a\b li\index{gnl}{Kali@K\=a\b li} la glorieuse, Koccai\index{gnl}{Koccai} la fertile [et] Ka\b lumalam\index{gnl}{Kalumalam@Ka\b lumalam} [c'est]\\
La ville du Seigneur dont la nature n'est pas connue\\
Des bouddhiste\index{gnl}{bouddhiste}s et des n\'efastes (ja\"in\index{gnl}{jain@ja\"in}s]),\\
Aux cheveux arrach\'es, qui sont sans asc\`ese. (II 74.11)\\
\end{verse}

\scriptsize
\begin{verse}
\textit{amm\=a\b n c\=er ka\b lumalam\index{gnl}{Kalumalam@Ka\b lumalam}, m\=ac cirapuram\index{gnl}{Cirapuram}, ve\.nkuru\index{gnl}{Venkuru@Ve\.nkuru}, koccai\index{gnl}{Koccai}, pu\b ravam\index{gnl}{Puravam@Pu\b ravam}, am c\=\i r\\
meym m\=a\b nattu o\d n pukali\index{gnl}{Pukali}, miku k\=a\b li, t\=o\d nipuram\index{gnl}{Tonipuram@T\=o\d nipuram}, t\=evark\=o\b n\=ur,\\
am m\=al ma\b n uyar ca\d npai\index{gnl}{Canpai@Ca\d npai}, tar\=ay\index{gnl}{Taray@Tar\=ay}, aya\b n\=ur, va\b li mu\d takkum \=avi\b np\=accal\\
tamm\=a\b n o\b n\b riya \~n\=a\b nacammanta\b n tami\b l ka\b rp\=or, takk\=ort\=am\=e.} (II 74.12)\\
\end{verse}
\normalsize
\begin{verse}
Ka\b lumalam\index{gnl}{Kalumalam@Ka\b lumalam} que rejoint le Seigneur,\\
La grande Cirapuram\index{gnl}{Cirapuram}, Ve\.nkuru\index{gnl}{Venkuru@Ve\.nkuru}, Koccai\index{gnl}{Koccai}, Pu\b ravam\index{gnl}{Puravam@Pu\b ravam},\\
La lumineuse Pukali\index{gnl}{Pukali} \`a la vraie renomm\'ee et \`a la belle gloire,\\
K\=a\b li\index{gnl}{Kali@K\=a\b li} l'excellente, T\=o\d nipuram\index{gnl}{Tonipuram@T\=o\d nipuram}, la ville du roi\index{gnl}{roi} des dieux,\\
La haute Ca\d npai\index{gnl}{Canpai@Ca\d npai} o\`u r\'eside la belle gloire,\\
Tar\=ay\index{gnl}{Taray@Tar\=ay} [et] la ville d'Aya\b n\index{gnl}{Brahma@Brahm\=a!Aya\b n} [sont pr\'esent\'es dans]\\
Le jet de vache qui zigzague sur le chemin,\\
Ceux qui apprennent [ce] tamoul de \~N\=a\b nacampanta\b n\index{gnl}{Campantar!N\=a\b nacampanta\b n@\~N\=a\b nacampanta\b n},\\
Ceux-l\`a sont dignes d'estime. (II 74.12)\\
\end{verse}

\subsection{Hymne III 67}

Cet hymne\index{gnl}{hymne} est construit selon la figure de style du \textit{va\b limo\b li}, \og mots du chemin\fg\ (voir 2.1.3) qui est mentionn\'ee dans l'envoi\index{gnl}{envoi}. Chacun des douze\index{gnl}{douze} toponymes est pr\'esent\'e avec sa légende\index{gnl}{legende@légende}. Les strophes 8, 9 et 10 d\'ecrivent respectivement le mythe\index{gnl}{mythe} de R\=ava\d na\index{gnl}{Ravana@R\=ava\d na}, la manifestation du \textit{li\.nga}\index{gnl}{linga@\textit{li\.nga}} et la critique des hérétique\index{gnl}{heretique@hérétique}s. Cette disposition est en d\'ecalage avec la structure typique de Campantar\index{gnl}{Campantar} car cet hymne\index{gnl}{hymne} comporte douze\index{gnl}{douze} strophes.
%qui est employ\'ee, selon T. V. Gopal Iyer, pour la premi\`ere fois dans la litt\'erature tamoule. Ce procédé\index{gnl}{procédé littéraire} consiste \`a reprendre en \textit{etukai}\index{gnl}{etukai@\textit{etukai}} (rime de la deuxi\`eme syllabe du vers ou du ciirXXXX) la deuxi\`eme syllabe des douze\index{gnl}{douze} noms de C\=\i k\=a\b li\index{gnl}{Cikali@C\=\i k\=a\b li}.

\scriptsize
\begin{verse}
\textit{curarulaku, nararka\d l payil tara\d nitalam, mura\d n a\b liya, ara\d namatil mup-\\
puram eriya, viravu vakai cara vicai ko\d l karam u\d taiya parama\b n i\d tam\=am---\\
varam aru\d la varalmu\b raiyi\b n niralni\b rai ko\d lvaru curuticira uraiyi\b n\=al,\\
pirama\b n uyar ara\b n e\b lil ko\d l cara\d nai\d nai parava, va\d lar piramapuram\index{gnl}{Piramapuram}\=e.} (III 67.1)\\
\end{verse}
\normalsize
\begin{verse}
Le lieu du Seigneur --- poss\'edant la main\\
Qui tue rapidement et unie \`a la fl\`eche\\
Pour consumer les trois forteresses\\
Aux fortifications protectrices\\
Et pour d\'etruire l'ennemi du monde des dieux\\
Et de la terre\index{gnl}{terre} o\`u r\'esident les gens ---\\
Est Piramapuram\index{gnl}{Piramapuram} qui s'\'el\`eve alors que Brahm\=a\index{gnl}{Brahma@Brahm\=a} r\'epand,\\
Avec le discours des \textit{ved\=anta} transmis\\
Dit de mani\`ere convenable et avec ordonnance\index{gnl}{ordonnance},\\
La [grandeur] de la paire de pieds qui re\c coivent\\
La beaut\'e du grand Hara\index{gnl}{Hara}\\
Pour qu'il (\'Siva\index{gnl}{Siva@\'Siva}) [lui] accorde gr\^ace. (III 67.1)\\
\end{verse}


\scriptsize
\begin{verse}
\textit{t\=a\d nu miku \=a\d n icaiko\d tu, \=a\d nu viyar p\=e\d numatu k\=a\d num a\d lavil,\\
k\=o\d num nutal n\=\i \d lnaya\b ni k\=o\d n il pi\d ti m\=a\d ni, matu n\=a\d num vakaiy\=e\\
\=e\d nu kari p\=u\d n a\b liya, \=a\d n iyal ko\d l m\=a\d ni pati---c\=e\d n amarark\=o\b n\\
v\=e\d nuvi\b nai \=e\d ni, nakar k\=a\d nil, tivi k\=a\d na, na\d tu v\=e\d nupuram\index{gnl}{Venupuram@V\=e\d nupuram}\=e.} (III 67.2)\\
\end{verse}
\normalsize
\begin{verse}
Le lieu du beau qui re\c coit la nature masculine\\
D\`es qu'[il] vit le fait d'aimer transpirer pour les dieux (?),\\
De fa\c con \`a r\'epugner la femme\index{gnl}{femme}, belle \'el\'ephante sans d\'efaut,\\
Celle aux yeux longs, au front courb\'e,\\
Pour d\'etruire les [actions] destructrices du puissant \'el\'ephant\\
Qui prit avec d\'esir la forme tr\`es masculine permanente ---\\
Est V\=e\d nupuram\index{gnl}{Venupuram@V\=e\d nupuram} o\`u a \'et\'e plant\'ee une \'echelle de bambou\index{gnl}{bambou}\\
Par le roi\index{gnl}{roi} des dieux du ciel\\
Pour voir la ville c\'eleste qui ne peut \^etre vue. (III 67.2)\\
\end{verse}

%TVG y voit le mythe\index{gnl}{mythe} de Ga\d ne\'sa\index{gnl}{Ganesa@Ga\d ne\'sa} engendr\'e par \'Siva \'el\'ephant et P\=arvat\=\i\index{gnl}{Parvati@P\=arvat\=\i} \'el\'ephante pour d\'etruire un démon\index{gnl}{demon@démon} \'el\'ephant qui par sa puissance rendait Matu (autre d\'emon) honteux de lui-m\^eme.

\scriptsize
\begin{verse}
\textit{pakal o\d licey naka ma\d niyai, mukai malarai, nika\b l cara\d na akavu mu\b nivarkku\\
akalam mali cakalakalai mika uraicey mukam u\d taiya pakava\b n i\d tam\=am-\\
pakai ka\d laiyum vakaiyil a\b rumukai\b raiyai mika aru\d la, nikar il imaiy\=or\\
puka, ulaku puka\b la, e\b lil tika\b la, nika\b l alar peruku pukalinakar\=e.} (III 67.3)\\
\end{verse}
\normalsize
\begin{verse}
Le lieu du Seigneur --- qui poss\`ede la bouche\\
Qui conseille avec \'elaboration tous les arts du vaste monde\\
Aux sages qui appellent la paire de pieds\\
Semblable \`a des fleurs en bouton\\
Et au rubis des montagnes qui brille comme le soleil ---\\
Est Pukali\index{gnl}{Pukali} dont la grandeur sans pareil augmente,\\
Alors que brille la beaut\'e et que le monde complimente,\\
Pour faire entrer les c\'elestes qui sont sans pareil\\
Et pour accorder grandement gr\^ace \`a Celui \`a six t\^etes\\
Afin de d\'eraciner son ennemi. (III 67.3)\\
\end{verse}

\scriptsize
\begin{verse}
\textit{am ka\d n mati, ka\.nkainati, ve\.nka\d n arava\.nka\d l, e\b lil ta\.nkum ita\b lit\\
tu\.nka malar, ta\.nku ca\d tai a\.nki nikar e\.nka\d l i\b rai ta\.nkum i\d tam\=am---\\
ve\.nkatir vi\d la\.nku ulakam e\.nkum etir po\.nku eri pula\b nka\d l ka\d laiv\=or\\
ve\.n kuru vi\d la\.nki umaipa\.nkar cara\d na\.nka\d l pa\d ni ve\.nkuruat\=e.} (III 67.4)\\
\end{verse}
\normalsize
\begin{verse}
Le lieu --- o\`u r\'eside notre Seigneur\\
Aux m\`eches pareilles au feu\index{gnl}{feu}\\
O\`u r\'esident la fleur pure de cassier pourvue de beaut\'e,\\
Les serpent\index{gnl}{serpent}s aux yeux cruels,\\
La rivi\`ere Ga\.ng\=a\index{gnl}{Ganga@Ga\.ng\=a} et la lune pourvue de beaut\'e :\\
Est Ve\.nkuru\index{gnl}{Venkuru@Ve\.nkuru} o\`u Ve\.nkuru\index{gnl}{Venkuru@Ve\.nkuru} \\
Qui a connu ceux qui ont ma\^itris\'e leur sens br\^ulants\\
Qui bouillonnent [venant] \`a la rencontre,\\
Partout dans le monde o\`u brille le soleil cruel\\
Et qui honore les pieds de Celui qui a pour moiti\'e Um\=a\index{gnl}{Uma@Um\=a}. (III 67.4)\\
\end{verse}

\scriptsize
\begin{verse}
\textit{\=a\d n iyalpu k\=a\d na, va\b nav\=a\d na iyal p\=e\d ni, etir p\=a\d nama\b lai c\=er\\
t\=u\d ni a\b ra, n\=a\d ni a\b ra, v\=e\d nu cilai p\=e\d ni a\b ra, n\=a\d ni vicaya\b n\\
p\=a\d ni amar p\=u\d na, aru\d l m\=a\d nu piram\=a\d ni i\d tam---\=e\d ni mu\b raiyil\\
p\=a\d ni ulaku \=a\d la, mika \=a\d ni\b n mali t\=o\d ni nikar t\=o\d nipuram\index{gnl}{Tonipuram@T\=o\d nipuram}\=e.} (III 67.5)\\
\end{verse}
\normalsize
\begin{verse}
Le lieu du Grand plein de gr\^ace ---\\
Qui, pour voir la nature de l'homme (Arjuna\index{gnl}{Arjuna}),\\
D\'esira [rev\^etir] la nature des habitants de la for\^et,\\
Alla \`a sa rencontre,\\
Brisa les carquois d'o\`u [sortait] jointe la pluie de fl\`eches,\\
Brisa la corde, \\
Et brisa l'arc de bambou\index{gnl}{bambou} avec plaisir,\\
Pour que Vijaya\b n\index{gnl}{Vijaya\b n} (Arjuna\index{gnl}{Arjuna}) honteux combatte avec les mains ---\\
Est T\=o\d nipuram\index{gnl}{Tonipuram@T\=o\d nipuram} pareil \`a une barque\index{gnl}{barque}\\
Qui a été \'etendue par le grand homme (\'Siva\index{gnl}{Siva@\'Siva})\\
Quand l'eau\index{gnl}{eau} r\'egnait sur la terre\index{gnl}{terre}\\
 \`A la mani\`ere d'une \'echelle (graduellement). (III 67.5)\\
\end{verse}

\scriptsize
\begin{verse}
\textit{"nir\=amaya! par\=apara! pur\=ata\b na! par\=avu civa! r\=aka! aru\d l!" e\b n\b ru,\\
ir\=avum etir\=ayatu par\=ay ni\b nai pur\=a\d na\b n, amar\=ati pati\=am---\\
ar\=amicai ir\=ata e\b lil taru \=aya ara par\=aya\d na var\=aka uru v\=a\\
Tar\=ay\index{gnl}{Taray@Tar\=ay}a\b nai vir\=ay eri par\=ay, miku tar\=ay\index{gnl}{Taray@Tar\=ay} mo\b li vir\=aya patiy\=e.} (III 67.6)\\
\end{verse}
\normalsize
\begin{verse}
Le lieu du Premier des dieux,\\
Celui des \textit{pur\=a\d na}\index{gnl}{Purana@\textit{Pur\=a\d na}} m\'emorables,\\
Qu'on prie nuit et son contre (jour) disant :\\
\og Celui qui est sans maladie, le grand pour les grands,\\
L'antique, \'Siva\index{gnl}{Siva@\'Siva} qui est honor\'e, d\'esir\'e par tous, accorde ta gr\^ace\fg,\\
Est le lieu o\`u est joint l'excellent mot \textit{Tar\=ay\index{gnl}{Taray@Tar\=ay}},\\
Pour d\'etruire la mal\'ediction dont est pourvu \\
Le dévot\index{gnl}{devot(e)@dévot(e)} de Hara\index{gnl}{Hara} qui donne la belle [terre\index{gnl}{terre}]\\
Qui n'est plus sur le serpent\index{gnl}{serpent},\\
Celui \`a forme de sanglier (Var\=aha\index{gnl}{Varaha@Var\=aha}),\\
Le roi\index{gnl}{roi} du souffle. (III 67.6)\\
\end{verse}

\scriptsize
\begin{verse}
\textit{ara\d nai u\b ru mura\d narpalar mara\d nam vara, ira\d nam matil aram mali pa\d taik\\
karam vici\b ru viraka\b n, amar kara\d na\b n, uyar para\b n, ne\b ri ko\d l kara\b natu i\d tam\=am---\\
paravu amutu virava, vi\d tal pura\d lam u\b rum aravai ari ciram ariya, ac\\
ciram ara\b na cara\d namavai parava, iru kirakam amar cirapuram\index{gnl}{Cirapuram}at\=e.} (III 67.7)\\
\end{verse}
\normalsize
\begin{verse}
Est le lieu --- de Celui qui est capable de lancer de la main \\
Les armes aiguis\'ees avec des pierres\\
Sur les fortifications bless\'ees \\
Pour que survienne la mort des nombreux \\
Qui sont contre et qui exp\'erimentent les fortifications;\\
Celui qui contr\^ole les sens qui r\'esident,\\
Le haut Seigneur,\\
Celui \`a la main qui montre le chemin, ---\\
Est Cirapuram\index{gnl}{Cirapuram} o\`u r\'esident les deux plan\`etes :\\
L'ambroisie\index{gnl}{ambroisie} \'etendue fut partag\'ee,\\
Hari coupa la t\^ete du serpent\index{gnl}{serpent} qui se roulait dans le poison,\\
Cette t\^ete (R\=ahu\index{gnl}{Rahu@R\=ahu}) pria les pieds de Hara\index{gnl}{Hara}. (III 67.7)\\
\end{verse}

\scriptsize
\begin{verse}
\textit{a\b ram a\b livu pe\b ra ulaku te\b ru puyava\b n vi\b ral a\b liya, ni\b ruvi viral, m\=a-\\
ma\b raiyi\b n oli mu\b rai muralcey pi\b raieyi\b ra\b n u\b ra, aru\d lum i\b raiva\b n i\d tam\=am-\\
ku\b raivu il mika ni\b raitai u\b li, ma\b rai amarar ni\b rai aru\d la, mu\b raiyo\d tu varum\\
pu\b rava\b n etir ni\b rai nilavu po\b raiya\b n u\d tal pe\b ra, aru\d lu pu\b ravamatuv\=e.} (III 67.8)\\
\end{verse}
\normalsize
\begin{verse}
Le lieu du Seigneur --- qui,\\
Pour qu'il obtienne la destruction du \textit{Dharma\index{gnl}{Dharma}} \\
Pour d\'etruire la force de celui aux bras qui font souffrir le monde \\
\'Ecrasa de son orteil, \\
Fit r\'esonner selon la norme le son du grand \textit{Veda}\index{gnl}{Veda@\textit{Veda}}\\
Par celui aux dents courb\'ees comme la lune, \\
Et qui accorda sa gr\^ace ---\\
Est Pu\b ravam\index{gnl}{Puravam@Pu\b ravam} qui accorde gr\^ace,\\
Ayant plac\'e le bon poids sans d\'efaut\\
Pour accorder gracieusement le poids du dieu\index{gnl}{dieu} cach\'e\\
Pour qu'obtienne un corps celui qui a de la patience,\\
Et qui installe [son] poids devant le pigeon\\
Qui vient selon la norme (?). (III 67.8)\\
\end{verse}

\scriptsize
\begin{verse}
\textit{vi\d n payila, ma\d n pakiri, va\d n pirama\b n e\d n periya pa\d n pa\d tai ko\d l m\=al,\\
ka\d n pariyum o\d npu o\b liya, nu\d nporu\d lka\d l ta\d n puka\b l ko\d l ka\d n\d ta\b n i\d tam\=am-\\
ma\d n pariyum o\d npu o\b liya, nu\d npu cakar pu\d n payila vi\d n pa\d tara, ac\\
ca\d npai\index{gnl}{Canpai@Ca\d npai} mo\b li pa\d npa mu\b ni ka\d n pa\b licey pa\d npu ka\d lai ca\d npainakar\=e.} (III 67.9)\\
\end{verse}
\normalsize
\begin{verse}
Le lieu --- de Celui \`a la gorge \\
Qui donne la grande gloire aux choses invisibles\\
Quand disparut la brillance qui d\'etruit les yeux\\
\lbrack Devant] M\=al\index{gnl}{Visnu@Vi\d s\d nu!Mal@M\=al} \`a l'arme pr\^ete \`a la grande renomm\'ee\\
Qui creusa la terre\index{gnl}{terre}\\
\lbrack Et devant] Brahm\=a\index{gnl}{Brahma@Brahm\=a} le fort\\
Qui erra dans le ciel ---\\
Est la ville de Ca\d npai\index{gnl}{Canpai@Ca\d npai} qui arrache les péché\index{gnl}{peche@péché}s\\
Quand le bon sage\index{gnl}{sage} pronon\c ca [le nom de] cette herbe,\\
Pour que la blessure demeure chez les Cents\\
Qui sont forts dans le combat,\\
Pour d\'etruire [leur] force qui garde difficilement la terre\index{gnl}{terre}\\
Pour qu'[ils] atteigent le ciel. (III 67.9)\\
\end{verse}

\scriptsize
\begin{verse}
\textit{p\=a\b li u\b rai v\=e\b lam nikar p\=a\b l ama\d nar, c\=u\b lum u\d tal\=a\d lar, u\d nar\=a\\
\=e\b li\b nicai y\=a\b li\b n mo\b li \=e\b laiava\d l v\=a\b lum i\b rai t\=a\b lum i\d tam\=am---\\
k\=\i \b l, icai ko\d l m\=elulakil, v\=a\b l aracu c\=u\b l aracu v\=a\b la, ara\b nukku\\
\=a\b liya cilk\=a\b li ceya, \=e\b lulakil \=u\b li va\d lar k\=a\b linakar\=e.} (III 67.10)\\
\end{verse}
\normalsize
\begin{verse}
Le lieu --- o\`u r\'eside le Seigneur\\
Qui vit avec la femme\index{gnl}{femme} au son du \textit{y\=a\b l} \`a sept gammes,\\
Qui n'est pas ressenti\\
Par ceux qui s'entourent [de v\^etement monastique]\\
Et par les ja\"in\index{gnl}{jain@ja\"in}s inutiles, semblables \`a des \'el\'ephants,\\
Qui vivent dans des monast\`ere\index{gnl}{monastère}s ---\\
Est la ville de K\=a\b li\index{gnl}{Kali@K\=a\b li} o\`u grandit le déluge\index{gnl}{deluge@déluge} dans les sept mondes,\\
Quand K\=a\b li\index{gnl}{Kali@K\=a\b li} la bruyante perdit [devant] Hara\index{gnl}{Hara}, \\
Alors que vivaient les rois environnants du bas-monde\\
Et les rois vivant dans le monde d'en haut\\
Faisaient de la musique\index{gnl}{musique}. (III 67.10)\\
\end{verse}

\scriptsize
\begin{verse}
\textit{naccu aravu kaccu e\b na acaiccu, mati ucciyi\b n milaiccu, oru kaiy\=al\\
meyc ciram a\d naiccu, ulakil niccam i\d tu piccai amar picca\b n i\d tam\=am-\\
maccam matam nacci matamac ci\b rumiyaic cey tava acca viratak\\
koccai\index{gnl}{Koccai} muravu accar pa\d niya, curarka\d l nacci mi\d tai koccai\index{gnl}{Koccai}nakar\=e.} (III 67.11)\\
\end{verse}
\normalsize
\begin{verse}
Le lieu --- o\`u r\'eside le fou\\
Qui mendie toujours de par le monde,\\
Ayant attach\'e telle une ceinture le serpent\index{gnl}{serpent} venimeux,\\
Couronn\'e de la lune sur le sommet,\\
Portant la t\^ete du corps dans une main ---\\
Est la ville de Koccai\index{gnl}{Koccai} d\'esir\'ee par les dieux\\
O\`u le sage\index{gnl}{sage} honore,\\
Ayant aim\'e l'intoxication du poisson,\\
\lbrack Puis] craignant ce qui a \'et\'e commis\\
Sur la jeune femme\index{gnl}{femme} des p\^echeurs,\\
Pour briser la bassesse [souillant son] asc\`ese. (III 67.11)\\
\end{verse}

\scriptsize
\begin{verse}
\textit{o\b lukal aritu a\b li kaliyil, u\b li ulaku pa\b li peruku va\b liyai ni\b naiy\=a,\\
mu\b lutu u\d talil e\b lum mayirka\d l ta\b luvum mu\b niku\b luvi\b no\d tu, ke\b luvu civa\b nait\\
to\b lutu, ulakil i\b lukum malam a\b liyum vakai ka\b luvum urai ka\b lumalanakar,\\
pa\b lutu il i\b rai e\b lutum mo\b li tami\b lviraka\b n va\b limo\b lika\d l mo\b li takaiyav\=e.} (III 67.12)\\
\end{verse}
\normalsize
\begin{verse}
Dans le \textit{kaliyuga} qui d\'etruit l'intelligence correcte,\\
Sans penser au chemin qui cro\^it\\
Dans le péché\index{gnl}{peche@péché} du monde apocalyptique,\\
Avec le groupe des sages aux poils qui poussent sur tout le corps,\\
Ayant honor\'e \'Siva\index{gnl}{Siva@\'Siva} qui est uni,\\
Il est correct de dire \og les mots du chemin\fg\\
De l'expert en tamoul aux mots qui louent le Seigneur sans d\'efaut\\
De la ville de Ka\b lumalam\index{gnl}{Kalumalam@Ka\b lumalam} qui lave\\
D\'etruisant les maux qui se r\'epandent dans le monde. (III 67.12)\\
\end{verse}


\subsection{Hymne III 110}

Ce poème\index{gnl}{poeme@poème} est compos\'e selon la figure de l'\textit{\=\i ra\d ti}\index{gnl}{irati@\textit{\=\i ra\d ti}}, \og deux pieds\fg\ (voir 2.1.3). Chaque strophe est d\'edi\'ee \`a un des noms de C\=\i k\=a\b li\index{gnl}{Cikali@C\=\i k\=a\b li} dans l'ordre\index{gnl}{ordre} d\'efini. Seuls les toponymes de Piramapuram\index{gnl}{Piramapuram}, V\=e\d nupuram\index{gnl}{Venupuram@V\=e\d nupuram} et T\=o\d nipuram\index{gnl}{Tonipuram@T\=o\d nipuram} sont pr\'esent\'es en r\'ef\'erence \`a leur légende\index{gnl}{legende@légende}.

\scriptsize
\begin{verse}
\textit{varamat\=e ko\d l\=a, uramat\=e ceyum puram erittava\b n---piramanalpurattu\\
ara\b n---na\b nn\=amam\=e paravuv\=arka\d l c\=\i r viravum, n\=\i \d l puviy\=e.} (III 110.1)\\
\end{verse}
\normalsize
\begin{verse}
Celui qui consuma les forteresses,\\
Qui ayant obtenu des faveurs montraient [leur] force,\\
Hara\index{gnl}{Hara} de la bonne ville de Brahm\=a\index{gnl}{Brahma@Brahm\=a};\\
La gloire de ceux qui louent [ses] bons noms\\
Se r\'epandra dans ce grand monde. (III 110.1)\\
\end{verse}

\scriptsize
\begin{verse}
\textit{c\=e\d n ul\=am matil v\=e\d nu ma\d nu\d l\=or k\=a\d na ma\b n\b ril \=ar v\=e\d nunalpurat\\
t\=a\d nuvi\b n ka\b lal p\=e\d nuki\b n\b ravar \=a\d ni ottavar\=e.} (III 110.2)\\
\end{verse}
\normalsize
\begin{verse}
Sont semblables à l'étalon [d'or]\\
Ceux qui aiment les pieds aux anneaux de cheville du Stable\\
De la bonne ville du bambou\index{gnl}{bambou} pleine d'assemblées\index{gnl}{assemblée}\\
Pour que les gens de la terre\index{gnl}{terre} voient les bambous \\
Tels des fortifications qui touchent le ciel. (III 110.2)\\
\end{verse}

\scriptsize
\begin{verse}
\textit{akalam \=ar taraip pukalum n\=alma\b raikku ikalil\=orka\d l v\=a\b l pukali\index{gnl}{Pukali} m\=a nakar,\\
pakal ceyv\=o\b n etirc cakala c\=ekara\b n akilan\=ayaka\b n\=e.} (III 110.3)\\
\end{verse}
\normalsize
\begin{verse}
Le Seigneur de l'univers est le manifeste \'Sekara\\
Qui s'oppose \`a celui qui fait le jour,\\
De la grande ville de Pukali\index{gnl}{Pukali} o\`u vivent\\
Ceux qui ne sont pas contraires aux quatre \textit{Veda}\index{gnl}{Veda@\textit{Veda}}\\
Que le très vaste monde loue. (III 110.3)\\
\end{verse}

\scriptsize
\begin{verse}
\textit{tu\.nka m\=akari pa\.nkam\=a a\d tum ce\.n kaiy\=a\b n nika\b l ve\.nkurut tika\b l\\
a\.nka\d n\=a\b n a\d ti tam kaiy\=al to\b la, ta\.nkum\=o, vi\b naiy\=e?} (III 110.4)\\
\end{verse}
\normalsize
\begin{verse}
Les actes r\'esideront-ils\\
Quand on honore avec les mains\\
Les pieds de Celui au beaux yeux \'eclatants\\
De la brillante Ve\.nkuru\index{gnl}{Venkuru@Ve\.nkuru},\\
De Celui \`a la main rouge\\
Qui d\'etruit faisant perdre\\
Le grand \'el\'ephant puissant? (III 110.4)\\
\end{verse}

\scriptsize
\begin{verse}
\textit{"k\=a\d ni, o\d n poru\d l, ka\b r\b ravarkku \=\i kai u\d taimaiy\=oravar k\=atal ceyyum nal\\
t\=o\d niva\d npurattu \=a\d ni" e\b npavar t\=u matiyi\b nar\=e.} (III 110.5)\\
\end{verse}
\normalsize
\begin{verse}
Ont l'esprit pur ceux qui disent:\\
\og Il est l'étalon [d'or]\\
De la ville fertile du bon radeau\index{gnl}{radeau}\\
Qui aime ceux qui poss\`edent\\
La qualit\'e de donner aux \'erudits\\
De brillantes richesses et des terres\index{gnl}{terre}\fg. (III 110.5)\\
\end{verse}

\scriptsize
\begin{verse}
\textit{\=entu ar\=a etir v\=aynta nu\d ni\d taip p\=un ta\d n \=otiy\=a\d l c\=ernta pa\.nki\b na\b n\\
p\=untar\=ay\index{gnl}{Taray@Tar\=ay} to\b lum m\=antar m\=e\b nim\=el c\=erntu ir\=a, vi\b naiy\=e.} (III 110.6)\\
\end{verse}
\normalsize
\begin{verse}
Les actes ne resteront pas coll\'es\\
Au corps des gens qui honorent la belle Tar\=ay\index{gnl}{Taray@Tar\=ay}\\
De Celui \`a la moiti\'e unie\\
\`A Celle \`a la chevelure fra\^iche et douce,\\
\`A la taille fine telle un serpent\index{gnl}{serpent} qui se dresse. (III 110.6)\\
\end{verse}

\scriptsize
\begin{verse}
\textit{"curapuratti\b nait tuyarcey t\=aruka\b n tuñca, veñci\b nak k\=a\d liyait tarum\\
cirapurattu u\d l\=a\b n" e\b n\b na vallavar citti pe\b r\b ravar\=e.} (III 110.7)\\
\end{verse}
\normalsize
\begin{verse}
Obtiendront la compl\'etude ceux qui sont capables de dire:\\
\og Celui de Cirapuram\index{gnl}{Cirapuram} donna K\=a\d li \`a la cruelle col\`ere\\
Pour d\'etruire T\=aruka\\
Qui faisait souffrir la ville des dieux\fg. (III 110.7)\\
\end{verse}

\scriptsize
\begin{verse}
\textit{"u\b ravum\=aki, a\b r\b ravarka\d lukku m\=a neti ko\d tuttu, n\=\i \d l puvi ila\.nku c\=\i rp\\
pu\b rava m\=a nakarkku i\b raiva\b n\=e!" e\b na, te\b rakil\=a, vi\b naiy\=e.} (III 110.8)\\
\end{verse}
\normalsize
\begin{verse}
Les actes ne d\'etruiront pas quand on dit:\\
\og \'Etant devenu un proche,\\
Ayant donn\'e une grande richesse\\
\`A ceux qui lui sont enti\`erement d\'evou\'es,\\
Il est le Seigneur de la grande ville de Pu\b ravam\index{gnl}{Puravam@Pu\b ravam}\\
\`A la gloire qui se r\'epand dans ce vaste monde\fg. (III 110.8)\\
\end{verse}

\scriptsize
\begin{verse}
\textit{pa\d npu c\=er ila\.nkaikku n\=ata\b n nal mu\d tika\d lpattaiyum ke\d ta nerittava\b n,\\
ca\d npai\index{gnl}{Canpai@Ca\d npai} \=atiyait to\b lum avarka\d laic c\=atiy\=a, vi\b naiy\=e.} (III 110.9)\\
\end{verse}
\normalsize
\begin{verse}
Les actes ne tourmenteront pas\\
Ceux qui honorent le Premier de Ca\d npai\index{gnl}{Canpai@Ca\d npai},\\
Celui qui \'ecrasa pour d\'etruire les dix [t\^etes]\\
Aux belles couronnes du Seigneur de la belle Ila\.nkai\index{gnl}{Srilanka!Ila\.nkai}. (III 110.9)\\
\end{verse}

\scriptsize
\begin{verse}
\textit{\=a\b li a\.nkaiyil ko\d n\d ta m\=al, aya\b n, a\b rivu o\d n\=atatu \=or va\d tivu ko\d n\d tava\b n---\\
k\=a\b li m\=a nakark ka\d tavu\d l---n\=amam\=e ka\b r\b ral naltavam\=e.} (III 110.10)\\
\end{verse}
\normalsize
\begin{verse}
Est un grand m\'erite\\
L'apprentissage des noms du dieu\index{gnl}{dieu} de la grande ville de K\=a\b li\index{gnl}{Kali@K\=a\b li},\\
De Celui qui prit la forme que ne pouvait sonder\\
L'intelligence d'Aya\b n\index{gnl}{Brahma@Brahm\=a!Aya\b n} et de M\=al\index{gnl}{Visnu@Vi\d s\d nu!Mal@M\=al}\\
Qui tient un disque dans la belle main. (III 110.10)\\
\end{verse}

\scriptsize
\begin{verse}
\textit{viccai o\b n\b ru il\=ac cama\d nar c\=akkiyappiccarta\.nka\d laik karicu a\b ruttava\b n\\
koccai\index{gnl}{Koccai} m\=a nakarkku a\b npu ceypavar ku\d na\.nka\d l k\=u\b rumi\b n\=e!} (III 110.11)\\
\end{verse}
\normalsize
\begin{verse}
Dites les qualit\'es des dévot\index{gnl}{devot(e)@dévot(e)}s de la grande ville de Koccai\index{gnl}{Koccai}\\
De Celui qui coupa la faute des fous\\
Que sont les bouddhiste\index{gnl}{bouddhiste}s et les ja\"in\index{gnl}{jain@ja\"in}s\\
Qui n'ont aucune connaissance\index{gnl}{connaissance}. (III 110.11)\\
\end{verse}

\scriptsize
\begin{verse}
\textit{ka\b lumalatti\b nu\d l ka\d tavu\d l p\=atam\=e karutu ñ\=a\b nacampanta\b n i\b ntami\b l\\
mu\b lutum vallavarkku i\b npam\=e tarum, mukka\d n em i\b raiy\=e.} (III 110.12)\\
\end{verse}
\normalsize
\begin{verse}
Notre seigneur aux trois yeux ne donne que du bonheur\\
 \`A ceux qui sont capables [de chanter] enti\`erement\\
 Le doux tamoul de \~N\=a\b nacampanta\b n\index{gnl}{Campantar!N\=a\b nacampanta\b n@\~N\=a\b nacampanta\b n}\\
 Qui ne m\'edite que les pieds du dieu\index{gnl}{dieu} de Ka\b lumalam\index{gnl}{Kalumalam@Ka\b lumalam}. (III 110.12)\\
\end{verse}

\subsection{Hymne III 113}

Ce poème\index{gnl}{poeme@poème} est construit selon le procédé\index{gnl}{procédé littéraire} litt\'eraire du \textit{iyamakam}\index{gnl}{iyamakam@\textit{iyamakam}} (voir 2.1.3). Il n'y a aucune allusion aux légende\index{gnl}{legende@légende}s des douze\index{gnl}{douze} toponymes. Notons que le mythe\index{gnl}{mythe} de R\=ava\d na\index{gnl}{Ravana@R\=ava\d na} est pr\'esent\'e dans la strophe 10, celui du \textit{li\.nga}\index{gnl}{linga@\textit{li\.nga}} dans la strophe 11 et la critique des ja\"in\index{gnl}{jain@ja\"in}s dans la derni\`ere strophe avec l'envoi\index{gnl}{envoi}.

\scriptsize
\begin{verse}
\textit{u\b r\b ru umai c\=ervatu meyyi\b naiy\=e; u\d narvatum ni\b n aru\d lmeyyi\b naiy\=e;\\
ka\b r\b ravar k\=ayvatu k\=ama\b naiy\=e; ka\b nal vi\b li k\=ayvatu k\=ama\b naiy\=e;\\
a\b r\b ram ma\b raippatu mu\b n pa\d niy\=e; amararka\d l ceyvatum u\b n pa\d niy\=e;\\
pe\b r\b ru mukantatu kanta\b naiy\=e; piramapurattai ukanta\b naiy\=e.} (III 113.1)\\
\end{verse}
\normalsize
\begin{verse}
Celui au corps que rejoint bien Um\=a\index{gnl}{Uma@Um\=a};\\
La v\'erit\'e de ta gr\^ace est ce qui est ressentie;\\
Le d\'esir est ce que consument les \'erudits;\\
K\=ama\index{gnl}{Kama@K\=ama} est celui que consume le regard de feu\index{gnl}{feu};\\
Le serpent\index{gnl}{serpent} devant est ce qui cache ce qui doit l'\^etre,\\
Ton service\index{gnl}{service} est ce que font les immortels;\\
Ayant donn\'e naissance\index{gnl}{naissance} avec joie \`a Kanta\b n\index{gnl}{Kantan@Kanta\b n};\\
Celui qui se r\'ejouit dans Piramapuram\index{gnl}{Piramapuram}. (III 113.1)\\
\end{verse}

\scriptsize
\begin{verse}
\textit{cati mika vanta calantara\b n\=e ta\d ti ciram n\=er ko\d l calam tara\b n\=e!\\
atir o\d li c\=er tikirippa\d taiy\=al amarnta\b nar umpar, tutippu a\d taiy\=al;\\
mati tava\b l ve\b rpuatu kaic cilaiy\=e; maru vi\d tam \=e\b rpatu kaiccilaiy\=e-\\
vitiyi\b nil i\d t\d tu avirum para\b n\=e! v\=e\d nupurattai virumpu ara\b n\=e!} (III 113.2)\\
\end{verse}
\normalsize
\begin{verse}
Jalandhara qui avan\c cait tr\`es rapidement,\\
\^O Porteur de la belle eau\index{gnl}{eau}, tu le d\'ecapitas, \\
Avec l'arme circulaire o\`u brille la peur,\\
R\'ealisant [ainsi] le souhait de ceux qui r\'esident dans le ciel; \\
L'arc dans ta main est la montagne o\`u rampe la lune;\\
Accepter le poison apparu n'est pas une amertume; \\
\^O rayonnant Seigneur qui pla\c ca [le monde] dans l'ordre\index{gnl}{ordre}! \\
\^O Hara\index{gnl}{Hara} qui aime V\=e\d nupuram\index{gnl}{Venupuram@V\=e\d nupuram}!\\
\end{verse}

\scriptsize
\begin{verse}
\textit{k\=atu amarat tika\b l t\=o\d ti\b na\b n\=e; k\=a\b nava\b n\=ayk ka\d titu \=o\d ti\b na\b n\=e;\\
p\=atamat\=al k\=u\b r\b ru\index{gnl}{Kurru@K\=u\b r\b ru} utaitta\b na\b n\=e; p\=artta\b n u\d talampu taitta\b na\b n\=e;\\
t\=atu avi\b l ko\b n\b rai taritta\b na\b n\=e; c\=arnta vi\b naiatu aritta\b na\b n\=e-\\
p\=otam amarum uraip poru\d l\=e, pukali\index{gnl}{Pukali} amarnta paramporu\d l\=e.} (III 113.3)\\
\end{verse}
\normalsize
\begin{verse}
Celui \`a la boucle qui demeure sur l'oreille;\\
Celui qui, devenu un chasseur, court vite;\\
Celui qui frappe K\=u\b r\b ru\index{gnl}{Kurru@K\=u\b r\b ru} avec le pied;\\
Celui qui perce d'une fl\`eche le corps de P\=artta\b n\index{gnl}{Parttan@P\=artta\b n} (Arjuna\index{gnl}{Arjuna});\\
Celui qui porte la fleur de cassier d'o\`u tombe le pollen;\\
Celui qui rompt le \textit{karma} qui approche;\\
\^O V\'erit\'e qui est la demeure o\`u r\'eside la sagesse;\\
\^O V\'erit\'e supr\^eme qui demeure \`a Pukali\index{gnl}{Pukali}. (III 113.3)\\
\end{verse}

\scriptsize
\begin{verse}
\textit{mait tika\b l nañcu umi\b l m\=acu\d nam\=e maki\b lntu arai c\=ervatum; m\=a cu(\d n)\d nam\=e\\
meyttu u\d tal p\=ucuvar; m\=el matiy\=e; v\=etamatu \=otuvar, m\=elmatiy\=e;\\
poyt talai\=o\d tu u\b rum, attamat\=e; purica\d tai vaittatu, mattamat\=e;\\
vittakar \=akiya em kuruv\=e virumpi amarnta\b nar, ve\.nkuruv\=e.} (III 113.4)\\
\end{verse}
\normalsize
\begin{verse}
Est joint \`a la taille, avec joie,\\
Le serpent\index{gnl}{serpent} qui crache le brillant venin noir;\\
Celui qui applique avec plaisir sur le corps la grande cendre\index{gnl}{cendre};\\
\lbrack Il porte] la lune sur la t\^ete;\\
Celui \`a la grande intelligence qui r\'ecite les \textit{Veda}\index{gnl}{Veda@\textit{Veda}};\\
Celui \`a la main qui tient la t\^ete sans vie;\\
La fleur de datura est plac\'ee dans les m\`eches torsad\'ees;\\
\^O mon ma\^itre qui est habile\\
Il s'est install\'e avec plaisir \`a Ve\.nkuru\index{gnl}{Venkuru@Ve\.nkuru}. (III 113.4)\\
\end{verse}

\scriptsize
\begin{verse}
\textit{u\d ta\b n payilki\b n\b ra\b na\b n, m\=atava\b n\=e, u\b ru po\b ri k\=ayntu icai m\=a tava\b n\=e;\\
ti\d tam pa\d ta m\=ama\b rai ka\d n\d ta\b na\b n\=e, tiriku\d nam m\=eviya ka\d n\d ta\b na\b n\=e;\\
pa\d tam ko\d l aravu arai ceyta\b na\b n\=e; paka\d tu uriko\d n\d tu araiceyta\b na\b n\=e;\\
to\d tarnta tuyarkku oru nañcu iva\b n\=e, t\=o\d nipurattu u\b rai nam civa\b n\=e.} (III 113.5)\\
\end{verse}
\normalsize
\begin{verse}
Celui qui est avec M\=atava\b n\index{gnl}{Visnu@Vi\d s\d nu!Matavan@M\=atava\b n} (Vi\d s\d nu\index{gnl}{Visnu@Vi\d s\d nu});\\
\^O grand asc\`ete de la voie qui d\'etruit les sens;\\
Celui qui a fait les grands \textit{Veda}\index{gnl}{Veda@\textit{Veda}} de fa\c con claire;\\
Celui qui renonce aux [doctrines de] trois qualit\'es;\\
Celui qui porte \`a la taille le serpent\index{gnl}{serpent} qui se dresse;\\
Celui qui d\'etruit l'\'el\'ephant en prenant sa d\'epouille;\\
Il est un poison pour les souffrances qui [nous] poursuivent,\\
Notre \'Siva\index{gnl}{Siva@\'Siva} qui demeure \`a T\=o\d nipuram\index{gnl}{Tonipuram@T\=o\d nipuram}. (III 113.5)\\
\end{verse}

\scriptsize
\begin{verse}
\textit{tika\b l kaiyatum pukai ta\.nku a\b lal\=e; t\=evar to\b luvatum tam ka\b lal\=e;\\
ika\b lpavart\=am oru m\=a\b n i\d tam\=e; irun ta\b nuv\=o\d tu e\b lil m\=a\b ni\d tam\=e;\\
mika varum n\=\i r ko\d lum mañcu a\d taiy\=e, mi\b n nikarki\b n\b ratum, am ca\d taiy\=e;\\
taka iratam ko\d l vacuntarar\=e, takka tar\=ay\index{gnl}{Taray@Tar\=ay} u\b rai cuntarar\=e.} (III 113.6)\\
\end{verse}
\normalsize
\begin{verse}
Celui du feu\index{gnl}{feu} port\'e dans la main brillante,\\
Celui aux anneaux de pieds honor\'es par les dieux,\\
\lbrack Celui qui porte] \`a gauche la gazelle des ennemis (les sages de la for\^et),\\
\lbrack Celui qui possède] \`a gauche, dans le grand corps, la belle femme\index{gnl}{femme},\\
Celui aux belles m\`eches qui brillent [comme] l'\'eclair\\
Que rejoignent les nuages qui prennent l'eau\index{gnl}{eau} en abondance,\\
Celui de la terre\index{gnl}{terre} qu'il prend pour char appropri\'e,\\
\^O le Beau qui demeure dans la parfaite Tar\=ay\index{gnl}{Taray@Tar\=ay}. (III 113.6)\\
\end{verse}

\scriptsize
\begin{verse}
\textit{\=orvu aru ka\.nka\d l i\d naikka(a)yal\=e; umaiyava\d l ka\.nka\d l i\d naik kayal\=e;\\
\=er maruvum ka\b lal n\=akamat\=e; e\b lil ko\d l ut\=aca\b na\b n, \=akamat\=e;\\
n\=\i r varu kontu a\d lakam kaiyat\=e, ne\d tuñca\d tai m\=eviya ka\.nkaiyat\=e;-\\
c\=ervu aru y\=oka tiyampaka\b n\=e! cirapuram\index{gnl}{Cirapuram} m\=eya ti ampu aka\b n\=e!} (III 113.7)\\
\end{verse}
\normalsize
\begin{verse}
Celui qui, \'etranger, ne peut \^etre approch\'e\\
Par les yeux difficiles qui ne voient pas [ses dévot\index{gnl}{devot(e)@dévot(e)}s];\\
Le poisson \textit{kayal} rejoint les yeux d'Um\=a\index{gnl}{Uma@Um\=a};\\
Le serpent\index{gnl}{serpent} aux anneaux embrasse la beaut\'e;\\
Celui \`a la forme du beau feu\index{gnl}{feu};\\
Les longues m\`eches o\`u habite la Ga\.ng\=a\index{gnl}{Ganga@Ga\.ng\=a} sont correctes\\
\lbrack M\^eme si] ce sont des cheveux en touffe o\`u vient l'eau\index{gnl}{eau};\\
\^O Yogi aux trois yeux difficiles \`a atteindre;\\
\^O Celui \`a la main pourvue d'une fl\`eche de feu\index{gnl}{feu}\\
Qui demeure \`a Cirapuram\index{gnl}{Cirapuram}. (III 113.7)\\
\end{verse}

\scriptsize
\begin{verse}
\textit{\=\i \d n\d tu tuyil amar appi\b na\b n\=e iru\.n ka\d n i\d tantu a\d ti appi\b na\b n\=e;\\
t\=\i \d n\d tal arum paricu ak karam\=e tika\b lntu o\d li c\=ervatu cakkaram\=e;\\
v\=e\d n\d ti varunta nakait talaiy\=e mikaittu avar\=o\d tu nakaittalaiy\=e\\
p\=u\d n\d ta\b nar; c\=eralum m\=a patiy\=e, pu\b ravam\index{gnl}{Puravam@Pu\b ravam} amarnta um\=apatiy\=e.} (III 113.8)\\
\end{verse}
\normalsize
\begin{verse}
Celui de l'eau\index{gnl}{eau} (mer\index{gnl}{mer} de lait\index{gnl}{lait}) qui demeure dans le sommeil\\
A appliqu\'e sur les pieds ses grands yeux, les ayant creus\'es;\\
Le disque rejoint la lumi\`ere brillante\\
Dans la main de qualit\'e difficile \`a toucher;\\
Celui qui porte le cr\^ane rieur en col\`ere contre ceux\\
Qui ont demand\'e avec effort le cr\^ane rieur (les sages de la for\^et);\\
Est atteinte la grande ville, Pu\b ravam\index{gnl}{Puravam@Pu\b ravam},\\
O\`u demeure le chef\index{gnl}{chef} d'Um\=a\index{gnl}{Uma@Um\=a}. (III 113.8)\\
\end{verse}

\scriptsize
\scriptsize
\begin{verse}
\textit{ni\b n ma\d ni v\=ayatu n\=\i \b lalaiy\=e n\=ecamatu \=a\b navar n\=\i \b lalaiy\=e;\\
u\b n\b ni, ma\b nattu, e\b lu ca\.nkamat\=e o\d liata\b n\=o\d tu u\b ru ca\.nkamat\=e;\\
ka\b n\b niyaraik kavarum ka(\d l)\d la\b n\=e! ka\d talvi\d tam u\d n\d ta karu\.n ka\d la\b n\=e;\\
ma\b n\b ni varaip pati, ca\d npu aiyat\=e v\=ari vayal mali ca\d npaiat\=e.} (III 113.9)\\
\end{verse}
\normalsize
\normalsize
\begin{verse}
L'ombre de ton entr\'ee \`a cloche [du temple\index{gnl}{temple}]\\
Est le refuge\index{gnl}{refuge} de ceux qui sont devenus dévot\index{gnl}{devot(e)@dévot(e)}s;\\
L'assemblée\index{gnl}{assemblée} [de dévot\index{gnl}{devot(e)@dévot(e)}s] qui m\'editent dans leur coeur se l\`eve\\
\lbrack Et part] avec des conques brillantes;\\
\^O Voleur qui ravit les vierges;\\
\^O Celui au cou noir qui avala le poison de la mer\index{gnl}{mer};\\
La ville o\`u il demeure avec coeur est Ca\d npai\index{gnl}{Canpai@Ca\d npai} o\`u abondent\\
Les rizi\`eres fertiles entour\'ees d'herbe \textit{ca\d npu}. (III 113.9)\\
\end{verse}

\scriptsize
\begin{verse}
\textit{ila\.nkai arakkartamakku i\b raiy\=e i\d tantu kayilai e\d tukka, i\b raiy\=e,\\
pula\b nka\d l ke\d ta u\d ta\b n p\=a\d ti\b na\b n\=e; po\b rika\d l ke\d ta u\d ta\b np\=a\d ti\b na\b n\=e;\\
ila\.nkiya m\=e\b ni ir\=a va\d na\b n\=e eytu peyarum ir\=ava\d na\b n\=e;\\
kalantu aru\d l pe\b r\b ratum m\=a vaciy\=e; k\=a\b li ara\b n a\d ti m\=a vaciy\=e.} (III 113.10)\\
\end{verse}
\normalsize
\begin{verse}
\^O Celui qui s'est accord\'e pour d\'etruire les sens\\
Du chef\index{gnl}{chef} des d\'emons de Ila\.nkai\index{gnl}{Srilanka!Ila\.nkai},\\
D\`es qu'il souleva un peu le mont \textit{Kail\=asa\index{gnl}{Kailasa@Kail\=asa}},\\
Il chanta d\`es que ses sens p\'erissaient,\\
Celui \`a la couleur de la nuit sur le corps brillant\\
A pris le nom de R\=ava\d na\index{gnl}{Ravana@R\=ava\d na};\\
Le glaive obtenu gr\^acieusement, [le coeur] consenti;\\
Les pieds du Hara\index{gnl}{Hara} de K\=a\b li\index{gnl}{Kali@K\=a\b li} sont une grande attraction. (III 113.10)\\
\end{verse}

\scriptsize
\begin{verse}
\textit{ka\d n nika\b l pu\d n\d tarikatti\b na\b n\=e, kalantu iri pu\d n tari katti\b na\b n\=e,\\
ma\d n nika\b lum paricu \=e\b namat\=e, v\=a\b nakam \=ey vakai c\=e\b namat\=e,\\
na\d n\d ni a\d timu\d ti eytalar\=e; na\d lir mali c\=olaiyil eytu alar\=e\\
pa\d n iyal koccai\index{gnl}{Koccai} pacupatiy\=e, pacu mika \=urvar, pacupatiy\=e.} (III 113.11)\\
\end{verse}
\normalsize
\begin{verse}
Celui au yeux de lotus brillants (Vi\d s\d nu\index{gnl}{Visnu@Vi\d s\d nu}),\\
Celui du lotus assis avec contentement,\\
Sanglier \`a la nature o\`u brille la terre\index{gnl}{terre},\\
Aigle [qui vole] de fa\c con \`a atteindre le ciel,\\
Ayant cherch\'e, ils n'atteignent pas la base ni le sommet\\
Du seigneur des \^ames de Koccai\index{gnl}{Koccai} de belle nature\\
Aux fleurs des jardins o\`u abonde la fra\^icheur;\\
Le seigneur des \^ames qui monte un grand bovin. (III 113.11)\\
\end{verse}

\scriptsize
\begin{verse}
\textit{paru matil maturai ma\b n avai etir\=e patikamatu e\b lutuilaiavai etir\=e\\
varu nat\=\i \d tai micai varu kara\b n\=e! vacaiyo\d tum alar ke\d ta aruku ara\b n\=e!\\
karutal il icai muraltarum maru\d l\=e, ka\b lumalam\index{gnl}{Kalumalam@Ka\b lumalam} amar i\b rai tarum aru\d l\=e;\\
maruviya tami\b lviraka\b na mo\b liy\=e vallavartam i\d tar, ti\d tam, o\b liy\=e.} (III 113.12)\\
\end{verse}
\normalsize
\begin{verse}
En face d'eux (l'assemblée\index{gnl}{assemblée})\\
Du roi\index{gnl}{roi} de Maturai\index{gnl}{Maturai} aux grandes fortifications,\\
\^O celui \`a l'acte qui fait venir\\
Les feuilles d'\'ecriture des chants\index{gnl}{chant}\index{gnl}{chant}\\
\`A l'encontre de la rivi\`ere qui coule;\\
\^O celui qui d\'etruit les ja\"in\index{gnl}{jain@ja\"in}s pour an\'eantir bl\^ame et bassesse;\\
L'\'etonnement que donne l'extension de la gloire sans pareille\\
Est la gr\^ace qu'octroie le Seigneur qui demeure \`a Ka\b lumalam\index{gnl}{Kalumalam@Ka\b lumalam};\\
Les souffrances p\'eriront certainement\\
Pour ceux qui sont forts dans les mots\\
De l'expert en tamoul embrass\'e. (III 113.12)\\
\end{verse}

Nous constatons, \`a la lecture de ces hymne\index{gnl}{hymne}s \`a douze\index{gnl}{douze} noms attribu\'es \`a Campantar\index{gnl}{Campantar}, que chaque toponyme est li\'e \`a une légende\index{gnl}{legende@légende} plus ou moins d\'etaill\'ee. Il est n\'ecessaire d'\'etudier le traitement de ces légende\index{gnl}{legende@légende}s dans l'ensemble du \textit{T\=ev\=aram}\index{gnl}{Tevaram@\textit{T\=ev\=aram}} pour consid\'erer leur place dans le corpus\index{gnl}{corpus}. Les donn\'ees de ces poème\index{gnl}{poeme@poème}s \`a douze\index{gnl}{douze} noms concordent-elles avec celles des autres poème\index{gnl}{poeme@poème}s attribu\'es \`a Campantar\index{gnl}{Campantar}? Quel est le t\'emoignage des deux autres \textit{m\=uvar}\index{gnl}{muvar@\textit{m\=uvar}}?

\section{Les douze légendes dans le \textit{T\=ev\=aram}}

Pour l'analyse des douze\index{gnl}{douze} légende\index{gnl}{legende@légende}s de C\=\i k\=a\b li\index{gnl}{Cikali@C\=\i k\=a\b li} dans le \textit{T\=ev\=aram}\index{gnl}{Tevaram@\textit{T\=ev\=aram}} examinons les textes attribu\'es \`a Campantar\index{gnl}{Campantar} et aux deux autres \textit{m\=uvar}\index{gnl}{muvar@\textit{m\=uvar}}.

\subsection{Les douze légendes chez Campantar}

Le tableau 3.2 ci-dessous classe les informations recueillies sur les légende\index{gnl}{legende@légende}s des douze\index{gnl}{douze} noms de C\=\i k\=a\b li\index{gnl}{Cikali@C\=\i k\=a\b li} \`a travers huit des hymne\index{gnl}{hymne}s traduits pr\'ec\'edemment. Nous ne pr\'esentons pas les poème\index{gnl}{poeme@poème}s I 117 et III 113 parce qu'ils ne contiennent aucune allusion aux légende\index{gnl}{legende@légende}s du site, ni l'hymne\index{gnl}{hymne} I 127 que nous n'avons pas traduit.
%\noindent
%\begin{table}
%\begin{center}
%\caption{\textsc{Les légendes dans les hymnes \`a douze noms}}
%\end{center}
%\end{table}

\begin{center}
\scriptsize
\begin{longtable}{|c|c|c|c|c|c|c|c|c|}
\caption{Les légendes dans les hymnes \`a douze noms}\endfirsthead
\hline
 & I 63& I 90& I 128& II 70& II 73& II 74& III 67& III 110\endhead
\hline
 & I 63& I 90& I 128& II 70& II 73& II 74& III 67& III 110\\
\hline
\hline
Pirama-& Brahm\=a\index{gnl}{Brahma@Brahm\=a}& ville de&&ville de &ville de&ville de&Brahm\=a\index{gnl}{Brahma@Brahm\=a}&ville de\\
 puram & y r\`egne&Brahm\=a\index{gnl}{Brahma@Brahm\=a}&&Brahm\=a\index{gnl}{Brahma@Brahm\=a}&Brahm\=a\index{gnl}{Brahma@Brahm\=a}&Brahm\=a\index{gnl}{Brahma@Brahm\=a}&y &Brahm\=a\index{gnl}{Brahma@Brahm\=a}\\
 &(st. 1)&(st. 1)&& (st. 1, 2,& (douze\index{gnl}{douze}& (douze\index{gnl}{douze}&honore& (st. 1)\\
 &&&& 5, 6, 7, 8,&r\'ef.)&r\'ef.)&\'Siva\index{gnl}{Siva@\'Siva}&\\
 &&&& 10 et 11)&&&(st. 1)&\\
\hline
V\=e\d nu-& Indra\index{gnl}{Indra}&ville du&&&ville& ville&Indra\index{gnl}{Indra}&ville du\\
 puram &y r\`egne&bambou\index{gnl}{bambou}&&&d'Indra\index{gnl}{Indra}&d'Indra\index{gnl}{Indra}&s'y&bambou\index{gnl}{bambou}\\
 &(st. 2)&(st. 2)&&&(st. 2, 3,&(st. 1, 2,&installe&(st. 1)\\
 &&&&&4, 5, 6, &3, 4, 6,&(st. 1)&\\
 &&&&&7, 9, 10 & 8, 10, 11&&\\
 &&&&&et 11)& et 12)&&\\
\hline
Pukali\index{gnl}{Pukali}& refuge\index{gnl}{refuge}&refuge\index{gnl}{refuge}&&&refuge\index{gnl}{refuge}&&refuge\index{gnl}{refuge}&\\
 &des dieux&(st. 3)&&&(st. 9)&&des dieux&\\
 &(st. 3)&&&&&&(st. 3)&\\
\hline
Ve\.n-& Dharma\index{gnl}{Dharma}&&&&ville de&&Ve\.nkuru\index{gnl}{Venkuru@Ve\.nkuru}&\\
 kuru &y r\`egne&&&&Dharma\index{gnl}{Dharma}&&y honore&\\
 &(st. 4)&&&&(st. 12)&&\'Siva\index{gnl}{Siva@\'Siva}&\\
 &&&&&&&(st. 4)&\\
\hline
T\=o\d ni-& ville&ville&ville&ville du&&&ville-&ville\\
 puram&flottant\index{gnl}{flotter}&du&flottant&déluge\index{gnl}{deluge@déluge}&&&radeau\index{gnl}{radeau}&du\\
 &au&radeau\index{gnl}{radeau}&au&(st. 1, 2&&&du&radeau\index{gnl}{radeau}\\
 & déluge\index{gnl}{deluge@déluge}&(st. 5)& déluge\index{gnl}{deluge@déluge}&3, 4 et 5)&&& déluge\index{gnl}{deluge@déluge}&(st. 5)\\
 &(st. 5)&&(l. 28-29)&&&&(st. 5)&\\
\hline
Tar\=ay\index{gnl}{Taray@Tar\=ay}& un roi\index{gnl}{roi}&&&&&&péché\index{gnl}{peche@péché}&\\
 &vaillant&&&&&&de&\\
 &y r\`egne&&&&&&Var\=aha\index{gnl}{Varaha@Var\=aha}&\\
 &(st. 6)&&&&&&(st. 6)&\\
\hline
Cira-&Cilam-&&&Cilampa\b n\index{gnl}{Cilampan@Cilampa\b n}&ville de&ville de&péché\index{gnl}{peche@péché}&\\
puram&pa\b n&&&(st. 5) ou&Cilampa\b n\index{gnl}{Cilampan@Cilampa\b n}&Cilampa\b n\index{gnl}{Cilampan@Cilampa\b n}&de&\\
 &y r\`egne&&&une t\^ete&(st. 2, 3,&(st. 9 &R\=ahu\index{gnl}{Rahu@R\=ahu}&\\
 &(st. 7)&&&(st. 4 et 6)&4, 5, 8, 9,&et 10)&(st. 7)&\\
 &&&&y r\`egne&10 et 12)&&&\\
\hline
Pu\b ra-&un roi\index{gnl}{roi} &&&&&&mythe\index{gnl}{mythe}&\\
vam&\`a char&&&&&&du roi\index{gnl}{roi}&\\
&y r\`egne&&&&&&\'Sibi\index{gnl}{Sibi@\'Sibi}&\\
&(st. 8)&&&&&&(st. 8)&\\
\hline
Ca\d npai\index{gnl}{Canpai@Ca\d npai}& Ca\d n\d ta\b n\index{gnl}{Cantan@Ca\d n\d ta\b n} &&&&&&péché\index{gnl}{peche@péché} &\\
&y r\`egne&&&&&&des Y\=adava\index{gnl}{Yadava@Y\=adava}&\\
&(st. 9)&&&&&&(st. 9)&\\
\hline
K\=a\b li\index{gnl}{Kali@K\=a\b li}& roi\index{gnl}{roi} des &&&&&&d\'efaite &\\
&serpent\index{gnl}{serpent}s&&&&&&de K\=a\d li&\\
&y r\`egne&&&&&&(st. 10)&\\
&(st. 10)&&&&&&&\\
\hline
Koccai\index{gnl}{Koccai}&Nanta\b n\index{gnl}{Nantan@Nanta\b n} &&&&&&souillure\index{gnl}{souillure} &\\
&y r\`egne&&&&&&d'un sage\index{gnl}{sage}&\\
&(st. 11)&&&&&&(st. 11)&\\
\hline
Ka\b lu-\index{gnl}{Kalumalam@Ka\b lumalam}& origine\index{gnl}{origine} du&&&&&&ville qui &\\
malam& poète\index{gnl}{poete@poète}&&&&&&lave les &\\
&(st. 12)&&&&&&péché\index{gnl}{peche@péché}s&\\
&&&&&&& (st. 12)&\\
\hline
\end{longtable}
\end{center}


\normalsize
\noindent
Ainsi, nous observons que les légende\index{gnl}{legende@légende}s les plus fr\'equemment mentionn\'ees sont celles qui fondent les toponymes de Piramapuram\index{gnl}{Piramapuram}, V\=e\d nupuram\index{gnl}{Venupuram@V\=e\d nupuram}, Cirapuram\index{gnl}{Cirapuram} et T\=o\d nipuram\index{gnl}{Tonipuram@T\=o\d nipuram}. C'est d'ailleurs la légende\index{gnl}{legende@légende} attach\'ee \`a cette derni\`ere appellation qui est la plus cit\'ee dans les autres hymne\index{gnl}{hymne}s attribu\'es \`a Campantar\index{gnl}{Campantar}.

%\subsubsection{La légende\index{gnl}{legende@légende} de T\=o\d nipuram\index{gnl}{Tonipuram@T\=o\d nipuram}}

\`A l'exception d'une allusion \`a l'\'etymologie de Piramapuram\index{gnl}{Piramapuram} dans un hymne\index{gnl}{hymne} d\'edi\'e \`a ce m\^eme toponyme (\textit{pirama\b n\=ur}, \og ville de Brahm\=a\index{gnl}{Brahma@Brahm\=a}\fg\ III 37.1), toutes les autres r\'ef\'erences aux légende\index{gnl}{legende@légende}s que nous avons relev\'ees portent sur celle de T\=o\d nipuram\index{gnl}{Tonipuram@T\=o\d nipuram}. Rappelons que d'apr\`es cette derni\`ere \'Siva\index{gnl}{Siva@\'Siva} et sa par\`edre sont venus, en barque\index{gnl}{barque}, s'installer au moment du déluge\index{gnl}{deluge@déluge} \`a C\=\i k\=a\b li\index{gnl}{Cikali@C\=\i k\=a\b li}, seule terre qui \'emergeait des eaux\index{gnl}{eau}.
\noindent
Nous trouvons des r\'ef\'erences \`a cette légende\index{gnl}{legende@légende} dans des poème\index{gnl}{poeme@poème}s c\'el\'ebrant C\=\i k\=a\b li\index{gnl}{Cikali@C\=\i k\=a\b li} sous divers noms (I 1.5; II 59.11, 83.10; III 2.9, 100.4, 100.5, 118.3 et 9) et dans l'envoi\index{gnl}{envoi} d'un hymne\index{gnl}{hymne} \`a la gloire d'un autre site (II 5.11):

\begin{enumerate}

\item
\scriptsize
\begin{verse}
\textit{\og orumai pe\d nmai u\d taiya\b n! ca\d taiya\b n! vi\d tai \=urum(m) iva\b n!\fg\ e\b n\b na\\
arumai \=aka uraiceyya amarntu, e\b natu u\d l\d lam kavar ka\d lva\b n---\\
\og \textbf{karumai pe\b r\b ra ka\d tal ko\d l\d la, mitantatu or k\=alam(m) itu}\fg\ e\b n\b nap\\
perumai pe\b r\b ra piram\=apuram m\=eviya pemm\=a\b n---iva\b n a\b n\b r\=e!} (I 1.5)\\
\end{verse}

\normalsize
\begin{verse}
Le voleur qui charme mon for int\'erieur\\
R\'esidant [l\`a] quand on dit excellemment:\\
\og Celui qui poss\`ede la f\'eminit\'e sur un c\^ot\'e,\\
Celui aux m\`eches,\\
Celui qui monte le teaureau\fg;\\
N'est-ce pas lui le Seigneur\\
Qui vit \`a Piramapuram\index{gnl}{Piramapuram} \`a la grandeur ainsi rapportée:\\
\og Un temps, quand la mer\index{gnl}{mer} noire recouvrait [tout],\\
Ceci flotta\index{gnl}{flotter}\fg. (I 1.5)\\
\end{verse}

\item
\scriptsize
\begin{verse}
\textit{\textbf{\=u\b li\=aya p\=aril \=o\.nkum uyar celvak\\
k\=a\b li} \=\i ca\b n ka\b lal\=e p\=e\d num campanta\b n,\\
t\=a\b lum ma\b natt\=al, uraitta tami\b lka\d livai vall\=ar,\\
v\=a\b li n\=\i \.nk\=a v\=a\b n\=or ulakil maki\b lv\=ar\=e.} (II 59.11)\\
\end{verse}

\normalsize
\begin{verse}
Ceux qui sont forts dans ces [vers] tamouls compos\'es,\\
Avec le c\oe ur qui r\'ev\`ere,\\
Par Campanta\b n\index{gnl}{Campantar!Campanta\b n} qui ne m\'edite que sur les pieds\\
Du Seigneur de K\=a\b li\index{gnl}{Kali@K\=a\b li}\\
\`A la prosp\'erit\'e qui monte et qui s'\'el\`eve dans le monde\\
\lbrack Au moment] du déluge\index{gnl}{deluge@déluge},\\
\lbrack Ceux-l\`a] seront heureux dans le monde des c\'elestes\\
O\`u ne cesse la vie. (II 59.11)\\
\end{verse}

\item
\scriptsize
\begin{verse}
\textit{i\b raiva\b nai, oppu il\=ata o\d li m\=e\b niy\=a\b nai, \textbf{ulaka\.nka\d l\=e\b lumu\d ta\b n\=e\\
ma\b raitaru ve\d l\d lam \=e\b ri va\d lar k\=oyil ma\b n\b ni i\b nit\=a irunta ma\d niyai},\\
ku\b raivu ila \~n\=a\b nam m\=evu ku\d lir panta\b n vaitta tami\b lm\=alai p\=a\d tumavar, p\=oy,\\
a\b rai ka\b lal \=\i ca\b n \=a\d lum nakar m\=evi, e\b n\b rum a\b lak\=a iruppatu a\b riv\=e.} (II 83.10)\\
\end{verse}

\normalsize
\begin{verse}
Ceux qui chantent la guirlande\index{gnl}{guirlande} tamoule ---\\
Pos\'ee par le frais Panta\b n\index{gnl}{Campantar!Panta\b n}\\
Qui habite dans la connaissance\index{gnl}{connaissance} sans manque\\
Sur le Seigneur,\\
Celui au corps brillant qui n'a pas d'\'egal,\\
Le joyau qui reste avec plaisir\\
Dans le grand temple\index{gnl}{temple} qui \'emergea pendant le déluge\index{gnl}{deluge@déluge}\\
Qui donne les \textit{Veda}\index{gnl}{Veda@\textit{Veda}} et les sept mondes ---\\
\lbrack Ceux-l\`a] vont habiter la ville\\
Que gouverne le Seigneur aux anneaux sonores (Kail\=asa\index{gnl}{Kailasa@Kail\=asa});\\
Il est connu qu'ils y restent toujours avec beaut\'e. (II 83.10)\\
\end{verse}

\item
\scriptsize
\begin{verse}
\textit{ko\.nku c\=er ku\b lal\=a\d l, ni\b lal ve\d nnakai, kovvaiv\=ay, ko\d ti \=er i\d taiy\=a\d lumai\\
pa\.nku c\=er tirum\=arpu u\d taiy\=ar; pa\d tar t\=\i uru \=ay,\\
ma\.nkulva\d n\d na\b num m\=a malar\=o\b num maya\.nka n\=\i \d n\d tavar; \textbf{v\=a\b nmicai vantue\b lu\\
po\.nkun\=\i ril mitanta na\b n p\=untar\=ay\index{gnl}{Taray@Tar\=ay}} p\=o\b r\b rutum\=e.} (III 2.9)\\
\end{verse}

\normalsize
\begin{verse}
Honorons la bonne et belle Tar\=ay\index{gnl}{Taray@Tar\=ay}\\
Qui flotta\index{gnl}{flotter} dans l'eau\index{gnl}{eau} d\'ebordante\\
Qui s'est \'elev\'ee jusqu'au ciel;\\
\lbrack Ville] de Celui qui s'est allong\'e,\\
Devenu un feu\index{gnl}{feu} se r\'epandant, pour confondre\\
Celui de la grande fleur et Celui \`a la couleur du nuage;\\
\lbrack Ville] de Celui qui poss\`ede un torse\\
Dont la moiti\'e est Um\=a\index{gnl}{Uma@Um\=a},\\
Celle \`a la taille de liane,\\
Celle \`a la bouche (couleur du fruit) \textit{kovvai},\\
Celle aux dents blanches et brillantes,\\
Celle aux cheveux pleins de parfum. (III 2.9)\\
\end{verse}

\item
\scriptsize
\begin{verse}
\textit{\og va\d l\d lal irunta malai ata\b nai valamceytal v\=aymai\fg\ e\b na\\
u\d l\d lam ko\d l\d l\=atu, kotittu e\b luntu, a\b n\b ru, e\d tutt\=o\b n uram neriya,\\
me\d l\d la viral vaittu, e\b n u\d l\d lam ko\d n\d t\=ar m\=evum i\d tamp\=olum---\\
\textbf{tu\d l oli ve\d l\d latti\b nm\=el mitanta t\=o\d nipuram\index{gnl}{Tonipuram@T\=o\d nipuram}}t\=a\b n\=e.} (III 100.4)\\
\end{verse}

\normalsize
\begin{verse}
Jadis, il se leva bouillonnant,\\
Posa doucement son orteil pour \'ecraser\\
La poitrine de celui qui souleva [la montagne] sans consid\'erer\\
Le bienfait\index{gnl}{bienfait} de circumambuler la montagne\\
O\`u r\'eside le G\'en\'ereux;\\
La demeure du Ravisseur de mon c\oe ur\\
Est bien T\=o\d nipuram\index{gnl}{Tonipuram@T\=o\d nipuram} qui,\\
Au clapotis des vague\index{gnl}{vague}s, flotta\index{gnl}{flotter} sur le déluge\index{gnl}{deluge@déluge}. (III 100.4)\\
\end{verse}

\item
\scriptsize
\begin{verse}
\textit{vel pa\b ravaik ko\d ti m\=alum, ma\b r\b rai viraimalarm\=el aya\b num,\\
pal pa\b ravaippa\d ti\=ay uyarntum, pa\b n\b ri atu\=ayp pa\d nintum,\\
celvu a\b ra n\=i\d n\d tu em cintai ko\d n\d ta celvar i\d tamp\=olum\\
\textbf{tol pa\b ravai cumantu \=o\.nku cemmait t\=o\d nipuram\index{gnl}{Tonipuram@T\=o\d nipuram}}t\=a\b n\=e.} (III 100.5)\\
\end{verse}

\normalsize
\begin{verse}
Malgr\'e la descente de M\=al\index{gnl}{Visnu@Vi\d s\d nu!Mal@M\=al} devenu sanglier,\\
A la banni\`ere figurant l'oiseau\index{gnl}{oiseau} victorieux,\\
Et l'ascension d'Aya\b n\index{gnl}{Brahma@Brahm\=a!Aya\b n} sur la fleur parfum\'ee,\\
Devenu une masse d'oiseau\index{gnl}{oiseau}x,\\
Il s'allongea pour que cesse [leur] d\'eplacement;\\
La demeure du Fortun\'e, ravisseur de notre pens\'ee,\\
Est bien T\=o\d nipuram\index{gnl}{Tonipuram@T\=o\d nipuram} la fertile,\\
Qui s'\'el\`eve port\'ee par d'antiques oiseau\index{gnl}{oiseau}x. (III 100.5)\\
\end{verse}

\item
\scriptsize
\begin{verse}
\textit{c\=\i r u\b ru to\d n\d tar, ko\d n\d tu a\d ti p\=o\b r\b ra, ce\b lu malar pu\b nalo\d tu t\=upam;\\
t\=ar u\b ru ko\b n\b rai tam mu\d ti vaitta caiva\b n\=ar ta\.nku i\d tam \textbf{e\.nkum\\
\=ur u\b ru patika\d l ulak\=u\d ta\b n po\.nki olipu\b nal ko\d la, u\d ta\b nmitanta},\\
k\=ar u\b ru cemmai na\b nmaiy\=al mikka \textbf{ka\b lumalanakar} e\b nal \=am\=e.} (III 118.3)\\
\end{verse}

\normalsize
\begin{verse}
Le lieu --- o\`u r\'eside le Shiva\"ite\index{gnl}{shiva\"ite}\\
Qui se couronne de la fleur de cassier en guirlande\index{gnl}{guirlande}\\
Alors que les dévot\index{gnl}{devot(e)@dévot(e)}s pleins de gloire louent [ses] pieds\\
Avec des fleurs fertiles, de l'eau\index{gnl}{eau} et de la fum\'ee ---\\
Est dit être la ville de Ka\b lumalam\index{gnl}{Kalumalam@Ka\b lumalam},\\
La prospère par le bienfait\index{gnl}{bienfait} de la fertilit\'e de la pluie,\\
Qui flotta\index{gnl}{flotter} quand l'eau\index{gnl}{eau} sonore, d\'ebordante,\\
Couvrit partout les lieux de ville\\
Et le [reste du] monde. (III 118.3)\\
\end{verse}

\item
\scriptsize
\begin{verse}
\textit{aru varai po\b rutta \=a\b r\b rali\b n\=a\b num, a\d ni ki\d lar t\=amaraiy\=a\b num,\\
iruvarum \=etta, eri uru \=a\b na i\b raiva\b n\=ar u\b raivu i\d tam vi\b navil,\\
\textbf{oruvar iv ulakil v\=a\b lkil\=a va\d n\d nam olipu\b nalve\d l\d lam mu\b n parappa,\\
karuvarai c\=u\b lnta ka\d tal i\d tai mitakkum ka\b lumalanakar} e\b nal \=am\=e.} (III 118.9)\\
\end{verse}

\normalsize
\begin{verse}
Si on demande le lieu o\`u demeure le Seigneur\\
Qui devint une forme de feu\index{gnl}{feu}\\
Sous les louanges des deux,\\
Celui \`a la force qui a soulev\'e la montagne rare\\
Et Celui du brillant et beau lotus,\\
On r\'epond que c'est la ville de Ka\b lumalam\index{gnl}{Kalumalam@Ka\b lumalam}\\
Qui flotta\index{gnl}{flotter} dans la mer\index{gnl}{mer} entour\'ee de montagnes noires\\
Quand, jadis, le déluge\index{gnl}{deluge@déluge} aux flots\index{gnl}{flots} sonores se propageait\\
De fa\c con \`a ce personne ne vive en ce monde. (III 118.9)\\
\end{verse}

\item
\scriptsize
\begin{verse}
\textit{\textbf{tollai \=u\b lip peyar t\=o\b n\b riya t\=o\d nipurattu i\b rai}---\\
nalla k\=e\d lvit tami\b l \~n\=a\b nacampanta\b n---nall\=arka\d lmu\b n\\
allal t\=\i ra uraiceyta a\b nekata\.nk\=avatam\\
colla, nalla a\d taiyum; a\d taiy\=a, cu\d tutu\b npam\=e.} (II 5.11)\\
\end{verse}

\normalsize
\begin{verse}
Quand on dit devant les [gens] bons\\
\lbrack Le chant sur] A\b nekata\.nk\=avatam\\
Compos\'e pour d\'etruire le malheur\\
Par le tamoul \~N\=a\b nacampanta\b n\index{gnl}{Campantar!N\=a\b nacampanta\b n@\~N\=a\b nacampanta\b n} \`a la bonne \'erudition\\
Sur le Seigneur de T\=o\d nipuram\index{gnl}{Tonipuram@T\=o\d nipuram}\\
Dont le nom apparut pendant l'ancien déluge\index{gnl}{deluge@déluge},\\
Le bien est atteint,\\
La souffrance qui br\^ule n'est pas atteinte. (II 5.11)\\
\end{verse}
\end{enumerate}

\noindent
C\=\i k\=a\b li\index{gnl}{Cikali@C\=\i k\=a\b li} est donc la ville qui, jadis (\textit{\=or k\=alam} I 1.5; \textit{tollai} II 5.11), flotta\index{gnl}{flotter} (\textit{mitantu} I 1.5, III 2.9, III 100.4, III 118.3 et 9) sur les eaux\index{gnl}{eau} (\textit{n\=\i r} III 2.9; \textit{ve\d l\d lam} II 83.10, III 100.4; \textit{pu\b nal} III 118.3; \textit{pu\b nalve\d l\d lam} III 118.9) ou la mer\index{gnl}{mer} (\textit{ka\d tal} I 1.5) du déluge\index{gnl}{deluge@déluge} (\textit{\=u\b li} II 59.11 et II 5.11). Si, en principe, T\=o\d nipuram\index{gnl}{Tonipuram@T\=o\d nipuram} est le nom attach\'e \`a ce mythe\index{gnl}{mythe} du déluge\index{gnl}{deluge@déluge} (III 100.4 et II 5.11), c'est aussi sous les appellations de Piramapuram\index{gnl}{Piramapuram} (I 1.5), K\=a\b li\index{gnl}{Kali@K\=a\b li} (II 59.11), Tar\=ay\index{gnl}{Taray@Tar\=ay} (III 2.9) et Ka\b lumalam\index{gnl}{Kalumalam@Ka\b lumalam} (III 118.3 et 9) que l'on s'y r\'ef\`ere. Dans l'hymne\index{gnl}{hymne} d\'edi\'e \`a Koccai\index{gnl}{Koccai} (II 83.10) il n'est m\^eme pas n\'ecessaire de pr\'eciser le toponyme pour situer la légende\index{gnl}{legende@légende}. Notons que d'apr\`es cette strophe c'est le temple\index{gnl}{temple} (\textit{k\=oyil}) qui flotte\index{gnl}{flotter} plut\^ot que le site. Ajoutons, enfin, que le quatrain III 100.5 mentionne un \'el\'ement particulier de la légende\index{gnl}{legende@légende} que nous n'avons pas encore rencontr\'e: le site s'\'el\`eve port\'e par des oiseau\index{gnl}{oiseau}x. Ce d\'etail apparaît également dans un hymne\index{gnl}{hymne} attribu\'e \`a Appar\index{gnl}{Appar}.

\subsection{Les douze légendes chez Appar et Cuntarar}

Appar\index{gnl}{Appar} et Cuntarar\index{gnl}{Cuntarar} connaissent la ville de C\=\i k\=a\b li\index{gnl}{Cikali@C\=\i k\=a\b li} dont est originaire Campantar\index{gnl}{Campantar} (voir 2.3.2). Cependant ils ne chantent ce site que sous les appellations de Ka\b lumalam\index{gnl}{Kalumalam@Ka\b lumalam} (IV 82, 83 et VII 58) et de T\=o\d nipuram\index{gnl}{Tonipuram@T\=o\d nipuram} (V 45). Le nom de K\=a\b li\index{gnl}{Kali@K\=a\b li} est donné dans VII 97.9. Aucun des neuf autres toponymes n'est mentionn\'e. Quant aux légende\index{gnl}{legende@légende}s, Appar et Cuntarar n'\'evoquent que celle de T\=o\d nipuram\index{gnl}{Tonipuram@T\=o\d nipuram}:
\begin{enumerate}

\item
\scriptsize
\begin{verse}
\textit{\textbf{p\=ar ko\d n\d tu m\=u\d tik ka\d tal ko\d n\d ta\d t\=a\b n\b ru ni\b n p\=atam ell\=am\\
n\=al a\~ncu pu\d li\b nam \=enti\b na e\b npar}; na\d lirmatiyam\\
k\=al ko\d n\d ta va\d nkaic ca\d tai virittu \=a\d tum ka\b lumalavarkku\\
\=a\d la\b n\b ri ma\b r\b rum u\d n\d too, am ta\d n \=a\b li akali\d tam\=e?} (IV 82.1)\\
\end{verse}

\normalsize
\begin{verse}
Le jour o\`u la mer\index{gnl}{mer} couvrit et prit la terre\index{gnl}{terre}\\
On dit que quatre [fois] cinq (vingt) oiseau\index{gnl}{oiseau}x\\
Ont port\'e tes pieds;\\
Y a-t-il sur cette vaste terre\index{gnl}{terre} [entour\'ee]\\
De beaux et frais oc\'eans\\
Autre chose que d'\^etre\\
Le serviteur\index{gnl}{serviteur} de Celui de Ka\b lumalam\index{gnl}{Kalumalam@Ka\b lumalam}\\
Qui a la main g\'en\'ereuse\\
Et qui danse\index{gnl}{danser} les cheveux l\^ach\'es\\
Dans lesquels r\'eside la fra\^iche lune? (IV 82.1)\\
\end{verse}

\item
\scriptsize
\begin{verse}
\textit{nilaiyum perumaiyum n\=\i tiyum c\=ala a\b laku u\d taittu\=ay,\\
\textbf{alaiyum peruve\d l\d lattu a\b n\b ru mitanta it t\=o\d nipuram\index{gnl}{Tonipuram@T\=o\d nipuram}},\\
cilaiyil tiri puramm\=u\b n\b ru eritt\=ar, tam ka\b lumalavar,\\
alarum ka\b lala\d ti n\=a\d lto\b rum namtamai \=a\d lva\b nav\=e.} (IV 82.6)\\
\end{verse}

\normalsize
\begin{verse}
Celui qui a consum\'e avec l'arc\\
Les trois citadelles errantes,\\
Celui là m\^eme de Ka\b lumalam\index{gnl}{Kalumalam@Ka\b lumalam},\\
De ce T\=o\d nipuram\index{gnl}{Tonipuram@T\=o\d nipuram} qui jadis\\
Flotta\index{gnl}{flotter} sur le grand déluge\index{gnl}{deluge@déluge} mouvant\\
Et qui a obtenu avec beaucoup de beaut\'e\\
La perp\'etuit\'e, la grandeur et la justice;\\
\lbrack Ses] pieds de fleurs, aux anneaux,\\
Nous gouverneront tous les jours. (IV 82.6)\\
\end{verse}

\item
\scriptsize
\begin{verse}
\textit{c\=atalum pi\b rattalum tavirttu, e\b nai vakuttu, ta\b n aru\d l tanta em talaiva\b nai; malaiyi\b n\\
m\=ati\b nai matittu, a\.nku orp\=al ko\d n\d ta ma\d niyai; varupu\b nal ca\d tai i\d tai vaitta emm\=a\b nai;\\
\=etile\b n ma\b nattukku or irumpu u\d n\d ta n\=\i rai; e\d nvakai oruva\b nai; e\.nka\d l pir\=a\b nai;\\
k\=atil ve\.nku\b laiya\b nai; \textbf{ka\d tal ko\d la mitanta ka\b lumala va\d la nakar}k ka\d n\d tuko\d n\d t\=e\b n\=e.} (VII 58.1)\\
\end{verse}

\normalsize
\begin{verse}
Dans la ville fertile de Ka\b lumalam\index{gnl}{Kalumalam@Ka\b lumalam}\\
Qui, envahie par la mer\index{gnl}{mer}, flotta\index{gnl}{flotter}\\
J'ai vu mon chef\index{gnl}{chef} qui accorde gr\^ace\\
\lbrack M']ayant retir\'e la mort et la renaissance\\
Et m'ayant gouvern\'e,\\
Le Joyau qui, ayant aim\'e la femme\index{gnl}{femme} de la montagne,\\
\lbrack La] porte sur une moiti\'e,\\
Mon Seigneur qui a mis dans ses m\`eches l'eau\index{gnl}{eau} qui coule,\\
Celui qui est [comme] l'eau\index{gnl}{eau} qui consomma le fer [chaud]\\
Dans mon coeur \'etranger,\\
Celui aux huit formes,\\
Notre seigneur,\\
Celui \`a la boucle blanche \`a l'oreille. (VII 58.1)\\
\end{verse}
\end{enumerate}

\noindent
C'est uniquement la légende\index{gnl}{legende@légende} du déluge\index{gnl}{deluge@déluge}, associ\'ee au toponyme de T\=o\d nipuram\index{gnl}{Tonipuram@T\=o\d nipuram}, qui est mentionn\'ee dans ces deux hymne\index{gnl}{hymne}s c\'el\'ebrant Ka\b lumalam\index{gnl}{Kalumalam@Ka\b lumalam}. Appar\index{gnl}{Appar} y fait r\'ef\'erence sur deux strophes dans lesquelles nous apprenons que T\=o\d nipuram\index{gnl}{Tonipuram@T\=o\d nipuram} est identifi\'e comme Ka\b lumalam\index{gnl}{Kalumalam@Ka\b lumalam} (IV 82.6) et que \'Siva\index{gnl}{Siva@\'Siva} est transport\'e dans les airs par des oiseau\index{gnl}{oiseau}x au moment du déluge\index{gnl}{deluge@déluge} (IV 82.1). Ainsi, dans l'ensemble des poème\index{gnl}{poeme@poème}s du \textit{T\=ev\=aram}\index{gnl}{Tevaram@\textit{T\=ev\=aram}}, \`a l'exception des onze hymne\index{gnl}{hymne}s sur les douze\index{gnl}{douze} noms attribu\'es \`a Campantar\index{gnl}{Campantar}, seule une légende\index{gnl}{legende@légende} sur douze\index{gnl}{douze}, celle de T\=o\d nipuram\index{gnl}{Tonipuram@T\=o\d nipuram}, est d\'ecrite. Cela signifie-t-il que les autres légende\index{gnl}{legende@légende}s ne sont pas encore d\'evelopp\'ees ou qu'elles ne sont pas encore li\'ees \`a C\=\i k\=a\b li\index{gnl}{Cikali@C\=\i k\=a\b li}?


\section{Les douze noms de C\=\i k\=a\b li, un artifice?}

R\'ecapitulons en guise de conclusion \`a ce chapitre les informations et probl\`emes recens\'es autour des douze\index{gnl}{douze} noms de C\=\i k\=a\b li\index{gnl}{Cikali@C\=\i k\=a\b li} dans le \textit{T\=ev\=aram}\index{gnl}{Tevaram@\textit{T\=ev\=aram}}.

\subsection{Les probl\`emes chez Campantar}

Avec un total de soixante-et-onze hymne\index{gnl}{hymne}s (soixante-sept attribu\'es \`a Campantar\index{gnl}{Campantar}, trois \`a Appar\index{gnl}{Appar} et un \`a Cuntarar\index{gnl}{Cuntarar}), C\=\i k\=a\b li\index{gnl}{Cikali@C\=\i k\=a\b li} est le site le plus c\'el\'ebr\'e dans le \textit{T\=ev\=aram}\index{gnl}{Tevaram@\textit{T\=ev\=aram}}. Ceci s'explique parce qu'il s'agit de la ville natale du poète\index{gnl}{poete@poète} Campantar\index{gnl}{Campantar} mais surtout parce que douze\index{gnl}{douze} appellations furent attribuées \`a cette ville.
En effet, C\=\i k\=a\b li\index{gnl}{Cikali@C\=\i k\=a\b li} poss\`ede douze\index{gnl}{douze} toponymes distincts qui sont pr\'esent\'es, parfois avec leurs mythe\index{gnl}{mythe}s fondateurs, dans dix poème\index{gnl}{poeme@poème}s \`a douze\index{gnl}{douze} strophes (I 63, I 90, I 117, I 127, II 70, II 73, II 74, III 67, III 110 et III 113) et dans un poème\index{gnl}{poeme@poème} en prose (I 128) attribu\'es \`a Campantar\index{gnl}{Campantar}. \`A l'exception de deux poème\index{gnl}{poeme@poème}s (II 73 et II 74), tous ces textes reprennent les douze\index{gnl}{douze} dans l'ordre\index{gnl}{ordre} suivant: Piramapuram\index{gnl}{Piramapuram}, V\=e\d nupuram\index{gnl}{Venupuram@V\=e\d nupuram}, Pukali\index{gnl}{Pukali}, Ve\.nkuru\index{gnl}{Venkuru@Ve\.nkuru}, T\=o\d nipuram\index{gnl}{Tonipuram@T\=o\d nipuram}, Tar\=ay\index{gnl}{Taray@Tar\=ay}, Cirapuram\index{gnl}{Cirapuram}, Pu\b ravam\index{gnl}{Puravam@Pu\b ravam}, Ca\d npai\index{gnl}{Canpai@Ca\d npai}, K\=a\b li\index{gnl}{Kali@K\=a\b li}, Koccai\index{gnl}{Koccai} et Ka\b lumalam\index{gnl}{Kalumalam@Ka\b lumalam}. Onze hymne\index{gnl}{hymne}s sont compos\'es selon des procédé\index{gnl}{procédé littéraire}s litt\'eraires sophistiqu\'es qui s'\'ecartent, par leur complexit\'e et leur exclusivit\'e, de l'esprit populaire du reste du corpus\index{gnl}{corpus}. Nous avons d'ailleurs sugg\'er\'e au chapitre pr\'ec\'edent que ces poème\index{gnl}{poeme@poème}s \`a procédé\index{gnl}{procédé littéraire}s pouvaient \^etre des ajouts post\'erieurs à un corpus\index{gnl}{corpus} \og premier\fg.

Les variations dans les légende\index{gnl}{legende@légende}s que dix de ces onze hymnes rapportent viennent appuyer l'hypothèse que ces onze textes n'ont pas \'et\'e \'ecrits \`a la m\^eme \'epoque par le m\^eme auteur. Les descriptions des mythe\index{gnl}{mythe}s fondateurs donn\'ees dans l'hymne\index{gnl}{hymne} III 67 refl\`etent une maturit\'e des légendes qui contraste tr\`es clairement avec les allusions l\'egendaires des autres hymne\index{gnl}{hymne}s. Alors que la strophe I 63.6 \'evoque le r\`egne d'un roi\index{gnl}{roi} vaillant \`a Tar\=ay\index{gnl}{Taray@Tar\=ay}, III 67.6 mentionne le péché\index{gnl}{peche@péché} commis par Var\=aha\index{gnl}{Varaha@Var\=aha}. Le r\`egne de Cilampa\b n\index{gnl}{Cilampan@Cilampa\b n}, ou plut\^ot de sa t\^ete, est chant\'e dans plusieurs poème\index{gnl}{poeme@poème}s (I 63, II 70, II 73 et II 74) alors que III 67 rapporte le péché\index{gnl}{peche@péché} de la plan\`ete R\=ahu\index{gnl}{Rahu@R\=ahu}. Il est question d'un simple roi\index{gnl}{roi} \`a char dans I 63 mais III 67 explique le nom de Pu\b ravam\index{gnl}{Puravam@Pu\b ravam} avec le mythe\index{gnl}{mythe} du roi\index{gnl}{roi} \'Sibi\index{gnl}{Sibi@\'Sibi}. Quand I 63 renvoie \`a un certain Ca\d n\d ta\b n\index{gnl}{Cantan@Ca\d n\d ta\b n} \`a Ca\d npai\index{gnl}{Canpai@Ca\d npai}, III 67 rappelle la mal\'ediction des cent Y\=adava\index{gnl}{Yadava@Y\=adava}. Alors que I 63 traite du r\`egne du roi\index{gnl}{roi} des serpent\index{gnl}{serpent}s \`a K\=a\b li\index{gnl}{Kali@K\=a\b li}, III 67 expose la d\'efaite de la d\'eesse\index{gnl}{deesse@déesse} K\=a\d li. Quant \`a Koccai\index{gnl}{Koccai}, I 63 d\'ecrit le r\`egne d'un certain Nanta\b n\index{gnl}{Nantan@Nanta\b n} et III 67 raconte la souillure\index{gnl}{souillure} d'un sage\index{gnl}{sage}. Ainsi, les douze strophes de III 67 présentent les douze légendes de C\=\i k\=a\b li telles qu'elles sont décrites dans le \textit{talapur\=a\d nam} du site. Par ailleurs, concernant le mythe fondateur du toponyme K\=a\b li\index{gnl}{Kali@K\=a\b li}, les principaux \textit{talapur\=a\d nam}\index{gnl}{Purana@\textit{Pur\=a\d na}!\textit{talapur\=a\d nam}} en sanskrit, le \textit{Cidambaram\=ah\=atmya}\index{gnl}{Cidambara@\textit{Cidambaram\=ah\=atmya}}, et en tamoul, le \textit{K\=oyi\b rpur\=a\d nam}\index{gnl}{Koyil@\textit{K\=oyi\b rpur\=a\d nam}}, ne mentionnent pas la comp\'etition de danse\index{gnl}{danse}. Il semble que la d\'efaite de la d\'eesse\index{gnl}{deesse@déesse} K\=a\d li lors d'une comp\'etition de danse\index{gnl}{danse} contre \'Siva\index{gnl}{Siva@\'Siva} est un fait qui est conté pour la premi\`ere fois dans une des versions sanskrites mineures du \textit{talapur\=a\d nam}\index{gnl}{Purana@\textit{Pur\=a\d na}!\textit{talapur\=a\d nam}} de Citamparam\index{gnl}{Citamparam}, le \textit{Vy\=aghrapuram\=ah\=atmya}\index{gnl}{Vyaghrapuram@\textit{Vy\=aghrapuram\=ah\=atmya}} dont la datation serait post\'erieure au \textsc{xiv}\up{e} si\`ecle (\textsc{Smith} 1998: 143-145). Ainsi, nous supposons que III 67 est un poème\index{gnl}{poeme@poème} ajout\'e au corpus\index{gnl}{corpus} \og premier\fg\ du \textit{T\=ev\=aram}\index{gnl}{Tevaram@\textit{T\=ev\=aram}}.

Nous avons remarqu\'e au chapitre pr\'ec\'edent que, dans les envois\index{gnl}{envoi} attribu\'es \`a Campantar\index{gnl}{Campantar}, les douze\index{gnl}{douze} toponymes ne sont pas consid\'er\'es sur le m\^eme plan. K\=a\b li\index{gnl}{Kali@K\=a\b li} pr\'edomine grandement suivi de Pukali\index{gnl}{Pukali}, Ka\b lumalam\index{gnl}{Kalumalam@Ka\b lumalam} et Ca\d npai\index{gnl}{Canpai@Ca\d npai}. Ce traitement in\'egal des noms dans les envois\index{gnl}{envoi} nous conduit \`a douter d'une entit\'e de douze\index{gnl}{douze} appellations parfaitement \'etablie d\`es les premières étapes de la formation du \textit{T\=ev\=aram}\index{gnl}{Tevaram@\textit{T\=ev\=aram}} actuel. Ce doute est renforc\'e par le t\'emoignage des hymne\index{gnl}{hymne}s attribu\'es aux deux autres \textit{m\=uvar}\index{gnl}{muvar@\textit{m\=uvar}}.

\subsection{Depuis Appar jusqu'aux inscriptions}

Nous avons observ\'e que les quatre hymne\index{gnl}{hymne}s attribu\'es \`a Appar\index{gnl}{Appar} (IV 82, 83 et V 45) et Cuntarar\index{gnl}{Cuntarar} (VII 58) glorifient C\=\i k\=a\b li\index{gnl}{Cikali@C\=\i k\=a\b li} sous les noms de Ka\b lumalam\index{gnl}{Kalumalam@Ka\b lumalam} et T\=o\d nipuram\index{gnl}{Tonipuram@T\=o\d nipuram}. Quand ces deux poète\index{gnl}{poete@poète}s mentionnent l'origine\index{gnl}{origine} g\'eographique de Campantar\index{gnl}{Campantar}, ils le lient \`a Ka\b lumalam\index{gnl}{Kalumalam@Ka\b lumalam} (IV 56.1) et \`a K\=a\b li\index{gnl}{Kali@K\=a\b li} (VII 97.9). Appar\index{gnl}{Appar} identifie m\^eme T\=o\d nipuram\index{gnl}{Tonipuram@T\=o\d nipuram} comme Ka\b lumalam\index{gnl}{Kalumalam@Ka\b lumalam} (IV 82.6).
Ajoutons que le poème\index{gnl}{poeme@poème} \textit{K\=\i rttitiruvakaval} du \textit{Tiruv\=acakam}\index{gnl}{Tiruvacakam@\textit{Tiruv\=acakam}}, attribu\'e \`a M\=a\d nikkav\=acakar\index{gnl}{Manikkavacakar@M\=a\d nikkav\=acakar} et \'etablissant une liste de lieux saints shiva\"ite\index{gnl}{shiva\"ite}s, ne mentionne que Ka\b lumalam\index{gnl}{Kalumalam@Ka\b lumalam} au vers 88. Ailleurs, le poète\index{gnl}{poete@poète} vishnouite\index{gnl}{vishnouite} Tiruma\.nkaiy\=a\b lv\=ar\index{gnl}{Tirumankaiyalvar@Tiruma\.nkaiy\=a\b lv\=ar} chante le site sous l'appellation de K\=a\b li\index{gnl}{Kali@K\=a\b li} dans le \textit{Periyatirumo\b li}\index{gnl}{Periyatirumoli@\textit{Periyatirumo\b li}} (1178-1197).
Ces poète\index{gnl}{poete@poète}s qui sont les plus proches de Campantar\index{gnl}{Campantar} dans le temps, entre le \textsc{vii}\up{e} et \textsc{ix}\up{e} si\`ecle, ne font nullement r\'ef\'erence \`a l'unit\'e des douze\index{gnl}{douze} toponymes de C\=\i k\=a\b li\index{gnl}{Cikali@C\=\i k\=a\b li}. Ils ne reprennent que les noms attest\'es dans les sources historiques, les textes \'epigraphiques.

Ka\b lumalam\index{gnl}{Kalumalam@Ka\b lumalam} et T\=o\d nipuram\index{gnl}{Tonipuram@T\=o\d nipuram} sont, en r\'ealit\'e, les toponymes qui d\'esignent le site dans les inscriptions, et ce jusqu'au \textsc{xiv}\up{e} si\`ecle. T\=o\d nipuram\index{gnl}{Tonipuram@T\=o\d nipuram} est le nom du temple\index{gnl}{temple}, ou du bourg l'entourant, et compose le nom de \'Siva\index{gnl}{Siva@\'Siva} ou du \textit{li\.nga}\index{gnl}{linga@\textit{li\.nga}} appel\'e T\=o\d nipuramu\d taiy\=ar. Quant \`a Ka\b lumalam\index{gnl}{Kalumalam@Ka\b lumalam}, il renvoie \`a une division territoriale plus grande, le \textit{va\d lan\=a\d tu}, englobant divers villages (voir la troisi\`eme partie). La premi\`ere attestation de source historique, que nous connaissions, du terme C\=\i k\=a\b li\index{gnl}{Cikali@C\=\i k\=a\b li} ou K\=a\b li\index{gnl}{Kali@K\=a\b li} se trouve dans une inscription du \textsc{xi}\up{e} si\`ecle du temple\index{gnl}{temple} de Ta\~nc\=av\=ur\index{gnl}{Tancavur@Ta\~nc\=av\=ur}. Elle entre dans la composition du nom d'un chanteur de \textit{tiruppatiyam}\index{gnl}{tiruppatiyam@\textit{tiruppatiyam}} appel\'e K\=a\b li\index{gnl}{Kali@K\=a\b li} Campanta\b n\index{gnl}{Campantar!Campanta\b n} (SII 2 65 l.11).
Ainsi, les textes litt\'eraires qui ne sont pas attribu\'es \`a Campantar\index{gnl}{Campantar} mais qui lui sont plus ou moins contemporains, ainsi que les textes \'epigraphiques du temple\index{gnl}{temple} m\^eme de C\=\i k\=a\b li\index{gnl}{Cikali@C\=\i k\=a\b li} ne pr\'esentent pas ce site comme poss\'edant douze\index{gnl}{douze} toponymes.
Ce constat nous conduit \`a proposer l'hypoth\`ese que certains des douze\index{gnl}{douze} noms de C\=\i k\=a\b li\index{gnl}{Cikali@C\=\i k\=a\b li} ne sont pas \`a l'origine\index{gnl}{origine} des toponymes renvoyant \`a la localit\'e de C\=\i k\=a\b li\index{gnl}{Cikali@C\=\i k\=a\b li} mais qu'ils ont \'et\'e ajout\'es \`a ce corpus\index{gnl}{corpus} du \textit{T\=ev\=aram} ou qu'ils ont \'et\'e attach\'es au site de C\=\i k\=a\b li arbitrairement\footnote{Nous connaissons des exemples de sites poss\'edant plusieurs temples\index{gnl}{temple}: deux \`a K\=a\d t\d tuppa\d l\d li (\textit{K\=\i \b lai} I 5 et \textit{M\=elai} III 29), deux \`a Kura\.nk\=a\d tutu\b rai (\textit{Te\b n} II 35 et \textit{Va\d ta} III 91), deux \`a Kacci (\=Ekampam I 133, II 12, III 41, III 114 et Ne\b rikk\=araikk\=a\d tu III 65).}.
Ces douze\index{gnl}{douze} appellations ne sont c\'el\'ebr\'ees qu'\`a partir de la premi\`ere phase de mise en légende\index{gnl}{legende@légende} des textes de la \textit{bhakti}\index{gnl}{bhakti@\textit{bhakti}} shiva\"ite\index{gnl}{shiva\"ite} tamoule et de leurs poète\index{gnl}{poete@poète}s, au \textsc{xii}\up{e} si\`ecle environ.

\begin{center}
*
\end{center}

Les poème\index{gnl}{poeme@poème}s du \textit{T\=ev\=aram}\index{gnl}{Tevaram@\textit{T\=ev\=aram}} c\'el\'ebrant C\=\i k\=a\b li\index{gnl}{Cikali@C\=\i k\=a\b li} repr\'esentent aujourd'hui les plus anciennes r\'ef\'erences au site et constituent donc la source principale de la premi\`ere partie de notre \'etude. Ces hymne\index{gnl}{hymne}s et le reste du corpus\index{gnl}{corpus} ont \'et\'e chant\'es et honor\'es dans l'enceinte du temple\index{gnl}{temple}. Ils le sont toujours. Leurs auteurs, d\`es le \textsc{xi}\up{e} si\`ecle, int\`egrent le panth\'eon divin et font l'objet d'un culte\index{gnl}{culte}s. Mais ce corpus du \textit{T\=ev\=aram}, consacr\'e par la tradition\index{gnl}{tradition}, nous paraît être un ensemble composite qui nécessite l'établissement d'une édition critique. L'analyse de quelques donn\'ees internes nous a permis de sugg\'erer des interpolation\index{gnl}{interpolation}s et l'unit\'e des douze\index{gnl}{douze} toponymes du site de C\=\i k\=a\b li\index{gnl}{Cikali@C\=\i k\=a\b li} nous est apparu comme étant un artifice, très probablement, form\'e postérieurement.

Les hymnes du \textit{T\=ev\=aram} célèbrent des sites. Ces hymne\index{gnl}{hymne}s et leurs poète\index{gnl}{poete@poète}s sont \`a leur tour c\'el\'ebr\'es au \textsc{xii}\up{e} si\`ecle. Notre deuxi\`eme partie, \`a travers l'\'etude de quelques textes du \textit{Tirumu\b rai}\index{gnl}{Tirumurai@\textit{Tirumu\b rai}}, examine la formation l\'egendaire de ce site et de son poète\index{gnl}{poete@poète} Campantar\index{gnl}{Campantar}. Des légende\index{gnl}{legende@légende}s se fixent. Des héros\index{gnl}{heros@héros} y naissent.
%Les donn\'ees nous manquent pour pr\'eciser avec exactitude le moment o\`u Campantar\index{gnl}{Campantar}, le poète\index{gnl}{poete@poète}, devient l'enfant\index{gnl}{enfant} prodigieux, auteur de miracle\index{gnl}{miracle}s. Cependant, nous constatons qu'il jouit d'une remarquable notori\'et\'e dans les textes assign\'es \`a Appar\index{gnl}{Appar}, et surtout, \`a Cuntarar\index{gnl}{Cuntarar}, en tant que poète\index{gnl}{poete@poète} originaire de C\=\i k\=a\b li\index{gnl}{Cikali@C\=\i k\=a\b li}. Puis, sa figure d'enfant\index{gnl}{enfant} fabuleux, qui ne semble \^etre attest\'ee qu'\`a partir du \textsc{x}\up{e} si\`ecle, au plus t\^ot, devient une v\'eritable légende\index{gnl}{legende@légende} dans les poème\index{gnl}{poeme@poème}s attribu\'es \`a Nampi \=A\d n\d t\=ar Nampi. Pr\'esentons maintenant les textes des \textit{Tirumu\b rai} \textsc{iv-vii} et \textsc{xi} mentionnant Campantar\index{gnl}{Campantar} selon l'ordre\index{gnl}{ordre} chronologique traditionnel.



\part{Héros}

Les hymne\index{gnl}{hymne}s du \textit{T\=ev\=aram}\index{gnl}{Tevaram@\textit{T\=ev\=aram}} ont int\'egr\'e le service\index{gnl}{service} religieux des temples\index{gnl}{temple} shiva\"ite\index{gnl}{shiva\"ite}s et de la vie domestique. Les images de leurs auteurs, \'elev\'es au rang de demi-dieux, sont install\'ees dans les lieux de culte\index{gnl}{culte} d\`es le \textsc{xi}\up{e} si\`ecle. Les r\'ecits de leurs vies, probablement transmis oralement tout d'abord, sont consign\'es au \textsc{xii}\up{e} si\`ecle par \'ecrit dans l'hagiographie\index{gnl}{hagiographie}\footnote{Bien que ce terme ait une signification particuli\`ere dans le christianisme, il s'applique, aujourd'hui, dans les \'etudes indiennes \`a une litt\'erature de biographies sacr\'ees. Nous suivons \textsc{Mallison} (2001: viii) qui, la derni\`ere en date, justifie cet emploi.} composée par C\=ekki\b l\=ar\index{gnl}{Cekkilar@C\=ekki\b l\=ar}, dans laquelle ces figures religieuses deviennent les héros\index{gnl}{heros@héros} de la conqu\^ete du shiva\"isme\index{gnl}{shivaisme@shiva\"isme} au Pays Tamoul\index{gnl}{Pays Tamoul}. La \og Légende dorée\fg\ de Campantar\index{gnl}{Campantar} est exemplaire: de poète\index{gnl}{poete@poète} originaire d'une famille de \textit{gotra\index{gnl}{gotra@\textit{gotra}} kau\d n\d dinya\index{gnl}{kaundinya@\textit{kau\d n\d dinya}}} de C\=\i k\=a\b li\index{gnl}{Cikali@C\=\i k\=a\b li} qui exprime dans ses poème\index{gnl}{poeme@poème}s son amour absolu de \'Siva\index{gnl}{Siva@\'Siva} tout en scandant sa haine des hérétique\index{gnl}{heretique@hérétique}s, il devient dans son hagiographie\index{gnl}{hagiographie} un enfant\index{gnl}{enfant} prodige\index{gnl}{prodige}, héros\index{gnl}{heros@héros} de C\=\i k\=a\b li\index{gnl}{Cikali@C\=\i k\=a\b li}, qui charme et convertit la population avec ses hymne\index{gnl}{hymne}s produisant des miracle\index{gnl}{miracle}s lors de ses pèlerinage\index{gnl}{pelerinage@pèlerinage}s-conqu\^etes dans le Pays Tamoul\index{gnl}{Pays Tamoul}. Dans cette deuxi\`eme partie de notre travail nous explorons la figure de Campantar\index{gnl}{Campantar}, l'histoire de sa légende\index{gnl}{legende@légende}.

Ainsi, le chapitre 4 d\'efinit et pr\'esente de fa\c con g\'en\'erale les textes du \textit{Tirumu\b rai}\index{gnl}{Tirumurai@\textit{Tirumu\b rai}} qui seront exploit\'es pour \'etudier l'\'evolution et la fixation des légende\index{gnl}{legende@légende}s de Campantar\index{gnl}{Campantar} et du temple\index{gnl}{temple} de C\=\i k\=a\b li\index{gnl}{Cikali@C\=\i k\=a\b li}. Leur choix repose sur l'importance qu'ils accordent au poète\index{gnl}{poete@poète} et au site pour les p\'eriodes ant\'erieure et, probablement, contemporaine des premiers t\'emoignages \'epigraphiques de C\=\i k\=a\b li\index{gnl}{Cikali@C\=\i k\=a\b li}\footnote{Parce qu'ils ne sont que le reflet de légende\index{gnl}{legende@légende}s parfaitement cristallis\'ees, les textes qui c\'el\`ebrent ce site et son poète\index{gnl}{poete@poète} post\'erieurs au \textsc{xiii}\up{e} si\`ecle ne sont pas \'etudi\'es dans cette partie.}.

Le chapitre 5 \'etudie les donn\'ees textuelles et iconographiques disponibles à propos du héros\index{gnl}{heros@héros} Campantar\index{gnl}{Campantar} pour comprendre l'image de l'enfant\index{gnl}{enfant} qui est, \`a notre avis, absente du \textit{T\=ev\=aram}\index{gnl}{Tevaram@\textit{T\=ev\=aram}}.

Le chapitre 6 essaie de reconstituer le m\'ecanisme hagiographique\index{gnl}{hagiographie!hagiographique} mis en place par le second héros\index{gnl}{heros@héros} de cette partie, C\=ekki\b l\=ar\index{gnl}{Cekkilar@C\=ekki\b l\=ar} qui fixa la légende\index{gnl}{legende@légende} du héros\index{gnl}{heros@héros} de C\=\i k\=a\b li\index{gnl}{Cikali@C\=\i k\=a\b li}.

\chapter{Les textes de la mise en légende}


Le \textit{Tirumu\b rai}\index{gnl}{Tirumurai@\textit{Tirumu\b rai}} renvoie aujourd'hui \`a douze\index{gnl}{douze} livres contenant divers textes religieux louant \'Siva\index{gnl}{Siva@\'Siva}, ses temples\index{gnl}{temple} et ses dévot\index{gnl}{devot(e)@dévot(e)}s. Ses sept premiers livres forment le \textit{T\=ev\=aram}\index{gnl}{Tevaram@\textit{T\=ev\=aram}}. Le huiti\`eme rassemble les \oe uvres attribu\'ees \`a M\=a\d nikkav\=acakar\index{gnl}{Manikkavacakar@M\=a\d nikkav\=acakar} (le \textit{Tiruv\=acakam}\index{gnl}{Tiruvacakam@\textit{Tiruv\=acakam}} et le \textit{Tirukk\=ovaiy\=ar}\index{gnl}{Tirukkovaiyar@\textit{Tirukk\=ovaiy\=ar}}) et le neuvi\`eme est un volume composite divis\'e en deux parties (le \textit{Tiruvicaipp\=a}\index{gnl}{Tiruvicaippa@\textit{Tiruvicaipp\=a}} et le \textit{Tiruppall\=a\d n\d tu}\index{gnl}{Tiruppallantu@\textit{Tiruppall\=a\d n\d tu}}) contenant les hymne\index{gnl}{hymne}s de neuf poète\index{gnl}{poete@poète}s c\'el\'ebrant au total quatorze sites. Le dixi\`eme est consacr\'e \`a l'ouvrage de Tirum\=ular\index{gnl}{Tirumular@Tirum\=ular}, le \textit{Tirumantiram}\index{gnl}{Tirumantiram@\textit{Tirumantiram}}. Le onzi\`eme regroupe en un m\'elange quarante textes de douze\index{gnl}{douze} auteurs dont K\=araikk\=alammaiy\=ar\index{gnl}{Karaikkalammaiyar@K\=araikk\=alammaiy\=ar}, Pa\d t\d ti\b nattuppi\d l\d lai\index{gnl}{Pattinattu Pillai@Pa\d t\d ti\b nattuppi\d l\d lai} et Nampi \=A\d n\d t\=ar Nampi\index{gnl}{Nampi \=A\d n\d t\=ar Nampi}. Et enfin, le dernier volume est l'hagiographie\index{gnl}{hagiographie} compos\'ee par C\=ekki\b l\=ar\index{gnl}{Cekkilar@C\=ekki\b l\=ar}, le \textit{Tirutto\d n\d tarpur\=a\d nam}\index{gnl}{Periyapuranam@\textit{Periyapur\=a\d nam}!\textit{Tirutto\d n\d tarpur\=a\d nam}}, nomm\'e aussi \textit{Periyapur\=a\d nam}\index{gnl}{Periyapuranam@\textit{Periyapur\=a\d nam}}.

Avant d'examiner les textes de la mise en légende\index{gnl}{legende@légende} de C\=\i k\=a\b li\index{gnl}{Cikali@C\=\i k\=a\b li} et de son poète\index{gnl}{poete@poète}, nous offrons une pr\'esentation du corpus\index{gnl}{corpus} du \textit{Tirumu\b rai}\index{gnl}{Tirumurai@\textit{Tirumu\b rai}} qui confronte la légende\index{gnl}{legende@légende} élaborée à son propos dans un texte litt\'eraire aux donn\'ees archéologiques fournies par l'\'epigraphie.

\section{Le \textit{Tirumu\b rai} entre légende et histoire}

\subsection{La légende du \textit{Tirumu\b rai}}

Un r\'ecit l\'egendaire, le \textit{Tirumu\b raika\d n\d tapur\=a\d nam}\index{gnl}{Tirumuraikantapuranam@\textit{Tirumu\b raika\d n\d tapur\=a\d nam}}, \og légende\index{gnl}{legende@légende} de la formation du Canon\index{gnl}{canon} sacr\'e\fg\footnote{Dans leurs traductions de ce titre, \textsc{Peterson} (*1991 [1989]: 15) met l'accent sur la d\'ecouverte du corpus\index{gnl}{corpus} unitaire \og The Story of the Discovery of the Tirumu\b rai\fg\ alors que \textsc{Prentiss} (2001a) souligne la compilation\index{gnl}{compilation} des \oe uvres constituantes du corpus\index{gnl}{corpus} \og The story of bringing together the holy collections\fg.}, explique l'ordonnance\index{gnl}{ordonnance} du corpus\index{gnl}{corpus} du \textit{Tirumu\b rai}, excluant le douzi\`eme volume. Ce texte de quarante-cinq strophes (sans l'invocation\index{gnl}{invocation}), traditionnellement attribu\'e \`a Um\=apati\index{gnl}{Umapati@Um\=apati} Civ\=ac\=ariyar, raconte comment Nampi \=A\d n\d t\=ar Nampi\index{gnl}{Nampi \=A\d n\d t\=ar Nampi} compile l'\oe uvre\footnote{Le r\'esum\'e qui suit est fond\'e sur l'examen du texte pr\'esent\'e dans le premier volume de l'\'edition du \textit{Periyapur\=a\d nam}\index{gnl}{Periyapuranam@\textit{Periyapur\=a\d nam}} de Ci. K\=e. \textsc{Cuppirama\d niya Mutaliy\=ar}, p.~33-38. Pour d'autres r\'esum\'es de cette légende\index{gnl}{legende@légende}, voir \textsc{Rangaswamy} (*1990 [1958]: 19-24), \textsc{Ve\d l\d laiv\=ara\d na\b n} (*1994 [1962 et 1969]: 9-15), \textsc{Gros} (2001: 23-24) et enfin, \textsc{Prentiss} qui analyse la cr\'eation du canon\index{gnl}{canon} (2001a) et traduit le texte (2001b).}: un roi\index{gnl}{roi} nomm\'e R\=acar\=aca Apaiyakulac\=ekara\b n (sk. R\=ajar\=aja Abhayakula\'sekhara\index{gnl}{Rajaraja A@R\=ajar\=aja Abhayakula\'sekhara}), extr\^emement \'emu par les chants\index{gnl}{chant}\index{gnl}{chant} des \textit{m\=uvar}\index{gnl}{muvar@\textit{m\=uvar}} r\'ecit\'es au temple\index{gnl}{temple} de Ty\=age\'sa \`a \=Ar\=ur\index{gnl}{Ar\=ur@\=Ar\=ur}, souhaite classer les hymne\index{gnl}{hymne}s des poète\index{gnl}{poete@poète}s. Mais en vain. Il demeure pein\'e (st. 1). Alors appara\^it un jeune brahmane\index{gnl}{brahmane} shiva\"ite\index{gnl}{shiva\"ite} de N\=araiy\=ur\index{gnl}{Naraiyur@N\=araiy\=ur}, n\'e dans une famille \textit{\=adi\'saiva}, officiant \og rempla\c cant\fg\ du temple\index{gnl}{temple} de Poll\=appi\d l\d laiy\=ar\index{gnl}{Pollappillaiyar@Poll\=appi\d l\d laiy\=ar} (Ga\d ne\'sa\index{gnl}{Ganesa@Ga\d ne\'sa}), qui, par d\'evotion\index{gnl}{devotion@dévotion}, nourrit v\'eritablement cette divinit\'e (st. 2-4) et apprend d'elle les textes sacr\'es. Il est nomm\'e Nampi \=A\d n\d t\=ar Nampi\index{gnl}{Nampi \=A\d n\d t\=ar Nampi} (st. 5). Le roi\index{gnl}{roi} a connaissance\index{gnl}{connaissance} de ce miracle\index{gnl}{miracle} et d\'ecide de le v\'erifier en apportant une quantit\'e gargantuesque d'offrandes destin\'ees \`a Ga\d ne\'sa\index{gnl}{Ganesa@Ga\d ne\'sa} pour mettre \`a l'épreuve\index{gnl}{epreuve@épreuve} le brahmane\index{gnl}{brahmane}. Poll\=appi\d l\d laiy\=ar\index{gnl}{Pollappillaiyar@Poll\=appi\d l\d laiy\=ar} consomme tous les mets \`a la demande de Nampi\index{gnl}{Nampi \=A\d n\d t\=ar Nampi} (st. 6-8). Le roi\index{gnl}{roi} heureux et convaincu des qualit\'es de Nampi\index{gnl}{Nampi \=A\d n\d t\=ar Nampi} lui confie alors la t\^ache de r\'eunir les hymne\index{gnl}{hymne}s des \textit{m\=uvar}\index{gnl}{muvar@\textit{m\=uvar}} (st. 9). Nampi\index{gnl}{Nampi \=A\d n\d t\=ar Nampi} accepte puis, inform\'e par Poll\=appi\d l\d laiy\=ar\index{gnl}{Pollappillaiyar@Poll\=appi\d l\d laiy\=ar} que les mains de \'Siva\index{gnl}{Siva@\'Siva} dansant lui-même indiquent l'emplacement des hymne\index{gnl}{hymne}s dans une pièce\index{gnl}{piece@pièce} fermée du temple de Citamparam\index{gnl}{Citamparam}\footnote{\textit{Tirumu\b raika\d n\d tapur\=a\d nam}\index{gnl}{Tirumuraikantapuranam@\textit{Tirumu\b raika\d n\d tapur\=a\d nam}} st. 12: \textit{"va\d n\d tami\b lka \d liruntavi\d ta ma\b n\b ru\d l\=a\d tu\.n, k\=urntaviru\d t ka\d n\d tarpu\b rak ka\d taiyi\b n p\=a\.nkark k\=olamalark \textbf{kaika\d la\d taiy\=a\d la m\=aka}c, c\=arnta\b na"}, \og Behind [the image of] the Lord of the Dark Throat who dances in the hall is the place where the Tamil manuscripts are kept; his beautiful lotus-like hands mark the spot\fg\ (traduction de \textsc{Prentiss} 2001b).}, il se rend là avec le roi\index{gnl}{roi} et sa cour (st. 10-17). Mais, les brahmane\index{gnl}{brahmane}s, les dévot\index{gnl}{devot(e)@dévot(e)}s et les gardiens du temple\index{gnl}{temple} de Citamparam\index{gnl}{Citamparam}, demandent que les \textit{m\=uvar}\index{gnl}{muvar@\textit{m\=uvar}} soient pr\'esents pour que la pièce\index{gnl}{piece@pièce} s'ouvre (st. 19: "\textit{tami\b lvaitta m\=uvarvant\=a l\textbf{a\b raiti\b rakkum}}"). Le roi\index{gnl}{roi} organise une procession\index{gnl}{procession} des image\index{gnl}{image}s des trois poète\index{gnl}{poete@poète}s et \`a l'issue de laquelle, la porte\index{gnl}{porte} s'ouvre offrant au grand bonheur de tous des hymne\index{gnl}{hymne}s sur \^oles en partie, cependant, rong\'ees par les termites (st. 18-20). La vision de l'\'etat d\'et\'erior\'e des feuilles de palmier\index{gnl}{palmier} plonge le roi\index{gnl}{roi} dans une profonde affliction, \`a laquelle la voix de \'Siva\index{gnl}{Siva@\'Siva} rem\'edie en lui signifiant qu'il est la cause des hymne\index{gnl}{hymne}s et de leur état (st. 21-22). Le roi\index{gnl}{roi} lui-m\^eme compile les sept premiers \textit{Tirumu\b rai}\index{gnl}{Tirumurai@\textit{Tirumu\b rai}}, comme ils \'etaient jadis, avec les hymne\index{gnl}{hymne}s des \textit{m\=uvar}\index{gnl}{muvar@\textit{m\=uvar}} (st. 23-24)\footnote{Soulignons que le roi\index{gnl}{roi} est le sujet dans ces deux strophes et qu'il appara\^it ainsi comme le compilateur des sept premiers livres.}. La seconde moiti\'e du \textit{Tirumu\b raika\d n\d tapur\=a\d nam}\index{gnl}{Tirumuraikantapuranam@\textit{Tirumu\b raika\d n\d tapur\=a\d nam}} d\'ecrit comment Nampi\index{gnl}{Nampi \=A\d n\d t\=ar Nampi} agen\c ca les autres \oe uvres dans le canon\index{gnl}{canon}. Il ajouta \`a la fin ses compositions dont la louange des soixante-trois dévot\index{gnl}{devot(e)@dévot(e)}s, en fonction des modes musicaux, \textit{pa\d n}, avec l'aide d'une sp\'ecialiste d\'esign\'ee par \'Siva\index{gnl}{Siva@\'Siva}\footnote{De nombreux chercheurs, dont \textsc{Rangaswamy} (*1990 [1958]: 23-24), \textsc{Zvelebil} (1975: 133) et \textsc{Gros} (2001: 24) voient dans cette spécialiste la descendante de Tirun\=\i laka\d n\d ta y\=a\b lpp\=a\d nar, joueur de \textit{y\=a\b l} qui accompagne Campantar\index{gnl}{Campantar} dans le \textit{Periyapur\=a\d nam}\index{gnl}{Periyapuranam@\textit{Periyapur\=a\d nam}}. Cependant, cette parent\'e n'est pas \'evoqu\'ee dans le \textit{Tirumu\b raika\d n\d tapur\=a\d nam}\index{gnl}{Tirumuraikantapuranam@\textit{Tirumu\b raika\d n\d tapur\=a\d nam}}.}.

Ce \textit{pur\=a\d nam} pose des probl\`emes liés à sa datation et à des passages interpolés. \textsc{Rangaswamy} (*1990 [1958]: 20) consid\`ere que les vingt-quatre premi\`eres strophes sont authentiques mais que le reste a été ajout\'e postérieurement, compte tenu du changement de m\`etre et de la pr\'esentation tr\`es abrupte des informations concernant la compilation\index{gnl}{compilation} des volumes \textsc{viii} \`a \textsc{xi}. Ajoutons que l'invocation\index{gnl}{invocation} ne mentionne que la compilation\index{gnl}{compilation} des \oe uvres des \textit{m\=uvar}\index{gnl}{muvar@\textit{m\=uvar}}, et que les vingt-quatre premiers quatrains se r\'ef\`erent seulement aux trois poète\index{gnl}{poete@poète}s (st. 1, 9-11, 13-16 et 19). \textsc{Zvelebil} (1995: 679) adh\`ere \'egalement au raisonnement de \textsc{Rangaswamy} et s'interroge même sur son auteur. Selon la tradition\index{gnl}{tradition} il s'agit d'Um\=apati\index{gnl}{Umapati@Um\=apati} Civ\=ac\=ariyar, un des quatre \textit{cant\=a\b nakuruvar}, \og ma\^itres de la lign\'ee\fg, shiva\"ite\index{gnl}{shiva\"ite}, aux c\^ot\'es de Meyka\d n\d tar\index{gnl}{Meyka\d n\d tar}, d'Aru\d nanti\index{gnl}{Arunanti@Aru\d nanti} et de Ma\b rai\~n\=a\b nacampantar\index{gnl}{Marainana@Ma\b rai\~n\=a\b nacampantar}. Issu d'une famille de \textit{d\=\i k\d sita} de Citamparam\index{gnl}{Citamparam}, il aurait v\'ecu au \textsc{xiv}\up{e} si\`ecle et aurait eu pour ma\^itre Ma\b rai\~n\=a\b nacampantar\index{gnl}{Marainana@Ma\b rai\~n\=a\b nacampantar} (\textsc{Dagens} 1979: 14). De nombreux textes philosophico-religieux en sanskrit et en tamoul lui sont attribu\'es (\textsc{Janaki} 1996 et \textsc{Smith} 1998) dont le \textit{C\=ekki\b l\=arpur\=a\d nam}\index{gnl}{Cekkilarpuranam@\textit{C\=ekki\b l\=arpur\=a\d nam}} et la version tamoule du \textit{talapur\=a\d nam}\index{gnl}{Purana@\textit{Pur\=a\d na}!\textit{talapur\=a\d nam}} de Citamparam\index{gnl}{Citamparam}, le \textit{K\=oyi\b rpur\=a\d nam}\index{gnl}{Koyil@\textit{K\=oyi\b rpur\=a\d nam}}. Le temple\index{gnl}{temple} et le \'Siva\index{gnl}{Siva@\'Siva} dansant de Citamparam\index{gnl}{Citamparam} sont grandement c\'el\'ebr\'es dans ces \oe uvres. Cependant, il est fort probable que plusieurs philosophes shiva\"ite\index{gnl}{shiva\"ite}s aient port\'e le nom d'Um\=apati\index{gnl}{Umapati@Um\=apati} Civ\=ac\=ariyar \`a l'\'epoque m\'edi\'evale\footnote{Hypoth\`ese soutenue aussi par \textsc{Cox} (2006a: 87, n. 73).}. \textsc{Ir\=acam\=a\d nikka\b n\=ar} (*1996 [1968]: 77-81) soutient par ailleurs de fa\c con convaincante qu'il existe des discordances narratives dans les r\'ecits des dévot\index{gnl}{devot(e)@dévot(e)}s pr\'esent\'es dans trois textes tamouls attribu\'es \`a ce brahmane\index{gnl}{brahmane} de Citamparam\index{gnl}{Citamparam}, \`a savoir le \textit{Tirutto\d n\d tarpur\=a\d nac\=aram}, le \textit{Tirumu\b raika\d n\d tapur\=a\d nam} et le \textit{C\=ekki\b l\=arpur\=a\d nam}\index{gnl}{Cekkilarpuranam@\textit{C\=ekki\b l\=arpur\=a\d nam}} (voir 4.3.2) qui est aussi appel\'e \textit{Tirutto\d n\d tarpur\=a\d navaral\=a\b ru}. \textsc{Colas-Chauhan} (2002: 305-306, n.~3) place le \textit{Pau\d skarabh\=a\d sya}\index{gnl}{Pauskarabhasya@\textit{Pau\d skarabh\=a\d sya}}, un texte sanskrit qui est attribué à Um\=apati\index{gnl}{Umapati@Um\=apati}, au \textsc{xvi}\up{e} si\`ecle parce que l'auteur de ce commentaire aurait eu connaissance\index{gnl}{connaissance} de quelques trait\'es du \textsc{xv}-\textsc{xvi}\up{e} si\`ecle. \textsc{Goodall} (2004: cxv-cxix) pense que le \textit{\'Sataratnasa\.ngraha}\index{gnl}{Sataratnasangraha@\textit{\'Sataratnasa\.ngraha}}, attribu\'e \`a Um\=apati\index{gnl}{Umapati@Um\=apati}, diff\`ere par ses sources et son contenu id\'eologique d'autres textes qu'il aurait \'ecrit, tels que le \textit{Pau\d skarabh\=a\d sya}\index{gnl}{Pauskarabhasya@\textit{Pau\d skarabh\=a\d sya}} et le \textit{Ca\.nka\b rpanir\=akara\d nam}\index{gnl}{Cankarpanirakaranam@\textit{Ca\.nka\b rpanir\=akara\d nam}}\footnote{Rappelons que c'est la datation de ce texte qui a conduit beaucoup de chercheurs \`a situer tous les Um\=apati\index{gnl}{Umapati@Um\=apati} au \textsc{xiv}\up{e} si\`ecle. En effet, il est mentionn\'e dans le \textit{p\=ayiram}\index{gnl}{payiram@\textit{p\=ayiram}} de ce texte qu'il date de 1313 (\textit{\'saka} 1235).}. L'attribution de cet ensemble d'\oe uvres, y compris le \textit{Tirumu\b raika\d n\d tapur\=a\d nam}\index{gnl}{Tirumuraikantapuranam@\textit{Tirumu\b raika\d n\d tapur\=a\d nam}} qui nous int\'eresse\footnote{\textsc{Gros} (2001: 24, n. 5) rappelle que le \textit{Tirumu\b raika\d n\d tapur\=a\d nam}\index{gnl}{Tirumuraikantapuranam@\textit{Tirumu\b raika\d n\d tapur\=a\d nam}} est pr\'esent\'e sans nom d'auteur dans des \'editions du \textit{Periyapur\=a\d nam}\index{gnl}{Periyapuranam@\textit{Periyapur\=a\d nam}} au \textsc{xix}\up{e} si\`ecle, et que dans celles de Cat\=acivappi\d l\d lai, le nom d'Um\=apati\index{gnl}{Umapati@Um\=apati} ne figure qu'\`a partir de la quatri\`eme \'edition. La premi\`ere \'edition de Cat\=acivappi\d l\d lai (1898) ne contient pas le \textit{Tirumu\b raika\d n\d tapur\=a\d nam}\index{gnl}{Tirumuraikantapuranam@\textit{Tirumu\b raika\d n\d tapur\=a\d nam}} qui appara\^it, sans mention d'auteur, dans la seconde (1912). La date de la quatri\`eme \'edition n'est pas pr\'ecis\'ee dans la note de \textsc{Gros} qui reprend les informations de \textsc{A. C.~\~N\=a\b nacampantar}, \textit{Periyapur\=a\d nam \=or \=ayvu}, Madras, 1994 [1987], p. 418 et suivantes.}, \`a un seul Um\=apati\index{gnl}{Umapati@Um\=apati} qui aurait vécu au \textsc{xiv}\up{e} si\`ecle semble donc tr\`es contestable.

Ainsi, c'est un texte fort controvers\'e qui d\'ecrit la d\'ecouverte miraculeuse des hymne\index{gnl}{hymne}s \`a Citamparam\index{gnl}{Citamparam} puis la compilation\index{gnl}{compilation} des onze premiers volumes du \textit{Tirumu\b rai}\index{gnl}{Tirumurai@\textit{Tirumu\b rai}}. En réalité, les donn\'ees historiques montrent qu'une intervention divine, une volont\'e royale ou m\^eme Citamparam\index{gnl}{Citamparam} ne sont pas n\'ecessaires pour trouver des hymne\index{gnl}{hymne}s, probablement d\'ej\`a constitu\'es en un corpus\index{gnl}{corpus} nomm\'e \textit{Tirumu\b rai}\index{gnl}{Tirumurai@\textit{Tirumu\b rai}}, qui furent enferm\'es, en proie aux insectes, dans une pièce\index{gnl}{piece@pièce} de temple\index{gnl}{temple}.

\subsection{\textit{Tirumu\b rai}: les donn\'ees historiques}

Les occurrences du terme \textit{tirumu\b rai}\index{gnl}{Tirumurai@\textit{Tirumu\b rai}} sont rares et tardives dans l'\'epigraphie. Dans l'\'etat actuel des recherches, \textit{tirumu\b rai}\index{gnl}{Tirumurai@\textit{Tirumu\b rai}} appara\^it, sous la dynastie \textit{c\=o\b la}\index{gnl}{cola@\textit{c\=o\b la}}, dans une inscription datant du r\`egne de Kulottu\.nga II\index{gnl}{Kulottu\.nga II} (1133-1150): CEC 26\footnote{Nous montrons dans le CEC que le r\'esum\'e de l'ARE de cette inscription \'evoquant des image\index{gnl}{image}s est erron\'e. Par cons\'equent, la pr\'esence du terme \textit{tirumu\b rai}\index{gnl}{Tirumurai@\textit{Tirumu\b rai}} y est rest\'ee inconnue pour de nombreux auteurs qui datent son \og apparition\fg\ \'epigraphique sous R\=ajar\=aja III\index{gnl}{Rajaraja III@R\=ajar\=aja III}, tel que \textsc{Swamy} (1972: 98).}; trois de R\=ajar\=aja III\index{gnl}{Rajaraja III@R\=ajar\=aja III} (1216-1256): ARE 1928-29 350, 1908 454 et SII 8 205\footnote{Cette inscription de Mu\b niy\=ur (P\=apan\=acam tk.) datant de la vingt-huiti\`eme ann\'ee du roi\index{gnl}{roi} a \'et\'e cit\'ee comme r\'ef\'erence dans de nombreuses \'etudes. Cependant, elle a \'et\'e publi\'ee avec des erreurs. La v\'erification de l'estampage nous a permis de constater avec certitude que le nom du temple\index{gnl}{temple} au sud du monast\`ere\index{gnl}{monastère} de Tirumu\b rai-t\=ev\=arac-celva\b n\index{gnl}{Tirumuraittevaraccelvan@Tirumu\b raitt\=ev\=araccelva\b n}, l.~1, n'est pas Tirutto\d nicuram (ou Tirutto\d n\d t\=\i \'svaram, \textsc{Rangaswamy} 1990 [1958]: 29) mais Tirutt\=o\d nipuram\index{gnl}{Tonipuram@T\=o\d nipuram!Tirutt\=o\d nipuram}, \textit{i.e.} C\=\i k\=a\b li\index{gnl}{Cikali@C\=\i k\=a\b li}, et que figure, l.~2, \textit{tirumu\b rait tirukk\=appu nikki} \og ayant ouvert le \textit{Tirumu\b rai}\index{gnl}{Tirumurai@\textit{Tirumu\b rai}}\fg\ au lieu de \textit{tirumu\b r\b ra tirukk\=appu nikki} qui ne fait pas sens. Sur l'expression \textit{tirukk\=appu nikki}, cf. CEC 26.}; une de R\=ajendra III\index{gnl}{Rajendra III@R\=ajendra III} (1246-1279): ARE 1918 10; ainsi que sous les P\=a\d n\d dya \og tardifs\fg\ (ARE 1907 92, 1908 414 et 1924 24).

Dans ces inscriptions le terme d\'esigne clairement des hymne\index{gnl}{hymne}s chant\'es. Les relev\'es de l'ARE 1928-29 350 et 1907 92 mentionnent des dons\index{gnl}{don} pour assurer la récitation\index{gnl}{recitation@récitation} du \textit{Tirumu\b rai}\index{gnl}{Tirumurai@\textit{Tirumu\b rai}}. ARE 1908 454\footnote{l. 3: \textit{tirumu\b rai otuv\=a\b rkku tiruvamutupa\d tikku u\d tal\=aka ivar ku\d tutta nilam\=ay}, \og terre\index{gnl}{terre} qu'il a donn\'ee comme capital pour l'offrande de nourriture au chanteur du \textit{Tirumu\b rai}\index{gnl}{Tirumurai@\textit{Tirumu\b rai}}\fg.} et ARE 1918 10\footnote{l. 2: \textit{tirumu\b rai e\b luntaru\d li irukkum tiruppa\d l\d li a\b rai nokkuv\=a\b rkkum tirupp\=a\d t\d tu otuv\=arkkum}, \og pour celui qui s'occupe de la pièce\index{gnl}{piece@pièce} \textit{tiruppa\d l\d li} o\`u se trouve install\'e le \textit{Tirumu\b rai}\index{gnl}{Tirumurai@\textit{Tirumu\b rai}} et pour le chanteur des hymne\index{gnl}{hymne}s sacr\'es\fg.} pr\'ecisent qu'il s'agit de dons\index{gnl}{don} de terre\index{gnl}{terre} pour nourrir le chanteur du \textit{Tirumu\b rai}\index{gnl}{Tirumurai@\textit{Tirumu\b rai}}. Une inscription de V\=\i \b limi\b lalai\index{gnl}{Vilimilalai@V\=\i \b limi\b lalai} (Na\b n\b nilam tk.\index{gnl}{Nannilam@Na\b n\b nilam tk.}), ARE 1908 414, que nous d\'etaillons plus bas, enregistre une donation pour faire des offrandes de nourriture au \textit{Tirumu\b rai}\index{gnl}{Tirumurai@\textit{Tirumu\b rai}}, sous forme manuscrite\index{gnl}{manuscrit}, qui avait \'et\'e install\'e, men\'e en procession\index{gnl}{procession}, chant\'e et honor\'e.

Certains textes épigraphiques nous informent par ailleurs que le \textit{Tirumu\b rai}\index{gnl}{Tirumurai@\textit{Tirumu\b rai}} \'etait conserv\'e dans une pièce\index{gnl}{piece@pièce}, ou un espace sp\'ecifique du temple\index{gnl}{temple} o\`u il \'etait chant\'e, appel\'ee souvent le \textit{tirukkaikk\=o\d t\d ti}\index{gnl}{tirukkaikkotti@\textit{tirukkaikk\=o\d t\d ti}}\footnote{Selon \textsc{Rangaswamy} (*1990 [1958]: 23) il s'agit probablement d'une forme tamoulis\'ee du sk. \textit{\'sr\=\i hastago\d s\d th\=\i}; ce terme d\'eriverait du fait que les hymne\index{gnl}{hymne}s \'etaient r\'ecit\'es par un groupe (\textit{go\d s\d th\=\i}) marquant le temps avec les mains (\textit{hasta}). L'hypoth\`ese de \textsc{Swamy} (1972: 108) qui y voit un comit\'e travaillant pour le temple\index{gnl}{temple} plut\^ot qu'un espace d\'efini consacr\'e \`a la récitation\index{gnl}{recitation@récitation} n'est absolument pas convaincante compte tenu des inscriptions que nous pr\'esentons ici. Pr\'ecisons toutefois que le terme \textit{\'sr\=\i hastago\d s\d th\=\i} ne se rencontre pas dans les textes sanskrits (Information de Dominic \textsc{Goodall}). Signalons enfin un exemple que pr\'esente \textsc{Hardy} (*2001 [1983]: 643) pour souligner le substrat tamoul de la langue du \textit{Bh\=agavatapur\=a\d na}\index{gnl}{Bhagavata@\textit{Bh\=agavatapur\=a\d na}}. D'apr\`es l'auteur, le nom K\=amako\d s\d n\=\i\ trouv\'e dans ce texte est une mauvaise re-sanskritisation du nom tamoul du temple\index{gnl}{temple} K\=amak\=o\d t\d ti \`a K\=a\~ncipuram\index{gnl}{Kancipuram@K\=a\~ncipuram} car le terme tam. \textit{k\=o\d t\d ti} est un d\'eriv\'e du sk. \textit{ko\d s\d tha} signifiant \og grenier, trésorerie\index{gnl}{tresorerie@trésorerie}\fg\ et non de \textit{ko\d s\d n\=\i} qui ne fait pas sens. \textit{K\=o\d t\d tam}, \og temple\index{gnl}{temple}\fg, est un autre d\'eriv\'e tamoul de ce terme. Ainsi, nous sugg\'erons que le terme \textit{tirukkaikk\=o\d t\d ti}\index{gnl}{tirukkaikkotti@\textit{tirukkaikk\=o\d t\d ti}} n'est pas une forme tamoulis\'ee du sk. \textit{\'sr\=\i hastago\d s\d th\=\i} mais que ce dernier est une mauvaise sanskritisation du mot tamoul qui renvoit certainement \`a un espace d\'efini du temple\index{gnl}{temple} (\textit{k\=o\d t\d ti}, \textit{k\=o\d t\d tam} du sk. \textit{ko\d s\d tha}) associ\'e, nous ne savons pas encore pourquoi, aux mains (tam. \textit{kai}).} (ARE 1908 203, 414, 454, 1928-29 350 et CEC 26). Cette partie du temple\index{gnl}{temple} semble avoir \'et\'e néglig\'ee dans quelques sites, au p\'eril des \^oles contenant le \textit{Tirumu\b rai}\index{gnl}{Tirumurai@\textit{Tirumu\b rai}}. \`A notre avis, deux inscriptions, non publi\'ees et tr\`es mal connues de la litt\'erature secondaire, font \'etat de cette situation. Elles proviennent de C\=\i k\=a\b li\index{gnl}{Cikali@C\=\i k\=a\b li} et de V\=\i \b limi\b lalai\index{gnl}{Vilimilalai@V\=\i \b limi\b lalai}, deux sites du delta de la K\=av\=eri\index{gnl}{Kaveri@K\=av\=eri} li\'es \`a la légende\index{gnl}{legende@légende} de Campantar\index{gnl}{Campantar}, et datent, respectivement, du \textsc{xii}\up{e} et du \textsc{xiii}\up{e} si\`ecle\footnote{Il nous semble abusif de douter de la v\'eracit\'e de ces textes \'epigraphiques sachant que nous n'avons pas affaire \`a un éloge\index{gnl}{eloge@éloge} royal, \`a une inscription contenant un éloge\index{gnl}{eloge@éloge}, \`a une copie d'inscription ant\'erieure ou \`a un r\'eemploi (types d'inscriptions qui sont susecptibles d'\^etre des \og faux\fg). De plus, compte tenu de la dimension \og locale\index{gnl}{local}\fg\ --- les donateurs sont le temple\index{gnl}{temple} et l'assemblée\index{gnl}{assemblée} villageoise --- de leurs donn\'ees nous croyons en l'authenticit\'e de leur t\'emoignage. Nous avons \'edit\'e la premi\`ere dans le CEC (26) et avons consult\'e la transcription de la seconde \`a Mysore en d\'ecembre 2006.}. Ainsi, CEC 26, qui date de la quatri\`eme ann\'ee de r\`egne de Kulottu\.nga II\index{gnl}{Kulottu\.nga II}, enregistre un don\index{gnl}{don} de l'assemblée\index{gnl}{assemblée} villageoise de C\=\i k\=a\b li\index{gnl}{Cikali@C\=\i k\=a\b li} pour que l'expert en tamoul de la chapelle de Campantar\index{gnl}{Campantar} rouvre le \textit{tirukkaikk\=o\d t\d ti}\index{gnl}{tirukkaikkotti@\textit{tirukkaikk\=o\d t\d ti}}, r\'epare les manuscrits\index{gnl}{manuscrit} d\'et\'erior\'es et en r\'einstalle de nouveaux.
Une inscription de V\=\i \b limi\b lalai\index{gnl}{Vilimilalai@V\=\i \b limi\b lalai} (ARE 1908 414)\footnote{L'\'epigraphe date de la neuvi\`eme ann\'ee de r\`egne de Ca\d taiyapa\b nmar Tirupuva\b naccakkaravattika\d l Cu\b ntarap\=a\d n\d tiya que \textsc{Mahalingam} (1992: 485) sugg\`ere d'identifier comme Ja\d t\=avarman Sundara P\=a\d n\d dya II\index{gnl}{Jatavarman S@Ja\d t\=avarman Sundara P\=a\d n\d dya II}, et date ainsi le texte de 1285.} est encore plus explicite sur les conditions de renaissance d'un \textit{tirukkaikk\=o\d t\d ti}\index{gnl}{tirukkaikkotti@\textit{tirukkaikk\=o\d t\d ti}} et, ce faisant, du \textit{Tirumu\b rai}\index{gnl}{Tirumurai@\textit{Tirumu\b rai}} qu'il contenait: \og le \textit{tirukkaikk\=o\d t\d ti}\index{gnl}{tirukkaikkotti@\textit{tirukkaikk\=o\d t\d ti}} du seigneur [de ce temple\index{gnl}{temple}] a été laiss\'e longtemps en ruine, sans que le \textit{Tirumu\b rai}\index{gnl}{Tirumurai@\textit{Tirumu\b rai}}, install\'e, puisse \'ecouter les chants\index{gnl}{chant}\index{gnl}{chant} sacr\'es\fg\ (l.~3-4)\footnote{[\dots] \textit{i\b n\b n\=ayan\=ar tirukkaikko\d t\d ti citilam\=ay ne\d tun\=a\d l pa\d tat tirumu\b r[ai]y\=ar e\b luntaru\d li iruntu tirupp\=a\d t\d tuk ke\d t\d taru\d lap pe\b r\=amal potukaiyil}[\dots]}. Le terme \textit{tirumu\b raiy\=ar}, nom appellatif au pluriel ou au singulier honorifique, qui pourrait renvoyer \`a des image\index{gnl}{image}s, souligne ici la d\'eification du texte. En effet, l.~20, la s\'equence \textit{tiruve\d tuka\d lum ta\b nittupp\=attu nokki}, \og ayant regard\'e s\'epar\'ement les \^oles\fg, qui d\'ecrit le \textit{tirumu\b raiy\=ar} confirme que ce dernier d\'esigne le texte sur feuilles de palmier\index{gnl}{palmier} et non les image\index{gnl}{image}s des auteurs de ce texte. Puis, un certain Tevar N\=araci\.nkatevar\index{gnl}{Tevar Naracinkatevar@Tevar N\=araci\.nkatevar},
%\footnote{Sur l'identification de cet individu, cf. p.XXXXX.}
d\'esireux d'entendre \`a nouveau les chants\index{gnl}{chant}\index{gnl}{chant} sacr\'es dans ce temple\index{gnl}{temple}, \og construisit un \textit{tirukkaikk\=o\d t\d ti}\index{gnl}{tirukkaikkotti@\textit{tirukkaikk\=o\d t\d ti}}, ainsi qu'un si\`ege de lion\fg\ et r\'eintroduisit procession\index{gnl}{procession}, chant et culte\index{gnl}{culte} pour ce texte (l.~4-6)\footnote{[\dots] \textit{pi\b npu tevar n\=araci\.nkatevar tiru\~n\=a\b naca\dots\ kka\d tava k\=a\b lit tirume\d ni mu\b npil\=a\d n\d tu u\d taiy\=ar tirukkaikko\d t\d tiyil tirupp\=a\d t\d tuk ke\d t\d taru\d lum pa\d tiye tirupp\=a\d t\d tuk ke\d t\d taru\d la ve\d nume\b n\b ru muta\dots\ tirukkaikko\d t\d tiyum amaittu ci\.nk\=aca\b namum amaittu oru[p*]pa\d ta tirumu\b raiy\=arum e\b luntaru\d lap pa\d n\d ni e\b riyaru\d lavum pa\d n\d ni tirupp\=a\d t\d tuk ke\d t\d taru\d lip p\=ucai ko\d n\d tu potukaiyil} [\dots]}. Enfin, les employ\'es du temple\index{gnl}{temple} d\'ecident \`a leur tour de destiner une terre\index{gnl}{terre} \`a l'offrande de nourriture pour ce \textit{Tirumu\b rai}\index{gnl}{Tirumurai@\textit{Tirumu\b rai}} et celui qui l'entretient (l.~6-7)\footnote{[\dots] \textit{tirumu\b raiy\=ar amutu caiytaru\d la ve\b nume\b n\b ru i\dots tikku amutupa\d tikkum tirupparikar\=am\=ay ni\b n\b ru tirumu\b raiy\=arai nokku[cey*]ki\b ra tirume\b nikku} [\dots]}. Des guirlandes (\textit{tiruppa\d l\d litt\=amam} l.~19) et des v\^etements (\textit{tirupacica\d t\d tam} l.~20) \'etaient pr\'evus pour orner ce texte.
%Voici cinq lignes pertinentes de cette inscription qui en compte trente. Le texte pr\'esent\'e est fond\'e sur l'unique examen de la transcription de l'ASI et les conventions sont celles de notre CEC.}:
%\begin{enumerate}
%\item \textbf{svasti} \textbf{\'sr\=\i}\ ko\b rcca\d taiyapa\b nmar tirupuva\b naccakkaravattika\d l \textbf{\'sr\=\i}\ cu\b ntara-p\=a[\d n\d ti]yateva\b rku y\=a\d n\d tu [9]vatu tul\=an\=aya\b r\b ru aparapa\textbf{k\d sa}ttu \textbf{saptami}-\textbf{yu}m [n]\=aya\b r\b rukki\b la
%\item maiyum pe\b r\b ra p\=ucattu n\=a\d l uyakko\d n\d t\=ar va\d lan\=a\d t\d tu [ve\d n]\d n\=a\d t\d tu u\d taiy\=ar tiruvi\b limi\b lalaiyu\d taiy\=ar koyil \=atica\d n\d taicuratevar tiruvaru\d l\=al
%\item[3.] tevarka\b nmi koyil ka\d nakkom i\b n\b n\=ayan\=ar tirukkaikko\d t\d ti citilam\=ay ne\d tun\=a\d l pa\d tat tirumu\b r[ai]y\=ar e\b luntaru\d li iruntu tirupp\=a\d t\d tuk ke\d t\d taru\d lap pe\b r\=ama
%\item[4.] l potukaiyil pi\b npu tevar n\=araci\.nkatevar tiru\~n\=a\b naca\dots kka\d tava k\=a\b littirume\d ni mu\b npil\=a\d n\d tu u\d taiy\=ar tirukkaikko\d t\d tiyil tirupp\=a\d t\d tuk ke\d t\d taru\d lumpa
%\item[5.] \d tiye tirupp\=a\d t\d tuk ke\d t\d taru\d la ve\d nume\b n\b ru muta\dots\ tirukkaikko\d t\d tiyum amaittu ci\.nk\=aca\b namum amaittu oru[p*]pa\d ta tirumu\b raiy\=arum e\b luntaru\d lap pa\d n\d ni e\b riyaru\d lavum
%\item[6.] pa\d n\d ni tirupp\=a\d t\d tuk ke\d t\d taru\d lip p\=ucai ko\d n\d tu potukaiyil tirumu\b raiy\=ar amutu caiytaru\d la ve\b nume\b n\b ru i\dots tikku amutupa\d tikkum tirupparikar\=am\=ay ni\b n\b ru tirumu\b raiy\=arai nokku[cey*]ki\b ra ti
%\item[7.] rume\b nikku amutukkumu\d tal\=alat tirukkaikko\d t\d tip pu\b ram\=aka veli nilam ko\d n\d tu i[\d tu]vat\=aka \dots\ tevat\=a\b nam \textbf{ja}ya\.nko\d n\d taco\b lanall\=uril ka\d niy\=a\d lar vi\b rki\b ra vilaippa\d ti veli nilat
%\end{enumerate}
%(8-17)
%\begin{enumerate}
%\item[18.] {\d n\d n\=a\b r\b ru mu\b n\b r\=a\~ncatirattu nilam irum\=avaraiyum \=aka nilam o\b n\b rai\dots\ koyil tirumu\b raiy\=ar amutu ceytaru\d la n\=a\d l o\b n\b rukku mu\b n\b n\=a\b li a[riciyum ka]\b riyamutum vi\~nca}
%\item[19.] {\b namum palakaiyile vi\d t\d tu tirumu\b raiy\=ar amutu ceytaru\d lavum amutu ceytaru\d li\b na pirac\=atam tirukkaikko\d t\d tikku \dots\ ceyki\b ra tirume\b ni tirumu\b raiy\=a\b rku tiruppa\d l\d litt\=amam ka\b littuc c\=attiyum \dots\ tirumu}
%\item[20.] {\b raiy\=arait tirukk\=appu nikki tiruve\d tuka\d lum ta\b nittupp\=attu nokki e\b lunta[ru\d la]ve\d n\d tum po\b lutu e\b luntaru\d lavum pa\d n\d na \dots kkavum tiruparica\d t\d tam tiru mu\b n viri tirumu\d t\d tukka\d lu\d l\d livaiyum \dots\ ippa\d ti}
%\item[21.] {ceyki\b ra tirume\b nikku\d tal\=aka ippirac\=atam pe\b rak ka\d tav\=ar\=akavum ippa\d ti cantir\=atitavaraiyum vi\d tak ka\d tavom\=akavum \dots t t\=a\b lvu\d ta\b n\=a\d l\=a\d n\d tatilum tirumu\b raiy\=ar amutu ceytaru\d l\=aki\b ra pa\d tiyil ku}
%\item[22.] {\b raiy\=amal vi\d tak ka\d tavom\=akavum tirumu\b raiy\=ar amutu ceytaru\dots\ ceyki\b ra tirume\b niye\b ravum koyil nimantam tirukkaikko\d t\d ti}
%\item[23.] {yile otum tirume\b ni\dots m ippa\d ti cammatittuk kalve\d t\d tik ku\d tuttom \=atica\d n\d te\textbf{\'sva}ratevar [ti]ruvaru\d lal tevarka\b nmi ko[yil] ka\d nakkarom ikk\=acu e\dots ttu mu\b n\b n\=a}
%\end{enumerate}
%(24-30)


Ainsi, nous sugg\'erons que le terme \textit{Tirumu\b rai}\index{gnl}{Tirumurai@\textit{Tirumu\b rai}} renvoyant \`a des textes chant\'es dans des inscriptions des \textsc{xii}\up{e} et \textsc{xiii}\up{e} si\`ecles pourrait se r\'ef\'erer \`a une compilation\index{gnl}{compilation}, ant\'erieure \`a 1136 (CEC 26), d'hymne\index{gnl}{hymne}s g\'en\'eralement appel\'es par le terme \textit{tiruppatiyam}\index{gnl}{tiruppatiyam@\textit{tiruppatiyam}} et dont le contenu exact reste à d\'efinir\footnote{Nous pensons que ces poème\index{gnl}{poeme@poème}s sont, en partie, ceux attribu\'es aux \textit{m\=uvar}\index{gnl}{muvar@\textit{m\=uvar}} et \`a d'autres poète\index{gnl}{poete@poète}s connus du \textit{Tirumu\b rai}\index{gnl}{Tirumurai@\textit{Tirumu\b rai}} actuel. Par exemple, des noms propres bas\'es sur ce terme et sur Campantar\index{gnl}{Campantar} nous permettent de supposer que les \oe uvres de ce dernier appartenaient \`a ce corpus\index{gnl}{corpus}: le monast\`ere\index{gnl}{monastère} Tirumu\b rait-t\=ev\=arac-celva\b n\index{gnl}{Tirumuraittevaraccelvan@Tirumu\b raitt\=ev\=araccelva\b n} est tr\`es probablement un monast\`ere\index{gnl}{monastère} de Campantar\index{gnl}{Campantar} dans SII 8 205 (voir 1.2) et la donatrice d'une image\index{gnl}{image} de Campantar\index{gnl}{Campantar} s'appelle \og Tirumu\b rai-N\=achchi alias Tiruj\~n\=anasambanda-na\.ngai\fg\ dans ARE 1924 24.}. Ensuite, pour des raisons qui nous sont encore obscures, le \textit{Tirumu\b rai}\index{gnl}{Tirumurai@\textit{Tirumu\b rai}} et la pièce\index{gnl}{piece@pièce} qui les contenait furent néglig\'es dans quelques temples\index{gnl}{temple} comme s'ils avaient connu une phase impopulaire. Ces faits sont d'autant plus surprenants qu'ils eurent lieu dans deux sites tr\`es importants de la légende\index{gnl}{legende@légende} de Campantar\index{gnl}{Campantar}. Et enfin, une autorit\'e locale\index{gnl}{local} (assemblée\index{gnl}{assemblée} ou temple\index{gnl}{temple}) intervient et rouvre les portes du \textit{tirukkaikk\=o\d t\d ti}\index{gnl}{tirukkaikkotti@\textit{tirukkaikk\=o\d t\d ti}} pour entretenir, honorer et ranimer ces hymne\index{gnl}{hymne}s. Il appara\^it donc, selon nous, que la trame du \textit{Tirumu\b raika\d n\d tapur\=a\d nam}\index{gnl}{Tirumuraikantapuranam@\textit{Tirumu\b raika\d n\d tapur\=a\d nam}} ne serait en fait que la reprise mythifi\'ee d'\'el\'ements historiques attest\'es dans l'\'epigraphie et mis en \oe uvre par des autorit\'es locales. Le g\'enie de l'auteur de cette légende\index{gnl}{legende@légende} est d'avoir vu en Nampi \=A\d n\d t\=ar Nampi\index{gnl}{Nampi \=A\d n\d t\=ar Nampi} le compilateur, figure semblable \`a l'expert en tamoul de CEC 26\footnote{Il est toutefois pr\'ecis\'e dans le \textit{Tirumu\b raika\d n\d tapur\=a\d nam}\index{gnl}{Tirumuraikantapuranam@\textit{Tirumu\b raika\d n\d tapur\=a\d nam}} st. 23 que les hymne\index{gnl}{hymne}s des \textit{m\=uvar}\index{gnl}{muvar@\textit{m\=uvar}} connaissaient une ordonnance\index{gnl}{ordonnance} harmonieuse avant leur perte et que le roi\index{gnl}{roi} d\'esirait la r\'etablir: \textit{pa\d n\d t\=aran ti\b rantu vi\d t\d t\=a\b n; parivu k\=urnt\=a\b n; intavakaip peru\.nka\d liko\d n ma\b n\b na\b n \b r\=a\b nu m\textbf{e\b li\b nmu\b raiyai mu\b np\=ola vakukka} ve\d n\d ni}, \og Then the king, filled with joy and love, opened the treasury, intending to put the beautiful collection (\textit{mu\b rai}) in the order it was previously\fg\ (traduction de \textsc{Prentiss} 2001b).}, et surtout, d'avoir, en quelque sorte, conf\'er\'e au corpus\index{gnl}{corpus} le statut de texte r\'ev\'el\'e: Ga\d ne\'sa\index{gnl}{Ganesa@Ga\d ne\'sa} localise les hymne\index{gnl}{hymne}s sacr\'es perdus dans la demeure du \'Siva\index{gnl}{Siva@\'Siva} dansant qui demande \`a les mettre en musique\index{gnl}{musique}.

Enfin, pour clore cette sous-partie, nous signalons que le terme \textit{tirumu\b rai}\index{gnl}{Tirumurai@\textit{Tirumu\b rai}} appara\^it dans la litt\'erature avec le \textit{Periyapur\=a\d nam}\index{gnl}{Periyapuranam@\textit{Periyapur\=a\d nam}} o\`u il semble attest\'e deux fois. Sa mention dans l'hagiographie\index{gnl}{hagiographie} de Campantar\index{gnl}{Campantar} (st. 2680) ne conduit pas, selon le commentaire de Ci. K\=e. \textsc{Cuppirama\d niya Mutaliy\=ar}, \`a l'id\'ee d'un corpus\index{gnl}{corpus} compos\'e des hymne\index{gnl}{hymne}s des \textit{m\=uvar}\index{gnl}{muvar@\textit{m\=uvar}} et d'autres, mais d\'esignerait plut\^ot, dans ce contexte, un \'ecrit religieux quelconque qui guide vers la d\'elivrance. Dans la légende\index{gnl}{legende@légende} du dévot\index{gnl}{devot(e)@dévot(e)} Ka\d nan\=atar\index{gnl}{Kananatar@Ka\d nan\=atar}, son occurrence (st. 3925) pourrait se r\'ef\'erer au corpus\index{gnl}{corpus} du \textit{Tirumu\b rai}\index{gnl}{Tirumurai@\textit{Tirumu\b rai}}. En effet, ce quatrain et le suivant énumèrent diff\'erents service\index{gnl}{service}s (\textit{to\d n\d tu}) envers \'Siva\index{gnl}{Siva@\'Siva} que Ka\d nan\=atar\index{gnl}{Kananatar@Ka\d nan\=atar} enseigne aux dévot\index{gnl}{devot(e)@dévot(e)}s qui viennent \`a lui. \'Ecrire, ou plut\^ot transcrire, et lire le \textit{Tirumu\b rai}\index{gnl}{Tirumurai@\textit{Tirumu\b rai}} sont consid\'er\'es là comme des actes m\'eritoires\footnote{ \textit{ellai yilvi\d lak kerippavar, \textbf{tirumu\b rai ye\b lutuv\=or v\=acipp\=or}} (3925d), \og ceux qui allument des lampe\index{gnl}{lampe}s sans fin, ceux qui transcrivent et lisent le \textit{tirumu\b rai}\index{gnl}{Tirumurai@\textit{Tirumu\b rai}}\fg.}. Les commentaires de Ci. K\=e. \textsc{Cuppirama\d niya Mutaliy\=ar} et de \textsc{Gopal Iyer} (1991: 10-11) s'accordent sur cette lecture. Toutefois, à notre avis, les occurrences sont insuffisantes dans le \textit{Periyapur\=a\d nam} pour d\'efinir précisément le sens du terme\footnote{Selon \textsc{Gopal Iyer} (1991: 5-8) dans la tradition\index{gnl}{tradition} des manuscrits\index{gnl}{manuscrit} du \textit{T\=ev\=aram}\index{gnl}{Tevaram@\textit{T\=ev\=aram}}, le terme \textit{tirumu\b rai}\index{gnl}{Tirumurai@\textit{Tirumu\b rai}} prend plusieurs sens. Il peut d\'esigner une strophe, un ensemble de poème\index{gnl}{poeme@poème}s d'un auteur (un des sept premiers \textit{Tirumu\b rai}\index{gnl}{Tirumurai@\textit{Tirumu\b rai}}) ou l'int\'egralit\'e des \oe uvres des \textit{m\=uvar}\index{gnl}{muvar@\textit{m\=uvar}}. \textit{A\d ta\.nka\b nmu\b rai}, \og Canon\index{gnl}{canon} entier\fg, est une autre appellation du \textit{T\=ev\=aram}\index{gnl}{Tevaram@\textit{T\=ev\=aram}}.}.


\textsc{Aravamuthan} (1934-35) et \textsc{Zvelebil} (1975: 130-151) offrent une introduction aux textes du \textit{Tirumu\b rai}\index{gnl}{Tirumurai@\textit{Tirumu\b rai}}. Notre choix des \oe uvres du \textit{Tirumu\b rai}\index{gnl}{Tirumurai@\textit{Tirumu\b rai}} dans ce chapitre est dict\'e par leur r\'ef\'erence, allusive ou d\'etaill\'ee, au site de C\=\i k\=a\b li\index{gnl}{Cikali@C\=\i k\=a\b li} et \`a son poète\index{gnl}{poete@poète}. Leur pr\'esentation se veut chronologique, selon la tradition\index{gnl}{tradition}, et g\'en\'erale: le \textit{Tirukka\b lumalamumma\d nikk\=ovai}\index{gnl}{Tirukkalumalamummanikkovai@\textit{Tirukka\b lumalamumma\d nikk\=ovai}} attribu\'e \`a Pa\d t\d ti\b nattuppi\d l\d lai\index{gnl}{Pattinattu Pillai@Pa\d t\d ti\b nattuppi\d l\d lai} (\textit{Tirumu\b rai}\index{gnl}{Tirumurai@\textit{Tirumu\b rai}} \textsc{xi}), six poème\index{gnl}{poeme@poème}s sur Campantar\index{gnl}{Campantar} qui auraient \'et\'e compos\'es par Nampi \=A\d n\d t\=ar Nampi\index{gnl}{Nampi \=A\d n\d t\=ar Nampi} (\textit{Tirumu\b rai}\index{gnl}{Tirumurai@\textit{Tirumu\b rai}} \textsc{xi}) et le \textit{Periyapur\=a\d nam}\index{gnl}{Periyapuranam@\textit{Periyapur\=a\d nam}} compos\'e par C\=ekki\b l\=ar\index{gnl}{Cekkilar@C\=ekki\b l\=ar} (\textit{Tirumu\b rai}\index{gnl}{Tirumurai@\textit{Tirumu\b rai}} \textsc{xii}).


\section{Le \textit{Tirumu\b rai} \textsc{xi}}

\subsection{Le \textit{Tirukka\b lumalamumma\d nikk\=ovai}}

Dans le \textit{Tirumu\b rai}\index{gnl}{Tirumurai@\textit{Tirumu\b rai}} \textsc{xi}, le \textit{Tirukka\b lumalamumma\d nikk\=ovai}\index{gnl}{Tirukkalumalamummanikkovai@\textit{Tirukka\b lumalamumma\d nikk\=ovai}}, \og Triple suite de gemmes sur Tirukka\b lumalam\index{gnl}{Kalumalam@Ka\b lumalam}\fg, est un des cinq poème\index{gnl}{poeme@poème}s attribu\'es à Pa\d t\d ti\b nattuppi\d l\d lai\index{gnl}{Pattinattu Pillai@Pa\d t\d ti\b nattuppi\d l\d lai}. Ce texte appartient au genre nomm\'e \textit{mumma\d nikk\=ovai} qui se caract\'erise par la succession de trois m\`etres (\textit{akava\b rp\=a}, \textit{ve\d np\=a} et \textit{ka\d t\d ta\d laikkalittu\b rai})\footnote{Pour une introduction \`a la m\'etrique tamoule, cf. \textsc{Niklas} 1988.} ordonn\'es en \textit{ant\=ati}\index{gnl}{antati@\textit{ant\=ati}} (reprise du dernier mot d'une strophe pour commencer la strophe suivante). Il comporte quatre triples suites sur un total de cent cinquante-six vers. Les strophes en \textit{ve\d np\=a} et en \textit{ka\d t\d ta\d laikkalittu\b rai} sont syst\'ematiquement des quatrains.

Tr\`es peu d'informations sont disponibles sur l'auteur Pa\d t\d ti\b nattuppi\d l\d lai. Voici un r\'esum\'e de l'entr\'ee de \textsc{Zvelebil} (1995: s.v.): son nom sugg\`ere qu'il est n\'e \`a Pa\d t\d ti\b nam, \textit{i.e.} K\=av\=eripa\d t\d ti\b nam\index{gnl}{Kaveripattinam@K\=av\=eripa\d t\d ti\b nam}. Il serait issu d'une famille de commer\c cants. D'apr\`es le fondement traditionnel de sa contemporan\'eit\'e avec C\=enta\b n\=ar\index{gnl}{Centanar@C\=enta\b n\=ar} et Karuv\=urtt\=evar\index{gnl}{Karuvurttevar@Karuv\=urtt\=evar} (poète\index{gnl}{poete@poète}s du livre \textsc{ix} du \textit{Tirumu\b rai}\index{gnl}{Tirumurai@\textit{Tirumu\b rai}}) et d'apr\`es ses r\'ef\'erences aux \textit{m\=uvar}\index{gnl}{muvar@\textit{m\=uvar}} et \`a M\=a\d nikkav\=acakar\index{gnl}{Manikkavacakar@M\=a\d nikkav\=acakar}, il aurait v\'ecu \`a la fin du \textsc{x}\up{e} si\`ecle et au d\'ebut du si\`ecle suivant. Selon sa légende\index{gnl}{legende@légende} narr\'ee dans le \textit{Pa\d t\d ti\b nattuppi\d l\d laipur\=a\b nam}\index{gnl}{Pattinattu Pillai@Pa\d t\d ti\b nattuppi\d l\d lai!\textit{Pa\d t\d ti\b nattuppi\d l\d laipur\=a\b nam}}, d'auteur inconnu, sa vie est marqu\'ee par le d\'ec\`es de son fils et par l'adoption d'un enfant\index{gnl}{enfant} pauvre. Plus tard, il se fait renon\c cant \`a Tiruvi\d taimarut\=ur\index{gnl}{Tiruviraimarutur@Tiruvi\d taimarut\=ur} qu'il c\'el\`ebre. Il a chant\'e, entre autres, K\=oyil (Citamparam\index{gnl}{Citamparam}), Tiruv\=ekampamu\d taiy\=ar \`a K\=a\~ncipuram\index{gnl}{Kancipuram@K\=a\~ncipuram}, Tiruvo\b r\b riy\=ur\index{gnl}{Tiruvorriyur@Tiruvo\b r\b riy\=ur} et, bien s\^ur, Tirukka\b lumalam\index{gnl}{Kalumalam@Ka\b lumalam!Tirukka\b lumalam} (C\=\i k\=a\b li\index{gnl}{Cikali@C\=\i k\=a\b li}). \textsc{Arunachalam} (1985) r\'esume les hagiographie\index{gnl}{hagiographie}s des dévot\index{gnl}{devot(e)@dévot(e)}s shiva\"ite\index{gnl}{shiva\"ite}s. Il consacre cinq pages \`a Pa\d t\d ti\b nattuppi\d l\d lai\index{gnl}{Pattinattu Pillai@Pa\d t\d ti\b nattuppi\d l\d lai} (p.~208-213). Cependant, il n'y pr\'esente pas le r\'ecit de vie de l'auteur qui nous int\'eresse mais celui d'un individu du m\^eme nom qui aurait v\'ecu au \textsc{xiv}\up{e} si\`ecle. Bien qu'il ne mentionne pas ses sources, l'histoire qu'il narre est semblable \`a celle du \textit{Pa\d t\d ti\b nattuppi\d l\d laipur\=a\b nam}\index{gnl}{Pattinattu Pillai@Pa\d t\d ti\b nattuppi\d l\d lai!\textit{Pa\d t\d ti\b nattuppi\d l\d laipur\=a\b nam}}: un marchand de K\=av\=eripa\d t\d ti\b nam\index{gnl}{Kaveripattinam@K\=av\=eripa\d t\d ti\b nam} renonce \`a la vie mondaine apr\`es la perte de son fils adoptif. Nous avons l'impression qu'il y a eu une confusion entre deux individus d'\'epoques diff\'erentes qui portent le m\^eme nom et que, par cons\'equent, on leur octroie une biographie\index{gnl}{biographie} et une bibliographie semblables. %\`A propos de son \oe uvre \textsc{Zvelebil} \'ecrit (1995, 539): \og It is of rather philos[ophical] nature, with echoes of asceticism, and enjoys great reputation. There is grandeur combined with simplicity, and his st[anzas] are of high poetic quality\fg.
Dans l'\'etat actuel des recherches il est difficile de d\'eterminer si l'auteur qui c\'el\`ebre C\=\i k\=a\b li\index{gnl}{Cikali@C\=\i k\=a\b li} appartient à la fin du \textsc{x}\up{e}, au \textsc{xiv}\up{e} ou \`a un autre si\`ecle.

\subsection{Les \oe uvres de Nampi \=A\d n\d t\=ar Nampi}

Les dix \oe uvres attribu\'ees \`a Nampi \=A\d n\d t\=ar Nampi\index{gnl}{Nampi \=A\d n\d t\=ar Nampi} concluent le livre \textsc{xi} du \textit{Tirumu\b rai}\index{gnl}{Tirumurai@\textit{Tirumu\b rai}}. Elles c\'el\`ebrent Ga\d ne\'sa\index{gnl}{Ganesa@Ga\d ne\'sa} \`a Tirun\=araiy\=ur\index{gnl}{Naraiyur@N\=araiy\=ur!Tirun\=araiy\=ur} (\textit{Tirun\=araiy\=ur vin\=ayakar ira\d t\d tai ma\d nim\=alai}), Citamparam\index{gnl}{Citamparam} (\textit{K\=oyil tiruppa\d n\d niyar viruttam}), Appar\index{gnl}{Appar} (\textit{Tirun\=avukkaracu t\=evar tiruv\=ek\=ataca m\=alai}), les soixante-trois \textit{n\=aya\b nm\=ar}\index{gnl}{nayanmar@\textit{n\=aya\b nm\=ar}} (\textit{Tirutto\d n\d tar tiruvant\=ati}) et, enfin, Campantar\index{gnl}{Campantar} dans six genres po\'etiques diff\'erents dans les titres desquels il est nomm\'e \=A\d lu\d taiyapi\d l\d laiy\=ar\index{gnl}{Campantar!Alutaiyapillaiyar@\=A\d lu\d taiyapi\d l\d laiy\=ar}\footnote{\=A\d lu\d taiyapi\d l\d laiy\=ar\index{gnl}{Campantar!Alutaiyapillaiyar@\=A\d lu\d taiyapi\d l\d laiy\=ar} est l'appellation r\'epandue de l'image\index{gnl}{image} de Campantar\index{gnl}{Campantar} en contexte \'epigraphique, voir notre troisi\`eme partie.}.

\subsubsection{Les textes sur Campantar\index{gnl}{Campantar}}

Le \textit{Tirutto\d n\d tar tiruvant\=ati}, \og Ant\=ati des saints serviteur\index{gnl}{serviteur}s\fg, est la r\'eplique du \textit{Tirutto\d n\d tattokai}\index{gnl}{Tiruttontattokai@\textit{Tirutto\d n\d tattokai}} de Cuntatar. Il pr\'esente dans le m\^eme ordre\index{gnl}{ordre} et en quatre-vingt-neuf quatrains les dévot\index{gnl}{devot(e)@dévot(e)}s mentionn\'es par Cuntarar\index{gnl}{Cuntarar} en d\'ecrivant, en un court \'episode, leur trait particulier. Il est pr\'efac\'e par une strophe ajout\'ee postérieurement, appel\'ee \textit{cirappupp\=ayiram}\index{gnl}{payiram@\textit{p\=ayiram}} (\og introduction sp\'eciale\fg)\footnote{Sur la valeur du \textit{cirappupp\=ayiram}, le choix de son auteur et la notori\'et\'e qu'il conf\`ere aux \oe uvres des lettr\'es tamouls du \textsc{xix}\up{e} si\`ecle, cf. \textsc{Ebeling} 2010: 38-42.}, dans laquelle l'auteur prend refuge\index{gnl}{refuge} aux pieds de Nampi\index{gnl}{Nampi \=A\d n\d t\=ar Nampi}, en pr\'ecisant que ce dernier a compos\'e en \textit{ant\=ati}\index{gnl}{antati@\textit{ant\=ati}} le \textit{to\d n\d tattokai} sur les soixante-trois avec la gr\^ace du dieu\index{gnl}{dieu} \`a t\^ete d'\'el\'ephant de Tirun\=araiy\=ur\index{gnl}{Naraiyur@N\=araiy\=ur!Tirun\=araiy\=ur}
\footnote{\textit{Tirutto\d n\d tar tiruvant\=ati cirappupp\=ayiram}:
\begin{verse}
\textit{po\d n\d ni va\d takarai c\=er\textbf{n\=arai y\=uri\b r pu\b laikkaimuka\\
ma\b n\b na\b n a\b rupattu m\=uvar} patit\=em marapuceyal\\
pa\b n\b na-at \textbf{to\d n\d tat tokai}vakai palkum\textbf{an t\=ati}ta\b naic\\
co\b n\b na ma\b raikkula \textbf{nampi}po\b r p\=atat tu\d naitu\d naiy\=e}
\end{verse}}.

Les six poème\index{gnl}{poeme@poème}s \`a la gloire de Campantar\index{gnl}{Campantar} sont de longueur et de style divers. L'\textit{\=A\d lu\d taiyapi\d l\d laiy\=ar\index{gnl}{Campantar!Alutaiyapillaiyar@\=A\d lu\d taiyapi\d l\d laiy\=ar} tiruvant\=ati} comporte cent quatrains en m\`etre \textit{ka\d t\d talaikkalittu\b rai} et un quatrain en \textit{ve\d np\=a}.
L'\textit{\=A\d lu\d taiyapi\d l\d laiy\=ar\index{gnl}{Campantar!Alutaiyapillaiyar@\=A\d lu\d taiyapi\d l\d laiy\=ar} tiruca\d npai viruttam} c\'el\`ebre Tiruca\d npai (C\=\i k\=a\b li\index{gnl}{Cikali@C\=\i k\=a\b li}) en onze quatrains de m\`etre \textit{viruttam}, ancienne nomination du mètre \textit{ka\d t\d talaikkalittu\b rai}\footnote{Cf. l'\'edition de Tarumapuram\index{gnl}{Tarumapuram}, livre \textsc{xi} p.~751.}.
L'\textit{\=A\d lu\d taiyapi\d l\d laiy\=ar\index{gnl}{Campantar!Alutaiyapillaiyar@\=A\d lu\d taiyapi\d l\d laiy\=ar} tirumumma\d nikk\=ovai} forme un ensemble de dix triples suites.
L'\textit{\=A\d lu\d taiyapi\d l\d laiy\=ar\index{gnl}{Campantar!Alutaiyapillaiyar@\=A\d lu\d taiyapi\d l\d laiy\=ar} tiruvul\=am\=alai} appartient au genre \textit{ul\=a} dont le sujet d\'ecrit la procession\index{gnl}{procession} d'une divinit\'e ou d'un roi\index{gnl}{roi} sous le regard enamour\'e des femme\index{gnl}{femme}s de tous \^ages. Il contient cent quarante-trois distiques en m\`etre \textit{kalive\d np\=a}.
L'\textit{\=A\d lu\d taiyapi\d l\d laiy\=ar\index{gnl}{Campantar!Alutaiyapillaiyar@\=A\d lu\d taiyapi\d l\d laiy\=ar} tirukkalampakam} est un pot-pourri contenant quarante-neuf strophes de m\`etre et style diff\'erents\footnote{Cf. \textsc{Zvelebil} (1995: 305-306) pour un bref expos\'e de ce genre.}.
L'\textit{\=A\d lu\d taiyapi\d l\d laiy\=ar\index{gnl}{Campantar!Alutaiyapillaiyar@\=A\d lu\d taiyapi\d l\d laiy\=ar} tiruttokai} est une collection des miracle\index{gnl}{miracle}s li\'es \`a Campantar\index{gnl}{Campantar} d\'ecrite en soixante-cinq vers. La multiplicit\'e des genres compos\'es sur Campantar\index{gnl}{Campantar} souligne non seulement les talents litt\'eraires de Nampi\index{gnl}{Nampi \=A\d n\d t\=ar Nampi}, s'il en est vraiment l'unique auteur, mais surtout, pour le propos de cette \'etude, le prestige du poète\index{gnl}{poete@poète} et de son temple\index{gnl}{temple} \`a date ancienne. En effet, ces six hymne\index{gnl}{hymne}s, souvent inexploit\'es, contiennent des r\'ef\'erences pr\'ecises et abondantes \`a la légende\index{gnl}{legende@légende} de Campantar\index{gnl}{Campantar}.

\subsubsection{Nampi \=A\d n\d t\=ar Nampi\index{gnl}{Nampi \=A\d n\d t\=ar Nampi}}

L'auteur, Nampi \=A\d n\d t\=ar Nampi\index{gnl}{Nampi \=A\d n\d t\=ar Nampi}, est le compilateur, d'apr\`es la tradition\index{gnl}{tradition}, des onze premiers livres du \textit{Tirumu\b rai}\index{gnl}{Tirumurai@\textit{Tirumu\b rai}} (cf. 4.1)\footnote{Sur l'hypoth\`ese consistant \`a voir en Um\=apati\index{gnl}{Umapati@Um\=apati} Civ\=ac\=ariy\=ar le v\'eritable compilateur du canon\index{gnl}{canon}, cf. \textsc{Prentiss} 2001a.}. La légende\index{gnl}{legende@légende} du \textit{Tirumu\b raika\d n\d tapur\=a\b nam} narre qu'il est n\'e d'un père\index{gnl}{pere@père} officiant shiva\"ite\index{gnl}{shiva\"ite}, et qu'il b\'en\'eficie tr\`es jeune des faveurs divines. Le Ga\d ne\'sa\index{gnl}{Ganesa@Ga\d ne\'sa} de Tirun\=araiy\=ur\index{gnl}{Naraiyur@N\=araiy\=ur!Tirun\=araiy\=ur} est son enseignant priv\'e. Puis, il devient son guide dans la recherche des textes perdus. Et enfin, l'ordre\index{gnl}{ordre} d'agencer les hymne\index{gnl}{hymne}s selon les modes musicaux lui est donn\'e par la voix de \'Siva\index{gnl}{Siva@\'Siva}. Il n'est pas le premier compilateur (st. 23), au moins de ce qui forme le \textit{T\=ev\=aram}\index{gnl}{Tevaram@\textit{T\=ev\=aram}}, et il a ajout\'e sa louange des soixante-trois dévot\index{gnl}{devot(e)@dévot(e)}s dans le livre \textsc{xi} du canon\index{gnl}{canon} (st. 29). Aucune mention des autres textes ne figure dans ce \textit{pur\=a\d nam}.

La datation de Nampi\index{gnl}{Nampi \=A\d n\d t\=ar Nampi} est un sujet fort d\'ebattu. Nous constatons que seuls les deux textes religieux du \textit{Tirutto\d n\d tar tiruvant\=ati} que la tradition\index{gnl}{tradition} lui attribue et du \textit{Tirumu\b raika\d n\d tapur\=a\b nam}, attribu\'e \`a un Um\=apati\index{gnl}{Umapati@Um\=apati}, constituent les sources de son \'etude.

Le \textit{Tirutto\d n\d tar tiruvant\=ati} mentionne \`a trois reprises des rois \textit{c\=o\b la}\index{gnl}{cola@\textit{c\=o\b la}} quand il est question de certains rois \textit{n\=aya\b nm\=ar}\index{gnl}{nayanmar@\textit{n\=aya\b nm\=ar}}: Puka\b lcc\=o\b la\index{gnl}{Pukalccola@Puka\b lcc\=o\b la} (st. 50), I\d ta\.nka\b li\index{gnl}{Itankali@I\d ta\.nka\b li} (st. 65) et K\=occe\.nka\d n\index{gnl}{Koccenkan@K\=occe\.nka\d n} (st. 81-82). Certains chercheurs per\c coivent dans ces strophes des r\'ef\'erences au roi\index{gnl}{roi} r\'egnant de la p\'eriode de Nampi \=A\d n\d t\=ar Nampi\index{gnl}{Nampi \=A\d n\d t\=ar Nampi}. Par exemple, \textsc{Ve\d l\d laiv\=ara\d na\b n} (*1994 [1962 et 1969]: 16-24) et \textsc{Rangaswamy} (*1990 [1958]: 22-23) pensent que Nampi\index{gnl}{Nampi \=A\d n\d t\=ar Nampi} est contemporain d'\=Aditya I\index{gnl}{Aditya I@\=Aditya I} (871-907) car ils per\c coivent une allusion \`a ce roi\index{gnl}{roi} \textit{c\=o\b la}\index{gnl}{cola@\textit{c\=o\b la}} dans la br\`eve description du roi\index{gnl}{roi}-dévot\index{gnl}{devot(e)@dévot(e)} Puka\b lcc\=o\b la\index{gnl}{Pukalccola@Puka\b lcc\=o\b la}: K\=oka\b nan\=ata\b n est un de ses noms\footnote{\textsc{Rangaswamy} (*1990 [1958]: 22): \og In verse 50 he refers to the contemporary C\=o\b la king as the victor of Ceylon and calls the king K\=oka\b nakan\=ata\b n. This term means the Lord of the lotus, \textit{i.e.}, the sun. The proper name equivalent to this as found in the list of C\=o\b la\index{gnl}{Cola@C\=o\b la} kings is \=Aditya\fg. Mais l'\'edition de Tarumapuram\index{gnl}{Tarumapuram}, qui lit K\=oka\b nan\=ata\b n, stipule dans son commentaire que ce terme est une d\'esignation g\'en\'erale de la dynastie solaire et qu'il ne renvoie \`a aucun roi\index{gnl}{roi} pr\'ecis. Quant \`a la victoire sur Srilanka\index{gnl}{Srilanka}, elle serait l'\oe uvre de diff\'erents rois de la dynastie \textit{c\=o\b la}\index{gnl}{cola@\textit{c\=o\b la}} commen\c cant par le l\'egendaire Karik\=ala\index{gnl}{Karikala@Karik\=ala}\b n.}. Puis, la st. 65 renvoie \`a la conqu\^ete du Pays Ko\.nku\index{gnl}{Pays Konku@Pays Ko\.nku} et de son or, avec lequel \=Atitta\b n couvre le toit de la \textit{ci\b r\b rampalam} du temple\index{gnl}{temple} de Citamparam\index{gnl}{Citamparam}\footnote{\textit{Tirutto\d n\d tar tiruvant\=ati}:
\begin{verse}
\textit{ci\.nkat turuva\b naic ce\b r\b rava\b n ci\b r\b ram palamuka\d tu\\
ko\.nki\b r ka\b naka ma\d nintav\=a titta\b n kulamutal\=o\b n} (65ab)\\
\end{verse} Litt\'eralement, \og [I\d ta\.nka\b li\index{gnl}{Itankali@I\d ta\.nka\b li}] est le premier de la lign\'ee d'\=Atitta\b n qui a orn\'e avec l'or de Ko\.nku\index{gnl}{Pays Konku@Pays Ko\.nku} le toit de la \textit{ci\b r\b rampalam} de Celui qui a d\'etruit l'Avatar de lion\fg. L'\'edition de Tarumapuram\index{gnl}{Tarumapuram} consid\`ere que la mention d'\=Atitta\b n sert uniquement \`a illustrer et \`a renforcer la grandeur d'I\d ta\.nka\b li\index{gnl}{Itankali@I\d ta\.nka\b li}, anc\^etre et en tant que tel, sup\'erieur exemplaire de ce roi\index{gnl}{roi} \textit{c\=o\b la}\index{gnl}{cola@\textit{c\=o\b la}} qui a couvert d'or le toit de Citamparam\index{gnl}{Citamparam} et, qu'il n'est pas n\'ecessaire d'\'etablir une contemporan\'eit\'e entre le roi\index{gnl}{roi} \=Aditya I\index{gnl}{Aditya I@\=Aditya I} et Nampi\index{gnl}{Nampi \=A\d n\d t\=ar Nampi}. Par ailleurs, il est int\'eressant de souligner l'existence explicite d'une forme de \'Siva\index{gnl}{Siva@\'Siva} destructeur de Vi\d s\d nu\index{gnl}{Visnu@Vi\d s\d nu}-Narasi\d mha, probablement celle de \'Sarabha, \`a l'\'epoque de Nampi\index{gnl}{Nampi \=A\d n\d t\=ar Nampi}.}, et enfin, la st. 82 \'evoque un roi\index{gnl}{roi} dévot\index{gnl}{devot(e)@dévot(e)} qui effectue la m\^eme action\footnote{\textit{Tirutto\d n\d tar tiruvant\=ati}:
\begin{verse}
\textit{cempo \b na\d nintuci\b r \b rampalat taicciva l\=okameyti\\
nampa\b n ka\b la\b rk\=\i\ \b lirunt\=o\b n kulamuta le\b nparnalla} (82ab)\\
\end{verse} \og [K\=occe\.nka\d n\index{gnl}{Koccenkan@K\=occe\.nka\d n}] est dit \^etre le premier de la lign\'ee de celui qui, ayant orn\'e d'or pur la \textit{ci\b r\b rampalam} qu'il consid\`ere comme le monde de \'Siva\index{gnl}{Siva@\'Siva}, reste aux pieds du Seigneur\fg. Aucun titre royal n'est donn\'e.}. Le raisonnement de ces chercheurs est tr\`es s\'eduisant \`a premi\`ere vue mais il ne nous convainc pas. Nous pensons que les appellations telles que K\=oka\b nan\=ata\b n et \=Atitta\b n sont employ\'ees pour signifier lexicalement l'appartenance du roi\index{gnl}{roi} \textit{c\=o\b la}\index{gnl}{cola@\textit{c\=o\b la}} \`a la dynastie solaire. En effet, elles d\'esignent toutes les deux le soleil: K\=oka\b nan\=ata\b n est le \og Seigneur aux lotus\fg, \textit{i.e.} S\=urya, de m\^eme qu'\=Atitta\b n (sk. \textit{\=aditya}, \og soleil\fg). De plus, les r\'ef\'erences \`a ce ou ces rois descendants des rois \textit{n\=aya\b nm\=ar}\index{gnl}{nayanmar@\textit{n\=aya\b nm\=ar}}, qui ont couvert d'or Citamparam\index{gnl}{Citamparam} et qui ont conquis Srilanka\index{gnl}{Srilanka} et le Pays Ko\.nku\index{gnl}{Pays Konku@Pays Ko\.nku}, sugg\`erent des image\index{gnl}{image}s exemplaires de rois \textit{c\=o\b la}\index{gnl}{cola@\textit{c\=o\b la}}, victorieux, dévot\index{gnl}{devot(e)@dévot(e)}s shiva\"ite\index{gnl}{shiva\"ite}s et g\'en\'ereux. Couvrir d'or Citamparam\index{gnl}{Citamparam} ou le toit de la Citsabh\=a est un acte d\'evotionnel\index{gnl}{devotionnel@dévotionnel} attribu\'e \`a de nombreux rois \textit{c\=o\b la}\index{gnl}{cola@\textit{c\=o\b la}}\footnote{Le premier roi\index{gnl}{roi} \textit{c\=o\b la}\index{gnl}{cola@\textit{c\=o\b la}} qui aurait accompli ce don\index{gnl}{don} est \=Aditya I\index{gnl}{Aditya I@\=Aditya I} (871-907) selon la tradition\index{gnl}{tradition} que reprend \textsc{Younger} (1995: 94-95). Cependant, son argumentation reposant sur des sources litt\'eraires tardives n'est absolument pas concluante. Cet auteur rel\`eve en effet les trois strophes, cit\'ees ci-dessus de Nampi \=A\d n\d t\=ar Nampi\index{gnl}{Nampi \=A\d n\d t\=ar Nampi}, mais une seule mentionne v\'eritablement le nom \=Atitta\b n. Puis, \textsc{Younger} renvoie au \textit{Periyapur\=a\d nam}\index{gnl}{Periyapuranam@\textit{Periyapur\=a\d nam}}. Or, ce texte qui fait r\'ef\'erence à la dorure de Citamparam\index{gnl}{Citamparam} par un roi\index{gnl}{roi} (st. 8) n'\'etablit jamais de lien entre cette donation et \=Atitta\b n. Enfin, \textsc{Younger} a recours \`a un texte attribu\'e \`a un Um\=apati\index{gnl}{Umapati@Um\=apati} Civ\=ac\=ariyar, le \textit{Tirutto\d n\d tarpur\=a\d nac\=aram} st. 59, datant au plus t\^ot du \textsc{xiv}\up{e} si\`ecle, qui n'est en fait qu'une reprise du texte de Nampi\index{gnl}{Nampi \=A\d n\d t\=ar Nampi} et qui donc, ne permet nullement d'identifier le roi\index{gnl}{roi} \`a \=Aditya I\index{gnl}{Aditya I@\=Aditya I}. Les tablettes de cuivre d'A\b npil\index{gnl}{Anpil@A\b npil} de Par\=antaka II\index{gnl}{Parantaka II@Par\=antaka II} (960), EI 15 5, qui c\'el\`ebrent ce roi\index{gnl}{roi} en lui accordant le patronage d'un grand nombre de temples\index{gnl}{temple}, et dont la v\'eracit\'e est remise en question par \textsc{Kaimal} (1996), ne parlent pas de la dorure de Citamparam\index{gnl}{Citamparam}.

Ensuite, les grandes tablettes de Leiden\index{gnl}{Leiden}, EI 22 34 v. 17 de la partie sanskrite, sous R\=ajar\=aja I\index{gnl}{Rajaraja I@R\=ajar\=aja I} nous informent que Par\=antaka I\index{gnl}{Parantaka I@Par\=antaka I} (907-955) a couvert d'or le temple\index{gnl}{temple} de \'Siva\index{gnl}{Siva@\'Siva} \`a Vy\=aghr\=agrah\=ara, ville identifi\'ee comme Citamparam\index{gnl}{Citamparam}, alors que celles de Tiruv\=ala\.nk\=a\d tu\index{gnl}{Alankatu@\=Ala\.nk\=a\d tu!Tiruvalankatu@Tiruv\=ala\.nk\=a\d tu}, SII 3 205, sous R\=ajendra I\index{gnl}{Rajendra I@R\=ajendra I} en 1018 lui attribuent la construction de l'assemblée\index{gnl}{assemblée} d'or. Ces mêmes tablettes, SII 3 205 v. 53, ne mentionne pas Citamparam\index{gnl}{Citamparam} mais la \textit{dabhrasabh\=a}, forme sanskrite de la \textit{ci\b r\b rampalam}, \og petite assemblée\index{gnl}{assemblée}\fg. Nous soulignons qu'aucun texte connu du r\`egne de Par\=antaka I\index{gnl}{Parantaka I@Par\=antaka I} ne mentionne ces faits qui apparaissent uniquement dans la glorification de la lign\'ee \textit{c\=o\b la}\index{gnl}{cola@\textit{c\=o\b la}} par leur descendant au \textsc{xi}\up{e} si\`ecle. De plus, les tablettes de Leiden\index{gnl}{Leiden} et de Tiruv\=ala\.nk\=a\d tu\index{gnl}{Alankatu@\=Ala\.nk\=a\d tu!Tiruvalankatu@Tiruv\=ala\.nk\=a\d tu} ne s'accordent pas sur l'acte. Signalons aussi qu'un \textit{Tiruvicaipp\=a}\index{gnl}{Tiruvicaippa@\textit{Tiruvicaipp\=a}} attribu\'e \`a Ka\d n\d tar\=atittar, fils (?) de Par\=antaka I\index{gnl}{Parantaka I@Par\=antaka I}, mentionne \`a la st. 8 qu'un roi\index{gnl}{roi} \textit{c\=o\b la}\index{gnl}{cola@\textit{c\=o\b la}}, qui a conquis le pays \textit{p\=a\d n\d dya}\index{gnl}{pandya@\textit{p\=a\d n\d dya}} et \=I\b lam\index{gnl}{Srilanka!Ilam@\=I\b lam}, a couvert d'or l'assemblée\index{gnl}{assemblée} de Tillai\index{gnl}{Citamparam!Tillai}. Bien que beaucoup de chercheurs, \textsc{Cox} (2006a: 43) par exemple, pensent que ce roi\index{gnl}{roi} est Par\=antaka I\index{gnl}{Parantaka I@Par\=antaka I}, l'identification de l'auteur et du roi\index{gnl}{roi} qu'il c\'el\`ebre, s'il ne repr\'esente pas une figure st\'er\'eotyp\'ee du roi\index{gnl}{roi} \textit{c\=o\b la}\index{gnl}{cola@\textit{c\=o\b la}} victorieux, g\'en\'ereux, dévot\index{gnl}{devot(e)@dévot(e)}, nous para\^it encore tr\`es incertaine. Par ailleurs, il semble que couvrir un site d'or rel\`eve aussi de l'hyperbole de la louange royale: le roi\index{gnl}{roi} mythique Karik\=ala\index{gnl}{Karikala@Karik\=ala} aurait r\'enov\'e la ville de K\=a\~nci avec de l'or (SII 3 205 v. 42).

Puis, il faut attendre le \textsc{xii}\up{e} si\`ecle pour que des figures royales s'attribuent cette donation dans des inscriptions qui leur sont contemporaines. En 1114, Kuntavai \=A\b lv\=ar\index{gnl}{Kuntavai@Kuntavai \=A\b lv\=ar}, la soeur de Kulottu\.nga I\index{gnl}{Kulottu\.nga I} (1070-1122), renouvelle l'offrande dans une inscription de Citamparam\index{gnl}{Citamparam} (EI 5 p.~105 l.~7-9). Mais il y est seulement dit qu'\og elle couvre d'or le temple\index{gnl}{temple} entier\fg\ et non un toit (\textit{g\^oyil=el\^am \'sem-bo\b n m\^eynd\^a\d l}). Est-ce une image\index{gnl}{image} pour signifier qu'elle a fait beaucoup d'offrandes en or\string? Vikramac\=ola (1118-1135) aussi se glorifie de beaucoup de dons\index{gnl}{don} \og dor\'es\fg\ \`a Citamparam\index{gnl}{Citamparam} dans son éloge\index{gnl}{eloge@éloge} royal intitul\'e \textit{p\=um\=alai mi\d taintu po\b nm\=alai tika\b ltara} (Cf. SII 5 458, \textsc{Nilakanta Sastri} (*2000 [1955]: 344-345) et \textsc{Cuppirama\d niyam} (1983: 112-118)). Enfin, Kulottu\.nga II\index{gnl}{Kulottu\.nga II} porte souvent le titre de celui \og qui a couvert d'or la grande assemblée\index{gnl}{assemblée}\fg\ (cf. SII 8 575 l.~7-8: \textit{tirupperampalam pon menta \'sr\=\i kulottu\.nkaco\b la}, ARE 1927 350 et § 24, 1928-29 315, ainsi que \textsc{Nilakanta Sastri} (*2000 [1955]: 348) sur la relation de ce roi\index{gnl}{roi} \`a Citamparam\index{gnl}{Citamparam} et ses diff\'erents travaux). Dans SII 7 782, un village est nomm\'e d'apr\`es ce titre (l. 2: \textit{tirupperampalam ponme[y*]ntaperum\=a\d lnall\=ur}).}.
Ce ou ces \og bons\fg\ rois historiques et descendants de la lign\'ee solaire servent \`a souligner, dans le \textit{Tirutto\d n\d tar tiruvant\=ati}, la grandeur de leurs anc\^etres mythiques, les rois \textit{n\=aya\b nm\=ar}\index{gnl}{nayanmar@\textit{n\=aya\b nm\=ar}}. Enfin, la mention tr\`es probable de \'Sarabha\index{gnl}{Sarabha@\'Sarabha}, dont la premi\`ere repr\'esentation daterait du r\`egne de R\=ajar\=aja II\index{gnl}{Rajaraja II@R\=ajar\=aja II} (1146-1173)\footnote{Cf. \textsc{Balambal} 1998, chapitre 10. Bien que le mythe\index{gnl}{mythe} de \'Sarabha\index{gnl}{Sarabha@\'Sarabha} soit attest\'e dans le \textit{Skandapur\=a\d na} ancien --- le manuscrit\index{gnl}{manuscrit} le plus ancien de ce texte date de 810, \textsc{Bakker} (2004: 1) --- c'est seulement dans les versions plus tardives du \textit{\'Sivapur\=a\d na} et du \textit{Li\.ngapur\=a\d na} que \'Sarabha\index{gnl}{Sarabha@\'Sarabha} combat r\'eellement l'indomptable Narasi\d mha, cf. \textsc{Granoff} 2004.}, laisse supposer une datation du texte plus tardive que la fin du \textsc{ix}\up{e} ou le d\'ebut du \textsc{x}\up{e} si\`ecle\footnote{S'il faut lier litt\'eralement comme dans le texte cette figure \`a Citamparam\index{gnl}{Citamparam}, la datation serait encore plus tardive. Bien que \textsc{Gopal Iyer} (1991: 358) liste la forme de \'Sarabha\index{gnl}{Sarabha@\'Sarabha} dans le \textit{T\=ev\=aram}\index{gnl}{Tevaram@\textit{T\=ev\=aram}}, VII 6 1, l'unique occurrence, dans le corpus\index{gnl}{corpus}, du terme \textit{ma\d ta\.nkal\=a\b nai} conserve une ambigu\"it\'e contextuelle qui ne permet pas de trancher entre Yama\index{gnl}{Yama} et Narasi\d mha. Voir aussi la traduction glos\'ee et les notes de V. M.~\textsc{Subramanya Aiyar} pour cette strophe de Cuntarar\index{gnl}{Cuntarar}.}. Ainsi, le \textit{Tirutto\d n\d tar tiruvant\=ati} attribu\'e \`a Nampi\index{gnl}{Nampi \=A\d n\d t\=ar Nampi} date d'une \'epoque o\`u la dorure de Citamparam\index{gnl}{Citamparam} par des rois \textit{c\=o\b la}\index{gnl}{cola@\textit{c\=o\b la}} et la forme de \'Siva\index{gnl}{Siva@\'Siva} destructeur de Narasi\d mha semblent bien connus dans le Pays Tamoul\index{gnl}{Pays Tamoul}.

Le \textit{Tirumu\b raika\d n\d tapur\=a\b nam} est la seconde source qui a été utilis\'ee pour dater Nampi\index{gnl}{Nampi \=A\d n\d t\=ar Nampi}. Ce texte pr\'ecise que le roi\index{gnl}{roi} qui demande \`a Nampi\index{gnl}{Nampi \=A\d n\d t\=ar Nampi} de compiler le \textit{Tirumu\b rai}\index{gnl}{Tirumurai@\textit{Tirumu\b rai}} est R\=acar\=aca Apaiyakulac\=ekara\b n (st. 1). \textsc{Nilakanta Sastri} (*2000 [1955]: 637), \textsc{Swamy} (1972: 120), \textsc{Gros} (1984: 11) et \textsc{Nagaswamy} (1989: 221), entre autres, sont s\'eduits par l'identification de ce roi\index{gnl}{roi} \`a R\=ajar\=aja I\index{gnl}{Rajaraja I@R\=ajar\=aja I} (985-1014). Mais, encore une fois, la datation n'est pas convaincante. En premier lieu, dans ce texte l\'egendaire, \`a la gloire de Citamparam\index{gnl}{Citamparam}, dat\'e du \textsc{xiv}\up{e} si\`ecle au plus t\^ot, il est possible que le titre royal \textit{c\=o\b la}\index{gnl}{cola@\textit{c\=o\b la}} ne renvoie pas \`a un homme r\'eel mais au pouvoir politique qu'il incarne. Le monarque \textit{c\=o\b la}\index{gnl}{cola@\textit{c\=o\b la}}, repr\'esentant par excellence du Pays Tamoul\index{gnl}{Pays Tamoul} m\'edi\'eval, \`a l'origine\index{gnl}{origine} de la compilation\index{gnl}{compilation}, sert \`a l\'egitimer cette derni\`ere et \`a lui conf\'erer une valeur d'autorit\'e terrestre. Les interventions de Ga\d ne\'sa\index{gnl}{Ganesa@Ga\d ne\'sa} et de \'Siva\index{gnl}{Siva@\'Siva} la consacrent doublement et elle acquiert un statut divin. D'ailleurs, si ce titre d\'esigne un roi\index{gnl}{roi} pr\'ecis, comment expliquer que R\=ajar\=aja I\index{gnl}{Rajaraja I@R\=ajar\=aja I} n'appara\^it pas une seule fois \`a Citamparam\index{gnl}{Citamparam} (\textsc{Younger} 1995: 98)? Le titre R\=ajar\=aja, \og roi\index{gnl}{roi} des rois\fg, connoterait simplement la grandeur d'un roi\index{gnl}{roi} majestueux d'antan. \textsc{Zvelebil} (1975: 133-134) propose une hypoth\`ese qui identifie ce roi\index{gnl}{roi} comme Kulottu\.nga I\index{gnl}{Kulottu\.nga I} (1070-1122). Le titre Apaya\b n\index{gnl}{Apayan@Apaya\b n} (sk. \textit{abhaya}, \og sans crainte\fg), moins g\'en\'erique que R\=ajar\=aja, lui est attribu\'e\footnote{Voir \textsc{Nilakanta Sastri} (*2000 [1955]: 330-331) et \textsc{Zvelebil} (1975: 133 et n.~18) pour quelques r\'ef\'erences litt\'eraires et \'epigraphiques. Toutefois, SII 6 1338, une des r\'ef\'erences donn\'ees par ce dernier auteur, est erron\'ee.}. Mais, s'il faut combiner cette identification avec un roi\index{gnl}{roi} \textit{c\=o\b la}\index{gnl}{cola@\textit{c\=o\b la}} qui a couvert d'or le toit de Citamparam\index{gnl}{Citamparam}, l'interpr\'etation de \textsc{Zvelebil} ne convainc plus car nulle part il n'est dit que Kulottu\.nga I\index{gnl}{Kulottu\.nga I} a effectu\'e ce don\index{gnl}{don}. Surgit alors une autre solution, celle de Kulottu\.nga II\index{gnl}{Kulottu\.nga II} (1133-1150) qui porte aussi le titre Apaya\b n\index{gnl}{Apayan@Apaya\b n}\footnote{Nous renvoyons \`a \textsc{Zvelebil} (1975: 133, n.~18) et ajoutons aux strophes relevées par cet auteur les strophes 149, 159 et 250 du \textit{Kul\=ottu\.nkacc\=o\b la\b nul\=a} compos\'e par O\d t\d takk\=uttar au \textsc{xii}\up{e} si\`ecle.} et qui a couvert d'or un des toits de Citamparam\index{gnl}{Citamparam} (voir \textit{supra} note 33). D'ailleurs, n'est-ce pas sous son r\`egne qu'\`a C\=\i k\=a\b li\index{gnl}{Cikali@C\=\i k\=a\b li}, selon CEC 26, des hymne\index{gnl}{hymne}s constitu\'es en un corpus\index{gnl}{corpus} nomm\'e \textit{Tirumu\b rai}\index{gnl}{Tirumurai@\textit{Tirumu\b rai}} sont enferm\'es en proie aux termites et qu'un expert tamoul les recueille, les nettoie et les installe de nouveau\string? Notons encore que R\=ajar\=aja II\index{gnl}{Rajaraja II@R\=ajar\=aja II} est d\'esign\'e par le titre Apaya\b n\index{gnl}{Apayan@Apaya\b n} dans l'\textit{Ir\=acar\=acac\=o\b la\b nul\=a}, st.~352 et 354, compos\'e aussi par O\d t\d takk\=uttar (\textsc{Zvelebil} 1995: 502-504). Faut-il alors voir en R\=ajar\=aja II le roi\index{gnl}{roi} commanditaire de la légende\index{gnl}{legende@légende}\string?

Bref, les \'el\'ements manquent et les sp\'eculations peuvent continuer. Il est surprenant de constater, sauf erreur, le silence des donn\'ees \'epigraphiques sur Nampi\index{gnl}{Nampi \=A\d n\d t\=ar Nampi} ou son \'eventuelle image\index{gnl}{image} de culte\index{gnl}{culte}. Dans le cadre notre \'etude, il faudra se contenter de savoir que Nampi\index{gnl}{Nampi \=A\d n\d t\=ar Nampi}, l'auteur pr\'esum\'e de dix poème\index{gnl}{poeme@poème}s du livre \textsc{xi} du \textit{Tirumu\b rai}\index{gnl}{Tirumurai@\textit{Tirumu\b rai}}, est post\'erieur aux \textit{n\=aya\b nm\=ar}\index{gnl}{nayanmar@\textit{n\=aya\b nm\=ar}}, \`a M\=a\d nikkav\=acakar\index{gnl}{Manikkavacakar@M\=a\d nikkav\=acakar} et qu'il a une connaissance\index{gnl}{connaissance} de la légende\index{gnl}{legende@légende} de Campantar\index{gnl}{Campantar} proche de celle de C\=ekki\b l\=ar\index{gnl}{Cekkilar@C\=ekki\b l\=ar}, l'auteur du \textit{Periyapur\=a\d nam}\index{gnl}{Periyapuranam@\textit{Periyapur\=a\d nam}}.

\section{Le \textit{Periyapur\=a\d nam}}

Le \textit{Tirutto\d n\d tarpur\=a\d nam}\index{gnl}{Periyapuranam@\textit{Periyapur\=a\d nam}!\textit{Tirutto\d n\d tarpur\=a\d nam}}, \og légende\index{gnl}{legende@légende} des serviteur\index{gnl}{serviteur}s\fg, ou plus commun\'ement le \textit{Periyapur\=a\d nam}\index{gnl}{Periyapuranam@\textit{Periyapur\=a\d nam}}, \og Grande légende\index{gnl}{legende@légende}\fg, douzi\`eme et dernier livre du Canon\index{gnl}{canon}, attribu\'e \`a C\=ekki\b l\=ar\index{gnl}{Cekkilar@C\=ekki\b l\=ar}, est le texte de la \textit{bhakti}\index{gnl}{bhakti@\textit{bhakti}} shiva\"ite\index{gnl}{shiva\"ite} tamoule qui compl\`ete et mod\`ele les légende\index{gnl}{legende@légende}s de soixante-trois \textit{n\=aya\b nm\=ar}\index{gnl}{nayanmar@\textit{n\=aya\b nm\=ar}}, les cristallise et les ancre sur le sol tamoul. Chaque dévot\index{gnl}{devot(e)@dévot(e)}, incarnation de la d\'evotion\index{gnl}{devotion@dévotion} absolue envers \'Siva\index{gnl}{Siva@\'Siva}, est pr\'esent\'e dans le cadre r\'ealiste d'une communaut\'e particuli\`ere, d'un temps historique r\'evolu et d'une g\'eographie pr\'ecise (fig. 4.1). L'incorporation d'\'el\'ements l\'egendaires attest\'es, amplifi\'es et simplement cr\'e\'es, puis leur assimilation et fusion avec la dynamique narrative c\'el\'ebrant l'amour envers \'Siva\index{gnl}{Siva@\'Siva} forgent une hagiographie\index{gnl}{hagiographie} si efficace au final que la post\'erit\'e ne jurera que par ce texte pour aborder la litt\'erature d\'evotionnelle\index{gnl}{devotionnel@dévotionnel} ant\'erieure\footnote{Pour une critique scientifique du \textit{Periyapur\=a\d nam}\index{gnl}{Periyapuranam@\textit{Periyapur\=a\d nam}} comme source fondamentale pour dater les \textit{m\=uvar}\index{gnl}{muvar@\textit{m\=uvar}}, cf. \textsc{Swamy} 1975b et \textsc{Gros} (1984: n.~10).}.
\begin{figure}[!h]
  \centering
 \includegraphics[width=14cm]{docthese/photoCIIKAALI/Toniyappartemple023.JPG}
  \caption{Les soixante-trois \textit{n\=aya\b nm\=ar}, galerie sud du temple de \'Siva, C\=\i k\=a\b li (cliché G. \textsc{Ravindran}, EFEO, 2005).}
\end{figure}

Le poids de cette \oe uvre dans l'histoire de la litt\'erature tamoule est donc consid\'erable\footnote{Le \textit{Periyapur\=a\d nam}\index{gnl}{Periyapuranam@\textit{Periyapur\=a\d nam}} conserve aujourd'hui encore une place pr\'epond\'erante dans la vie cultuelle et religieuse de la soci\'et\'e tamoule. En effet, le texte compose le \textit{pa\~ncapur\=a\d nam}, \og Cinq \textit{pur\=a\d nam}\fg, r\'epertoire chant\'e par un \textit{\=otuv\=ar}\index{gnl}{otuvar@\textit{\=otuv\=ar}} \`a la fin des \textit{p\=uj\=a}\index{gnl}{puja@\textit{p\=uj\=a}} \=agamiques des temples\index{gnl}{temple}. Ce dernier est constitu\'e de cinq strophes tir\'ees respectivement du \textit{T\=ev\=aram}\index{gnl}{Tevaram@\textit{T\=ev\=aram}}, \textit{Tiruv\=acakam}\index{gnl}{Tiruvacakam@\textit{Tiruv\=acakam}}, \textit{Tiruvicaipp\=a}\index{gnl}{Tiruvicaippa@\textit{Tiruvicaipp\=a}}, \textit{Tiruppall\=a\d n\d tu}\index{gnl}{Tiruppallantu@\textit{Tiruppall\=a\d n\d tu}} et du \textit{Periyapur\=a\d nam}\index{gnl}{Periyapuranam@\textit{Periyapur\=a\d nam}} (informations communiqu\'ees par l'\textit{\=otuv\=ar}\index{gnl}{otuvar@\textit{\=otuv\=ar}} de C\=\i k\=a\b li\index{gnl}{Cikali@C\=\i k\=a\b li}). De plus, le \textit{Periyapur\=a\d nam}\index{gnl}{Periyapuranam@\textit{Periyapur\=a\d nam}} est fr\'equemment le sujet de d\'ebat, d'enseignement ou de discours religieux donn\'es dans les temples\index{gnl}{temple} et monast\`ere\index{gnl}{monastère}s \`a l'occasion des f\^etes\index{gnl}{fete@fête} ou c\'er\'emonies particuli\`eres. Ainsi, en 2006, T. V.~\textsc{Gopal Iyer} avait l'habitude de se rendre une fois par mois dans un temple\index{gnl}{temple} de Ce\b n\b nai pour exposer un point doctrinal, mythologique ou litt\'eraire touchant au \textit{Periyapur\=a\d nam}\index{gnl}{Periyapuranam@\textit{Periyapur\=a\d nam}}.}. De nombreuses \'etudes existent: \textsc{Peterson} (1994) et \textsc{Gros} (2001) offrent une pr\'esentation g\'en\'erale du texte ainsi que \textsc{Zvelebil} (1995), s.v. \textit{Periya-pur\=a\d nam}\index{gnl}{Periyapuranam@\textit{Periyapur\=a\d nam}}, qui ajoute un historique \'editorial du texte. Plus traditionnel et d\'etaill\'e est le grand expos\'e de \textsc{Ve\d l\d laiv\=ara\d na\b n} (*1994 [1962 et 1969]: 1012-1340). \textsc{Ir\=acam\=a\d nikka\b n\=ar} (*1996 [1968]) concentre sa recherche sur l'auteur C\=ekki\b l\=ar\index{gnl}{Cekkilar@C\=ekki\b l\=ar}\footnote{Un de ses autres ouvrages en tamoul, \textit{Periyapur\=a\d na \=ar\=aycci}, Madras, 1948 n'a pu \^etre consult\'e.}. Ailleurs, diff\'erents th\`emes de l'hagiographie\index{gnl}{hagiographie} ont \'et\'e abord\'es. L'image\index{gnl}{image} de la violence d\'evotionnelle\index{gnl}{devotionnel@dévotionnel}, sujet tr\`es exploit\'e, est analys\'ee par \textsc{Hudson} (*1990 [1989]) et par \textsc{Monius} (2004) par exemple, qui, faisant \'etat des \'etudes pr\'ec\'edentes, justifie la violence dans un contexte de r\'edaction r\'eactionnaire face aux \'ecrits ja\"in\index{gnl}{jain@ja\"in}s\footnote{Voir aussi l'ouvrage que nous n'avons pu consulter de Chandraleka \textsc{Vamadeva}, \textit{The concept of va\b n\b na\b npu (violent love) in Tamil Saivism, with special reference to the Periyapur\=a\d nam}, Suède: Uppsala University, 1995.}. \textsc{Peterson} (1983) pr\'esente l'\'elaboration d'une identit\'e\index{gnl}{identit\'e} religieuse communautaire \`a travers le syst\`eme de p\`elerinage\index{gnl}{pelerinage@pèlerinage} fix\'e par C\=ekki\b l\=ar\index{gnl}{Cekkilar@C\=ekki\b l\=ar}. L'antagonisme envers les mouvements ja\"in\index{gnl}{jain@ja\"in} et bouddhiste\index{gnl}{bouddhiste} consid\'er\'es comme \'etrangers et hérétique\index{gnl}{heretique@hérétique}s, d\`es le \textit{T\=ev\=aram}\index{gnl}{Tevaram@\textit{T\=ev\=aram}}, est soulign\'e par \textsc{Peterson} (*1999 [1998]) et \textsc{Davis} (*1999 [1998]). \textsc{Marr} (1979) et \textsc{L'Hernault} (1987: 96-107) d\'ecrivent l'iconographie narrative des soixante-trois dévot\index{gnl}{devot(e)@dévot(e)}s. Quelques \textit{n\=aya\b nm\=ar}\index{gnl}{nayanmar@\textit{n\=aya\b nm\=ar}} ont \'et\'e l'objet d'\'etudes particuli\`eres accompagn\'ees souvent de la traduction int\'egrale de leur hagiographie\index{gnl}{hagiographie}: Ci\b rutto\d n\d tar\index{gnl}{Ciruttontar@Ci\b rutto\d n\d tar} (\textsc{Hart} 1980 avec trad. et \textsc{Shulman} 1993), Cuntarar\index{gnl}{Cuntarar} (\textsc{Rangaswamy} *1990 [1958] et \textsc{Shulman} 1990), K\=araikk\=alammaiy\=ar\index{gnl}{Karaikkalammaiyar@K\=araikk\=alammaiy\=ar} (\textsc{Karavelane} 1982 avec trad. de J. \textsc{Vinson}, \textsc{Ramachandran} 1993 et \textsc{Prentiss} 2006 avec trad.), I\d laiy\=a\b nku\d ti M\=ara\b n\index{gnl}{Ilaiyankuti@I\d laiy\=a\b nku\d ti M\=ara\b n} (\textsc{Veluppillai} 2003b avec trad.), Ka\d n\d nappar\index{gnl}{Kannappar@Ka\d n\d nappar} (\textsc{Cox} 2005) et Nanta\b nar\index{gnl}{Nantanar@Nanta\b nar} (\textsc{Prentiss} 2005 avec trad.). Enfin, deux traductions compl\`etes du \textit{Periyapur\=a\d nam}\index{gnl}{Periyapuranam@\textit{Periyapur\=a\d nam}} sont disponibles: \textsc{Ramachandran} (1990-1995) et \textsc{McGlashan} (2006).

La pr\'esentation qui suit propose une introduction \`a la composition du \textit{Periyapur\=a\d nam}\index{gnl}{Periyapuranam@\textit{Periyapur\=a\d nam}}, un r\'esum\'e de la légende\index{gnl}{legende@légende} de sa formation et enfin, quelques remarques historiques.

\subsection{La composition du texte}

C\=ekki\b l\=ar\index{gnl}{Cekkilar@C\=ekki\b l\=ar} annonce ses sources dans le \textit{Periyapur\=a\d nam}\index{gnl}{Periyapuranam@\textit{Periyapur\=a\d nam}}: le \textit{Tirutto\d n\d tattokai}\index{gnl}{Tiruttontattokai@\textit{Tirutto\d n\d tattokai}} attribu\'e \`a Cuntarar\index{gnl}{Cuntarar} (st. 47-48 et 349) et un texte de Nampi \=A\d n\d t\=ar Nampi\index{gnl}{Nampi \=A\d n\d t\=ar Nampi} (st. 49), tr\`es certainement le \textit{Tirutto\d n\d tar tiruvant\=ati}. En effet, l'\textit{ant\=ati} puis l'hagiographie\index{gnl}{hagiographie} suivent fid\`element l'ordre\index{gnl}{ordre} de pr\'esentation des \textit{n\=aya\b nm\=ar}\index{gnl}{nayanmar@\textit{n\=aya\b nm\=ar}}, ainsi que des neuf groupes de dévot\index{gnl}{devot(e)@dévot(e)}s de Cuntarar\index{gnl}{Cuntarar}, et \'etoffent progressivement les légende\index{gnl}{legende@légende}s. \textsc{Ir\=acam\=a\d nikka\b n\=ar} (*1996 [1968], chapitre 7) pense que C\=ekki\b l\=ar\index{gnl}{Cekkilar@C\=ekki\b l\=ar} serait lui-m\^eme parti en p\`elerinage\index{gnl}{pelerinage@pèlerinage} visiter les diff\'erents temples\index{gnl}{temple} chant\'es ou li\'es \`a un dévot\index{gnl}{devot(e)@dévot(e)} particulier pour recueillir les donn\'ees l\'egendaires et qu'il aurait aussi fait usage des inscriptions lues sur les sites. Cette hypoth\`ese inv\'erifiable est parfaitement reformul\'ee par \textsc{Cox} (2006a: 73-93 et 2006b) qui pr\'esente, de fa\c con convaincante, les nombreux autres textes litt\'eraires et \'epigraphiques que C\=ekki\b l\=ar\index{gnl}{Cekkilar@C\=ekki\b l\=ar} aurait int\'egr\'es \`a son \oe uvre. \textsc{Monius} (2004) propose que le \textit{Periyapur\=a\d nam}\index{gnl}{Periyapuranam@\textit{Periyapur\=a\d nam}} aurait \'et\'e compos\'e en r\'eaction \`a l'\'epop\'ee ja\"in\index{gnl}{jain@ja\"in}e, le \textit{C\=\i vakacint\=ama\d ni}\index{gnl}{Civakacintamani@\textit{C\=\i vakacint\=ama\d ni}}. Toute son argumentation repose sur une information donn\'ee dans la légende\index{gnl}{legende@légende} de C\=ekki\b l\=ar\index{gnl}{Cekkilar@C\=ekki\b l\=ar} (\'ecrite au moins deux si\`ecles apr\`es le \textit{Periyapur\=a\d nam}\index{gnl}{Periyapuranam@\textit{Periyapur\=a\d nam}}!), nullement confirm\'ee par l'hagiographe. Les autres versions dans d'autres langues peignant la vie des \textit{n\=aya\b nm\=ar}\index{gnl}{nayanmar@\textit{n\=aya\b nm\=ar}} constitueraient une source suppl\'ementaire\footnote{\textsc{Swamy} (1975: 121) donne une liste de neuf textes en kanna\d da, \textsc{Gros} (2001: 23, n.~4 et 30, n.~15) \'evoque les versions t\'elougoue et sanskrites et \textsc{Nambi Arooran} (1977: 21-24) les trois. Voir aussi \textsc{Rao} (1990) pour une traduction du \textit{Basava Pur\=a\d na} (t\'elougou).}. Mais une \'etude comparative rigoureuse, que nous ne pouvons mener ici, est n\'ecessaire pour se prononcer avec certitude sur les influences mutuelles.

Le texte du \textit{Periyapur\=a\d nam}\index{gnl}{Periyapuranam@\textit{Periyapur\=a\d nam}} est organis\'e autour de l'hymne\index{gnl}{hymne} de Cuntarar\index{gnl}{Cuntarar}. Il se compose de quatre mille deux cent quatre-vingt-un quatrains r\'epartis en treize chapitres pr\'ec\'ed\'es d'une introduction (st. 1-10): les premier et dernier chapitres encadrent les onze autres intitul\'es suivant les premiers mots des onze strophes de Cuntarar\index{gnl}{Cuntarar}\footnote{Ainsi, le deuxi\`eme chapitre est intitul\'e \textit{Tillaiv\=a\b lanta\b nar carukkam}, le troisi\`eme \textit{Ilaimalinta carukkam}, le quatri\`eme \textit{Mummaiy\=al ulak\=a\d n\d ta carukkam}, le cinqui\`eme \textit{Tiruni\b n\b ra carukkam}, le sixi\`eme \textit{Vampa\b r\=a variva\d n\d tu carukkam}, le septi\`eme \textit{V\=arko\d n\d ta va\b namulaiy\=a\d l carukkam}, le huiti\`eme \textit{Poyya\d timaiyill\=ata pulavar carukkam}, le neuvi\`eme \textit{Ka\b raikka\d n\d ta\b n carukkam}, le dixi\`eme \textit{Ka\d tac\=u\b lnta carukkam}, le onzi\`eme \textit{Pattar\=ayppa\d niv\=ar carukkam} et le douzi\`eme \textit{Ma\b n\b niyac\=\i r carukkam}.}. Les chapitres 1 (st. 11-349) et 13 (st. 4229-4281), les hagiographie\index{gnl}{hagiographie}s de Kalikk\=ama\b n\index{gnl}{Kalikkaman@Kalikk\=ama\b n} (st. 3155-3563) et de C\=eram\=a\b n Perum\=a\d l\index{gnl}{Ceraman Perumal@C\=eram\=a\b n Perum\=a\d l} (st. 3748-3922), ainsi que le dernier quatrain de chaque chapitre, forment la légende\index{gnl}{legende@légende} de Cuntarar\index{gnl}{Cuntarar}\footnote{L'ordonnance\index{gnl}{ordonnance} des chapitres du \textit{Periyapur\=a\d nam}\index{gnl}{Periyapuranam@\textit{Periyapur\=a\d nam}} en fonction de Cuntarar\index{gnl}{Cuntarar} conduit \textsc{Ir\=acam\=a\d nikka\b n\=ar} (*1996: 119-121) \`a consid\'erer que Cuntarar\index{gnl}{Cuntarar} est le personnage principal de l'ouvrage et que l'hagiographie\index{gnl}{hagiographie} enti\`ere ne serait que le \textit{Cuntarar\index{gnl}{Cuntarar}pur\=a\d nam}. Le m\^eme auteur exploite ailleurs cette id\'ee (\textsc{Rajamanickam} 1964: 211-213) pour identifier le texte \textit{\=A\d lu\d taiyanampi \'Sr\=\i pur\=a\d nam} d'une inscription de Tiruvo\b r\b riy\=ur\index{gnl}{Tiruvorriyur@Tiruvo\b r\b riy\=ur} (SII 5 1358, l.~4) comme le \textit{Cuntararpur\=a\d nam} et donc comme le \textit{Periyapur\=a\d nam}\index{gnl}{Periyapuranam@\textit{Periyapur\=a\d nam}}.}. Le tableau 4.1, fond\'e sur l'organisation des onze strophes de Cuntarar\index{gnl}{Cuntarar}, illustre la fid\'elit\'e de l'agencement reproduit par Nampi\index{gnl}{Nampi \=A\d n\d t\=ar Nampi} et C\=ekki\b l\=ar\index{gnl}{Cekkilar@C\=ekki\b l\=ar}. Il pr\'esente les \textit{n\=aya\b nm\=ar}\index{gnl}{nayanmar@\textit{n\=aya\b nm\=ar}} et les groupes de dévot\index{gnl}{devot(e)@dévot(e)}s selon leur ordre\index{gnl}{ordre} d'apparition dans l'hymne\index{gnl}{hymne} du \textit{T\=ev\=aram}. Chaque encadr\'e correspond \`a un quatrain de Cuntarar\index{gnl}{Cuntarar} et/ou \`a un chapitre de C\=ekki\b l\=ar\index{gnl}{Cekkilar@C\=ekki\b l\=ar}. A l'exception du cas particulier de Cuntarar\index{gnl}{Cuntarar} les autres r\'ecits hagiographique\index{gnl}{hagiographie!hagiographique}s se succ\`edent selon l'ordre\index{gnl}{ordre} de l'hymne\index{gnl}{hymne} du \textit{T\=ev\=aram}\index{gnl}{Tevaram@\textit{T\=ev\=aram}}\footnote{Les st. 87-89 de Nampi\index{gnl}{Nampi \=A\d n\d t\=ar Nampi} ne portent pas sur Cuntarar\index{gnl}{Cuntarar} mais c\'el\`ebrent de fa\c con g\'en\'erale les dévot\index{gnl}{devot(e)@dévot(e)}s.}. Il appara\^it clairement dans le tableau 4.1 que les \textit{pur\=a\d nam} individuels ne jouissent pas d'un traitement \'egal: leur longueur diff\`ere et aussi par cons\'equent, leur importance. Ainsi, les légende\index{gnl}{legende@légende}s des groupes n'exc\`edent jamais dix quatrains. Les groupes nomm\'es \textit{Cittattaic civa\b np\=al\=e vaitt\=ar}, \og ceux qui posent leur esprit sur \'Siva\index{gnl}{Siva@\'Siva}\fg, et \textit{App\=alum a\d ticc\=arnta a\d tiy\=ar}, \og les dévot\index{gnl}{devot(e)@dévot(e)}s d'ailleurs qui atteignent les pieds de \'Siva\index{gnl}{Siva@\'Siva}\fg, ne b\'en\'eficient que d'une strophe chacun. Certains \textit{n\=aya\b nm\=ar}\index{gnl}{nayanmar@\textit{n\=aya\b nm\=ar}} connaissent aussi ce sort: par exemple, les parents de Cuntarar\index{gnl}{Cuntarar} n'ont qu'un quatrain chacun, Ma\.nkaiyarkkaraciy\=ar\index{gnl}{Mankaiyarkkaraci@Ma\.nkaiyarkkaraci} trois, C\=om\=aci M\=a\b ra\b n cinq, Ci\b rappuli six et enfin, pour abr\'eger la liste, Ka\d nan\=ata\b n, Catti, Ceruttu\d nai et Puka\b lttu\d nai sont d\'ecrits en sept strophes. Par opposition, les longues hagiographie\index{gnl}{hagiographie}s de deux auteurs du \textit{T\=ev\=aram}\index{gnl}{Tevaram@\textit{T\=ev\=aram}} occupent une place centrale: Appar\index{gnl}{Appar} est c\'el\'ebr\'e en quatre cent vingt-neuf quatrains et Campantar\index{gnl}{Campantar} en mille deux cent cinquante-six, soit sur plus d'un quart de l'\oe uvre int\'egrale!

%\noindent
%\begin{table}
%\begin{center}
%\caption{\textsc{Les soixante-trois n\=aya\b nm\=ar}}
%\end{center}
%\end{table}
\begin{center}
\scriptsize
\begin{longtable}{|l|c|r|}
\caption{Les soixante-trois \textit{n\=aya\b nm\=ar}}\endfirsthead
\hline
\textit{Tirutto\d n\d tattokai}\index{gnl}{Tiruttontattokai@\textit{Tirutto\d n\d tattokai}} & \textit{Tirutto\d n\d tar tiruvant\=ati} & \textit{Tirutto\d n\d tarpur\=a\d nam}\index{gnl}{Periyapuranam@\textit{Periyapur\=a\d nam}!\textit{Tirutto\d n\d tarpur\=a\d nam}}\endhead
\hline
\textit{Tirutto\d n\d tattokai}\index{gnl}{Tiruttontattokai@\textit{Tirutto\d n\d tattokai}} & \textit{Tirutto\d n\d tar tiruvant\=ati} & \textit{Tirutto\d n\d tarpur\=a\d nam}\index{gnl}{Periyapuranam@\textit{Periyapur\=a\d nam}!\textit{Tirutto\d n\d tarpur\=a\d nam}}\\
\hline\hline
 & & introduction st. 1-10\\
\hline\hline
 & & chap. 1 11-349\\
\hline\hline
st. 1 Tillaiv\=a\b lanta\b nar & st. 1 & chap. 2 350-359\\
Nilaka\d n\d ta\b n & 2 & 360-403\\
Iya\b rpakai & 3 & 404-439\\
I\d laiy\=a\.nku\d ti M\=ara\b n\index{gnl}{Ilaiyankuti@I\d laiy\=a\b nku\d ti M\=ara\b n} & 4 & 440-466\\
Meypporu\d l\index{gnl}{Meypporu\d l} & 5 & 467-490\\
Vi\b ra\b nmi\d n\d ta\b n & 6 & 491-501\\
Amarn\=\i ti & 7 & 502-549\\
Cuntarar\index{gnl}{Cuntarar} & 8 & 550\\
\hline\hline
st. 2 E\b ripatta\b n & 9 & 551-607\\
\=E\b n\=ati & 10 & 608-649\\
Ka\d n\d nappa\b n & 11 & 650-830\\
Ka\d tav\=ur Kalaya\b n & 12 Ku\.nkuliyakkalaya\b n & 831-865\\
M\=a\b nakka\~nc\=a\b ra\b n & 13 & 866-902\\
T\=aya\b n & 14 Ariv\=a\d t\d t\=aya\b n & 903-925\\
Ma\.nkai \=A\b n\=aya\b n & 15 & 926-966\\
Cuntarar\index{gnl}{Cuntarar} & 16 & 967\\
\hline\hline
st. 3 M\=urtti & 17 & 968-1016\\
Muruka\b n\index{gnl}{Murukan@Muruka\b n} & 18 & 1017-1030\\
Uruttira Pacupati & 19 & 1031-1040\\
N\=alaipp\=ov\=ar & 20 & 1041-1077\\
Ku\b ripputto\d n\d ta\b n & 21 & 1078-1205\\
Ca\d n\d ti\index{gnl}{Candesa@Ca\d n\d de\'sa!Ca\d n\d ti} & 22 & 1206-1264\\
Cuntarar\index{gnl}{Cuntarar} & 23 & 1265\\
\hline\hline
st. 4 N\=avukkaraca\b n & 24-25 & 1266-1694\\
Kulacci\b rai\index{gnl}{kulaccirai@Kulacci\b rai} & 26 & 1695-1705\\
Perumi\b lalai Ku\b rumpa\b n & 27 & 1706-1716\\
P\=ey & 28 K\=araikk\=al Ammaiy\=ar\index{gnl}{Karaikkalammaiyar@K\=araikk\=alammaiy\=ar} & 1717-1782\\
App\=uti & 29 & 1783-1827\\
N\=\i lanakka\b n\index{gnl}{Nilanakkan@N\=\i lanakka\b n} & 30 & 1828-1865\\
Naminanti & 31 & 1866-1897\\
Cuntarar\index{gnl}{Cuntarar} & 32 & 1898\\
\hline\hline
st. 5 Campanta\b n\index{gnl}{Campantar!Campanta\b n} & 33-34 & chap. 6 1899-3154\\
Kalikk\=ama\b n\index{gnl}{Kalikkaman@Kalikk\=ama\b n} & 35 & 3155-3563\\
Tirum\=ula\b n & 36 & 3564-3591\\
Ta\d n\d ti & 37 & 3592-3617\\
M\=urkka\b n & 38 & 3618-3629\\
C\=om\=aci M\=a\b ra\b n & 39 & 3630-3634\\
Cuntarar\index{gnl}{Cuntarar} & 40 & 3635\\
\hline\hline
st. 6 C\=akkiya\b n & 41 & 3636-3653\\
Ci\b rappuli & 42 & 3654-3659\\
Ci\b rutto\d n\d ta\b n & 43 & 3660-3747\\
Ka\b la\b ri\b r\b ra\b riv\=a\b n & 44-45 C\=eram\=a\b n Perum\=a\d l\index{gnl}{Ceraman Perumal@C\=eram\=a\b n Perum\=a\d l} & 3748-3922\\
Ka\d nan\=ata\b n & 46 & 3923-3929\\
K\=u\b r\b ra\b n & 47 & 3930-3937\\
Cuntarar\index{gnl}{Cuntarar} & 48 & 3938\\
\hline\hline
st. 7 Poyya\d timai ill\=ata pulavar & 49 & 3939-3941\\
Puka\b l c\=ola\b n & 50 & 3942-3982\\
Naraci\.nka Mu\b naiyaraiya\b n & 51 & 3983-3991\\
Atipatta\b n & 52 & 3992-4011\\
Kalikkampa\b n & 53 & 4012-4021\\
Kaliya\b n & 54 & 4022-4038\\
Catti & 55 & 4039-4045\\
Aiya\d tikal & 56 & 4046-4053\\
Cuntarar\index{gnl}{Cuntarar} & 57 & 4054\\
\hline\hline
st. 8 Ka\d nampulla\b n & 58 & 4055-4063\\
K\=ari & 59 & 4064-4068\\
Ne\d tum\=a\b ra\b n\index{gnl}{Netumaran@Ne\d tum\=a\b ra\b n} & 60 & 4069-4078\\
V\=ayil\=a\b n & 61 & 4079-4088\\
Mu\b naiya\d tuv\=a\b n & 62 & 4089-4094\\
Cuntarar\index{gnl}{Cuntarar} & 63 & 4095\\
\hline\hline
st. 9 Ka\b la\b rci\.nka\b n & 64 & 4096-4108\\
I\d ta\.nka\b li\index{gnl}{Itankali@I\d ta\.nka\b li} & 65 & 4109-4119\\
Ceruttu\d nai & 66 & 4120-4126\\
Puka\b lttu\d nai & 67 & 4127-4133\\
K\=o\d tpuli & 68 & 4134-4145\\
Cuntarar\index{gnl}{Cuntarar} & 69 & 4146\\
\hline\hline
st. 10 Pattar\=appa\d niv\=ar & 70 & chap. 11 4147-4154\\
Parama\b naiy\=e p\=a\d tuv\=ar & 71 & 4155-4156\\
Cittattaic civa\b np\=al\=e vaitt\=ar & 72 & 4157\\
Tiruv\=ar\=ur\index{gnl}{Ar\=ur@\=Ar\=ur!Tiruv\=ar\=ur} pirant\=ar & 73 & 4158-4159\\
Mupp\=otum tirum\=e\d ni t\=\i \d n\d tuv\=ar & 74 & 4160-4162\\
Mu\b lun\=\i \b ru p\=uciya mu\b nivar & 75 & 4163-4168\\
App\=alum a\d ticc\=arnta a\d tiy\=ar & 76 & 4169\\
Cuntarar\index{gnl}{Cuntarar} & 77 & 4170\\
\hline\hline
st. 11 Ni\b n\b rav\=ur P\=ucal & 78 & 4171-4188\\
M\=a\b ni & 79 Ma\.nkaiyarkkaraci\index{gnl}{Mankaiyarkkaraci@Ma\.nkaiyarkkaraci} & 4189-4191\\
N\=eca\b n & 80 & 4192-4196\\
Ce\.nka\d n & 81-82 & 4197-4214\\
Tirun\=\i laka\d n\d tattu P\=a\d na\b n\=ar & 83 & 4215-4226\\
Ca\d taiya\b n & 84 & 4227\\
Icai\~n\=a\b niy\=ar & 85 & 4228\\
Cuntarar\index{gnl}{Cuntarar} & 86 & \\
\hline\hline
 & & chap. 13 4229-4281\\
\hline
\end{longtable}
\end{center}


\normalsize


\subsection{La légende de C\=ekki\b l\=ar}

Le \textit{Tirutto\d n\d tarpur\=a\d navaral\=a\b ru}, \og Histoire de la légende\index{gnl}{legende@légende} des serviteur\index{gnl}{serviteur}s\fg, plus g\'en\'eralement appel\'e \textit{C\=ekki\b l\=arpur\=a\d nam}\index{gnl}{Cekkilarpuranam@\textit{C\=ekki\b l\=arpur\=a\d nam}}, \og légende\index{gnl}{legende@légende} de C\=ekki\b l\=ar\index{gnl}{Cekkilar@C\=ekki\b l\=ar}\fg, attribu\'e \`a un Um\=apati\index{gnl}{Umapati@Um\=apati} Civ\=ac\=ariyar est un texte compos\'e de cent trois quatrains qui narre les conditions de r\'edaction du \textit{Periyapur\=a\d nam}\index{gnl}{Periyapuranam@\textit{Periyapur\=a\d nam}}\footnote{Notre \'etude suit le texte pr\'esent\'e en introduction du premier volume du \textit{Periyapur\=a\d nam}\index{gnl}{Periyapuranam@\textit{Periyapur\=a\d nam}} \'edit\'e par Ci. K\=e. \textsc{Cuppirama\d niya Mutaliy\=ar}, p.~55-72.}.

D'apr\`es ce texte, l'auteur de l'hagiographie\index{gnl}{hagiographie} appartient au clan (\textit{ku\d ti}) des C\=ekki\b l\=ar\index{gnl}{Cekkilar@C\=ekki\b l\=ar} et se nomme Aru\d nmo\b litt\=evar. Il est issu d'une famille de \textit{v\=e\d l\=a\d lar} (caste de propri\'etaires terriens) de la r\'egion de Ku\b n\b ratt\=ur\index{gnl}{Kunrattur@Ku\b n\b ratt\=ur} situ\'ee dans la banlieue de la m\'etropole actuelle de Ce\b n\b nai. Il devient premier ministre\index{gnl}{ministre} du roi\index{gnl}{roi} et prend le titre d'Uttamac\=o\b lapallava\b n\index{gnl}{Uttamacolapallavan@Uttamac\=o\b lappallava\b n} (st. 18). Fervent dévot\index{gnl}{devot(e)@dévot(e)} du temple\index{gnl}{temple} shiva\"ite\index{gnl}{shiva\"ite} de N\=ak\=ecuram il fait construire un temple\index{gnl}{temple} du m\^eme nom dans son village natal. Un jour, constatant avec d\'eception que le roi\index{gnl}{roi} se r\'ejouit de la lecture du [\textit{C\=\i vaka}]\textit{cint\=ama\d ni}, texte ja\"in\index{gnl}{jain@ja\"in}, C\=ekki\b l\=ar\index{gnl}{Cekkilar@C\=ekki\b l\=ar} lui apprend que ces histoires futiles ne m\`enent pas \`a la v\'erit\'e, contrairement aux textes shiva\"ite\index{gnl}{shiva\"ite}s. Le roi\index{gnl}{roi} le questionne alors sur la nature de ces textes lib\'erateurs (st. 20-21). Le ministre\index{gnl}{ministre} lui pr\'esente les hymne\index{gnl}{hymne}s de Cuntarar\index{gnl}{Cuntarar} et de Nampi\index{gnl}{Nampi \=A\d n\d t\=ar Nampi} qui c\'el\`ebrent les dévot\index{gnl}{devot(e)@dévot(e)}s shiva\"ite\index{gnl}{shiva\"ite}s (st. 23) et lui d\'etaille quelques légende\index{gnl}{legende@légende}s. Le roi\index{gnl}{roi} s\'eduit lui demande de composer une grande \oe uvre po\'etique (\textit{peru\.nk\=aviyam}) d\'ecrivant les pays, les villes, les clans, les noms et les actes de ces dévot\index{gnl}{devot(e)@dévot(e)}s (st. 28). C\=ekki\b l\=ar\index{gnl}{Cekkilar@C\=ekki\b l\=ar} se rend \`a Citamparam\index{gnl}{Citamparam}, honore le \'Siva\index{gnl}{Siva@\'Siva} dansant et m\'edite sur son projet. La voix de \'Siva\index{gnl}{Siva@\'Siva} se fait entendre, prononce \textit{ulakel\=am} et donne ainsi le d\'ebut du texte (st. 31). L'auteur s'installe dans le pavillon \`a mille piliers et compose un ouvrage en deux parties, treize chapitres et quatre mille deux cent cinquante-trois strophes\footnote{Le nombre de strophes diff\`ere selon les \'editeurs. Par exemple, \textsc{\=A\b rumuka N\=avalar} en compte quatre mille deux cent quatre-vingt-six et Ci. K\=e. \textsc{Cuppirama\d niya Mutaliy\=ar} quatre mille deux cent quatre-vingt-un. Une trentaine de strophes sont consid\'er\'ees comme des interpolation\index{gnl}{interpolation}s (\textsc{Nambi Arooran} 1977: 19-20). Cependant il n'existe aucune \'edition critique du \textit{Periyapur\=a\d nam}\index{gnl}{Periyapuranam@\textit{Periyapur\=a\d nam}}.} appel\'e le \textit{Tirutto\d n\d tarpur\=a\d nam}\index{gnl}{Periyapuranam@\textit{Periyapur\=a\d nam}!\textit{Tirutto\d n\d tarpur\=a\d nam}} (st. 52-53). Le roi\index{gnl}{roi} apprend la fin de la r\'edaction du texte et se rend \`a Tillai\index{gnl}{Citamparam!Tillai} en grande pompe pour honorer le \'Siva\index{gnl}{Siva@\'Siva} dansant, C\=ekki\b l\=ar\index{gnl}{Cekkilar@C\=ekki\b l\=ar} et son poème\index{gnl}{poeme@poème}. Accueilli par les brahmane\index{gnl}{brahmane}s de Citamparam\index{gnl}{Citamparam} (\textit{Tillai\index{gnl}{Citamparam!Tillai} v\=a\b l anta\b nar}), il entre adorer la divinit\'e du temple\index{gnl}{temple} avec son ministre\index{gnl}{ministre}. \'Siva\index{gnl}{Siva@\'Siva} donne l'ordre\index{gnl}{ordre} de r\'eciter le texte \`a C\=ekki\b l\=ar\index{gnl}{Cekkilar@C\=ekki\b l\=ar} à qui il a octroy\'e le premier mot (st. 64). Le roi\index{gnl}{roi} envoie d\`es lors des invitations dans toutes les directions pour convier \`a la récitation\index{gnl}{recitation@récitation} (st. 66). La ville de Tillai\index{gnl}{Citamparam!Tillai}, les rues et le temple\index{gnl}{temple} sont orn\'es pour l'occasion. Un tr\^one (\textit{p\=\i \d tam}) \'erig\'e selon les normes \=agamiques re\c coit le texte (st. 78). La récitation\index{gnl}{recitation@récitation} d\'ebute le jour de l'ast\'erisme de naissance\index{gnl}{naissance} de Campantar\index{gnl}{Campantar} pour se terminer l'ann\'ee suivante (st. 80). Pendant toute cette ann\'ee le roi\index{gnl}{roi} s'est charg\'e de financer les c\'er\'emonies et de nourrir les dévot\index{gnl}{devot(e)@dévot(e)}s venus \'ecouter le poème\index{gnl}{poeme@poème}. Ensuite le texte, consid\'er\'e comme le cinqui\`eme \textit{Veda}\index{gnl}{Veda@\textit{Veda}} tamoul (st. 86), est honor\'e et men\'e en procession\index{gnl}{procession} \`a dos d'\'el\'ephant avec C\=ekki\b l\=ar\index{gnl}{Cekkilar@C\=ekki\b l\=ar} et le roi\index{gnl}{roi} qui le ventile avec un chasse-mouche. Les dieux font tomber une pluie de p\'etales. De retour au temple\index{gnl}{temple}, C\=ekki\b l\=ar\index{gnl}{Cekkilar@C\=ekki\b l\=ar} d\'epose le manuscrit\index{gnl}{manuscrit} devant le \'Siva\index{gnl}{Siva@\'Siva} dansant et l'honore. Le texte forme d\'esormais le douzi\`eme livre du \textit{Tirumu\b rai}\index{gnl}{Tirumurai@\textit{Tirumu\b rai}} (st. 96). C\=ekki\b l\=ar\index{gnl}{Cekkilar@C\=ekki\b l\=ar} se retire de la politique et consacre le restant de ses jours \`a m\'editer \`a Citamparam\index{gnl}{Citamparam} sur les soixante-trois \textit{n\=aya\b nm\=ar}\index{gnl}{nayanmar@\textit{n\=aya\b nm\=ar}}. Son fr\`ere cadet devient le nouveau ministre\index{gnl}{ministre} et prend le titre de To\d n\d taim\=a\b n.

\subsection{Les rep\`eres historiques}

Certains auteurs, comme \textsc{Nilakanta Sastri} (*2000 [1955]: 675-676) et \textsc{Zvelebil} (1975: 135), s'accordent pour dater le \textit{Periyapur\=a\d nam}\index{gnl}{Periyapuranam@\textit{Periyapur\=a\d nam}} du r\`egne de Kulottu\.nga II\index{gnl}{Kulottu\.nga II} (1133-1150). Ils se basent sur l'hagiographie\index{gnl}{hagiographie} qui mentionne, \`a plusieurs reprises, un roi\index{gnl}{roi} nomm\'e Anap\=aya (du sk., \og imp\'erissable\fg) identifi\'e comme Kulottu\.nga II\index{gnl}{Kulottu\.nga II} gr\^ace aux sources \'epigraphiques.

\`A notre connaissance\index{gnl}{connaissance}, dix occurrences du terme \textit{anap\=aya} (orthographi\'e aussi \textit{a\b nap\=aya}), d\'esignant un roi\index{gnl}{roi} \textit{c\=o\b la}\index{gnl}{cola@\textit{c\=o\b la}}, se trouvent dans le \textit{Periyapur\=a\d nam}\index{gnl}{Periyapuranam@\textit{Periyapur\=a\d nam}} et ce terme peut être appliqué à trois objets différents.
En premier lieu, le terme \textit{anap\=aya} renvoie \`a la figure g\'en\'erale d'un roi\index{gnl}{roi} repr\'esentant de la dynastie \textit{c\=o\b la}\index{gnl}{cola@\textit{c\=o\b la}}. Une partie du premier chapitre d\'ecrit la \og glorification du pays\fg\ (\textit{tirun\=a\d t\d tucci\b rappu}) et c\'el\`ebre le territoire \textit{c\=o\b la}\index{gnl}{cola@\textit{c\=o\b la}} sur lequel r\`egne un roi\index{gnl}{roi} imp\'erissable au sceptre juste, protecteur et triomphant (st.~22); plus loin, ce roi est l'h\'eritier de grands monarques mythiques, dévot\index{gnl}{devot(e)@dévot(e)}s de \'Siva\index{gnl}{Siva@\'Siva} (st.~404, 552, 2745, 3949 et 4210). Ensuite, un roi\index{gnl}{roi} l\'egendaire, dont le r\'ecit est narr\'e dans la \og glorification de la ville de Tiruv\=ar\=ur\index{gnl}{Ar\=ur@\=Ar\=ur!Tiruv\=ar\=ur}\fg\ (\textit{tiruv\=ar\=urt tirunakaracci\b rappu}, st.~86-135), s'appelle Anap\=aya\b n\index{gnl}{Anapayan@Anap\=aya\b n} (st.~85 et 98). Cette légende\index{gnl}{legende@légende}, bien connue par d'autres sources\footnote{Cf. les tablettes de cuivre de Leiden\index{gnl}{Leiden} (EI 22 34 l. 8) par exemple.}, raconte comment un roi\index{gnl}{roi} juste, descendant de Manu, condamne son fils \`a mourir \'ecras\'e par le char avec lequel ce dernier avait accidentellement tu\'e un veau. Enfin, deux strophes semblent faire allusion au roi\index{gnl}{roi} régnant au moment de la composition de l'hagiographie\index{gnl}{hagiographie}; ce roi\index{gnl}{roi} nomm\'e Anap\=aya\b n\index{gnl}{Anapayan@Anap\=aya\b n} est li\'e \`a Citamparam\index{gnl}{Citamparam}. La premi\`ere strophe appartient \`a l'\textit{avaiya\d takkam} (passage exprimant la modestie de l'auteur). Elle nomme le roi\index{gnl}{roi} Anap\=aya\b n\index{gnl}{Anapayan@Anap\=aya\b n} et mentionne qu'il est \og le C\=o\b la\index{gnl}{Cola@C\=o\b la} qui a couvert d'or rouge et pur la grande salle sacr\'ee (de Citamparam\index{gnl}{Citamparam}) du (Seigneur) Rouge\fg\footnote{Le commentaire de Ci. K\=e. \textsc{Cuppirama\d niya Mutaliy\=ar} et la traduction de \textsc{Ramachandran} (1990) sugg\`erent que ce roi\index{gnl}{roi} nomm\'e Anap\=aya\b n\index{gnl}{Anapayan@Anap\=aya\b n} a aussi commandit\'e l'hagiographie\index{gnl}{hagiographie}. Cependant, nous ne pouvons pas appuyer cette interpr\'etation qui n'est pas \'evidente \`a la lecture de cette strophe dont la structure demeure incompr\'ehensible;

\begin{verse}
\textit{m\=eya vivvurai ko\d n\d tu virumpum\=am\\
c\=eya va\b n\b rirup p\=erampala\~n ceyya\\
t\=uya po\b n\b na\d ni c\=o\b la\b n\=\i\ \d t\=u\b lip\=ar\\
\=aya c\=\i r-ana p\=aya \b naracavai} (8)\\
\end{verse}
%\begin{verse}\og Wide and extensive is the earth ruled by the Chola\\Who gold-plated the divine roof of the great Ambalam\\Where is enshrined the Ruddy One.\\As Anapayan's royal court, in love, willed it\\(We compose this opus)\fg\footnote{(Traduction de \textsc{Ramachandran} 1990)}\\\end{verse}
%\og The C\=o\b la\index{gnl}{Cola@C\=o\b la}, he who decorated Ceyava\b n's holy \textit{p\=erampalam} with pure gold, Anap\=aya\b n\index{gnl}{Anapayan@Anap\=aya\b n}, whose merits are known throughout the wide earth---it was in fact his royal assembly who wished for these loving words\fg. Cox (2006a: 78)
}. La seconde strophe qui nous intéresse s'ins\`ere dans le r\'ecit de Ca\d n\d de\'sa\index{gnl}{Candesa@Ca\d n\d de\'sa} dont l'introduction d\'ecrit la ville de C\=ey\~nal\=ur\index{gnl}{Ceynalur@C\=ey\~nal\=ur} qui est digne d'\^etre le lieu de couronnement de la lign\'ee du roi\index{gnl}{roi} Anap\=aya\b n\index{gnl}{Anapayan@Anap\=aya\b n}, \`a nouveau associ\'e \`a Citamparam\index{gnl}{Citamparam}. La strophe ne mentionne pas explicitement qu'il a couvert d'or un des toits de ce temple\index{gnl}{temple}:

\scriptsize
\begin{verse}
\textit{ce\b n\b ni, yApaya\b n\index{gnl}{Apayan@Apaya\b n}, kul\=ottu\.nkac c\=o\b la\b n, \b rillait tiruvellai\\
po\b n\b ni\b n mayam\=ak kiyava\d lavar p\=or\=e, \b re\b n\b rum puvik\=akku\\
ma\b n\b nar perum\=a \b nAnap\=aya\b n\index{gnl}{Anapayan@Anap\=aya\b n} varunto\b n marapi\b n mu\d tic\=u\d t\d tun\\
ta\b nmai nilavu patiyainti \b no\b n\b r\=ay vi\d la\.nkun takaittavv\=ur.} (1213)\\
\end{verse}
\normalsize
\begin{quote}
La ville brillante [de C\=ey\~nal\=ur\index{gnl}{Ceynalur@C\=ey\~nal\=ur} est] comme une des cinq villes permanentes dignes de la qualit\'e de couronner l'ancienne lign\'ee d'o\`u vient Anap\=aya\b n\index{gnl}{Anapayan@Anap\=aya\b n}, le seigneur des rois qui prot\`ege la terre\index{gnl}{terre}, et qu'on appelle aussi l'Apaya\b n\index{gnl}{Apayan@Apaya\b n} couronn\'e, le Kul\=ottu\.nkacc\=o\b la\b n, le héros\index{gnl}{heros@héros} capable qui a donn\'e la beauté de l'or au site sacr\'e de Tillai\index{gnl}{Citamparam!Tillai} (Citamparam\index{gnl}{Citamparam}).
\end{quote}

\normalsize \noindent Ainsi, seules les deux occurrences des st.~8 et 1213 peuvent nous permettre d'identifier ce roi\index{gnl}{roi} \`a un monarque historique qui a accompli les m\^emes faits. Les titres Apaya\b n\index{gnl}{Apayan@Apaya\b n} et Kul\=ottu\.nkacc\=o\b la\b n qui lui sont attribu\'es (st. 1213) et la dorure de Citamparam\index{gnl}{Citamparam} qu'\'evoquent les deux strophes semblent soutenir son identification comme Kulottu\.nga II\index{gnl}{Kulottu\.nga II}.

Les donn\'ees historiques confirment cette identification\footnote{Cf. ARE 1912 para. 27. Selon \textsc{Nilakanta Sastri} (*2000 [1955]: 349) \textit{Anap\=aya\b n\index{gnl}{Anapayan@Anap\=aya\b n}} est un titre de Kulottu\.nga II\index{gnl}{Kulottu\.nga II}.}. Quelques \'epigraphes mentionnent des villages (Anap\=ayanall\=ur dans les ARE 1921 533, 1915 271 et 1911 363) et des officier\index{gnl}{officier}s royaux (Anap\=ayam\=uventav\=e\d l\=a\b n dans les ARE 1911 346 et 359) nomm\'es d'apr\`es ce titre. Bien que certaines inscriptions (ARE 1911 346, 359 et 363 du Ce\.nka\b rpa\d t\d tu dt.\index{gnl}{Ce\.nka\b rpa\d t\d tu dt.}) soient dat\'ees par \textsc{Mahalingam} (1989) du r\`egne de Kulottu\.nga II\index{gnl}{Kulottu\.nga II} sans justification, d'autres nous donnent plus de pr\'ecision sur leur datation: ARE 1915 271 (Va\d t\=a\b rk\=a\d tu dt.\index{gnl}{Va\d t\=a\b rk\=a\d tu dt.}) date d'un Kulottu\.nga et ARE 1921 533 (Te\b n\b n\=a\b rk\=a\d tu dt.\index{gnl}{Te\b n\b n\=a\b rk\=a\d tu dt.}) contient l'éloge\index{gnl}{eloge@éloge} royal \textit{p\=um\=evi va\d lar} attribu\'e \`a Kulottu\.nga II\index{gnl}{Kulottu\.nga II}\footnote{Sur les diff\'erentes \textit{meykk\=\i rtti}\index{gnl}{meykkirtti@\textit{meykk\=\i rtti}} de Kulottu\.nga II\index{gnl}{Kulottu\.nga II}, cf. \textsc{Cuppirama\d niyam} (1983: 121-131).}. Une inscription de \=Ar\=ur\index{gnl}{Ar\=ur@\=Ar\=ur} (SII 7 485) commen\c cant par \textit{p\=uma\b n\b nu patumam}, \textit{meykk\=\i rtti}\index{gnl}{meykkirtti@\textit{meykk\=\i rtti}} \'egalement compos\'ee \`a la gloire de ce souverain, pr\'ecise l.~32, dans la partie sanskrite, que le roi\index{gnl}{roi} est Anap\=aya (\textit{devo'nap\=ayo n\textsubring{r}pa\d h}). Ainsi, en contexte \'epigraphique, Anap\=aya\b n\index{gnl}{Anapayan@Anap\=aya\b n} semble \^etre un titre exclusif de Kulottu\.nga II\index{gnl}{Kulottu\.nga II}.

La litt\'erature de cour t\'emoigne aussi que le titre Anap\=aya\b n\index{gnl}{Anapayan@Anap\=aya\b n} est attribu\'e \`a Kulottu\.nga II\index{gnl}{Kulottu\.nga II}. Dans le \textit{Kul\=ottu\.nkacc\=o\b la\b nul\=a} compos\'e par O\d t\d takk\=uttar au \textsc{xii}\up{e} si\`ecle (\textsc{Zvelebil} 1995: 502-504) et d\'edi\'e \`a la c\'el\'ebration de Kulottu\.nga II\index{gnl}{Kulottu\.nga II}, nous retrouvons deux occurrences du titre Anap\=aya\b n\index{gnl}{Anapayan@Anap\=aya\b n} (st. 159 et 315) qui d\'esignent ce roi\index{gnl}{roi}.

Enfin, l'analyse de la frise narrative\index{gnl}{frise narrative} du temple\index{gnl}{temple} de T\=ar\=acuram\index{gnl}{Taracuram@T\=ar\=acuram} repr\'esentant les \'episodes des \textit{n\=aya\b nm\=ar}\index{gnl}{nayanmar@\textit{n\=aya\b nm\=ar}} conduit \textsc{Marr} (1979) \`a la conclusion que les panneaux forment une illustration fid\`ele du \textit{Periyapur\=a\d nam}\index{gnl}{Periyapuranam@\textit{Periyapur\=a\d nam}}. Parce que T\=ar\=acuram\index{gnl}{Taracuram@T\=ar\=acuram} a été construit sous le r\`egne de R\=ajar\=aja II\index{gnl}{Rajaraja II@R\=ajar\=aja II} (1150-1173), l'auteur propose de dater le \textit{Periyapur\=a\d nam}\index{gnl}{Periyapuranam@\textit{Periyapur\=a\d nam}} sous le r\`egne de Kulottu\.nga II\index{gnl}{Kulottu\.nga II} (1133-1150). \textsc{L'Hernault} (1987: 96-107) remet en question cette fid\'elit\'e \`a cause de l'existence d'une frise ant\'erieure, plus ou moins similaire, \`a M\=elakka\d tamp\=ur\index{gnl}{Melakkatampur@M\=elakka\d tamp\=ur}. Elle argue que les \'el\'ements des panneaux de T\=ar\=acuram\index{gnl}{Taracuram@T\=ar\=acuram} n'ont pas forc\'ement leur origine\index{gnl}{origine} dans le \textit{Periyapur\=a\d nam}\index{gnl}{Periyapuranam@\textit{Periyapur\=a\d nam}} et sugg\`ere que les deux frises auraient \'et\'e influenc\'ees par le \textit{Tirutto\d n\d tar tiruvant\=ati} attribu\'e \`a Nampi \=A\d n\d t\=ar Nampi\index{gnl}{Nampi \=A\d n\d t\=ar Nampi} et par d'autres versions orales qui auraient \'et\'e en circulation \`a cette \'epoque. \textsc{Gros} (2001: 25) critique cette id\'ee car, pour reprendre ses mots, \og la fid\'elit\'e quasi servile de C\=ekki\b l\=ar\index{gnl}{Cekkilar@C\=ekki\b l\=ar} \`a la lettre de l'\textit{Ant\=ati} rend futile la controverse sur Darasuram o\`u rien n'existe qui contredise le \textit{Periya Pur\=a\d nam}\fg. L'observation des image\index{gnl}{image}s des deux temples\index{gnl}{temple} et la lecture du \textit{Periyapur\=a\d nam}\index{gnl}{Periyapuranam@\textit{Periyapur\=a\d nam}} nous permettent d'apporter un argument suppl\'ementaire qui soutient l'hypoth\`ese de \textsc{Marr}: contrairement aux panneaux du temple\index{gnl}{temple} de M\=elakka\d tamp\=ur\index{gnl}{Melakkatampur@M\=elakka\d tamp\=ur} ceux de T\=ar\=acuram\index{gnl}{Taracuram@T\=ar\=acuram} repr\'esentent parfois le couple de \'Siva\index{gnl}{Siva@\'Siva} et P\=arvat\=\i\index{gnl}{Parvati@P\=arvat\=\i}\ mont\'es sur le taureau\index{gnl}{taureau}\footnote{Voir dans \textsc{L'Hernault} (1987) les Ph. 73 (fig. 4, 5, 7, 9, 12, 14, 17, 20, 28, 36, 38, 41 et 43) et 74 (fig. 44 et 57).}. Cette représentation iconographique correspond \`a la description litt\'eraire du \textit{deus ex machina} de nombreuses hagiographie\index{gnl}{hagiographie}s: \'Siva\index{gnl}{Siva@\'Siva} et P\=arvat\=\i\index{gnl}{Parvati@P\=arvat\=\i}\ apparaissent mont\'es sur le taureau\index{gnl}{taureau} \`a la fin de l'\'episode pour sauver ou bénir le \textit{n\=aya\b n\=ar}\index{gnl}{nayanmar@\textit{n\=aya\b nm\=ar}!\textit{n\=aya\b n\=ar}}. Par exemple, la situation finale de l'hagiographie\index{gnl}{hagiographie} d'I\d laiy\=a\b nku\d tim\=ara\b n\index{gnl}{Ilaiyankuti@I\d laiy\=a\b nku\d ti M\=ara\b n} est d\'ecrite ainsi:

\scriptsize
\begin{verse}
\textit{m\=alaya\b r kariya n\=ata\b n va\d tivoru c\=oti y\=akac\\
c\=alav\=e maya\.nku v\=arkkuc ca\.nkara\b n \b r\=a\b nma ki\b lnt\=e\\
y\=elav\=ar ku\b lal\=a \d ta\b n\b n\=o \d ti\d tapav\=a ka\b na\b n \=ayt t\=o\b n\b ric\\
c\=\i lam\=ar p\=ucai ceyta tirutto\d n\d tar tammai n\=okki} (\textit{PP} 464)\\

\textit{a\b npa\b n\=e ya\b npar p\=ucai ya\d littan\=\i\ ya\d na\.nki \b n\=o\d tum\\
e\b nperu mulaka meyti yirunitik ki\b lava\b n \b r\=a\b n\=e\\
mu\b nperu nitiya m\=enti mo\b liva\b li y\=eval k\=e\d tpa\\
vi\b npam\=arn tirukka ve\b n\b r\=e yaru\d lceyt\=a \b nevarkku mikk\=a\b n.} (\textit{PP} 465)\\
\end{verse}

\normalsize

\begin{verse}
Le Seigneur difficile [\`a trouver] par M\=al\index{gnl}{Visnu@Vi\d s\d nu!Mal@M\=al} et Aya\b n\index{gnl}{Brahma@Brahm\=a!Aya\b n}\\
Prit la forme d'une lumi\`ere\\
\'Sa\.nkara se r\'ejouissant lui-m\^eme pour [les deux] confus extr\^emement,\\
Apparut tel Celui \`a la monture de taureau\index{gnl}{taureau}\\
Avec Celle \`a la chevelure pleinement parfum\'ee;\\
Il regarda les saints serviteur\index{gnl}{serviteur}s qui [lui] avaient rendu\\
Des culte\index{gnl}{culte}s [hospitaliers] parfaitement dignes, (\textit{PP} 464)\\
\end{verse}
\begin{verse}
\og \^O dévot\index{gnl}{devot(e)@dévot(e)}! toi qui as rendu un culte\index{gnl}{culte} au dévot\index{gnl}{devot(e)@dévot(e)}, avec ton \'epouse,\\
Atteins mon grand monde, du Possesseur des deux tr\'esors\\
Re\c cois la grande richesse de jadis, \'ecoute les directives des textes,\\
Sois pleinement joyeux\fg\ parlant ainsi,\\
Celui [qui est] sup\'erieur \`a tous accorda sa gr\^ace. (\textit{PP} 465)\\
\end{verse}

Le couple divin n'appara\^it pas sur leur monture de taureau dans les panneaux de M\=elakka\d tamp\=ur\index{gnl}{Melakkatampur@M\=elakka\d tamp\=ur} ni dans l'\textit{Ant\=ati} attribuée à Nampi \=A\d n\d t\=ar Nampi\index{gnl}{Nampi \=A\d n\d t\=ar Nampi}. Seuls les descriptions du \textit{Periyapur\=a\d nam}\index{gnl}{Periyapuranam@\textit{Periyapur\=a\d nam}}\footnote{Pour d'autres apparitions du couple divin sur le taureau\index{gnl}{taureau} voir les \textit{pur\=a\d na} de Nilaka\d n\d ta\b n (\textit{PP} 399), Iya\b rpakai (\textit{PP} 434), M\=a\b nakka\~nc\=a\b rar (\textit{PP} 896), \=A\b n\=ayar (\textit{PP} 963-965), Kuripputto\d n\d tar (\textit{PP} 1203), Ca\d n\d ti\index{gnl}{Candesa@Ca\d n\d de\'sa!Ca\d n\d ti} (\textit{PP} 1257), Campantar\index{gnl}{Campantar} (\textit{PP} 1962) et C\=akkiya\b n (\textit{PP} 3651-3652). Parfois \'Siva\index{gnl}{Siva@\'Siva} intervient seul sur sa monture: T\=aya\b n (\textit{PP} 923), Atipattar (\textit{PP} 4009), Kaliya\b n (\textit{PP} 4037). Il peut \^etre sans taureau\index{gnl}{taureau}: avec sa par\`edre (Amarn\=\i ti \textit{PP} 547), seul (E\b n\=atin\=atar \textit{PP} 647 et K\=o\d tpuli \textit{PP} 4144). Enfin, une fois, \'Siva\index{gnl}{Siva@\'Siva} se manifeste sous la forme de Som\=askanda\index{gnl}{Somaskanda@Som\=askanda}, avec Um\=a\index{gnl}{Uma@Um\=a} et Skanda\index{gnl}{Skanda}, devant le dévot\index{gnl}{devot(e)@dévot(e)} Ci\b rutto\d n\d tar\index{gnl}{Ciruttontar@Ci\b rutto\d n\d tar} (\textit{PP} 3743-3744).} se superposent donc parfaitement sur les panneaux de T\=ar\=acuram\index{gnl}{Taracuram@T\=ar\=acuram}.

Ainsi, compte tenu des donn\'ees litt\'eraires, \'epigraphiques et iconographiques dont nous disposons il est difficile d'identifier le roi\index{gnl}{roi} Anap\=aya\b n\index{gnl}{Anapayan@Anap\=aya\b n} \`a un autre souverain \textit{c\=o\b la}\index{gnl}{cola@\textit{c\=o\b la}} que Kulottu\.nga II\index{gnl}{Kulottu\.nga II}. Il nous para\^it donc probable que le \textit{Periyapur\=a\d nam}\index{gnl}{Periyapuranam@\textit{Periyapur\=a\d nam}} date de la seconde moiti\'e du \textsc{xii}\up{e} si\`ecle.
% Cependant, les \'el\'ements exploit\'es dans cette sous-partie sont limit\'es; une \'etude approfondie est n\'ecessaire pour pouvoir confirmer ou discuter nos analyses.\\

Mis \`a part les informations donn\'ees par le texte l\'egendaire du \textit{C\=ekki\b l\=arpur\=a\d nam}\index{gnl}{Cekkilarpuranam@\textit{C\=ekki\b l\=arpur\=a\d nam}}, nous avons peu de renseignements sur l'auteur du \textit{Periyapur\=a\d nam}\index{gnl}{Periyapuranam@\textit{Periyapur\=a\d nam}}, si C\=ekki\b l\=ar\index{gnl}{Cekkilar@C\=ekki\b l\=ar} est bien le compositeur de cette hagiographie\index{gnl}{hagiographie}.
\textsc{Ir\=acam\=a\d nikka\b n\=ar} (*1996 [1968]: 16-18) souhaite reconna\^itre la famille du poète\index{gnl}{poete@poète} dans les inscriptions mentionnant des hommes du \og clan\fg\ C\=ekki\b l\=ar\index{gnl}{Cekkilar@C\=ekki\b l\=ar} de Ku\b n\b ratt\=ur\index{gnl}{Kunrattur@Ku\b n\b ratt\=ur}, conform\'ement \`a la légende\index{gnl}{legende@légende} qui nous informe que les proches du poète\index{gnl}{poete@poète} \'etaient actifs dans cette r\'egion. Il donne ainsi une liste de neuf \'epigraphes sans, malheureusement, fournir les r\'ef\'erences des relev\'es. Nous en avons retrouv\'e cinq. Trois inscriptions du temple\index{gnl}{temple} de N\=ak\=ecuram de Ku\b n\b ratt\=ur\index{gnl}{Kunrattur@Ku\b n\b ratt\=ur} (\'Sr\=\i perumput\=ur tk.)\footnote{ARE 1929-30 230, 218 et 208 dat\'ees respectivement de 1182, 1241 et 1268 par \textsc{Mahalingam} (1989: 438-446).}, qui a \'et\'e parrain\'e par l'auteur du \textit{Periyapur\=a\d nam}\index{gnl}{Periyapuranam@\textit{Periyapur\=a\d nam}} selon la légende\index{gnl}{legende@légende}, et une du temple\index{gnl}{temple} de Tirupp\=alaiva\b nam\index{gnl}{Tiruppalaivanam@Tirupp\=alaiva\b nam} (ARE 1928-29 314 datant de 1226), toutes du district de Ce\.nka\b rpa\d t\d tu\index{gnl}{Ce\.nka\b rpa\d t\d tu dt.}, enregistrent des donations d'individus portant le nom de C\=ekki\b l\=a\b n\index{gnl}{Cekkilan@C\=ekki\b l\=a\b n} et originaires de Ku\b n\b ratt\=ur\index{gnl}{Kunrattur@Ku\b n\b ratt\=ur}. La datation de deux \'epigraphes que l'auteur pr\'esente nous semble erron\'ee: les dates de 1164 et 1179 donn\'ees pour ces inscriptions \'evoquant un certain C\=ekki\b l\=a\b n\index{gnl}{Cekkilan@C\=ekki\b l\=a\b n} P\=ala\b r\=av\=aya\b n\index{gnl}{Palaravayan@P\=ala\b r\=av\=aya\b n} Ka\d lapp\=a\d lar\=aya\b n\index{gnl}{Kalappalarayan@Ka\d lapp\=a\d lar\=aya\b n}, que \textsc{Ir\=acam\=a\d nikka\b n\=ar} associe sans fondement \`a Ku\b n\b ratt\=ur\index{gnl}{Kunrattur@Ku\b n\b ratt\=ur}, ne concordent pas avec celle de l'ARE 1928-29 221 (Virutt\=acalam\index{gnl}{Viruttacalam@Virutt\=acalam} tk., Te\b n\b n\=a\b rk\=a\d tu dt.\index{gnl}{Te\b n\b n\=a\b rk\=a\d tu dt.}) o\`u figure le m\^eme homme et qui date de 1235 selon \textsc{Mahalingam} (1988: 512). Nous pensons que \textsc{Ir\=acam\=a\d nikka\b n\=ar} a \'et\'e influenc\'e par la st. 98 du \textit{C\=ekki\b l\=arpur\=a\d nam}\index{gnl}{Cekkilarpuranam@\textit{C\=ekki\b l\=arpur\=a\d nam}} qui nomme P\=ala\b r\=av\=aya\b n\index{gnl}{Palaravayan@P\=ala\b r\=av\=aya\b n} le fr\`ere successeur de C\=ekki\b l\=ar\index{gnl}{Cekkilar@C\=ekki\b l\=ar}, et qu'il a voulu \`a tout prix confondre le personnage littéraire P\=ala\b r\=av\=aya\b n avec cet individu de l'inscription (C\=ekki\b l\=a\b n P\=ala\b r\=av\=aya\b n Ka\d lapp\=a\d lar\=aya\b n) qui n'est pas originaire de Ku\b n\b ratt\=ur\index{gnl}{Kunrattur@Ku\b n\b ratt\=ur}.
Une inscription \'evoquerait notre hagiographe selon \textsc{Ir\=acam\=a\d nikka\b n\=ar}. Le r\'esum\'e que donne l'ARE 1920 95, provenant du temple\index{gnl}{temple} de Tiruma\b lap\=a\d ti\index{gnl}{Tirumalapati@Tiruma\b lap\=a\d ti} (U\d taiy\=arp\=a\d laiyam tk., Tirucci\index{gnl}{Tirucci dt.} dt.) et datant de la dix-septi\`eme ann\'ee de r\`egne d'un R\=ajar\=aja, mentionne une donation de quatre-vingt-dix moutons pour l'entretien d'une lampe\index{gnl}{lampe} perp\'etuelle par un individu appelé \og Ku\b n\b ratt\=ur\index{gnl}{Kunrattur@Ku\b n\b ratt\=ur} \'S\=ekki\b l\=a\b n M\=ad\=eva\d diga\d l R\=amad\=eva alias Uttamach\=o\d la Pallavaraya\b n\fg. Dans le \textit{C\=ekki\b l\=arpur\=a\d nam}\index{gnl}{Cekkilarpuranam@\textit{C\=ekki\b l\=arpur\=a\d nam}} (st. 18), C\=ekki\b l\=ar\index{gnl}{Cekkilar@C\=ekki\b l\=ar} porte le titre Uttamac\=o\b lappallava\b n\index{gnl}{Uttamacolapallavan@Uttamac\=o\b lappallava\b n}. \textsc{Ir\=acam\=a\d nikka\b n\=ar} et quelques autres auteurs, comme \textsc{Cox} (2006b: 7), en concluent que ce donateur est notre poète\index{gnl}{poete@poète}. Cependant, l'\'epigraphe n'est pas publi\'ee et sa datation est incertaine. Ni l'ARE ni \textsc{Mahalingam} (1991b: 388) ne proposent une identification du roi\index{gnl}{roi}. Il nous semble que, dans l'\'etat actuel de nos connaissance\index{gnl}{connaissance}s, nous ne pouvons pas soutenir l'identification de ce donateur comme l'auteur du \textit{Periyapur\=a\d nam}\index{gnl}{Periyapuranam@\textit{Periyapur\=a\d nam}} sur la seule autorit\'e du \textit{C\=ekki\b l\=arpur\=a\d nam}\index{gnl}{Cekkilarpuranam@\textit{C\=ekki\b l\=arpur\=a\d nam}}.
Deux inscriptions not\'ees par \textsc{Ir\=acam\=a\d nikka\b n\=ar} demeurent introuvables. Ainsi, les ARE \og retrouv\'es\fg\ et datables placent ces différents C\=ekki\b l\=ar\index{gnl}{Cekkilar@C\=ekki\b l\=ar} de Ku\b n\b ratt\=ur\index{gnl}{Kunrattur@Ku\b n\b ratt\=ur} principalement au \textsc{xiii}\up{e} si\`ecle.

Par ailleurs, la concordance des noms des inscriptions \textit{c\=o\b la}\index{gnl}{cola@\textit{c\=o\b la}} (\textsc{Karashima, Subbarayalu, Matsui} 1978), recense six autres \'epigraphes publi\'ees,
%\footnote{Nous n'approuvons pas la remarque de Cox 2006b: 7 qui stipule que les auteurs da la concordance ont interpr\'et\'e le segment `C\=ekki\b l\=a\b n\index{gnl}{Cekkilan@C\=ekki\b l\=a\b n}' comme \'etant un toponyme. A aucun endroit ces aureurs ne font \'etat de cela. De plus, le chiffre `1' de la seconde colonne signifie la position du segment tel qu'il appara\^it dans le nom.}
 datant du \textsc{xi}\up{e} et \textsc{xii}\up{e} si\`ecles et, ce faisant, ant\'erieures \`a celles discut\'ees plus haut, et mentionnant des C\=ekki\b l\=a\b n\index{gnl}{Cekkilan@C\=ekki\b l\=a\b n}. Si deux d'entre elles enregistrent des donateurs originaires du district de Ce\.nka\b rpa\d t\d tu\index{gnl}{Ce\.nka\b rpa\d t\d tu dt.} (SII 5 473 et 7 476), les autres viennent de l'ancien district de Ta\~nc\=av\=ur\index{gnl}{Tancavur@Ta\~nc\=av\=ur} (SII 19 78, 8 226 et 220) et du taluk de Citamparam\index{gnl}{Citamparam} (SII 13 146). Il nous appara\^it donc que le \og clan\fg\ des C\=ekki\b l\=ar\index{gnl}{Cekkilar@C\=ekki\b l\=ar} n'est pas uniquement enracin\'e dans la r\'egion de Ce\.nka\b rpa\d t\d tu\index{gnl}{Ce\.nka\b rpa\d t\d tu dt.}, et plus particuli\`erement de Ku\b n\b ratt\=ur\index{gnl}{Kunrattur@Ku\b n\b ratt\=ur}, comme le suppose \textsc{Ir\=acam\=a\d nikka\b n\=ar} mais qu'un nombre substantiel de hauts officier\index{gnl}{officier}s ou de propri\'etaires terriens poss\'edant ce nom vivaient aussi dans la plaine delta\"ique. De plus, relier tous les C\=ekki\b l\=ar\index{gnl}{Cekkilar@C\=ekki\b l\=ar} de Ku\b n\b ratt\=ur\index{gnl}{Kunrattur@Ku\b n\b ratt\=ur} \`a la famille du poète\index{gnl}{poete@poète} sur l'unique autorit\'e d'un texte l\'egendaire nous semble contestable.

Dans l'\'etat actuel des recherches, il nous semble que deux \'epigraphes, non publi\'ees, peuvent faire allusion \`a l'auteur du \textit{Periyapur\=a\d nam}\index{gnl}{Periyapuranam@\textit{Periyapur\=a\d nam}}\footnote{L'identification du texte \textit{\=A\d lu\d taiya Nampi \'Sr\=\i pur\=a\d nam} de SII 5 1358 avec le \textit{Periyapur\=a\d nam}\index{gnl}{Periyapuranam@\textit{Periyapur\=a\d nam}} par \textsc{Rajamanickam} (1964: 211-213) reste, \`a notre avis, discutable.}: ARE 1958-59 313 et ARE 1938-39 229. La premi\`ere provient de Citamparam\index{gnl}{Citamparam}. Le r\'esum\'e de cette inscription nous informe qu'elle contient l'éloge\index{gnl}{eloge@éloge} royal \textit{puyal v\=ayttu va\d lam peruka} attribu\'e \`a Kulottu\.nga III\index{gnl}{Kulottu\.nga III} (1178-1218), qu'elle date de la huiti\`eme ann\'ee de r\`egne de ce roi, 1186, et enfin, qu'elle enregistre un ordre\index{gnl}{ordre royal} royal qui d\'etaxe quelques terres\index{gnl}{terre} donn\'ees par un certain \og \'S\=ekki\b l\=a\b n Araiya\b n Edirili\'so\b la\b n\fg\ de Ku\b n\b ratt\=ur\index{gnl}{Kunrattur@Ku\b n\b ratt\=ur} pour former un jardin nomm\'e \textit{tirutto\d n\d tar c\=\i ruraitt\=ar}\footnote{\textsc{Cox} (2006b: 7) classe cette inscription parmi celles qui attestent l'existence d'un clan C\=ekki\b la\b n mais ne rel\`eve pas le nom du jardin, et par cons\'equence, son importance.}. Le nom appelatif \textit{tirutto\d n\d tar c\=\i ruraitt\=ar}, \og Celui qui a racont\'e la gloire des saints serviteur\index{gnl}{serviteur}s\fg, peut \^etre une d\'esignation de Cuntarar\index{gnl}{Cuntarar}, de Nampi \=A\d n\d t\=ar Nampi\index{gnl}{Nampi \=A\d n\d t\=ar Nampi} ou de C\=ekki\b l\=ar\index{gnl}{Cekkilar@C\=ekki\b l\=ar}. Mais, le titre et l'origine\index{gnl}{origine} g\'eographique du donateur nous permettent de supposer que ce nom appellatif renvoie \`a C\=ekki\b l\=ar\index{gnl}{Cekkilar@C\=ekki\b l\=ar} le poète\index{gnl}{poete@poète}\footnote{Par ailleurs, \`a la st. 95 du \textit{C\=ekki\b l\=arpur\=a\d nam}\index{gnl}{Cekkilarpuranam@\textit{C\=ekki\b l\=arpur\=a\d nam}}, C\=ekki\b l\=ar\index{gnl}{Cekkilar@C\=ekki\b l\=ar} l'hagiographe y re\c coit du roi\index{gnl}{roi} le titre de \textit{to\d n\d ta c\=\i rparavuv\=ar}, \og celui qui r\'epand la gloire des serviteur\index{gnl}{serviteur}s\fg.} ou \`a un de ses descendants.
La seconde date de la vingt-cinqui\`eme ann\'ee de r\`egne de Kulottu\.nga III\index{gnl}{Kulottu\.nga III}, soit de 1203, et se trouve \`a \'Sr\=\i v\=a\~nciyam\index{gnl}{Srivanciyam@\'Sr\=\i v\=a\~nciyam} (Na\b n\b nilam tk.\index{gnl}{Nannilam@Na\b n\b nilam tk.}, Ta\~nc\=av\=ur\index{gnl}{Tancavur@Ta\~nc\=av\=ur} dt.). Selon le r\'esum\'e de l'ARE, elle relate une donation pour maintenir le culte\index{gnl}{culte} et les offrandes faits aux trois image\index{gnl}{image}s d'\textit{Emberumakka\d l} (litt\'eralement, \og nos \^etres chers\fg, identifi\'es comme les \textit{m\=uvar}\index{gnl}{muvar@\textit{m\=uvar}}) et \`a celle de \textit{Tirutto\d n\d dar S\=\i ruraitt\=ar}. Alors que l'ARE identifie ce dernier comme M\=a\d nikkav\=acakar\index{gnl}{Manikkavacakar@M\=a\d nikkav\=acakar}, \textsc{Nagaswamy} (1989: 227), dans un paragraphe confus, qui nous semble-t-il, r\'efute (sans la mentionner) l'identification propos\'ee par l'ARE, pense qu'il s'agit de C\=ekki\b l\=ar\index{gnl}{Cekkilar@C\=ekki\b l\=ar}. L'interpr\'etation de l'ARE nous para\^it mauvaise car les \oe uvres attribu\'ees \`a M\=a\d nikkav\=acakar\index{gnl}{Manikkavacakar@M\=a\d nikkav\=acakar} ne permettent pas de le qualifier de \og Celui qui a racont\'e la gloire des saints serviteur\index{gnl}{serviteur}s\fg.
Cependant, aucune information intrins\`eque du r\'esum\'e concernant le donateur, un certain Anap\=aya\b n\index{gnl}{Anapayan@Anap\=aya\b n}, ou l'emplacement des image\index{gnl}{image}s, dans la chapelle de la d\'eesse\index{gnl}{deesse@déesse} du temple\index{gnl}{temple} de Tiruma\d nakk\=oyilu\d taiy\=ar \`a C\=ev\=ur, ne permet d'établir formellement une identification. Il faudrait \'etudier en d\'etail ces inscriptions inédites pour confirmer notre hypoth\`ese.
\vspace*{0.5cm}

\begin{center}
*
\end{center}

Nous avons vu, tout au long de ce chapitre introductif aux textes de la mise en légende\index{gnl}{legende@légende} de C\=\i k\=a\b li\index{gnl}{Cikali@C\=\i k\=a\b li} et de Campantar\index{gnl}{Campantar}, que l'histoire de nombreux textes du \textit{Tirumu\b rai}\index{gnl}{Tirumurai@\textit{Tirumu\b rai}} reste incertaine malgr\'e une bibliographie abondante et quelques nouvelles donn\'ees \'epigraphiques. Ces \oe uvres sont nimb\'ees de légende\index{gnl}{legende@légende}s tellement influentes que, malheureusement, la plus grande partie de la litt\'erature secondaire repose sur ces derni\`eres pour \'etablir la chronologie des poète\index{gnl}{poete@poète}s et des textes qui leur sont attribu\'es. Tr\`es peu de chercheurs remettent en cause par exemple la paternit\'e de certains textes fond\'ee sur les informations de ces r\'ecits mythologiques souverains.
% Ainsi, c'est principalement sur l'autorit\'e du \textit{Periyapur\=a\d nam}\index{gnl}{Periyapuranam@\textit{Periyapur\=a\d nam}} que les textes du \textit{T\=ev\=aram}\index{gnl}{Tevaram@\textit{T\=ev\=aram}} sont rattach\'es \`a un trio d'auteurs\footnote{Nous n'oublions pas les envois\index{gnl}{envoi} (\textit{tirukka\d taikk\=appu}\index{gnl}{tirukkataikkappu@\textit{tirukka\d taikk\=appu}}) des poème\index{gnl}{poeme@poème}s du \textit{T\=ev\=aram}\index{gnl}{Tevaram@\textit{T\=ev\=aram}} qui portent souvent la griffe des hymnistes. Cependant, nous consid\'erons l'hypoth\`ese que beaucoup d'entre eux soient des ajouts post\'erieurs, cf. chapitre 2.} et que ces derniers sont dat\'es.
Le \textit{Tirumu\b raika\d n\d tapur\=a\d nam}\index{gnl}{Tirumuraikantapuranam@\textit{Tirumu\b raika\d n\d tapur\=a\d nam}} et le \textit{C\=ekki\b l\=arpur\=a\d nam}\index{gnl}{Cekkilarpuranam@\textit{C\=ekki\b l\=arpur\=a\d nam}} racontent respectivement la compilation\index{gnl}{compilation} des onze premiers livres du canon\index{gnl}{canon} shiva\"ite\index{gnl}{shiva\"ite} par Nampi\index{gnl}{Nampi \=A\d n\d t\=ar Nampi} et la composition du douzi\`eme livre par C\=ekki\b l\=ar\index{gnl}{Cekkilar@C\=ekki\b l\=ar}. Ces légende\index{gnl}{legende@légende}s ont \'et\'e essentiellement exploit\'ees pour rattacher tel texte \`a tel auteur et pour les dater.

\`A travers les textes du \textit{Tirumu\b rai}\index{gnl}{Tirumurai@\textit{Tirumu\b rai}} pr\'esent\'es, nous voyons se dessiner deux phases de mise en légende\index{gnl}{legende@légende} qui ont permis de revaloriser et reg\'en\'erer les textes de la \textit{bhakti}\index{gnl}{bhakti@\textit{bhakti}} shiva\"ite\index{gnl}{shiva\"ite} tamoule. La premi\`ere, celle qui concerne notre \'etude, a eu lieu avec le \textit{Periyapur\=a\d nam}\index{gnl}{Periyapuranam@\textit{Periyapur\=a\d nam}} qui consacre une grande portion de son \oe uvre aux \textit{m\=uvar}\index{gnl}{muvar@\textit{m\=uvar}} (plus du quart \`a Campantar\index{gnl}{Campantar}). Rappelons seulement ici que les textes de \textit{bhakti}\index{gnl}{bhakti@\textit{bhakti}} shiva\"ite\index{gnl}{shiva\"ite} ont connu, pour des raisons obscures, un certain \og abandon\fg\ ou \og oubli\fg\ \`a la p\'eriode \textit{c\=o\b la}\index{gnl}{cola@\textit{c\=o\b la}} dans quelques temples\index{gnl}{temple} comme l'attestent les deux inscriptions de V\=\i \b limi\b lalai\index{gnl}{Vilimilalai@V\=\i \b limi\b lalai} (ARE 1908 414) et, de mani\`ere plus surprenante, de C\=\i k\=a\b li\index{gnl}{Cikali@C\=\i k\=a\b li} (CEC 26). Ce dernier temple\index{gnl}{temple} est fortement associ\'e au poète\index{gnl}{poete@poète} Campantar\index{gnl}{Campantar} qui appara\^it aujourd'hui comme le meneur de la \textit{bhakti}\index{gnl}{bhakti@\textit{bhakti}} shiva\"ite\index{gnl}{shiva\"ite} tamoule. Comment expliquer que le temple\index{gnl}{temple} de Campantar\index{gnl}{Campantar} s'est retrouv\'e dans cette situation au \textsc{xii}\up{e} si\`ecle? De plus, il nous semble int\'eressant de souligner que les \oe uvres attribu\'ees \`a Nampi\index{gnl}{Nampi \=A\d n\d t\=ar Nampi} et \`a C\=ekki\b l\=ar\index{gnl}{Cekkilar@C\=ekki\b l\=ar}, forgeant le mythe\index{gnl}{mythe} de ce personnage, \'emergent pr\'ecis\'ement dans ce contexte historique. Y a-t-il un lien de causalit\'e entre la renaissance du \textit{tirukkaikk\=o\d t\d ti}\index{gnl}{tirukkaikkotti@\textit{tirukkaikk\=o\d t\d ti}} de C\=\i k\=a\b li\index{gnl}{Cikali@C\=\i k\=a\b li} et la naissance\index{gnl}{naissance} de la légende\index{gnl}{legende@légende} fig\'ee de Campantar\index{gnl}{Campantar}?
La seconde phase concerne le \textit{Tirumu\b raika\d n\d tapur\=a\d nam}\index{gnl}{Tirumuraikantapuranam@\textit{Tirumu\b raika\d n\d tapur\=a\d nam}} et le \textit{C\=ekki\b l\=arpur\=a\d nam}\index{gnl}{Cekkilarpuranam@\textit{C\=ekki\b l\=arpur\=a\d nam}}. Bien que ces deux r\'ecits l\'egendaires soient eux-m\^emes sujets de controverses et que leur attribution \`a un seul Um\=apati\index{gnl}{Umapati@Um\=apati} Civ\=ac\=ariyar nous paraisse tr\`es improbable, ils semblent illustrer le besoin qu'a ressenti, aux \textsc{xiv}-\textsc{xvi}\up{e} si\`ecles, la soci\'et\'e shiva\"ite\index{gnl}{shiva\"ite} tamoule de l\'egitimer, de sanctifier et d'\'etablir un corpus\index{gnl}{corpus} canonique\index{gnl}{canon} d'hymne\index{gnl}{hymne}s tamouls bhaktiques; et ce, assur\'ement sous la direction incorporatrice du mouvement \'Saiva Siddh\=anta\index{gnl}{Saiva@\'Saiva Siddh\=anta} tamoul pr\'epond\'erant \`a Citamparam\index{gnl}{Citamparam} \`a partir du \textsc{xiv}\up{e} si\`ecle\footnote{Nous renvoyons \`a \textsc{Prentiss} (1996) en gardant toutefois des r\'eserves sur les identifications et les datations qu'elle suit. Pour une introduction au \'Saiva Siddh\=anta panindien, cf. le premier chapitre de \textsc{Davis} (*2000 [1991]); pour une mise au point sur l'étude de ce mouvement dans la littérature secondaire, cf. la préface de \textsc{Goodall} (2004) et enfin, pour une pr\'esentation g\'en\'erale des textes philosophico-religieux du \'Saiva Siddh\=anta\index{gnl}{Saiva@\'Saiva Siddh\=anta} tamoul, cf. \textsc{Zvelebil} (1975: 198-207).}.


Dans les deux chapitres suivants, nous essayons de comprendre comment la légende\index{gnl}{legende@légende} de Campantar\index{gnl}{Campantar} et celle de sa ville natale furent construites en examinant, dans la limite du possible, selon l'ordre\index{gnl}{ordre} chronologique \'etabli par la tradition\index{gnl}{tradition}, les textes du \textit{Tirumu\b rai}\index{gnl}{Tirumurai@\textit{Tirumu\b rai}} qui c\'el\`ebrent le poète.
Campantar\index{gnl}{Campantar} appartient en effet \`a la cat\'egorie des \textit{n\=aya\b nm\=ar}\index{gnl}{nayanmar@\textit{n\=aya\b nm\=ar}} poète\index{gnl}{poete@poète}s --- comprenant les \textit{m\=uvar}\index{gnl}{muvar@\textit{m\=uvar}}, K\=araikk\=alammaiy\=ar\index{gnl}{Karaikkalammaiyar@K\=araikk\=alammaiy\=ar}, Tirum\=ular\index{gnl}{Tirumular@Tirum\=ular} et le roi\index{gnl}{roi} C\=eram\=a\b n Perum\=a\d l\index{gnl}{Ceraman Perumal@C\=eram\=a\b n Perum\=a\d l} --- dont les \oe uvres, pass\'ees \`a la post\'erit\'e, ont \'et\'e exploit\'ees pour constituer leurs biographies sacr\'ees.
%Une fois les diff\'erents strates de l'\'elaboration mythique d\'efinis pour Campantar\index{gnl}{Campantar} puis C\=\i k\=a\b li\index{gnl}{Cikali@C\=\i k\=a\b li}, nous viendrons \`a consid\'erer les conditions historiques de cette mise en légende\index{gnl}{legende@légende} et les liens que l'on peut tisser entre l'Histoire et la légende\index{gnl}{legende@légende}.


\chapter{Aux origines d'un héros l\'egendaire}

Le \textit{Periyapur\=a\d nam}\index{gnl}{Periyapuranam@\textit{Periyapur\=a\d nam}}, ensemble des hagiographie\index{gnl}{hagiographie}s des soixante-trois \textit{n\=aya\b nm\=ar}, narre la vie exemplaire, sacr\'ee et l\'egendaire de serviteur\index{gnl}{serviteur}s qui incarnent une d\'evotion\index{gnl}{devotion@dévotion} extr\^eme envers \'Siva\index{gnl}{Siva@\'Siva}. Ce texte de r\'ef\'erence, de la seconde moitié du \textsc{xii}\up{e} si\`ecle, met en forme, int\'egralement et vraisemblablement pour la premi\`ere fois, en tamoul, le r\'ecit de vie du \textit{n\=aya\b n\=ar}\index{gnl}{nayanmar@\textit{n\=aya\b nm\=ar}!\textit{n\=aya\b n\=ar}} Campantar\index{gnl}{Campantar}, reconnu comme l'auteur des trois premiers livres du \textit{T\=ev\=aram}\index{gnl}{Tevaram@\textit{T\=ev\=aram}}. Campantar\index{gnl}{Campantar}, figure embl\'ematique du shiva\"isme\index{gnl}{shivaisme@shiva\"isme} tamoul, n'est per\c cu qu'\`a travers ce \textit{pur\=a\d nam} qui sert le plus souvent de base \`a son \'etude historique. En effet, la litt\'erature secondaire, qui essaie de pr\'esenter sa v\'eritable biographie\index{gnl}{biographie}, repose sur l'unique autorit\'e du \textit{Periyapur\=a\d nam}\index{gnl}{Periyapuranam@\textit{Periyapur\=a\d nam}} et identifie, dans le texte, les pr\'etendus contemporains du poète\index{gnl}{poete@poète} pour dater ce dernier (\textsc{Gros} 1984)\footnote{Cette m\'ethode est aussi appliqu\'ee pour \'etudier la datation des poète\index{gnl}{poete@poète}s vishnouite\index{gnl}{vishnouite}s tamouls; cf. \textsc{Hardy} (*2001 [1983]: 243-244 et n. 4) qui critique cette d\'emarche non scientifique.}.

La datation de Campantar\index{gnl}{Campantar} est fond\'ee principalement sur celle de trois figures religieuses b\'en\'eficiant chacune d'une hagiographie\index{gnl}{hagiographie} dans le \textit{Periyapur\=a\d nam}\index{gnl}{Periyapuranam@\textit{Periyapur\=a\d nam}}: le poète\index{gnl}{poete@poète} Appar\index{gnl}{Appar}, le roi\index{gnl}{roi} \textit{p\=a\d n\d dya}\index{gnl}{pandya@\textit{p\=a\d n\d dya}} Ne\d tum\=a\b ra\b n\index{gnl}{Netumaran@Ne\d tum\=a\b ra\b n} que Campantar\index{gnl}{Campantar} a converti, ainsi que le \textit{n\=aya\b n\=ar}\index{gnl}{nayanmar@\textit{n\=aya\b nm\=ar}!\textit{n\=aya\b n\=ar}} Ci\b rutto\d n\d tar\index{gnl}{Ciruttontar@Ci\b rutto\d n\d tar} qu'il a rencontr\'e \`a l'occasion d'un de ses p\`elerinage\index{gnl}{pelerinage@pèlerinage}s (st.~2366 et 2382).
Ainsi, \og sous l'influence hypnotique de C\=ekki\b l\=ar\index{gnl}{Cekkilar@C\=ekki\b l\=ar}\fg\footnote{Pour reprendre une formulation de \textsc{Gros} (1984: xiii).}, Appar\index{gnl}{Appar}, \`a qui sont attribu\'es les livres \textsc{iv} \`a \textsc{vi} du \textit{T\=ev\=aram}\index{gnl}{Tevaram@\textit{T\=ev\=aram}}, est un contemporain plus \^ag\'e de Campantar\index{gnl}{Campantar} qui, ayant abjur\'e le ja\"in\index{gnl}{jain@ja\"in}isme, a converti au shiva\"isme\index{gnl}{shivaisme@shiva\"isme} son pers\'ecuteur, un roi\index{gnl}{roi} \textit{pallava}\index{gnl}{pallava@\textit{pallava}}. Beaucoup de chercheurs ont identifi\'e ce dernier comme Mahendra I~(600-625). Aucun \'el\'ement n'est assez pr\'ecis et convainquant pour soutenir cette identification (\textsc{Francis} 2009: 437, n.607) et, par cons\'equent, la datation d'Appar\index{gnl}{Appar}.
Ensuite, d'apr\`es le \textit{pur\=a\d nam}, Campantar\index{gnl}{Campantar} a d\'etourn\'e du ja\"in\index{gnl}{jain@ja\"in}isme un roi\index{gnl}{roi} \textit{p\=a\d n\d dya}\index{gnl}{pandya@\textit{p\=a\d n\d dya}} identifi\'e comme M\=a\b ravarman Arikesari\index{gnl}{Maravarman@M\=a\b ravarman Arikesari}, souverain qui aurait régné dans la seconde moiti\'e du \textsc{vii}\up{e} si\`ecle. Aucune donn\'ee fiable ne vient soutenir, ici encore, ce synchronisme (\textsc{Swamy} 1975b: 129). Par ailleurs, n'est-il pas vain de chercher \`a identifier ces personnages royaux convertis au shiva\"isme\index{gnl}{shivaisme@shiva\"isme} qui illustrent la marche conqu\'erante des deux poète\index{gnl}{poete@poète}s sur diff\'erentes dynasties et contr\'ees du Pays Tamoul\index{gnl}{Pays Tamoul}.
Enfin, Ci\b rutto\d n\d tar\index{gnl}{Ciruttontar@Ci\b rutto\d n\d tar} est, dans le \textit{pur\=a\d nam}, un ancien guerrier, appel\'e aussi Para\~nc\=oti\index{gnl}{Parancoti@Para\~nc\=oti}, qui m\`ene une vie de parfait dévot\index{gnl}{devot(e)@dévot(e)} en offrant quotidiennement un repas \`a un shiva\"ite\index{gnl}{shiva\"ite}. Un jour, il est mis \`a l'épreuve\index{gnl}{epreuve@épreuve} par \'Siva\index{gnl}{Siva@\'Siva} d\'eguis\'e en asc\`ete bhairavique qui lui demande de servir son fils pour le repas. Ci\b rutto\d n\d tar\index{gnl}{Ciruttontar@Ci\b rutto\d n\d tar} ex\'ecute sa volont\'e, le fils est ressuscit\'e et toute la famille obtient, au final, la gr\^ace divine. Seules deux informations vague\index{gnl}{vague}s donn\'ees dans les dix premiers quatrains de l'hagiographie\index{gnl}{hagiographie} (st.~3660-3669) qui retracent sa vie de guerrier ont fond\'e l'identification qu'il a reçue: il porte le nom de Para\~nc\=oti\index{gnl}{Parancoti@Para\~nc\=oti} (st.~3661) et a mis \`a sac la cit\'e de V\=at\=api\index{gnl}{Vatapi@V\=at\=api} (st.~3665). Ainsi, Ci\b rutto\d n\d tar\index{gnl}{Ciruttontar@Ci\b rutto\d n\d tar} a été identifi\'e comme un g\'en\'eral \textit{pallava}\index{gnl}{pallava@\textit{pallava}}, nomm\'e Para\~nc\=oti\index{gnl}{Parancoti@Para\~nc\=oti}, vainqueur de la bataille de V\=at\=api\index{gnl}{Vatapi@V\=at\=api} en 642. Or, aucune source \textit{pallava}\index{gnl}{pallava@\textit{pallava}} ne donne le nom de ce g\'en\'eral (information communiqu\'ee par \textsc{E. Francis}, voir aussi \textsc{Francis} 2009: 443, n. 632) qui semble donc \^etre une pure cr\'eation du \textit{Periyapur\=a\d nam}\index{gnl}{Periyapuranam@\textit{Periyapur\=a\d nam}}. Par ailleurs, aucun \'el\'ement du \textit{pur\=a\d nam} ne pr\'ecise que le roi\index{gnl}{roi} que servait le guerrier-dévot\index{gnl}{devot(e)@dévot(e)} \'etait \textit{pallava}\index{gnl}{pallava@\textit{pallava}}. La capitale \textit{c\=a\d lukya}\index{gnl}{calukya@\textit{c\=a\d lukya}} a \'et\'e prise maintes fois (\textsc{Gros} 1984: xii-xiii). Enfin, les noms de Para\~nc\=oti\index{gnl}{Parancoti@Para\~nc\=oti} et de V\=at\=api\index{gnl}{Vatapi@V\=at\=api} n'ont pas \'et\'e retenus dans la version t\'elougoue de la légende\index{gnl}{legende@légende}\footnote{Dans le \textit{Basavapur\=a\d nu}, le serviteur\index{gnl}{serviteur}, nomm\'e Siriy\=ala, est simplement un marchand de K\=a\~ncipuram\index{gnl}{Kancipuram@K\=a\~ncipuram} (\textsc{Rao} 1990: 144-147).}.
Ainsi, ce triple synchronisme avec ces figures religieuses, attest\'e dans ce \textit{pur\=a\d nam} du \textsc{xii}\up{e} si\`ecle et nullement confirm\'e par d'autres sources historiques\footnote{Nous soulignons que les occurrences de \textit{ci\b rutto\d n\d tar}\index{gnl}{ciruttontar@\textit{ci\b rutto\d n\d tar}}, litt\'eralement \og petit serviteur\index{gnl}{serviteur}\fg, dans les hymne\index{gnl}{hymne}s attribu\'es \`a Campantar\index{gnl}{Campantar}, ne d\'esignent pas un individu particulier mais l'arch\'etype du dévot\index{gnl}{devot(e)@dévot(e)} humble (I~45.7; I~61.10; I~99.5; I~103.6; III~46.3 et III~63). Notre interpr\'etation est soutenue par le m\^eme usage qu'en fait Appar\index{gnl}{Appar} en IV 109.2.}, nous semble relever de la fiction narrative.

S'appuyant uniquement sur ces identifications douteuses fond\'ees sur l'autorit\'e du \textit{Periyapur\=a\d nam}\index{gnl}{Periyapuranam@\textit{Periyapur\=a\d nam}}, de nombreux chercheurs ont plac\'e Campantar\index{gnl}{Campantar} dans la seconde moiti\'e du \textsc{vii}\up{e} si\`ecle. Rappelons par exemple la datation que suit \textsc{Peterson} (*1991 [1989]: 19), qui reprend \textsc{Zvelebil} (1975: 139-141).
Ailleurs, dans son introduction \`a la traduction du \textit{Periyapur\=a\d nam}\index{gnl}{Periyapuranam@\textit{Periyapur\=a\d nam}}, \textsc{Ramachandran} (1995: xxii) suit la datation extr\^emement pr\'ecise propos\'ee par un certain \textsc{Sivagurunaatha Pillai}:

\scriptsize
\begin{quote}
%Thanks to the patient and painstaking research conducted by Sri Sivagurunaatha Pillai of Sundaraperumaall Koyil, we now know that
\dots\ St. Sambandhar made his avatar on the Adirai Day, the 29th of Chittirai, Vikaari (3740, Kaliyabda) corresponding to 12 May 639.
\end{quote}
\normalsize
\noindent Les exemples du genre abondent\footnote{Cf., entre autres, Somasundaram (1986: 3-9) et Soundra (1979: 46-53).}; \textit{a contrario}, vouloir d\'em\^eler le mythe\index{gnl}{mythe} de l'histoire est une t\^ache difficile qui d\'epasse le cadre de notre recherche pr\'esente. Nous nous contentons d'\'etudier ici, non pas l'histoire de Campantar\index{gnl}{Campantar}, mais celle de sa légende\index{gnl}{legende@légende} qui est inextricablement associ\'ee \`a C\=\i k\=a\b li\index{gnl}{Cikali@C\=\i k\=a\b li}.

Nous \'etudions dans un premier temps, \`a travers des textes du \textit{Tirumu\b rai}\index{gnl}{Tirumurai@\textit{Tirumu\b rai}} \textsc{xi}, l'origine\index{gnl}{origine} de la légende\index{gnl}{legende@légende} de Campantar\index{gnl}{Campantar} que nous mettons, ensuite, en rapport avec ses premi\`eres image\index{gnl}{image}s et enfin, nous examinons les légende\index{gnl}{legende@légende}s de son lieu d'origine\index{gnl}{origine}, C\=\i k\=a\b li\index{gnl}{Cikali@C\=\i k\=a\b li}.

\section{\`A la recherche de l'origine de la légende}

Seules deux informations, figurant dans un envoi\index{gnl}{envoi} et une strophe du \textit{T\=ev\=aram}\index{gnl}{Tevaram@\textit{T\=ev\=aram}} (II 84.11 et III 39.1) de l'authenticit\'e desquels nous doutons (voir 2.2 et 2.3.1), nous apprennent que Campantar\index{gnl}{Campantar} est un enfant\index{gnl}{enfant}. Cette caract\'eristique fondamentale, vraisemblablement, absente, \`a notre avis, des envois\index{gnl}{envoi} et de l'\oe uvre enti\`ere semble appartenir au poète\index{gnl}{poete@poète} l\'egendaire. Examinons maintenant ces textes du \textit{Tirumu\b rai}\index{gnl}{Tirumurai@\textit{Tirumu\b rai}} \textsc{xi}, souvent peu exploités, \`a la recherche de l'origine\index{gnl}{origine} de cette image\index{gnl}{image} de l'enfant\index{gnl}{enfant} divin.

\subsection{Enfant b\'eni chez Pa\d t\d ti\b nattuppi\d l\d lai}

Le \textit{Tirukka\b lumalamumma\d nikk\=ovai}\index{gnl}{Tirukkalumalamummanikkovai@\textit{Tirukka\b lumalamumma\d nikk\=ovai}} (cf. 1.2.2), qui appartient au \textit{Tirumu\b rai}\index{gnl}{Tirumurai@\textit{Tirumu\b rai}} \textsc{xi}, est un des cinq textes attribu\'es \`a Pa\d t\d ti\b nattuppi\d l\d lai\index{gnl}{Pattinattu Pillai@Pa\d t\d ti\b nattuppi\d l\d lai} (entre le \textsc{x}\up{e} et le \textsc{xiv}\up{e} si\`ecle). Dix-huit des trente strophes qui l'auraient compos\'e manquent\footnote{L'\'edition du monast\`ere\index{gnl}{monastère} de Tarumapuram\index{gnl}{Tarumapuram} ne pr\'esente que douze\index{gnl}{douze} strophes et pr\'ecise qu'elle n'a pas eu acc\`es \`a celle de \textsc{Ci\.nk\=arav\=el Mutaliy\=ar} qui aurait fourni le texte complet (p. 622).}. Les vers 23 \`a 33 de la premi\`ere partie
%, en m\`etre \textit{akaval},
 pr\'esentent l'\'episode du don\index{gnl}{don} de la nourriture de connaissance\index{gnl}{connaissance}, ainsi que la premi\`ere strophe de l'hymne\index{gnl}{hymne} inaugural du corpus\index{gnl}{corpus} \'etabli du \textit{T\=ev\=aram}\index{gnl}{Tevaram@\textit{T\=ev\=aram}} mais sans mentionner le nom de Campantar\index{gnl}{Campantar}:

\scriptsize
\begin{verse}
\textit{t\=ataiyo\d tu vanta v\=etiyac ci\b ruva\b n\\
ta\d larna\d taip paruvattu va\d larpaci varutta\\
`a\b n\b n\=a y\=o've\b n \b ra\b laippamu\b n ni\b n\b ru\\
\~n\=a\b na p\=o\b nakam aru\d la\d t\d tik ku\b laitta\\
\=a\b n\=at tira\d lai ava\b nvayi\b n aru\d la\\
anta\d na\b n mu\b nintu `tant\=ar y\=ar'e\b na\\
`ava\b naik k\=a\d t\d tuva\b n appa, v\=a\b n\=ar\\
\textbf{t\=oo \d tu\d taiya ceviya\b n}' e\b n\b rum\\
\textbf{p\=\i i \d tu\d taiya pemm\=a\b n} e\b n\b rum\\
kaiyil cu\d t\d tik k\=a\d t\d ta\\
aiyan\=\i\ ve\d lippa\d t \d taru\d li\b nai \=a\.nk\=e.} (l. 23-33)\\
\end{verse}

\normalsize
\begin{verse}
L'enfant\index{gnl}{enfant} v\'edisant, venu avec le père\index{gnl}{pere@père},\\
Alors que la faim de l'\^age \`a la d\'emarche titubante [le] tourmentait\\
Et qu'il appellait `\^o m\`ere',\\
Tu te tins devant\\
Et tu lui donnas dans sa bouche une boule de riz\index{gnl}{riz}, perpétuelle,\\
P\'etrie en y m\^elant la gr\^ace de la nourriture de connaissance\index{gnl}{connaissance};\\
Quand le brahmane\index{gnl}{brahmane}, en col\`ere, demanda: `Qui [te l']a donn\'ee?'\\
Il dit: `je le montre, père\index{gnl}{pere@père}, le c\'eleste,\\
Celui \`a l'oreille pourvue d'une boucle,\\
Le seigneur pourvu d'excellence'\\
Et il le montra du doigt;\\
Seigneur, tu fis la gr\^ace de te manifester l\`a! (l. 23-33)\\
\end{verse}

\normalsize
\noindent
L'action se d\'eroule \`a Ka\b lumalam\index{gnl}{Kalumalam@Ka\b lumalam} (1.~17). Un enfant\index{gnl}{enfant} brahmane\index{gnl}{brahmane} pleure de faim. \'Siva\index{gnl}{Siva@\'Siva} se manifeste et lui donne du riz\index{gnl}{riz}, nourriture de la connaissance\index{gnl}{connaissance}. Quand le père\index{gnl}{pere@père} de l'enfant\index{gnl}{enfant}, en col\`ere, le questionne sur l'origine\index{gnl}{origine} de ce mets, ce dernier pointe du doigt le ciel et prononce les mots \textit{t\=o\d tu\d taiya ceviya\b n} et \textit{p\=\i \d tu\d taiya pemm\=a\b n}. Ces deux \'epith\`etes sont, respectivement, celles des premier et quatri\`eme vers de la strophe ouvrant le corpus\index{gnl}{corpus} actuel du \textit{T\=ev\=aram}\index{gnl}{Tevaram@\textit{T\=ev\=aram}}, d\'edi\'ee \`a Piramapuram\index{gnl}{Piramapuram}, \textit{i.e.} C\=\i k\=a\b li\index{gnl}{Cikali@C\=\i k\=a\b li}:

\scriptsize
\begin{verse}
\textit{\textbf{t\=o\d tu u\d taiya ceviya\b n}, vi\d tai \=e\b ri, \=or t\=u ve\d nmati c\=u\d ti,\\
k\=a\d tu u\d taiya cu\d talaip po\d ti p\=uci, e\b n u\d l\d lam kavar ka\d lva\b n~---\\
\=e\d tu u\d taiya malar\=a\b n mu\b nain\=a\d l pa\d nintu \=etta, aru\d lceyta,\\
\textbf{p\=\i \d tu u\d taiya} piram\=apuram m\=eviya \textbf{pemm\=a\b n} --- iva\b n a\b n\b r\=e!} (I 1.1)\\
\end{verse}

\normalsize
\begin{verse}
Le voleur qui ravit mon for int\'erieur,\\
Celui \`a l'oreille pourvue d'une boucle,\\
Mont\'e sur le taureau\index{gnl}{taureau},\\
Couronn\'e de la pure lune blanche,\\
Enduit de la cendre\index{gnl}{cendre} des b\^uchers des bois [cr\'ematoires];\\
N'est-ce pas lui,\\
Le seigneur pourvu d'exellence qui vit \`a Piramapuram\index{gnl}{Piramapuram},\\
Accorda [sa] gr\^ace\\
Quand celui de la fleur aux p\'etales, inclin\'e, le loua? (I 1.1)\\
\end{verse}

\noindent
Cette citation nous permet d'affirmer que l'enfant\index{gnl}{enfant} brahmane\index{gnl}{brahmane} nourri par \'Siva\index{gnl}{Siva@\'Siva} est bien Campantar\index{gnl}{Campantar}. Aucune autre information n'est donn\'ee sur le poète\index{gnl}{poete@poète} dans le reste du texte disponible.

Nous observons dans ce poème\index{gnl}{poeme@poème} que le site de C\=\i k\=a\b li\index{gnl}{Cikali@C\=\i k\=a\b li} n'est d\'esign\'e que par trois toponymes: Ka\b lumalam\index{gnl}{Kalumalam@Ka\b lumalam} (1.17; 5.3; 6.4; 7.5; 8.4; 9.3), Pukali\index{gnl}{Pukali} (2.2; 3.3; 4.21; 10.5; 11.4) et T\=o\d nipuram\index{gnl}{Tonipuram@T\=o\d nipuram} (12.3). Toutefois, il est pr\'ecis\'e aux vers 2-5 de la strophe 10 que le \og site conna\^it un nom distinct dans chacun des douze\index{gnl}{douze} \textit{yuga}\fg\ (\textit{pa\b n\b n\=\i rukattu v\=e\b ruv\=e\b ru peyari\b n \=ur}). Mais ces douze\index{gnl}{douze} appellations ne sont pas donn\'ees.

En r\'esum\'e, le \textit{Tirukka\b lumalamumma\d nikk\=ovai}\index{gnl}{Tirukkalumalamummanikkovai@\textit{Tirukka\b lumalamumma\d nikk\=ovai}}, s'il est ant\'erieur aux textes attribu\'es \`a Nampi \=A\d n\d t\=ar Nampi\index{gnl}{Nampi \=A\d n\d t\=ar Nampi} et au \textit{Periyapur\=a\d nam}\index{gnl}{Periyapuranam@\textit{Periyapur\=a\d nam}}, serait le premier texte \`a faire r\'ef\'erence \`a l'enfant\index{gnl}{enfant} poète\index{gnl}{poete@poète} Campantar\index{gnl}{Campantar} et \`a l'associer au miracle\index{gnl}{miracle} du don\index{gnl}{don} de la nourriture de connaissance\index{gnl}{connaissance} et \`a l'hymne\index{gnl}{hymne} I 1. Dans cet \'episode, notons que c'est \'Siva\index{gnl}{Siva@\'Siva} lui-m\^eme, et non pas la d\'eesse\index{gnl}{deesse@déesse}, qui nourrit le jeune poète\index{gnl}{poete@poète} affam\'e et ce, avec une boule de riz\index{gnl}{riz} (\textit{tira\d lai}). Cette version diff\`ere l\'eg\`erement de celle du \textit{Periyapur\=a\d nam}\index{gnl}{Periyapuranam@\textit{Periyapur\=a\d nam}} dans laquelle Campantar\index{gnl}{Campantar} enfant\index{gnl}{enfant} ne pleure pas de faim mais parce qu'il constate l'absence de son père\index{gnl}{pere@père} (st. 1959), puis il boit le lait\index{gnl}{lait} du sein de P\=arvat\=\i\index{gnl}{Parvati@P\=arvat\=\i}\ dans une coupelle\index{gnl}{coupelle} (st. 1965-66). Ajoutons qu'il est question de l'unit\'e des douze\index{gnl}{douze} toponymes qui est une caract\'eristique fondamentale du site dans certains textes attribu\'es \`a Nampi \=A\d n\d t\=ar Nampi\index{gnl}{Nampi \=A\d n\d t\=ar Nampi}.

\subsection{Enfant divin chez Nampi \=A\d n\d t\=ar Nampi}

Sept textes attribu\'es \`a Nampi \=A\d n\d t\=ar Nampi\index{gnl}{Nampi \=A\d n\d t\=ar Nampi} (\textsc{xi}\up{e}-\textsc{xii}\up{e} si\`ecles) c\'el\`ebrent Campantar\index{gnl}{Campantar} (cf. 4.2.2). Six d'entre eux lui sont enti\`erement consacr\'es\footnote{Nous adoptons les abr\'eviations suivantes: \textit{TTA} pour le \textit{Tirutto\d n\d tar tiruvant\=ati}, \textit{APCV} pour l'\textit{\=A\d lu\d taiyapi\d l\d laiy\=ar\index{gnl}{Campantar!Alutaiyapillaiyar@\=A\d lu\d taiyapi\d l\d laiy\=ar} tiruca\d npai viruttam}, \textit{APK} pour l'\textit{\=A\d lu\d taiyapi\d l\d laiy\=ar\index{gnl}{Campantar!Alutaiyapillaiyar@\=A\d lu\d taiyapi\d l\d laiy\=ar} tirukkalampakam}, \textit{APMK} pour l'\textit{\=A\d lu\d taiyapi\d l\d laiy\=ar\index{gnl}{Campantar!Alutaiyapillaiyar@\=A\d lu\d taiyapi\d l\d laiy\=ar} tirumumma\d nikk\=ovai}, \textit{APUM} pour l'\textit{\=A\d lu\d taiyapi\d l\d laiy\=ar\index{gnl}{Campantar!Alutaiyapillaiyar@\=A\d lu\d taiyapi\d l\d laiy\=ar} tiruvul\=am\=alai}, \textit{APA} pour l'\textit{\=A\d lu\d taiyapi\d l\d laiy\=ar\index{gnl}{Campantar!Alutaiyapillaiyar@\=A\d lu\d taiyapi\d l\d laiy\=ar} tiruvant\=ati} et \textit{APT} pour l'\textit{\=A\d lu\d taiyapi\d l\d laiy\=ar\index{gnl}{Campantar!Alutaiyapillaiyar@\=A\d lu\d taiyapi\d l\d laiy\=ar} tiruttokai}.}. Le \textit{Tirutto\d n\d tar tiruvant\=ati}, cons\'ecration de tous les \textit{n\=aya\b nm\=ar}\index{gnl}{nayanmar@\textit{n\=aya\b nm\=ar}}, se distingue en lui d\'ediant seulement deux strophes. La premi\`ere mentionne le nom du poète\index{gnl}{poete@poète}: \~N\=a\b nacampanta\b n\index{gnl}{Campantar!N\=a\b nacampanta\b n@\~N\=a\b nacampanta\b n}. Il est d\'ecrit comme un enfant\index{gnl}{enfant} qui, ayant obtenu la gr\^ace de P\=arvat\=\i\index{gnl}{Parvati@P\=arvat\=\i}, chante pour les habitants de Piramapuram\index{gnl}{Piramapuram} afin de r\'ejouir le monde et de d\'etruire les ja\"in\index{gnl}{jain@ja\"in}s. La seconde, plus ambig\"ue, \'evoque des \textit{n\=aya\b nm\=ar}\index{gnl}{nayanmar@\textit{n\=aya\b nm\=ar}} tels que le \textit{C\=o\b la\index{gnl}{Cola@C\=o\b la}} Ce\.nka\d n, Muruka\b n\index{gnl}{Murukan@Muruka\b n} et N\=\i lanakka\b n\index{gnl}{Nilanakkan@N\=\i lanakka\b n}:

\scriptsize
\begin{verse}
\textit{vaiya maki\b lay\=am v\=a\b la vama\d nar valitolaiya\\
aiya\b n pirama purattara\b r kamme\b n kutalaiccevv\=ay\\
paiya mi\b la\b r\b rum paruvattup p\=a\d tap paruppatatti\b n\\
taiya laru\d lpe\b r \b ra\b na\b ne\b npar \~n\=a\b nacam panta\b naiy\=e.} (\textit{TTA} 33)\\

\textit{pant\=ar viraliyar v\=e\d lce\.nka\d t c\=o\b la\b n muruka\b nnalla\\
cant\=a rakalattu n\=\i lanak ka\b npeyar t\=a\b nmo\b lintu\\
kont\=ar ca\d taiyar patikatti li\d t\d ta\d ti y\=e\b nko\d tutta\\
ant\=ati ko\d n\d ta pir\=a\b naru\d t k\=a\b liyar ko\b r\b rava\b n\=e.} (\textit{TTA} 34)\\
\end{verse}

\normalsize
\begin{verse}
Pour que le monde se r\'ejouisse,\\
Que nous vivions,\\
Que les ja\"in\index{gnl}{jain@ja\"in}s perdent leur force,\\
Il chante \`a l'\^age o\`u on gazouille lentement\\
De sa douce et belle bouche rouge babillante,\\
Pour ceux de Piramapuram\index{gnl}{Piramapuram} d'Aiya\b n,\\
Il obtient la gr\^ace de la femme\index{gnl}{femme} de la montagne,\\
On dit que c'est \~N\=a\b nacampanta\b n\index{gnl}{Campantar!N\=a\b nacampanta\b n@\~N\=a\b nacampanta\b n}. (\textit{TTA} 33)\\

\end{verse}
\begin{verse}
Le C\=o\b la\index{gnl}{Cola@C\=o\b la} Ce\.nka\d n d\'esir\'e de celles aux doigts [jouant] \`a la balle\footnote{Plusieurs autres lectures sont possibles: \begin{itemize}
\item V\=e\d l (K\=ama\index{gnl}{Kama@K\=ama}) de celles aux doigts [jouant] \`a la balle, Ce\.nka\d n le C\=o\b la\index{gnl}{Cola@C\=o\b la}
\item Celles aux doigts [jouant] \`a la balle, le d\'esirable Ce\.nka\d n le C\=o\b la\index{gnl}{Cola@C\=o\b la}
\end{itemize}},\\
Muruka\b n\index{gnl}{Murukan@Muruka\b n}, N\=\i lanakka\b n\index{gnl}{Nilanakkan@N\=\i lanakka\b n} au torse enduit de bon santal,\\
Ayant dit leurs noms,\\
Les ayant plac\'es dans les d\'ecades (offertes)\\
 \`A Celui aux m\`eches pourvues de fleurs,\\
Il re\c coit l'\textit{ant\=ati} donn\'e par moi le serviteur\index{gnl}{serviteur},\\
Le Ko\b r\b rava\b n de ceux de K\=a\b li\index{gnl}{Kali@K\=a\b li}\\
Qui a la gr\^ace du Seigneur. (\textit{TTA} 34)\\
\end{verse}

\normalsize
Bien que les \'el\'ements hagiographique\index{gnl}{hagiographie!hagiographique}s contenus dans ce passage soient peu nombreux, ils r\'esument le caract\`ere essentiel du personnage l\'egendaire. Campantar\index{gnl}{Campantar} est en premier lieu pr\'esent\'e comme l'ennemi des ja\"in\index{gnl}{jain@ja\"in}s. Il est ensuite associ\'e \`a la localit\'e de Piramapuram\index{gnl}{Piramapuram}. Bien qu'il s'agisse du toponyme utilis\'e pour c\'el\'ebrer C\=\i k\=a\b li\index{gnl}{Cikali@C\=\i k\=a\b li} dans l'hymne\index{gnl}{hymne} inaugural du \textit{T\=ev\=aram}\index{gnl}{Tevaram@\textit{T\=ev\=aram}} il nous semble fragile d'identifier ici une allusion directe \`a ce corpus\index{gnl}{corpus}. Puis, le \textit{Tirutto\d n\d tar tiruvant\=ati} souligne le tr\`es jeune \^age de Campantar\index{gnl}{Campantar}. Il poss\`ede une bouche qui babille (\textit{kutalai}) et chante \`a un \^age o\`u l'on gazouille doucement (\textit{paiya mi\b la\b r\b rum paruvam}). La facult\'e de chanter, obtenue par la gr\^ace divine de P\=arvat\=\i\index{gnl}{Parvati@P\=arvat\=\i}, d\`es sa prime enfance le rend exceptionnel. Campantar\index{gnl}{Campantar} est un enfant\index{gnl}{enfant} prodige\index{gnl}{prodige}.

La seconde strophe est probl\'ematique du fait de sa construction syntaxique et de l'expression \textit{pant\=ar viraliyar}. Deux lectures nous semblent possibles. Si \textit{ko\b r\b rava\b n}\index{gnl}{korravan@\textit{ko\b r\b rava\b n}}, \textit{i.e} Campantar\index{gnl}{Campantar}, est le sujet des absolutifs \textit{mo\b lintu} et \textit{i\d t\d tu}, ainsi que du participe \textit{ko\d n\d ta}, nous pouvons lire qu'il a mentionn\'e les quelques dévot\index{gnl}{devot(e)@dévot(e)}s (Ce\.nka\d n le \textit{C\=o\b la\index{gnl}{Cola@C\=o\b la}}, Muruka\b n\index{gnl}{Murukan@Muruka\b n} et N\=\i lanakka\b n\index{gnl}{Nilanakkan@N\=\i lanakka\b n}) dans ses d\'ecades en l'honneur de \'Siva\index{gnl}{Siva@\'Siva}. Cependant, si \textit{a\d tiy\=e\b n}, premi\`ere personne du singulier renvoyant \`a l'auteur du \textit{Tirutto\d n\d tar tiruvant\=ati}, est le sujet des absolutifs et du participe \textit{ko\d tutta} alors c'est plut\^ot le poète\index{gnl}{poete@poète} Nampi qui \'evoque les \textit{n\=aya\b nm\=ar}\index{gnl}{nayanmar@\textit{n\=aya\b nm\=ar}} dans son \textit{ant\=ati} qu'il donne \`a Campantar\index{gnl}{Campantar}. De plus, l'identification de la ou des femme\index{gnl}{femme}s d\'esign\'ees par la m\'etonymie \textit{pant\=ar viraliyar}, \og celle aux doigts [jouant] \`a la balle\fg, est difficile. Le commentaire de l'\'edition de Tarumapuram\index{gnl}{Tarumapuram} propose d'en faire un compl\'ement de nom de \textit{v\=e\d l} qui signifierait le dieu\index{gnl}{dieu} de l'amour K\=ama\index{gnl}{Kama@K\=ama}, et de lire \og K\=ama\index{gnl}{Kama@K\=ama} de celles aux doigts [jouant] \`a la balle\fg. Or, inclure K\=ama\index{gnl}{Kama@K\=ama} dans une \'enum\'eration de \textit{n\=aya\b nm\=ar}\index{gnl}{nayanmar@\textit{n\=aya\b nm\=ar}} n'est pas satisfaisant. Cette m\'etonymie pourrait aussi d\'esigner un personnage \`a part enti\`ere du groupe de dévot\index{gnl}{devot(e)@dévot(e)}s shiva\"ite\index{gnl}{shiva\"ite}s tel que la reine\index{gnl}{reine} \textit{p\=a\d n\d dya}\index{gnl}{pandya@\textit{p\=a\d n\d dya}} ou la m\`ere de Cuntarar\index{gnl}{Cuntarar}. Ajoutons enfin qu'une strophe de l'\textit{APA}, attribu\'e aussi \`a Nampi \=A\d n\d t\=ar Nampi\index{gnl}{Nampi \=A\d n\d t\=ar Nampi}, pr\'esente des rimes initiales similaires\footnote{\textit{APA} 19: \textit{pant\=ar a\d niviral} \dots/ \textit{kont\=ar na\b ru\.nku\b lal} \dots/ \textit{nant\=a vi\d lakki\b naik} \dots/ \textit{cant\=ar akalattu} \dots}. Ceci nous permet de supposer que l'expression \textit{pant\=ar viraliyar}, relevant clairement de la formule\footnote{Signalons que cette image\index{gnl}{image} d\'ecrit exclusivement P\=arvat\=\i\index{gnl}{Parvati@P\=arvat\=\i}\ dans le \textit{T\=ev\=aram}\index{gnl}{Tevaram@\textit{T\=ev\=aram}} (I 8.5, 17.5, 70.5, 100.9, 107.1, 120.7; II 57.11, 72.1, 109.11; III 2.1, 12.6, 28.3, 58.5, 120.3; VI 4.1, 6.10, 46.10, 73.3, 86.8; VII 25.7, 27.5, 49.10 et 85.4).}, a peut-\^etre servi ici uniquement \`a la versification.

Ainsi, ce passage du \textit{Tirutto\d n\d tar tiruvant\=ati} pr\'esente succinctement les caract\`eres constitutifs du personnage l\'egendaire de Campantar\index{gnl}{Campantar}: un tr\`es jeune enfant\index{gnl}{enfant} poète\index{gnl}{poete@poète}, originaire de C\=\i k\=a\b li\index{gnl}{Cikali@C\=\i k\=a\b li}, qui a obtenu la gr\^ace divine et qui combat les ja\"in\index{gnl}{jain@ja\"in}s. Les six autres textes, attribu\'es \`a Nampi \=A\d n\d t\=ar Nampi\index{gnl}{Nampi \=A\d n\d t\=ar Nampi}, qui lui sont consacr\'es abondent en r\'ef\'erences hagiographique\index{gnl}{hagiographie!hagiographique}s pour certains et illustrent un traitement in\'egal des miracle\index{gnl}{miracle}s octroy\'es \`a Campantar\index{gnl}{Campantar}. Nous constatons, \`a premi\`ere vue, que l'\textit{APA}, l'\textit{APUM} et surtout l'\textit{APT} exposent plus de miracle\index{gnl}{miracle}s que les autres.

Nous examinons maintenant l'image\index{gnl}{image} du poète\index{gnl}{poete@poète} qui se dessine dans ces six textes, analysant, en particulier, les miracles qu'ils mentionnent et les difficultés qu'ils soulèvent.

\subsubsection{L'image\index{gnl}{image} du poète\index{gnl}{poete@poète} Campantar}

La repr\'esentation de Campantar\index{gnl}{Campantar}, personnage principal de ces textes, suivant ses d\'esignations, ses \oe uvres et sa caract\'eristique primordiale, est celle de l'ennemi des ja\"in\index{gnl}{jain@ja\"in}s.

Si tous les titres de ces \oe uvres comportent le nom \=A\d lu\d taiyapi\d l\d laiy\=ar\index{gnl}{Campantar!Alutaiyapillaiyar@\=A\d lu\d taiyapi\d l\d laiy\=ar}, \og l'enfant meneur d'hommes\fg, ce dernier n'appara\^it jamais dans les textes mêmes\footnote{\=A\d lu\d taiyapi\d l\d laiy\=ar\index{gnl}{Campantar!Alutaiyapillaiyar@\=A\d lu\d taiyapi\d l\d laiy\=ar} est le nom employ\'e dans les inscriptions pour d\'esigner l'image\index{gnl}{image} de Campantar\index{gnl}{Campantar}; voir partie III, le CEC en particulier.}, où le poète\index{gnl}{poete@poète} est nomm\'e Campanta\b n\index{gnl}{Campantar!Campanta\b n}, \~N\=a\b nacampanta\b n\index{gnl}{Campantar!N\=a\b nacampanta\b n@\~N\=a\b nacampanta\b n} ou Tiru\~n\=a\b nacampanta\b n\index{gnl}{Campantar!Tiru\~n\=a\b nacampanta\b n}\footnote{Nous relevons les occurrences de Campanta\b n\index{gnl}{Campantar!Campanta\b n} (\textit{APCV} en fin de chaque strophe; \textit{APK} 26, 42; \textit{APMK} 2, 7.5, 11, 13.15, 14, 20, 21, 30; \textit{APA} 1, 16, 18, 37), de \~N\=a\b nacampanta\b n\index{gnl}{Campantar!N\=a\b nacampanta\b n@\~N\=a\b nacampanta\b n} (\textit{APK} 1.7, 3, 4, 40, 41, 47; \textit{APMK} 4.25, 5, 9, 18, 22.12, 24, 27; \textit{APA} 8, 22, 25, 29, 30, 38, 42, 48, 50, 51, 56, 59, 61, 83, 90, 92) et de Tiru\~n\=a\b nacampanta\b n\index{gnl}{Campantar!Tiru\~n\=a\b nacampanta\b n} (\textit{APK} 2, 8, 10; \textit{APUM} 69; \textit{APA} 80, 84, 99, 100). Ajoutons que ces appellations ne figurent pas dans l'\textit{APT} o\`u il est nomm\'e simplement \textit{pi\d l\d lai} (\og enfant\index{gnl}{enfant}\fg\ \textit{APT} 29 et 60).}.
Il est aussi d\'esign\'e en rapport avec C\=\i k\=a\b li\index{gnl}{Cikali@C\=\i k\=a\b li}\footnote{\textit{APMK} 5, 9, 11, 12, 25.6, 26, 27, 28.4, 29; \textit{APK} 15, 25, 33, 40, 42, 44, 46, 47; \textit{APA} 1, 8, 9, 50, 52 et 53.}: il est le seigneur de cette localit\'e\footnote{\textit{APMK} 13.13, 15, 17; \textit{APK} 1.6, 2-12, 14, 16-23, 26-31, 35, 36, 37.30, 38, 39, 43, 48; \textit{APA} 2, 4, 5, 7, 20, 21, 24, 29, 32, 36, 43, 44, 46, 55, 57, 58, 61-64, 69, 77, 79, 86, 87, 93-95, 98; \textit{APUM} 70 et 132.} o\`u il est n\'e (\textit{APMK} 1.10, 7.5; \textit{APK} 0.4; \textit{APUM} 59-63). Nous observons toutefois un traitement in\'egal des douze\index{gnl}{douze} toponymes sur le mod\`ele des envois\index{gnl}{envoi} attribu\'es \`a Campantar\index{gnl}{Campantar}. K\=a\b li\index{gnl}{Kali@K\=a\b li}, Ca\d npai\index{gnl}{Canpai@Ca\d npai}, Pukali\index{gnl}{Pukali} et Ka\b lumalam\index{gnl}{Kalumalam@Ka\b lumalam} sont les plus fr\'equents. L'unit\'e des douze\index{gnl}{douze} noms n'est mentionn\'ee que dans deux textes, \textit{APUM} 56-58 et \textit{APA} 100, et ceci dans un ordre\index{gnl}{ordre} diff\'erent de celui pr\'esent\'e dans le corpus\index{gnl}{corpus} \'etabli du \textit{T\=ev\=aram}\index{gnl}{Tevaram@\textit{T\=ev\=aram}}. Enfant\index{gnl}{enfant} --- son jeune \^age est signifi\'e par les termes \og petit (d'une vache, d'un arbre)\fg\ (\textit{ka\b n\b ru} \textit{APA} 13 et 73) et \og enfant\index{gnl}{enfant}\fg\ (\textit{pi\d l\d lai} \textit{APMK} 6, 10.10, 26; \textit{APK} 0.32; \textit{APT} 29 et 60) ---, il re\c coit la gr\^ace divine (\textit{APMK} 19.7-11, 22.10 et \textit{APK} 0.5) et t\'emoigne d'une \'erudition exceptionnelle. Il conna\^it les \textit{Veda}\index{gnl}{Veda@\textit{Veda}} (\textit{APK} 0.19-20, 5, 15, 24, 48; \textit{APA} 61; \textit{APUM} 68). Flambeau du \textit{kavu\d n\d dinya gotra\index{gnl}{gotra@\textit{gotra}}} (\textit{APMK} 1.10, 10.4, 25.6; \textit{APK} 14, 17, 28, 34, 37.30; \textit{APA} 3, 12, 23, 27, 67, 98; \textit{APUM} 131), il est aussi le \og joyau du diad\`eme brahmane\index{gnl}{brahmane}\fg\ (\textit{vipracik\=ama\d ni} \textit{APK} 6, 11, 19) et \og shiva\"ite\index{gnl}{shiva\"ite}\fg\ (\textit{caivacik\=ama\d ni} \textit{APMK} 3, 13.15; \textit{APK} 15, 31, 37.31; \textit{APA} 11, 14, 78; \textit{APUM} 65; \textit{APCV} 9).
Campantar\index{gnl}{Campantar} d\'eifi\'e (\textit{APMK} 5, 12, 28.4; \textit{APK} 5, 9, 10, 24; \textit{APA} 2 et 94) est le fils d'Um\=a\index{gnl}{Uma@Um\=a} (\textit{malaimaka\d l putalva\b n} \textit{APMK} 1.11 et 10.6). N\'e de la gr\^ace divine (\textit{APMK} 4.2), il est ador\'e (\textit{APMK} 2, 3, 8, 14; \textit{APK} 3, 17, 18, 28, 36, 37.33-37, 40, 43; \textit{APA} 26, 42 et 90) et parfois m\^eme identifi\'e comme Muruka\b n\index{gnl}{Murukan@Muruka\b n} et K\=ama\index{gnl}{Kama@K\=ama} (\textit{APUM} 124).
Mis \`a part sa d\'eification et son jeune \^age, caract\'eristiques de ces textes, les autres \'el\'ements identitaires sont les m\^emes que dans les envois\index{gnl}{envoi} des hymne\index{gnl}{hymne}s du \textit{T\=ev\=aram}\index{gnl}{Tevaram@\textit{T\=ev\=aram}} qui lui sont attribu\'es.

Ensuite, Campantar\index{gnl}{Campantar} est un \og poète\index{gnl}{poete@poète}\fg\ (\textit{kavi APMK} 10.4; \textit{APK} 13, 33, 41; \textit{APA} 38), \og dou\'e en musique\index{gnl}{musique}\fg\ (\textit{icai vittaka\b n APT} 15) et \og ma\^itre des arts\fg\ (\textit{kalai talaiva\b n APMK} 14 et \textit{kalai vittaka\b n APK} 6). Il est l'\og expert tamoul\fg\ par excellence\footnote{\textit{tami\b l viraka\b n APMK} 1.9, 10.4, 16.14, 19.12, 23, 25.7; \textit{APK} 1.1, 6, 17, 21-24, 27, 32-34, 36, 37.31, 38, 39, 44-46; \textit{APA} 35, 45, 47, 52, 68, 70, 72, 74, 75, 79, 93, 94, 96 et 98.}. Il compose des chants\index{gnl}{chant}\index{gnl}{chant}\footnote{\textit{APMK} 1.12, 6, 21; \textit{APK} 0.2, 9, 29, 30, 35; \textit{APA} 2, 31, 80; \textit{APT} 9-10 et \textit{APCV} 6.}, en l'honneur de \'Siva\index{gnl}{Siva@\'Siva} (\textit{APK} 0.12, 7 et \textit{APUM} 85), appel\'es \og tamoul\fg\ (\textit{tami\b l APK} 18 et \textit{APMK} 11), \og guirlande\index{gnl}{guirlande}\fg\ (\textit{m\=alai\index{gnl}{malai@\textit{m\=alai}} APMK} 11) et \textit{Veda}\index{gnl}{Veda@\textit{Veda}} (\textit{curuti}, sk. \textit{\'sruti}, \textit{APK} 4; \textit{APA} 46 et 48), qui g\'en\`erent des miracle\index{gnl}{miracle}s (\textit{APMK} 4.6, 4.8-9, 28.2-3; \textit{APA} 39; \textit{APT} 38-39 et \textit{APCV} 3). Le vocabulaire employ\'e pour d\'esigner le poète\index{gnl}{poete@poète} et les poème\index{gnl}{poeme@poème}s est similaire \`a celui des envois\index{gnl}{envoi} du \textit{T\=ev\=aram}\index{gnl}{Tevaram@\textit{T\=ev\=aram}}. Notons que la nouveaut\'e ici est que les hymne\index{gnl}{hymne}s prennent une importance d\'emesur\'ee: ils acqui\`erent la sacralit\'e des \textit{Veda}\index{gnl}{Veda@\textit{Veda}} et le pouvoir de g\'en\'erer des miracle\index{gnl}{miracle}s. Ils se comptent par milliers. En effet, trois textes s'accordent pour attribuer \`a Campantar\index{gnl}{Campantar} seize milles hymne\index{gnl}{hymne}s ou strophes\footnote{\textit{pati\b n\=a\b r\=ayiram patikam} (\textit{APA} 15), \textit{pati\b n\=a\b r\=ayirana\b rpa\b nuval} (\textit{APUM} 62) et \textit{pati\b n\=a\b r\=ayiram p\=a} (\textit{APT} 42).}. Nous trouvons enfin des r\'ef\'erences pr\'ecises \`a certains hymne\index{gnl}{hymne}s du corpus\index{gnl}{corpus} actuel du \textit{T\=ev\=aram}\index{gnl}{Tevaram@\textit{T\=ev\=aram}} comme le poème\index{gnl}{poeme@poème} inaugural I 1 (\textit{APK} 0.9 et \textit{APA} 13), celui intitul\'e \textit{y\=a\b lm\=uri}\index{gnl}{yalmuri@\textit{y\=a\b lm\=uri}} I 136 (\textit{APK} 26; \textit{APA} 39, 91; \textit{APUM} 76 et \textit{APT} 13) ou encore celui surnomm\'e \textit{paccai patikam} I 49 (\textit{APT} 42)\footnote{L'\'edition de Tarumapuram\index{gnl}{Tarumapuram} explique que ce \textit{paccai patikam}, litt\'eralement \og d\'ecade verte\fg, est un hymne\index{gnl}{hymne} d\'edi\'e \`a Na\d l\d l\=a\b ru\index{gnl}{Nallaru@Na\d l\d l\=a\b ru}. Il s'agit d'apr\`es le \textit{Periyapur\=a\d nam}\index{gnl}{Periyapuranam@\textit{Periyapur\=a\d nam}} de l'hymne\index{gnl}{hymne} qui sort intact du feu\index{gnl}{feu} devant les ja\"in\index{gnl}{jain@ja\"in}s \`a Maturai\index{gnl}{Maturai} et qui est consacr\'e \`a Na\d l\d l\=a\b ru\index{gnl}{Nallaru@Na\d l\d l\=a\b ru} (st. 2354). Signalons aussi que deux inscriptions du temple\index{gnl}{temple} de Na\d l\d l\=a\b ru\index{gnl}{Nallaru@Na\d l\d l\=a\b ru} semblent faire allusion \`a ce surnom dans l'appelation du village Paccapatiyanall\=ur (cf. inscriptions 455 et 471 dans \textsc{Vijayavenugopal} 2006).}. L'\textit{APUM} mentionne quelques procédé\index{gnl}{procédé littéraire}s litt\'eraires (\textit{APUM} 83-85) et l'\textit{APT} la formule \textit{\=a\d nai namat\=e} (\textit{APT} 45) qui est pr\'esente dans les envois\index{gnl}{envoi} des hymne\index{gnl}{hymne}s II 84, 85 et III 78\footnote{Cette formule est aussi associ\'ee \`a Campantar\index{gnl}{Campantar} dans les inscriptions; voir SII 8 442 l. 24 par exemple.}.

Enfin, Campantar\index{gnl}{Campantar} porte le titre de \og lion pour les hérétique\index{gnl}{heretique@hérétique}s\fg\ (\textit{paracamaya k\=o\d lari APK} 5, 24, 37.32; \textit{APA} 4 et 54). Il est l'ennemi des ja\"in\index{gnl}{jain@ja\"in}s (\textit{APMK} 8 et \textit{APA} 43), qualifi\'es souvent de \og gros\fg\ (\textit{ku\d n\d tar} \textit{APK} 20 et 41). Il est n\'e pour les vaincre (\textit{APMK} 6 et \textit{APCV} 1). De nombreux passages \'evoquent sa victoire au bord du fleuve Vaikai\index{gnl}{Vaikai} (\textit{APMK} 13.14, 21, 26, 29; \textit{APA} 12 et 54) ou \`a K\=u\d tal, \textit{i.e.} Maturai\index{gnl}{Maturai} (\textit{APUM} 73), contre les ja\"in\index{gnl}{jain@ja\"in}s qu'il a fait empaler\index{gnl}{empaler}\footnote{\textit{APMK} 4.18-19, 26; \textit{APK} 0.23, 8, 9, 35; \textit{APA} 6, 28, 51, 66, 81; \textit{APUM} 74, 134-135 et \textit{APT} 12.}. La haine envers les ja\"in\index{gnl}{jain@ja\"in}s d\'ej\`a exprim\'ee dans le \textit{T\=ev\=aram}\index{gnl}{Tevaram@\textit{T\=ev\=aram}} est accompagn\'ee dor\'enavant de disputes sanglantes au bord de la Vaikai\index{gnl}{Vaikai}. Ajoutons que ces passages ne se r\'ef\`erent jamais de fa\c con pr\'ecise aux confrontations de Maturai\index{gnl}{Maturai} narr\'ees dans le \textit{Periyapur\=a\d nam}\index{gnl}{Periyapuranam@\textit{Periyapur\=a\d nam}} (la guérison\index{gnl}{guerison@guérison} du roi\index{gnl}{roi}, les épreuve\index{gnl}{epreuve@épreuve}s du feu\index{gnl}{feu}, de l'eau\index{gnl}{eau}, etc.). Campantar\index{gnl}{Campantar} vainc les ja\"in\index{gnl}{jain@ja\"in}s et les fait empaler\index{gnl}{empaler} au bord du fleuve (fig. 5.1). Quant aux bouddhiste\index{gnl}{bouddhiste}s, seuls deux textes les mentionnent. Ils sont vaincus (\textit{APA} 66) et d\'ecapit\'es (\textit{APT} 38-39).
\begin{figure}[!h]
  \centering
 \includegraphics[width=14cm]{docthese/photoCIIKAALI/chapdeesse8.JPG}
  \caption{Empalement des ja\"ins. Détail de la frise peinte sur le plafond du \textit{ma\d n\d dapa} de Skanda, C\=\i k\=a\b li (cliché U. \textsc{Veluppillai}, 2006).}
\end{figure}
\noindent
Ainsi, les actions de Campantar\index{gnl}{Campantar} contre ces deux religions sont peu d\'evelopp\'ees dans les textes attribu\'es \`a Nampi\index{gnl}{Nampi \=A\d n\d t\=ar Nampi}. Par exemple, les \'episodes de Maturai\index{gnl}{Maturai} qui forment un passage crucial et qui occupent un quart de l'hagiographie\index{gnl}{hagiographie} de Campantar\index{gnl}{Campantar} dans le \textit{Periyapur\=a\d nam}\index{gnl}{Periyapuranam@\textit{Periyapur\=a\d nam}} ne sont \'evoqu\'es que dans deux textes seulement: l'\textit{APUM} et l'\textit{APT}. Certains autres miracle\index{gnl}{miracle}s sont plus fr\'equemment attest\'es.

\subsubsection{Les miracles}

Dans les textes attribu\'es \`a Nampi \=A\d n\d t\=ar Nampi\index{gnl}{Nampi \=A\d n\d t\=ar Nampi} certains miracle\index{gnl}{miracle}s sont plus fr\'equents que d'autres, et un petit groupe d'entre eux n'appara\^it que dans deux poème\index{gnl}{poeme@poème}s.
Parmi les miracle\index{gnl}{miracle}s les plus r\'ecurrents se trouvent ceux d\'ej\`a \'evoqu\'es dans les hymne\index{gnl}{hymne}s attribu\'es \`a Appar\index{gnl}{Appar} et \`a Cuntarar\index{gnl}{Cuntarar}: le don\index{gnl}{don} de la connaissance\index{gnl}{connaissance}, des cymbale\index{gnl}{cymbale}s, des pièce\index{gnl}{piece@pièce}s \`a \=Ava\d tutu\b rai\index{gnl}{Avatuturai@\=Ava\d tutu\b rai} et \`a V\=\i \b limi\b lalai\index{gnl}{Vilimilalai@V\=\i \b limi\b lalai} et le miracle\index{gnl}{miracle} des portes de Ma\b raikk\=a\d tu\index{gnl}{Maraikkatu@Ma\b raikk\=a\d tu}. \`A ceux-ci s'ajoutent le don\index{gnl}{don} du palanquin\index{gnl}{palanquin}, la guérison\index{gnl}{guerison@guérison} de l'amant mordu par un serpent\index{gnl}{serpent} \`a Marukal\index{gnl}{Marukal}, le prodige\index{gnl}{prodige} de la barque\index{gnl}{barque} de P\=ut\=ur\index{gnl}{Kollamputur@Ko\d l\d lamp\=ut\=ur!Putur@P\=ut\=ur} et celui des palmier\index{gnl}{palmier}s d'\=Ott\=ur\index{gnl}{Ottur@\=Ott\=ur}.

Le tableau 5.1 recense les prodige\index{gnl}{prodige}s narr\'es dans le \textit{Periyapur\=a\d nam}\index{gnl}{Periyapuranam@\textit{Periyapur\=a\d nam}} et \'etablit leur fr\'equence dans les \'ecrits attribu\'es \`a Nampi\index{gnl}{Nampi \=A\d n\d t\=ar Nampi}\footnote{La num\'erotation des strophes \'etant erron\'ee dans l'\'edition du monast\`ere\index{gnl}{monastère} de Tarumapuram\index{gnl}{Tarumapuram} nous suivons celle de l'\'edition de Tiruppa\b nant\=a\d l.}.


\scriptsize
\begin{center}
\begin{longtable}{|l|c|c|c|c|c|c|}
\caption{Les miracles de Campantar}\endfirsthead
\hline
\ Miracle\index{gnl}{miracle}s du & \textit{APCV} & \textit{APK} & \textit{APMK} & \textit{APA} & \textit{APUM} & \textit{APT}\\
\textit{Periyapur\=a\d nam}\index{gnl}{Periyapuranam@\textit{Periyapur\=a\d nam}} &&&&&&\endhead
\hline
\ Miracle\index{gnl}{miracle}s du & \textit{APCV} & \textit{APK} & \textit{APMK} & \textit{APA} & \textit{APUM} & \textit{APT}\\
\textit{Periyapur\=a\d nam}\index{gnl}{Periyapuranam@\textit{Periyapur\=a\d nam}} &&&&&&\\
\hline\hline
Lait, gr\^ace, connaissance\index{gnl}{connaissance} & st. 2 & st. 0.5-8, & st. 1.5, 4.5 & st. 40, 72 & st. 67 & l. 1-8\\
&&9, 35, 48 & & & &\\
\hline
Cymbale\index{gnl}{cymbale}s & & 9 & 4.6-7 & 40, 82 & 82 & 22\\
\hline
Palanquin\index{gnl}{palanquin} & & 9 & 4.12-13, 28.3 & 40, 83 & 79 & 23-24 \\
\hline
Guérison\index{gnl}{guerison@guérison} de l'\'epilepsie & & & & & & 31-32\\
\hline
Guérison\index{gnl}{guerison@guérison} de la fièvre\index{gnl}{fievre@fièvre} & & & & & 78 & \\
\hline
Contre la chaleur & & & & & & \\
\hline
Pièce\index{gnl}{piece@pièce}s d'or \`a \=Ava\d tutu\b rai\index{gnl}{Avatuturai@\=Ava\d tutu\b rai} & & & 4.17 & 40, 85 & 80 & 18-19\\
\hline
Amant gu\'eri du venin & & & 4.9 & 49 & 137-138 & 21\\
\`a Marukal\index{gnl}{Marukal} &&&&&&\\
\hline
Pièce\index{gnl}{piece@pièce}s d'or \`a V\=\i \b limi\b lalai\index{gnl}{Vilimilalai@V\=\i \b limi\b lalai} & & 24 & & 41, 80 & 78-79 & 11\\
\hline
Portes de Ma\b raikk\=a\d tu\index{gnl}{Maraikkatu@Ma\b raikk\=a\d tu} & 7 & 35 & 4.10-11 & 39, 91 & 77 & 26-27\\
\hline
Transfert du feu\index{gnl}{feu} sur & & & & & & 47-50\\
le roi\index{gnl}{roi} \textit{p\=a\d n\d dya}\index{gnl}{pandya@\textit{p\=a\d n\d dya}} &&&&&&\\
\hline
Guérison\index{gnl}{guerison@guérison} du roi\index{gnl}{roi} & & & & & & \\
\hline
\'Epreuve\index{gnl}{epreuve@épreuve} du feu\index{gnl}{feu} & & & & & 76 & 13-14\\
\hline
\'Epreuve\index{gnl}{epreuve@épreuve} de l'eau\index{gnl}{eau} & & & & & 76 & 33-34\\
\hline
Barque\index{gnl}{barque} de P\=ut\=ur\index{gnl}{Kollamputur@Ko\d l\d lamp\=ut\=ur!Putur@P\=ut\=ur} & & 35 & 4.22-23 & 39 & 77 & 29-30\\
\hline
Bouddhiste d\'ecapit\'e & & & & & & 38-39\\
\hline
Palmier\index{gnl}{palmier}s d'\=Ott\=ur\index{gnl}{Ottur@\=Ott\=ur} & & & 4.14-15 & 39 & 81 & 27-28\\
\hline
Jeune fille ressuscit\'ee & & & & & & 35-38\\
\hline
Mariage\index{gnl}{mariage} & 10 &&& 60 && 61-64\\
\hline
\end{longtable}
\end{center}


\normalsize
Le don\index{gnl}{don} de la connaissance\index{gnl}{connaissance}, \textit{i.e.} celui du lait\index{gnl}{lait}, est mentionn\'e dans tous les textes. Campantar\index{gnl}{Campantar} re\c coit d\`es son jeune \^age la gr\^ace de \'Siva\index{gnl}{Siva@\'Siva} (\textit{APK} 0.5-8, \textit{APMK} 1.5, \textit{APA} 72 et \textit{APT} 1-10) et celle de la d\'eesse\index{gnl}{deesse@déesse} (\textit{APCV} 2, \textit{APK} 9, 35, et 48) qui le nourrit avec l'ambroisie\index{gnl}{ambroisie} de la connaissance\index{gnl}{connaissance} (\textit{APK} 0.5-8, \textit{APMK} 4.5, \textit{APUM} 67, \textit{APA} 40, 72 et \textit{APT} 1-10) dans une coupelle\index{gnl}{coupelle} en or (\textit{APK} 0.5-8 et \textit{APUM} 67). Notons qu'un seul texte pr\'ecise que l'enfant\index{gnl}{enfant} pleure de faim en tapant des pieds (\textit{APT} 1-10)\footnote{Cette version des faits n'est pas celle du \textit{Periyapur\=a\d nam}\index{gnl}{Periyapuranam@\textit{Periyapur\=a\d nam}} mais celle retenue par Pa\d t\d ti\b nattuppi\d l\d lai\index{gnl}{Pattinattu Pillai@Pa\d t\d ti\b nattuppi\d l\d lai} (voir 5.1.1).}.

Le don\index{gnl}{don} des cymbale\index{gnl}{cymbale}s est souvent coupl\'e avec celui du palanquin\index{gnl}{palanquin} (\textit{APK} 9, \textit{APA} 40, \textit{APA} 82-83, \textit{APT} 22-24). Ces deux objets constituent les attributs du poète\index{gnl}{poete@poète}-enfant\index{gnl}{enfant}.
Campantar\index{gnl}{Campantar} obtient des cymbale\index{gnl}{cymbale}s d'or (\textit{APA} 9, \textit{APMK} 6-7, \textit{APUM} 82) \`a K\=olakk\=a\index{gnl}{Kolakka@K\=olakk\=a} (\textit{APT} 22, \textit{APA} 40, \textit{APA} 82, \textit{APMK} 6-7), localit\'e adjacente \`a [C\=\i ]k\=a\b li (\textit{APUM} 82), pour que ses petites mains ne souffrent pas en marquant le rythme du chant\index{gnl}{chant} (\textit{APA} 82 et \textit{APUM} 82):

\scriptsize
\begin{verse}
\textit{ka\d n\d n\=ar tirunuta l\=o\b nk\=olak k\=avil karano\d tiy\=al\\
pa\d n\d n\=ar tarapp\=a\d tu ca\d npaiyar k\=o\b np\=a\d ni nonti\d tume\b n(\b ru)\\
e\d n\d n\=a ve\b lutta\~ncu mi\d t\d tapo\b n t\=a\d la\.nka \d l\=\i yakka\d n\d tum\\
ma\d n\d n\=ar cilarca\d npai n\=ata\b nai y\=ett\=ar varuntuvat\=e.} (\textit{APA} 82)\\
\end{verse}

\normalsize
\begin{verse}
Certains habitants de la terre\index{gnl}{terre}\\
Qui ne louent pas le seigneur de Ca\d npai\index{gnl}{Canpai@Ca\d npai},\\
Bien qu'ayant vu le don\index{gnl}{don} de cymbale\index{gnl}{cymbale}s d'or\\
Marqu\'es des cinq syllabes d'estime,\\
Pensant: \og les paumes du roi\index{gnl}{roi} des habitants de Ca\d npai\index{gnl}{Canpai@Ca\d npai},\\
Qui chante en donnant la m\'elodie en frappant des mains,\\
Dans [le temple\index{gnl}{temple} de] de K\=olakk\=a\index{gnl}{Kolakka@K\=olakk\=a}\\
De Celui au front pourvu d'oeil,\\
Vont avoir mal\fg,\\
\lbrack Ces habitants-l\`a] souffrent. (\textit{APA} 82)\\
\end{verse}

\normalsize
La raison du don\index{gnl}{don} de palanquin\index{gnl}{palanquin} n'est pas expliqu\'ee dans ces textes. Nous apprenons qu'il est fait de perles (\textit{APT} 23, \textit{APMK} 12-13, \textit{APMK} 28.3, \textit{APUM} 79) et qu'il fut obtenu \`a Arattu\b rai\index{gnl}{Arattu\b rai} (\textit{APA} 40, \textit{APA} 83, \textit{APMK} 12-13). Cependant, selon l'\textit{APUM}, ce don\index{gnl}{don} a eu lieu \`a Nelv\=ayil, un site associ\'e \`a Arattu\b rai\index{gnl}{Arattu\b rai} aujourd'hui, pr\`es de Citamparam\index{gnl}{Citamparam}\footnote{Notons que dans le \textit{Periyapur\=a\d nam}\index{gnl}{Periyapuranam@\textit{Periyapur\=a\d nam}} ce sont les brahmane\index{gnl}{brahmane}s de Nelv\=ayil qui apportent ce palanquin\index{gnl}{palanquin} \`a Arattu\b rai\index{gnl}{Arattu\b rai} pour l'offrir \`a Campantar\index{gnl}{Campantar} (st. 2083-2130).}.

Le miracle\index{gnl}{miracle} de Ma\b raikk\=a\d tu\index{gnl}{Maraikkatu@Ma\b raikk\=a\d tu}, dans ces six poème\index{gnl}{poeme@poème}s, se limite \`a la fermeture des portes par Campantar\index{gnl}{Campantar}. Parfois, le toponyme est omis (\textit{APCV} 7, \textit{APK} 35, \textit{APUM} 77 et \textit{APA} 39). Notons que, contrairement \`a ce qui se passe dans le \textit{T\=ev\=aram}\index{gnl}{Tevaram@\textit{T\=ev\=aram}} V 50.8 ou dans le \textit{Periyapur\=a\d nam}\index{gnl}{Periyapuranam@\textit{Periyapur\=a\d nam}} (st. 2477-2485), il n'y a aucune allusion \`a la pr\'esence d'Appar\index{gnl}{Appar} ou au fait que ce dernier ait ouvert les portes du temple\index{gnl}{temple} que le poète\index{gnl}{poete@poète}-enfant\index{gnl}{enfant} referme.

\'Siva\index{gnl}{Siva@\'Siva} octroie des pièce\index{gnl}{piece@pièce}s d'or \`a Campantar\index{gnl}{Campantar}: \`a V\=\i \b limi\b lalai\index{gnl}{Vilimilalai@V\=\i \b limi\b lalai} il les fait pleuvoir (\textit{APK} 24 et \textit{APA} 41) et \`a \=Ava\d tutu\b rai\index{gnl}{Avatuturai@\=Ava\d tutu\b rai} il les donne par milliers (\textit{APMK} 4.17 et \textit{APUM} 80). Les raisons de ces dons\index{gnl}{don} diff\`erent. L'\'episode de V\=\i \b limi\b lalai\index{gnl}{Vilimilalai@V\=\i \b limi\b lalai} est d\'evelopp\'e uniquement dans \textit{APA} 80:

\scriptsize
\begin{verse}
\textit{p\=a\d tiya centami \b l\=a\b rpa\b la\.n k\=acu paricil pe\b r\b ra\\
n\=\i \d tiya c\=\i rttiru \~n\=a\b nacam panta\b n ni\b raipuka\b l\=a\b n\\
n\=e\d tiya p\=untiru n\=avuk karac\=o \d te\b lilmi\b lalaik\\
k\=u\d tiya ku\d t\d tatti \b n\=alu\d la t\=ayttik kuvalayam\=e.} (\textit{APA} 80)\\
\end{verse}

\normalsize
\begin{verse}
Celui \`a la gloire pleine,\\
Tiru\~n\=a\b nacampanta\b n\index{gnl}{Campantar!Tiru\~n\=a\b nacampanta\b n} \`a la gloire longue,\\
Qui obtint un cadeau d'anciennes pièce\index{gnl}{piece@pièce}s\\
Gr\^ace au beau tamoul chant\'e,\\
Avec Tirun\=avukkaracu \`a la beaut\'e estimable,\\
Par leur rassemblement, \`a Mi\b lalai la belle, \\
Le monde subsista. (\textit{APA} 80)\\
\end{verse}

\noindent Dans le \textit{Periyapur\=a\d nam}\index{gnl}{Periyapuranam@\textit{Periyapur\=a\d nam}}, Campantar\index{gnl}{Campantar} et Appar\index{gnl}{Appar} chantent devant le \'Siva\index{gnl}{Siva@\'Siva} de V\=\i \b limi\b lalai\index{gnl}{Vilimilalai@V\=\i \b limi\b lalai} pour obtenir des pièce\index{gnl}{piece@pièce}s d'or afin de nourrir les villageois souffrant de famine. Contrairement \`a celles d'Appar\index{gnl}{Appar}, les pièce\index{gnl}{piece@pièce}s acquises par Campantar\index{gnl}{Campantar} sont anciennes et causent pour cette raison un retard dans la distribution des repas. Campantar\index{gnl}{Campantar} demande alors \`a \'Siva\index{gnl}{Siva@\'Siva} d'avoir des pièce\index{gnl}{piece@pièce}s de m\^eme valeur que celles d'Appar\index{gnl}{Appar}. Son d\'esir est exauc\'e et le monde ne succombe pas \`a la famine (st. 2460-2470). Ici nous nous \'ecartons un peu de ce r\'ecit: Campantar\index{gnl}{Campantar} obtient des pièce\index{gnl}{piece@pièce}s anciennes gr\^ace \`a son chant, il est avec Appar\index{gnl}{Appar} et les deux poètes sauvent le monde.

\noindent L'\'episode d'\=Ava\d tutu\b rai\index{gnl}{Avatuturai@\=Ava\d tutu\b rai} n'est r\'esum\'e que dans \textit{APA} 85:

\scriptsize
\begin{verse}
\textit{cintaiyait t\=e\b nait tiruv\=a va\d tutu\b rai yu\d ltika\b lum\\
entaiyaip p\=a\d tal icaittut tolaiy\=a nitiyameytit\\
tantaiyait t\=\i tto\b lil m\=u\d t\d tiya k\=o\b ncara\d n c\=arvilar\=el\\
nintaiyaip pe\b r\b ro\b li y\=atiran t\=ekaram n\=\i \d t\d tuvar\=e.} (\textit{APA} 85)\\
\end{verse}

\normalsize
\begin{verse}
Il chanta [celui qui est] pens\'ee, miel,\\
Notre père\index{gnl}{pere@père} qui brille dans Tiruv\=ava\d tutu\b rai\index{gnl}{Avatuturai@\=Ava\d tutu\b rai!Tiruv\=ava\d tutu\b rai},\\
Il obtint un tr\'esor imp\'erissable,\\
Le roi\index{gnl}{roi} rattacha son père\index{gnl}{pere@père} au m\'etier du feu\index{gnl}{feu};\\
S'ils ne prennent refuge\index{gnl}{refuge} [en lui],\\
Bl\^am\'es, sans fin, suppliant ils qu\'emanderont\footnote{Litt\'eralement \og tendront la main\fg. \textsc{T. S. Gangadharan} (1929-2009) nous a guidée dans la lecture de cette strophe.}. (\textit{APA} 85)\\
\end{verse}

\normalsize
\noindent Dans le \textit{Periyapur\=a\d nam}\index{gnl}{Periyapuranam@\textit{Periyapur\=a\d nam}}, Campantar\index{gnl}{Campantar} cong\'edie son père\index{gnl}{pere@père} avec des pièce\index{gnl}{piece@pièce}s d'or pour qu'il aille conduire ses sacrifices du feu\index{gnl}{feu} \`a C\=\i k\=a\b li\index{gnl}{Cikali@C\=\i k\=a\b li} (st. 2320-2327).

Le miracle\index{gnl}{miracle} de la guérison\index{gnl}{guerison@guérison} du venin est un th\`eme appr\'eci\'e dans ces six textes. Cependant, si plusieurs passages y font bri\`evement allusion (\textit{APCV} 3, \textit{APK} 35, \textit{APUM} 76, \textit{APA} 39 et 91), seuls quatre poème\index{gnl}{poeme@poème}s placent ce prodige\index{gnl}{prodige} dans la localit\'e de Marukal\index{gnl}{Marukal} (\textit{APMK} 4.8-9, \textit{APUM} 137-138, \textit{APA} 49 et \textit{APT} 21). Campantar\index{gnl}{Campantar} sauve l'amant d'une jeune femme\index{gnl}{femme} en d\'etresse mordu par un serpent\index{gnl}{serpent}:

\scriptsize
\begin{verse}
\textit{vayal\=ar Marukal\index{gnl}{Marukal} patita\b n\b nu\d l v\=a\d larav\=a\b rka\d tiyu\d n\\
\d tayal\=a vi\b lunta ava\b nuk kira\.nki a\b riva\b linta\\
kayal\=ar karu\.nka\d n\d ni ta\b ntuyar t\=\i rtta karu\d naive\d l\d la(m)} (\textit{APA} 49a-c)\\
\end{verse}

\normalsize
\begin{verse}
Celui qui est abondance de bont\'e,\\
Compatissant pour celui qui tomba,\\
\`A c\^ot\'e, mordu par un serpent\index{gnl}{serpent} brillant,\\
Dans la ville de Marukal\index{gnl}{Marukal} pleine de rizi\`eres,\\
Supprima la souffrance\\
De celle aux yeux noirs [en forme] de poisson \textit{kayal}\\
Qui avait perdu la raison. (\textit{APA} 49a-c)\\
\end{verse}

\normalsize
Ajoutons enfin que deux miracle\index{gnl}{miracle}s mineurs reviennent souvent et bri\`evement dans ces textes: la barque\index{gnl}{barque} qui traverse le courant pour aller sur l'autre rive \`a P\=ut\=ur\index{gnl}{Kollamputur@Ko\d l\d lamp\=ut\=ur!Putur@P\=ut\=ur} et les palmier\index{gnl}{palmier}s m\^ales qui deviennent femelles \`a \=Ott\=ur\index{gnl}{Ottur@\=Ott\=ur}.

Ainsi, nous observons un traitement in\'egal des miracle\index{gnl}{miracle}s dans les textes attribu\'es \`a Nampi\index{gnl}{Nampi \=A\d n\d t\=ar Nampi}. Certains \'ev\`enements importants du \textit{Periyapur\=a\d nam}\index{gnl}{Periyapuranam@\textit{Periyapur\=a\d nam}} ne sont nullement ou \`a peine mentionn\'es dans ces six poème\index{gnl}{poeme@poème}s. Par exemple, les miracle\index{gnl}{miracle}s de Pa\d t\d t\=\i caram\index{gnl}{Patticaram@Pa\d t\d t\=\i caram} ou de la guérison\index{gnl}{guerison@guérison} du roi\index{gnl}{roi} \textit{p\=a\d n\d dya}\index{gnl}{pandya@\textit{p\=a\d n\d dya}} sont absents. Ensuite, quelques \'episodes de Maturai\index{gnl}{Maturai} ne sont \'evoqu\'es, nous l'avons dit, que dans deux textes: \textit{APUM} et \textit{APT}. M\^eme s'il y a plusieurs r\'ef\'erences \`a la victoire de Campantar\index{gnl}{Campantar} sur les ja\"in\index{gnl}{jain@ja\"in}s et \`a leur empalement\index{gnl}{empaler} au bord du fleuve Vaikai\index{gnl}{Vaikai} dans \textit{APMK} et \textit{APA}, elles ne sont situ\'ees dans aucun contexte narratif. De plus, si la reine\index{gnl}{reine} et le ministre\index{gnl}{ministre} \textit{p\=a\d n\d dya}\index{gnl}{pandya@\textit{p\=a\d n\d dya}} sont cit\'es dans \textit{APT} 48 et 49 respectivement\footnote{Les autres dévot\index{gnl}{devot(e)@dévot(e)}s apparaissant dans les textes attribu\'es \`a Nampi\index{gnl}{Nampi \=A\d n\d t\=ar Nampi} sont Ci\b rutto\d n\d tar\index{gnl}{Ciruttontar@Ci\b rutto\d n\d tar} (\textit{APA} 72 et \textit{APUM} 71), Muruka\b n\index{gnl}{Murukan@Muruka\b n} (\textit{APA} 71, \textit{APUM} 71 et \textit{APT} 72) et N\=\i lanakka\b n\index{gnl}{Nilanakkan@N\=\i lanakka\b n} (\textit{APA} 71, \textit{APUM} 71 et \textit{APT} 54).}, le roi\index{gnl}{roi} \textit{p\=a\d n\d dya}\index{gnl}{pandya@\textit{p\=a\d n\d dya}} est absent dans ces textes. Pourtant, dans le \textit{Periyapur\=a\d nam}\index{gnl}{Periyapuranam@\textit{Periyapur\=a\d nam}}, c'est la conversion au shiva\"isme\index{gnl}{shivaisme@shiva\"isme} de ce personnage cl\'e qui entra\^ine la d\'efaite et l'empalement\index{gnl}{empaler} de huit milles ja\"in\index{gnl}{jain@ja\"in}s. Enfin, seul l'APT \'evoque un des miracle\index{gnl}{miracle}s les plus c\'el\`ebres, celui de P\=ump\=avai\index{gnl}{Pumpavai@P\=ump\=avai}.

Nous constatons aussi qu'il existe des variations narratives entre les textes attribu\'es \`a Nampi\index{gnl}{Nampi \=A\d n\d t\=ar Nampi} et le \textit{Periyapur\=a\d nam}\index{gnl}{Periyapuranam@\textit{Periyapur\=a\d nam}}. Lors du miracle\index{gnl}{miracle} du lait\index{gnl}{lait} Campantar\index{gnl}{Campantar} pleure de faim dans l'\textit{APT} et dans le \textit{Tirukka\b lumalamumma\d nikk\=ovai}\index{gnl}{Tirukkalumalamummanikkovai@\textit{Tirukka\b lumalamumma\d nikk\=ovai}} alors que dans l'hagiographie\index{gnl}{hagiographie} il pleure parce qu'il ne voit plus son père\index{gnl}{pere@père} qui a plong\'e dans le bassin. Ensuite, selon \textit{APUM} le palanquin\index{gnl}{palanquin} est offert \`a Nelv\=ayil. Pourtant, tous les autres textes et le \textit{Periyapur\=a\d nam}\index{gnl}{Periyapuranam@\textit{Periyapur\=a\d nam}} s'accordent pour situer ce fait \`a Arattu\b rai\index{gnl}{Arattu\b rai}. De plus, aucun texte attribu\'e \`a Nampi\index{gnl}{Nampi \=A\d n\d t\=ar Nampi} ne mentionne la pr\'esence d'Appar\index{gnl}{Appar} \`a Ma\b raikk\=a\d tu\index{gnl}{Maraikkatu@Ma\b raikk\=a\d tu} lors de la fermeture des portes. Or, le r\^ole d'Appar est capital dans l'hagiographie\index{gnl}{hagiographie} puisque Campantar\index{gnl}{Campantar} ferme les portes qu'Appar\index{gnl}{Appar} avait ouvertes. Signalons enfin qu'un prodige\index{gnl}{prodige} \'evoqu\'e dans \textit{APT} 15-17, \textit{APUM} 75, \textit{APA} 17 et \textit{APCV} 4 n'est absolument pas d\'evelopp\'e dans le \textit{Periyapur\=a\d nam}\index{gnl}{Periyapuranam@\textit{Periyapur\=a\d nam}}: Campantar\index{gnl}{Campantar} transforme le d\'esert (\textit{p\=alai}) de Na\b nipa\d l\d li\index{gnl}{Nanipalli@Na\b nipa\d l\d li} en r\'egion c\^oti\`ere (\textit{neytal}). Signalons toutefois qu'il y a m\^eme parmi les textes attribu\'es \`a Nampi\index{gnl}{Nampi \=A\d n\d t\=ar Nampi} des variations narratives. Quand \textit{APUM} 75 et \textit{APCV} 4 mentionnent clairement ce prodige\index{gnl}{prodige}, \textit{APT} 15-17 se r\'ef\`ere \`a ces deux r\'egions ou paysages int\'erieurs pour dire que Campantar\index{gnl}{Campantar} est \og capable de chanter \textit{p\=alai} et \textit{neytal}\fg\ (\textit{p\=alaiyum neytalum p\=a\d taval\=a\b n}), c'est-\`a-dire des poème\index{gnl}{poeme@poème}s relevant de ces r\'egions qui symbolisent un \'etat psychologique d'apr\`es les conventions litt\'eraires du \textit{Ca\.nkam}\index{gnl}{Cankam@\textit{Ca\.nkam}}. Dans \textit{APA} 17 Campantar\index{gnl}{Campantar} transforme la for\^et (\textit{k\=a\b nakam}) de Na\b nipa\d l\d li\index{gnl}{Nanipalli@Na\b nipa\d l\d li} en rizi\`ere (\textit{vayal}), ce qui correspond \`a deux autres r\'egions symboliques.

Compte tenu de tous ces \'el\'ements nous supposons que Nampi \=A\d n\d t\=ar Nampi\index{gnl}{Nampi \=A\d n\d t\=ar Nampi} n'est pas l'auteur unique de ces six poème\index{gnl}{poeme@poème}s qui t\'emoignent clairement des diff\'erents stades de maturation des miracle\index{gnl}{miracle}s. Parce qu'ils mentionnent le plus grand nombre de miracle\index{gnl}{miracle}s, qu'ils sont les seuls \`a le faire pour deux \'episodes importants et parce qu'ils font r\'ef\'erence aux procédé\index{gnl}{procédé littéraire}s litt\'eraires et aux envois\index{gnl}{envoi} de Campantar\index{gnl}{Campantar} que nous avons suspect\'es, pour beaucoup, d'interpolation\index{gnl}{interpolation}, l'\textit{APUM} et, surtout, l'\textit{APT} nous semblent \^etre les textes les plus tardifs qui rel\`event d'une transmission diff\'erente de celle du \textit{Periyapur\=a\d nam}\index{gnl}{Periyapuranam@\textit{Periyapur\=a\d nam}}.


Les \'etudes secondaires menées jusqu'ici concluaient que la légende\index{gnl}{legende@légende} de Campantar\index{gnl}{Campantar}, suivant l'ordre\index{gnl}{ordre} chronologique impos\'e par la tradition\index{gnl}{tradition}, s'était form\'ee dans les textes attribu\'es \`a Nampi \=A\d n\d t\=ar Nampi\index{gnl}{Nampi \=A\d n\d t\=ar Nampi} avant de se cristalliser dans le \textit{Periyapur\=a\d nam}\index{gnl}{Periyapuranam@\textit{Periyapur\=a\d nam}}. Selon nous, la lecture attentive de ces poème\index{gnl}{poeme@poème}s \og transitoires\fg\ remet en question cette id\'ee. Nous avons vu que les six poème\index{gnl}{poeme@poème}s attribu\'es \`a Nampi\index{gnl}{Nampi \=A\d n\d t\=ar Nampi} ne r\'esultent probablement pas d'un m\^eme auteur et que \textit{APUM} et \textit{APT} peuvent parfaitement \^etre des textes parall\`eles ou post\'erieurs au \textit{Periyapur\=a\d nam}\index{gnl}{Periyapuranam@\textit{Periyapur\=a\d nam}}. Sous la plume qu'aurait tenue Nampi \=A\d n\d t\=ar Nampi\index{gnl}{Nampi \=A\d n\d t\=ar Nampi} nous observons le d\'eveloppement de nouveaux \'el\'ements biographiques\index{gnl}{biographie!biographique} de la légende\index{gnl}{legende@légende}: Campantar est ainsi l'ennemi, par excellence, des ja\"in\index{gnl}{jain@ja\"in}s qu'il combat au bord de la Vaikai\index{gnl}{Vaikai} et qu'il fait empaler\index{gnl}{empaler}.
La lecture des textes mentionn\'es dans le chapitre 4 nous conduit d'autre part \`a constater une rupture entre les poète\index{gnl}{poete@poète}s du \textit{T\=ev\=aram}\index{gnl}{Tevaram@\textit{T\=ev\=aram}} et ceux du \textit{Tirumu\b rai}\index{gnl}{Tirumurai@\textit{Tirumu\b rai}} \textsc{xi} dans la pr\'esentation de Campantar\index{gnl}{Campantar}: ce dernier, poète\index{gnl}{poete@poète} tamoul de K\=a\b li\index{gnl}{Kali@K\=a\b li} pour Appar\index{gnl}{Appar} et Cuntarar\index{gnl}{Cuntarar}, et m\^eme dans les \'ecrits qui lui sont attribu\'es, devient l'enfant\index{gnl}{enfant} prodige\index{gnl}{prodige} pour Pa\d t\d ti\b nattuppi\d l\d lai\index{gnl}{Pattinattu Pillai@Pa\d t\d ti\b nattuppi\d l\d lai} et Nampi \=A\d n\d t\=ar Nampi\index{gnl}{Nampi \=A\d n\d t\=ar Nampi} et entre ainsi dans la légende\index{gnl}{legende@légende}: c'est, plus ou moins, \`a cette \'epoque que l'image\index{gnl}{image} cultuelle de Campantar\index{gnl}{Campantar} représentant le poète en enfant int\`egre l'espace sacr\'e du temple\index{gnl}{temple} de Ta\~nc\=av\=ur\index{gnl}{Tancavur@Ta\~nc\=av\=ur} sous le patronage de R\=ajar\=aja I\index{gnl}{Rajaraja I@R\=ajar\=aja I} (SII 2 38).

\section{\`A l'origine des images}

D'après les donn\'ees arch\'eologiques et \'epigraphiques à notre disposition la repr\'esentation plastique du poète\index{gnl}{poete@poète} Campantar\index{gnl}{Campantar} ne semble appara\^itre dans l'enceinte sacr\'ee du temple\index{gnl}{temple} qu'\`a la fin du \textsc{x}\up{e} si\`ecle pour conna\^itre son apog\'ee au \textsc{xiii}\up{e} si\`ecle quand les installation\index{gnl}{installation d'une image}s d'image\index{gnl}{image}s, de chapelles et de monast\`ere\index{gnl}{monastère}s \'etablis en son honneur se multiplient dans l'ensemble du Pays Tamoul\index{gnl}{Pays Tamoul} (voir notamment \textsc{Dehejia} 1987 et \textsc{Orr} 2009).
Dans cette sous-partie, consacr\'ee \`a l'iconographie de Campantar\index{gnl}{Campantar}, nous examinons les image\index{gnl}{image}s les plus anciennes disponibles pour ensuite sonder leur origine\index{gnl}{origine} iconographique et, enfin, les mettre en rapport avec les textes litt\'eraires.

\subsection{Les images}

Aujourd'hui, dans les temples\index{gnl}{temple} shiva\"ite\index{gnl}{shiva\"ite}s les représentations iconographiques des soixante-trois \textit{n\=aya\b nm\=ar}\index{gnl}{nayanmar@\textit{n\=aya\b nm\=ar}} sont fréquentes. Elles habillent souvent les galeries int\'erieures sud des murs d'enceinte. En ce qui concerne Campantar\index{gnl}{Campantar}, il s'agit d'un dévot\index{gnl}{devot(e)@dévot(e)} tr\`es repr\'esent\'e, surtout dans les sites mentionn\'es par son hagiographie\index{gnl}{hagiographie}. \`A C\=\i k\=a\b li\index{gnl}{Cikali@C\=\i k\=a\b li}, par exemple, ses image\index{gnl}{image}s \og modernes\fg, peintes et stuqu\'ees, foisonnent. Dans le cadre de notre \'etude sur leur origine\index{gnl}{origine}, nous étudions les image\index{gnl}{image}s les plus anciennes disponibles: celles, souvent perdues, qui sont d\'ecrites dans les inscriptions et celles qui subsistent dans les temples et les musées.

\subsubsection{Les image\index{gnl}{image}s \'epigraphiques}

Nous avons vu dans le premier chapitre (1.3) que les \'epigraphes mentionnent d\`es la fin de la p\'eriode dite \textit{pallava}\index{gnl}{pallava@\textit{pallava}} au Pays Tamoul (\textsc{vi}\up{e}-\textsc{ix}\up{e} siècle) des dons\index{gnl}{don} pour établir ou maintenir les récitation\index{gnl}{recitation@récitation}s des \textit{tiruppatiyam}\index{gnl}{tiruppatiyam@\textit{tiruppatiyam}}. Rappelons que ces \textit{tiruppatiyam}\index{gnl}{tiruppatiyam@\textit{tiruppatiyam}} peuvent renvoyer aux chants\index{gnl}{chant}\index{gnl}{chant} de n'importe quelle secte et qu'ils englobent fort probablement des hymne\index{gnl}{hymne}s appartenant au corpus\index{gnl}{corpus} actuel du \textit{T\=ev\=aram}\index{gnl}{Tevaram@\textit{T\=ev\=aram}}. Peu d'attestations de lien entre un chant et son auteur nous sont parvenues. Parall\`element, dans les si\`ecles suivants, des inscriptions \'evoquent des offrandes d'image\index{gnl}{image}s de dévot\index{gnl}{devot(e)@dévot(e)}s, dont les auteurs pr\'esum\'es du \textit{T\=ev\=aram}\index{gnl}{Tevaram@\textit{T\=ev\=aram}}. Ainsi, la premi\`ere r\'ef\'erence, datable et disponible, \`a une image\index{gnl}{image} de Campantar\index{gnl}{Campantar} se trouve dans une inscription, d\'ej\`a \'etudi\'ee en 1.3, du temple\index{gnl}{temple} de Ta\~nc\=av\=ur\index{gnl}{Tancavur@Ta\~nc\=av\=ur} (SII 2 38) et date du r\`egne de R\=ajar\=aja I\index{gnl}{Rajaraja I@R\=ajar\=aja I} (985-1014). Cette repr\'esentation, offerte avec celles de Cuntarar\index{gnl}{Cuntarar}, de son \'epouse Paravai\index{gnl}{Paravai}, d'Appar\index{gnl}{Appar}, du roi\index{gnl}{roi}, de la reine\index{gnl}{reine} et de Candra\'sekhara\index{gnl}{Candra\'sekhara}, est un don\index{gnl}{don} royal\footnote{D'autres image\index{gnl}{image}s de \textit{n\=aya\b nm\=ar}\index{gnl}{nayanmar@\textit{n\=aya\b nm\=ar}} ont \'et\'e offertes \`a Ta\~nc\=av\=ur\index{gnl}{Tancavur@Ta\~nc\=av\=ur}: Ca\d n\d ti\index{gnl}{Candesa@Ca\d n\d de\'sa!Ca\d n\d ti} (SII 2 29), Meypporu\d l\index{gnl}{Meypporu\d l} (SII 2 40) et Ci\b rutto\d n\d tar\index{gnl}{Ciruttontar@Ci\b rutto\d n\d tar} (SII 2 43).}. Elle est nomm\'ee Tiru\~n\=a\b nacampanta\d tika\d l (l. 25), poss\`ede deux bras et est par\'ee de divers bijoux dont, \`a la diff\'erence des autres image\index{gnl}{image}s de la donation, une ceinture \textit{tiruppa\d tikai} (l. 46) qui est un ornement port\'e en g\'en\'eral par les femme\index{gnl}{femme}s et les enfant\index{gnl}{enfant}s (\textsc{Subbarayalu} 2002-3). Nous pouvons ainsi supposer que ce bronze du d\'ebut du \textsc{xi}\up{e} si\`ecle figure Campantar\index{gnl}{Campantar} sous les traits d'un enfant\index{gnl}{enfant}.
Quelques ann\'ees plus tard, une inscription non publi\'ee du temple\index{gnl}{temple} de Vaittiyan\=ata\b n \`a Ma\b lav\=a\d ti\index{gnl}{Tirumalapati@Tiruma\b lap\=a\d ti!Ma\b lav\=a\d ti} (Tirucci\index{gnl}{Tirucci dt.} dt.), ARE 1920 37, datant de 1032\footnote{Cette datation nous para\^it fort probable car le rapport pr\'ecise qu'il s'agit de la vingti\`eme ann\'ee de r\`egne de R\=ajendra identifi\'e comme R\=ajendra I\index{gnl}{Rajendra I@R\=ajendra I} (1012-1044) gr\^ace au \textit{meykk\=\i rtti}\index{gnl}{meykkirtti@\textit{meykk\=\i rtti}} d\'ebutant par \og \textit{tirumanniva\d lara}\fg\ (pour une version de cet éloge\index{gnl}{eloge@éloge} voir \textsc{Cuppirama\d niyam} (1983: 26)).}, mentionne l'installation\index{gnl}{installation d'une image} des image\index{gnl}{image}s des \textit{m\=uvar}\index{gnl}{muvar@\textit{m\=uvar}} nomm\'es \og Pi\d l\d laiy\=ar Tiruj\~n\=ana\'sambanda\d diga\d l\fg, \og Tirun\=avukkaraiyad\=eva\fg\ et \og Nambi \=Ar\=uran\=ar\fg. Si le terme \og pi\d l\d laiy\=ar\fg\ est r\'eellement employ\'e dans l'\'epigraphe, il renvoie certainement \`a une image\index{gnl}{image} du poète\index{gnl}{poete@poète}-enfant\index{gnl}{enfant}. Ces deux inscriptions soulignent le fait que d\`es le d\'ebut du \textsc{xi}\up{e} si\`ecle les trois auteurs du corpus\index{gnl}{corpus} actuel du \textit{T\=ev\=aram}\index{gnl}{Tevaram@\textit{T\=ev\=aram}} \'etaient associ\'es.

C'est v\'eritablement \`a partir du \textsc{xii}\up{e} si\`ecle que les r\'ef\'erences aux image\index{gnl}{image}s de Campantar\index{gnl}{Campantar} commencent \`a abonder dans les inscriptions, surtout dans les sites li\'es au poète\index{gnl}{poete@poète} selon l'hagiographie\index{gnl}{hagiographie}. Attardons-nous bri\`evement sur les corpus\index{gnl}{corpus} \'epigraphiques de C\=\i k\=a\b li\index{gnl}{Cikali@C\=\i k\=a\b li}, d'\=Acc\=a\d lpuram\index{gnl}{Accalpuram@\=Acc\=a\d lpuram} et de Na\b nipa\d l\d li\index{gnl}{Nanipalli@Na\b nipa\d l\d li} par exemple.
\`A C\=\i k\=a\b li\index{gnl}{Cikali@C\=\i k\=a\b li}, lieu de naissance\index{gnl}{naissance} du poète\index{gnl}{poete@poète}, \`a proximit\'e du temple\index{gnl}{temple} principal de T\=o\d nipuram\index{gnl}{Tonipuram@T\=o\d nipuram} U\d taiy\=ar, un templion ind\'ependant \'etait destin\'e au culte\index{gnl}{culte} de Campantar\index{gnl}{Campantar} au \textsc{xii}\up{e} si\`ecle. Le CEC nous apprend qu'un culte\index{gnl}{culte} quotidien avec offrande de nourriture était rendu (CEC 25, 27, 29, 31), que des hymne\index{gnl}{hymne}s étaient récités (CEC 26), que des f\^etes\index{gnl}{fete@fête} annuelles (CEC 31) étaient célébrées et que la chapelle poss\'edait de nombreuses terres\index{gnl}{terre} dans le voisinage (CEC 27, 28, 29, 30, 31, 32). L'\'epigraphe CEC 26 (voir 1.3) t\'emoigne de l'existence d'un culte\index{gnl}{culte} des hymne\index{gnl}{hymne}s \'etablis en un corpus\index{gnl}{corpus} nomm\'e \textit{Tirumu\b rai}\index{gnl}{Tirumurai@\textit{Tirumu\b rai}} dans ce templion en 1136.
Ensuite, \`a une dizaine de kilom\`etres au Nord-Est s'\'el\`eve le temple\index{gnl}{temple} de \'Sivalokaty\=age\'sa \`a \=Acc\=a\d lpuram\index{gnl}{Accalpuram@\=Acc\=a\d lpuram} o\`u aurait eu lieu le mariage\index{gnl}{mariage} de Campantar\index{gnl}{Campantar} durant lequel ce dernier entra dans le feu\index{gnl}{feu} sacrificiel accompagn\'e de son \'epouse et de tous les convives pour rejoindre les pieds de \'Siva\index{gnl}{Siva@\'Siva}\footnote{Les inscriptions de ce temple\index{gnl}{temple} ont \'et\'e relev\'ees et r\'esum\'ees dans les ARE 1918 522 \`a 540. En 2006, nous avons relev\'e \`a nouveau ces \'epigraphes que nous avons lues \textit{in situ} avec l'aide du professeur \textsc{G. Vijayavenugopal}. Nous pr\'eparons une \'edition de ces textes in\'edits.}. Nous constatons que la chapelle d\'edi\'ee \`a Campantar\index{gnl}{Campantar} et \`a son \'epouse Cokkiy\=ar\index{gnl}{Cokkiyar@Cokkiy\=ar} y existe d\`es la fin du \textsc{xii}\up{e} si\`ecle. Des terres\index{gnl}{terre} sont donn\'ees pour \'etablir un monast\`ere\index{gnl}{monastère}, une rue de procession\index{gnl}{procession} et un jardin \`a fleurs (ARE 1918 531). Une autre inscription du même site, ARE 1918 527, enregistre une offrande de terre\index{gnl}{terre} pour nourrir Campantar\index{gnl}{Campantar} et son \'epouse. Elle \'evoque aussi \`a la premi\`ere ligne des villages au sud de la K\=av\=eri\index{gnl}{Kaveri@K\=av\=eri} dans lesquels les image\index{gnl}{image}s de Campantar\index{gnl}{Campantar} et de Cokkiy\=ar\index{gnl}{Cokkiyar@Cokkiy\=ar} partent en procession\index{gnl}{procession} (Ve\.nk\=a\d tu\index{gnl}{Venkatu@Ve\.nk\=a\d tu}, Na\b nipa\d l\d li\index{gnl}{Nanipalli@Na\b nipa\d l\d li}, \=Akk\=ur\index{gnl}{Akkur@\=Akk\=ur} et Citamparam\index{gnl}{Citamparam}) avant d'effectuer un tour de leur village, \textit{kir\=amappiratek\d sa\d nam} (sk. \textit{gr\=amapradak\d si\d na}).
Enfin, Na\b nipa\d l\d li\index{gnl}{Nanipalli@Na\b nipa\d l\d li}, aussi appel\'e Pu\~ncai, est un site marqu\'e, d\`es la seconde moiti\'e du \textsc{x}\up{e} si\`ecle (ARE 1925 192), par des inscriptions. Beaucoup d'entre elles contiennent, souvent, des éloge\index{gnl}{eloge@éloge}s royaux. Si les r\'ef\'erences \`a Campantar\index{gnl}{Campantar} ne sont pas nombreuses dans ce corpus\index{gnl}{corpus} l'\'epigraphe qui nous int\'eresse apporte des informations substantielles le concernant. Cette inscription non publi\'ee (ARE 1925 180) et grav\'ee sur le mur nord du temple\index{gnl}{temple}, enregistre la création d'un lot de terres\index{gnl}{terre} appel\'e Tiru\~n\=a\b nacampantanall\=ur (l. 3) \`a partir de parcelles confisqu\'ees (?) \`a des vishnouites (\textit{vi\d s\d nukka\d lai m\=a\b ri\b na nilattile}, l. 2) pour assurer les culte\index{gnl}{culte}s rendus \`a l'image\index{gnl}{image} de Campantar\index{gnl}{Campantar} install\'ee dans la vieille demeure o\`u il serait n\'e (\textit{pi\d l\d laiy\=ar tiruvot\=aram pa\d n\d niyaru\d li\b na pa\b la m\=a\d tattile}, l. 2)\footnote{Nous remercions Charlotte \textsc{Schmid} de nous avoir signal\'e cette inscription et de nous avoir fourni une version du corpus\index{gnl}{corpus} in\'edit de Pu\~ncai qu'elle \'edite en collaboration avec \textsc{G. Vijayavenugopal}.}. Ce texte, ne comportant pas d'éloge royal, date de la douzi\`eme ann\'ee de r\`egne d'un Kulottu\.nga. Aucune information interne ou externe
%\footnote{Voir si son emplacement par rapport aux autres inscriptions sur le mur nord permet de lui donner un date relative approximativeXXXXX}
\`a l'\'epigraphe ne nous permet dans l'imm\'ediat de pr\'eciser l'identit\'e\index{gnl}{identit\'e} du roi\index{gnl}{roi}. Il peut s'agir de Kulottu\.nga I\index{gnl}{Kulottu\.nga I}, II ou III; le texte peut ainsi dater respectivement de 1082, de 1145 ou de 1190. L'image\index{gnl}{image} install\'ee de Campantar\index{gnl}{Campantar} est celle d'un enfant\index{gnl}{enfant} d\'esign\'e par les termes \textit{pi\d l\d laiy\=ar} ou \textit{u\d taiya pi\d l\d laiy\=ar}\footnote{Rappelons qu'\`a C\=\i k\=a\b li\index{gnl}{Cikali@C\=\i k\=a\b li} ou \`a \=Acc\=a\d lpuram\index{gnl}{Accalpuram@\=Acc\=a\d lpuram} il est nomm\'e \=A\d lu\d taiyapi\d l\d laiy\=ar\index{gnl}{Campantar!Alutaiyapillaiyar@\=A\d lu\d taiyapi\d l\d laiy\=ar}.}.

Ajoutons enfin que les monast\`ere\index{gnl}{monastère}s nomm\'es d'apr\`es Campantar\index{gnl}{Campantar} apparaissent au \textsc{xii}\up{e} si\`ecle\footnote{\`A \=Acc\=a\d lpuram\index{gnl}{Accalpuram@\=Acc\=a\d lpuram} par exemple, en 1121, il existait un monast\`ere\index{gnl}{monastère} nomm\'e d'apr\`es Paracamayak\=o\d lari, un titre de Campantar\index{gnl}{Campantar} signifiant le \og lion (contre) les h\'er\'esies\fg, o\`u venaient se restaurer les dévot\index{gnl}{devot(e)@dévot(e)}s \textit{mahe\'svara} (ARE 1918 534).} et se multiplient dans tout le Pays Tamoul\index{gnl}{Pays Tamoul} \`a partir du \textsc{xiii}\up{e} si\`ecle; voir \textsc{Swamy} (1972: 113-115).

Ainsi, la r\'ef\'erence \'epigraphique la plus ancienne, disponible et datable, \`a une image\index{gnl}{image} de Campantar\index{gnl}{Campantar} remonte au d\'ebut du \textsc{xi}\up{e} si\`ecle. Le poète\index{gnl}{poete@poète} rentre dans l'enceinte sacr\'e du temple\index{gnl}{temple} sur ordre\index{gnl}{ordre royal} royal, il est associ\'e aux deux autres \textit{m\=uvar}\index{gnl}{muvar@\textit{m\=uvar}} et semble avoir \'et\'e figur\'e sous les traits d'un enfant\index{gnl}{enfant}. Examinons maintenant les repr\'esentations iconographiques disponibles.

\subsubsection{L'iconographie de Campantar}

Nous traiterons successivement de la premi\`ere image\index{gnl}{image} pr\'esum\'ee en pierre de Campantar\index{gnl}{Campantar}, des bronzes types et des frises narratives.

La plus ancienne repr\'esentation iconographique en pierre de Campantar\index{gnl}{Campantar}, qui daterait du milieu du \textsc{x}\up{e} si\`ecle, se trouve sur le mur sud du temple\index{gnl}{temple} de Vasi\d s\d the\'svara \`a Karunta\d t\d t\=a\.nku\d ti\index{gnl}{Karuntattankuti@Karunta\d t\d t\=a\.nku\d ti} (Karuntai\index{gnl}{Karuntattankuti@Karunta\d t\d t\=a\.nku\d ti!Karuntai}, Ta\~nc\=av\=ur\index{gnl}{Tancavur@Ta\~nc\=av\=ur} dt.) selon \textsc{L. Orr} (2009) qui reprend des informations donn\'ees par \textsc{P. R. Srinivasan} dans \textit{Important Works of Art of early Chola period from near Tanjore} in \textit{Transactions for the year 1956-57 of the Archaeological Society of South India}, vol. II, p. 56-59, fig. 10 et 11, Madras\footnote{\textsc{P. R. Srinivasan} (*1994 [1963]: 225) revient bri\`evement sur ces image\index{gnl}{image}s de pierre des hymnistes et pr\'ecise qu'elles sont ant\'erieures au \textsc{xi}\up{e} si\`ecle de quelques d\'ecennies.}. L'examen de l'\'epigraphie de ce site nous permet d'\'elargir l'estimation de la datation de cette première image\index{gnl}{image} de Campantar\index{gnl}{Campantar}. En effet, les inscriptions aujourd'hui disponibles sur ce temple\index{gnl}{temple}\footnote{Les relev\'es 42 \`a 51 de l'ARE 1897 ainsi que \textsc{Mahalingam} (1992: 581-585) pr\'esentent des r\'esum\'es de ces \'epigraphes publi\'ees dans SII 5 1405-1414. Signalons que c'est dans cette localit\'e que furent exhum\'ees les plaques de cuivre dites de Karantai (ARE 1949-50 p. 3-5) datant de la huiti\`eme ann\'ee de r\`egne de R\=ajendra I\index{gnl}{Rajendra I@R\=ajendra I}, soit de 1020.} commencent d'être gravées \`a partir de 909 (SII 5 1412). De plus, une visite du site en avril 2011 nous a permis de constater que les image\index{gnl}{image}s d'Appar\index{gnl}{Appar} et de Campantar\index{gnl}{Campantar} encadrent un \'Siva\index{gnl}{Siva@\'Siva} dansant et que le mur entre ce \'Siva\index{gnl}{Siva@\'Siva} dansant et Campantar\index{gnl}{Campantar} porte l'\'epigraphe \'edit\'ee dans SII 5 1407 datant de la troisi\`eme ann\'ee de r\`egne de R\=ajendra I\index{gnl}{Rajendra I@R\=ajendra I}, soit 1015 et qui fait \'etat d'une vente de terre\index{gnl}{terre} au temple\index{gnl}{temple} par l'assemblée\index{gnl}{assemblée} villageoise en \'echange de soixante-quinze pièce\index{gnl}{piece@pièce}s (\textit{k\=acu}). Cette inscription se poursuit sur le mur \`a gauche de Campantar\index{gnl}{Campantar}, juste au-dessus du \textit{li\.nga}\index{gnl}{linga@\textit{li\.nga}} que ce dernier honore. Cette position du texte grav\'e sur le mur, par rapport aux image\index{gnl}{image}s de Campantar\index{gnl}{Campantar} et du \textit{li\.nga}\index{gnl}{linga@\textit{li\.nga}}, montre que l'\'epigraphe est post\'erieure \`a ces repr\'esentations. Ainsi, nous pouvons supposer que ces image\index{gnl}{image}s de poète\index{gnl}{poete@poète}s, effectivement les plus anciennes disponibles, sont datables entre le \textsc{x}\up{e} et le d\'ebut du \textsc{xi}\up{e} si\`ecle. Ce site pionnier dans la repr\'esentation des poète\index{gnl}{poete@poète}s shiva\"ite\index{gnl}{shiva\"ite}s, ayant b\'en\'efici\'e de donations de la famille royale (SII 5 1405 et 1409), reste pourtant silencieux dans ses inscriptions sur les \textit{tirupatiyam}, leur récitation\index{gnl}{recitation@récitation}, etc. Notons aussi qu'il n'est pas non plus c\'el\'ebr\'e dans le corpus\index{gnl}{corpus} actuel du \textit{T\=ev\=aram}\index{gnl}{Tevaram@\textit{T\=ev\=aram}}.

\noindent Tournons-nous vers les image\index{gnl}{image}s de ces poète\index{gnl}{poete@poète}s. Sur le mur sud, dans le sens de la circumambulation, sont plac\'ees les repr\'esentations d'un \textit{li\.nga}\index{gnl}{linga@\textit{li\.nga}}, de Campantar\index{gnl}{Campantar}, de \'Siva\index{gnl}{Siva@\'Siva} dansant, d'Appar\index{gnl}{Appar} et de \'Siva\index{gnl}{Siva@\'Siva} mendiant dans la for\^et de pins. Les image\index{gnl}{image}s n'ont pas la m\^eme dimension. Les formes de \'Siva\index{gnl}{Siva@\'Siva} sont deux \`a trois fois plus grandes que celles des poète\index{gnl}{poete@poète}s et du \textit{li\.nga}\index{gnl}{linga@\textit{li\.nga}}, plus petit que ces derniers.
Appar\index{gnl}{Appar} est debout, v\^etu d'un simple cache-sexe (fig. 5.2). Il porte la tonsure. Deux rosaires ornent son front et son torse. Il est pourvu de deux bras; la main gauche tient le manche d'une houe pos\'ee sur l'épaule\index{gnl}{epaule@épaule} gauche et la droite serr\'ee pointe le pouce et l'index au milieu du torse comme s'il m\'editait. Nous apercevons d'ailleurs un petit \textit{li\.nga}\index{gnl}{linga@\textit{li\.nga}} pos\'e \`a sa gauche, \`a la hauteur de ses cuisses. Des feuillages comblent l'arri\`ere-plan. Appar\index{gnl}{Appar} semble m\'editer devant un \textit{li\.nga}\index{gnl}{linga@\textit{li\.nga}} dans la for\^et, peut-\^etre dans la for\^et de cèdres o\`u se prom\`ene le \'Siva\index{gnl}{Siva@\'Siva} qui se tient \`a sa droite.
Campantar\index{gnl}{Campantar}, de la m\^eme taille qu'Appar\index{gnl}{Appar}, est plac\'e \`a gauche du \'Siva\index{gnl}{Siva@\'Siva} dansant (fig. 5.3). Il est debout, v\^etu, pareillement, d'un cache-sexe. Il porte aussi la tonsure et est par\'e de deux rosaires. Son physique et ses ornements sont pareil à ceux d'Appar\index{gnl}{Appar}. Inclin\'e vers sa gauche il joue de la cymbale\index{gnl}{cymbale} en regardant le \textit{li\.nga}\index{gnl}{linga@\textit{li\.nga}} du panneau suivant. Campantar\index{gnl}{Campantar} semble marquer le rythme de ses chants\index{gnl}{chant}\index{gnl}{chant} adress\'es au \textit{li\.nga}\index{gnl}{linga@\textit{li\.nga}} qui accompagnent aussi la danse\index{gnl}{danse} du \'Siva\index{gnl}{Siva@\'Siva} se trouvant \`a sa droite. Aucun attribut, parure ou \'el\'ement corporel ne le caract\'erise comme un enfant\index{gnl}{enfant}. L'image\index{gnl}{image} de Campantar\index{gnl}{Campantar}, faisant \'echo à celle d'Appar\index{gnl}{Appar}, est celle d'un dévot\index{gnl}{devot(e)@dévot(e)} adulte qui honore \'Siva\index{gnl}{Siva@\'Siva} au son de ses cymbale\index{gnl}{cymbale}s.

\begin{figure}[!h]
  %\begin{minipage}[c]{0.5\textwidth}
  \centering
  \includegraphics[height=8cm]{docthese/KARUNTAITNC/SL2011132.JPG}
  \caption{Appar portant une houe à main, face sud, temple de Vasi\d s\d the\'svara \`a Karunta\d t\d t\=a\.nku\d ti (cliché U. \textsc{Veluppillai}, 2011).}
  \end{figure}
  %\end{minipage}
  %\begin{minipage}[c]{0.5\textwidth}
  \begin{figure}[!h]
  \centering
  \includegraphics[height=8cm]{docthese/KARUNTAITNC/SL2011130.JPG}
  \caption{Campantar jouant des cymbales, face sud, temple de Vasi\d s\d the\'svara \`a Karunta\d t\d t\=a\.nku\d ti (cliché U. \textsc{Veluppillai}, 2011).}
  %\end{minipage}
  \end{figure}

\noindent Ainsi, nous constatons que la plus ancienne image\index{gnl}{image} disponible de Campantar\index{gnl}{Campantar}, provenant de la r\'egion de Ta\~nc\=av\=ur\index{gnl}{Tancavur@Ta\~nc\=av\=ur}, ne le repr\'esente pas sous l'aspect d'un enfant\index{gnl}{enfant}. Nous supposons donc que la légende\index{gnl}{legende@légende} de l'enfant\index{gnl}{enfant} poète\index{gnl}{poete@poète} Campantar\index{gnl}{Campantar} se d\'eveloppe pleinement apr\`es le d\'ebut du \textsc{xi}\up{e} si\`ecle, p\'eriode \`a partir de laquelle ses image\index{gnl}{image}s de bronze, le repr\'esentant sous les traits d'un enfant\index{gnl}{enfant}, sont install\'ees dans les temples\index{gnl}{temple}.

%LES BRONZES TYPES

Les nombreuses image\index{gnl}{image}s en bronze de Campantar\index{gnl}{Campantar} disponibles aujourd'hui, celles qui d\'efilent lors des procession\index{gnl}{procession}s de f\^etes\index{gnl}{fete@fête} annuelles dans les temples\index{gnl}{temple}, le repr\'esentent toujours sous les traits d'un enfant\index{gnl}{enfant}.
Campantar\index{gnl}{Campantar} est nu, il est par\'e d'une ceinture \`a pendeloque et porte parfois la double bandouli\`ere (sk. \textit{channav\=\i ra}?). Sa coiffure varie: il peut porter la tonsure, avoir les cheveux courts ou \^etre couronn\'e d'une tiare. Suivant sa position et ses attributs nous pouvons d\'efinir trois types d'image\index{gnl}{image}s en bronze. Campantar\index{gnl}{Campantar} peut \^etre repr\'esent\'e debout tenant des cymbale\index{gnl}{cymbale}s dans les mains\footnote{Ce type reprend le mod\`ele de son image\index{gnl}{image} en pierre de Karuntai\index{gnl}{Karuntattankuti@Karunta\d t\d t\=a\.nku\d ti!Karuntai}.}. Ensuite, il peut \^etre debout, portant une coupelle\index{gnl}{coupelle} dans une main et pointant le ciel avec l'index de l'autre main\footnote{Cf. l'image\index{gnl}{image} de procession\index{gnl}{procession} actuelle de C\=\i k\=a\b li\index{gnl}{Cikali@C\=\i k\=a\b li}. Pour des image\index{gnl}{image}s qui ont \'et\'e dat\'ees du \textsc{xi}\up{e} si\`ecle cf. \textsc{Dehejia} (1987: 54), \textsc{Dehejia} (2002: 153) et \textsc{Srinivasan} (*1994 [1963]: fig. 125).}. Enfin, il peut danse\index{gnl}{danser}r, une jambe fl\'echie suspendue et l'autre fl\'echie \`a terre\index{gnl}{terre}, \`a la mani\`ere de K\textsubring{r}\d s\d na\index{gnl}{Krsna@K\textsubring{r}\d s\d na} sur le serpent\index{gnl}{serpent} K\=aliya\index{gnl}{Kaliya@K\=aliya}, en pointant le ciel de son index droit et en gardant le bras gauche pendant selon le geste\index{gnl}{geste} du \textit{gajahasta} (fig. 5.4)\footnote{L'une des premi\`eres image\index{gnl}{image}s de Campantar\index{gnl}{Campantar} dansant, datant du \textsc{xi}\up{e} si\`ecle, serait celle de Tiruve\.nk\=a\d tu\index{gnl}{Venkatu@Ve\.nk\=a\d tu!Tiruve\.nk\=a\d tu} expos\'ee \`a l'Art Gallery de Ta\~nc\=av\=ur\index{gnl}{Tancavur@Ta\~nc\=av\=ur} (\textsc{Dehejia} *2002 [1988]: fig. 16).}.

\noindent Ainsi, \`a partir du \textsc{xi}\up{e} si\`ecle, ces repr\'esentations en bronze mettent en sc\`ene, par leur attribut (la coupelle\index{gnl}{coupelle}) et leur gestuelle (l'index pointant le ciel), le Campantar\index{gnl}{Campantar} enfant\index{gnl}{enfant} de la légende\index{gnl}{legende@légende}. Au si\`ecle suivant, la légende\index{gnl}{legende@légende} est \'ecrite et d\'epeinte en d\'etail sur des frises de murs de temples\index{gnl}{temple}.

%LES FRISES NARRATIVES

Les petits encadr\'es des soubassements des temples\index{gnl}{temple} de M\=elakka\d tamp\=ur\index{gnl}{Melakkatampur@M\=elakka\d tamp\=ur} et de T\=ar\=acuram\index{gnl}{Taracuram@T\=ar\=acuram}, formant des frises narratives, illustrent les exploits des soixante-trois dévot\index{gnl}{devot(e)@dévot(e)}s shiva\"ite\index{gnl}{shiva\"ite}s. Campantar\index{gnl}{Campantar} est repr\'esent\'e \`a plusieurs reprises: dans l'\'episode du don\index{gnl}{don} de lait\index{gnl}{lait} et en compagnie d'autres \textit{n\=aya\b nm\=ar}\index{gnl}{nayanmar@\textit{n\=aya\b nm\=ar}} qu'il rencontre selon son hagiographie\index{gnl}{hagiographie}. Rappelons que les image\index{gnl}{image}s de M\=elakka\d tamp\=ur\index{gnl}{Melakkatampur@M\=elakka\d tamp\=ur} sont ant\'erieures \`a celles de T\=ar\=acuram\index{gnl}{Taracuram@T\=ar\=acuram} et qu'elles semblent suivre une version l\'egendaire diff\'erente de celle transmise par le \textit{Periyapur\=a\d nam}\index{gnl}{Periyapuranam@\textit{Periyapur\=a\d nam}} qui est repr\'esent\'ee \`a T\=ar\=acuram\index{gnl}{Taracuram@T\=ar\=acuram} (voir 4.3.3).
Sur le panneau 22, pour reprendre la num\'erotation de \textsc{L'Hernault} (1987: 96-107), \`a M\=elakka\d tamp\=ur\index{gnl}{Melakkatampur@M\=elakka\d tamp\=ur}, un personnage adulte habill\'e d'un manteau se tient pr\`es d'un \^etre plus petit, nu et pourvu de cymbale\index{gnl}{cymbale}s \`a la main, certainement l'enfant\index{gnl}{enfant} Campantar\index{gnl}{Campantar}. \`A T\=ar\=acuram\index{gnl}{Taracuram@T\=ar\=acuram} est repr\'esent\'ee l'épreuve\index{gnl}{epreuve@épreuve} de l'eau\index{gnl}{eau} dans laquelle le manuscrit\index{gnl}{manuscrit} de Campantar\index{gnl}{Campantar} remonte le courant \`a contre-sens alors que ceux des ja\"in\index{gnl}{jain@ja\"in}s sont emport\'es par les flots\index{gnl}{flots} de la Vaikai\index{gnl}{Vaikai}. Sur l'image\index{gnl}{image}, d'un c\^ot\'e du fleuve se trouve un personnage adulte, le ministre\index{gnl}{ministre} Kulacci\b rai\index{gnl}{kulaccirai@Kulacci\b rai}, les mains jointes en adoration devant l'enfant\index{gnl}{enfant} Campantar\index{gnl}{Campantar}, tenant un objet dans la main. \textsc{L'Hernault} y voit un manuscrit\index{gnl}{manuscrit} mais ce n'est pas clair; il peut aussi s'agir de cymbale\index{gnl}{cymbale}s. Sur l'autre rive, quatre personnages asc\'etiques, des ja\"in\index{gnl}{jain@ja\"in}s, tiennent dans leur main gauche un objet que \textsc{L'Hernault} identifie, ici encore, comme un manuscrit\index{gnl}{manuscrit}. Cependant, une des extr\^emit\'es de cet objet est arrondie et rappelle plut\^ot la plume de paon (ou un autre attribut des asc\`etes ja\"in\index{gnl}{jain@ja\"in}s). Le dernier de ces quatre personnages se dirige vers sa condamnation, la mort par emplament, repr\'esent\'ee par un groupe de quatre ja\"in\index{gnl}{jain@ja\"in}s empal\'es\index{gnl}{empaler}.
Le panneau 28 illustre l'\'episode du don\index{gnl}{don} de lait\index{gnl}{lait}. \`A M\=elakka\d tamp\=ur\index{gnl}{Melakkatampur@M\=elakka\d tamp\=ur}, \`a gauche, la d\'eesse\index{gnl}{deesse@déesse}, main gauche sur le sein, semble tirer son lait\index{gnl}{lait}. L'objet qu'elle tient dans la main droite n'est pas identifiable (\textsc{L'Hernault} ne le mentionne pas). Au centre, un personnage barbu, le père\index{gnl}{pere@père} de Campantar\index{gnl}{Campantar}, menace de son b\^aton l'enfant\index{gnl}{enfant} Campantar\index{gnl}{Campantar} qui montre de sa coupelle\index{gnl}{coupelle} le couple divin, \'Siva\index{gnl}{Siva@\'Siva} et P\=arvat\=\i\index{gnl}{Parvati@P\=arvat\=\i}, assis confortablement sur leur tr\^one. \`A T\=ar\=acuram\index{gnl}{Taracuram@T\=ar\=acuram}, \'Siva\index{gnl}{Siva@\'Siva} et P\=arvat\=\i\index{gnl}{Parvati@P\=arvat\=\i}, mont\'es sur le taureau\index{gnl}{taureau}, apparaissent, alors que l'enfant\index{gnl}{enfant} Campantar\index{gnl}{Campantar}, menac\'e d'un b\^aton par son père\index{gnl}{pere@père} barbu, pointe son index vers le ciel. La repr\'esentation est tellement recouverte de stuc qu'il est difficile de savoir si l'enfant\index{gnl}{enfant} tient quelque chose \`a la main.
Le panneau 66 est la repr\'esentation de la reine\index{gnl}{reine} \textit{p\=a\d n\d dya}\index{gnl}{pandya@\textit{p\=a\d n\d dya}}. \`A M\=elakka\d tamp\=ur\index{gnl}{Melakkatampur@M\=elakka\d tamp\=ur}, la reine\index{gnl}{reine} assise sur son si\`ege est entour\'ee de ses suivantes assises au sol. Campantar\index{gnl}{Campantar} n'est pas figur\'e. \`A T\=ar\=acuram\index{gnl}{Taracuram@T\=ar\=acuram}, la sc\`ene narre l'accueil que re\c coit l'enfant\index{gnl}{enfant} Campantar\index{gnl}{Campantar} \`a son arriv\'ee \`a Maturai\index{gnl}{Maturai}. Campantar\index{gnl}{Campantar} assis sur son si\`ege fait le geste\index{gnl}{geste} d'absence de crainte de la main droite. Derri\`ere lui, quelqu'un le rafraîchit avec un chasse-mouche. Devant lui, deux femme\index{gnl}{femme}s le saluent les mains jointes. Il peut s'agir de la reine\index{gnl}{reine} et de sa suivante; plus loin, deux hommes font aussi l'\textit{a\~njali}, le ministre\index{gnl}{ministre} et son serviteur\index{gnl}{serviteur}. Le point surprenant dans cette repr\'esentation est la dimension de l'enfant\index{gnl}{enfant} Campantar\index{gnl}{Campantar} qui est pratiquement deux fois plus imposant que les autres personnages.
Le panneau 69 est consacr\'e au joueur de \textit{y\=a\b l}, fid\`ele compagnon de Campantar\index{gnl}{Campantar} dans ses pèlerinage\index{gnl}{pelerinage@pèlerinage}s. \`A M\=elakka\d tamp\=ur\index{gnl}{Melakkatampur@M\=elakka\d tamp\=ur}, le musicien joue du \textit{y\=a\b l} au milieu de la sc\`ene. Il est encadr\'e de son \'epouse, assise, qui joue des cymbale\index{gnl}{cymbale}s en le regardant et d'un petit personnage, debout, qui a les mains jointes. Il s'agit certainement de Campantar\index{gnl}{Campantar}. Ce dernier chante ou adore un \textit{li\.nga}\index{gnl}{linga@\textit{li\.nga}} qui aurait pu se trouver \`a sa gauche. En effet, Campantar\index{gnl}{Campantar} est tourn\'e vers sa gauche, vers un espace laissé vide, o\`u il aurait \'et\'e possible de figurer un \textit{li\.nga}\index{gnl}{linga@\textit{li\.nga}}.

Nous constatons donc une \'evolution dans la repr\'esentation de Campantar\index{gnl}{Campantar}. \`A l'origine\index{gnl}{origine}, sur le mur ext\'erieur du temple\index{gnl}{temple} de Karuntai\index{gnl}{Karuntattankuti@Karunta\d t\d t\=a\.nku\d ti!Karuntai}, il est un poète\index{gnl}{poete@poète} adulte qui joue des cymbale\index{gnl}{cymbale}s devant un \textit{li\.nga}\index{gnl}{linga@\textit{li\.nga}}. Ensuite, quand il entre \`a l'int\'erieur du temple\index{gnl}{temple}, en image\index{gnl}{image} de bronze, il est devenu un enfant\index{gnl}{enfant} poète\index{gnl}{poete@poète} (avec cymbale\index{gnl}{cymbale}s) ou un enfant\index{gnl}{enfant} divin qui a re\c cu le lait\index{gnl}{lait} de la d\'eesse\index{gnl}{deesse@déesse} dans une coupelle\index{gnl}{coupelle}. Sa gestuelle, pointer l'index droit vers le ciel, renvoie aussi \`a l'\'episode du don\index{gnl}{don} de lait\index{gnl}{lait} lors duquel il d\'esigne le couple \'Siva\index{gnl}{Siva@\'Siva}-P\=arvat\=\i\index{gnl}{Parvati@P\=arvat\=\i}\ comme \'etant ceux qui l'ont nourri, ceux \`a qui il est li\'e. Enfin, une fois sa légende\index{gnl}{legende@légende} d\'evelopp\'ee et fix\'ee par \'ecrit, il n'est plus seulement l'enfant\index{gnl}{enfant} prodige\index{gnl}{prodige} qui a re\c cu la gr\^ace divine mais aussi celui qui apparaît également dans les hagiographie\index{gnl}{hagiographie}s d'autres dévot\index{gnl}{devot(e)@dévot(e)}s.

L'image\index{gnl}{image} du Campantar\index{gnl}{Campantar} adulte de Karuntai\index{gnl}{Karuntattankuti@Karunta\d t\d t\=a\.nku\d ti!Karuntai}, datable du \textsc{x}\up{e} au d\'ebut du \textsc{xi}\up{e} si\`ecle, nous permet de supposer que les deux passages du \textit{T\=ev\=aram}\index{gnl}{Tevaram@\textit{T\=ev\=aram}}, qui sugg\`erent que le poète\index{gnl}{poete@poète} est un enfant\index{gnl}{enfant} et dont nous doutions de l'authenticit\'e pour d'autres raisons (voir 2.3.1), sont fort probablement des interpolation\index{gnl}{interpolation}s faites apr\`es le d\'ebut du \textsc{xi}\up{e} si\`ecle. Ajoutons, selon ce m\^eme raisonnement, que tous les textes \'ecrits disponibles qui louent un Campantar\index{gnl}{Campantar} enfant\index{gnl}{enfant} sont post\'erieurs \`a cette p\'eriode; nous pensons en particulier au texte attribu\'e \`a Pa\d t\d ti\b nattuppi\d l\d lai\index{gnl}{Pattinattu Pillai@Pa\d t\d ti\b nattuppi\d l\d lai}. Cette figure de l'enfant\index{gnl}{enfant} Campantar\index{gnl}{Campantar} dont nous percevons une \og origine\index{gnl}{origine}\fg\ dans les image\index{gnl}{image}s, avant les textes, semble avoir \'et\'e influenc\'ee par l'iconographie d\'ej\`a connue et parfaitement assimil\'ee de divinit\'es enfant\index{gnl}{enfant}s dans le Pays Tamoul\index{gnl}{Pays Tamoul}.

\subsection{La formation d'une iconographie}

La r\'ecup\'eration ou la ressemblance iconographique dans l'histoire de l'art religieux en Inde est un ph\'enom\`ene parfois observable lorsqu'il y a cr\'eation d'une nouvelle image\index{gnl}{image}. Ainsi, la repr\'esentation de \'Siva\index{gnl}{Siva@\'Siva} enseignant assis sous un banian, appel\'ee la Dak\d si\d n\=am\=urti\index{gnl}{Daksina@Dak\d si\d n\=am\=urti}, n\'ee sous les Pallava\index{gnl}{Pallava}, semble avoir partiellement puis\'e sur le mod\`ele, apparu des si\`ecles plus t\^ot dans l'Inde septentrionale, du Bouddha\index{gnl}{Bouddha} pr\^echant assis sous un pipal. \`A l'aube du deuxième mill\'enaire au Pays Tamoul\index{gnl}{Pays Tamoul}, il y a eu, vraisemblablement, un besoin de repr\'esenter le poète\index{gnl}{poete@poète} Campantar\index{gnl}{Campantar} ou plut\^ot la figure de l'enfant\index{gnl}{enfant} divin Campantar\index{gnl}{Campantar} dont les hymne\index{gnl}{hymne}s sont chant\'es lors des culte\index{gnl}{culte}s de temples\index{gnl}{temple} et dont la légende\index{gnl}{legende@légende} s'\'etablit. Des mod\`eles pr\'e-existent.

L'iconographie de l'enfant\index{gnl}{enfant} divin est tr\`es pr\'esente dans le Pays Tamoul\index{gnl}{Pays Tamoul} avec les image\index{gnl}{image}s de Skanda\index{gnl}{Skanda} et de K\textsubring{r}\d s\d na\index{gnl}{Krsna@K\textsubring{r}\d s\d na}. L'enfant\index{gnl}{enfant} Skanda\index{gnl}{Skanda} est rarement seul, semble-t-il, dans les premi\`eres image\index{gnl}{image}s. Il est accompagn\'e de ses parents, \'Siva\index{gnl}{Siva@\'Siva} et P\=arvat\=\i\index{gnl}{Parvati@P\=arvat\=\i}, dans la composition du Som\=askanda\index{gnl}{Somaskanda@Som\=askanda} d\`es les Pallava\index{gnl}{Pallava}. Nu, g\'en\'eralement debout, il tient des lotus dans les mains\footnote{Cf. \textsc{L'Hernault} 1978 et \textsc{Schmid} à paraître (a).}. Fils de \'Siva\index{gnl}{Siva@\'Siva} et de P\=arvat\=\i\index{gnl}{Parvati@P\=arvat\=\i}, Skanda\index{gnl}{Skanda} aurait \'et\'e un exemple parfait pour repr\'esenter Campantar\index{gnl}{Campantar}. N'est-ce pas \`a lui qu'est identifi\'e Campantar\index{gnl}{Campantar} dans les textes attribu\'es \`a Nampi\index{gnl}{Nampi \=A\d n\d t\=ar Nampi} (\textit{APUM} 124, \textit{APMK} 1.11 et 10.4)?
Mais c'est dans une secte concurrente, chez les vishnouite\index{gnl}{vishnouite}s, que l'image\index{gnl}{image} de l'enfant\index{gnl}{enfant} Campantar\index{gnl}{Campantar} trouvera son moule.

\begin{figure}[!h]
  \centering
 \includegraphics[height=8cm]{docthese/bronzeTNCcikali.jpg}
  \caption{L'enfant Campantar dansant, bronze, temple de C\=\i k\=a\b li (cliché U. \textsc{Veluppillai}, 2005).}
\end{figure}

\noindent La ressemblance entre les image\index{gnl}{image}s de K\textsubring{r}\d s\d na\index{gnl}{Krsna@K\textsubring{r}\d s\d na} et de Campantar\index{gnl}{Campantar} a d\'ej\`a \'et\'e soulign\'ee (voir \textsc{Dehejia} 1987 et \textsc{Lef\`evre} 2001). La forme dansante du poète\index{gnl}{poete@poète} enfant\index{gnl}{enfant} est un calque de celle de K\textsubring{r}\d s\d na\index{gnl}{Krsna@K\textsubring{r}\d s\d na} dansant sur le serpent\index{gnl}{serpent} d\'emoniaque, \`a la diff\'erence du geste\index{gnl}{geste} de la main droite: Campantar\index{gnl}{Campantar} pointe du doigt vers le ciel alors que K\textsubring{r}\d s\d na\index{gnl}{Krsna@K\textsubring{r}\d s\d na} fait le geste\index{gnl}{geste} d'absence de crainte. Plusieurs hypoth\`eses peuvent tenter d'expliquer les raisons de ce choix krishna\"ite: parce que K\textsubring{r}\d s\d na\index{gnl}{Krsna@K\textsubring{r}\d s\d na} est l'enfant\index{gnl}{enfant}-dieu\index{gnl}{dieu} le plus repr\'esent\'e, semble-t-il, sur le territoire \textit{c\=o\b la}\index{gnl}{cola@\textit{c\=o\b la}} au \textsc{x-xi}\up{e} si\`ecles, parce qu'il appartient \`a une secte concurrente du shiva\"isme\index{gnl}{shivaisme@shiva\"isme}, et/ou parce qu'il est un dieu\index{gnl}{dieu} qui descend sur terre\index{gnl}{terre} pour d\'etruire les d\'emons comme Campantar\index{gnl}{Campantar} qui est n\'e pour d\'etruire l'h\'er\'esie et faire briller le shiva\"isme\index{gnl}{shivaisme@shiva\"isme}. Mais aussi, peut-\^etre, parce qu'un texte sanskrit, le \textit{Bh\=agavatapur\=a\d na}\index{gnl}{Bhagavata@\textit{Bh\=agavatapur\=a\d na}}, r\'edig\'e au Pays Tamoul\index{gnl}{Pays Tamoul} au \textsc{x-xi}\up{e} si\`ecle, moment o\`u nous pensons que la légende\index{gnl}{legende@légende} de Campantar\index{gnl}{Campantar} s'est form\'ee, met K\textsubring{r}\d s\d na\index{gnl}{Krsna@K\textsubring{r}\d s\d na} sur le devant de la sc\`ene religieuse avec de nouvelles image\index{gnl}{image}s. En effet, la légende\index{gnl}{legende@légende} de l'enfant\index{gnl}{enfant} K\textsubring{r}\d s\d na\index{gnl}{Krsna@K\textsubring{r}\d s\d na} est cont\'ee dans le \textit{Hariva\d m\'sa}\index{gnl}{Harivamsa@\textit{Hariva\d m\'sa}} (\textsc{iii-iv}\up{e} s.), appendice du \textit{Mah\=abh\=arata}\index{gnl}{Mahabharata@\textit{Mah\=abh\=arata}} qui pour sa part en pr\'esente pour l'essentiel l'incarnation adulte, et dans le \textit{Vi\d s\d nupur\=a\d na}\index{gnl}{Visnupurana@\textit{Vi\d s\d nupur\=a\d na}} (\textsc{v}\up{e} s.). Au Pays Tamoul\index{gnl}{Pays Tamoul}, cette enfance est chant\'ee dans les poème\index{gnl}{poeme@poème}s de \textit{bhakti}\index{gnl}{bhakti@\textit{bhakti}} vishnouite\index{gnl}{vishnouite}s attribu\'es aux \textit{\=a\b lv\=ar} entre le \textsc{viii}\up{e} et le \textsc{ix}\up{e} si\`ecle. Ensuite, au \textsc{x-xi}\up{e} si\`ecle, le \textit{Bh\=agavatapur\=a\d na}\index{gnl}{Bhagavata@\textit{Bh\=agavatapur\=a\d na}} d\'eveloppe cet \^age de la vie dans son livre \textsc{x}. \textsc{Schmid} (2002), \`a travers l'\'etude des frises narratives krishna\"ites, permet de pr\'eciser l'apparition du \textit{Bh\=agavatapur\=a\d na}\index{gnl}{Bhagavata@\textit{Bh\=agavatapur\=a\d na}}. En partant d'un corpus\index{gnl}{corpus} de six temples\index{gnl}{temple} shiva\"ite\index{gnl}{shiva\"ite}s du \textsc{x}\up{e} si\`ecle du delta de la K\=av\=eri\index{gnl}{Kaveri@K\=av\=eri} et d'un temple\index{gnl}{temple} vishnouite\index{gnl}{vishnouite} du \textsc{xi}\up{e} si\`ecle qu'elle confronte avec les textes sanskrits et tamouls d\'ecrivant l'enfance de K\textsubring{r}\d s\d na\index{gnl}{Krsna@K\textsubring{r}\d s\d na}, Charlotte \textsc{Schmid} souligne l'influence du \textit{Bh\=agavatapur\=a\d na}\index{gnl}{Bhagavata@\textit{Bh\=agavatapur\=a\d na}} sur l'iconographie vishnouite\index{gnl}{vishnouite}, du temple\index{gnl}{temple} de Varadar\=ajaperum\=a\d l \`a Tirupuva\b nai\index{gnl}{Tirupuva\b nai}, au \textsc{xi}\up{e} si\`ecle, avec de nouvelles mises en sc\`ene du dieu\index{gnl}{dieu}-enfant\index{gnl}{enfant} jouant de la fl\^ute ou volant le beurre, et pr\'ecise ainsi la datation du texte sanskrit. Le fait que ce texte r\'eactualise l'enfance de K\textsubring{r}\d s\d na\index{gnl}{Krsna@K\textsubring{r}\d s\d na} et que ceci se voit sur les temples\index{gnl}{temple} peut \^etre, \`a notre avis, un des arguments principaux d\'eterminant le mod\`ele iconographique de Campantar\index{gnl}{Campantar}. Nous sugg\'erons, par exemple, que ces nouvelles repr\'esentations krishna\"ites sculpt\'ees sur les frises des murs ont probablement influenc\'e des \'el\'ements des frises narratives shiva\"ite\index{gnl}{shiva\"ite}s de M\=elakka\d tamp\=ur\index{gnl}{Melakkatampur@M\=elakka\d tamp\=ur} et de T\=ar\=acuram\index{gnl}{Taracuram@T\=ar\=acuram} illustrant la d\'evotion\index{gnl}{devotion@dévotion} des soixante-trois \textit{n\=aya\b nm\=ar}\index{gnl}{nayanmar@\textit{n\=aya\b nm\=ar}}. L'image\index{gnl}{image} de Campantar\index{gnl}{Campantar} enfant\index{gnl}{enfant} tenant une coupelle\index{gnl}{coupelle} \`a la main et menac\'e d'un b\^aton par son père\index{gnl}{pere@père} peut \^etre mise en parall\`ele avec celle de K\textsubring{r}\d s\d na\index{gnl}{Krsna@K\textsubring{r}\d s\d na} volant le beurre et menac\'e d'un b\^aton par Ya\'sod\=a\index{gnl}{Yasoda@Ya\'sod\=a} (\textsc{Schmid} 2002: 45, fig. 22 a/b)\footnote{Nous pouvons \'etablir une autre correspondance iconographique sur ces frises entre le \textit{n\=aya\b n\=ar}\index{gnl}{nayanmar@\textit{n\=aya\b nm\=ar}!\textit{n\=aya\b n\=ar}} \=A\b naya\b n, bouvier-fl\^utiste, et K\textsubring{r}\d s\d na\index{gnl}{Krsna@K\textsubring{r}\d s\d na} jouant de la fl\^ute.}.

Ainsi, la légende\index{gnl}{legende@légende} du héros\index{gnl}{heros@héros} Campantar\index{gnl}{Campantar} est à l'origine d'une iconographie nouvelle, influenc\'ee par le krishna\"isme\index{gnl}{krishnaisme@krishna\"isme}, et qui inspire les textes shiva\"ite\index{gnl}{shiva\"ite}s.

\subsection{Des textes selon les images}

Les panneaux narratifs de M\=elakka\d tamp\=ur\index{gnl}{Melakkatampur@M\=elakka\d tamp\=ur} et de T\=ar\=acuram\index{gnl}{Taracuram@T\=ar\=acuram} renvoient \`a la légende\index{gnl}{legende@légende} \'etablie et \'ecrite de Campantar\index{gnl}{Campantar} qui leur est contemporaine. En revanche, ce sont des textes du \textsc{xii}\up{e} si\`ecle, ceux qui fixent cette légende\index{gnl}{legende@légende}, qui viennent l\'egitimer, en quelque sorte, la posture dansante de l'enfant\index{gnl}{enfant}-poète\index{gnl}{poete@poète}, influenc\'ee par le krishna\"isme\index{gnl}{krishnaisme@krishna\"isme}, qui appara\^it d\`es le \textsc{xi}\up{e} si\`ecle. Nous pensons donc que les image\index{gnl}{image}s de bronze repr\'esentant Campantar\index{gnl}{Campantar} dansant, parce que d\'ecrites postérieurement, ont inspir\'e les hagiographes qui les ont placées au centre de leurs légendes\footnote{Nous n'aborderons pas les textes \=agamiques qui d\'ecrivent les personnages \og saints\fg\ et leur culte\index{gnl}{culte} parce qu'ils sont post\'erieurs aux culte\index{gnl}{culte}s rendus \`a Campantar\index{gnl}{Campantar} ou \`a ses hymne\index{gnl}{hymne}s dans les temples\index{gnl}{temple}.}. Selon les textes, le père\index{gnl}{pere@père} de Campantar\index{gnl}{Campantar} est furieux et souhaite conna\^itre l'origine\index{gnl}{origine} du lait\index{gnl}{lait} consomm\'e. L'enfant\index{gnl}{enfant} Campantar\index{gnl}{Campantar}, tout en dansant, pointe du doigt vers le ciel o\`u se trouve le couple \'Siva\index{gnl}{Siva@\'Siva}-P\=arvat\=\i\index{gnl}{Parvati@P\=arvat\=\i}.


\scriptsize
\begin{verse}
ci\b rupara\b r karanta vi\d likura\b r ki\.nki\d ni\\
c\=eva\d ti pullic cilkural iya\b r\b ri\\
amutu\d n cevv\=ay aruvi t\=u\.nkat\\
t\=a\d lam piriy\=at ta\d takkai acaittuc \\
\textbf{ci\b ruk\=ut tiya\b r\b ri}c civa\b naru\d l pe\b r\b ra\\
na\b r\b rami\b l viraka\b n ... (APMK 19.7-12)\\
\end{verse}

\normalsize
\begin{verse}
L'expert en bon tamoul qui obtint la gr\^ace de \'Siva\index{gnl}{Siva@\'Siva}\\
Ayant mis \`a ses pieds rouges des grelots\\
Au son chantant et contenant de petites graines,\\
S'exprima d'une petite voix,\\
Alors que coulait le flot (de lait\index{gnl}{lait})\\
De sa bouche rouge qui consomma l'ambroisie\index{gnl}{ambroisie}\\
Il agita sa large main qui ne quitte jamais les cymbale\index{gnl}{cymbale}s\\
Et fit une petite danse\index{gnl}{danse}. (APMK 19.7-12)\\
\end{verse}

\scriptsize
\begin{verse}
eccil maya\.nki\d ta u\b nakku \=\i tu i\d t\d t\=araik k\=a\d t\d tu e\b n\b ru\\
kaic ci\b riyatu orum\=a\b ru ko\d n\d tu \=occak \textbf{k\=al e\d tutt\=e}\\
ac ci\b riya peruntakaiy\=ar \=a\b nantak ka\d n tu\d li peytu\\
ucciyi\b nil e\d tuttu aru\d lum oru tirukkai viral cu\d t\d ti (PP 1971)\\
\end{verse}

\normalsize
\begin{verse}
Alors qu'il [le père] brandit un petit b\^aton en disant:\\
\og Montre(-moi) celui qui t'a donn\'e ce qui te fait saliver\fg,\\
Le jeune et très honorable éleva une jambe,\\
Pleura de joie,\\
Prit au-dessus (de la t\^ete) un doigt de sa main gracieuse\\
Et pointa (vers le ciel). (PP 1971)\\
\end{verse}

Les auteurs de ces strophes \'evoquent la danse\index{gnl}{danse} de Campantar\index{gnl}{Campantar} au moment du miracle\index{gnl}{miracle} du don\index{gnl}{don} de lait\index{gnl}{lait} et int\`egrent ainsi les image\index{gnl}{image}s dansantes en bronze calqu\'ees sur K\textsubring{r}\d s\d na\index{gnl}{Krsna@K\textsubring{r}\d s\d na} dans la légende\index{gnl}{legende@légende} \'ecrite. C'est dans cette derni\`ere que nous retrouvons aussi l'unit\'e des douze\index{gnl}{douze} noms de la ville d'origine\index{gnl}{origine} du poète\index{gnl}{poete@poète}.

\section{La ville d'origine aux douze noms}

\subsection{Les douze légendes dans les \textit{Tirumu\b rai} \textsc{xi} et \textsc{xii}}

Si nous trouvons des r\'ef\'erences \`a l'entit\'e des douze\index{gnl}{douze} noms de C\=\i k\=a\b li\index{gnl}{Cikali@C\=\i k\=a\b li} dans les textes attribu\'es \`a Pa\d t\d ti\b nattuppi\d l\d lai\index{gnl}{Pattinattu Pillai@Pa\d t\d ti\b nattuppi\d l\d lai}, Nampi \=A\d n\d t\=ar Nampi\index{gnl}{Nampi \=A\d n\d t\=ar Nampi} et C\=ekki\b l\=ar\index{gnl}{Cekkilar@C\=ekki\b l\=ar}, en revanche, nous ne relevons dans ces mêmes textes des allusions qu'\`a une seule légende\index{gnl}{legende@légende}, celle du déluge\index{gnl}{deluge@déluge} de T\=o\d nipuram\index{gnl}{Tonipuram@T\=o\d nipuram}.

Pa\d t\d tinattu Pi\d l\d lai, dans le \textit{Tirukka\b lumalamumma\d nikk\=ovai}\index{gnl}{Tirukkalumalamummanikkovai@\textit{Tirukka\b lumalamumma\d nikk\=ovai}}, \'evoque le mythe\index{gnl}{mythe} du déluge\index{gnl}{deluge@déluge} \`a deux reprises:

\scriptsize
\begin{verse}
\textit{n\=akar n\=a\d tu m\=\i micai mitantu\\
m\=\i micai ulaka\.n k\=\i \b lmutal t\=a\b lnti\.nku\\
o\b n\b r\=a vanta ku\b n\b r\=a ve\d l\d lattu \\
ulakamm\=u\b n\b rukkum ka\d laika\d n \=aki\\
mutalil k\=alam i\b nitu v\=\i \b r\b riruntu\b lit} (1.18-22)\\
\end{verse}

\normalsize
\begin{verse}
Dans les eaux\index{gnl}{eau} venues ensemble en montagne (de vague\index{gnl}{vague}s)\\
(D'abord) le pays des c\'elestes flotta\index{gnl}{flotter} bien \\
(Puis) les mondes \`a commencer par le bas furent bien engloutis \\
\lbrack Alors \'Siva\index{gnl}{Siva@\'Siva}] devint le support des trois mondes\\
Et demeura avec plaisir \`a l'\^age premier [\`a Ka\b lumalam\index{gnl}{Kalumalam@Ka\b lumalam}]. (1.18-22)\\
\end{verse}

\scriptsize
\begin{verse}
\textit{ne\d tunila va\d l\=akamum a\d tukatir v\=a\b namum\\
a\d taiyap paranta \=ative\d l\d lattu\\
nuraiye\b nac cita\b ri irucu\d tar mitappa\\
varaipa\b rittiya\.nkum m\=arutam ka\d tuppa\\
m\=alum pirama\b num mutaliya v\=a\b navar\\
k\=alam ituve\b nak kala\.nk\=a ni\b n\b ru\b li\\
ma\b r\b ravar uyyap pa\b r\b riya pu\d naiy\=ay\\
mikana\b ni mitanta pukali\index{gnl}{Pukali} n\=ayaka} (4.14-21)\\
\end{verse}

\normalsize
\begin{verse}
Alors que les deux astres flottaient\index{gnl}{flotter}, \'eclat\'es en \'ecume,\\
Sous les flots\index{gnl}{flots} premiers qui se r\'epandaient\\
Pour atteindre le vaste monde et le ciel du soleil br\^ulant,\\
Alors que le vent qui arrache les montagnes soufflait,\\
Alors que les c\'elestes \`a commencer par M\=al\index{gnl}{Visnu@Vi\d s\d nu!Mal@M\=al} et Brahm\=a\index{gnl}{Brahma@Brahm\=a}\\
\'Etaient troubl\'es en pensant que c'était le moment [de la fin],\\
Il demeura,\\
Devint le radeau\index{gnl}{radeau} qui lib\`ere les autres\\
Le héros\index{gnl}{heros@héros} de Pukali\index{gnl}{Pukali} qui flotta\index{gnl}{flotter} parfaitement. (4.14-21)\\
\end{verse}

\noindent
Au moment du déluge\index{gnl}{deluge@déluge} apocalyptique, alors que le monde est recouvert d'eau\index{gnl}{eau}, \'Siva\index{gnl}{Siva@\'Siva} vient s'installer dans la ville qui flottait\index{gnl}{flotter}: Ka\b lumalam\index{gnl}{Kalumalam@Ka\b lumalam} ou Pukali\index{gnl}{Pukali}.\\

Ensuite, parmi les sept textes attribu\'es \`a Nampi \=A\d n\d t\=ar Nampi\index{gnl}{Nampi \=A\d n\d t\=ar Nampi}, c\'el\'ebrant Campantar\index{gnl}{Campantar} et C\=\i k\=a\b li\index{gnl}{Cikali@C\=\i k\=a\b li}, seul un poème\index{gnl}{poeme@poème}, l'\textit{APUM}, fait r\'ef\'erence au mythe\index{gnl}{mythe} du déluge\index{gnl}{deluge@déluge} dans un passage pr\'ec\'edant l'\'enum\'eration des douze\index{gnl}{douze} toponymes:


\scriptsize
\begin{quotation}
\textit{va\d larve\d l\d lat tumparo\d tum pa\b n\b niruk\=al n\=\i ril mitantav\=ur} (\textit{APUM} 55)
\end{quotation}

\normalsize
\begin{verse}
Sur les flots\index{gnl}{flots} grandissants,\\
Avec les c\'elestes,\\
La ville qui flotta\index{gnl}{flotter}\\
\`A douze\index{gnl}{douze} reprises sur l'eau\index{gnl}{eau}. (\textit{APUM} 55)\\
\end{verse}

\noindent
Cette description du site qui r\'esiste au déluge\index{gnl}{deluge@déluge} apocalyptique douze\index{gnl}{douze} fois, c'est-\`a-dire sous douze\index{gnl}{douze} noms diff\'erents, ressemble \`a celle faite par Pa\d t\d ti\b nattuppi\d l\d lai\index{gnl}{Pattinattu Pillai@Pa\d t\d ti\b nattu Pi\d l\d lai} dans le \textit{Tirukka\b lumalamumma\d nikk\=ovai}\index{gnl}{Tirukkalumalamummanikkovai@\textit{Tirukka\b lumalamumma\d nikk\=ovai}} 10.2-5: \og la ville au nom distinct dans chacun des douze\index{gnl}{douze} \textit{yuga}\fg\ (\textit{pa\b n\b n\=\i rukattu v\=e\b ruv\=e\b ru peyari\b n \=ur}).\\

Enfin, dans le \textit{Periyapur\=a\d nam}\index{gnl}{Periyapuranam@\textit{Periyapur\=a\d nam}}, c'est aussi le mythe\index{gnl}{mythe} du déluge\index{gnl}{deluge@déluge} qui est associ\'e au site de C\=\i k\=a\b li\index{gnl}{Cikali@C\=\i k\=a\b li}.

Dans l'hagiographie\index{gnl}{hagiographie} de Cuntarar\index{gnl}{Cuntarar} nous trouvons en effet une r\'ef\'erence \`a un hymne\index{gnl}{hymne} qu'aurait chant\'e Cuntarar\index{gnl}{Cuntarar} sur T\=o\d nipuram\index{gnl}{Tonipuram@T\=o\d nipuram}:

\scriptsize
\begin{verse}
\textit{ma\d n\d tiya p\=er a\b npi\b n\=al va\b n to\d n\d tar ni\b n\b ri\b rai\~ncit\\
\textbf{te\d n tirai v\=elaiyil mitanta tirut t\=o\d nipuratt\=arai}k\\
ka\d n\d tu ko\d n\d t\=e\b n kayilaiyi\b nil v\=\i \b r\b ru irunta pa\d ti e\b n\b ru\\
pa\d n\d tu arum i\b n\b nicai payi\b n\b ra tirup patikam p\=a\d ti\b n\=ar.} (\textit{PP} 259)\\
\end{verse}

\normalsize
\begin{verse}
Par un grand amour abondant,\\
Le s\'ev\`ere dévot\index{gnl}{devot(e)@dévot(e)},\\
Priant debout, chanta un hymne\index{gnl}{hymne}\\
Compos\'e selon une douce m\'elodie d'antan:\\
\og J'ai vu, assis majestueusement au Kail\=asa\index{gnl}{Kailasa@Kail\=asa},\\
Celui de T\=o\d nipuram\index{gnl}{Tonipuram@T\=o\d nipuram} qui flotta\index{gnl}{flotter} \\
Sur les vague\index{gnl}{vague}s de la mer\index{gnl}{mer} pure\fg. (\textit{PP} 259)\\
\end{verse}

\noindent
L'expression \textit{ka\d n\d tu ko\d n\d t\=e\b n} figure dans l'unique poème\index{gnl}{poeme@poème} de Cuntarar\index{gnl}{Cuntarar} en l'honneur de C\=\i k\=a\b li\index{gnl}{Cikali@C\=\i k\=a\b li}, VII 58. Cependant, c'est la ville de Ka\b lumalam\index{gnl}{Kalumalam@Ka\b lumalam} qui y flotte\index{gnl}{flotter} et non celle de T\=o\d nipuram\index{gnl}{Tonipuram@T\=o\d nipuram}. C\=ekki\b l\=ar\index{gnl}{Cekkilar@C\=ekki\b l\=ar} semble avoir identifi\'e l'un à l'autre ces deux toponymes.

Dans l'hagiographie\index{gnl}{hagiographie} d'Appar\index{gnl}{Appar} (\textit{PP} 1266-1694), lors de son pèlerinage\index{gnl}{pelerinage@pèlerinage} \`a C\=\i k\=a\b li\index{gnl}{Cikali@C\=\i k\=a\b li} (\textit{PP} 1442-1454), nous relevons deux allusions \`a la légende\index{gnl}{legende@légende} du déluge\index{gnl}{deluge@déluge}:
\begin{enumerate}

\item
\scriptsize
\begin{verse}
\textit{ka\d taiyukattil \=a\b liyi\b n m\=el mitanta ka\b lumalatti\b n irunta cemka\d n\\
vi\d tai ukaitt\=ar} \dots\ (\textit{PP} 1442)
\end{verse}

\normalsize
\begin{verse}
Le cavalier du taureau\index{gnl}{taureau} aux yeux rouges\\
Install\'e \`a Ka\b lumalam\index{gnl}{Kalumalam@Ka\b lumalam}\\
Qui, \`a la fin du \textit{yuga},\\
Flotta\index{gnl}{flotter} sur le déluge\index{gnl}{deluge@déluge}. (\textit{PP} 1442)\\
\end{verse}


\item
\scriptsize
\begin{verse}
\textit{\dots}\\
\textit{ve\d l\d lan\=\i rt tirutt\=o\d ni v\=\i \b r\b rirunt\=ar ka\b lal va\d na\.nkum viruppi\b n mikk\=ar.} (\textit{PP} 1449)\\
\end{verse}

\normalsize
\begin{verse}
Celui qui excelle dans le d\'esir d'honorer\\
Les [Pieds] aux anneaux\\
De Celui qui est assis majestueusement\\
Sur le radeau\index{gnl}{radeau} sacré des eaux\index{gnl}{eau} du déluge\index{gnl}{deluge@déluge}. (\textit{PP} 1449)\\
\end{verse}
\end{enumerate}

\noindent
D'apr\`es cette derni\`ere strophe ce n'est pas le site de T\=o\d nipuram\index{gnl}{Tonipuram@T\=o\d nipuram} qui se trouve sur les eaux\index{gnl}{eau} du déluge\index{gnl}{deluge@déluge} mais le radeau\index{gnl}{radeau} (\textit{t\=o\d ni}) sur lequel est install\'e \'Siva\index{gnl}{Siva@\'Siva}. En effet, \`a plusieurs reprises dans le \textit{Periyapur\=a\d nam}\index{gnl}{Periyapuranam@\textit{Periyapur\=a\d nam}}, \'Siva\index{gnl}{Siva@\'Siva} est d\'ecrit comme celui qui r\'eside dans le temple\index{gnl}{temple} que forme le radeau\index{gnl}{radeau} de la ville de C\=\i k\=a\b li\index{gnl}{Cikali@C\=\i k\=a\b li}: \og le temple\index{gnl}{temple}-montagne du radeau\index{gnl}{radeau} de T\=o\d nipuram\index{gnl}{Tonipuram@T\=o\d nipuram}\fg\ (\textit{t\=o\d nipuratt\=o\d nic cikarakk\=oyil}, \textit{PP} 2004); \og Celui \`a la gorge noire qui est assis majestueusement sur le radeau\index{gnl}{radeau} sacré de V\=e\d nupuram\index{gnl}{Venupuram@V\=e\d nupuram}\fg\ (\textit{v\=e\d nupurattirutt\=o\d ni v\=\i \b r\b rirunta ka\d la\.nko\d l ka\d n\d tar}, \textit{PP} 2128); \og le héros\index{gnl}{heros@héros} du grand radeau\index{gnl}{radeau} sacré qui se trouve \`a Ca\d npai\index{gnl}{Canpai@Ca\d npai}\fg\ (\textit{ca\d npaiyilamar peruntirutt\=o\d ni n\=aya\b n\=ar}, \textit{PP} 3924).

Dans l'hagiographie\index{gnl}{hagiographie} de Ka\d nan\=atar\index{gnl}{Kananatar@Ka\d nan\=atar} (\textit{PP} 3923-3929), dévot\index{gnl}{devot(e)@dévot(e)} originaire de C\=\i k\=a\b li\index{gnl}{Cikali@C\=\i k\=a\b li}, la strophe inaugurale pr\'esente le site en rapport avec Campantar\index{gnl}{Campantar} et avec la légende\index{gnl}{legende@légende} du déluge\index{gnl}{deluge@déluge}:

\scriptsize
\begin{verse}
\textit{\=a\b li m\=anilattu akilam \=\i \b n\b ru a\d littava\d l tirumulai amutu\d n\d ta\\
v\=a\b li \~n\=a\b nacampantar vantaru\d liya va\b nappi\b natu \textbf{a\d lappill\=a\\
\=u\b li m\=aka\d tal ve\d l\d lattu mitantu ulaki\b nukku oru mutal \=ay\\
k\=a\b li m\=a nakar} tiruma\b raiyavar kulakk\=avalar ka\d nan\=atar.} (\textit{PP} 3923)\\
\end{verse}

\normalsize
\begin{verse}
C'est une merveille la venue du prosp\`ere \~N\=a\b nacampantar\index{gnl}{Campantar}\\
Qui consomma l'ambroisie\index{gnl}{ambroisie} du sein divin\\
De Celle qui, l'ayant engendr\'ee, donna\\
Toute la terre\index{gnl}{terre} [entour\'ee] d'oc\'eans;\\
Ka\d nan\=atar\index{gnl}{Kananatar@Ka\d nan\=atar} le protecteur du clan des brahmane\index{gnl}{brahmane}s v\'ediques\\
De la grande ville de K\=a\b li\index{gnl}{Kali@K\=a\b li}\\
Qui, devenue la premi\`ere dans le monde,\\
Flotta\index{gnl}{flotter} sur les flots\index{gnl}{flots} de la grande mer\index{gnl}{mer} du déluge\index{gnl}{deluge@déluge}. (\textit{PP} 3923)\\
\end{verse}

\noindent
Ainsi, Campantar\index{gnl}{Campantar} et la légende\index{gnl}{legende@légende} du déluge\index{gnl}{deluge@déluge} sont retenus comme les deux caract\'eristiques principales pour d\'ecrire la ville de C\=\i k\=a\b li\index{gnl}{Cikali@C\=\i k\=a\b li}.

Enfin, dans l'hagiographie\index{gnl}{hagiographie} de Campantar\index{gnl}{Campantar} (\textit{PP} 1899-3154), nous relevons quatre strophes faisant allusion \`a la légende\index{gnl}{legende@légende} du déluge\index{gnl}{deluge@déluge}:

\begin{enumerate}
\item
\scriptsize
\begin{verse}
\textit{piramapuram\index{gnl}{Piramapuram} v\=e\d nupuram\index{gnl}{Venupuram@V\=e\d nupuram} pukali\index{gnl}{Pukali} peruve\.nkuru \textbf{n\=\i rp\\
poruvil tirutt\=o\d nipuram\index{gnl}{Tonipuram@T\=o\d nipuram!Tirutt\=o\d nipuram}} p\=umtar\=ay\index{gnl}{Taray@Tar\=ay} cirapuram\index{gnl}{Cirapuram} mu\b n\\
varupu\b ravam ca\d npai\index{gnl}{Canpai@Ca\d npai} nakar va\d lar k\=a\b li koccai\index{gnl}{Koccai} vayam\\
paravu tirukka\b lumalam\index{gnl}{Kalumalam@Ka\b lumalam} \=am pa\b n\b nira\d n\d tu tiruppeyartt\=al.} (\textit{PP} 1912)\\
\end{verse}

\normalsize
\begin{verse}
Piramapuram\index{gnl}{Piramapuram}, V\=e\d nupuram\index{gnl}{Venupuram@V\=e\d nupuram}, Pukali\index{gnl}{Pukali}, la grande Ve\.nkuru\index{gnl}{Venkuru@Ve\.nkuru},\\
L'incomparable T\=o\d nipuram\index{gnl}{Tonipuram@T\=o\d nipuram} des eaux\index{gnl}{eau}, la belle Tar\=ay\index{gnl}{Taray@Tar\=ay}, Cirapuram\index{gnl}{Cirapuram},\\
Pu\b ravam\index{gnl}{Puravam@Pu\b ravam} qui vient d'antan, la ville de Ca\d npai\index{gnl}{Canpai@Ca\d npai},\\
La fertile K\=a\b li\index{gnl}{Kali@K\=a\b li}, Koccai\index{gnl}{Koccai}vayam, Ka\b lumalam\index{gnl}{Kalumalam@Ka\b lumalam} l'adorable,\\
Tels sont les douze\index{gnl}{douze} noms saints. (\textit{PP} 1912)\\
\end{verse}

\noindent
\textit{PP} 1912 est l'unique strophe qui \'enum\'ere, dans cet ordre\index{gnl}{ordre} d\'efini, les douze\index{gnl}{douze} toponymes de C\=\i k\=a\b li\index{gnl}{Cikali@C\=\i k\=a\b li} parmi lesquels seul T\=o\d nipuram\index{gnl}{Tonipuram@T\=o\d nipuram} est pr\'esent\'e avec son mythe\index{gnl}{mythe} fondateur.

\item
\scriptsize
\begin{verse}
\textit{t\=ekkiya m\=ama\b rai ve\d l\d lat tirutt\=o\d ni v\=\i \b r\b rirunt\=arai} (\textit{PP} 2173c)\\
\end{verse}

\normalsize
\begin{verse}
Celui qui est assis majestueusement\\
Sur le saint radeau\index{gnl}{radeau} des flots\index{gnl}{flots}\\
Des grands \textit{Veda}\index{gnl}{Veda@\textit{Veda}} abondants (\textit{PP} 2173c)\\
\end{verse}

\item
\scriptsize
\begin{verse}
\textit{\=u\b li mu\d tivil uyarnta ve\d l\d lattu \=o\.nkiya k\=a\b li uyar patiyil\\
\dots} (\textit{PP} 2449)\\
\end{verse}

\normalsize
\begin{verse}
Dans la haute ville de K\=a\b li\index{gnl}{Kali@K\=a\b li} qui s'\'eleva\\
Sur les flots\index{gnl}{flots} grandissants \`a la fin du \textit{yuga}\\
\dots\ (\textit{PP} 2449)\\
\end{verse}

\item
\scriptsize
\begin{verse}
\textit{p\=ota n\=\i \d tu m\=a ma\b raiyavar etir ko\d lap pukali\index{gnl}{Pukali} k\=avalarum tam\\
c\=\i ta muttu a\d nic civikai ni\b n\b ru i\b lintu etir celpavar tirutt\=o\d ni\\
n\=atar k\=oyil mu\b n to\b n\b ri\d ta nakai malark karam kuvittu i\b raiñcip p\=oy\\
\=ota \textbf{n\=\i ri\b n m\=el \=o\.nku k\=oyil}i\b n ma\d nik k\=opuram ce\b n\b ru\b r\b r\=ar.} (\textit{PP} 2850)\\
\end{verse}

\normalsize
\begin{verse}
Quand les grands [brahmane\index{gnl}{brahmane}s]\\
Des \textit{Veda}\index{gnl}{Veda@\textit{Veda}} \`a la haute sagesse\\
Vinrent \`a [sa] rencontre\\
Le protecteur de Pukali\index{gnl}{Pukali} aussi,\\
Descendant du palanquin\index{gnl}{palanquin} orn\'e de perles fra\^iches,\\
Alla \`a [leur] rencontre;\\
Alors qu'apparut devant [eux]\\
Le temple\index{gnl}{temple} du Seigneur du radeau\index{gnl}{radeau} sacré\\
Il chanta en honorant, les mains jointes,\\
Telles des fleurs parfum\'ees,\\
Et avan\c ca jusqu'au beau pavillon d'entr\'ee\\
Du temple\index{gnl}{temple} qui s'\'el\`eve sur les eaux\index{gnl}{eau}. (\textit{PP} 2850)\\
\end{verse}
\end{enumerate}

Ainsi, mis \`a part les huit hymne\index{gnl}{hymne}s attribu\'es \`a Campantar\index{gnl}{Campantar} (sur les onze qui c\'el\`ebrent les douze\index{gnl}{douze} toponymes de C\=\i k\=a\b li\index{gnl}{Cikali@C\=\i k\=a\b li}), tous les autres textes attribu\'es \`a Campantar\index{gnl}{Campantar}, Appar\index{gnl}{Appar}, Cuntarar\index{gnl}{Cuntarar}, Pa\d t\d ti\b nattuppi\d l\d lai\index{gnl}{Pattinattu Pillai@Pa\d t\d ti\b nattu Pi\d l\d lai}, Nampi \=A\d n\d t\=ar Nampi\index{gnl}{Nampi \=A\d n\d t\=ar Nampi} et C\=ekki\b l\=ar\index{gnl}{Cekkilar@C\=ekki\b l\=ar}, m\^eme s'ils mentionnent parfois l'unit\'e des douze\index{gnl}{douze} noms (\textit{Tirumu\b rai}\index{gnl}{Tirumurai@\textit{Tirumu\b rai}} \textsc{xi} et \textsc{xii}), ne font allusion qu'\`a une seule légende\index{gnl}{legende@légende}, celle la ville de T\=o\d nipuram\index{gnl}{Tonipuram@T\=o\d nipuram} qui flotta\index{gnl}{flotter} sur les eaux\index{gnl}{eau} du déluge\index{gnl}{deluge@déluge} apocalyptique. Cette in\'egalit\'e dans le traitement des douze\index{gnl}{douze} légende\index{gnl}{legende@légende}s ne signalerait-elle pas une forme de \og bricolage mythique\fg\footnote{Nous remercions Charlotte \textsc{Schmid} pour cette expression.}\ autour de l'entit\'e des douze\index{gnl}{douze} noms?

\subsection{Mise en légende: de Pa\d t\d ti\b nattuppi\d l\d lai \`a C\=ekki\b l\=ar}

Nous avons vu que Pa\d t\d ti\b nattuppi\d l\d lai\index{gnl}{Pattinattu Pillai@Pa\d t\d ti\b nattuppi\d l\d lai} consacre un texte au site de C\=\i k\=a\b li\index{gnl}{Cikali@C\=\i k\=a\b li}, le \textit{Tirukka\b lumalamumma\d nikk\=ovai}\index{gnl}{Tirukkalumalamummanikkovai@\textit{Tirukka\b lumalamumma\d nikk\=ovai}}: il y chante Campantar\index{gnl}{Campantar} (le premier hymne\index{gnl}{hymne} du corpus\index{gnl}{corpus} du \textit{T\=ev\=aram}, un des miracles de Campantar\index{gnl}{miracle}) et son site (la légende\index{gnl}{legende@légende} de T\=o\d nipuram\index{gnl}{Tonipuram@T\=o\d nipuram} et l'unit\'e de douze\index{gnl}{douze} toponymes).

Sept textes attribu\'es \`a Nampi \=A\d n\d t\=ar Nampi\index{gnl}{Nampi \=A\d n\d t\=ar Nampi} c\'el\`ebrent grandement Campantar\index{gnl}{Campantar} et son site. Nous y avons relev\'e de nombreuses r\'ef\'erences aux miracle\index{gnl}{miracle}s et aux hymne\index{gnl}{hymne}s de Campantar\index{gnl}{Campantar} ainsi qu'\`a la légende\index{gnl}{legende@légende} de T\=o\d nipuram\index{gnl}{Tonipuram@T\=o\d nipuram}. Le tableau 5.2 ci-dessous recense toutes les occurrences des douze\index{gnl}{douze} toponymes dans ces textes\footnote{\textit{TTA} est notre abbr\'eviation pour le \textit{Tirutto\d n\d tar Tiruvant\=ati}. Cf. note 7 de 5.1.2 pour les autres abbr\'eviations.}.


\scriptsize
\begin{center}
\begin{longtable}{|c|c|c|c|c|c|c|c|}
\caption{Les douze noms dans les textes attribu\'es \`a Nampi \=A\d n\d t\=ar Nampi}\endfirsthead
\hline
\textit{Toponymes}&\textit{TTA}&\textit{APCV}&\textit{APK}&\textit{APMK}&\textit{APA}&\textit{APUM}&\textit{APT}\endhead
\hline
\textit{Toponymes}&\textit{TTA}&\textit{APCV}&\textit{APK}&\textit{APMK}&\textit{APA}&\textit{APUM}&\textit{APT}\\
\hline
\hline
Piramapuram\index{gnl}{Piramapuram}&34&&&&2, 100&56&\\
\hline
V\=e\d nupuram\index{gnl}{Venupuram@V\=e\d nupuram}&&&9, 31&&7, 100&56&\\
\hline
Pukali\index{gnl}{Pukali}&&&4, 5, 12,&7.4, 10.3&17, 35, 40,&58, 70,&\\
&&& 22, 23, 25,&& 43, 54, 66,& 88(x2)&\\
&&& 28(x2), 29,&& 100&&\\
&&& 33, 34, 38,&&&&\\
&&& 43, 48, 49&&&&\\
\hline
Ve\.nkuru\index{gnl}{Venkuru@Ve\.nkuru}&&&&&57, 100&57&\\
\hline
T\=o\d nipuram\index{gnl}{Tonipuram@T\=o\d nipuram}&&&34, 41&&91, 94,&57&\\
&&&&& 100, 101&&\\
\hline
Tar\=ay\index{gnl}{Taray@Tar\=ay}&&&&&4, 100&57&\\
\hline
Cirapuram\index{gnl}{Cirapuram}&&&24&&26, 39, 100&56&\\
\hline
Pu\b ravam\index{gnl}{Puravam@Pu\b ravam}&&&35&&29, 30, 100&58&\\
\hline
Ca\d npai\index{gnl}{Canpai@Ca\d npai}&19, 84&toutes les& 1, 3, 8,&&5, 9, 14,&56&\\
&& strophes& 11, 14, 16,&& 20, 21, 31,&&\\
&&& 20, 43, 45&& 46, 51, 62,&&\\
&&&&& 64, 65, 77,&&\\
&&&&& 82(x2), 86,&&\\
&&&&& 87, 98, 100&&\\
\hline
K\=a\b li\index{gnl}{Kali@K\=a\b li}&35, 47, 61&&1, 2, 13, 19,&1.10, 5,&3, 15, 24,&56, 125&l. 9\\
&&&30, 40& 6, 9, 11,& 44, 58, 60,&&\\
&&&&12, 14, 15,& 61, 69, 95,&&\\
&&&& 16, 17, 23,& 96, 100&&\\
&&&& 25-29&&&\\
\hline
Koccai\index{gnl}{Koccai}&&&10, 26, 27,&&8, 32, 36,&57&l. 41\\
&&&44, 47&& 46, 47, 53,&&\\
&&&&& 55, 56, 79,&&\\
&&&&& 81, 98, 100&&\\
\hline
Ka\b lumalam\index{gnl}{Kalumalam@Ka\b lumalam}&&&17, 18, 37,&4, 13&22, 34, 50,&58, 132&\\
&&&42, 46&&93, 94, 100&&\\
\hline
\end{longtable}
\end{center}


\noindent
\normalsize
L'\'etude des miracle\index{gnl}{miracle}s de Campantar\index{gnl}{Campantar} dans ces sept textes nous avait conduit \`a poser plusieurs hypoth\`eses: Nampi \=A\d n\d t\=ar Nampi\index{gnl}{Nampi \=A\d n\d t\=ar Nampi} n'est pas l'unique auteur de ces \oe uvres; l'\textit{APUM} et l'\textit{APT} sont les poème\index{gnl}{poeme@poème}s les plus tardifs de cet ensemble; ces deux poèmes ne viennent pas de la m\^eme transmission que le \textit{Periyapur\=a\d nam}\index{gnl}{Periyapuranam@\textit{Periyapur\=a\d nam}}. Nous constatons ici que tous les textes ne pr\'esentent pas les douze\index{gnl}{douze} noms de C\=\i k\=a\b li\index{gnl}{Cikali@C\=\i k\=a\b li}. Seuls l'\textit{APA} 100\footnote{L'\textit{APA} 1 mentionne l'unit\'e des douze\index{gnl}{douze} noms sans les citer.} et l'\textit{APUM} 56-58 \'enum\`erent les douze\index{gnl}{douze} toponymes dans un ordre\index{gnl}{ordre} diff\'erent de celui observ\'e dans les hymne\index{gnl}{hymne}s attribu\'es \`a Campantar\index{gnl}{Campantar} dans le \textit{T\=ev\=aram} et de celui adopté dans le \textit{Periyapur\=a\d nam}\index{gnl}{Periyapuranam@\textit{Periyapur\=a\d nam}}. Ainsi, notre hypoth\`ese de transmission multiple en sort renforc\'ee.

Dans le \textit{Periyapur\=a\d nam}\index{gnl}{Periyapuranam@\textit{Periyapur\=a\d nam}} l'unit\'e des douze\index{gnl}{douze} noms de C\=\i k\=a\b li\index{gnl}{Cikali@C\=\i k\=a\b li} n'est pr\'esent\'ee que dans la strophe 1912 et ce, nous l'avons dit, dans le m\^eme ordre\index{gnl}{ordre} que celui des hymne\index{gnl}{hymne}s attribu\'es \`a Campantar\index{gnl}{Campantar}. Si les onze hymne\index{gnl}{hymne}s aux douze\index{gnl}{douze} noms sont des ajouts comme nous le supposons, il s'agirait d'additions influenc\'ees par le \textit{Periyapur\=a\d nam}\index{gnl}{Periyapuranam@\textit{Periyapur\=a\d nam}} plut\^ot que par les textes attribu\'es \`a Nampi \=A\d n\d t\=ar Nampi\index{gnl}{Nampi \=A\d n\d t\=ar Nampi}. Ajoutons que T\=o\d nipuram\index{gnl}{Tonipuram@T\=o\d nipuram} s'y distingue tr\`es nettement parce que seul son mythe\index{gnl}{mythe} fondateur est \'evoqu\'e et parce qu'il sert \`a d\'ecrire de mani\`ere exclusive le temple\index{gnl}{temple} de C\=\i k\=a\b li\index{gnl}{Cikali@C\=\i k\=a\b li} et le \'Siva\index{gnl}{Siva@\'Siva} qui y r\'eside. Nous observons aussi un traitement in\'egal de ces douze\index{gnl}{douze} toponymes dans l'hagiographie\index{gnl}{hagiographie} de Campantar\index{gnl}{Campantar}: Pukali\index{gnl}{Pukali}, Ca\d npai\index{gnl}{Canpai@Ca\d npai}, K\=a\b li\index{gnl}{Kali@K\=a\b li}, T\=o\d nipuram\index{gnl}{Tonipuram@T\=o\d nipuram} et Ka\b lumalam\index{gnl}{Kalumalam@Ka\b lumalam} sont les noms les plus employ\'es. Notons que cet emploi disproportionn\'e des noms ressemble fortement \`a ce que nous avons constat\'e dans les envois\index{gnl}{envoi} attribu\'es \`a Campantar\index{gnl}{Campantar}.

\begin{center}
*
\end{center}

La reprise des poème\index{gnl}{poeme@poème}s du corpus\index{gnl}{corpus} \'etabli du \textit{T\=ev\=aram}\index{gnl}{Tevaram@\textit{T\=ev\=aram}} pour ce qui est des douze noms de la ville dans le \textit{Periyapur\=a\d nam}\index{gnl}{Periyapuranam@\textit{Periyapur\=a\d nam}} a pouss\'e jusqu'\`a pr\'esent les chercheurs \`a consid\'erer que C\=ekki\b l\=ar\index{gnl}{Cekkilar@C\=ekki\b l\=ar} a int\'egr\'e dans son hagiographie\index{gnl}{hagiographie} sur Campantar\index{gnl}{Campantar} tous les \'el\'ements qu'il aurait pu relever dans les hymne\index{gnl}{hymne}s. Or, l'examen interne des hymne\index{gnl}{hymne}s attribu\'es \`a Campantar\index{gnl}{Campantar} et toutes ces correspondances \og \'etonnantes\fg\ avec le \textit{Periyapur\=a\d nam}\index{gnl}{Periyapuranam@\textit{Periyapur\=a\d nam}} nous permet dans ce cas de consid\'erer les informations sous un autre angle. Si l'auteur du \textit{Periyapur\=a\d nam}\index{gnl}{Periyapuranam@\textit{Periyapur\=a\d nam}} n'avait pas simplement repris ou exploit\'e des hymne\index{gnl}{hymne}s existants mais avait aussi, pour construire et justifier son propre r\'ecit, particip\'e aux ajouts de certains passages et \`a l'\'etablissement du corpus\index{gnl}{corpus} du \textit{T\=ev\=aram}\index{gnl}{Tevaram@\textit{T\=ev\=aram}}?



\chapter{La mécanique hagiographique}

%\epigraphhead[450]{
\epigraph{C\=ekki\b l\=ar\index{gnl}{Cekkilar@C\=ekki\b l\=ar} s'est exactement incorpor\'e au \textit{T\=ev\=aram}\index{gnl}{Tevaram@\textit{T\=ev\=aram}} et plus personne apr\`es lui n'a su le lire sans lui.}{F. \textsc{Gros} (1984: xi), \textit{Pour lire le T\=ev\=aram}}

En introduction \`a ce chapitre qui cherche \`a reconstituer le travail de C\=ekki\b l\=ar\index{gnl}{Cekkilar@C\=ekki\b l\=ar}, nous pr\'esentons un r\'esum\'e de la légende\index{gnl}{legende@légende} \'etablie de Campantar\index{gnl}{Campantar}, telle qu'elle nous est parvenue, cristallis\'ee, par le \textit{Periyapur\=a\d nam}\index{gnl}{Periyapuranam@\textit{Periyapur\=a\d nam}}.

L'hagiographie\index{gnl}{hagiographie} de Campantar\index{gnl}{Campantar} b\'en\'eficie d'un traitement particulier au sein du \textit{Periyapur\=a\d nam}\index{gnl}{Periyapuranam@\textit{Periyapur\=a\d nam}}. Nous avons soulign\'e sa longueur exceptionnelle au chapitre pr\'ec\'edent (mille deux cent cinquante-six strophes). Contrairement aux autres légende\index{gnl}{legende@légende}s de dévot\index{gnl}{devot(e)@dévot(e)}s exemplaires, il n'y a là ni mise \`a l'épreuve\index{gnl}{epreuve@épreuve}, ni \textit{deus ex machina}. Campantar\index{gnl}{Campantar} est un héros\index{gnl}{heros@héros} ma\^itre de ses actions, charg\'e d'une mission d\`es la naissance\index{gnl}{naissance}: restaurer le shiva\"isme\index{gnl}{shivaisme@shiva\"isme} face aux hérétique\index{gnl}{heretique@hérétique}s, ja\"in\index{gnl}{jain@ja\"in}s et bouddhiste\index{gnl}{bouddhiste}s. Par cons\'equent, le poète obtient t\^ot la gr\^ace divine, la connaissance\index{gnl}{connaissance} et la puissance. Campantar\index{gnl}{Campantar} est un enfant\index{gnl}{enfant} de l'\^age Kali, qui n'est pas joueur \`a l'image\index{gnl}{image} de ce K\textsubring{r}\d s\d na\index{gnl}{Krsna@K\textsubring{r}\d s\d na} qui l'inspire en partie, mais qui endosse la responsabilit\'e de (re)conqu\'erir pour \'Siva le Pays Tamoul\index{gnl}{Pays Tamoul} par ses p\`elerinage\index{gnl}{pelerinage@pèlerinage}s, arm\'e de ses poème\index{gnl}{poeme@poème}s.

Le \textit{pur\=a\d nam} de Campantar\index{gnl}{Campantar} s'organise autour de six p\`elerinage\index{gnl}{pelerinage@pèlerinage}s dont le premier est fait \`a K\=olakk\=a\index{gnl}{Kolakka@K\=olakk\=a} (st.~1998-2003), au nord-ouest de C\=\i k\=a\b li\index{gnl}{Cikali@C\=\i k\=a\b li}\footnote{\textsc{Barnoud-Sethupathy} (1994: 47-48) reprend un texte tamoul --- le \og \textit{Tiruttala\.nga\d l Varal\=a\b ru} (Histoire des Sites) 1990 (Madras - U. V. Sv\=amin\=athaiyar Library)\fg\ --- qui d\'ecrit \og la situation des sites les uns par rapport aux autres en terme de distance mesur\'ee en temps de marche \`a pied\fg\ avec le \textit{ka\d tikai}, unit\'e de mesure de temps \'equivalant \`a vingt-quatre minutes. Elle montre ainsi que K\=olakk\=a\index{gnl}{Kolakka@K\=olakk\=a} est \`a un demi \textit{ka\d tikai} de C\=\i k\=a\b li\index{gnl}{Cikali@C\=\i k\=a\b li}, soit \`a douze\index{gnl}{douze} minutes de marche.}. Le deuxi\`eme (st.~2010-2025) et le troisi\`eme (st.~2027-2028), tr\`es courts aussi, s'inscrivent dans la zone g\'eographique au sud-est de C\=\i k\=a\b li\index{gnl}{Cikali@C\=\i k\=a\b li}, englobant par exemple Na\b nipa\d l\d li\index{gnl}{Nanipalli@Na\b nipa\d l\d li}, Talaicca\.nk\=a\d tu\index{gnl}{Talaiccankatu@Talaicca\.nk\=a\d tu}, Ve\.nk\=a\d tu\index{gnl}{Venkatu@Ve\.nk\=a\d tu} et Mullaiv\=ayil\index{gnl}{Mullaivayil@Mullaiv\=ayil}. Lors du quatri\`eme (st.~2040-2153), Campantar\index{gnl}{Campantar} se rend \`a Tillai\index{gnl}{Citamparam!Tillai} (Citamparam\index{gnl}{Citamparam}) et dans ses environs, puis descend \`a l'ouest dans la r\'egion de C\=ey\~nal\=ur\index{gnl}{Ceynalur@C\=ey\~nal\=ur}, de Vicayama\.nkai\index{gnl}{Vicayama\.nkai}, de Viyal\=ur\index{gnl}{Viyal\=ur}, etc. Le cinqui\`eme p\`elerinage\index{gnl}{pelerinage@pèlerinage} est long en dur\'ee et en distance (st.~2177-2848). Il s'\'etend vers les r\'egions de Tirucci\index{gnl}{Tirucci dt.}, puis couvre des sites tels qu'\=Ava\d tutu\b rai\index{gnl}{Avatuturai@\=Ava\d tutu\b rai}, Mayil\=a\d tutu\b rai\index{gnl}{Mayilatuturai@Mayil\=a\d tutu\b rai}, \=Ar\=ur\index{gnl}{Ar\=ur@\=Ar\=ur} et Ma\b raik\=a\d tu\index{gnl}{maraikatu@Ma\b raik\=a\d tu}, avant de se prolonger jusqu'\`a \=Alav\=ay\index{gnl}{Maturai!Alavay@\=Alav\=ay} (Maturai\index{gnl}{Maturai}). Le dernier p\`elerinage (st.~2860-3043) est effectu\'e tr\`es au nord de C\=\i k\=a\b li\index{gnl}{Cikali@C\=\i k\=a\b li}: \`a Tiruva\d n\d n\=amalai\index{gnl}{Tiruvannamalai@Tiruva\d n\d n\=amalai}, \`a K\=a\~ncipuram\index{gnl}{Kancipuram@K\=a\~ncipuram} et \`a Mayil\=apuri\index{gnl}{Mayil\=apuri} (quartier de Ce\b n\b nai). Campantar\index{gnl}{Campantar} foule ainsi la quasi-totalit\'e du sol tamoul actuel.

Entre chaque p\'er\'egrination, Campantar\index{gnl}{Campantar} revient dans sa ville natale, C\=\i k\=a\b li\index{gnl}{Cikali@C\=\i k\=a\b li}, qui marque ainsi le d\'ebut et la fin de ses circuits%\footnote{Cette image\index{gnl}{image} d'un retour syst\'ematique \`a la ville principale du personnage, apr\`es et avant chaque d\'eplacement, n'est pas sans rappeler celle du roi\index{gnl}{roi} victorieux, dans les pan\'egyriques, qui revient \`a sa capitale avant de repartir \`a la conqu\^ete d'une nouvelle contr\'ee. Par exemple, XXXXX}
. C\=\i k\=a\b li\index{gnl}{Cikali@C\=\i k\=a\b li} est le th\'e\^atre des moments marquants de son parcours personnel. En effet, \`a l'\^age de trois ans, il y obtient la gr\^ace divine en buvant le lait\index{gnl}{lait} de la d\'eesse\index{gnl}{deesse@déesse} (st.~1952-1996). D\`es lors, il poss\`ede la connaissance\index{gnl}{connaissance} divine, la ma\^itrise de la langue tamoule et le statut de chef\index{gnl}{chef}. \`A sept ans, il y re\c coit l'initiation religieuse (\textit{upanayana}) en tant que brahmane\index{gnl}{brahmane} d'une famille du \textit{kau\d n\d dinya\index{gnl}{kaundinya@\textit{kau\d n\d dinya}} gotra\index{gnl}{gotra@\textit{gotra}}} (st.~2162-2164). Nous avons constat\'e que c'est seulement apr\`es cet \'ev\'enement que ses hymne\index{gnl}{hymne}s g\'en\`erent des miracle\index{gnl}{miracle}s. C\=\i k\=a\b li\index{gnl}{Cikali@C\=\i k\=a\b li} est aussi le lieu de rencontres exceptionnelles. Campantar\index{gnl}{Campantar} y fait la connaissance\index{gnl}{connaissance} de N\=\i laka\d n\d tay\=a\b lp\=a\d nar\index{gnl}{Nilakantayalpanar@N\=\i laka\d n\d tay\=a\b lp\=a\d nar}, un joueur de \textit{y\=a\b l}, qui l'accompagne dans ses p\'er\'egrinations (st.~2029-2039), ainsi que celle de Tirun\=avukkaracar\index{gnl}{Appar!Tirunavukkaracar@Tirun\=avukkaracar} qui prend le surnom d'Appar\index{gnl}{Appar} parce qu'il est interpel\'e ainsi par Campantar\index{gnl}{Campantar} (st.~2166-2172). C\=\i k\=a\b li\index{gnl}{Cikali@C\=\i k\=a\b li} est enfin le si\`ege de sa cr\'eativit\'e po\'etique. Campantar\index{gnl}{Campantar} profite de ses retours \`a la source pour composer des hymne\index{gnl}{hymne}s selon des procédé\index{gnl}{procédé littéraire}s litt\'eraires complexes (st.~2174-2176).

C\=ekk\=\i \b l\=ar a tr\`es habilement nou\'e la vie de Campantar\index{gnl}{Campantar} \`a celle d'autres \textit{n\=aya\b nm\=ar}\index{gnl}{nayanmar@\textit{n\=aya\b nm\=ar}} en les mettant en scène dans le r\'ecit de vie du poète\index{gnl}{poete@poète}. Les personnages se croisent, les trames se r\'ep\`etent d'une hagiographie\index{gnl}{hagiographie} \`a l'autre et l'unit\'e du \textit{Periyapur\=a\d nam}\index{gnl}{Periyapuranam@\textit{Periyapur\=a\d nam}} en sort renforc\'ee. Campantar\index{gnl}{Campantar} rencontre dans ses d\'eplacements N\=\i lanakkar\index{gnl}{Nilanakkar@N\=\i lanakkar} (st.~2358), Ci\b rutto\d n\d tar\index{gnl}{Ciruttontar@Ci\b rutto\d n\d tar} (st.~2367), Muruka\b n\index{gnl}{Murukan@Muruka\b n} (st.~2387), Ku\.nkuliyakkalayar\index{gnl}{Kunkuliyakkalayar@Ku\.nkuliyakkalayar} (st.~2431), le roi\index{gnl}{roi}, la reine\index{gnl}{reine} et le ministre\index{gnl}{ministre} \textit{p\=a\d n\d dya}\index{gnl}{pandya@\textit{p\=a\d n\d dya}} (st.~2552 et suiv.). Par ailleurs il rend hommage \`a certains autres dévot\index{gnl}{devot(e)@dévot(e)}s dans leur villages natals: Ca\d n\d ti\index{gnl}{Candesa@Ca\d n\d de\'sa!Ca\d n\d ti} \`a C\=ey\~nal\=ur\index{gnl}{Ceynalur@C\=ey\~nal\=ur} (st.~2140), Ce\.nka\d n \`a \=A\b naikk\=a\index{gnl}{Anaikka@\=A\b naikk\=a} (st.~2242) et K\=araikk\=alammaiy\=ar\index{gnl}{Karaikkalammaiyar@K\=araikk\=alammaiy\=ar} \`a \=Ala\.nk\=a\d tu\index{gnl}{Alankatu@\=Ala\.nk\=a\d tu} (st.~2906).

En route et dans les temples\index{gnl}{temple} qui sont ses destinations, Campantar\index{gnl}{Campantar} est associ\'e \`a de nombreux miracle\index{gnl}{miracle}s. Les premiers sont le fait de \'Siva\index{gnl}{Siva@\'Siva} ou de sa par\`edre: Campantar\index{gnl}{Campantar} boit le lait\index{gnl}{lait} divin \`a C\=\i k\=a\b li\index{gnl}{Cikali@C\=\i k\=a\b li} (st.~1952-1996), re\c coit des cymbale\index{gnl}{cymbale}s d'or \`a K\=olakk\=a\index{gnl}{Kolakka@K\=olakk\=a} (st.~1998-2003), acquiert un palanquin\index{gnl}{palanquin} et un parasol\index{gnl}{parasol} \`a Arattu\b rai\index{gnl}{Arattu\b rai} (st.~2083-2130) et il est escort\'e par les \textit{ga\d na} de \'Siva\index{gnl}{Siva@\'Siva} qui le prot\`egent du soleil \`a Pa\d t\d t\=\i caram\index{gnl}{Patticaram@Pa\d t\d t\=\i caram} (st.~2289-2296). Ensuite, Campantar\index{gnl}{Campantar}, ou plut\^ot ses poème\index{gnl}{poeme@poème}s, deviennent l'auteur des prodige\index{gnl}{prodige}s. Ainsi, dans l'ordre\index{gnl}{ordre} de la narration,
\`a P\=accil\=accir\=amam\index{gnl}{Paccilacciramam@P\=accil\=accir\=amam}, il gu\'erit la jeune fille du chef\index{gnl}{chef} Kollima\b lava\b n atteinte par \textit{muyalaka\b n}, l'\'epilepsie (st.~2208-2218).
\`A Ko\d tim\=a\d tacce\.nku\b n\b r\=ur\index{gnl}{cenkunrur@Ce\.nku\b n\b r\=ur}, il soigne la fièvre\index{gnl}{fievre@fièvre} des dévot\index{gnl}{devot(e)@dévot(e)}s caus\'ee par le froid hivernal (st.~2222-2234).
\`A \=Ava\d tutu\b rai\index{gnl}{Avatuturai@\=Ava\d tutu\b rai}, pour que son père\index{gnl}{pere@père} aille c\'el\'ebrer un sacrifice \`a C\=\i k\=a\b li\index{gnl}{Cikali@C\=\i k\=a\b li}, il obtient de \'Siva mille pièce\index{gnl}{piece@pièce}s d'or que les \textit{ga\d na} d\'eposent sur l'autel \`a offrande (st.~2315-2328).
\`A Marukal\index{gnl}{Marukal}, il ressuscite un jeune homme mordu par un serpent\index{gnl}{serpent} (st.~2370-2381).
\`A V\=\i \b limi\b lalai\index{gnl}{Vilimilalai@V\=\i \b limi\b lalai}, o\`u la s\`echeresse a entra\^in\'e la famine, avec Appar\index{gnl}{Appar}, il re\c coit du dieu\index{gnl}{dieu} des pièce\index{gnl}{piece@pièce}s d'or afin de nourrir quotidiennement les dévot\index{gnl}{devot(e)@dévot(e)}s venus aux monast\`ere\index{gnl}{monastère}s (st.~2460-2470).
\`A Ma\b raikk\=a\d tu\index{gnl}{Maraikkatu@Ma\b raikk\=a\d tu}, alors qu'Appar\index{gnl}{Appar} chante plusieurs strophes pour ouvrir les portes du temple\index{gnl}{temple}, rest\'e ferm\'e depuis que les \textit{Veda}\index{gnl}{Veda@\textit{Veda}} l'ont honor\'e, Campantar\index{gnl}{Campantar}, en un seul quatrain, les referme (st.~2474-2488).
\`A \=Alav\=ay\index{gnl}{Maturai!Alavay@\=Alav\=ay}, \`a l'issue de quatre confrontations, il vainc les ja\"in\index{gnl}{jain@ja\"in}s et convertit le roi\index{gnl}{roi} \textit{p\=a\d n\d dya}\index{gnl}{pandya@\textit{p\=a\d n\d dya}} au shiva\"isme\index{gnl}{shivaisme@shiva\"isme} (st.~2497-2782).
\`A Mu\d l\d liv\=aykkarai\index{gnl}{Mullivaykkarai@Mu\d l\d liv\=aykkarai}, il conduit, malgr\'e le courant, une barque\index{gnl}{barque} jusqu'\`a la rive oppos\'ee o\`u se trouve Ko\d l\d lamp\=ut\=ur\index{gnl}{Kollamputur@Ko\d l\d lamp\=ut\=ur} (st.~2794-2798).
\`A Te\d licc\=eri\index{gnl}{Telicceri@Te\d licc\=eri}, il d\'efait des bouddhiste\index{gnl}{bouddhiste}s et les convertit (st.~2802-2823).
\`A \=Ott\=ur\index{gnl}{Ottur@\=Ott\=ur}, il transforme des palmier\index{gnl}{palmier}s m\^ales en femelles afin qu'ils donnent des fruits (st.~2871-2881).
\`A Mayil\=apuri\index{gnl}{Mayil\=apuri}, il fait rena\^itre des cendre\index{gnl}{cendre}s une jeune fille, nomm\'ee P\=ump\=avai\index{gnl}{Pumpavai@P\=ump\=avai}, tu\'ee par un serpent\index{gnl}{serpent} (st.~2931-3018).
Enfin, \`a Peruma\d nam (\=Acc\=a\d lpuram\index{gnl}{Accalpuram@\=Acc\=a\d lpuram}), le jour de son mariage\index{gnl}{mariage}, il entre dans le feu\index{gnl}{feu} de \'Siva\index{gnl}{Siva@\'Siva} et rejoint ses pieds en compagnie de tous les invit\'es (st.~3053-3152).\\

Le \textit{Periyapur\=a\d nam}\index{gnl}{Periyapuranam@\textit{Periyapur\=a\d nam}} associe donc chaque miracle\index{gnl}{miracle} et autres faits marquants de la vie de Campantar\index{gnl}{Campantar} \`a un site g\'eographique pr\'ecis et matériel, mais surtout \`a un hymne\index{gnl}{hymne} existant dans le corpus\index{gnl}{corpus} actuel du \textit{T\=ev\=aram}\index{gnl}{Tevaram@\textit{T\=ev\=aram}}. C'est pourquoi les hymne\index{gnl}{hymne}s attribu\'es \`a Campantar\index{gnl}{Campantar} sont souvent lus et interpr\'et\'es \`a la lumi\`ere du discours hagiographique\index{gnl}{hagiographie!hagiographique}. Nous avons procédé\index{gnl}{procédé littéraire} \`a une d\'emarche inverse en commen\c cant par analyser les donn\'ees internes aux poème\index{gnl}{poeme@poème}s du \textit{T\=ev\=aram}\index{gnl}{Tevaram@\textit{T\=ev\=aram}}. Examinons maintenant le travail de C\=ekki\b l\=ar\index{gnl}{Cekkilar@C\=ekki\b l\=ar}.

\section{C\=ekki\b l\=ar le grand assimilateur}

Dans le contexte du \textit{T\=ev\=aram}\index{gnl}{Tevaram@\textit{T\=ev\=aram}}, \`a propos du g\'enie cr\'eatif de l'hagiographe C\=ekki\b l\=ar\index{gnl}{Cekkilar@C\=ekki\b l\=ar} et de l'autorit\'e de son \oe uvre momunentale, le \textit{Periyapur\=a\d nam}\index{gnl}{Periyapuranam@\textit{Periyapur\=a\d nam}}, \textsc{Gros} (1984: xi) \'ecrit:

\scriptsize
\begin{quote}
Son \'erudition aux multiples facettes demeure \`a la mesure de l'\'elite indienne: une formidable m\'emoire des textes, des r\'ecits puraniques et des anecdotes, un sens du r\'ecit, des personnages et des situations, et tout cela au service\index{gnl}{service} d'une noble cause. Sa force est d'avoir, sur le m\'etier, mont\'e d'abord la cha\^ine des hymne\index{gnl}{hymne}s eux-m\^emes, auxquels il est litt\'eralement fid\`ele dans les termes et dans les m\`etres. D\`es lors la navette peut courir entre les lisses, charg\'ee tour \`a tour d'histoire ou de légende\index{gnl}{legende@légende}, de mythologie ou de doctrine, de tradition\index{gnl}{tradition} authentique ou de pieux mensonge: au terme de l'\oe uvre, la trame la plus diverse est inextricablement int\'egr\'ee \`a la v\'erit\'e sup\'erieure et \`a l'autorit\'e des hymne\index{gnl}{hymne}s. C\=ekki\b l\=ar\index{gnl}{Cekkilar@C\=ekki\b l\=ar} s'est exactement incorpor\'e au \textit{T\=ev\=aram}\index{gnl}{Tevaram@\textit{T\=ev\=aram}} et plus personne apr\`es lui n'a su le lire sans lui.
\end{quote}

\normalsize
Cette \og formidable m\'emoire des textes, des r\'ecits puraniques et des anecdotes\fg, perceptible tout le long de l'hagiographie\index{gnl}{hagiographie}, t\'emoigne du g\'enie d'un auteur appartenant \`a la soci\'et\'e prosp\`ere de l'apog\'ee \textit{c\=o\b la}\index{gnl}{cola@\textit{c\=o\b la}} marqu\'ee par la floraison du \'Saiva Siddh\=anta\index{gnl}{Saiva@\'Saiva Siddh\=anta}, par la r\'edaction de grandes \oe uvres litt\'eraires (\textit{C\=\i vakacint\=ama\d ni}\index{gnl}{Civakacintamani@\textit{C\=\i vakacint\=ama\d ni}}, \textit{Kampar\=am\=aya\b nam}\index{gnl}{Kamparamayanam@\textit{Kampar\=am\=aya\b nam}}, etc.) et par l'aura croissante de Citamparam\index{gnl}{Citamparam}.
Les sources officielles du \textit{Periyapur\=a\d nam}\index{gnl}{Periyapuranam@\textit{Periyapur\=a\d nam}} donn\'ees dans le texte m\^eme sont: le \textit{Tirutto\d n\d tar tokai} attribu\'e \`a Cuntarar\index{gnl}{Cuntarar} (st. 47-48 et 349) ainsi qu'un texte de Nampi \=A\d n\d t\=ar Nampi\index{gnl}{Nampi \=A\d n\d t\=ar Nampi} (st. 49), fort probablement le \textit{Tirutto\d n\d tar ant\=ati}. L'hagiographie\index{gnl}{hagiographie} de C\=ekki\b l\=ar et l'\textit{ant\=ati} de Nampi reprennent fid\`element l'ordre\index{gnl}{ordre} d'\'enum\'eration des soixante-trois \textit{n\=aya\b nm\=ar}\index{gnl}{nayanmar@\textit{n\=aya\b nm\=ar}} et des neuf groupes de dévot\index{gnl}{devot(e)@dévot(e)}s pr\'esent\'es dans l'hymne\index{gnl}{hymne} du \textit{T\=ev\=aram}\index{gnl}{Tevaram@\textit{T\=ev\=aram}}. \`A partir de cette base structurale C\=ekki\b l\=ar\index{gnl}{Cekkilar@C\=ekki\b l\=ar} d\'eveloppe les r\'ecits de vie des dévot\index{gnl}{devot(e)@dévot(e)}s exemplaires. Vraisemblablement, il s'appuie aussi sur d'autres textes \'epigraphiques et litt\'eraires. Whitney \textsc{Cox}, dans son travail consacr\'e aux textes de la p\'eriode tardive de la dynastie \textit{c\=o\b la}\index{gnl}{cola@\textit{c\=o\b la}} \`a Citamparam\index{gnl}{Citamparam}, recense de fa\c con convaincante les \'ecrits sanskrits et tamouls qui ont tr\`es certainement influenc\'e l'auteur du \textit{Periyapur\=a\d nam}\index{gnl}{Periyapuranam@\textit{Periyapur\=a\d nam}} (\textsc{Cox} 2006a: 73-93). Pour reprendre ses termes : \og C\=ekki\b l\=ar\index{gnl}{Cekkilar@C\=ekki\b l\=ar} was a voracious assimilator of other texts\fg\ (\textsc{Cox} 2006a:77).

Concentrons-nous sur l'hagiographie\index{gnl}{hagiographie} de Campantar\index{gnl}{Campantar} pour rendre compte de l'\'etat du corpus\index{gnl}{corpus} du \textit{T\=ev\=aram}\index{gnl}{Tevaram@\textit{T\=ev\=aram}} au milieu du \textsc{xii}\up{e} si\`ecle.
S'il peut y avoir des r\'ef\'erences aux hymne\index{gnl}{hymne}s attribu\'es \`a Campantar\index{gnl}{Campantar} dans les r\'ecits de vie des autres \textit{n\=aya\b nm\=ar}\index{gnl}{nayanmar@\textit{n\=aya\b nm\=ar}}\footnote{Cf. la légende\index{gnl}{legende@légende} de Ci\b rappuli (PP 3654).} c'est, essentiellement, dans la partie du \textit{Periyapur\=a\d nam} organisée autour de Campantar que nous trouvons des citations, des paraphrases ou des r\'ef\'erences pr\'ecises aux poème\index{gnl}{poeme@poème}s de l'enfant\index{gnl}{enfant}-poète\index{gnl}{poete@poète}\footnote{Nous partons de l'\'etude de ces r\'ef\'erences faite par \textsc{Gopal Iyer} (1991: 18-20).}. C\=ekki\b l\=ar\index{gnl}{Cekkilar@C\=ekki\b l\=ar} cite les premiers mots d'un hymne\index{gnl}{hymne}\footnote{PP 1974, 2000, 2005, 2018, 2020, 2060, 2072, 2112, 2216, 2248, 2252, 2311, 2334, 2345, 2354, 2680, 2384, 2395, 2397, 2405, 2416, 2422, 2430, 2432, 2440, 2443, 2455, 2485, 2494, 2514, 2561, 2602, 2637c, 2658, 2682, 2720-2743, 2763, 2779, 2796, 2800, 2852, 2863, 2865, 2870, 2893, 2908, 2910, 2929, 2986, 3045, 3143.}
ou, moins fr\'equemment, d'autres passages du premier quatrain\footnote{PP 2193, 2216, 2252, 2272, 2311, 2380, 2396, 2405, 2986, 3045.}.
Parfois, il rapporte simplement les refrains\footnote{PP 2079, 2082, 2164, 2270, 2305, 2322, 3026, 3031, 3046.}
ou une portion de l'envoi\index{gnl}{envoi}\footnote{PP 2013, 2878, 3029.}.
Les citations des textes que nous poss\'edons aujourd'hui sont pr\'esent\'ees quelquefois avec de l\'eg\`eres variations\footnote{PP 2024, 2081, 2201, 2204, 2233, 2270, 2272, 2305, 2380, 2468, 2565, 2637a, 2720-2743, 2748, 2868, 2918, 2926.}
ou par des paraphrases\footnote{PP 2193, 2662, 2682, 2852.}.
Un hymne\index{gnl}{hymne}, III 54, b\'en\'eficie d'un traitement exceptionnel. Les premiers mots de chaque strophe sont repris et expliqu\'es. C\=ekki\b l\=ar\index{gnl}{Cekkilar@C\=ekki\b l\=ar} fournit un v\'eritable commentaire de ce poème\index{gnl}{poeme@poème} (PP 2720-2743).
Dans le tableau qui suit nous avons aussi relev\'e les r\'ef\'erences aux procédé\index{gnl}{procédé littéraire}s litt\'eraires\footnote{PP 2080, 2174-2175, 2195, 2323, 2768, 2897, 3021.} et \`a un hymne\index{gnl}{hymne} particulier nomm\'e \textit{namacciv\=aya patikam} (PP 3146).

Ainsi, nous avons r\'epertori\'e quatre-vingt-cinq r\'ef\'erences directes aux hymne\index{gnl}{hymne}s attribu\'es \`a Campantar\index{gnl}{Campantar}. Aucune citation donn\'ee par le \textit{pur\=a\d nam} n'appartient \`a un texte perdu ou inexistant dans le corpus\index{gnl}{corpus} du \textit{T\=ev\=aram}\index{gnl}{Tevaram@\textit{T\=ev\=aram}} aujourd'hui à notre disposition. Sur les soixante-sept poème\index{gnl}{poeme@poème}s d\'edi\'es \`a C\=\i k\=a\b li\index{gnl}{Cikali@C\=\i k\=a\b li} seuls cinq sont cit\'es; et onze parmi les dix-huit procédé\index{gnl}{procédé littéraire}s litt\'eraires vus en 2.1.3 sont mentionn\'es. Il faut noter que tous les poème\index{gnl}{poeme@poème}s contenant une allusion autobiographique\index{gnl}{autobiographique} sont relev\'es dans le \textit{Periyapur\=a\d nam}\footnote{PP 2013, 2468, 2485, 2561, 2602, 2658, 2662, 2682, 2720-2743, 2748, 2768, 2779, 2796, 2852 et 2878.}.


\scriptsize
\begin{center}
\begin{longtable}{|c|c|c|l|l|}
\caption{Le \textit{T\=ev\=aram} dans le \textit{Periyapur\=a\d nam}}\endfirsthead
\hline
& \textit{Periyapur\=a\d nam}\index{gnl}{Periyapuranam@\textit{Periyapur\=a\d nam}} & Site & Citation & \textit{T\=ev\=aram}\index{gnl}{Tevaram@\textit{T\=ev\=aram}}\endhead
\hline
& \textit{Periyapur\=a\d nam}\index{gnl}{Periyapuranam@\textit{Periyapur\=a\d nam}} & Site & Citation & \textit{T\=ev\=aram}\index{gnl}{Tevaram@\textit{T\=ev\=aram}}\\
\hline\hline
1 & st. 1974 & C\=\i k\=a\b li\index{gnl}{Cikali@C\=\i k\=a\b li} & \textit{t\=o\d tu\d taiya ceviya\b n} & I 1.1a (Piram\=apuram)\\
\hline
2 & 2000& K\=olakk\=a\index{gnl}{Kolakka@K\=olakk\=a} & \textit{ma\d taiyil v\=a\d laika\d l p\=aya} & I 23.1a\\
\hline
3 & 2005 & C\=\i k\=a\b li\index{gnl}{Cikali@C\=\i k\=a\b li} & \textit{p\=uv\=ar ko\b n\b rai} & I 24.1a (K\=a\b li\index{gnl}{Kali@K\=a\b li})\\
\hline
4 & 2013 & Na\b nipa\d l\d li\index{gnl}{Nanipalli@Na\b nipa\d l\d li} & \textit{k\=araika\d l k\=ukai mullai} & II 84.1a\\
\hline
5 & 2018 & Valampuram & \textit{ko\d tiyu\d tai} & III 103.1a\\
\hline
6 & 2020 & C\=aykk\=a\d tu & \textit{ma\d n puk\=ar} & II 41.1a\\
\hline
7& 2024 & Tiruve\d nk\=a\d tu & \textit{ka\d n k\=a\d t\d tu nuta\b n} & II 48.1a\\
& & & & \textit{ka\d n k\=a\d t\d tum nutal\=a\b num}\\
\hline
8 & 2060 & Tillai\index{gnl}{Citamparam!Tillai} & \textit{ka\b r\b r\=a\.n keriy\=ompi} & I 80.1a\\
\hline
9 & 2072 & " & \textit{\=a\d ti\b n\=ay na\b ru} & III 1.1a\\
&&& \textit{neyyo\d tu p\=a\b rayir}&\\
\hline
10 &2079& Mutuku\b n\b ram & \textit{mutuku\b n\b rai a\d taiv\=om} & refrain de I 12\\
&&&&\textit{mutuku\b n\b ru a\d taiv\=om\=e}\\
\hline
11& 2080 & " & irukkukku\b ra\d l & procédé\index{gnl}{procédé littéraire} de I 93\\
\hline
12& 2081& "& \textit{muracatirnte\b lum} & III 99.1a\\
&&&&\textit{muracatirnte\b lutaru}\\
\hline
13& 2082& T\=u\.n k\=a\b naim\=a\d tam & \textit{t\=\i \.n ku n\=\i \.n kuv\=\i r to\b lumi\b nka\d l} & refrain de I 59\\
&&&&\textit{to\b lumi\b nka\d l\=e}\\
\hline
14& 2112& Arattu\b rai\index{gnl}{Arattu\b rai} & \textit{entai y\=\i ca\b n} & II 90.1a\\
\hline
15& 2164& C\=\i k\=a\b li\index{gnl}{Cikali@C\=\i k\=a\b li}& \textit{a\~nce\b luttum\=e}& refrain de III 22 (\textit{potu})\\
&&&&pa\~nc\=akkara patikam\\
\hline
16& 2174-2175& C\=\i k\=a\b li\index{gnl}{Cikali@C\=\i k\=a\b li}& mo\b lim\=a\b r\b ru&procédé\index{gnl}{procédé littéraire} de I 117\\
&&& m\=alaim\=a\b r\b ru&procédé\index{gnl}{procédé littéraire} de III 117\\
&&& yamakam&procédé\index{gnl}{procédé littéraire} de III 113\\
&&& \=ekap\=atam&procédé\index{gnl}{procédé littéraire} de I 127\\
&&& irukkukkura\d l&procédé\index{gnl}{procédé littéraire} de I 90\\
&&& e\b luk\=u\b r\b rirukkai&procédé\index{gnl}{procédé littéraire} de I 128\\
&&& \=\i ra\d ti&procédé\index{gnl}{procédé littéraire} de III 110\\
&&& \=\i ra\d tivaippu&procédé\index{gnl}{procédé littéraire} de III 5\\
&&& n\=ala\d tim\=elvaippu&procédé\index{gnl}{procédé littéraire} de III 3\\
&&& ir\=akam&procédé\index{gnl}{procédé littéraire} de I 19, II 29,\\
&&&& II 97, III 75, III 81\\
&&&& et III 84 \\
&&&cakkaram& II 70 et II 73\\
\hline
17& 2193& T\=eva\b nku\d ti& \textit{marunto\d tu mantiram\=aki}& III 25.1 \textit{maruntu} [...]\\
&&& \textit{ma\b r\b rum ivar v\=e\d tam\=am}& \textit{mantira\.n ka\d l} [...]\\
&&&& \textit{v\=e\d ta\.nka\d l\=e}\\
\hline
18& 2195& I\b n\b nampar& i\d tai ma\d takku& procédé\index{gnl}{procédé littéraire} de III 95\\
\hline
19& 2201& Aiy\=a\b ru& \textit{k\=o\d tal k\=o\.n ka\.n ku\d lir} & II 6.1a \textit{k\=o\d tal k\=o\.n kam} \\
&&&\textit{k\=uvi\d lam}&\textit{ku\d lir k\=uvi\d la}\\
&&&puis \textit{\=a\d tum\=a\b ratu vall\=a\b n} &puis II 6.1d \textit{\=a\d tum\=a\b ru}\\
&&&\textit{aiy\=a\b r\b ru emmaiya\b n\=e}&\textit{vall\=a\b num aiy\=a\b ru}\\
&&&& \textit{u\d tai aiya\b n\=e}\\
\hline
20& 2204& Ma\b lap\=a\d ti& \textit{a\.n kaiy\=ara\b lal} & III 48.1a \textit{a\.n kai \=ar a\b lala\b n}\\
\hline
21& 2216& P\=accil\=accir\=amam\index{gnl}{Paccilacciramam@P\=accil\=accir\=amam} & \textit{ma\d ni va\d lar ka\d n\d tar\=o} & I 44.1d\\
&&&\textit{ma\.n kaiyai v\=a\d ta mayal}&\\
&&&\textit{ceyvat\=ovivar m\=a\d npatu}&\\
\hline
22& 2233& Ce\.nku\b n\b r\=ur\index{gnl}{cenkunrur@Ce\.nku\b n\b r\=ur}& \textit{avvi\b naikkivvi\b nai} & I 116.1a \textit{avvvi\b naikku}\\
&&&puis \textit{ceyvi\b nai t\=\i \d n\d t\=a} & \textit{ivvi\b nai \=am} (\textit{potu})\\
&&&\textit{tirun\=\i laka\d n\d tam}&puis I 116.1d \textit{ceyvi\b nai}\\
&&&& \textit{vantu emait t\=\i \d n\d tappe\b r\=a}\\
&&&& \textit{tirun\=\i laka\d n\d tam}\\
\hline
23& 2248& K\=a\d t\d tupa\d l\d li& \textit{v\=aru ma\b n\b nummulai}& III 29.1a\\
\hline
24& 2252& C\=o\b r\b ruttu\b rai & \textit{appar c\=o\b r\b ruttu\b rai}& I 28.1d\\
&&& \textit{ce\b n\b ra\d taiv\=om} &\\
\hline
25& 2270& Karuk\=av\=ur & \textit{antamillavar} & refrain de III 46 \\
&&&\textit{va\d n\d nam\=ara\b lal} &\textit{va\d n\d nam a\b lalum}\\
&&&\textit{va\d n\d nam}& \textit{a\b lal va\d n\d nam\=e}\\
\hline
26& 2272& Ava\d l-iva\d l-nall\=ur& \textit{tamparicu\d taiy\=ar} & III 82.1b \textit{tam parici\b n\=o\d tu}\\
\hline
27& 2305& Ku\d tam\=ukku& \textit{ku\d tam\=ukkai yuvantirunta} & refrain III 59 \textit{ku\d tam\=ukku} \\
&&&\textit{perum\=a\b nemmi\b rai} &\textit{i\d tam\=a ... irunt\=a\b n ava\b n}\\
&&&& \textit{em i\b raiy\=e}\\
\hline
28& 2311& I\d taimarut\=ur& \textit{\=o\d t\=ekala\b n} & I 32.1a puis I 32.1d\\
&&&puis \textit{i\d taimarut\=\i t\=o}& \\
\hline
29& 2322& \=Ava\d tutu\b rai\index{gnl}{Avatuturai@\=Ava\d tutu\b rai} & \textit{\=\i vato\b n\b ruma\b r\b ril\=e\b nu\b n\b na\d ti }& refrain de III 4?\\
&&&\textit{yallato\b n\b ra\b riy\=e\b n}&\textit{\=\i vatu o\b n\b ru emakku}\\
&&&& \textit{illaiy\=el}\\
&2323&"&n\=ala\d tiyi\b nm\=eliruc\=\i r& procédé\index{gnl}{procédé littéraire} de III 4\\
\hline
30& 2334& Turutti& \textit{varaittalaippacum po\b n} & II 98.1a\\
\hline
31& 2345& Tarumapuram\index{gnl}{Tarumapuram}& \textit{m\=atarma\d tappi\d ti} & I 136.1a\\
\hline
32 & 2354 et 2680& Na\d l\d l\=a\b ru\index{gnl}{Nallaru@Na\d l\d l\=a\b ru}& \textit{p\=okam\=artta p\=u\d n m\=ulaiy\=a\d l}& I 49.1a\\
\hline
33& 2380& Marukal\index{gnl}{Marukal}& \textit{u\d taiy\=a\b n\=e takum\=o} & II 18.1d \textit{u\d taiy\=ay} \\
&&&\textit{inta o\d l\d li\b laiy\=a\d l u\d nmelivu}& \textit{takum\=o iva\d l u\d l meliv\=e}\\
\hline
34& 2384& Ce\.nk\=a\d t\d ta\.nku\d ti\index{gnl}{Cenkattankuti@Ce\.nk\=a\d t\d ta\.nku\d ti}& \textit{a\.n kamum v\=etamum}& I 6.1a\\
\hline
35& 2395& Vi\b rku\d ti v\=\i ra\d t\d tam& \textit{p\=a\d tala\b n\=a\b nma\b rai} & I 105.1a\\
& 2396& "& \textit{a\d l\d la\b r ka\b la\b ni y\=ar\=ur}& I 105.3d\\
&&& \textit{a\d taiv\=om}&\\
\hline
36 & 2397& \=Ar\=ur\index{gnl}{Ar\=ur@\=Ar\=ur}& \textit{parukkaiy\=a\b nai}& II 101.1a\\
\hline
37& 2405& "& \textit{antam\=ayukak\=atiy\=am}& III 45.1a\\
&&&puis \textit{entai t\=a\b ne\b nai}&puis III 45.1d\\
&&&\textit{y\=e\b n\b ruko\d lu\.n kol}&\\
\hline
38& 2416& "&\textit{pava\b nam\=aycc\=o\d taiy\=ay}&II 79.1a\\
\hline
39& 2422& Pukal\=ur& \textit{ku\b rikalanticai}& I 2.1a\\
\hline
40& 2430& Ampar &\textit{pulkupo\b n\b ni\b ram}&II 103.1a\\
\hline
41& 2432 & Ka\d tav\=ur& \textit{ca\d taiyu\d taiy\=a\b n}& III 8.1a\\
\hline
42& 2440& V\=\i \b limi\b lalai\index{gnl}{Vilimilalai@V\=\i \b limi\b lalai} & \textit{araiy\=ar virik\=ova\d nav\=a\d tai}& I 35.1a\\
\hline
43& 2443& "&\textit{ca\d taiy\=ar pu\b nalu\d taiy\=a\b n}& I 11.1a\\
\hline
44& 2455&"& \textit{maimmarup\=u\.n ku\b lal}& I 4.1a\\
\hline
45& 2468&"& \textit{v\=acit\=\i rttaru\d lum}& I 92.1a \textit{v\=aci t\=\i rav\=e}\\
&&&& \textit{k\=acu nalkuv\=\i r}\\
\hline
46& 2485& Ma\b raikk\=a\d tu\index{gnl}{Maraikkatu@Ma\b raikk\=a\d tu}& \textit{caturam}& II 37.1a\\
\hline
47& 2494& V\=aym\=ur& \textit{ta\d liri\d lava\d lar}& II 111.1a\\
\hline
48& 2514& Ma\b raikk\=a\d tu\index{gnl}{Maraikkatu@Ma\b raikk\=a\d tu}& \textit{v\=eyu\b rut\=o\d li}& II 85.1a\\
\hline
49& 2561& \=Alav\=ay\index{gnl}{Maturai!Alavay@\=Alav\=ay}& \textit{ma\.nkaiyarkkaraci}& III 120.1a\\
\hline
50& 2565& "& \textit{n\=\i la m\=ami\d ta\b r\b ralav\=ay\=a\b n}& I 94.1a \textit{n\=\i la m\=ami\d ta\b r\b ru}\\
&&&&\textit{alav\=ayil\=a\b n}\\
\hline
51& 2602& "& \textit{ceyya\b n\=e tiruv\=alav\=ay}& III 51.1a\\
\hline
52& 2637a&"& \textit{k\=a\d t\d tum\=avuri}& III 47.1a \textit{k\=a\d t\d tu m\=a atu}\\
&&&&\textit{urittu}\\
\hline
53& 2637c&"& \textit{v\=eta v\=e\d lvi}& III 108.1a\\
\hline
54& 2658& "&\textit{m\=a\b ni\b n\=er vi\b liyin\=ay}& III 39.1a\\
\hline
55& 2662&"& \textit{n\=\i \b r\=e ma\b n\b numantiramum\=aki}& II 66.1a \textit{mantiram \=avatu} \\
&&& \textit{maruntum\=ayt t\=\i rppatu}&\textit{n\=\i \b ru}\\
\hline
56& 2682& "& \textit{ta\d lir i\d la va\d lar o\d li}& III 87.1a (Na\d l\d l\=a\b ru\index{gnl}{Nallaru@Na\d l\d l\=a\b ru})\\
\hline
57&2720-2743&"&&III 54 (\textit{potu})\\
\hline
& 2720&"& \textit{anta\d nar t\=evar \=a\b n} & III 54.1a \textit{v\=a\b lka anta\d nar} \\
&&&\textit{i\b na\.nka\d l v\=a\b lka} &\textit{v\=a\b navar \=a\b ni\b nam}\\
\hline
& 2721 &"&\textit{v\=\i \b l pu\b nal}& III 54.1b \textit{v\=\i \b lka ta\d n pu\b nal}\\
&&&puis \textit{ma\b n\b na\b nai v\=a\b lttiyatu} &puis III 54.1b \textit{v\=enta\b num oo\.nkuka}\\
\hline
&2722 &"& \textit{\=a\b lka t\=\i yatu} &III 54.1c \textit{\=a\b lka t\=\i yatu}\\
&&&puis \textit{ell\=am ara\b n peyar c\=u\b lka} &puis III 54.1cd \textit{ell\=am ara\b n} \\
&&&& \textit{n\=amam\=e c\=u\b lka}\\
\hline
&2723&"& \textit{vaiyakamum tuyar t\=\i rkav\=e}&III 54.1d\\% \textit{vaiyakamum tuyar}\\
%&&&& \textit{t\=\i rkav\=e}
\hline
&2724&"& \textit{ariya k\=a\d tciyar} &III 54.2a \textit{ariya k\=a\d tciyar\=ay}\\
\hline
&2725 &"&\textit{\=ayi\b num periy\=ar} &III 54.2cd\\ %\textit{\=ayi\b num periyar}
\hline
&2726 &"&\textit{\=ar a\b riv\=ar avar pe\b r\b riy\=e} &III 54.2d\\ %\textit{\=ar a\b riv\=ar avar pe\b r\b riy\=e}
\hline
&2727&"& \textit{venta c\=ampal virai} &III 54.3a\\ %\textit{venta c\=ampal virai}
\hline
&2728&"& \textit{tantaiyar t\=ay ilar} &III 54.3b \textit{tantaiy\=aro\d tu t\=ay ilar}\\
\hline
&2729&"& \textit{tammaiy\=e cintiy\=ar} & III 54.3bc \textit{tammaiy\=e cintiy\=a}\\
&&&& \textit{e\b luv\=ar}\\
\hline
&2730 &"&\textit{entaiy\=ar avar evvakaiy\=ar kol} &III 54.3d\\ %\textit{entaiy\=ar avar} \\
%&&&&\textit{evvakaiy\=ar koloo}
\hline
&2731&"&\textit{ \=a\d tp\=al avarkku aru\d lum} & III 54.4a\\ %\textit{\=a\d tp\=alavarkku aru\d lum}
\hline
&2733 &"&\textit{\=etukka\d l\=al} &III 54.5a \textit{\=etukka\d l\=alum}\\
\hline
&2734 &"&\textit{cu\d tar vi\d t\d tu u\d la\b n} &III 54.5b\\ %\textit{cu\d tarvi\d t\d tu u\d la\b n}
\hline
&2735&"& \textit{m\=atukkam n\=\i kkal} &III 54.5c \textit{m\=a tukkam n\=\i \.nkal} \\
&&&\textit{u\b ruv\=\i r ma\b nam pa\b r\b rum} &\textit{u\b ruv\=\i r ma\b nampa\b r\b ri}\\
\hline
&2736&"& \textit{c\=atukka\d l} &III 54.5d \textit{c\=atukka\d l}\\
&&&puis \textit{c\=armi\b n} &puis III 54.5d \textit{c\=armi\b n ka\d l\=e}\\
\hline
&2737 &"&\textit{\=a\d tum} &III 54.6a\\ %\textit{\=a\d tum}
\hline
&2738&"& \textit{ka\d tic\=ernta} &III 54.7a\\ %\textit{ka\d ti c\=ernta}
\hline
 &2739 &"&\textit{v\=eta mutalva\b n} &III 54.8a\\ %\textit{v\=etamutalva\b n}
\hline
&2740 &"&\textit{p\=ar \=a\b li va\d t\d tam} &III 54.9a\\ %\textit{p\=ar \=a\b liva\d t\d tam}
\hline
&2741 &"&\textit{m\=al\=ayava\b n} & III 54.10a\\ %\textit{m\=al \=ayava\b num}
\hline
&2742 &"&\textit{a\b r\b ru a\b n\b ri} &III 54.11a\\ %\textit{a\b r\b ru a\b n\b ri}
\hline
&2743 &"& \textit{p\=acurattai} &III 54.12a \textit{p\=acuram}\\
\hline
58& 2748& \=E\d takam\index{gnl}{Etakam@\=E\d takam}& \textit{va\b n\b ni}& III 32.1a \textit{va\b n\b niyum}\\
\hline
59& 2763& \=Alav\=ay\index{gnl}{Maturai!Alavay@\=Alav\=ay}& \textit{v\=\i \d tal\=alav\=ay}& III 52.1a\\
\hline
60& 2768& "& yamakam& procédé\index{gnl}{procédé littéraire} de III 115\\
\hline
61& 2779& C\=\i k\=a\b li\index{gnl}{Cikali@C\=\i k\=a\b li}& \textit{ma\d n\d ni\b nalla}& III 24.1a (Ka\b lumalam\index{gnl}{Kalumalam@Ka\b lumalam})\\
\hline
62& 2796& Mu\d l\d liv\=aykkarai\index{gnl}{Mullivaykkarai@Mu\d l\d liv\=aykkarai}& \textit{ko\d t\d tam}& III 6.1a\\
\hline
63& 2800& Na\d l\d l\=a\b ru\index{gnl}{Nallaru@Na\d l\d l\=a\b ru}& \textit{p\=a\d takamella\d ti}& I 7.1a\\
\hline
64& 2852& C\=\i k\=a\b li\index{gnl}{Cikali@C\=\i k\=a\b li}& \textit{u\b ru\b rumai c\=ervatu}& III 113.1a\\
&&&yamakam&procédé\index{gnl}{procédé littéraire} de III 113\\
\hline
65& 2863& Atikaiv\=\i ra\d t\d ta\b nam& \textit{ku\d n\d taikku\b ra\d tp\=utam}& I 46.1a\\
\hline
66& 2865& \=Amatt\=ur& \textit{ku\b n\b rav\=arcilai}& II 50.1a\\
\hline
67& 2868& A\d n\d n\=amalai& \textit{u\d n\d n\=amulaiy\=a\d l}& I 10.1a \textit{u\d n\d n\=amulai umaiy\=a\d l}\\
\hline
68& 2870&"& \textit{p\=uv\=ar malar}& I 69.1a\\
\hline
69& 2878&\=Ott\=ur\index{gnl}{Ottur@\=Ott\=ur}&\textit{kurumpaiy\=a\d n}&I 54.11a (envoi\index{gnl}{envoi})\\
&&& \textit{pa\b naiy\=\i \b num}&\\
\hline
70& 2893& Kacci& \textit{ma\b raiy\=a\b n}& II 12.1a \textit{ma\b raiy\=a\b nai}\\
\hline
71& 2897& " & yamakam& procédé\index{gnl}{procédé littéraire} de III 114\\
&&&irukkukku\b ra\d l& procédé\index{gnl}{procédé littéraire} de III 41\\
\hline
72& 2908& pr\`es d'\=Ala\.nk\=a\d tu\index{gnl}{Alankatu@\=Ala\.nk\=a\d tu}& \textit{tu\~ncavaruv\=ar}& I 45.1a\\
\hline
73& 2910& P\=ac\=ur& \textit{cintaiyi\d taiy\=ar}& II 60.1a\\
\hline
74& 2918& K\=a\d latti& \textit{v\=a\b navarka\d t\=a\b navar}& III 69.1a \textit{v\=a\b navarka\d l}\\
&&&&\textit{t\=a\b navarka\d l}\\
\hline
75& 2926&"& \textit{entaiy\=ari\d naiya\d tiye\b n}& III 36.1d \textit{entaiy\=ar i\d nai}\\
&&& \textit{ma\b natta}& \textit{a\d ti e\b n ma\b nattu u\d l\d lav\=e}\\
\hline
76& 2929& O\b r\b riy\=ur& \textit{vi\d taiyava\b n}& III 57.1a\\
\hline
77& 2986& Mayil\=app\=ur& \textit{ma\d t\d ti\d t\d ta} puis \textit{p\=otiy\=o}&II 47.1a puis II 47.1d\\
\hline
78& 3021& V\=a\b nmiy\=ur& vi\b n\=avurai& procédé\index{gnl}{procédé littéraire} de II 4\\
\hline
79& 3026& I\d taiccuram& \textit{iruntavi\d taiccuram} & refrain de I 78 \textit{i\d taiccuram}\\
&&&\textit{m\=evumivar}&\textit{m\=eviya ivar va\d nam e\b n\b n\=e}\\
&&& \textit{va\d n\d name\b n\b n\=e}&\\
\hline
80& 3029& Ka\b lukku\b n\b ram& \textit{k\=atalceyu\.n k\=oyil}& I 103.10a (envoi\index{gnl}{envoi})\\
&&& \textit{ka\b lukku\b n\b ru}&\\
\hline
81& 3031& Acci\b rup\=akkam& \textit{\=a\d tciko\d n\d t\=ar}& refrain de I 77\\
\hline
82& 3045& C\=\i k\=a\b li\index{gnl}{Cikali@C\=\i k\=a\b li} & \textit{va\d n\d t\=ar ku\b lalarivai}& I 9.1a (V\=e\d nupuram\index{gnl}{Venupuram@V\=e\d nupuram})\\
&&&puis \textit{vi\d n\d t\=a\.nkuvap\=ol}& puis I 9.1d \textit{vi\d n t\=a\.nkuva p\=olum}\\
&&& \textit{V\=e\d nupuram\index{gnl}{Venupuram@V\=e\d nupuram}}& \textit{miku V\=e\d nupuram\index{gnl}{Venupuram@V\=e\d nupuram} atuv\=e}\\
\hline
83& 3046&"& \textit{k\=a\b linakar c\=ermi\b n}& refrain de II 97 \textit{k\=a\b li c\=ermi\b n}\\ &&&&(K\=a\b li\index{gnl}{Kali@K\=a\b li})\\
\hline
84& 3143& Nall\=urpperuma\d nam& \textit{kall\=urpperuma\d nam}& III 125.1a\\
\hline
85& 3146& "& namacciv\=ayat& III 49 (\textit{potu}) namacciv\=aya\\
&&&tiruppatikam& patikam\\
\hline
\end{longtable}
\end{center}

\normalsize
La légende\index{gnl}{legende@légende} de Campantar\index{gnl}{Campantar} se construit tr\`es probablement d\`es le \textsc{xi}\up{e} si\`ecle. Elle circule, influence l'iconographie du poète\index{gnl}{poete@poète} et g\'en\`ere des textes de transmission diff\'erente (voir 5.1). C\=ekki\b l\=ar\index{gnl}{Cekkilar@C\=ekki\b l\=ar}, au milieu du \textsc{xii}\up{e} si\`ecle, ne l'invente pas. Il la d\'eveloppe et la fixe en s'appuyant solidement sur l'\oe uvre attribu\'ee \`a Campantar\index{gnl}{Campantar} --- qui est la même que celle que nous possédons aujourd'hui --- qu'il cite.
L'\'etude de ces citations nous \'eclaire sur la m\'ethode employ\'ee par C\=ekki\b l\=ar\index{gnl}{Cekkilar@C\=ekki\b l\=ar} pour assimiler le corpus\index{gnl}{corpus} du \textit{T\=ev\=aram}\index{gnl}{Tevaram@\textit{T\=ev\=aram}} et forger la légende\index{gnl}{legende@légende} de l'enfant\index{gnl}{enfant}-poète\index{gnl}{poete@poète}\footnote{C\=ekki\b l\=ar\index{gnl}{Cekkilar@C\=ekki\b l\=ar} effectue un travail de même type en \'ecrivant les légende\index{gnl}{legende@légende}s d'Appar\index{gnl}{Appar}, de Cuntarar\index{gnl}{Cuntarar} et de K\=araikk\=alammaiy\=ar\index{gnl}{Karaikkalammaiyar@K\=araikk\=alammaiy\=ar}. Concernant cette derni\`ere voir la postface de \textsc{F. Gros} dans \textsc{Karavelane} (1982).}.

Il semble que C\=ekki\b l\=ar\index{gnl}{Cekkilar@C\=ekki\b l\=ar} invente parfois un \'episode l\'egendaire en faisant une lecture litt\'erale d'un poème\index{gnl}{poeme@poème}. Par exemple, certains hymne\index{gnl}{hymne}s attribu\'es \`a Campantar\index{gnl}{Campantar}, influenc\'es par la litt\'erature du \textit{Ca\.nkam}\index{gnl}{Cankam@\textit{Ca\.nkam}} (voir 2.1.2), sont exploit\'es pour \'elaborer un miracle\index{gnl}{miracle}: les voix des narratrices qui expriment la souffrance de la s\'eparation amoureuse avec \'Siva\index{gnl}{Siva@\'Siva} dans le \textit{T\=ev\=aram}\index{gnl}{Tevaram@\textit{T\=ev\=aram}} sont donn\'ees aux femme\index{gnl}{femme}s que Campantar\index{gnl}{Campantar} secourt dans le \textit{pur\=a\d nam}. Ainsi, Campantar\index{gnl}{Campantar} chante l'hymne\index{gnl}{hymne} I 44 d\'edi\'e \`a P\=accil\=accir\=amam\index{gnl}{Paccilacciramam@P\=accil\=accir\=amam} pour sauver une jeune femme\index{gnl}{femme} sujette à des crises d'\'epilepsie (PP 2216). Il r\'ecite II 18 \`a Marukal\index{gnl}{Marukal} pour ressusciter l'amant d'une femme\index{gnl}{femme} mordu par un serpent\index{gnl}{serpent} (PP 2380) et II 47 \`a Mayil\=ap\=ur pour faire rena\^itre des cendre\index{gnl}{cendre}s une jeune femme\index{gnl}{femme} nomm\'ee P\=ump\=avai\index{gnl}{Pumpavai@P\=ump\=avai} (PP 2986). Ces trois chants\index{gnl}{chant}\index{gnl}{chant} engendrent trois miracle\index{gnl}{miracle}s dans le \textit{pur\=a\d nam}.
Sous l'influence de la po\'esie h\'ero\"ique du \textit{Ca\.nkam}\index{gnl}{Cankam@\textit{Ca\.nkam}} \'Siva\index{gnl}{Siva@\'Siva} est pr\'esent\'e comme un roi\index{gnl}{roi} vaillant et g\'en\'ereux \`a qui le poète\index{gnl}{poete@poète} Campantar\index{gnl}{Campantar}, \`a l'image\index{gnl}{image} des bardes auprès du roi\index{gnl}{roi}, demande divers bienfaits, dont des pièce\index{gnl}{piece@pièce}s, dans l'hymne\index{gnl}{hymne} I 92 c\'el\'ebrant V\=\i \b limi\b lalai\index{gnl}{Vilimilalai@V\=\i \b limi\b lalai}. Ce poème\index{gnl}{poeme@poème} est utilis\'e pour authentifier l'\'episode de la famine qui touche V\=\i \b limi\b lalai\index{gnl}{Vilimilalai@V\=\i \b limi\b lalai} et \`a laquelle Campantar\index{gnl}{Campantar} rem\'edie en obtenant des pièce\index{gnl}{piece@pièce}s d'or (PP 2468).
Parfois il est possible d'inférer la connaissance\index{gnl}{connaissance} par C\=ekki\b l\=ar\index{gnl}{Cekkilar@C\=ekki\b l\=ar} d'un hymne\index{gnl}{hymne} particulier sans qu'il y fasse r\'ef\'erence. Ainsi, il est possible que C\=ekki\b l\=ar\index{gnl}{Cekkilar@C\=ekki\b l\=ar} connaisse le poème\index{gnl}{poeme@poème} III 63 d\'edi\'e \`a Ce\.nk\=a\d t\d ta\.nku\d ti\index{gnl}{Cenkattankuti@Ce\.nk\=a\d t\d ta\.nku\d ti} mettant en sc\`ene un humble serviteur\index{gnl}{serviteur} (\textit{ci\b rutto\d n\d ta\b n}) qui envoie des messages \`a \'Siva\index{gnl}{Siva@\'Siva} par le biais de divers oiseau\index{gnl}{oiseau}x parce qu'il semble tirer de cet hymne\index{gnl}{hymne} la rencontre du \textit{n\=aya\b n\=ar}\index{gnl}{nayanmar@\textit{n\=aya\b nm\=ar}!\textit{n\=aya\b n\=ar}} Ci\b rutto\d n\d ta\b n avec Campantar\index{gnl}{Campantar} (PP 2367) et le nom de son fils C\=\i r\=a\d la\b n (PP 3676), \'epith\`ete de \'Siva\index{gnl}{Siva@\'Siva} dans III 63.8.

Cependant, notre \'etude interne des hymne\index{gnl}{hymne}s attribu\'es \`a Campantar\index{gnl}{Campantar} nous a montr\'e qu'il y a des probl\`emes d'interpolation\index{gnl}{interpolation} certains (voir 2.3.1 et 3.3). Ainsi, lorsque ces passages probl\'ematiques sont cit\'es par C\=ekki\b l\=ar\index{gnl}{Cekkilar@C\=ekki\b l\=ar} trois hypoth\`eses sont possibles: soit ces ajouts furent op\'er\'es, avant le \textit{Periyapur\=a\d nam}\index{gnl}{Periyapuranam@\textit{Periyapur\=a\d nam}}, au moment de la formation de la légende\index{gnl}{legende@légende} de Campantar\index{gnl}{Campantar}; soit ils furent int\'egr\'es au corpus\index{gnl}{corpus} pendant la r\'edaction du \textit{Periyapur\=a\d nam}\index{gnl}{Periyapuranam@\textit{Periyapur\=a\d nam}}, soit, encore, ils sont post\'erieurs au \textit{Periyapur\=a\d nam}\index{gnl}{Periyapuranam@\textit{Periyapur\=a\d nam}} qui aurait lui-m\^eme subi des interpolation\index{gnl}{interpolation}s.
Nous ne pouvons pas apporter d'\'el\'ements de r\'eponse dans le cadre de cette th\`ese sur le site de C\=\i k\=a\b li\index{gnl}{Cikali@C\=\i k\=a\b li}. Toutefois, nous envisageons un travail plus profond sur le \textit{pur\=a\d nam} de Campantar\index{gnl}{Campantar} qui permettra, peut-\^etre, de nous \'eclairer.
Pour l'instant, examinons certaines de ces citations qui renvoient souvent \`a des strophes contenant des allusions \og autobiographique\index{gnl}{autobiographique}s\fg\ (voir 2.3.1).
C\=ekki\b l\=ar\index{gnl}{Cekkilar@C\=ekki\b l\=ar} cite des d\'ebuts de strophes qui nous semblent \^etre des ajouts. Par exemple, PP 2485 donne les premiers mots du poème\index{gnl}{poeme@poème} II 37.1a d\'edi\'e \`a Ma\b raikk\=a\d tu\index{gnl}{Maraikkatu@Ma\b raikk\=a\d tu} dans le contexte de l'\'episode de la fermeture des portes du temple\index{gnl}{temple}. Or nous avons vu que cette strophe du \textit{T\=ev\=aram}\index{gnl}{Tevaram@\textit{T\=ev\=aram}} diff\'erait des autres par sa structure et son th\`eme au point de nous apparaître comme une interpolation dans l'hymne\index{gnl}{hymne}. Deux autres strophes du corpus\index{gnl}{corpus} pr\'esentent ce m\^eme cas de figure. PP 2742 mentionne le d\'ebut du onzi\`eme quatrain du poème\index{gnl}{poeme@poème} III 54, hymne sans site (\textit{potu}) et \`a douze\index{gnl}{douze} strophes. Notre \'etude de l'hymne\index{gnl}{hymne} a montr\'e que la r\'ef\'erence aux \^ole\index{gnl}{ole@\^ole}s qui remontent \`a contre-courant la Vaikai\index{gnl}{Vaikai} \`a Maturai\index{gnl}{Maturai} est mal venue \`a cet emplacement. Ensuite, PP 2852 cite la strophe III 113.1a et paraphrase l'envoi\index{gnl}{envoi} pour justifier le m\^eme \'episode. L'appartenance de ce quatrain \`a un hymne\index{gnl}{hymne} c\'el\'ebrant les douze\index{gnl}{douze} noms de C\=\i k\=a\b li\index{gnl}{Cikali@C\=\i k\=a\b li} nous commande de le lire avec pr\'ecaution.

Parfois C\=ekki\b l\=ar\index{gnl}{Cekkilar@C\=ekki\b l\=ar} donne les premiers mots d'un poème\index{gnl}{poeme@poème} pour appuyer un miracle\index{gnl}{miracle} ou un \'episode alors que les allusions \og autobiographique\index{gnl}{autobiographique}s\fg\ relatant le miracle\index{gnl}{miracle} ou l'\'episode en question se trouvent dans les envois\index{gnl}{envoi}. Ainsi, PP 2662 cite II 66.1a quand Campantar\index{gnl}{Campantar} gu\'erit la fièvre\index{gnl}{fievre@fièvre} du roi\index{gnl}{roi} \textit{p\=a\d n\d dya}\index{gnl}{pandya@\textit{p\=a\d n\d dya}} avec de la cendre\index{gnl}{cendre} sacr\'ee; PP 2682 mentionne III 87.1a lorsque les \^ole\index{gnl}{ole@\^ole}s sont jet\'ees dans le feu\index{gnl}{feu} et PP 2748 renvoie \`a III 32.1a au moment o\`u Campantar\index{gnl}{Campantar} stoppe \`a \=E\d takam\index{gnl}{Etakam@\=E\d takam} les \^ole\index{gnl}{ole@\^ole}s plac\'ees dans le fleuve. Pourquoi ces envois\index{gnl}{envoi} ne sont-ils pas mentionn\'es par C\=ekki\b l\=ar\index{gnl}{Cekkilar@C\=ekki\b l\=ar}? Faut-il remettre en cause sa \og formidable m\'emoire\fg\ ou plut\^ot sugg\'erer des ajouts au \textit{T\=ev\=aram} effectu\'es en fonction de la narration du \textit{Periyapur\=a\d nam}\index{gnl}{Periyapuranam@\textit{Periyapur\=a\d nam}}?

\og L'\'erudition aux multiples facettes\fg\ de C\=ekki\b l\=ar\index{gnl}{Cekkilar@C\=ekki\b l\=ar} comprend aussi une excellente connaissance\index{gnl}{connaissance} de la topographie. \`A partir des centaines de site chant\'es par la figure de Campantar\index{gnl}{Campantar} C\=ekki\b l\=ar\index{gnl}{Cekkilar@C\=ekki\b l\=ar} dresse une v\'eritable carte shiva\"ite\index{gnl}{shiva\"ite} du Pays Tamoul\index{gnl}{Pays Tamoul}.

\section{C\=ekki\b l\=ar le topographe}

C\=ekki\b l\=ar\index{gnl}{Cekkilar@C\=ekki\b l\=ar} mentionne les sites avec rigueur et m\'ethodologie. Son travail semble avoir \'et\'e de les classer par r\'egion g\'eographique et ensuite de mettre en place six pèlerinage\index{gnl}{pelerinage@pèlerinage}s, avec minutie\footnote{Le r\'ecent travail de Schmid (à paraître b) s'appuie sur la description d'un pèlerinage\index{gnl}{pelerinage@pèlerinage} de Campantar\index{gnl}{Campantar} dans le \textit{Periyapur\=a\d nam}\index{gnl}{Periyapuranam@\textit{Periyapur\=a\d nam}} pour rattacher l'hymne\index{gnl}{hymne} I 111 d\'edi\'e \`a Ka\d taimu\d ti au site abandonn\'e de Tirucce\b n\b namp\=u\d n\d ti plut\^ot qu'au site de K\=\i \b laiy\=ur comme il l'est actuellement.}, pour que Campantar\index{gnl}{Campantar} visite et c\'el\`ebre tous ces lieux (voir l'introduction de ce chapitre).

En associant systématiquement un miracle\index{gnl}{miracle} \`a un hymne\index{gnl}{hymne} et, ce faisant, \`a un site, C\=ekki\b l\=ar permet un ancrage profond des poème\index{gnl}{poeme@poème}s du \textit{T\=ev\=aram}\index{gnl}{Tevaram@\textit{T\=ev\=aram}} sur le sol tamoul. Il fixe aussi topographiquement des hymne\index{gnl}{hymne}s \`a caract\`ere g\'en\'eral (\textit{potu}). Par exemple, il fait r\'eciter le poème\index{gnl}{poeme@poème} aux cinq syllabes (III 22) \`a C\=\i k\=a\b li\index{gnl}{Cikali@C\=\i k\=a\b li} apr\`es l'initiation de Campantar\index{gnl}{Campantar} (PP 2164), le \textit{Tirun\=\i laka\d n\d tam} (I 116) \`a Ce\.nku\b n\b rur\index{gnl}{cenkunrur@Ce\.nku\b n\b r\=ur} pour soigner la fièvre\index{gnl}{fievre@fièvre} des dévot\index{gnl}{devot(e)@dévot(e)}s (PP 2233), le \textit{Tirupp\=acuram} (III 54) pour faire remonter les \^ole\index{gnl}{ole@\^ole}s \`a contre-courant (PP 2720-2743) et le \textit{Namacciv\=aya patikam} (III 49) quand Campantar\index{gnl}{Campantar} entre dans le feu\index{gnl}{feu} sacrificiel du mariage\index{gnl}{mariage} pour atteindre le monde de \'Siva\index{gnl}{Siva@\'Siva} (PP 3146).

Cependant des probl\`emes se posent à propos de l'identification des sites et de leur r\'ealit\'e géographique. Concentrons-nous sur le cas de C\=\i k\=a\b li\index{gnl}{Cikali@C\=\i k\=a\b li} aux douze\index{gnl}{douze} noms dans le cadre de notre \'etude\footnote{Un autre cas exceptionnel est le site d'\=Alav\=ay\index{gnl}{Maturai!Alavay@\=Alav\=ay} dont les dix hymne\index{gnl}{hymne}s attribu\'es \`a Campantar\index{gnl}{Campantar} pr\'esentent tous des allusions \og autobiographique\index{gnl}{autobiographique}s\fg\ douteuses (voir 2.3.1) et sont tous cit\'es dans le \textit{Periyapur\=a\d nam}\index{gnl}{Periyapuranam@\textit{Periyapur\=a\d nam}}. Parmi les interpolation\index{gnl}{interpolation}s sugg\'er\'ees dans le \textit{pur\=a\d nam} de Campantar\index{gnl}{Campantar} par \textsc{Nampi Arooran} (1977:19) cinq strophes se trouvent dans l'\'episode de Maturai\index{gnl}{Maturai} (PP 2603, 2613, 2614, 2632 et 2633). Nous envisageons un travail exclusif sur ce lieu.}. L'analyse interne des hymne\index{gnl}{hymne}s d\'edi\'es \`a la ville natale de Campantar\index{gnl}{Campantar} dans le chapitre 3 a permis de soulever de nombreuses questions quant \`a l'unit\'e des douze\index{gnl}{douze} toponymes. Un constat similaire se dresse \`a la lecture de l'hagiographie\index{gnl}{hagiographie}. Certains noms sont plus fr\'equents que d'autres: Pukali\index{gnl}{Pukali} avec cent dix-neuf occurences devance Ca\d npai\index{gnl}{Canpai@Ca\d npai} (quatre-vingt-quatorze), T\=o\d nipuram\index{gnl}{Tonipuram@T\=o\d nipuram} (soixante-deux), K\=a\b li\index{gnl}{Kali@K\=a\b li} ou C\=\i k\=a\b li (quarante-neuf), Cirapuram\index{gnl}{Cirapuram} (vingt-huit) et enfin Ka\b lumalam\index{gnl}{Kalumalam@Ka\b lumalam} (quatorze)\footnote{Rappelons que dans les envois\index{gnl}{envoi} nous avions plus ou moins cet ordre\index{gnl}{ordre}: Ka\b li avec cent cinquante-trois occurences devance Pukali\index{gnl}{Pukali} (quarante-quatre), Ka\b lumalam\index{gnl}{Kalumalam@Ka\b lumalam} (vingt-et-un), Ca\d npai\index{gnl}{Canpai@Ca\d npai} (seize) et Cirapuram\index{gnl}{Cirapuram} (sept).}. Parmi les cinq citations de poème\index{gnl}{poeme@poème}s c\'el\'ebrant C\=\i k\=a\b li\index{gnl}{Cikali@C\=\i k\=a\b li} dans le \textit{pur\=a\d nam} seuls quatre hymne\index{gnl}{hymne}s sont clairement associ\'es \`a un toponyme: Piramapuram\index{gnl}{Piramapuram} (I 1) dans PP 1974, K\=a\b li\index{gnl}{Kali@K\=a\b li} (I 24) dans PP 2005, Ka\b lumalam\index{gnl}{Kalumalam@Ka\b lumalam} (III 24) dans PP 2779 et V\=e\d nupuram\index{gnl}{Venupuram@V\=e\d nupuram} (I 9) dans PP 3045\footnote{III 113 (PP 2852) est d\'edi\'e aux douze\index{gnl}{douze} noms.}. De plus, T\=o\d nipuram\index{gnl}{Tonipuram@T\=o\d nipuram} conserve son statut particulier. Si, dans les hymne\index{gnl}{hymne}s, il s'agit d'un toponyme qui renvoie \`a la légende\index{gnl}{legende@légende} la plus mentionn\'ee, celle du déluge\index{gnl}{deluge@déluge}, dans le \textit{Periyapur\=a\d nam}\index{gnl}{Periyapuranam@\textit{Periyapur\=a\d nam}}, T\=o\d nipuram\index{gnl}{Tonipuram@T\=o\d nipuram} se distingue aussi. Il est r\'eserv\'e pour la description du \'Siva\index{gnl}{Siva@\'Siva} et du temple\index{gnl}{temple} de C\=\i k\=a\b li\index{gnl}{Cikali@C\=\i k\=a\b li} comme s'il d\'esignait sp\'ecifiquement le temple\index{gnl}{temple} plut\^ot que la localit\'e. Des expressions telles que \textit{pukalit tirutt\=o\d ni} (PP 1918), \og la barque\index{gnl}{barque}(-temple\index{gnl}{temple}) sacrée de Pukali\index{gnl}{Pukali}\fg, ou \textit{ca\d npaiyi\b n amar perum tirutt\=o\d ni n\=aya\b nar} (PP 3924), \og le seigneur de la grande et sacrée barque\index{gnl}{barque}(-temple\index{gnl}{temple}) qui demeure \`a Ca\d npai\index{gnl}{Canpai@Ca\d npai}\fg, confortent cette id\'ee et correspondent à la r\'ealit\'e historique refl\'et\'ee dans les inscriptions (voir CEC).
Ensuite, l'ordre\index{gnl}{ordre} de pr\'esentation des douze\index{gnl}{douze} toponymes observ\'e dans le \textit{T\=ev\=aram}\index{gnl}{Tevaram@\textit{T\=ev\=aram}} est le m\^eme dans l'hagiographie\index{gnl}{hagiographie} alors que les textes attribu\'es \`a Nampi \=A\d n\d tar Nampi\index{gnl}{Nampi \=A\d n\d t\=ar Nampi} ne suivent pas cet encha\^inement des noms qui est pourtant flagrant dans les poème\index{gnl}{poeme@poème}s \`a douze\index{gnl}{douze} strophes compos\'es selon des procédé\index{gnl}{procédé littéraire}s litt\'eraires complexes.
Enfin, bien que C\=ekki\b l\=ar\index{gnl}{Cekkilar@C\=ekki\b l\=ar} mentionne les noms de onze figures po\'etiques (PP 2174-2175), il ne cite qu'un seul hymne\index{gnl}{hymne} \`a douze\index{gnl}{douze} noms construit selon le procédé\index{gnl}{procédé littéraire} du \textit{yamakam} (PP 2852). Notons qu'il ne fait r\'ef\'erence ni au \textit{k\=om\=uttiri\index{gnl}{komuttiri@\textit{k\=om\=uttiri}} ant\=ati} (II 74) ni au \textit{va\b limo\b li tiruvir\=akam} (III 67). La maturit\'e des douze\index{gnl}{douze} légende\index{gnl}{legende@légende}s d\'ecrites dans III 67 nous avait conduit \`a sugg\'erer un ajout (3.3.1). L'absence de citation dans le \textit{pur\=a\d nam} permet maintenant de supposer un ajout post\'erieur au \textsc{xii}\up{e} si\`ecle.
Ainsi, l'\'etude succincte des douze\index{gnl}{douze} toponymes de C\=\i k\=a\b li\index{gnl}{Cikali@C\=\i k\=a\b li} dans le \textit{Periyapur\=a\d nam}\index{gnl}{Periyapuranam@\textit{Periyapur\=a\d nam}} confirme nos doutes à propos de cette unité qui nous apparaît toute artificielle (voir 3.3 et 5.3) et renforce l'hypoth\`ese selon laquelle des poème\index{gnl}{poeme@poème}s compos\'es selon des procédé\index{gnl}{procédé littéraire}s litt\'eraires constitueraient des ajouts (2.1.3).


Les citations et la pr\'esentation des sites \`a travers les pèlerinage\index{gnl}{pelerinage@pèlerinage}s de l'enfant\index{gnl}{enfant}-poète\index{gnl}{poete@poète} donn\'ees par C\=ekki\b l\=ar\index{gnl}{Cekkilar@C\=ekki\b l\=ar} refl\`etent l'\'etat du corpus\index{gnl}{corpus} du \textit{T\=ev\=aram}\index{gnl}{Tevaram@\textit{T\=ev\=aram}} au milieu du \textsc{xii}\up{e} si\`ecle, du moins pour ce qui concerne les hymne\index{gnl}{hymne}s attribu\'es \`a Campantar\index{gnl}{Campantar}, tr\`es proche de ce que nous avons actuellement. Le texte de C\=ekki\b l\=ar\index{gnl}{Cekkilar@C\=ekki\b l\=ar} est pr\'ecis et organis\'e. Cependant son recoupement avec les donn\'ees internes des poème\index{gnl}{poeme@poème}s intensifie, voire confirme, nos doutes à propos de passages que nous croyons dès lors interpol\'es. Ces ajouts et/ou possibles ajouts auraient \'et\'e effectu\'es avant, pendant et apr\`es le \textit{Periyapur\=a\d nam}\index{gnl}{Periyapuranam@\textit{Periyapur\=a\d nam}}. S'il est possible un v\'eritable travail d'\'edition critique du \textit{Periyapur\=a\d nam} s'impose.
\begin{center}
*
\end{center}
Les poème\index{gnl}{poeme@poème}s du \textit{T\=ev\=aram}\index{gnl}{Tevaram@\textit{T\=ev\=aram}} attribu\'es \`a Campantar\index{gnl}{Campantar} et d\'edi\'es \`a C\=\i k\=a\b li\index{gnl}{Cikali@C\=\i k\=a\b li}, datables du \textsc{vii}\up{e} au \textsc{ix}\up{e} si\`ecle, constituent la source principale de la premi\`ere partie de notre thèse. L'auteur et ses hymne\index{gnl}{hymne}s sont \`a leur tour c\'el\'ebr\'es dans quelques textes du \textit{Tirumu\b rai}\index{gnl}{Tirumurai@\textit{Tirumu\b rai}} qui fixent leurs légendes\index{gnl}{legende@légende} au \textsc{xii}\up{e} si\`ecle. L'\'etude des textes et des image\index{gnl}{image}s dans la deuxi\`eme partie nous permet de poser plusieurs hypoth\`eses:
\begin{enumerate}
\item l'attribution de douze toponymes au site de C\=\i k\=a\b li paraît être un développement postérieur à un corpus \og premier\fg\ du \textit{T\=ev\=aram} qui aurait eu lieu sous l'influence du \textit{Periyapur\=a\d nam}. Par conséquent, les hymnes aux douze noms attribués à Campantar dans le \textit{T\=ev\=aram} nous semblent être des interpolations.
\item L'analyse des miracles et l'observation de variantes narratives dans les textes attribués à Nampi dans le \textit{Tirumu\b rai} \textsc{xi} montrent, à notre avis, que ce poète n'est pas l'unique auteur de ces textes qui rendent compte, pour certains, d'une transmission de la \og Légende dorée\fg\ de Campantar différente de celle du \textit{Periyapur\=a\d nam}.
\item L'étude iconographique de Campantar nous permet de suggérer que la légende\index{gnl}{legende@légende} de l'enfant-poète ne se d\'eveloppe qu'\`a partir de l'extr\^eme fin du \textsc{x}\up{e} si\`ecle pour se cristalliser au \textsc{xii}\up{e} si\`ecle dans le \textit{Periyapur\=a\d nam}\index{gnl}{Periyapuranam@\textit{Periyapur\=a\d nam}} de C\=ekki\b l\=ar\index{gnl}{Cekkilar@C\=ekki\b l\=ar} dans laquelle l'hagiographe int\`egre le corpus\index{gnl}{corpus} du \textit{T\=ev\=aram}\index{gnl}{Tevaram@\textit{T\=ev\=aram}} qu'il a pu, selon nous, compiler.
\end{enumerate}
Parall\`element \`a ce contexte de rédaction du \textit{Periyapur\=a\d nam} l'importance du site de C\=\i k\=a\b li\index{gnl}{Cikali@C\=\i k\=a\b li} se manifeste concrètement dans l'espace, comme en t\'emoignent les inscriptions qui couvrent les murs du temple\index{gnl}{temple} \`a partir du \textsc{xii}\up{e} si\`ecle.


\part{Histoire}

Du \textsc{vii}\up{e} au \textsc{xii}\up{e} si\`ecle, les textes litt\'eraires subliment le temple\index{gnl}{temple} de C\=\i k\=a\b li\index{gnl}{Cikali@C\=\i k\=a\b li} et b\^atissent la renomm\'ee l\'egendaire de son enfant\index{gnl}{enfant}-poète\index{gnl}{poete@poète} Campantar\index{gnl}{Campantar}, aux grands exploits miraculeux, dans tout le Pays Tamoul\index{gnl}{Pays Tamoul}. Cependant, nous constatons des dissonnances. Les doutes concernant l'appartenance de certains passages au corpus premier s'installent. Par exemple, l'unit\'e des douze\index{gnl}{douze} noms ne peut être originelle. Les trois toponymes qui ont une importance constante dans les textes du \textit{Tirumu\b rai}--- T\=o\d nipuram\index{gnl}{Tonipuram@T\=o\d nipuram}, K\=a\b li\index{gnl}{Kali@K\=a\b li} et Ka\b lumalam\index{gnl}{Kalumalam@Ka\b lumalam} --- sont précisément ceux qui correspondent \`a une r\'ealit\'e g\'eographique, ceux qui apparaissent dans les inscriptions.



Le temple\index{gnl}{temple} de C\=\i k\=a\b li\index{gnl}{Cikali@C\=\i k\=a\b li} est au centre de la ville. Il est ouvert \`a l'est (voir les fig. 6.1 et 8.3)\footnote{Pour plus de précision voir 8.3 et \textsc{Veluppillai} (2003a: 25-32).}. Il se caract\'erise par l'embo\^itement de trois enceintes.
Dans la premi\`ere, en partant du centre, se trouve le temple\index{gnl}{temple} de \'Siva\index{gnl}{Siva@\'Siva} [A].
Dans la seconde, dans le sens de la circumambulation, se situent
le bureau du \textit{devasth\=anam} [3], le \textit{ma\d n\d dapa} de la balan\c coire [4] et la cuisine [5], dans l'angle sud-est;
la chapelle des huit Bhairava\index{gnl}{Bhairava} au sud [7];
l'ancienne \'etable pour l'\'el\'ephant [8] et des petites chapelles dont deux sont d\'edi\'ees \`a Ga\d ne\'sa\index{gnl}{Ganesa@Ga\d ne\'sa} ([9] et [10]), une \`a Skanda\index{gnl}{Skanda} [11], une \`a un \textit{li\.nga}\index{gnl}{linga@\textit{li\.nga}} [12] et une au \textit{n\=aya\b n\=ar}\index{gnl}{nayanmar@\textit{n\=aya\b nm\=ar}!\textit{n\=aya\b n\=ar}} Ka\d nan\=atar\index{gnl}{Kananatar@Ka\d nan\=atar} [13], dans l'angle sud-ouest;
la chapelle de la d\'eesse\index{gnl}{deesse@déesse} [C] et une petite chapelle pour Skanda\index{gnl}{Skanda} [15], dans l'angle nord-ouest;
et le bassin dans l'angle nord-est [D].
C'est dans la troisi\`eme enceinte, au nord-ouest que s'\'el\`eve la chapelle de Campantar\index{gnl}{Campantar} [B]. Les jardins qui semblent occuper la moiti\'e de la superficie totale du temple\index{gnl}{temple} sont inclus dans cette même troisi\`eme enceinte où s'y trouve aussi une \'etable avec douze\index{gnl}{douze} vaches [1].

Les trois sanctuaires majeurs sont ceux de \'Siva\index{gnl}{Siva@\'Siva}, de la d\'eesse\index{gnl}{deesse@déesse} et de Campantar\index{gnl}{Campantar} localisés respectivement, donc, dans la première, la deuxième et la troisième enceinte.
Dans le temple\index{gnl}{temple} de \'Siva\index{gnl}{Siva@\'Siva}, deux manifestations viennent suppl\'eer le \textit{li\.nga}\index{gnl}{linga@\textit{li\.nga}} Brahm\=apure\'svara abrit\'e dans la cella. En effet, un b\^atiment \`a \'etages, accol\'e au mur ouest du corps principal qui contient la cella (voir fig. 8.2), est habit\'e au premier \'etage par \'Siva\index{gnl}{Siva@\'Siva} T\=o\d niyappar\index{gnl}{Toniyappar@T\=o\d niyappar} \og le P\`ere au radeau\index{gnl}{radeau}\fg\ et P\=arvat\=\i\index{gnl}{Parvati@P\=arvat\=\i}\ Periyan\=acciy\=ar \og la grande Dame\fg. Il est aussi fréquent que les fidèles appellent le couple divin Ammaiyappa\b n, \og mère et père\fg. Au second \'etage, se tient debout \'Siva\index{gnl}{Siva@\'Siva} Ca\d t\d tain\=atar\index{gnl}{Cattainatar@Ca\d t\d tain\=atar} \og le Seigneur \`a la chemise\fg, une forme de Bhairava\index{gnl}{Bhairava} (voir la conclusion).
La d\'eesse\index{gnl}{deesse@déesse} principale du temple\index{gnl}{temple} se nomme Tirunilain\=ayaki \og la Dame du site\fg.
Campantar\index{gnl}{Campantar} possède sa propre chapelle o\`u se dresse une biblioth\`eque \`a sa gauche, fermée actuellement.

Cette organisation est le fruit de plusieurs si\`ecles de constructions et de r\'enovations commandit\'ees au niveau local\index{gnl}{local}.

Notre derni\`ere partie est consacr\'ee aux donn\'ees archéologiques, constituées essentiellement des inscriptions grav\'ees sur le temple\index{gnl}{temple} de C\=\i k\=a\b li\index{gnl}{Cikali@C\=\i k\=a\b li}. \`A partir du \textsc{xii}\up{e} si\`ecle, ces textes \'epigraphiques t\'emoignent d'un site en activit\'e certes, mais dont le rayonnement\index{gnl}{rayonnement} est local\index{gnl}{local} et limit\'e par rapport \`a ce qui serait attendu d'un lieu sacr\'e dont nous avons constaté qu'il fut tant c\'el\'ebr\'e dans les textes littéraires.

La pr\'esentation du corpus\index{gnl}{corpus} \'epigraphique inédit de C\=\i k\=a\b li\index{gnl}{Cikali@C\=\i k\=a\b li} (chapitre 7) forme la base de l'analyse historique du site que nous esquissons dans notre chapitre 8.


 \begin{figure}
  \centering
  \includegraphics[width=14cm]{docthese/Copiedeplan.jpg}
  \caption{Plan approximatif du temple de C\=\i k\=a\b li.}
  \end{figure}

\chapter{Le corpus \'epigraphique de C\=\i k\=a\b li}

%\epigraphhead[300]{
\epigraph{Unless we read the full text of inscriptions, how can we perceive their whisperings, or have a dialogue with this pristine source material?}{Noboru \textsc{Karashima} (*2004 [2001]: 58),
 \textit{Whispering of Inscriptions}.}

%\newpage
%\section*{Pr\'esentation du corpus\index{gnl}{corpus}}

Les inscriptions du temple\index{gnl}{temple} de C\=\i k\=a\b li\index{gnl}{Cikali@C\=\i k\=a\b li} ont \'et\'e l'objet de l'attention des \'epigraphistes il y a plus d'un si\`ecle. Trois d'entre elles ont \'et\'e relev\'ees en 1896 (ARE 1896 123 \`a 125). Puis en 1918, lors d'une mission plus longue dans la r\'egion, quarante-deux textes furent copi\'es (ARE 360 \`a 401) mais seuls sept ont b\'en\'efici\'e d'une publication (SII 12 210, 211, 252 et 253; SII 5 988, 989 et 990). \textsc{Mahalingam} (1992: 547-554) reprend les r\'esum\'es des ARE et des SII de trente-deux inscriptions pour lesquelles, souvent, il essaie d'\'etablir une datation pr\'ecise.

Le corpus\index{gnl}{corpus} \'epigraphique de C\=\i k\=a\b li\index{gnl}{Cikali@C\=\i k\=a\b li} pr\'esent\'e ici est le fruit de trois s\'ejours de recherche en Inde du Sud\footnote{Les bourses EFEO (2004, 2005 et 2006) et Aires culturelles (2005) ont favoris\'e un travail de terrain de pr\`es d'un an et demi au total. Lors de ces s\'ejours, plusieurs d\'eplacements \`a Mysore et C\=\i k\=a\b li\index{gnl}{Cikali@C\=\i k\=a\b li} nous ont permis de recopier \`a la main les transcriptions de l'ASI, de consulter les estampages d'un texte contenant un éloge\index{gnl}{eloge@éloge} royal, de lire les inscriptions accessibles \textit{in situ} avec G. \textsc{Vijayavenugopal} (\'epigraphiste du centre EFEO de Putucc\=eri\index{gnl}{Putucc\=eri}) et de photographier les \'epigraphes lisibles avec l'aide de \textsc{N. Ramaswamy} et de \textsc{G. Ravindran}.}. Il rassemble cinquante-cinq \'epigraphes: trente-sept inscriptions\footnote{Le texte de ARE 1918 361 qui a été relev\'e comme appartenant \`a C\=\i k\=a\b li\index{gnl}{Cikali@C\=\i k\=a\b li} n'est pas donn\'e. Ce texte est un ordre\index{gnl}{ordre royal} royal qui s'adresse aux employ\'es du temple\index{gnl}{temple} de K\=a\d latti. Il ne mentionne \`a aucun moment le temple\index{gnl}{temple} de T\=o\d nipuram\index{gnl}{Tonipuram@T\=o\d nipuram}. De plus, il ne figure ni sur les murs sud du temple\index{gnl}{temple} principal (ARE), ni sur ceux du \textit{ma\d n\d dapa,} ni ailleurs. Le classement de l'ASI nous semble erron\'e dans ce cas.} que nous avons class\'ees en fonctions d'une chronologie probable et sur chaque monument, ainsi que dix-huit fragments. Se succ\`edent ainsi les textes du temple\index{gnl}{temple} de \'Siva\index{gnl}{Siva@\'Siva} (\textit{ma\d n\d dapa}, enceinte et galeries int\'erieures), ceux de la chapelle de Campantar\index{gnl}{Campantar} (temple\index{gnl}{temple} principal, \textit{ma\d n\d dapa} et enceinte) et les inscriptions fragmentaires. Remarquons que la chapelle de la déesse ne présente pas d'inscription, ce qui souligne dès l'abord son caractère récent.

Chaque texte poss\`ede un num\'ero CEC (pour \og corpus\index{gnl}{corpus} \'epigraphique de C\=\i k\=a\b li\index{gnl}{Cikali@C\=\i k\=a\b li}\fg). \`A l'exception des fragments, la pr\'esentation est organisée en trois parties g\'en\'eralement comme suit:
\begin{enumerate}
\item les remarques contiennent les r\'ef\'erences (ARE, SII, etc.), la datation, la description et un r\'esum\'e succinct du texte épigraphique.
\item le texte m\^eme avec des notes d'\'edition.
\item une traduction annot\'ee ou, \`a d\'efaut, lorsque le texte est lacunaire, nous proposons un r\'esum\'e pr\'ecis.
\end{enumerate}

%\section*{Conventions}
La translitt\'eration que nous avons adopt\'ee est celle du \textit{Tamil Lexicon}. Sur la pierre, les textes \'epigraphiques tamouls pr\'esent\'es ici ne connaissent, en g\'en\'eral, ni espace ni \textit{pu\d l\d li}. Pour faciliter la lecture, des espaces sont plac\'es entre les mots en l'absence totale de liaison et entre les liaisons consonantiques (ex.: \textit{tirukka\b lumalattuttirutto\d nipuram} devient \textit{tirukka\b lumalattut tirutto\d nipuram}). Nous considérons que les préfixes de bonne augure \textit{tiru} et \textit{\'sr\=\i}\ appartiennent aux noms qui les suivent dans le texte. Ainsi, nous ne les séparons pas des noms qui les suivent dans notre édition. La formule \textit{svasti \'sr\=\i}\ est un ensemble à part comme vient le conformer la ponctuation des textes épigraphiques (CEC 2, 3, 6, etc.). Les liaisons vocaliques sont conserv\'ees pour rester fid\`ele aux graph\`emes utilis\'es par le lapicide. Les \textit{pu\d l\d li} qui marquent une consonne non suivie de voyelle sont suppl\'e\'es selon l'interpr\'etation (ex.: l'absolutif \textit{ko\d n\d tu} sera choisi \`a la place de \textit{ko\d na\d tu}, d\'epourvu de sens). Le sens et l'interpr\'etation d\'eterminent le choix entre le \textit{a} long et le \textit{r}/\textit{ra} qui sont graphiquement identiques (ex.: la lecture pr\'ef\'erera \textit{k\=a\d ni} \`a \textit{kar\d ni} ou \textit{kara\d ni}).

Les textes utilisent un certain nombre d'abr\'eviations, surtout dans les mesures de terrain. Seules celles identifi\'ees ont un \'equivalent en lettre capitale. Les autres sont not\'ees uniform\'ement \og A\fg. Ainsi:
 \begin{verse}
 N remplace l'abr\'eviation pour \textit{nilam}, \og terre\index{gnl}{terre}\fg. Nous connaissons deux abr\'eviations pour ce terme: la premi\`ere ressemble \`a l'\textit{ak\d sara} \textit{ru} dont la boucle se prolonge en un trait horizontal vers la droite, et la seconde est semblable au \textit{n\=\i} dont la boucle est suivie d'une autre boucle, plus petite, avant de s'achever en trait horizontal.\\
 K pour \textit{k\=a\d ni}, \og droit, possession\fg, ressemble au \textit{ma} de l'\'ecriture grantha.\\
 V pour \textit{v\=eli} (une mesure de terre\index{gnl}{terre}) est not\'ee par \textit{li}.\\
 M pour \textit{m\=a} (une mesure de terre\index{gnl}{terre}) est marqu\'ee par \textit{pa}.\\
 Q pour \textit{muntiri} (une mesure de terre\index{gnl}{terre}). Cette abr\'eviation ressemble au \textit{ta} dont la partie haute gauche est not\'ee d'un petit trait vertical\footnote{Mentionnons que \textit{v\=eli}$>$\textit{m\=a}$>$\textit{muntiri}. La mesure \textit{m\=a} ou \textit{m\=acci\b n\b nam} correspond \`a 1/20 \textit{v\=eli} et \`a 100 \textit{ku\b li} (\textit{TL} et \textsc{Subbarayalu} *2001c [?]: 35). Cette \'equivalence est v\'erifi\'ee dans CEC 2. En effet, l.~26-30, la somme de 13 \textit{m\=a} des trois terres\index{gnl}{terre} d\'eduites (2 + 3 + 8 = 13) vaut, l.~30-1, un demi \textit{v\=eli} et 3 \textit{m\=a}, c'est-\`a-dire (20/2) + 3 \textit{m\=a} car 1 \textit{v\=eli} vaut 20 \textit{m\=a}.}.\\
 H pour \textit{mukk\=a\d ni} (une mesure de terre\index{gnl}{terre}) est semblable au \textit{ka} souscrit d'un \textit{ma} de l'\'ecriture grantha.\\
 C pour \textit{ka\d n\d n\=a\b ru}, \og canalicule\fg, est not\'ee par \textit{k\=u}.\\
 P pour \textit{nellu}, \og riz\index{gnl}{riz} non d\'ecortiqu\'e\fg, ressemble au \textit{ja} de l'\'ecriture grantha sans la boucle finale.\\
 \=A pour \textit{y\=a\d n\d tu}, \og ann\'ee\fg, est figur\'ee par un \textit{ru} dont la boucle englobe le chiffre qui le pr\'ec\`ede avant de se refermer.\\
 \=M pour \textit{m\=atam}, \og mois\fg, est semblable au \textit{m\=\i} dont la boucle est suivie d'une autre boucle, plus petite, avant de s'achever en trait horizontal.\\
 E pour \textit{ki\b l}, \og \`a l'est; en dessous\fg, est marqu\'ee par un \textit{k\=\i} dont la boucle est suivie d'une autre boucle, plus petite, avant de s'achever en trait horizontal.\\
 Z pour \textit{a\b ru} de \textit{ki\b la\b ru} dont le sens est inconnu. Cette abr\'eviation ressemble \`a la `corne' (\textit{kompu}, qui forme les voyelles \textit{e} et \textit{o} br\`eves) suivie d'un \textit{pa}.\\
 \end{verse}

Des signes ponctuent parfois le texte. \og U\fg\ rend compte du signe qui ressemble \`a la voyelle initiale \textit{u}. Plac\'e au d\'ebut (CEC 6), \`a la fin (CEC 8, 9, 11 et 17) ou au milieu du texte pour indiquer un changement de phrase (CEC 8, 19, 34), il est tr\`es vraisemblablement un signe de ponctuation \`a valeur propitiatoire. Deux ou trois \textit{da\d n\d da} sont souvent pr\'esents en d\'ebut ou en fin d'inscription. Ils sont maintenus tels quels.

Les autres conventions sont les suivantes:
\begin{verse}
\textbf{abc} marquent les \textit{ak\d sara} en \'ecriture grantha pour les mots d'origine\index{gnl}{origine} sanskrite.\\
(abc) contiennent les \'el\'ements graphiques qui sont difficilement lisibles sur la pierre.\\
{[abc] contiennent les graph\`emes illisibles et restaur\'es par conjecture.}\\
{[abc*] contiennent les \'el\'ements graphiques supplé\'es par conjecture parce que nécessaires à la lecture mais n'ayant jamais \'et\'e grav\'es.}\\
Chaque point marque un \textit{ak\d sara} manquant. Les points de suspension, plus espac\'es (\dots), sont utilis\'es quand la quantit\'e manquante est inconnue.\\
\end{verse}

%
%\part{Temple de T\=o\d nipuram\index{gnl}{Tonipuram@T\=o\d nipuram}}
%\thispagestyle{empty}
\newpage
\section{Temple de T\=o\d nipuram}
\begin{figure}[!h]
  \centering
  \includegraphics[height=7cm]{docthese/photoCIIKAALI/sivatpl10.jpg}
  \caption{Face sud du temple de \'Siva, vue de l'intérieur de l'enceinte, C\=\i k\=a\b li (cliché U. \textsc{Veluppillai}, 2006).}
  \end{figure}
\section*{A. Ma\d n\d dapa}

\section*{CEC 1}
\subsection*{CEC 1.1 Remarques}

Selon l'ARE, CEC 1 se trouve sur le mur sud du temple\index{gnl}{temple} principal de \'Siva\index{gnl}{Siva@\'Siva} et date de la sixi\`eme ann\'ee de r\`egne du roi\index{gnl}{roi} \textit{c\=o\b la}\index{gnl}{cola@\textit{c\=o\b la}} Tribhuvanacakravartin V\=\i rar\=ajendradeva.
\textsc{Mahalingam} (1992: 549, Tj. 2408) identifie ce roi\index{gnl}{roi} comme Kulottu\. nga III et pr\'ecise ainsi l'ann\'ee \textbf{1184}.

Le texte enregistre une donation de terres\index{gnl}{terre} pour approvisionner quotidiennement en huile les deux lampe\index{gnl}{lampe}s offertes au temple\index{gnl}{temple} de \'Siva\index{gnl}{Siva@\'Siva} par un natif de Pa\b laiya\b n\=urn\=a\d tu dans le Jeya\. nko\d n\d taco\b lama\d n\d talam.

Situ\'ee, plus exactement, sur deux portions du mur sud du \textit{ma\d n\d dapa} s\'epar\'ees par un pilastre, cette inscription est inédite; de plus, seul le texte d'une portion du mur a \'et\'e relev\'e par l'ARE et ce, de mani\`ere incompl\`ete (ARE 1918 363). Malgr\'e les nombreuses couches de peinture, il nous a \'et\'e possible de compl\'eter largement la premi\`ere partie de l'épigraphe (l.~1 \`a 23) et de d\'echiffrer la seconde qui est tr\`es probablement sa suite étant donné son contenu, son emplacement et sa pal\'eographie (l.~24 \`a 51). En effet, dans la seconde partie, sur la seconde portion du mur, sont pr\'ecis\'ees les diff\'erentes terres\index{gnl}{terre} acquises pour \^etre donn\'ees et la formule de protection finale. Ce texte est \'edit\'e sur l'examen de la transcription de l'ASI, de photographies (G. \textsc{Ravindran}, EFEO) et de la lecture \textit{in situ} avec G. \textsc{Vijayavenugopal}.

\subsection*{CEC 1.2 Texte}
\begin{enumerate}
	\item \textbf{svasti \'sr\=\i} tiripu[vanaca]kkara
	\item varttika(\d l\index{cec}{Tribhuvanacakravarti} \textbf{\'sr\=\i}virar\=a)[jendra]te\index{cec}{Virarajendre@V\=\i rar\=ajendra}
	\item varkku\footnote{Consid\'erant le nombre d'\textit{ak\d sara} manquants cette conjecture du titre royal \'etablie par l'ARE, suivie par \textsc{Mahalingam}, semble acceptable. De plus, l'attribution de ce titre \`a Kulottu\. nga III para\^it irr\'efutable, \textsc{Nilakanta Sastri} (*2000 [1955]: 397).} y\=a\d n\d tu [6...]\footnote{\`A d\'efaut de pouvoir v\'erifier l'ann\'ee de r\`egne \textit{in situ}, la le\c con de la transcription est adopt\'ee.} ir\=a
	\item c\=atir\=acava\d lan\=a\d t\d tut\index{cec}{Rajadhirajavala@R\=aj\=adhir\=ajava\d lan\=a\d tu} tirukka\b lu
	\item malan\=a\d t\d tu\index{cec}{Tirukka\b lumalan\=a\d tu} \textbf{bra}[ma]teya\index{cec}{brahmadeya@\textit{brahmadeya}}
	\item m tirukka\b lumattu\index{cec}{Tirukkalumalam@Tirukka\b lumalam} u
	\item \d taiy\=ar tirutto\d nipuramu
	\item \d taiya n\=ayann\=arkku\index{cec}{Tiruttoni@Tirutt\=o\d nipuramu\d taiya n\=aya\b n\=ar, \'Siva} \textbf{je}ya\.nko
	\item \d n\d taco\b lama\d n\d talalattu\index{cec}{Jayankontacolamandala@Jaya\.nko\d n\d tac\=o\b lama\d n\d dalam} melma
	\item laippa\b laiya\b n\=urn\=a\d t\d tup\index{cec}{Melmalai@Melmalaippa\b laiya\b n\=urn\=a\d tu} pa\b lai
	\item yan\=uru\d taiy\=a\b n\index{cec}{palaiuanur@Pa\b laiyan\=uru\d taiy\=a\b n} vetavanamu\d taiy\=a\index{cec}{vetavanam@V\=etava\b namu\d taiy\=a\b n}
	\item \b n karu\d n\=akaratevan\=ana v\=a\d n\=atir\=a
	\item ya\b n\index{cec}{Karu\d n\=akaratevan}\index{cec}{V\=a\d n\=atir\=aya\b n} i\b n\b n\=ayan\=ar tirumunpu vaitta vi
	\item \d lakku ira\d n\d tukku n\=a\d l o\b n\b rukku .. e
	\item \d n\d nai uriy\=aka ve\d n\d tum e\d n\d nai
	\item kku ivar ko\d n[\d tu vi\d t]\d ta \textbf{bra}mmateya\index{cec}{brahmadeya@\textit{brahmadeya}}
	\item m tirukka\b lumalattu\index{cec}{Tirukkalumalam@Tirukka\b lumalam} talaicca\.nk\=a\d t\d tu\index{cec}{Talaiccankatu@Talaicca\.nk\=a\d tu} vati\index{cec}{vati@\textit{vati}}
	\item kkuk ki\b lakkut tillaivi\d ta\.nka\index{cec}{Tillaivitankan@Tillaivi\d ta\.nka\b n} v\=aykk\=a
	\item {[lukku]t\index{cec}{vaykkal@\textit{v\=aykk\=al} canal} te\b rku muta\b rka\d n\d n\=a\b r\b ru muta\b r}
	\item catirattu.......y\=ana
	\item ...........ko\d n\d tu
	\item {[tiru\~n\=a\b nacampanta\b n]\footnote{Conjecture fond\'ee sur la transcription.}}
	\item ...
	\item k\=a.mutiri......
	\item tikkuk ki\b lakkut (tiru)....
	\item ka v\=aykk\=alukku\index{cec}{vaykkal@\textit{v\=aykk\=al} canal}....
	\item \d n\d t\=a\.n ka\d n\d n\=a\b r\b ru muta....
	\item tirattuccakkaravartti vi\d l\=a
	\item kame\b n\b ru per k\=uva pa\d t
	\item \d ta nilattu m\=a.lattu..to
	\item \d nipuramu\d taiy\=a\b n\index{cec}{Tonipuramutaiyan@T\=o\d nipuramu\d taiy\=a\b n} tiru.\b la\d lai.yu\d tai\footnote{La lecture \textit{mu} pour \textit{\b la} est tout aussi possible.}
	\item y\=a\b n pakkal ko\d n\d ta nilam ira\d n\d tu m\=a
	\item k\=a\d ni araikk\=a\d nik ki\b laraiye ira\d n\d tu m\=avu
	\item m v\=acciya\b n\index{cec}{Vacciyan@V\=acciya\b n} pira\d laiyavi\d ta\.nkan\index{cec}{Piralaiya@Pira\d laiyavi\d ta\.nka\b n} tirutto\d nipu
	\item ramu\d taiy\=an\index{cec}{Tonipuramutaiyan@T\=o\d nipuramu\d taiy\=a\b n} pakkal\index{cec}{pakkal@\textit{pakkal} auprès de} ko\d n\d ta nilam\index{cec}{nilam@\textit{nilam} terre} k\=a(\d niyu)\index{cec}{kani@\textit{k\=a\d ni} droit, propriété}
	\item m...k\=a\d ni\index{cec}{kani@\textit{k\=a\d ni} droit, propriété} vi..ka..ttil..n\=a
	\item yakan pakkal\index{cec}{pakkal@\textit{pakkal} auprès de} ko\d n\d ta nilam\index{cec}{nilam@\textit{nilam} terre} k\=a\d niyum\index{cec}{kani@\textit{k\=a\d ni} droit, propriété} ka
	\item vu\d niyan\index{cec}{kavuniyan@\textit{kavu\d niya\b n gotra}} tirun\=a(vukkarai)......m pir\=an
	\item pakkal\index{cec}{pakkal@\textit{pakkal} auprès de} ko\d n\d ta nilam\index{cec}{nilam@\textit{nilam} terre} o.....\=aka..pa
	\item \d t\d ta..\=aru m\=a muk\=a\d nik ki\b laraiye i
	\item .....vum ko\d n\d tu teva(r tirun\=ayaka ti)
	\item \dots
	\item \dots
	\item \dots
	\item ..........\'sr\=\i pa\d n\d t\=aratte\index{cec}{pantaram@\textit{pa\d n\d t\=aram} trésorerie du temple}.
	\item ......tta\d laiy\=al munnilait tiru
	\item ....kku ira\d n\d tilum .......
	\item \dots
	\item .....r\=atittava\b r cellak ka\d tavat\=aka
	\item .....\textbf{mahe[\'s]va}ra\index{cec}{srimahesvara@\textit{\'sr\=\i mahe\'svara} dévot, surveillant} ra[\textbf{k\d s}]ai...........
	\item ........ttu\tdanda
\end{enumerate}

\subsection*{CEC 1.3 Traduction}
(1-16) Que la prosp\'erit\'e soit!
La [6\up{e}] ann\'ee [de r\`egne de V\=\i rar\=ajendra]deva,
empereur des trois mondes,
pour le Seigneur propri\'etaire de Tirutt\=o\d nipuram\index{gnl}{Tonipuram@T\=o\d nipuram!Tirutt\=o\d nipuram}\footnote{\textit{u\d taiy\=a\b n}, pr\'ec\'ed\'e d'un toponyme et appliqu\'e \`a un individu, indique probablement que ce dernier jouit de la possession de terre\index{gnl}{terre}(s) en cet endroit; cf. \textsc{Karashima, Subbarayalu, Matsui} (1978: xlv-xlvii). Cependant, quand ce terme est pr\'ec\'ed\'e d'un toponyme renvoyant \`a un lieu saint, comme le temple\index{gnl}{temple} ou le \textit{li\.nga}\index{gnl}{linga@\textit{li\.nga}} en contexte shiva\"ite\index{gnl}{shiva\"ite}, il d\'esigne la divinit\'e en tant que propri\'etaire des biens du temple\index{gnl}{temple}. Sur la notion de propri\'et\'e divine voir \textsc{Reiniche} (1989: 3 et 169) et \textsc{Sanderson} (2003-4: n. 250).}
dans Tirukka\b lumalam\index{cec}{Tirukkalumalam@Tirukka\b lumalam}, \textit{brahmadeya}\index{cec}{brahmadeya@\textit{brahmadeya}}\footnote{Le \textit{brahmadeya} est un territoire, g\'en\'eralement un village, donn\'e \`a des brahmane\index{gnl}{brahmane}s et administr\'e au niveau local\index{gnl}{local} par ces derniers r\'eunis en assemblée\index{gnl}{assemblée} (\textit{sabh\=a}); cf. \textsc{Karashima} (*2001a [1966]), \textsc{Stein} (1980: chapitre 4), \textsc{Champakalakshmi} (*2004 [2001]) sur le cas particulier des \textit{brahmadeya} appel\'es \textit{ta\b niy\=ur} et \textsc{Veluthat} (1993: 196-211) pour une pr\'esentation incluant la r\'egion du Kerala actuel.}
du Tirukka\b lumalan\=a\d tu\index{cec}{Tirukka\b lumalan\=a\d tu}, dans le R\=aj\=adhir\=ajava\d la-n\=a\d tu\index{cec}{Rajadhirajavala@R\=aj\=adhir\=ajava\d lan\=a\d tu};
Karu\d n\=akaratevan alias V\=a\d n\=atir\=aya\b n\index{cec}{Karu\d n\=akaratevan}\index{cec}{V\=a\d n\=atir\=aya\b n}\footnote{\`A l'\'epoque \textit{c\=o\b la}\index{gnl}{cola@\textit{c\=o\b la}}, le titre V\=a\d n\=atir\=aya\b n a au moins seize occurrences, selon \textsc{Karashima, Subbarayalu, Matsui} (1978), qui t\'emoignent clairement d'une fonction substantielle dans le syst\`eme administratif. En effet, ce titre, employé exceptionnellement avec un nom propre (ARE 1931-32 74 publi\'e dans part. II p. 50), qui appara\^it dans la seconde partie de la p\'eriode \textit{c\=o\b la}\index{gnl}{cola@\textit{c\=o\b la}}, est principalement celui d'un des signataires l\'egalisant le contenu d'une inscription (SII 4 529, 7 780 et 5 478). Le porteur de ce titre appartient souvent \`a un groupe de signataires pr\'esid\'e par un haut officier\index{gnl}{officier} royal \textit{tirumantira \=olai}\index{cec}{olai@\textit{olai} ôle!tirumantiravolai@\textit{tirumantirav\=olai} officier-scribe} (ARE 1927 148 l.~15, 1931-32 74 avec texte en part. II p. 52; EI 21 31; SII 5 663, 6 2 et 438, 7 433, 17 730; IPS 163 et 166; SITI 64 et 518 et Dar. c.1). Il est parfois celui qui \'ecrit le texte de l'inscription, \textit{\=olai} (SII 8 252 et 17 452). Ainsi, le donateur Karu\d n\=akaratevan alias V\=a\d n\=atir\=aya\b n \'etait tr\`es probablement un notable au service\index{gnl}{service} de l'\'Etat.

De plus, trois inscriptions de l'ancien district de Ta\~nc\=av\=ur (ARE 1927 152 l.~2; SII 8 216 et 257) nomment un donateur de la m\^eme origine\index{gnl}{origine} et appel\'e Karu\d n\=akarateva\b n alias Amarak\=o\b n\=ar. Amarak\=o\b n ou Amarak\=o\b n\=ar\index{cec}{Amarako\b n}, absent de la liste des soixante et onze titres de vassalit\'e de \textsc{Karashima, Subbarayalu, Matsui} (1978: xxxiv), devient un v\'eritable titre \`a partir de la seconde moiti\'e du \textsc{xii}\up{e} si\`ecle. En effet, l'emploi du participe \textit{\=a\b na} signifiant \textit{alias} dans ces trois inscriptions soutient cette id\'ee. Par ailleurs, le terme \textit{amarak\=o\b n}\index{cec}{Amarako\b n} sans nom propre appartient souvent, lui aussi, \`a un groupe de signataires pr\'esid\'e par un haut officier\index{gnl}{officier} royal (EI 21 31; SII 17 135, 585 et 587, 6 436, 3 87; SITI 18, 19 et 628) et, il est parfois doubl\'e d'un titre important comme Pallavar\=aya\b n\index{cec}{Pallavar\=aya\b n} (CEC 3).
Les trois inscriptions mentionnant Karu\d n\=akarateva\b n alias Amarak\=o\b n\=ar\index{cec}{Amarako\b n} se situent dans une aire g\'eographique limit\'ee c'est-\`a-dire dans un espace de donation r\'ealiste \`a l'\'echelle humaine (sur les r\'eseaux des donateurs, cf. \textsc{Heitzman} (*2001 [1997]: chapitre 6)). Cet espace, d\'elimit\'e par Talai\~n\=ayi\b ru (ARE 1927 152), Tiruvala\~ncu\b li (SII 8 216) et Tirukka\d lar (SII 8 257), englobe C\=\i k\=a\b li\index{gnl}{Cikali@C\=\i k\=a\b li}. De plus, ces textes datent, avec certitude pour SII 8 216 et 257, respectivement, de 1172 et 1173. Ils sont ant\'erieurs d'une dizaine d'ann\'ees seulement au CEC 1 de C\=\i k\=a\b li\index{gnl}{Cikali@C\=\i k\=a\b li}.

Enfin, les titres \'etaient octroy\'es vraisemblablement selon une certaine hi\'erarchie. Ainsi, parmi les -\textit{r\=aya\b n}, les Brahmar\=ayar et les Pallavar\=ayar\index{cec}{Pallavar\=aya\b n} occupaient des postes de grande importance dans l'administration et principalement dans le fisc selon \textsc{Karashima, Subbarayalu, Matsui} (1978: lii-lv). Ceci laisse penser qu'une \'evolution \'etait possible et qu'elle pouvait engendrer ce faisant un changement de titre. \textsc{Subbarayalu} (*2001a [1983]: 18) \'evoque le cas de changement de titre au nouveau r\`egne. Par ailleurs, il existe des exemples de changement de titre d'officier\index{gnl}{officier}s \`a l'\'epoque \textit{c\=o\b la}\index{gnl}{cola@\textit{c\=o\b la}} sous un m\^eme r\`egne. Un officier\index{gnl}{officier} militaire de R\=aj\=adhir\=aja II, Ammai Appa\b n alias R\=ajar\=ajavi\b lupparaiya\b n (SII 17 583) devient Ammai Appa\b n alias Pallavar\=aya\b n\index{cec}{Pallavar\=aya\b n} (EI 21 31; SII 17 585 et 587) \`a Tiruv\=ar\=ur\index{gnl}{Ar\=ur@\=Ar\=ur!Tiruv\=ar\=ur}. Un autre officier\index{gnl}{officier}, donateur \`a Citamparam\index{gnl}{Citamparam}, sous R\=ajar\=aja III\index{gnl}{Rajaraja III@R\=ajar\=aja III} (?), appel\'e Civetava\b na Perum\=a\b n\=a\b na K\=ali\.nkar\=aya\b n la dixi\`eme et quatorzi\`eme ann\'ee de r\`egne (SITI 18 et 19) est Civetava\b na Perum\=a\b n\=a\b na To\d n\d taim\=a\b n la seizi\`eme ann\'ee (SITI 20).

Ainsi, nous supposons que Karu\d n\=akarateva\b n alias Amarak\=o\b n\=ar\index{cec}{Amarako\b n} est celui qui est devenu Karu\d n\=akarateva\b n alias V\=a\d n\=atir\=aya\b n.

Sur ce donateur et son \'eventuelle parent\'e avec Ammai Appa\b n alias Pallavar\=aya\b n\index{cec}{Pallavar\=aya\b n} un propriétaire [terrien] de Vetavanam et de Pa\b laiyan\=ur dans le Melmalaippa\b laiyan\=urn\=a\d tu du Jeya\.nko\d n\d taco\b lama\d n\d talam\index{cec}{Jayankontacolamandala@Jaya\.nko\d n\d tac\=o\b lama\d n\d dalam}, mentionn\'e ci-dessus, voir \textsc{Nilakanta Sastri} (*2000 [1955]: 369 et 373).
CEC 4, certainement contemporaine de CEC 1, mentionne un individu de la m\^eme origine\index{gnl}{origine} (l.~8-11). Cependant, aucun autre \'el\'ement ne permet d'identifier le ou les donateurs de ces inscritptions.}, un propriétaire [terrien] de Vetavanam et de Pa\b laiyan\=ur dans le Melmalaippa\b laiyan\=urn\=a\d tu du Jeya\.nko\d n\d taco\b lama\d n\d talam
pour les deux lampe\index{gnl}{lampe}s [qu'il a] plac\'ees devant l'[image\index{gnl}{image}] divine\footnote{\textit{tirumunpu}.} de ce Seigneur, [parce qu']il faut chaque jour X unit\'e(s) d'\textit{uri}\footnote{Unit\'e utilis\'ee pour mesurer les grains et les liquides comme le beurre clarifi\'e (SII 14 27 l.~21-2 \textit{ney uri}), le yaourt (SII 3 128 l.~40 \textit{tayiramutu potu uri}), etc.} d'huile,
[voici les terres\index{gnl}{terre}] qu'il laisse pour l'huile apr\`es les avoir achet\'ees:\\
(16-51) \dots\ du premier carr\'e du premier canalicule, au sud du canal de Tillaivi\d ta\.nkar\index{cec}{Tillaivitankan@Tillaivi\d ta\.nka\b n} et \`a l'est de la \textit{vati}\index{cec}{vati@\textit{vati}} de Talaicca\.nk\=a\d tu\index{cec}{Talaiccankatu@Talaicca\.nk\=a\d tu}\index{gnl}{Talaiccankatu@Talaicca\.nk\=a\d tu} dans le \textit{brahmadeya}\index{cec}{brahmadeya@\textit{brahmadeya}} de Tirukka\b lumalam\index{cec}{Tirukkalumalam@Tirukka\b lumalam}\footnote{L'emplacement des parcelles donn\'ees est pr\'ecis\'e par rapport aux diff\'erents canaux d'irrigation. Sur les travaux d'irrigation et la sp\'ecificit\'e des canaux; cf. \textsc{Heitzman} (*2001 [1997]: 42 et en particulier n. 4). Nous ne pensons pas, contrairement à \textsc{Gros} (1970: 91), que le terme \textit{vati}\index{cec}{vati@\textit{vati}} renvoie dans les inscriptions médiévales à une route.} \dots;
la terre\index{gnl}{terre} nomm\'ee Cakkaravarttivi\d l\=akam du premier carr\'e du\dots\ canalicule\dots\ canal\dots, \`a l'est de\dots;
la terre\index{gnl}{terre} de deux \textit{m\=a}\dots\ achet\'ee aupr\`es d'un propriétaire \dots, Tirutt\=o\d nipuram\index{gnl}{Tonipuram@T\=o\d nipuram!Tirutt\=o\d nipuram}u\d taiy\=a\b n;
la terre\index{gnl}{terre} achet\'ee aupr\`es de V\=acciya\b n\index{cec}{Vacciyan@V\=acciya\b n} Pira\d laiyavi\d ta\.nkan\index{cec}{Piralaiya@Pira\d laiyavi\d ta\.nka\b n} Tirutt\=o\d nipuram\index{gnl}{Tonipuram@T\=o\d nipuram!Tirutt\=o\d nipuram}u\d taiy\=a\b n;
la terre\index{gnl}{terre} achet\'ee aupr\`es de\dots;
la terre\index{gnl}{terre} achet\'ee aupr\`es de Kavu\d niya\b n\index{cec}{kavuniyan@\textit{kavu\d niya\b n gotra}} Tirun\=a\dots mpir\=an\dots\
trésorerie\index{gnl}{tresorerie@trésorerie} (du temple\index{gnl}{temple})\dots\ tant que durent lune et soleil\dots\ protection des \textit{Mahe\'svara}\index{cec}{srimahesvara@\textit{\'sr\=\i mahe\'svara} dévot, surveillant}\footnote{Ce terme, quand il n'est pas pr\'ecis\'e qu'il s'agit des surveillants du temple\index{gnl}{temple} \textit{ka\d nk\=a\d ni}\index{cec}{kani@\textit{k\=a\d ni} droit, propriété} (CEC 28 l.~7, 9 l.~4 et 6, 12 l.~18, 8 l.~6), renvoie g\'en\'eralement \`a l'ensemble des dévot\index{gnl}{devot(e)@dévot(e)}s shiva\"ite\index{gnl}{shiva\"ite}s. En effet, la protection des actes immortalis\'es sur les murs du temple\index{gnl}{temple} repose entre les mains des dévot\index{gnl}{devot(e)@dévot(e)}s qui en font un service\index{gnl}{service} pour la divinit\'e \textit{tirutto\d n\d tu} (CEC 10 l.~7), cf. \textsc{Reiniche} (1989: 138-140).}
\dots


\section*{CEC 2}
\subsection*{CEC 2.1 Remarques}

Cette inscription, r\'esum\'ee dans l'ARE 1918 360, a \'et\'e localis\'ee sur le mur sud du temple\index{gnl}{temple} principal de \'Siva\index{gnl}{Siva@\'Siva} au moment de son relev\'e. Elle date de la septi\`eme ann\'ee de r\`egne du roi\index{gnl}{roi} \textit{c\=o\b la}\index{gnl}{cola@\textit{c\=o\b la}} Tribhuvanacakravartin V\=\i rar\=ajendradeva. \textsc{Mahalingam} (1992: 549, Tj. 2410) identifie ce roi\index{gnl}{roi} comme Kulottu\.nga III\index{gnl}{Kulottu\.nga III} et date l'\'epigraphe de \textbf{1185}.

Le texte est inédit et sa localisation n'a pu \^etre v\'erifi\'ee \textit{in situ}. Cependant, il est certain que l'inscription ne se trouve pas aujourd'hui sur le mur sud du temple\index{gnl}{temple} principal et il est possible qu'elle ait été gravée sur le mur sud du \textit{ma\d n\d dapa}. En effet, aucune inscription n'est grav\'ee sur les murs actuels du temple\index{gnl}{temple} principal de \'Siva\index{gnl}{Siva@\'Siva} et quelques textes du mur sud du \textit{ma\d n\d dapa} n'ont pu \^etre identifi\'es et lus \`a cause des couches de peinture qui les recouvrent. Le texte que nous \'editons est donc fond\'e sur seulement l'examen de la transcription de l'ASI. Ainsi, les conjectures propos\'ees en note, bien que probables, sont \`a consid\'erer avec r\'eserve car elles ne restituent peut-\^etre pas le nombre d'\textit{ak\d sara} manquants.

L'inscription enregistre une donation de terres\index{gnl}{terre} par Utaiya\~nceyt\=a\b n T\=a\b li alias Co\d lentiraci\.nka Vi\b lupparaya\b n, un propriétaire terrien de Karupp\=ur, pour offrir quotidiennement, et \'eternellement, des feuilles de b\'etel et des noix d'arec au couple divin.

\subsection*{CEC 2.2 Texte}
\begin{enumerate}
	\item {\textbf{svasti \'sr\=\i} \ddanda\ tiripuvanacakkara}
	\item {[va]rttika\d l\index{cec}{Tribhuvanacakravarti} \textbf{\'sr\=\i}virar\=a\textbf{jenti}rateva}\index{cec}{Virarajendre@V\=\i rar\=ajendra}
	\item {[\b r]ku y\=a\d n\d tu e\b l\=avatu \textbf{\'sr\=\i}p\=atant\=a}
	\item \.nkum tiru[pa\d l]\d li\index{cec}{tiruppalli@\textit{tiruppa\d l\d li}}\footnote{Conjecture que nous proposons suivant \textit{devar pa\d l\d liccivikai} dans SII 3 128 l.~85.}c civiy\=arkku\index{cec}{civiyar@\textit{civiy\=ar} porteur de palanquins}\footnote{Vraisemblablement une variante de \textit{civikaiy\=ar} \og ceux du palanquin\index{gnl}{palanquin}\fg\, c'est-\`a-dire les porteurs de palanquin\index{gnl}{palanquin} (cf. SII 3 128 l.~85-6: \textit{civikai k\=avu\~ncivikaiy\=ar}).}cc\=amu
	\item {[t\=ayam]\index{cec}{camutayam@\textit{c\=amut\=ayam}}\footnote{Propos\'ee par \textsc{G. Vijayavenugopal}, cette conjecture est fond\'ee sur l'examen des l.~19 et 20 de PI 491.}\dots karupp\=uru\d taiy\=a\b n\index{cec}{Karuppurutaiyan@Karupp\=uru\d taiy\=a\b n} utaiya\~nce}
	\item yt\=a\b n\index{cec}{utaiyanceytan@Utaiya\~nceyt\=a\b n} t\=a\b liy\=a\b na\index{cec}{tali@T\=a\b li} co\d lentiraci
	\item \.nka\index{cec}{colentira@C\=o\d lentiraci\.nka} vi\b lupparaya\b ne\b n\index{cec}{vilupparayan@Vi\b lupparaya\b n} ir\=aj\=ati
	\item r\=ajava\d lan\=a\d t\d tup\index{cec}{Rajadhirajavala@R\=aj\=adhir\=ajava\d lan\=a\d tu} piramateyam\index{cec}{brahmadeya@\textit{brahmadeya}}
	\item tirukka\b lumalattu\index{cec}{Tirukkalumalam@Tirukka\b lumalam} u\d taiy\=ar tirutto
	\item \d nipuramu\d taiyarkkum\index{cec}{Tiruttoni@Tirutt\=o\d nipuramu\d taiya n\=aya\b n\=ar, \'Siva} periyan\=acciy\=ar\index{cec}{periyanacciyar@Periyan\=acciy\=ar}
	\item kkum cantir\=atuttavarai a\d taikk\=ayamutu\index{cec}{ataikkay@\textit{a\d taikk\=ay} noix d'arec} p\=a
	\item kkum tev\=ur\index{cec}{tevur@T\=ev\=ur} ilaiyamutu\index{cec}{ilaiyamutu@\textit{ilaiyamutu} feuille de bétel} pa\b r\b rum amutu cetaru\d la
	\item n\=a\b n vi\d t\d ta nilam\=avatu\index{cec}{nilam@\textit{nilam} terre} i\b n\b n\=a\d t\d tu n\=a\.nk\=ura\b na\index{cec}{nankur@N\=a\.nk\=ur} tiru
	\item cci\b r\b rampalamu\d taiy\=ar\index{cec}{tiruccirrampalamutaiyar@Tirucci\b r\b rampalamu\d taiy\=ar} \textbf{\'sr\=\i}p\=atat\=u\d liccarupvita
	\item ma\.nkalattu\index{cec}{caturvedimangalam@Caturvedima\.ngalam!P\=atat\=u\d liccaturvedima\.ngalam} te\b n pi\d t\=akai\index{cec}{pitakai@\textit{pi\d t\=akai} hameau} ki\d t\=ara\.nko\d n\d taco\b la
	\item {[na]lluril\index{cec}{kitarankontacola@Ki\d t\=ara\.nko\d n\d taco\b lanall\=ur} k\=a\d ni\index{cec}{kani@\textit{k\=a\d ni} droit, propriété} u\d taiya po\b n\b nu\b l\=a\b n\index{cec}{ponnulan@Po\b n\b nu\b l\=a\b n} aiyya\index{cec}{Aiyya}}
	\item {[nam]pi\index{cec}{Nampi} u\d taiy\=a\b num tiruv\=aykkulamu\d taiy\=a\b n\index{cec}{tiruvaykkulam@Tiruv\=aykkulamu\d taiy\=a\b n} ai}
	\item {[y]ya nampiyum\index{cec}{Nampi} aiyya\index{cec}{Aiyya} nampiteva\b num\index{cec}{Nampiteva\b n} u\d l\d li\d t\d t}
	\item {[\=ar pa]kkal n\=a\b n pe\b r\b ru\d taiyen\=ay e\b n\b nut\=ay varuki\b ra ko}
	\item llai nilattukku\index{cec}{nilam@\textit{nilam} terre} melp\=a\b rkellai k\=a\b rai \dots
	\item k\=alukku ki\b lakkum va\d tap\=a\b rkellai k\=averi \dots
	\item kum tiruve\d nk\=a\d t\d tumu.ka \dots
	\item \d l\d laiy\=ar ilaiyamutu\index{cec}{ilaiyamutu@\textit{ilaiyamutu} feuille de bétel} \dots
	\item \dots tiruve\d nk\=a\d t\d tumu \dots
	\item peru n\=a\b nkellai\dots
	\item \b n \=atitta\b n\index{cec}{atittan@\=Atitta\b n} n\=ar\=aya\d na\b n\index{cec}{narayanan@N\=ar\=aya\d na\b n} uyyakko\d n\d t\=a\b n\index{cec}{uyyakkontan@Uyyakko\d n\d t\=a\b n} nilam\index{cec}{nilam@\textit{nilam} terre} ira\d n\d tu m\=acci\b n\b namum ai
	\item yya nampi\index{cec}{Nampi} u\d taiy\=a\b n i\textbf{r\=ajar\=aja}pperuvilai\index{cec}{vilai@\textit{vilai} prix!peruvilai@\textit{peruvilai} prix fixé (?)}ko\d n\d tu a\b nupa[vi]kki\b ra nilam\index{cec}{nilam@\textit{nilam} terre}
	\item mu\b n\b ru m\=avum ivarka\d l pakkal\index{cec}{pakkal@\textit{pakkal} auprès de} ka\d latt\=uru\d taiy\=a\b n\index{cec}{kalattur@Ka\d latt\=uru\d taiy\=a\b n} t\=ayilum nall\=a\b n\index{cec}{tayilum@T\=ayilum nall\=a\b n} vi
	\item {[lai]ko[\d n\d tu] mu\b npu ilaiyamutu\index{cec}{ilaiyamutu@\textit{ilaiyamutu} feuille de bétel} to\d t\d tam\=aka vi\d lai}
	\item nilam\index{cec}{nilam@\textit{nilam} terre} e\d t\d tum\=avum \=aka nilam\index{cec}{nilam@\textit{nilam} terre} araiye
	\item mu\b n\b ru m\=acci\b n\b namum nikki n\=\i kki ni\b n\b ra e\b n\b nu
	\item t\=ay varuki\b ra nilam\index{cec}{nilam@\textit{nilam} terre} araiyum innilattu\index{cec}{nilam@\textit{nilam} terre}
	\item op\=ati \=a\b ri\d tu pa\d tukaiyum potuvum po
	\item t\=ariyu\.n ki\d na\b ru\.n ku\d lamum ma
	\item \b r\b rum epperppa\d t\d ta urimaika
	\item \d lum akappa\d ta vanta nilam\index{cec}{nilam@\textit{nilam} terre} ci
	\item van\=amattukk\=a\d niy\=akakkai\index{cec}{kani@\textit{k\=a\d ni} droit, propriété}
	\item kko\d n\d tu i\b rai i\b ruttu\index{cec}{iruttu@\textit{i\b ruttu} payer un impôt} i
	\item {[\b r]ai mikutikku nitta\b rpa\d ti n\=a\d l o}
	\item \b n\b rukku a\d taikk\=ayamutu\index{cec}{ataikkay@\textit{a\d taikk\=ay }noix d'arec} p\=akku irun\=u
	\item \b rum tev\=ur\index{cec}{tevur@T\=ev\=ur} ilaiyamutu\index{cec}{ilaiyamutu@\textit{ilaiyamutu} feuille de bétel} pa\b r\b ru \=a\b rum ni
	\item tta nimantam\=aka cantir\=atittavarai c
	\item ellak ka\d tavat\=aka vi\d t\d tuk ku\d tutte\b n ka
	\item {[rupp\=ur u]\d taiy\=an uta}
	\item ya\~nce[yt\=a\b n t\=a\b li]y\=a\b na\index{cec}{tali@T\=a\b li} co\d lentiraci\.nka\index{cec}{colentira@C\=o\d lentiraci\.nka} vi\b luppar\=aya
	\item {[\b ne\b n]\index{cec}{vilupparayan@Vi\b lupparaya\b n}\footnote{Reconstitution du nom du donateur des l.~43-6 suivant les l.~5-7.} itu \textbf{\'sr\=\i}m\=a\textbf{he\'sva}ra ira\textbf{k\d sai}\ddanda}
\end{enumerate}

\subsection*{CEC 2.3 Traduction}
(1-13) Que la prosp\'erit\'e soit! En la septi\`eme ann\'ee [de r\`egne] de V\=\i rar\=ajendra-deva, empereur des trois mondes;
moi Utaiya\~nceyt\=a\b n\index{cec}{utaiyanceytan@Utaiya\~nceyt\=a\b n} T\=a\b li\index{cec}{tali@T\=a\b li} alias Co\d lentiraci\.nka\index{cec}{colentira@C\=o\d lentiraci\.nka} Vi\b lupparaya\b n\index{cec}{vilupparayan@Vi\b lupparaya\b n}\footnote{Dans l'\'etat actuel des recherches, l'identit\'e\index{gnl}{identit\'e} de ce donateur reste obscure mais son titre de Vi\b lupparaya\b n pr\'ec\'ed\'e du titre royal Co\d lentiraci\.nka sugg\`ere qu'il est une autorit\'e politique de poids, au moins au niveau local\index{gnl}{local}, proche du pouvoir royal; \textsc{Karashima, Subbarayalu, Matsui} (1978: lii-lv).}, un propriétaire [terrien] de Karupp\=ur\index{cec}{Karuppurutaiyan@Karupp\=uru\d taiy\=a\b n}, repr\'esentant des porteurs de palanquin\index{gnl}{palanquin}\footnote{Ce donateur assume aussi une fonction li\'ee au \textit{c\=amut\=ayam}\index{cec}{camutayam@\textit{c\=amut\=ayam}}. Plusieurs inscriptions de Citamparam\index{gnl}{Citamparam} du \textsc{xiii}\up{e} si\`ecle \'evoquent, parmi les groupes employ\'es dans le temple\index{gnl}{temple}, destinataires des actes que constituent les inscriptions, celui des \textit{c\=amut\=aya\~nceyv\=arka\d l} `ceux qui font \textit{c\=amut\=ayam}' (SII 4 222 l.~2 et 229 l.~6; SII 8 44 l.~2, 47 l.~2, 48 l.~1, 49 l.~1, 51 l.~2, 52 l.~1, 53 l.~2, 54 l.~3, 55 l.~2, 56 l.~1; SII 12 149 l.~2, 151 l.~3, 152 l.~3, 154 l.~2, 159 l.~2, 160 l.~7, 171 l.~2, 172 l.~2, 173 l.~3, 174 l.~3, 175 l.~6, 201 l.~2, 209 l.~2; SITI 18, 19 et 20). Un individu s'y distingue par sa nomination personnelle: `Tirum\=a\d likaikk\=u\b ru' Tillaiyampalap Pallavar\=aya\b n\index{cec}{Pallavar\=aya\b n} \textit{camut\=ayam} du temple\index{gnl}{temple} d'\=A\d lu\d taiy\=ar (SII 4 222 l.~1-2). La traduction propos\'ee par \textsc{Orr} (2004: 234) pour ce groupe, \og those who do [the task of] the assembly\fg, qui figure, selon cet auteur parmi les comit\'es qui veillaient au bon fonctionnement des affaires \'economiques et cultuelles du temple\index{gnl}{temple}, ne nous convainc pas. De quelle assemblée\index{gnl}{assemblée} s'agit-il? Quelle est sa fonction? \textsc{Subbarayalu} (2003), s.v. \textit{s\=amut\=aya\~n ceyv\=arka\d l}, comprend qu'il s'agit d'un groupe important attach\'e au temple\index{gnl}{temple} de Citamparam\index{gnl}{Citamparam} et donne la r\'ef\'erence SII 12 154. Or, cette d\'efinition vague est inexacte car ce groupe se rencontre ailleurs.

En effet, le texte de SII 8 205 enregistre une transaction sign\'ee par les membres d'une assemblée\index{gnl}{assemblée} villageoise, un \textit{\=ur}, \`a Mu\b niy\=ur (Kumpak\=o\d nam tk.) la vingt-huitième ann\'ee de règne de R\=ajar\=aja III\index{gnl}{Rajaraja III@R\=ajar\=aja III}. Deux de ces membres sont d\'esign\'es par le terme \textit{c\=amut\=ayam} suivi du lieu d'origine\index{gnl}{origine} (l.~4): il est clair ici que le terme s'applique aussi à des individus.
De plus, une inscription datant de la huiti\`eme ann\'ee de r\`egne de Kulottu\.nga III\index{gnl}{Kulottu\.nga III} \`a T\=ar\=acuram\index{gnl}{Taracuram@T\=ar\=acuram} (Dar. a.8 l.~4) compte deux occurrences du terme \textit{c\=amut\=ayam}. Ce terme est pr\'ec\'ed\'e d'un nom propre (\textit{Vatuli \=Ar\=a amudu \'sr\=\i ...\b n\=a\b na Madur\=antakap-Pirammar\=aya\b n}) et, plus bas, d'un groupe particulier au datif (\textit{tirupa\d l\d litto[\.n]ga\d lu\d daiy\=arga\d lukku}). Il nous appara\^it \'evident ici que \textit{c\=amut\=ayam} d\'esigne la fonction d'un individu li\'e \`a un groupe desservant le temple\index{gnl}{temple}.
Enfin, une \'epigraphe \textit{p\=a\d n\d dya}\index{gnl}{pandya@\textit{p\=a\d n\d dya}} de Tiruna\d l\d l\=a\b ru\index{gnl}{Nallaru@Na\d l\d l\=a\b ru!Tiruna\d l\d l\=a\b ru} (PI 491), qui daterait de 1333, enregistre la vente d'un service\index{gnl}{service} de \textit{c\=amut\=ayam} (\textit{c\=amat\=ayappa\d ni o\b n\b ru}) du temple\index{gnl}{temple} \`a un certain V\=a\d n\=atar\=aja Brahm\=ar\=aya\b n pour cinquante \textit{pa\d nam}, l.~4 et 6. Elle contient aussi deux occurrences du terme \textit{c\=amut\=ayam} dans une liste des signataires, aux c\^ot\'es des surveillants, des comptables et des officiants du temple\index{gnl}{temple}. Le terme s'y rattache clairement \`a deux individus en rapport avec deux groupes professionnels de gardiens et de porteurs: l.~19 \textit{tirumeykk\=app\=arkku c\=amut\=ayam n\=a\d tu\d tain\=ayakap Pallavaraiya\b n\index{cec}{Pallavar\=aya\b n}} et l.~20 \textit{cip\=atant\=a\. nkum c\=amut\=ayam periyan\=ayaka\b n tiruvala\~ncu\b lipicca\b n}.

Ainsi, suivant l'ARE 1965-66, introduction p.~7, sur ce texte publié dans PI 491, \textit{c\=amut\=ayappa\d ni} serait un service\index{gnl}{service} effectu\'e par un groupe dans le temple\index{gnl}{temple}, dont les droits\index{gnl}{droits (\textit{k\=a\d ni})} d'acquisition sont monnayables et dont le repr\'esentant est qualifi\'e de \textit{c\=amut\=ayam}. P. R. \textsc{Srinivasan} suit cette interpr\'etation, pour Dar. a.8 l.~4, et propose la traduction suivante: \og V\=atuli \=Ar\=a amudu \'Sr\=\i\ ...n alias Madhur\=antakap-Piramm\=ar\=aya\b n (\dots) should stand as their representative. For his work, he should get (\dots), as well as an annual cash amount equal to that which was received by the representative of the class of people called Tirupa\d l\d litto\.nga\d lu\d daiy\=ar\fg. Puis, ce chercheur ajoute dans ses notes: \og The real purport of the record was to make provision for the gold workers of the temple\index{gnl}{temple} through the institution of a \textit{samud\=ayam} which was entrusted to the care of a Piramm\=ar\=aya\b n\fg.

Pour nous, consid\'erant la ressemblance des titres de haut rang des repr\'esentants et de l'acqu\'ereur des droits\index{gnl}{droits (\textit{k\=a\d ni})} (Vi\b lupparaya\b n\index{cec}{vilupparayan@Vi\b lupparaya\b n}, Pallavaraiya\b n\index{cec}{Pallavar\=aya\b n} et Brahm\=ar\=aya\b n) dans ces inscriptions, nous pensons que le repr\'esentant peut être celui qui poss\`ede les droits\index{gnl}{droits (\textit{k\=a\d ni})} d'un service\index{gnl}{service} particulier dans le temple\index{gnl}{temple}. Et s'il en \'etait ainsi, le donateur serait ici le repr\'esentant des porteurs de palanquin\index{gnl}{palanquin} et le propri\'etaire des droits\index{gnl}{droits (\textit{k\=a\d ni})} de ce service\index{gnl}{service}.} [de la couche sacr\'ee] qui portent les pieds sacr\'es\footnote{La relative \textit{p\=atant\=a\.nkum} a pour sujet les porteurs \textit{civiy\=ar} et non la couche \textit{tirupa\d l\d li} qui est \`a consid\'erer comme leur compl\'ement. La relative renvoie \`a l'image\index{gnl}{image} classique des dévot\index{gnl}{devot(e)@dévot(e)}s qui se couronnent des pieds d'une figure sainte. Ainsi, le dévot\index{gnl}{devot(e)@dévot(e)} est souvent d\'esign\'e par le terme \textit{a\d tiya\b n}, \og celui qui est aux pieds [du seigneur]\fg.},
pour la grande Dame Periyan\=acciy\=ar\index{cec}{periyanacciyar@Periyan\=acciy\=ar} et le Seigneur propri\'etaire de Tirutt\=o\d nipuram\index{gnl}{Tonipuram@T\=o\d nipuram!Tirutt\=o\d nipuram} de Tirukka\b lu-malam\index{cec}{Tirukkalumalam@Tirukka\b lumalam}, \textit{brahmadeya}\index{cec}{brahmadeya@\textit{brahmadeya}} du R\=aj\=adhir\=ajava\d lan\=a\d tu\index{cec}{Rajadhirajavala@R\=aj\=adhir\=ajava\d lan\=a\d tu}, tant que durent lune et soleil, pour qu'[ils] fassent la gr\^ace de manger des bottes de feuilles [de b\'etel] de Tev\=ur\index{cec}{tevur@T\=ev\=ur} et des noix d'arec\footnote{Les \'el\'ements comestibles offerts aux divinit\'es et aux personnages saints sont suffix\'es par -\textit{amutu\index{gnl}{amutu@\textit{amutu}}}, \og ambroisie\index{gnl}{ambroisie}\fg, nourriture par excellence des dieux. Par exemple, dans SII 5 642 l.~44-47, comme ici et ailleurs dans le corpus\index{gnl}{corpus}, les diff\'erents composants du repas divin sont ainsi assimil\'es \`a de l'ambroisie\index{gnl}{ambroisie}: \textit{tiruvamutu} est le riz\index{gnl}{riz}, l'ambroisie\index{gnl}{ambroisie} sacr\'ee (parce que l.~44 du riz\index{gnl}{riz} d\'ecortiqu\'e \textit{arici} est offert pour le pr\'eparer), \textit{ka\b riyamutu} les mets, \textit{mi\d lakamutu} le poivre, \textit{uppamutu} le sel, \textit{neyyamutu} le beurre clarifi\'e et \textit{a\d taikk\=ayamutu}\index{cec}{ataikkay@\textit{a\d taikk\=ay} noix d'arec} les noix d'arec. Sur l'usage identique de ce terme dans la litt\'erature religieuse; cf. \textsc{Veluppillai} (2013).}, j'ai donn\'e la terre\index{gnl}{terre} suivante:\\
(13-43) [situ\'ee] \`a Ki\d t\=ara\.nko\d n\d taco\b lanallur\index{cec}{kitarankontacola@Ki\d t\=ara\.nko\d n\d taco\b lanall\=ur}, hameau au sud de N\=a\.nk\=ur\index{cec}{nankur@N\=a\.nk\=ur} de ce \textit{n\=a\d tu}\footnote{N\=a\.nk\=ur est une localit\'e du N\=a\.nk\=urn\=a\d tu dans le R\=aj\=adhir\=ajava\d lan\=a\d tu\index{cec}{Rajadhirajavala@R\=aj\=adhir\=ajava\d lan\=a\d tu} (\textsc{Subbarayalu} (1973: 104) et carte 10). N\=a\.nk\=urn\=a\d tu n'\'etant pas mentionn\'e dans le texte, le d\'emonstratif renvoie dans le cas pr\'esent \`a la division territoriale du \textit{va\d lan\=a\d tu}, le R\=aj\=adhir\=ajava\d lan\=a\d tu\index{cec}{Rajadhirajavala@R\=aj\=adhir\=ajava\d lan\=a\d tu}, l.~7-8.} alias Tirucci\b r\b rampalamu\d taiy\=ar\index{cec}{tiruccirrampalamutaiyar@Tirucci\b r\b rampalamu\d taiy\=ar} \'Sr\=\i p\=atat\=u\d liccaruppetima\.nkalam\index{cec}{caturvedimangalam@Caturvedima\.ngalam!P\=atat\=u\d liccaturvedima\.ngalam}. J'ai obtenu [cette terre\index{gnl}{terre}] aupr\`es\footnote{Le terme \textit{u\d l\d li\d t\d t\=ar} signifierait \og les autres\fg\ (\textsc{Karashima} (*2001a [1966]: 181, n.~5) et impliquerait alors des individus qui ne sont pas mentionn\'es dans l'inscription. Litt\'eralement, il a le sens de \og ceux qui sont inclus\fg.}
de l'ayant droit\footnote{Sur le terme \textit{k\=a\d ni} et ses diff\'erents sens et emplois dans les inscriptions, cf. \textsc{Heitzman} (*2001 [1997]: 54) \textit{sq}.} Po\b n\b nul\=a\b n\index{cec}{ponnulan@Po\b n\b nu\b l\=a\b n} Aiyya\index{cec}{Aiyya} Nampi\index{cec}{Nampi} U\d taiy\=a\b n, d'Aiyya Nampi\index{cec}{Nampi}, un propriétaire [terrien] de Tiruv\=aykkulam, et d'Aiyya Nampiteva\b n\index{cec}{Nampi}. De cette terre\index{gnl}{terre} de verger, devenu mienne, la limite ouest est l'est de\dots, la limite nord est [le sud de]\dots\ de la K\=averi\dots\ Tiruve\.nk\=a\d tu\index{gnl}{Venkatu@Ve\.nk\=a\d tu!Tiruve\.nk\=a\d tu}\dots; ainsi sont les quatre grandes limites. \dots\ [ayant d\'eduit] deux \textit{m\=acci\b n\b nam} de la terre\index{gnl}{terre} d'\=Atitta\b n\index{cec}{atittan@\=Atitta\b n} N\=ar\=aya\d na\b n\index{cec}{narayanan@N\=ar\=aya\d na\b n} Uyyako\d n\d t\=a\b n\index{cec}{uyyakkontan@Uyyakko\d n\d t\=a\b n}, trois \textit{m\=a} de la terre\index{gnl}{terre} achet\'ee au prix fix\'e par Ir\=ajar\=aja dont jouit Aiyya\index{cec}{Aiyya} Nampi\index{cec}{Nampi} U\d taiy\=a\b n, et huit \textit{m\=a} de la terre\index{gnl}{terre} achet\'ee jadis en tant que verger d'ar\'equiers, aupr\`es d'eux\footnote{Il est difficile de d\'efinir s'il s'agit d'un singulier honorifique renvoyant \`a Aiyya Nampi\index{cec}{Nampi} U\d taiy\=a\b n ou d'un pluriel d\'esignant Atitta\b n N\=ar\=aya\d na\b n\index{cec}{narayanan@N\=ar\=aya\d na\b n} Uyyako\d n\d t\=a\b n\index{cec}{uyyakkontan@Uyyakko\d n\d t\=a\b n} et Aiyya Nampi\index{cec}{Nampi} U\d taiy\=a\b n.} par T\=ayilum Nall\=a\b n, un propriétaire [terrien] de Ka\d latt\=ur\index{cec}{kalattur@Ka\d latt\=uru\d taiy\=a\b n}, soit ayant d\'eduit une demi [\textit{v\=eli}] et trois \textit{m\=acci\b n\b nam} de terre\index{gnl}{terre}, [puis de] la terre\index{gnl}{terre} restante d'une demi [\textit{v\=eli}], qui est mienne, sont inclus les [droits\index{gnl}{droits (\textit{k\=a\d ni})} sur] les \textit{op\=ati}\footnote{Du sk. \textit{up\=adhi}, taxe pr\'elev\'ee sur les propri\'etaires, \textsc{Subbarayalu} (*2001f [1984]: 61).}, les terres\index{gnl}{terre} au bord des rivi\`eres, les \textit{pa\d tikai}, les terres\index{gnl}{terre} communes, les \textit{pot\=ari}, les puits, les bassins et comprenant toutes autres sortes de droits\index{gnl}{droits (\textit{k\=a\d ni})}. De cette terre\index{gnl}{terre}, ayant fait une propri\'et\'e au nom de \'Siva\index{gnl}{Siva@\'Siva} et ayant pay\'e les taxes, et pour les taxes suppl\'ementaires, moi, Utaiya\~nceyt\=a\b n\index{cec}{utaiyanceytan@Utaiya\~nceyt\=a\b n} T\=a\b li\index{cec}{tali@T\=a\b li} alias Co\d lentiraci\.nka\index{cec}{colentira@C\=o\d lentiraci\.nka} Vi\b lupparaya\b n\index{cec}{vilupparayan@Vi\b lupparaya\b n}, un propriétaire [terrien] de Karupp\=ur\index{cec}{Karuppurutaiyan@Karupp\=uru\d taiy\=a\b n}, \'eternellement et une fois par jour, tant que durent lune et soleil, en tant que service\index{gnl}{service} \'eternel au temple\index{gnl}{temple}, je donne deux cents noix d'arec et six bottes de feuilles [de b\'etel] de T\=ev\=ur. Ceci est sous la protection des \textit{\'Sr\=\i m\=ahe\'svara}\index{cec}{srimahesvara@\textit{\'sr\=\i mahe\'svara} dévot, surveillant}.


\section*{CEC 3}
\subsection*{CEC 3.1 Remarques}

L'inscription a \'et\'e relev\'ee \`a deux reprises par l'ASI (ARE 1896 125 et ARE 1918 365) qui la date de la neuvi\`eme ann\'ee de Tribhuvanacakravartin Kulottu\.nga C\=o\b ladeva `who took Madura'. \textsc{Mahalingam} (1992: 550, Tj. 2411) identifie ce roi\index{gnl}{roi} comme Kulottu\.nga III\index{gnl}{Kulottu\.nga III} et date le texte de \textbf{1187}. Selon leurs r\'esum\'es, il s'agit d'une donation de terre\index{gnl}{terre} pour offrir des lampe\index{gnl}{lampe}s au temple\index{gnl}{temple}. Le texte fait r\'ef\'erence au cadastre effectu\'e la seizi\`eme ann\'ee de règne du Kulottu\.nga qui a aboli les douanes.

Le texte a \'et\'e publi\'e dans SII 5 990. Mal localis\'ee sur le mur sud du temple\index{gnl}{temple} principal dans ARE 1896 125, que reprend SII 5 990, l'\'epigraphe se trouve, en r\'ealit\'e, sur toute la longueur (onze m\`etres et vingt centim\`etres) d'un \'el\'ement saillant du soubassement nord du \textit{ma\d n\d dapa} devant le temple\index{gnl}{temple} principal de \'Siva\index{gnl}{Siva@\'Siva}. Son \'edition est fond\'ee sur un examen de la pierre, de clich\'es (E. \textsc{Francis}), de la transcription de l'ASI et de la publication dans SII. Situé sous deux becs d'\'evacuation, le texte est marqu\'e par deux espaces aux trois premi\`eres lignes car le lapicide n'y a vraisemblablement pas eu acc\`es. Ils sont not\'es par \og $\Psi$\fg.

\subsection*{CEC 3.2 Texte}
\begin{enumerate}

	\item \textbf{svasti \'sr\=\i} U tiripuva\b naccakkaravatti(ka)\d l\index{cec}{Tribhuvanacakravarti} maturai\index{cec}{Maturai} ko\d n\d taru\d li\b na \textbf{\'sr\=\i}kulottu\.nka-co\b la\textbf{de}varkku\index{cec}{kulottungacoladevar@Kulottu\.ngac\=o\b ladeva} y\=a\d n\d tu 9 \=avatu n\=a\d l 100 7 10 6 l antar\=ayam\index{cec}{antarayam@\textit{antar\=ayam} taxe en argent} p\=a\d t\d tam\index{cec}{pattam@\textit{p\=a\d t\d tam} taxe en argent} u\d lpa\d ta tirunu\b nt\=avi\d lakkup (pu\b ra)\index{cec}{puram@\textit{pu\b ram} terre de donation!\textit{nu\b nt\=avi\d lakkuppu\b ram} terre donnée pour allumer une lampe perpétuelle} i\b raiyili\index{cec}{iraiyili@\textit{i\b raiyili} non imposable} i\d t\d ta ir\=a\textbf{j\=adhi}r\=a\textbf{\textbf{ja}}va\d lan\=a\d t\d tut tirukka\b lumalat-tu\index{cec}{Tirukkalumalam@Tirukka\b lumalam} u\d taiy\=ar tiruto$\Psi$\d nipuramu\d taiy\=arkum\index{cec}{Tiruttoni@Tirutt\=o\d nipuramu\d taiya n\=aya\b n\=ar, \'Siva} periyan\=acciy\=arkum\index{cec}{periyanacciyar@Periyan\=acciy\=ar} tirunu\b nt\=avi\d lakkup pu\b ra i\b raiyiliy\=aka\index{cec}{iraiyili@\textit{i\b raiyili} non imposable} k\=a\d niy\=a\d lar\index{cec}{kani@\textit{k\=a\d ni} droit, propriété} nilai\index{cec}{nilam@\textit{nilam} terre} ni\b n\b ru payir ce(y)tu ka\d tamai\index{cec}{katamai@\textit{ka\d tamai} taxe foncière} i\b r\=atu\index{cec}{iruttu@\textit{i\b ruttu} payer un impôt!\textit{i\b r\=atu} négation} pa\b r\b riliy\=ayi-varka\d lukku i\b raippu\d naip pa\d t\d ta karaippa\d taiyil\=arkku\footnote{SII 5 990 ne lit pas le redoublement de la gutturale tamoule.} ta\d n\d tal n\=ayakam araiya\.n\index{cec}{Araiya\b n} (ka)\d talko\d lamitant\=a\b n\=ana\index{cec}{katalkola@Ka\d talko\d lamitant\=a\b n} amarako\b n\index{cec}{Amarako\b n} pal(la)$\Psi$varaiyanum\index{cec}{Pallavar\=aya\b n} araiyan tiruna\d t\d tam\=a-\d tiy\=ana\index{cec}{Tirunattamati@Tiruna\d t\d tam\=a\d ti} vetavanan\=ayaka(p\index{cec}{vetavananayaka@Vetavanan\=ayaka} pa)llavaraiyanu(m cantan ko)van\=ana\index{cec}{Kova\b n} tirucci\b r\b ram-pala\index{cec}{Tirucirrampala@Tirucci\b ra\b rampala} vi\b lupparaiyanum\index{cec}{vilupparayan@Vi\b lupparaya\b n} ca\b nta\b n\index{cec}{cantan@Ca\b nta\b n}
	\item kul\=ava\b n\=ana\index{cec}{Kul\=ava\b n}\footnote{Ce nom est parfaitement lisible sur la pierre bien que la publication utilise des crochets.} ci\.nka\d l\=a\b ntakap\index{cec}{cinkalantaka@Ci\.nka\d l\=antaka} pallavaraiyanum\index{cec}{Pallavar\=aya\b n} ce\.nka\d nm\=al\index{cec}{cenkanmal@Ce\.nka\d nm\=al} p(e)riy\=an\=ana\index{cec}{periyan@Periy\=a\b n} nampiy\=ar\=urp\index{cec}{Nampiy\=ar\=ur} pallavaraiyanum\index{cec}{Pallavar\=aya\b n} pakkal\index{cec}{pakkal@\textit{pakkal} auprès de} ikkoyil\index{cec}{koyil@\textit{k\=oyil} temple} \=atica\d n\d te\textbf{\'sva}ra\textbf{de}var\index{cec}{adicandesvara@\=Adica\d n\d de\'svaradeva} tirun\=amattu vilai ko\d n\d ta\index{cec}{vilai@\textit{vilai} prix!vilai-kol@\textit{vilai-ko\d l} acheter} piram\=a\d na(p)pa\d ti\index{cec}{piramanam@\textit{pram\=a\d nam} document} k\=a[\d ni m\=a\b ri\b na ve\d n]\footnote{Le d\'ecollement de la pierre en cet endroit ne permet pas de lire la le\c con du texte publi\'e. Cependant, le nombre d'\textit{ak\d sara} manquants, environ huit, laisse l'accepter.}\d naiy\=urn\=a\d t\d tu nakka\b np\=a\d tiy\=a\b na\index{cec}{Nakka\b np\=a\d ti} a\b lakiyar\=amapa\d tinam\index{cec}{alakiya@A\b lakiyar\=amapa\d t\d ti\b nam}\footnote{\textit{a\b lakiyar\=amapa\d tinam}: la publication s\'epare les mots autrement, \textit{a\b lakiyar\=amapa\d ti nam}.} u\d taiy\=ar cu\.nka(n)tavuttaru\d lina $\Psi$ kulottu\.nkaco\b la\textbf{de}-varkku\index{cec}{kulottungacoladevar@Kulottu\.ngac\=o\b ladeva} 10 (6 \=a)vatu tiruvulaka\d lanta ka\d nakkuppa\d ti ni\.nkal ni(k)ki [pa\b nai ni\b n\b ra kollai]\footnote{\textit{pa\b nai ni\b n\b ra kollai}: le nombre d'\textit{ak\d sara} manquants, environ douze\index{gnl}{douze}, converge vers la le\c con de SII 5 990.}(yu)m uvarum\index{cec}{uvar@\textit{uvar} marais salant} uppuma\d n\d num \=ay (a)\b ruke(\b lu)\b nta pa\d tuta\b raiyum e\b n\b ru (a\d lanta) nilattu\index{cec}{nilam@\textit{nilam} terre} nel payirum pu\b n pa(yiru)m ceytum maranilaiyum (\=a)y varu-ki\b ra nilam\index{cec}{nilam@\textit{nilam} terre} u\d lpa\d ta ce[yyal\=am payir cetu]\footnote{\textit{ce[yyal\=am payir cetu]}: le nombre d'\textit{ak\d sara} manquants, environ douze\index{gnl}{douze}, permet d'accepter la le\c con du texte publi\'e.}$\Psi$m maram\=akkiyum u..\d t\d tu\footnote{La voyelle initiale \textit{i} pr\'esent\'ee dans \textit{i\d t\d tu} par la publication n'est pas reprise car elle n'est pas lisible et elle ne formerait pas sens avec les \'el\'ements lus.} tirununt\=avi\d lakku erikka i\d tuki\b ra nila(me\b n)\b nu\d taiya\index{cec}{nilam@\textit{nilam} terre} ... ...kku\footnote{La conjecture \textit{k\=a\d nikku} propos\'ee par \textsc{G. Vijayavenugopal} n'est pas suivie ici car le nombre d'\textit{ak\d sara} qui semblent manquer, six, est trop grand.} (a)\d taipp\=ay varuki\b ra-pa\d tiyum tavir\b ntu y\=a\d n[\d tu 9]\footnote{\textit{y\=a\d n[\d tu 9]}: reconstitution des deux \textit{ak\d sara} manquants \`a partir du texte publi\'e.} \=avatu pac\=anam\index{cec}{pacanam@\textit{pac\=anam} moisson} mutal a\b nta
	\item r\=ayam\index{cec}{antarayam@\textit{antar\=ayam} taxe en argent} p\=a\d t\d tam\index{cec}{pattam@\textit{p\=a\d t\d tam} taxe en argent} u\d lpa\d ta tirunu\b nt\=avi\d lakkup pu\b ra\index{cec}{puram@\textit{pu\b ram} terre de donation!\textit{nu\b nt\=avi\d lakkuppu\b ram} terre donnée pour allumer une lampe perpétuelle} i\b raiyili\index{cec}{iraiyili@\textit{i\b raiyili} non imposable} i\d t\d ta ni(la)m 5 10 7 4 M Q E 1/2 M AA P 1000 7 100 10 AA\footnote{Nous pensons qu'il s'agit ici de calculer une taxe en paddy par rapport \`a la superficie de la terre\index{gnl}{terre} donn\'ee. La lecture \textit{in situ} ne suit pas SII 5 990. Cependant, notre version est inexacte: les deux graph\`emes pr\'ec\'edant l'abr\'eviation pour paddy (not\'e P), ressemblant \`a \textit{\b n\=al}, et les deux graph\`emes finaux pr\'ec\'edant \textit{y\=a\d n\d tu}, identiques \`a \textit{\d lam}, sont des abr\'eviations non identifi\'ees.} y(\=a)\d n\d tu 9 \=avatu pac\=anam\index{cec}{pacanam@\textit{pac\=anam} moisson} mutal antar\=ayam\index{cec}{antarayam@\textit{antar\=ayam} taxe en argent} p\=a\d t\d tam\index{cec}{pattam@\textit{p\=a\d t\d tam} taxe en argent} u\d lpa\d ta tirunu\b nt\=avi\d lakkup pu\b ra\index{cec}{puram@\textit{pu\b ram} terre de donation!\textit{nu\b nt\=avi\d lakkuppu\b ram} terre donnée pour allumer une lampe perpétuelle} i\b raiyili\index{cec}{iraiyili@\textit{i\b raiyili} non imposable} i\d t\d tamaikku i(vv\=ur\footnote{\textit{ivv\=ur}: la lecture \textit{in situ} ne correspond pas \`a celle du texte publi\'e \textit{[t]e[var]}.} 16 \=avatu) a\d lavil ka\d talil tirai e\b riv\=ay aruku ilipp\=u(\d tu)m ir\=ava$\Psi$\d na\b n merv\=ayum (e)\b lu\b ntu ma\d nal ku\b n\b r\=ana nilamum\index{cec}{nilam@\textit{nilam} terre} pa\d t\d tinavar ku\d ti irupp\=ana nilamum\=ay\index{cec}{nilam@\textit{nilam} terre} (ni)\.nka[l\=ana nir..\d ni](lattu)\footnote{\textit{(ni)\.nka[l\=ana nir..\d ni](lattu)}: le nombre d'\textit{ak\d sara} manquants, environ huit, laisse accepter la conjecture de la publication.} pa\b naiyum iluppayum u\d l\d li\d t\d tu maram\=ak-kal\=am nilam\index{cec}{nilam@\textit{nilam} terre} \=akki ituvum tirunu\b nt\=avi\d lakkukku u\d tal\=avatu ivai puravuvari\index{cec}{vari@\textit{vari} taxe!\textit{puravuvari} officier des impôts} cika-ra\d na \b n\=a(ya)kam panta\d nainall\=ur\index{cec}{Pantanai@Pa\d n\d ta\d nainalluru\d taiy\=a\b n} u\d tai(y\=a\b n) e\b luttu $\Psi$ ivai (puravuvari\index{cec}{vari@\textit{vari} taxe!\textit{puravuvari} officier des impôts} cikara\d na) \b n\=ayakam [pirimaya\b n e\b lu]\footnote{\textit{[pirimaya\b n e\b lu]ttu}: lecture de la publication compte tenu des \textit{ak\d sara} manquants, environ 6.}ttu--\footnote{Ponctuation figurant telle quelle sur la pierre.}ivai v\=a\b na(va\.ncikara\d na\.n... v\=ur ki\b la)va\b n e\b lut-tu\footnote{\textit{ivai v\=a\b na(va\.ncikara\d na\.n... v\=ur ki\b la)va\b n e\b luttu}: le texte publi\'e, qui propose \textit{ivai puravuvari\index{cec}{vari@\textit{vari} taxe!\textit{puravuvari} officier des impôts} c\=\i kara\d nan\=ayakam .... ki\b lava\b n--e\b luttu}, est tr\`es loin de ce qui peut \^etre lu sur la pierre.}--ivai pu(ravuvari\index{cec}{vari@\textit{vari} taxe!\textit{puravuvari} officier des impôts} ci)kara\d na n\=a(ya)kam \=ar\=ur\index{cec}{arurutaiyan@\=Ar\=uru\d taiy\=a\b n} u\d taiy\=an e\b luttu
	\item ivai puravuvari\index{cec}{vari@\textit{vari} taxe!\textit{puravuvari} officier des impôts} cikara\d nattu\footnote{SII 5 990 a omis \textit{ci} dans \textit{cikara\d nattu}.} mukave\d t\d ti\index{cec}{mukavetti@\textit{mukave\d t\d ti} officier qui pose un sceau} kuruk\=a\d ti\index{cec}{Kuruk\=a\d ti} ki\b l\=a\b n e\b luttu--ivai puravuvari\index{cec}{vari@\textit{vari} taxe!\textit{puravuvari} officier des impôts} cikara\d nattu mukave\d t\d ti\index{cec}{mukavetti@\textit{mukave\d t\d ti} officier qui pose un sceau} kum\=arama\.nkalamu\d taiy\=a\b n\index{cec}{Kum\=arama\.nkalamu\d taiy\=a\b n} e\b luttu--ivai puravuvari\index{cec}{vari@\textit{vari} taxe!\textit{puravuvari} officier des impôts} cikara\d nattu mukave\d t\d ti\index{cec}{mukavetti@\textit{mukave\d t\d ti} officier qui pose un sceau} peruma\.nkalamu\d taiy\=a\b n\index{cec}{Peruma\.nkalamu\d taiy\=a\b n} e\b luttu--ivai (puravuvari)\index{cec}{vari@\textit{vari} taxe!\textit{puravuvari} officier des impôts} cika-(ra\d nat)tu mukave\d t\d ti\index{cec}{mukavetti@\textit{mukave\d t\d ti} officier qui pose un sceau} mel\=ur u\d taiy\=a\b n\index{cec}{Mel\=uru\d taiy\=a\b n} e\b luttu--ivai ce\b rako\b n e\b luttu--ivai kurukular\=aya\b n\index{cec}{Kurukular\=aya\b n} e\b luttu--ivai co\b lavicc\=atirap\index{cec}{colavicca@Co\b lavicc\=atira} pallavaraiya\b n\index{cec}{Pallavar\=aya\b n} e\b luttu--ivai vil\=a\d tattaraya\b n\index{cec}{vilata@Vil\=a\d tattaraya\b n} e\b luttu--ivai pa\.nka\d lar\=aya\b n\index{cec}{pankala@Pa\.nka\d lar\=aya\b n} e\b luttu--ivai meln\=a\d t\d taraya\b n\index{cec}{Meln\=a\d t\d taraya\b n} e\b luttu--ivai vec\=alipparaya\b n\index{cec}{Vec\=alipparaya\b n} e\b luttu--ivai v\=a\d luvar\=aya\b n\index{cec}{valuva@V\=a\d luvar\=aya\b n} e\b luttu--ivai ........ (e\b luttu)--i(vai ...\b n\=atapiriya\b n e\b luttu A\footnote{Probablement un signe de ponctuation finale qui ressemble \`a l'\textit{ak\d sara} \textit{\d la}.})\footnote{Il est difficile d'accepter la le\c con de la publication (\textit{ivai amarako\b n\index{cec}{Amarako\b n} e\b luttu--ivai kuruku[lat]taraya\b n e\b luttu--ivai ...pa\b na e\b luttu}) car les \textit{ak\d sara} manquants et la structure ne correspondent pas \`a ce qui peut \^etre lu \textit{in situ}.}--
\end{enumerate}

\subsection*{CEC 3.3 Traduction}
Que la prosp\'erit\'e soit ! En la 9\up{e} ann\'ee, le 176\up{e} jour [du r\`egne] de \'Sr\=\i kulottu\.nga-c\=o\b ladeva\index{cec}{kulottungacoladevar@Kulottu\.ngac\=o\b ladeva} qui fit la gr\^ace de prendre Maturai\index{cec}{Maturai}\footnote{Cette relative, version br\`eve de \textit{maturaiyum\index{cec}{Maturai} p\=a\d n\d tiya\b n mu\d tittalaiyum ko\d n\d taru\d li\b na}, qui appara\^it d\`es la deuxi\`eme ann\'ee de r\`egne de Kulottu\.nga III\index{gnl}{Kulottu\.nga III}, fait r\'ef\'erence \`a la campagne victorieuse du roi \textit{c\=o\b la}\index{gnl}{cola@\textit{c\=o\b la}} contre les P\=a\d n\d dya, \textsc{Nilakanta Sastri} (*2000 [1955]: 377).}, empereur des trois mondes; la terre\index{gnl}{terre} donn\'ee comme non imposable\footnote{Voir \textsc{Nilakanta Sastri} (*2000 [1955]: 534-6).}, incluant [les taxes] \textit{antar\=ayam} et \textit{p\=a\d t\d tam}\index{cec}{pattam@\textit{p\=a\d t\d tam} taxe en argent}\footnote{\textsc{Subramaniam} (1957) d\'efinit \textit{antar\=ayam}\index{cec}{antarayam@\textit{antar\=ayam} taxe en argent} comme une taxe pr\'elev\'ee par le corps local\index{gnl}{local} et \textit{p\=a\d t\d tam} comme une taxe ou un loyer qui toucherait l'industrie ou la profession. Et, suivi par \textsc{Sircar} (1966), il lit aussi ces termes ensemble dans la m\^eme entr\'ee que \textit{antar\=ayakk\=a\'su}\index{cec}{antarayam@\textit{antar\=ayam} taxe en argent} et glose \og internal taxes, minor taxes like the profession tax, etc. payable to the village assembly\fg. La fr\'equente absence de \textit{sandhi} entre ces termes (CEC 3, ARE 1918 361 l.~28 et 57, SII 5 663 l.~6, SII 6 44 l.~6, 456 l.~42, SII 17 730 l.~6) et l'usage de \textit{p\=a\d t\d tam}\index{cec}{pattam@\textit{p\=a\d t\d tam} taxe en argent} seul (SII 7 454 l.~7) laissent penser qu'il s'agit de deux taxes distinctes. Cependant, \textit{p\=a\d t\d tam}\index{cec}{pattam@\textit{p\=a\d t\d tam} taxe en argent} \'etant souvent pr\'ec\'ed\'e dans l'\'enum\'eration des taxes d'\textit{antar\=ayam}\index{cec}{antarayam@\textit{antar\=ayam} taxe en argent}, telle une formule, nous pensons qu'il existe une affinit\'e entre ces deux taxes. Elles sont souvent pay\'ees en argent, \textsc{Appadorai} (1936: 695) et \textsc{Heitzman} (*2001 [1997]: 166-7) ou en nature, \textsc{Veluthat} (1993: 147).}, pour [offrir/entretenir] une lampe\index{gnl}{lampe} perp\'etuelle\footnote{Le suffixe \textit{pu\b ram} d\'esigne une terre\index{gnl}{terre} non imposable donn\'ee au service\index{gnl}{service} d'une institution religieuse (\textsc{Subramaniam} 1957 et \textsc{Subbarayalu} 2003). Ainsi, \textit{nantava\b napu\b ram}\index{cec}{puram@\textit{pu\b ram} terre de donation!\textit{nantava\b nappu\b ram} terre donnée pour créer un jardin à fleur} est une terre\index{gnl}{terre} destin\'ee au jardin \`a fleurs (CEC 10 et 11), \textit{p\=a\b rponakapu\b ram}\index{cec}{puram@\textit{pu\b ram} terre de donation!\textit{p\=a\d rponakappu\b ram} terre donnée pour offrir du riz au lait}, \`a offrir du riz\index{gnl}{riz} au lait\index{gnl}{lait} (CEC 25 et 28), \textit{ma\d tapa\d l\d lipu\b ram}\index{cec}{puram@\textit{pu\b ram} terre de donation!\textit{ma\d tapa\d l\d lippu\b ram} terre donnée pour la cuisine}, \`a la cuisine (CEC 27) et \textit{ma\d tapu\b ram}\index{cec}{puram@\textit{pu\b ram} terre de donation!\textit{ma\d tappu\b ram} terre donnée pour le monastère}, au monast\`ere\index{gnl}{monastère} (CEC 17 et 18).} au Seigneur propri\'etaire de Tirutt\=o\d nipuram\index{gnl}{Tonipuram@T\=o\d nipuram!Tirutt\=o\d nipuram} \`a Tirukka\b lumalam\index{cec}{Tirukkalumalam@Tirukka\b lumalam} dans Ir\=aj\=adhir\=ajava\d lan\=a\d tu\index{cec}{Rajadhirajavala@R\=aj\=adhir\=ajava\d lan\=a\d tu} et \`a Periyan\=acciy\=ar\index{cec}{periyanacciyar@Periyan\=acciy\=ar}, en tant que terre\index{gnl}{terre} non imposable pour une lampe\index{gnl}{lampe} perp\'etuelle [est] devenue une terre\index{gnl}{terre} incultiv\'ee [car] les ayant-droits\index{gnl}{droits (\textit{k\=a\d ni})}, qui l'occupaient et la cultivaient, n'avaient pas pay\'e la taxe \textit{ka\d tamai}\index{cec}{katamai@\textit{ka\d tamai} taxe foncière}\footnote{Une taxe fonci\`ere; sur les taxes dans les inscriptions \textit{c\=o\b la} voir \textsc{Karashima} (*2001c [1972]) et \textsc{Subbarayalu} (*2001f [1984]).}.

[Je], un propri\'etaire [terrien] de Nakka\b np\=a\d ti alias A\b lakiyar\=amapa\d t\d tinam dans Ve\b n\b naiy\=urn\=a\d tu\footnote{Situ\'ee dans le R\=aj\=adhir\=ajava\d lan\=a\d tu\index{cec}{Rajadhirajavala@R\=aj\=adhir\=ajava\d lan\=a\d tu}, au nord de Tirukka\b lumalan\=a\d tu\index{cec}{Tirukka\b lumalan\=a\d tu}, cette division territoriale \'etait travers\'ee en son centre par le Ko\d l\d li\d tam, bras de la K\=av\=eri, qui se jette dans la mer\index{gnl}{mer}; \textsc{Subbarayalu} (1973, carte 10).}, ai acquis les droits\index{gnl}{droits (\textit{k\=a\d ni})}\footnote{La transaction effectu\'ee n'est absolument pas claire. Est-ce que le donateur r\'ecup\`ere la terre\index{gnl}{terre} confisqu\'ee (\textit{k\=a\d nim\=a\b ri\b na}\index{cec}{kani@\textit{k\=a\d ni} droit, propriété}, \textsc{Subramaniam} 1957 et \textsc{Heitzman} *2001 [1997]: 156) ou uniquement des droits qu'il exerce \`a partir d'une nouvelle terre\index{gnl}{terre} qu'il donne\string?}, selon le document de l'achat au nom d'\=Adica\d n\d de\'svaradeva\index{cec}{adicandesvara@\=Adica\d n\d de\'svaradeva}
%\footnote{XXXXX\=Adica\d n\d de\'svaradeva le comptable, cf. ARE 1917 227 de Kurukkai}
de ce temple\index{gnl}{temple}, aupr\`es de [ceux qui] s'\'etaient port\'es garants pour eux (\textit{i.e.} \textit{k\=a\d niy\=alar}):
[aupr\`es d']Araiyan\index{cec}{Araiya\b n} Ka\d talko\d lamitant\=a\b n\index{cec}{katalkola@Ka\d talko\d lamitant\=a\b n} alias Amarako\b n\index{cec}{Amarako\b n} Pallavaraiyan\index{cec}{Pallavar\=aya\b n}, \textit{ta\d n\d tal n\=ayakam}\footnote{Un officier\index{gnl}{officier} militaire; \textsc{Subramaniam} (1957), \textsc{Veluthat} (1993: 91) et \textsc{Subbarayalu} (2003).} pour les \textit{karaippa\d taiyil\=ar}\footnote{Litt\'eralement \og ceux de l'arm\'ee c\^oti\`ere\fg. Ainsi, nous supposons que Araiyan\index{cec}{Araiya\b n} Ka\d talko\d lamitant\=a\b n\index{cec}{katalkola@Ka\d talko\d lamitant\=a\b n} alias Amarako\b n\index{cec}{Amarako\b n} Pallavaraiyan\index{cec}{Pallavar\=aya\b n} est un responsable militaire de cette branche de l'arm\'ee.}, d'Araiyan\index{cec}{Araiya\b n} Tiruna\d t\d ta-m\=a\d tiy\=a\b n alias Vetavanan\=ayaka Pallavaraiyan\index{cec}{Pallavar\=aya\b n}, de Cantan Kovan alias Tirucci\b r\b ram-pala\index{cec}{Tirucirrampala@Tirucci\b ra\b rampala} Vi\b lupparaiyan\index{cec}{vilupparayan@Vi\b lupparaya\b n}, de Ca\b nta\b n\index{cec}{cantan@Ca\b nta\b n} Kul\=ava\b n alias Ci\.nka\d l\=a\b ntaka\index{cec}{cinkalantaka@Ci\.nka\d l\=antaka} Pallavaraiyan\index{cec}{Pallavar\=aya\b n}, et de Ce\.nka\d nm\=al Periy\=an\index{cec}{periyan@Periy\=a\b n} alias Nampiy\=arur Pallavaraiyan\index{cec}{Pallavar\=aya\b n}.

Selon les comptes qui ont permis d'établir le cadastre du \og territoire consacr\'e\fg\ (\textit{tiruvulaka\d lanta ka\d nakkuppa\d ti}) la 16\up{e} ann\'ee [de r\`egne] du Kulottu\.ngac\=o\b ladeva\index{cec}{kulottungacoladevar@Kulottu\.ngac\=o\b ladeva}\footnote{Sur l'identification de ce roi\index{gnl}{roi} comme Kulottu\.nga I\index{gnl}{Kulottu\.nga I}; cf. \textsc{Nilakanta Sastri} (*2000 [1955]: 331-2).} qui a an\'eanti les douanes\footnote{Pour un compte rendu des diff\'erents cadastres effectu\'es sous les C\=o\b la\index{gnl}{Cola@C\=o\b la}; cf. \textsc{Subbarayalu} (1973: 67-8) et \textsc{Veluthat} (1993: 104, n.~141). Et pour d'autres occurrences du cadastre de la seizi\`eme ann\'ee de r\`egne du Kulottu\.nga I\index{gnl}{Kulottu\.nga I}; cf. SII 6 34 l.~7; SII 23 289 l.~3, 23 483 l.~6-7; ARE 1900 paragraphe 25; ARE 1912 440, 1913 66, 1910 52 et 98.},
la terre\index{gnl}{terre} donn\'ee pour allumer une lampe\index{gnl}{lampe} perp\'etuelle --- ayant retir\'e ce qui est \`a retirer, faisant du \textit{nelpayir} et \textit{pu\b npayir}\footnote{\textit{nelpayir} serait la culture des graines et \textit{pu\b npayir} celle des l\'egumineuses.} sur les terres\index{gnl}{terre} cadastr\'ees comme palmeraies, comme [marais] salants, comme terres\index{gnl}{terre} sal\'ees et comme terres\index{gnl}{terre} difficiles o\`u s'\'el\`eve l'\textit{a\b ruku}, faisant aussi les \textit{payir} faisables sur la terre\index{gnl}{terre} qui vient comme terre\index{gnl}{terre} \`a arbres, faisant des palmeraies et exluant la portion qui vient comme limite\dots\ ma\dots --- est [cette] terre\index{gnl}{terre} donn\'ee comme non imposable pour [entretenir] une lampe\index{gnl}{lampe} perp\'etuelle, incluant [les taxes suivantes] \textit{antar\=ayam}\index{cec}{antarayam@\textit{antar\=ayam} taxe en argent} et \textit{p\=a\d t\d tam}\index{cec}{pattam@\textit{p\=a\d t\d tam} taxe en argent}, \`a partir de la moisson de la 9\up{e} ann\'ee [de r\`egne], [terre\index{gnl}{terre} qui vaut] 1710 [\textit{kalam}]\index{cec}{kalam@\textit{kalam} unité de mesure du paddy} de paddy pour 57 [\textit{v\=eli}] 4 \textit{m\=a muntiri k\=\i \b l} demi \textit{m\=a}\footnote{Le terme \textit{v\=eli} n'appara\^it pas dans le texte mais il est d'usage de l'omettre quand il est suivi par une plus petite mesure. La valeur d'une terre\index{gnl}{terre} (\textsc{Subbarayalu} *2001g [?]), \'etablie sur sa nature, son emplacement, sa productivit\'e, etc., est g\'en\'eralement exprim\'ee en \textit{kalam}\index{cec}{kalam@\textit{kalam} unité de mesure du paddy} par \textit{v\=eli} (k/v). La moyenne serait de cent k/v. La terre\index{gnl}{terre} donn\'ee ici n'est pas tr\`es productive, environ trente k/v. En effet, elle est saline et n\'ecessite des transformations pour la cultiver. Nous supposons qu'elle se trouve dans la localit\'e du donateur, sur la c\^ote dans le Ve\d n\d naiy\=urn\=a\d tu, en zone portuaire comme l'indique le suffixe toponymique -\textit{pa\d t\d ti\b nam}, \textit{TL} (-\textit{pa\d ti\b nam} dans le texte). Sur les dons\index{gnl}{don} de terres\index{gnl}{terre} en friche aux temples\index{gnl}{temple} pour favoriser l'expansion agraire; cf. \textsc{Heitzman} (*2001 [1997]: 107).}.

Pour l'exemption de la terre\index{gnl}{terre} pour [entretenir] une lampe\index{gnl}{lampe} perp\'etuelle incluant les taxes comme \textit{antar\=ayam}\index{cec}{antarayam@\textit{antar\=ayam} taxe en argent} et \textit{p\=a\d t\d tam}\index{cec}{pattam@\textit{p\=a\d t\d tam} taxe en argent}, \`a partir de la moisson de la 9\up{e} ann\'ee [de r\`egne], selon le cadastre de la 16\up{e} ann\'ee de cette ville, [je donne] la terre\index{gnl}{terre} devenue colline de sable o\`u poussent l'\textit{\=\i lipp\=u\d tu}, l'\textit{ir\=ava\d na\b n} et le \textit{merv\=ay}\footnote{Probablement des v\'eg\'etaux.} pr\`es de l'embouchure qui renvoie les vague\index{gnl}{vague}s dans la mer\index{gnl}{mer}, \dots\ la terre\index{gnl}{terre} o\`u demeurent les p\^echeurs \dots, et faisant d'[elle] une terre\index{gnl}{terre} \`a faire des arbres incluant les palmier\index{gnl}{palmier}s et les \textit{iluppai}. Elle forme aussi le capital de la terre\index{gnl}{terre} pour [entretenir] la lampe\index{gnl}{lampe} perp\'etuelle.

Ceci [est l\'egalis\'e par] la signature d'un propriétaire [terrien] de Panta\d nainall\=ur, \textit{puravuvari\index{cec}{vari@\textit{vari} taxe!\textit{puravuvari} officier des impôts} cikara\d na \b n\=ayakam}; ceci [est l\'egalis\'e par] la signature de Pirimaya\b n, \textit{puravuvari\index{cec}{vari@\textit{vari} taxe!\textit{puravuvari} officier des impôts} cikara\d na \b n\=ayakam}; ceci [est l\'egalis\'e par] la signature du \textit{ki\b lava\b n}\footnote{Litt\'eralement \og ancien\fg, ce titre, qui aurait \'et\'e port\'e \`a l'origine\index{gnl}{origine} par les chef\index{gnl}{chef}s des quartiers brahmane\index{gnl}{brahmane}s (\textsc{Champakalakshmi} *2004 [2001]: 63 et \textsc{Karashima} *2001a [1966]: 6), semble s'appliquer aux leaders (\textsc{Nilakanta Sastri} *2000 [1955]: 464) d'une division territoriale ou aux propri\'etaires terriens (\textsc{Subbarayalu} 2003, s.v.). Son sens serait assez proche de celui d'\textit{u\d taiy\=a\b n}.} de\dots\ V\=a\b nava\.ncikara\d na\.n\dots; ceci [est l\'egalis\'e par] la signature d'un propriétaire [terrien] d'\=Ar\=ur\index{cec}{arurutaiyan@\=Ar\=uru\d taiy\=a\b n}\index{gnl}{Ar\=ur@\=Ar\=ur}, \textit{puravuvari\index{cec}{vari@\textit{vari} taxe!\textit{puravuvari} officier des impôts} cikara\d na n\=ayakam}; ceci [est l\'egalis\'e par] la signature du \textit{ki\b l\=a\b n}\footnote{Vraisemblablement une variante de \textit{ki\b lava\b n}.} de Kuruk\=a\d ti, \textit{puravuvari\index{cec}{vari@\textit{vari} taxe!\textit{puravuvari} officier des impôts} cikara\d nattu mukave\d t\d ti}\index{cec}{mukavetti@\textit{mukave\d t\d ti} officier qui pose un sceau}; ceci [est l\'egalis\'e par] la signature d'un propriétaire [terrien] du Kum\=arama\.nkalam, \textit{puravuvari\index{cec}{vari@\textit{vari} taxe!\textit{puravuvari} officier des impôts} cikara\d nattu mukave\d t\d ti}\index{cec}{mukavetti@\textit{mukave\d t\d ti} officier qui pose un sceau}; ceci [est l\'egalis\'e par] la signature d'un propriétaire [terrien] de Peruma\.nkalam, \textit{puravuvari\index{cec}{vari@\textit{vari} taxe!\textit{puravuvari} officier des impôts} cikara\d nattu mukave\d t\d ti}\index{cec}{mukavetti@\textit{mukave\d t\d ti} officier qui pose un sceau}; ceci [est l\'egalis\'e par] la signature d'un propriétaire [terrien] de Mel\=ur, \textit{puravuvari\index{cec}{vari@\textit{vari} taxe!\textit{puravuvari} officier des impôts} cikara\d nattu mukave\d t\d ti}\index{cec}{mukavetti@\textit{mukave\d t\d ti} officier qui pose un sceau}; ceci [est l\'egalis\'e par] la signature de Ce\b rako\b n; ceci [est l\'egalis\'e par] la signature de Kurukular\=aya\b n; ceci [est l\'egalis\'e par] la signature de Co\b lavicc\=atira\index{cec}{colavicca@Co\b lavicc\=atira} Pallavaraiya\b n\index{cec}{Pallavar\=aya\b n}; ceci [est l\'egalis\'e par] la signature de Vil\=a\d tattaraya\b n; ceci [est l\'egalis\'e par] la signature de Pa\.nka\d lar\=aya\b n; ceci [est l\'egalis\'e par] la signature de Meln\=a\d t\d taraya\b n; ceci [est l\'egalis\'e par] la signature de Vec\=alipparaya\b n; ceci [est l\'egalis\'e par] la signature de V\=a\d luvar\=aya\b n; ceci [est l\'egalis\'e par] la signature de\dots; ceci [est l\'egalis\'e par] la signature de \dots \b n\=atapiriya\b n\footnote{L'ordre\index{gnl}{ordre} de pr\'esentation des percepteurs d'imp\^ots laisse penser qu'il y a une hi\'erarchie: le \textit{puravuvari\index{cec}{vari@\textit{vari} taxe!\textit{puravuvari} officier des impôts} cikara\d na \b n\=ayakam} pr\'ec\`ede le \og scelleur\fg\ \textit{puravuvari\index{cec}{vari@\textit{vari} taxe!\textit{puravuvari} officier des impôts} cikara\d nattu mukave\d t\d ti}\index{cec}{mukavetti@\textit{mukave\d t\d ti} officier qui pose un sceau}. Ce ph\'enom\`ene, non normalis\'e, est observable quand il y a un classement des signataires (SII 5 662 l.~8-10; CEC 8). Cf. \textsc{Veluthat} (1993: 92-94) sur la hi\'erarchisation de ces officier\index{gnl}{officier}s et \textit{id.}, p. 95-96 sur l'existence de promotions.}.

\section*{CEC 4}
\subsection*{CEC 4.1 Remarques}

L'inscription a \'et\'e relev\'ee dans ARE 1896 124 et dans ARE 1918 364. Le premier la situe par erreur sur le mur sud du temple\index{gnl}{temple} principal de \'Siva\index{gnl}{Siva@\'Siva} alors que le second pr\'ecise avec justesse qu'elle se trouve sur les murs nord et ouest du \textit{ma\d n\d dapa} devant le temple\index{gnl}{temple} principal. Le texte a \'et\'e publi\'e dans SII 5 989. Mais les \'editeurs de cette inscription, comme \textsc{Mahalingam} (1992: 550, Tj. 2413), reprennent l'erreur de localisation de l'ARE de 1896. Le texte date de la quatorzi\`eme année de r\`egne de Tribhuvanacakravartin Kulottu\.ngac\=o\b ladeva\index{cec}{kulottungacoladevar@Kulottu\.ngac\=o\b ladeva}, \og who was pleased to take Madura and the crowned head of the P\=a\d n\d dya\fg. \textsc{Mahalingam} identifie le roi\index{gnl}{roi} comme Kulottu\.nga III\index{gnl}{Kulottu\.nga III} et date le texte de \textbf{1192}.

Le texte enregistre un don\index{gnl}{don} de cinq terres\index{gnl}{terre} pour faire des jardins \`a fleurs pour \'Siva\index{gnl}{Siva@\'Siva}. Les donatrices sont la fille et la petite-fille d'un certain Jenan\=ataka\b rpakam\index{cec}{Jenan\=ataka\b rpakam} Araiya\b n\index{cec}{Araiya\b n}, \textit{ki\b l\=a\b n} d'\=An\=a\.nk\=urkku\b n\b ram\index{cec}{Anankur@\=An\=a\.nk\=urkku\b n\b ram} dans le Na\d tuviln\=a\d tu\index{cec}{Natuvil@Na\d tuviln\=a\d tu} alias Ir\=ajar\=ajava\d lan\=a\d tu. Elle sont, respectivement, \'epouse d'U\d taiya N\=ayaka\b n, un propriétaire [terrien] de Vetava\b nam\index{cec}{Vetava\b nam} et de Pa\b laiya\b n\=ur\index{cec}{Palaiyanur@Pa\b laiya\b n\=ur} dans le Me\b nmalaippa\b laiya\b n\=urn\=a\d tu\index{cec}{Melmalaippa\b laiya\b n\=urn\=a\d tu}, et \'epouse de Tiruvekampamu-\d taiy\=a\b n\index{cec}{Tiruvekampamu\d taiy\=a\b n} N\=aya\b n un propriétaire [terrien] de Perump\=ur\index{cec}{Perump\=ur} dans le Ve\d n\d nik\=u\b r\b ram\index{cec}{Venni@Ve\d n\d nik\=u\b r\b ram} du Cutta-maliva\d lan\=a\d tu\index{cec}{Cuttamaliva\d lan\=a\d tu}.

Le texte pr\'esent\'e est fond\'e sur l'examen de la publication confrontée avec une premi\`ere lecture \textit{in situ} en 2004. Les murs ont \'et\'e recouverts ensuite d'une peinture trop \'epaisse. L'inscription est r\'epartie, dans le sens de la lecture, sur trois portions du mur nord (l.~1-16, 17-30 et 31-43) et deux portions du mur ouest (l.~44-55 et 56-69) du niveau de piliers du \textit{ma\d n\d dapa}. Les s\'eparations sont marqu\'ees, dans l'ordre\index{gnl}{ordre}, par un pilastre (entre les murs 1 et 2), une niche sans image\index{gnl}{image} (entre 2 et 3), le changement de mur (entre 3 et 4) et une autre niche sans image\index{gnl}{image} (entre 4 et 5).

\subsection*{CEC 4.2 Texte}
\begin{enumerate}
	\item \textbf{svasti \'sr\=\i} \textbf{tri}ripuva\b naccakkarava
	\item t[tika\d l]\index{cec}{Tribhuvanacakravarti} maturaiyum\index{cec}{Maturai} p\=a[\d n]\d ti
	\item {[ya\b n mu]\d titta[lai]yu\.n ko\d n\d ta}
	\item ru\d li\b na [\textbf{\'sr\=\i}]kul[ottu]\.nkaco\b lateva
	\item ku [y\=a]\d n\d tu 10 4 vatu ir\=aj\=adhi[r\=a]ja
	\item va\d lan\=a\d t\d tut\index{cec}{Rajadhirajavala@R\=aj\=adhir\=ajava\d lan\=a\d tu} tirukka\b lumalun\=a\d t\d tup\index{cec}{Tirukka\b lumalan\=a\d tu} pira
	\item matecam tirukka[\b lumala]ttut tirutto
	\item \d nipuramu\d taiya n\=aya[n\=a]\b rku jeya\.n
	\item ko\d n\d taco\b la ma\d n\d talattu me\b nmalai
	\item ppa\b laya\b n\=ur n\=a\d t\d tup pa\b laiya\b n\=u
	\item ru\d taiy\=a\b n\index{cec}{palaiuanur@Pa\b laiyan\=uru\d taiy\=a\b n} vetava\b namu\d taiy\=a\b n u\index{cec}{vetavanam@V\=etava\b namu\d taiy\=a\b n}
	\item \d taiya n\=ayaka\b nukkup pukka\index{cec}{pukka@\textit{pukka} épouse} na\d tuvi
	\item ln\=a\d t\=a\b na ir\=ajar\=ajava\d lan\=a\d t\d tu
	\item \=an\=a\.nk\=urkku\b n\b ra\.n ki\b l\=a\b n jena
	\item {[n\=a]taka\b rpakam araiya\b n\index{cec}{Araiya\b n} maka\d l eti}
	\item ril\=apperum\=a\d lum iva\d l maka
	\item \d l cut[tamaliva]\d lan\=a\d t\d tu\index{cec}{Cuttamaliva\d lan\=a\d tu} ve\d n
	\item \d nikk\=u\b r\b rattup perumuru\d tai[y\=a]
	\item \b n [n\=a]ya\b n tiruvekampamu\d taiy\=a[\b nu]\index{cec}{Tiruvekampamu\d taiy\=a\b n}
	\item {[kku]p pukka\index{cec}{pukka@\textit{pukka} épouse} umaiy\=a\b lviyum ivvi}
	\item ruvom e[\.n*]ka\d l partt\=akka\d lukkum\index{cec}{partta@\textit{partt\=a} époux} [e]
	\item \.nka\d lukkum e\.nka\d l va\.ncattukku
	\item m na\b n\b r\=aka ir\=ajar\=ajava\d lan\=a\d t\d tu
	\item m\=att\=ur[n\=a]\d t\d tu\index{cec}{Mattur@M\=att\=urn\=a\d tu} o\b luka\b raiy\=a\b na\index{cec}{Olukarai@O\b luka\b rai}
	\item kulottu\.nkaco\b lanall\=ur\index{cec}{kulottungacolanallur@Kulottu\.ngac\=o\b lanall\=ur}kka\b nai
	\item y\=ur nantima\.nkala\.n ki\b l\=a\b n c\=uriya
	\item teva\b n\index{cec}{Curiya@C\=uriyat\=eva\b n} tiru\~n\=a\b nacampanta\b nar\index{cec}{tirunana@Tiru\~n\=a\b nacampanta\b n} mu\b n
	\item {[\b ni]laiy\=aka ivar pakkal\index{cec}{pakkal@\textit{pakkal} auprès de} k\=acu\index{cec}{kacu@\textit{k\=acu} pièces de monnaie} ku\d tuttu}
	\item ivar palar perilum k\=acu\index{cec}{kacu@\textit{k\=acu} pièces de monnaie} ku\d tuttuk ko\d n[\d tu*]
	\item ivar perile piram\=a\d nam\index{cec}{piramanam@\textit{pram\=a\d nam} document} pa\d n\d ni
	\item n\=a\.nka\d l ko\d n\d tu vi\d t\d ta \=ur
	\item kka\d nak(k)kuc ce\b r\b r\=uru\d tai
	\item y\=a\b nu\~n capaiy\=arume\b lutti\d t\d ta\index{cec}{sabha@\textit{sabh\=a} assemblée}
	\item cap\=a niyokappa\d ti\index{cec}{niyoka@\textit{niyokam} ordre} cantir\=atitta
	\item varai k\=acu\index{cec}{kacu@\textit{k\=acu} pièces de monnaie}
	\item ko\d l\d l\=avi\b raiyiliy\=aka\index{cec}{iraiyili@\textit{i\b raiyili} non imposable} vi\d t\d ta ni
	\item lam\=avatu tirukka\b lumalattuc\index{cec}{Tirukkalumalam@Tirukka\b lumalam} cut
	\item tamali vatikku\index{cec}{vati@\textit{vati}} me\b rkut tillaivi\d ta
	\item \.nka\index{cec}{Tillaivitankan@Tillaivi\d ta\.nka\b n} v\=aykk\=alukkut\index{cec}{vaykkal@\textit{v\=aykk\=al} canal} te\b rku 1 ka\d n
	\item \d n\=a\b r\b ru 2 \~ncatirattu v\=acciya\b n\index{cec}{Vacciyan@V\=acciya\b n} tiru
	\item tto\d nipuramu\d taiy\=a\b n\index{cec}{Tonipuramutaiyan@T\=o\d nipuramu\d taiy\=a\b n} pakkal\index{cec}{pakkal@\textit{pakkal} auprès de}
	\item ko\d n\d ta kollai ku\b li [2\footnote{La publication lit 3 mais compte tenu du calcul il faut vraisemblablement lire 2. Voir justification n. 74 du chapitre 7.}] 100 m k\=a
	\item cipan ciru\d taikka\b lal \=a\b lv\=a
	\item n pakkal\index{cec}{pakkal@\textit{pakkal} auprès de} ko\d n\d ta kollaiy ku
	\item \b li 3 100 m k\=acivan uyya nin\b r\=a\d ti pa
	\item kkal ko\d n\d ta kollai ku\b li mu\b nn\=u
	\item \b rum cuttamali\index{cec}{Cuttamali} vati[k]ku\index{cec}{vati@\textit{vati}} me\b rku ti
	\item llaivi\d ta\.nka v\=akk\=alukku te\b r
	\item ku 2 \d n\d t\=a\.n ka[\d n\d n\=a]\b r\b ru mun\b r\=a
	\item \~n catirattu tu\d n\d tattu c\=av\=anti vira
	\item pattiran uyyakko\d n\d ta pi\d l\d lai\index{cec}{uyyakkontap@Uyyakko\d n\d tapi\d l\d lai} pak
	\item kal ki\b lakki\d t\d tu te\b rka\d taiya ampa
	\item tu ku\b li nikkikko\d n\d ta kollaiy ku\b li
	\item 100 m muta\b r ka\d n\d n\=a\b r\b ru mu\b n\b r\=a\~n cati
	\item rattu viracci
	\item ya\b n\index{cec}{Viracciya\b n} centana tiruve\d nk\=a\d tu\d taiy\=a\b n\index{cec}{Tiruvenkatu@Tiruve\d nk\=a\d tu\d taiy\=a\b n}
	\item u\d l\d li\d t\d t\=ar pakkal\index{cec}{pakkal@\textit{pakkal} auprès de} ko\d n\d ta kollai
	\item ku\b li 100 m \=akak kollai ku\b li 1000 nilam\index{cec}{nilam@\textit{nilam} terre} ½ i\b n
	\item nilam\index{cec}{nilam@\textit{nilam} terre} araiyil ivaka\d lukku \textbf{ji}vanattuk
	\item ku vi\d t\d ta nilam\index{cec}{nilam@\textit{nilam} terre} tiruna\b ntava\b nam ceytananilam\index{cec}{nilam@\textit{nilam} terre} k\=alum kaikk
	\item o\d n\d tu t\=a\.nka\d l ve\d n\d tina payi\b r cey
	\item tu ko\d n\d tu nikkini\b n\b ra nilam\index{cec}{nilam@\textit{nilam} terre} n\=ayan\=a\b r
	\item kkut tiruna\b ntavanam ceytu na\b rutiruppa\d li
	\item tt\=amam \=akavum pa\d takku\d tik ku\b r\b ram pa\d t\=ama\d l c
	\item aiytu a\d laka ka\d tav\=ar\=akavum innilam\index{cec}{nilam@\textit{nilam} terre}
	\item arai[yu]m catir\=atittavarai k\=acuko\d l\d l\=a\index{cec}{kacu@\textit{k\=acu} pièces de monnaie} yi\b rai
	\item yiliy\=akakko\d n\d tu vi\d t\d tom etiril\=apperu
	\item m\=a\d lum iva\d l\footnote{La publication lit un masculin \textit{iva\b n}. Or il est \'evident que ce possessif ne peut que renvoyer \`a la donatrice m\`ere.} maka\d l umaiy\=a\b lviyum yivviruv
	\item om itu \textbf{\'sr\=\i m\=ahe\'svara rak\d sai}\index{cec}{srimahesvara@\textit{\'sr\=\i mahe\'svara} dévot, surveillant}\tdanda |
\end{enumerate}

\subsection*{CEC 4.3 Traduction}
(1-23) Que la prosp\'erit\'e soit!
En la 14\up{e} ann\'ee [de r\`egne] de Kulottu\.ngac\=o\b ladeva\index{cec}{kulottungacoladevar@Kulottu\.ngac\=o\b ladeva} qui fit la gr\^ace de prendre Maturai\index{cec}{Maturai} et la t\^ete couronn\'ee du [roi\index{gnl}{roi}] \textit{p\=a\d n\d dya}\index{gnl}{pandya@\textit{p\=a\d n\d dya}}, empereur des trois mondes;
pour le Seigneur propri\'etaire de Tirutt\=o\d nipuram\index{gnl}{Tonipuram@T\=o\d nipuram} dans Tirukka\b lumalam\index{cec}{Tirukkalumalam@Tirukka\b lumalam}, \textit{brahmadeya}\index{cec}{brahmadeya@\textit{brahmadeya}} de Tirukka\b lumalan\=a\d tu\index{cec}{Tirukka\b lumalan\=a\d tu}, dans le R\=aj\=adhir\=ajava\d lan\=a\d tu\index{cec}{Rajadhirajavala@R\=aj\=adhir\=ajava\d lan\=a\d tu},
nous deux --- Etiril\=apperum\=a\d l, \'epouse\footnote{Le terme \textit{pukka}\index{cec}{pukka@\textit{pukka} épouse} employ\'e pour signifier \og \'epouse\fg, pr\'ec\'ed\'e du nom de l'\'epoux au datif, est le relatif pass\'e du verbe \textit{puku-tal} dont le sens principal est \og entrer\fg\ (\textit{TL} s.v.). Ainsi, il semble que l'\'epouse de X est litt\'eralement \og celle qui est entr\'ee chez X\fg.} d'U\d taiya N\=ayaka\b n un propriétaire [terrien] de Vetava\b nam et de Pa\b laiya\b n\=ur dans le Me\b nmalaippa\b laiya\b n\=urn\=a\d tu du Jeya\.nko\d n\d ta-co\b lama\d n\d talam\index{cec}{Jayankontacolamandala@Jaya\.nko\d n\d tac\=o\b lama\d n\d dalam}\footnote{Nous supposons que l'\'epoux de la donatrice est un parent du donateur de CEC 1.} et fille de Jenan\=ataka\b rpakam Araiya\b n\index{cec}{Araiya\b n}, \textit{ki\b l\=a\b n} d'\=An\=a\.nk\=urku\b n\b ram, dans Na\d tuviln\=a\d tu alias le R\=ajar\=ajava\d lan\=a\d tu\footnote{Cette division territoriale est dans la r\'egion actuelle de Vi\b luppuram; \textsc{Subbarayalu} (1973: 86-87 et carte 9).}, et sa fille, Umaiy\=a\b lvi, \'epouse de N\=aya\b n Tiruvekampamu\d taiy\=a\b n\index{cec}{Tiruvekampamu\d taiy\=a\b n}\footnote{Plac\'e apr\`es N\=aya\b n, ce nom ne semble pas correspondre \`a un v\'eritable toponyme mais au nom d'une personne nomm\'ee d'apr\`es un \'Siva\index{gnl}{Siva@\'Siva} de K\=a\~ncipuram\index{gnl}{Kancipuram@K\=a\~ncipuram}, Ek\=amran\=atha ou U\d taiy\=ar Tiruvekampamu\d taiyan\=aya\b n\=ar\index{cec}{Tiruvekampamu\d taiy\=a\b n} dans les inscriptions (SII 4 350 l.~2).}, un propriétaire [terrien] de Perumur dans le Ve\d n\d ni-k\=u\b r\b ram du Cuttamaliva\d lan\=a\d tu\index{cec}{Cuttamaliva\d lan\=a\d tu}\footnote{Perum\=ur est identifi\'e \`a Peramp\=ur dans le taluk actuel de Ma\d n\d n\=arku\d ti; \textsc{Subbarayalu} (1973: 99 et carte 7).} --- pour que prosp\`erent nos \'epoux, nous-m\^emes et notre lign\'ee%\footnote{Il semble rare de pr\'eciser comme ici la raison d'une donation, \textsc{Orr} 2000, n. 21.}
\footnote{Ces deux femme\index{gnl}{femme}s sont identifi\'ees, principalement, par leur parent\'e masculine. La m\`ere, Etiril\=apperum\=a\d l, est pr\'esent\'ee par son \'epoux puis par son père\index{gnl}{pere@père}. Sa fille Umaiy\=a\b lvi est présentée par son \'epoux. Leur statut d'\'epouse prime sur leur individualit\'e f\'eminine comme il semble fr\'equent \`a la fin du \textsc{xii}\up{e} si\`ecle; \textsc{Orr} (*2004 [2001]: 215-222).}.

(23-37) Devant C\=uriyateva\b n Tiru\~n\=a\b nacampantar\index{cec}{tirunana@Tiru\~n\=a\b nacampanta\b n}\footnote{Tiru\~n\=a\b nacampantar, l'enfant\index{gnl}{enfant} poète\index{gnl}{poete@poète}, ou plut\^ot son image\index{gnl}{image} divine, n'est jamais d\'esign\'e ainsi dans CEC. Il est \=A\d lu\d taiyappi\d l\d laiy\=ar (CEC 25 l.~7, 28 l.~6, 30 l.~2, 32 l.~5, 35 l.~4, etc.) ou N\=aya\b n\=ar \=A\d lu\d taiyappi\d l\d laiy\=ar (CEC 17 l.~1, 18 l.~2, etc.). Ainsi, il est vraisemblable que la transaction n'est pas effectu\'ee devant l'image\index{gnl}{image} de Tiru\~n\=a\b nacampantar\index{cec}{tirunana@Tiru\~n\=a\b nacampanta\b n} mais en pr\'esence d'une personne habilit\'ee \`a agir (\textit{mutuka\d n}) pour ces femme\index{gnl}{femme}s et nomm\'ee ici d'apr\`es le poète\index{gnl}{poete@poète}. Sur \textit{mutuka\d n}, cf. \textsc{Orr} (*2004 [2001]: 228).}, \textit{ki\b l\=a\b n} de Nantima\.nkalam qui est attach\'e \`a O\b luka\b rai alias Kul\=ottu\.nkaco\b lanall\=ur dans le M\=att\=urn\=a\d tu\index{cec}{Mattur@M\=att\=urn\=a\d tu}, dans Ir\=ajar\=ajava\d lan\=a\d tu\footnote{O\b luka\b rai est identifi\'e \`a \og Oulgaret\fg\ dans le taluk de Vi\b luppuram; \textsc{Subbarayalu} (1973, carte 9).}, nous lui avons donn\'e l'argent (\dots)\footnote{Le sens de la l.~29 reste obscur car l'identit\'e\index{gnl}{identit\'e} de celui ou ceux qui re\c coivent l'argent des femme\index{gnl}{femme}s n'est pas claire. En effet, \`a qui renvoie \textit{ivar palar peril}?} et avons \'etabli le document \`a son nom. Voici les terres\index{gnl}{terre} que nous avons achet\'ees et donn\'ees non imposables et invendables tant que durent lune et soleil selon l'ordre\index{gnl}{ordre} de l'assemblée\index{gnl}{assemblée} sign\'e par les membres de l'assemblée\index{gnl}{assemblée} et le comptable du village, un propriétaire [terrien] de Ce\b r\b r\=ur :

(37-59) \`a l'ouest de la \textit{vati}\index{cec}{vati@\textit{vati}} de Cuttamali\index{cec}{Cuttamali} \`a Tirukka\b lumalam\index{cec}{Tirukkalumalam@Tirukka\b lumalam}, au sud du canal Tillaivi\d ta\.nka\index{cec}{Tillaivitankan@Tillaivi\d ta\.nka\b n}, le 2\up{e} carr\'e du 1\up{er} canalicule : [2]00\footnote{La conjecture `[3]00' propos\'ee par la publication n'est pas correcte car la somme des \textit{ku\b li} des cinq parcelles donn\'ees est alors \'egale \`a 1100: [3]00 (l.~42) + 300 (l.~45) + trois cents (l.~46) + 100 (l.~54) + 100 (l.~58). Or, elle doit \^etre \'egale \`a 1000 (l.~58). Ainsi, il semble plus coh\'erent de conjecturer `[2]00' l.~42.} \textit{ku\b li} de terre\index{gnl}{terre} de jardin obtenue aupr\`es de V\=acciya\b n\index{cec}{Vacciyan@V\=acciya\b n}  un propriétaire [terrien] de Tirutt\=o\d nipuram\index{gnl}{Tonipuram@T\=o\d nipuram!Tirutt\=o\d nipuram}; 300 \textit{ku\b li} de terre\index{gnl}{terre} de jardin obtenue aupr\`es de K\=acipan Ciru\d taikka\b lal \=a\b lv\=an; trois cents \textit{ku\b li} de terre\index{gnl}{terre} de jardin obtenue aupr\`es de K\=acivan Uyyanin\b r\=a\d ti. A l'ouest de la \textit{vati}\index{cec}{vati@\textit{vati}} de Cuttamali\index{cec}{Cuttamali} et au sud du canal de Tillaivi\d tanka\index{cec}{Tillaivitankan@Tillaivi\d ta\.nka\b n}, la portion du troisi\`eme carr\'e du 2\up{e} canalicule: 100 \textit{ku\b li} de terre\index{gnl}{terre} de jardin obtenue en retirant cinquante \textit{ku\b li} au sud-est  aupr\`es de C\=av\=anti Virapattiran Uyyakko\d n\d ta Pi\d l\d lai\index{cec}{uyyakkontanp@Uyyakko\d n\d tapi\d l\d lai}; le troisi\`eme carr\'e du premier canalicule: 100 \textit{ku\b li} de terre\index{gnl}{terre} de jardin obtenue aupr\`es de Viracciya\b n Centana un propriétaire [terrien] de Tiruve\d nk\=a\d tu\index{cec}{Tiruvenkatu@Tiruve\d nk\=a\d tu\d taiy\=a\b n} et d'autres; soit [un total] de 1000 \textit{ku\b li} [c'est-\`a-dire] une terre\index{gnl}{terre} d' 1/2 [\textit{v\=eli}].

(59-69) Sur cette terre\index{gnl}{terre} d'un demi [\textit{v\=eli}], qu'ils\footnote{Ce pluriel renvoie aux cultivateurs qui habitent --- ils payent la taxe d'habitation \textit{ku\d ti} --- et cultivent la terre\index{gnl}{terre} donn\'ee. En \'echange de ceci ils doivent fournir au temple\index{gnl}{temple} des guirlandes du soir compos\'ees des fleurs du jardin.} prennent en main la terre\index{gnl}{terre} laiss\'ee pour leur vivre, un quart de la terre\index{gnl}{terre} faite pour le jardin, qu'ils [la] cultivent selon leur besoin, qu'ils fassent de la terre\index{gnl}{terre} restante un jardin pour le Seigneur et qu'ils r\`eglent sans faillir le devoir de \textit{ku\d ti} en faisant des guirlandes parfum\'ees pour la chambre \`a coucher.

Nous deux, Etiril\=apperum\=a\d l et ma fille Umaiy\=a\d lvi, nous avons acquis et laiss\'e ces terres\index{gnl}{terre} non imposables et invendables d'1/2 [\textit{v\=eli}] tant que durent lune et soleil. Ceci est [sous] la protection des \textit{\'Sr\=\i mahe\'svara}\index{cec}{srimahesvara@\textit{\'sr\=\i mahe\'svara} dévot, surveillant}.


\section*{CEC 5}
\subsection*{CEC 5.1 Remarques}

L'inscription, relev\'ee dans ARE 1918 362, a \'et\'e localis\'ee sur le mur sud du temple\index{gnl}{temple} principal de \'Siva\index{gnl}{Siva@\'Siva}. Elle se situe plus exactement sur le mur sud du \textit{ma\d n\d dapa}. Elle date de la dix-septi\`eme ann\'ee (\textbf{1233?}) de Tribhuvanacakravartin R\=ajar\=ajadeva que \textsc{Mahalingam} (1992: 551, Tj. 2421) sugg\`ere d'identifier comme R\=ajar\=aja III\index{gnl}{Rajaraja III@R\=ajar\=aja III}.

Le texte pr\'esent\'e, tr\`es lacunaire, est bas\'e sur le seul examen de la transcription de l'ASI.

L'inscription semble enregistrer un don\index{gnl}{don} de vaisselle en or pour offrir \`a boire.

\subsection*{CEC 5.2 Texte}
\begin{enumerate}
	\item \textbf{tribhu}va\b naccakkaravarttika\d l\index{cec}{Tribhuvanacakravarti} \textbf{\'sr\=\i}[\textbf{r\=ajar\=a}]
	\item \textbf{ja}teva\b rku y\=a\d n\d tu 10 7 \=avatu n\=a\d l 4
	\item 100 5 10 8 \b n\=al r\=a\textbf{j\=a}tir\=a\textbf{ja}va\d lan\=a\d t
	\item \d tut tirukka\b lumala[n\=a\d t\d tut\index{cec}{Tirukka\b lumalan\=a\d tu} tiru]kka\b lumalat\index{cec}{Tirukkalumalam@Tirukka\b lumalam}
	\item tu u\d taiy\=ar tirutto\d nipuramu\d taiy
	\item \=ar\index{cec}{Tiruttoni@Tirutt\=o\d nipuramu\d taiya n\=aya\b n\=ar, \'Siva} koyil\index{cec}{koyil@\textit{k\=oyil} temple} \dots ta\d n\d niramu
	\item tu ceyta \dots pa\d t
	\item \d tarka\d lil \dots
	\item \b r\b rattu \dots caruppeti
	\item ma\.nkalat\index{cec}{caturvedimangalam@Caturvedima\.ngalam} \dots \textbf{dak\d si}naim\=urttipa
	\item \d t\d tar i \dots \b n\b ni\b n va\d t\d til o\b n\b ri\b n\=al
	\item o\b npate \dots le araikk\=al m\=a\b ri e\b luttu
	\item ve\d t\d tupa \dots \b n irupatto\b npati\b n ka\b la\~nce mukk\=al
\end{enumerate}

\subsection*{CEC 5.3 R\'esum\'e}
Le texte date du 458\up{e} jour\footnote{Une v\'erification de l'estampage est n\'ecessaire pour confirmer la lecture de la transcription qui mentionne 458 jours.} de la 17\up{e} ann\'ee [de r\`egne] de R\=ajar\=ajadeva, empereur des trois mondes, et pr\'esente une donation pour le temple\index{gnl}{temple} du Seigneur propri\'etaire de Tirutt\=o\d nipuram\index{gnl}{Tonipuram@T\=o\d nipuram!Tirutt\=o\d nipuram} \`a Tirukka\b lumalam\index{cec}{Tirukkalumalam@Tirukka\b lumalam} dans le Tirukka\b lumalan\=a\d tu\index{cec}{Tirukka\b lumalan\=a\d tu} du R\=aj\=atir\=ajava\d lan\=a\d tu\index{cec}{Rajadhirajavala@R\=aj\=adhir\=ajava\d lan\=a\d tu}. Il est question d'offrir une pièce de vaisselle (l.~11) pour l'eau\index{gnl}{eau} \`a boire (l.~6). La mention de l'unit\'e de masse qui permet de peser l'or \textit{ka\b la\~ncu} (l.~13) convainc que la pièce de vaisselle est faite en or (ou achet\'ee en or). Le donateur est manquant. Cependant, les occurrences de \textit{pa\d t\d tar} (sk. \textit{bha\d t\d ta}, ma\^itre, officiant) indiquent que ce don\index{gnl}{don} est fortement associ\'e au milieu brahmane\index{gnl}{brahmane}.


\section*{CEC 6}
\subsection*{CEC 6.1 Remarques}

L'inscription a \'et\'e relev\'ee dans ARE 1918 366. Elle a \'et\'e localis\'ee sur le mur sud du \textit{ma\d n\d dapa} devant le temple\index{gnl}{temple} principal. Elle date d'un roi\index{gnl}{roi} \textit{p\=a\d n\d dya}\index{gnl}{pandya@\textit{p\=a\d n\d dya}} du titre de Tribhuvanacakravartin K\=on\=erinmaiko\d n\d t\=a\b n que \textsc{Mahalingam} (1992: 552, Tj. 2426) identifie comme M\=a\b ravarman Vikrama P\=a\d n\d dya III en datant le texte aux environs de 1283. \textsc{Sethuraman} (1978: 216) argue de fa\c con convaincante, en identifiant le fr\`ere du donateur mentionné à la l. 2, qu'il s'agit de M\=a\b ravarman Vikrama P\=a\d n\d dya IV (1333-1340). Le texte daterait alors de \textbf{1339}.

Le texte pr\'esent\'e repose sur l'examen de la transcription de l'ASI. L'inscription ne se trouve pas actuellement sur le mur sud du \textit{ma\d n\d dapa}. Aujourd'hui, nous ne pouvons accéder au soubassement sud du \textit{ma\d n\d dapa}, accol\'e \`a une plateforme o\`u se dresse l'image\index{gnl}{image} mobile de Campantar\index{gnl}{Campantar}. Ainsi, nous supposons que CEC 6 se situe, compte tenu du nombre et de la longueur des lignes, sur tout le long du soubassement sud du \textit{ma\d n\d dapa}, parall\`element \`a CEC 3 au nord.

CEC 6 enregistre un don\index{gnl}{don} de terres\index{gnl}{terre} pour le culte\index{gnl}{culte} instaur\'e au nom du roi\index{gnl}{roi} Ir\=ac\=akka\d l n\=aya\b n, pour nourrir et honorer les image\index{gnl}{image}s d'U\d taiy\=ar Ir\=ac\=akka\d n\=aya\b n\=ar et N\=acciy\=ar Marakataccokkiy\=a\b r, pour nourrir quotidiennement douze\index{gnl}{douze} dévot\index{gnl}{devot(e)@dévot(e)}s venus au monast\`ere\index{gnl}{monastère} et pour entretenir les dévot\index{gnl}{devot(e)@dévot(e)}s qui r\'esident au monast\`ere\index{gnl}{monastère}. Les image\index{gnl}{image}s ont \'et\'e install\'ees par le m\^eme donateur: U\d taiyan\=ayaka\b n un propriétaire [terrien] d'E\d t\d tir\=ama Po\b npa\b r\b ri dans le Na\d tuvilk\=u\b r\b ru du Mi\b lalaikk\=u\b r\b ram dans le P\=a\d n\d tima\d n\d talam.

\subsection*{CEC 6.2 Texte}
\begin{enumerate}
	\item \textbf{svasti \'sr\=\i} U \textbf{tribhu}va\b na\textbf{chakra}vatti ko\b neri\b nmaiko\d n\d t\=a\b n u\d taiy\=ar tirucci\b r\b ram-palamu\d taiy\=ar\index{cec}{Tirucirrampalamu@Tirucci\b ra\b rampalamu\d taiy\=ar} tevat\=a\b nam\index{cec}{tevatanam@\textit{tevat\=a\b nam} propriété divine} ir\=ac\=atir\=acava\d lan\=a\d t\d tu\index{cec}{Rajadhirajavala@R\=aj\=adhir\=ajava\d lan\=a\d tu} n\=a\d tava\b rkku p\=a\d n\d tima\d n\d talattu mi\b lalaikk\=u\b r\b rattu na\d tuvilk\=u\b r\b ru e\d t\d tir\=ama po\b npa\b r\b ri u\d taiy\=a\b n u\d taiyan\=ayaka\b n ta\.n-ka\d l n\=a\d t\d tut tirukka\b lumalattu\index{cec}{Tirukkalumalam@Tirukka\b lumalam} u\d taiy\=ar tirutto\d nipuramu\d taiy\=ar\index{cec}{Tiruttoni@Tirutt\=o\d nipuramu\d taiya n\=aya\b n\=ar, \'Siva} koyil\index{cec}{koyil@\textit{k\=oyil} temple} u\d l\d lil tirukku-\d lattukkuk ki\b l karaiyil namper\=al e\b luntaru\d lap pa\d n\d ni\b na u\d taiy\=ar ir\=ac\=akka\d n\=aya\b n\=a-rkkum\index{cec}{iracakka@Ir\=ac\=akka\d n\=aya\b n\=ar} n\=acciy\=ar marakataccokkiy\=a\b rkum\index{cec}{Marakataccokkiy\=ar} i\b n\b n\=aya\b n\=arkku ir\=ac\=akka\d l n\=aya\b n\index{cec}{iracakka@Ir\=ac\=akka\d n\=aya\b n\=ar} cantikkum ir\=ac\=akka\d l n\=aya\b n\index{cec}{iracakka@Ir\=ac\=akka\d n\=aya\b n\=ar} tiruttoppukkum ittiruttopp
	\item ai\d t\d taka\d lukkum nampim\=ankum ir\=ac\=akka\d l\d n\=aya\b n\index{cec}{iracakka@Ir\=ac\=akka\d n\=aya\b n\=ar} ma\d tattukkum\index{cec}{matam@\textit{ma\d tam} monastère} i\b n\b n\=a\d t\d til ti\b rappil a\d n\d n\=a\b lvi\index{cec}{annalvi@\textit{a\d n\d n\=a\b lvi} frère aîné} cuntarap\=a\d n\d tiyateva\b rku\index{cec}{Cuntarap\=a\d n\d dyadeva} patine\d t\d t\=avatu varaiyum a\d taipp\=a\b na\index{cec}{ataippu@\textit{a\d taippu} limite} nilattu\index{cec}{nilam@\textit{nilam} terre} pa-yir ceyy\=amal p\=a\b lki\d tanta nilam\=ay\index{cec}{nilam@\textit{nilam} terre} variyil\index{cec}{vari@\textit{vari} taxe} ka\b litta nilattu\index{cec}{nilam@\textit{nilam} terre} nam olaippa\d ti\index{cec}{olai@\textit{olai} ôle} vi\d t\d ta nilam\index{cec}{nilam@\textit{nilam} terre} muppati\b r\b ru veli i\b n\b nilam\index{cec}{nilam@\textit{nilam} terre} muppati\b r\b ru veliyum ta\.nka\d lukku cervaiy\=a\b na i\d ta\.nka\d lile ka\d tamai\index{cec}{katamai@\textit{ka\d tamai} taxe foncière} u\d l\d li\d t\d ta\b na celav\=akkuvat\=akap pa\b r\b rip payir ceytu ko\d n\d tu t\=a\.nka\d l i\b n\b nilattukkut\index{cec}{nilam@\textit{nilam} terre} talaim\=a\b ru ta\.nka\d l peril a\~nc\=avatukku a\d taippa\b nai\index{cec}{ataippu@\textit{a\d taippu} limite} nila\index{cec}{nilam@\textit{nilam} terre} \dots l ta\.nka\d l vi\d t\d ta pa\d tikku t\=a\.nka\d l e\b lutik ku\d tutta \=a[v]volaippa\d ti\index{cec}{olai@\textit{olai} ôle} tirutto\d nipurattu
	\item ti\d t\d taiyil pi\b rinta ir\=acentiraco\b lanall\=ur\=al\index{cec}{Iracentira@Ir\=acentiraco\b lanall\=ur} vi\d t\d ta nilam\index{cec}{nilam@\textit{nilam} terre} e\b laraiyum tirukka\b lumalattu\index{cec}{Tirukkalumalam@Tirukka\b lumalam} vi\d t\d ta nilam\index{cec}{nilam@\textit{nilam} terre} ira\d n\d taraiyum \=aka vi\d t\d ta nilam\index{cec}{nilam@\textit{nilam} terre}muppati\b r\b ru veliyum u\d taiy\=ar ir\=a-c\=akka\d n\=aya\b n\=arkkum\index{cec}{iracakka@Ir\=ac\=akka\d n\=aya\b n\=ar} n\=acciy\=ar maratakaccokkiya\b rkkum amutupa\d ti c\=attuppa\d ti u\d l\d li\d t\d ta nitta Nn ta\.nka\d lukku tevat\=a\b nam\=aka\index{cec}{tevatanam@\textit{tevat\=a\b nam} propriété divine} i\d t\d ta nilam\index{cec}{nilam@\textit{nilam} terre} pati\b n\=al veliyum u\d taiy\=ar tirutto\d nipuramu\d taiy\=ar\index{cec}{Tiruttoni@Tirutt\=o\d nipuramu\d taiya n\=aya\b n\=ar, \'Siva} tevat\=a\b nattu\d ta\b ne\index{cec}{tevatanam@\textit{tevat\=a\b nam} propriété divine} k\=u\d t\d tik ko\d n\d tu payir ceytu ita\b nu\d tal koyil\index{cec}{koyil@\textit{k\=oyil} temple} pa\d n\d t\=arattile\index{cec}{pantaram@\textit{pa\d n\d t\=aram} trésorerie du temple} k\=u\d t\d tik ko\d n\d tu u\d taiy\=ar ir\=ac\=akka\d n\=aya\b n\=arkkum\index{cec}{iracakka@Ir\=ac\=akka\d n\=aya\b n\=ar} n\=acciy\=ar maratakaccokkiy\=a\b rkum amutupa\d ti c\=attuppa\d ti u\d l\d li\d t\d ta palanimanta\.nka\d lukkum
	\item ma\d tattil\index{cec}{matam@\textit{ma\d tam} monastère} n\=a\d lva\b ra u\d n\d naniccayitta \textbf{m\=ahe}curar per pa\b n\b nira\d n\d tukkum imma\d tattu\index{cec}{matam@\textit{ma\d tam} monastère} nokki u\d n\d naniccayitta \textbf{m\=ahe}curar\b rkku \=akki i\d t\d tu irukkum \textbf{m\=ahe}curarkkum mu\b n\b rarai veliyum \=aka nilam\index{cec}{nilam@\textit{nilam} terre} pati\b na\b ru veliyum \=aka tevat\=a\b nam\index{cec}{tevatanam@\textit{tevat\=a\b nam} propriété divine} u\d tpa\d ta nilam\index{cec}{nilam@\textit{nilam} terre} muppati\b r\b ru veliyum ipperka\d l\=al a\b nupavittuk ko\d l\d lum \=a\b r\=avatu mutal mutala\d ta\.n-kalum i\b raiyiliy\=akak\index{cec}{iraiyili@\textit{i\b raiyili} non imposable} ku\d tuttom i\b n\b nilattu\index{cec}{nilam@\textit{nilam} terre} cettava\b n ti\d tarum ku\d lamum ku\d ti iruppu nattamum itil ku\d lamum i\b n\b nilam\index{cec}{nilam@\textit{nilam} terre} muppati\b r\b ru veliyum ni\.nkal\=aka i\b raiyi[li]y\=aka ivaiyum ivarka\d l a\b nupovikkak ku\d tuttom ivaiyi\b r\b rukku varum ka\d tamai\index{cec}{katamai@\textit{ka\d tamai} taxe foncière} v\=acal vi\b na\b nipokam palalai \dots ppe\b rum olai\index{cec}{olai@\textit{olai} ôle} \dots
\end{enumerate}

\subsection*{CEC 6.3 R\'esum\'e}
Le texte ne mentionne pas l'ann\'ee de manière conventionnelle. Il date du r\`egne de \og Tribhuva\b nachakravatti Ko\b neri\b nmaiko\d n\d t\=a\b n\fg\footnote{Il est fort probable que l'ann\'ee figurant l.~4, \textit{\=a\b r\=avatu mutal}, qui marque la mise en place de la donation soit l'ann\'ee de r\`egne. Selon \textsc{Sethuraman} (1978: 19) le titre royal \og Tribhuva\b nachakravatti Ko\b neri\b nmaiko\d n\d t\=a\b n\fg\ est attribu\'e aux rois \textit{p\=a\d n\d dya}\index{gnl}{pandya@\textit{p\=a\d n\d dya}} post\'erieurs \`a Kulottu\.nga I\index{gnl}{Kulottu\.nga I} qui a inaugur\'e le titre Tribhuva\b nacakravarti.}. Il s'adresse aux \textit{n\=a\d tavar}\footnote{Ce sont les membres d'une assemblée\index{gnl}{assemblée} de cultivateurs au niveau de la division territoriale du \textit{n\=a\d tu}; \textsc{Subbarayalu} (1973: 33-36) et \textsc{Karashima, Subbarayalu, Matsui} (1978: lv-lvi).} du R\=aj\=adhir\=ajava\d lan\=a\d tu\index{cec}{Rajadhirajavala@R\=aj\=adhir\=ajava\d lan\=a\d tu} qui est un \textit{devad\=ana} du Seigneur propri\'etaire de Tirucci\b r\b ram-palam.
%\footnote{La pr\'esentation de la division territoriale du R\=aj\=adhir\=ajava\d lan\=a\d tu comme une terre\index{gnl}{terre} de la divinit\'e de Citamparam\index{gnl}{Citamparam} appara\^it aussi dans CEC 28, 29 et 397 qui datent de Kulottu\.nga II\index{gnl}{Kulottu\.nga II}. M\^eme chose dans Talainayiru 142 (l.~6-8) et Accalpuram 192 sous KIII? (l.~1) et 395 sous Rajadhiraja II (l.~2). Et Kurukkai, dans ARE 1927 228, il y a GKCmutaiyar tevatanam Virutarajapayankaravalanatu. XXXXX}.

Le donateur est U\d taiyan\=ayaka\b n un propriétaire terrien d'E\d t\d tir\=ama Po\b npa\b r\b ri du Na\d tuvilk\=u\b r\b ru dans le Mi\b lalaikk\=u\b r\b ram du P\=a\d n\d tima\d n\d talam\footnote{\textsc{Subbarayalu} (1973, cartes 4 et 8).}. Il donne des terres\index{gnl}{terre} pour les image\index{gnl}{image}s d'U\d taiy\=ar Ir\=ac\=akka\d n\=aya\b n\=ar\index{cec}{iracakka@Ir\=ac\=akka\d n\=aya\b n\=ar}\footnote{\textsc{Sethuraman} (1978: 208-218) d\'emontre que Ir\=ac\=akka\d n\=aya\b n\=ar est un titre du roi\index{gnl}{roi} M\=a\b ravarman Vikrama P\=a\d n\d dya IV (1333-1340) et que de nombreuses inscriptions font \'etat, comme ici, de l'institution de culte\index{gnl}{culte} \textit{canti}, d'image\index{gnl}{image}s, de f\^etes\index{gnl}{fete@fête}, de village, etc. portant le titre de ce roi\index{gnl}{roi}.} et de la Dame Marakataccokkiy\=ar\index{cec}{Marakataccokkiy\=ar} qui ont \'et\'e \'erig\'ees par lui sur le bord est du bassin sacr\'e qui se trouve \`a l'int\'erieur du temple\index{gnl}{temple} du Seigneur propri\'etaire de Tirutt\=o\d nipuram\index{gnl}{Tonipuram@T\=o\d nipuram!Tirutt\=o\d nipuram} dans Tirukka\b lumalam\index{cec}{Tirukkalumalam@Tirukka\b lumalam}. Ce don\index{gnl}{don} b\'en\'eficie aussi au Seigneur de Tirutt\=o\d nipuram\index{gnl}{Tonipuram@T\=o\d nipuram!Tirutt\=o\d nipuram}, \`a la c\'er\'emonie d'Ir\=ac\=akka\d ln\=aya\b n, au jardin Ir\=ac\=akka\d ln\=aya\b n, aux brahmane\index{gnl}{brahmane}s (\string?) de ce jardin, aux officiants et au monast\`ere\index{gnl}{monastère} Ir\=ac\=akka\d ln\=aya\b n.

La transaction et le partage de la terre\index{gnl}{terre} donn\'ee ne sont pas clairs. Une terre\index{gnl}{terre} de trente \textit{v\=eli} qui a été laiss\'ee en friche et d\'eduite de l'imposition jusqu'\`a la dix-huiti\`eme ann\'ee de r\`egne du fr\`ere a\^in\'e Cuntarap\=a\d n\d dyateva\index{cec}{Cuntarap\=a\d n\d dyadeva}\footnote{\textsc{Sethuraman} (1978: 216) identifie ce roi\index{gnl}{roi} comme Ja\d tavarman Sundara P\=a\d n\d dya IV (1318-1342) et souligne les occurrences \`a cette dix-huiti\`eme ann\'ee dans SII 7 818 et 819.} aurait \'et\'e \'echang\'ee (\textit{talaim\=a\b ru}) la cinqui\`eme ann\'ee de r\`egne\footnote{Il s'agirait de la cinqui\`eme ann\'ee de M\=a\b ravarman Vikrama P\=a\d n\d dya IV.} avec une terre\index{gnl}{terre} de trente \textit{v\=eli} qui se situe \`a Ir\=acentiraco\b lanall\=ur\index{cec}{iracentira@Ir\=acentiraco\b lanall\=ur}, hameau voisin de la colline de Tirutt\=o\d nipuram\index{gnl}{Tonipuram@T\=o\d nipuram!Tirutt\=o\d nipuram}. De cette terre\index{gnl}{terre}, quatorze \textit{v\=eli} seraient laiss\'ees en tant que \textit{devad\=ana}, terre\index{gnl}{terre} permanente destin\'ee \`a orner et nourrir U\d taiy\=ar Ir\=ac\=akka\d n\=aya\b n\=ar\index{cec}{iracakka@Ir\=ac\=akka\d n\=aya\b n\=ar} et la dame Maratakaccokkiyar; elles viendraient s'ajouter au \textit{devad\=ana} du Seigneur propri\'etaire de Tirutt\=o\d nipuram\index{gnl}{Tonipuram@T\=o\d nipuram!Tirutt\=o\d nipuram}. Elle serait cultiv\'ee, ajout\'ee au capital de la trésorerie\index{gnl}{tresorerie@trésorerie} du temple\index{gnl}{temple} et utilis\'ee pour les diff\'erentes d\'epenses incluant les ornements et la nourriture d'U\d taiy\=ar Ir\=ac\=akka\d n\=aya\b n\=ar et de la dame Maratakaccokkiyar. Puis, une terre\index{gnl}{terre} de seize \textit{v\=eli} servirait aux douze\index{gnl}{douze} dévot\index{gnl}{devot(e)@dévot(e)}s \textit{mahe\'svara} venus manger quotidiennement au monast\`ere\index{gnl}{monastère}, \`a la cuisine des \textit{mahe\'svara}\index{cec}{srimahesvara@\textit{\'sr\=\i mahe\'svara} dévot, surveillant} venus manger au monast\`ere\index{gnl}{monastère} et aux \textit{mahe\'svara} qui y r\'esident. Ainsi, cette terre\index{gnl}{terre} de trente \textit{v\=eli} est donn\'ee comme non imposable avec tout ce qu'elle contient (points d'eau\index{gnl}{eau}, taxes, etc.) \`a partir de la sixi\`eme ann\'ee.

\newpage


\section*{B. Enceinte}
\section*{CEC 7}
\subsection*{CEC 7.1 Remarques}

L'inscription, relev\'ee dans l'ARE 1918 392, a été localis\'ee sur le mur est de l'enceinte principale. Elle date de l'ann\'ee suivant la septi\`eme du r\`egne du roi\index{gnl}{roi} \textit{c\=o\b la}\index{gnl}{cola@\textit{c\=o\b la}} \og R\=ajak\=esarivarman alias [R\=ajar\=ajad\=eva]\fg. L'ARE remarque que les pierres en d\'esordre sont tr\`es endommag\'ees et propose le r\'esum\'e suivant: \og The introduction commences with the words \textit{c\=\i rma\b n\b nimalarmaka\d lum}, etc. Seems to record a sale in public auction of a land situated in Pa\b na\.ngu\d di a hamlet of Tiruv\=ali alias Mummu\d di\'s\=o\b lachaturv\=edima\.ngalam, in R\=aj\=adhir\=ajava\d lan\=a\d du\index{cec}{Rajadhirajavala@R\=aj\=adhir\=ajava\d lan\=a\d tu}, to the temple\index{gnl}{temple} of Tirutt\=o\d nipuram\index{gnl}{Tonipuram@T\=o\d nipuram!Tirutt\=o\d nipuram}u\d daiy\=ar and the shrine of Tiruve\.nk\=a\d du\d daiy\=ar set up in it by a certain K\=ali\.ngar\=aya\b n. Mentions the Royal Secretary (\textit{tirumantirav\=olai}) Ne\b riyu\d daichch\=o\b la M\=uv\=endav\=e\d la\b n\fg\footnote{Les mots en italique de la citation sont en \'ecriture tamoule dans le relev\'e.}. \textsc{Mahalingam} (1992: 548, Tj. 2404) reprend la localisation de l'ARE, identifie le roi\index{gnl}{roi} comme R\=ajar\=aja II\index{gnl}{Rajaraja II@R\=ajar\=aja II}, tout en soulignant que R\=ajak\=esarivarma\b n est une erreur pour Parak\=esari, et date ainsi le texte de 1154. Cependant, dans son r\'esum\'e, qui reprend en g\'en\'eral mot \`a mot ceux des ARE, sa lecture de la \textit{meykk\=\i rtti}\index{gnl}{meykkirtti@\textit{meykk\=\i rtti}} diff\`ere: \og \textit{chirma\b n\b nima\d lar}ma\.ngalam\fg.

L'inscription se situe sur le mur ext\'erieur est de la premi\`ere enceinte, au nord du pavillon d'entr\'ee. Le texte contient l'unique \textit{meykk\=\i rtti}\index{gnl}{meykkirtti@\textit{meykk\=\i rtti}} tamoule du corpus\index{gnl}{corpus} du temple\index{gnl}{temple}, qui ne semble pas avoir \'et\'e publi\'ee, et sa transcription a disparu \`a l'ASI de Mysore. L'examen de l'estampage ne permet pas, dans l'\'etat actuel des recherches, la reconstitution int\'egrale du texte. CEC 7 est compos\'ee de quinze lignes au minimum sur une longueur d'environ seize m\`etres. Les pierres, d\'et\'erior\'ees, sont effectivement pour la plupart dans le d\'esordre. Toutefois, nous pouvons ajouter des pr\'ecisions.

\subsection*{CEC 7.2 Donn\'ees historiques}

Le texte semble \^etre constitu\'e de trois parties. Il commence (l.~1-7) par un éloge\index{gnl}{eloge@éloge} royal d\'ebutant par \textit{c\=\i r ma\b n\b ni malar maka\d lum c\=\i .......c celviyu[m]} comme l'a relev\'e l'ARE. Cet éloge est in\'edit\footnote{La \textit{meykk\=\i rtti}\index{gnl}{meykkirtti@\textit{meykk\=\i rtti}} est absente dans \textsc{Cuppirama\d niyam} (1983) qui utilise les SII, dans PI, dans IPS, dans les inscriptions publi\'ees par le Tamilnadu State Department (Na\b n\b nilam\index{cec}{nilam@\textit{nilam} terre}, Ka\b n\b niy\=akumari, Tiruv\=\i \b limi\b lalai\index{gnl}{Vilimilalai@V\=\i \b limi\b lalai}, Tiruvala\~ncu\b li, Tiruttu\b raipp\=u\d n\d ti, Tarumapuri, Ta\~nc\=av\=ur\index{gnl}{Tancavur@Ta\~nc\=av\=ur} va\d t\d tam, T\=amaraipp\=akkam, Perumukkal), ainsi que dans celles de Tiruva\d n\d n\=amalai\index{gnl}{Tiruvannamalai@Tiruva\d n\d n\=amalai} et des \textit{\=Ava\d nam}.}. \textsc{Mahalingam} a vraisemblablement mal lu les premiers mots. Ensuite (l.~7-11), une terre\index{gnl}{terre} confisqu\'ee est vendue aux ench\`eres au temple\index{gnl}{temple}. Cette deuxième partie se cl\^ot sur les signatures authentifiant la vente et sur le signe de ponctuation U. Enfin (l.~11-15), une derni\`ere partie, qui commence avec l'ann\'ee de r\`egne, semble enregistrer, sur ordre\index{gnl}{ordre royal} royal (l.~14), une seconde transaction ou r\'ecapituler la premi\`ere.

Aucun \'el\'ement ne confirme pour le moment l'existence d'une chapelle de Tiruve\.nk\=a\d tu\d taiy\=ar\index{cec}{Tiruvenkatu@Tiruve\d nk\=a\d tu\d taiy\=a\b n} install\'ee \`a l'int\'erieur du temple\index{gnl}{temple} de Tirutt\=o\d nipuram\index{gnl}{Tonipuram@T\=o\d nipuram!Tirutt\=o\d nipuram}u\d taiy\=ar par un d\'enomm\'e K\=ali\.nkar\=aya\b n. En effet, ces noms propres apparaissent mais ils sont dispersés dans un texte tr\`es lacunaire qui ne permet pas d'\'elaborer des liens entre eux.\\

Deux points de l'inscription laissent penser que l'identification du roi\index{gnl}{roi} et partant, la datation avanc\'ees par \textsc{Mahalingam} sont sans fondement.

Un des signataires de l'inscription, l.~15, est Ne\b riyu\d taicco\b lamuventa[ve\d l\=a\b n]\index{cec}{Neriyutai@Ne\b riy\=u\d taicco\b lamuventave\d l\=a\b n}\footnote{Sur le titre de -\textit{m\=uv\=entav\=e\d l\=a\b n} et sa proximit\'e avec le pouvoir royal et le fisc, cf. \textsc{Karashima, Subbarayalu, Matsui} (1978: xlvii-li); \textsc{Veluthat} (1993: 82) et \textsc{Subbarayalu} (*2001h [1982]: 98-99).}. L'ARE traduit sa fonction de \textit{tirumantirav\=olai} par \og Royal Secretary\fg. Cet officier\index{gnl}{officier}, scribe royal, met par \'ecrit, \textit{\=olai}, les ordre\index{gnl}{ordre royal}s du roi\index{gnl}{roi}, \textit{mantiram}, et est charg\'e de les faire appliquer sur le terrain (\textsc{Subbarayalu} *2001h [1982]: 104). Ses occurrences \'epigraphiques nombreuses (ARE 1918 506; SII 17 135, 23 309 et 5 477) le d\'esignent comme un \textit{tirumantirav\=olai}\index{cec}{olai@\textit{olai} ôle!tirumantiravolai@\textit{tirumantirav\=olai} officier-scribe} qui agit souvent sur ordre\index{gnl}{ordre royal} royal (ARE 1925 179, 1918 530, 1970-71 567; SII 8 593; SITI 518 et Dar. a.8 et c.1) ou sur la demande d'un officier\index{gnl}{officier} royal (ARE 1918 513, 1927 148 et IPS 153). Certains textes datent avec certitude du r\`egne de Kulottu\.nga III\index{gnl}{Kulottu\.nga III} (1178-1218) (ARE 1925 179, 1918 530, 1970-71 567; SII 17 135 et 5 477 et Dar. a.8 et c.1). Les autres mentionnent un titre \og Tribhuvanacakravartin K\=o\b n\=eri\b nmaiko\d n\d t\=a\b n\fg\ que les historiens attribuent \`a Kulottu\.nga III\index{gnl}{Kulottu\.nga III}\footnote{\textsc{Nilakanta Sastri} (*2000 [1955]: 428) pense cependant que ARE 1918 506 date du r\`egne affaibli de R\=ajar\=aja III\index{gnl}{Rajaraja III@R\=ajar\=aja III} en raison, principalement, de la pr\'esence de Ne\b riyu\d taicco\b lamuventave\d l\=a\b n\index{cec}{Neriyutai@Ne\b riy\=u\d taicco\b lamuventave\d l\=a\b n}. S'il est vraiment question de R\=ajar\=aja III\index{gnl}{Rajaraja III@R\=ajar\=aja III} alors l'officier\index{gnl}{officier} scribe serait au moins \`a sa cinquante-quatrième ann\'ee de service\index{gnl}{service}.}. Ainsi, Ne\b riyu\d taicco\b lamuventave\d l\=a\b n\index{cec}{Neriyutai@Ne\b riy\=u\d taicco\b lamuventave\d l\=a\b n} ex\'ecute les ordre\index{gnl}{ordre royal}s royaux de Kulottu\.nga III\index{gnl}{Kulottu\.nga III} de 1180 (ARE 1918 513) \`a 1216 (SITI 518). Il nous para\^it inconcevable qu'il exer\c ca aussi sous R\=ajar\=aja II\index{gnl}{Rajaraja II@R\=ajar\=aja II}. Il aurait eu en 1216 au moins 62 ans de service\index{gnl}{service} avec une absence de 26 ans entre 1154 et 1180! Ainsi, nous pensons que cet officier\index{gnl}{officier} scribe a v\'ecu et servi sous Kulottu\.nga III\index{gnl}{Kulottu\.nga III} et R\=ajar\=aja III\index{gnl}{Rajaraja III@R\=ajar\=aja III} sur pr\`es de 46 ans.

De plus, l'ARE 1918 504 pr\'esente le r\'esum\'e d'une inscription de Tiruve\.nk\=a\d tu\index{gnl}{Venkatu@Ve\.nk\=a\d tu!Tiruve\.nk\=a\d tu} datant de la 4\up{e} ann\'ee de r\`egne de \og R\=ajak\=esarivarman alias Tribhuvanachakravartin R\=ajar\=ajad\=eva\fg, situ\'ee sur le mur nord de la premi\`ere enceinte. Les donn\'ees astronomiques compl\`etes \og V\textsubring{r}\'schika, \'su. di. da\'sam\=\i\, Monday, R\=evati\fg\ permettent \`a \textsc{Mahalingam} (1992: 571) de dater exactement le texte: lundi 18 novembre 1219. Le roi\index{gnl}{roi} est donc R\=ajar\=aja III\index{gnl}{Rajaraja III@R\=ajar\=aja III}. Or, cette inscription contient selon l'ARE une \textit{meykk\=\i rtti}\index{gnl}{meykkirtti@\textit{meykk\=\i rtti}} d\'ebutant par \og \textit{c\=\i rma\b n\b numalarmaka\d l}\fg. Il est tr\`es probable que l'éloge\index{gnl}{eloge@éloge} royal de Tiruve\.nk\=a\d tu\index{gnl}{Venkatu@Ve\.nk\=a\d tu!Tiruve\.nk\=a\d tu} et de C\=\i k\=a\b li\index{gnl}{Cikali@C\=\i k\=a\b li} soit le m\^eme\footnote{Il est fr\'equent de trouver des variations textuelles pour un m\^eme éloge\index{gnl}{eloge@éloge}. Cf. les travaux de Charlotte \textsc{Schmid} sur les \textit{meykk\=\i rtti}\index{gnl}{meykkirtti@\textit{meykk\=\i rtti}} du Tiruna\b nipa\d l\d li\index{gnl}{Nanipalli@Na\b nipa\d l\d li!Tiruna\b npa\d l\d li} (Pu\~ncai) dans le cadre des conf\'erences EPHE 2005-2006, ainsi que sa préface, en collaboration avec Emmanuel \textsc{Francis}, du second volume des inscriptions de Putucc\=eri\index{gnl}{Putucc\=eri}.}.

Ainsi, la concordance d'un nom propre et celle de l'éloge\index{gnl}{eloge@éloge} soutiennent notre hypoth\`ese que CEC 7 date de la huiti\`eme ann\'ee de r\`egne de R\=ajar\=aja III\index{gnl}{Rajaraja III@R\=ajar\=aja III}, soit de \textbf{1224}.

\section*{CEC 8}
\subsection*{CEC 8.1 Remarques}

L'inscription, relev\'ee sous ARE 1918 393 et localis\'ee sur le mur est de la premi\`ere enceinte, date du trois cent dix-septi\`eme jour et de l'ann\'ee suivant la septi\`eme du r\`egne du roi\index{gnl}{roi} \textit{c\=o\b la}\index{gnl}{cola@\textit{c\=o\b la}} Tribhuvanacakravartin R\=ajar\=ajadeva. \textsc{Mahalingam} (1992: 548, Tj. 2403) identifie ce roi\index{gnl}{roi} comme R\=ajar\=aja II\index{gnl}{Rajaraja II@R\=ajar\=aja II} et propose la date de 1154.

L'\'epigraphe contient neuf lignes qui s'\'etendent sur environ seize m\`etres. Elle se trouve sur le mur est de l'enceinte, en-dessous de CEC 7. La lecture \textit{in situ} a été impossible \`a cause des couches de peinture. Le texte que nous pr\'esentons et son analyse sont bas\'es sur le seul examen de la transcription de l'ASI.

L'inscription se compose de deux parties. La premi\`ere (l.~1-5), datant des huiti\`eme et neuvi\`eme ann\'ees de r\`egne, enregistre, sur ordre\index{gnl}{ordre royal} royal, la vente aux ench\`eres d'une terre\index{gnl}{terre} confisqu\'ee \`a des gens pour trahison (l. 1 \textit{turokam}\index{cec}{turokam@\textit{tur\=okam} trahison}, sk. \textit{droha})\footnote{Sur les confiscations et ventes des terres\index{gnl}{terre} de tra\^itres, cf. SII 23 310 \`a Tiruvi\d taimarut\=ur\index{gnl}{Tiruviraimarutur@Tiruvi\d taimarut\=ur}, ARE 1918 506 \`a Tiruve\.nk\=a\d tu\index{gnl}{Venkatu@Ve\.nk\=a\d tu!Tiruve\.nk\=a\d tu}, ARE 1917 244 \`a K\=oyil Tirumalam (Na\b n\b nilam), ainsi que ARE 1911 paragraphe 30 et \textsc{Nilakanta Sastri} (*2000 [1955]: 426-428) qui relie ce ph\'enom\`ene essentiellement \`a l'affaiblissement du r\`egne de R\=ajar\=aja III\index{gnl}{Rajaraja III@R\=ajar\=aja III}.}. Cette transaction pr\'esente une affinit\'e certaine avec celle de CEC 7. En effet, cette terre\index{gnl}{terre}, achet\'ee par le temple\index{gnl}{temple}, se situe \`a Pa\b na\.nku\d ti\index{cec}{panankuti@Pa\b na\.nku\d ti}, hameau de Tiruv\=ali alias Mummu\d ticco\b lacaturvetima\.nkalam, qui appara\^it dans CEC 7 l.~12. De plus, l'officier\index{gnl}{officier}-scribe Ne\b riyu\d taicco\b lam\=uventave\d l\=a\b n est pr\'esent aux l.~2, 3, 4 et 9. Et enfin, un segment de phrase de CEC 7 l.~14\footnote{\textit{\dots\ piram\=a\d nam\index{cec}{piramanam@\textit{pram\=a\d nam} document} ku\d tukkavum tiruv\=aymo[\b linta]ru\d li\b namaiyil [u\d taiy\=ar] (tirukka\b lu)malattu\index{cec}{Tirukkalumalam@Tirukka\b lumalam} tirutto\d nipuramu\d taiy\=ar\index{cec}{Tiruttoni@Tirutt\=o\d nipuramu\d taiya n\=aya\b n\=ar, \'Siva} koyil\index{cec}{koyil@\textit{k\=oyil} temple} \=a[tica\d n\d te\'svara]tevarka\b nmika\d lukku\index{cec}{adicandesvara@\=Adica\d n\d de\'svaradeva} \dots}} se retrouve ici l.~2. Ainsi, cette partie semble fortement li\'ee par son emplacement, son contenu et sa syntaxe \`a CEC 7. Nous supposons que la nature de la trahison est mentionn\'ee dans CEC 7. La seconde partie (fin de la l.~5-9), datant de la dixi\`eme ann\'ee, fixe le changement de statut en \textit{devad\=ana} de certaines terres\index{gnl}{terre} du temple\index{gnl}{temple}.

Compte tenu de la datation des parties du texte (huiti\`eme, neuvi\`eme et dixi\`eme ann\'ee de r\`egne de Tirupuvanaccakkaravarttika\d l\index{cec}{Tribhuvanacakravarti} \'Sr\=\i r\=ajar\=ajatevar) et de sa ressemblance avec CEC 7 nous proposons de dater ces fractions de texte de \textbf{1224}, \textbf{1225} et de \textbf{1226}, sous le r\`egne de R\=ajar\=aja III\index{gnl}{Rajaraja III@R\=ajar\=aja III}.

\subsection*{CEC 8.2 Texte}
\begin{enumerate}
	\item \textbf{svasti} tirupuvanaccakkarava(ttika\d l\index{cec}{Tribhuvanacakravarti} \textbf{\'sr\=\i}r\=a\textbf{ja})r\=a\textbf{ja}tevarku y\=a\d n\d tu e\b l\=avati\b n etir\=a-m\=a\d n\d tu n\=a\d l 3 100 10 7 inn\=a\d l r\=ajar\=a\dots ni\b naippi\b npa\d ti turokam\index{cec}{turokam@\textit{tur\=okam} trahison} ceyta marutaiki\b l\=a\b n\index{cec}{Marutai} tillaipperum\=a\b num\index{cec}{Tillaipperum\=a\b n} vikkiraco\b la\dots virapperum\=a\b num perumuru\d taiy\=a\b n\index{cec}{Perumuru\d taiy\=a\b n} co\b la\b num ivarka\d l u\b ravumu\b raiy\=arilum k\=ari[ya\~nceyt\=a]rilum a\d timaip perilum turokattu\index{cec}{turokam@\textit{tur\=okam} trahison} u\d tpa\d t\d t\=arum k\=a[\d ni]y\=ay m\=a\b rina nilattu\index{cec}{nilam@\textit{nilam} terre} k\=averikku va\d takaraip pa\d t\d ta n\=a\d tuka\d lil k\=a\d niko\d lv\=a\b rkku\index{cec}{kani@\textit{k\=a\d ni} droit, propriété}
	\item vayir\=atar\=ayarum\dots r\=a\textbf{ja}r\=a\textbf{ja}pperuvilaivi\b r\b ra\index{cec}{vilai@\textit{vilai} prix!peruvilai@\textit{peruvilai} prix fixé (?)}\dots laipa\d t\dots ttu o\d tukkavum innilattukku\index{cec}{nilam@\textit{nilam} terre} ivarka\d lum k\=a\.nkeyar\=ayarum\index{cec}{Kankeyar@K\=a\.nkeyar\=ayar} v\=a\d n\=atar\=ayarum\index{cec}{Vanatara@V\=a\d n\=atar\=ayar} kacciyar\=ayarum\index{cec}{Kacciyar\=ayar} malaiyappir\=ayarum\index{cec}{Malaiyappir\=ayar} puravari\index{cec}{vari@\textit{vari} taxe!\textit{puravuvari} officier des impôts} cikara\d na n\=ayakam pa\d n\d ta\d nainalluru\d taiy\=a\b n\index{cec}{Pantanai@Pa\d n\d ta\d nainalluru\d taiy\=a\b n} mukave\d t\d ti\index{cec}{mukavetti@\textit{mukave\d t\d ti} officier qui pose un sceau} ne\b r-kuppai\index{cec}{Ne\b rkuppaiyu\d taiy\=a\b n} u\d taiy\=a\b num ir\=a\textbf{j\=a}tir\=a\textbf{ja}\dots \=urka\d lil vi\b rki\b ra nilattukku\index{cec}{nilam@\textit{nilam} terre} tirumantira olai\index{cec}{olai@\textit{olai} ôle!tirumantiravolai@\textit{tirumantirav\=olai} officier-scribe} ne\b riy\=u\d taicco\b lamuventave\d l\=anum\index{cec}{Neriyutai@Ne\b riy\=u\d taicco\b lamuventave\d l\=a\b n}\footnote{\textit{ne\b ri\textbf{y\=u}\d tai}, la voyelle initiale \textit{u} est ajout\'ee au-dessus de \textit{yu} sur la graphie de la transcription.} e\b lutti\d t\d ta piram\=a\d nam\index{cec}{piramanam@\textit{pram\=a\d nam} document} ko\d tukkavum tiruv\=ay mo\b lintaru\d li\b namaiyil [u\d taiy\=ar tirukka\b lu]malattu\index{cec}{Tirukkalumalam@Tirukka\b lumalam} tirutto\d nipuramu\d taiy\=ar\index{cec}{Tiruttoni@Tirutt\=o\d nipuramu\d taiya n\=aya\b n\=ar, \'Siva} koyil\index{cec}{koyil@\textit{k\=oyil} temple} \=atica\textbf{\d n\d de\'sva}ratevarka\b nmika\d lukku r\=a\textbf{ja}r\=a\textbf{ja}pperuvilaivi\b r\b ra\index{cec}{vilai@\textit{vilai} prix!peruvilai@\textit{peruvilai} prix fixé (?)} [r\=a\textbf{j\=a}tir\=a\textbf{ja}va\d lan\=a\d t-\d tut] tiruv\=alin\=a\d t\d tu [tiruv\=aliy\=a\b na] mummu\d tico\b laccaruppetima\.nkalattup\index{cec}{caturvedimangalam@Caturvedima\.ngalam!Mummu\d tico\b lacaturvedima\.ngalam} pi\d t\=akai\index{cec}{pitakai@\textit{pi\d t\=akai} hameau} pa\b na\.nku\d tiyil\index{cec}{panankuti@Pa\b na\.nku\d ti} pa[ricai] ki\b l\=a\b n virapperum\=a\b nai
	\item kk\=a\d ni\index{cec}{kani@\textit{k\=a\d ni} droit, propriété} m\=a\b ri\b na N\dots kku 8 2 100 matippa\d ti \dots\ ikk\=acu\index{cec}{kacu@\textit{k\=acu} pièces de monnaie} irup\=ati\b n\=ay\=attukkum innilam\index{cec}{nilam@\textit{nilam} terre} pati\b r\b ru veliyum \dots \b r\b ruk ku\d tutta\b namakku ivai puravari\index{cec}{vari@\textit{vari} taxe!\textit{puravuvari} officier des impôts} cikara\d na n\=ayakam panta\d nainall\=ur u\d taiy\=a\b ne\b luttu\index{cec}{Pantanai@Pa\d n\d ta\d nainalluru\d taiy\=a\b n} ivai puravari\index{cec}{vari@\textit{vari} taxe!\textit{puravuvari} officier des impôts} cikara\d nattu mukave\d t\d ti\index{cec}{mukavetti@\textit{mukave\d t\d ti} officier qui pose un sceau} ne\b rkuppai-yu\d taiy\=a\b ne\b luttu\index{cec}{Ne\b rkuppaiyu\d taiy\=a\b n} ivai va\.nkattaraiyya\b ne\b luttu\index{cec}{vankattarai@Va\.nkattaraiyya\b n} ivai malaiyappiyar\=aya\b n\index{cec}{Malaiyappir\=ayar} e\b luttu ivai kacciya\b r\=aya\b n\index{cec}{Kacciyar\=ayar} e\b luttu ivai vayir\=atar\=aya\b n\index{cec}{Vayir\=atar\=aya\b n} e\b luttu ivai v\=a\d n\=atar\=aya\b n\index{cec}{Vanata@V\=a\d n\=atar\=aya\b n} e\b luttu ivai ka\.nkaiyar\=aya\b n\index{cec}{Kankai@Ka\.nkaiyar\=aya\b n} e\b luttu ivai ne\b riyu\d taico\b lamuventave\d l\=a\b n\index{cec}{Neriyutai@Ne\b riy\=u\d taicco\b lamuventave\d l\=a\b n} e\b luttu U \=e\b l\=avati\b n etir\=a-m\=a\d n\d tu u\d taiy\=ar tirukka\b lumalattut\index{cec}{Tirukkalumalam@Tirukka\b lumalam} tiruto\d nipuramu\d taiy\=ar\index{cec}{Tiruttoni@Tirutt\=o\d nipuramu\d taiya n\=aya\b n\=ar, \'Siva} koyil\index{cec}{koyil@\textit{k\=oyil} temple} \=atica\d n\d te\textbf{\'sva}ra-tevarka\b nmika\d lukku\index{cec}{adicandesvara@\=Adica\d n\d de\'svaradeva} r\=a\textbf{j\=a}tir\=a\textbf{ja}va\d lan\=a\d t\d tu tiruv\=a\d lin\=a\d t\d tu tiruv\=aliy\=a\b na mummu\d ti-co\b lacaturvetima\.nkalattu pi\d t\=akai\index{cec}{pitakai@\textit{pi\d t\=akai} hameau}
	\item pa\b na\.nku\d tiyil\index{cec}{panankuti@Pa\b na\.nku\d ti} paricaiki\b l\=a\b n virapperum\=a\b naik k\=a\d nim\=a\b ri\index{cec}{kani@\textit{k\=a\d ni} droit, propriété} ir\=acar\=acapperuvilai\index{cec}{vilai@\textit{vilai} prix!peruvilai@\textit{peruvilai} prix fixé (?)} vi\b r\b ra nilattukku\index{cec}{nilam@\textit{nilam} terre} vilaippa\d ti\index{cec}{vilai@\textit{vilai} prix} k\=a\d ni\index{cec}{kani@\textit{k\=a\d ni} droit, propriété}\dots ru\d ta\b n\dots karuvukalattu o\d tukki\b na k\=acukku\index{cec}{kacu@\textit{k\=acu} pièces de monnaie} \dots \d ti\b na 3 100il k\=acu\index{cec}{kacu@\textit{k\=acu} pièces de monnaie} 6 1000 8 100m 3 100 10 6l k\=acu 10 1000 4 100 5 10 \dots kk\=acu\index{cec}{kacu@\textit{k\=acu} pièces de monnaie} 2 10 1000 2 100 5 10 4l k\=a\d ni\index{cec}{kani@\textit{k\=a\d ni} droit, propriété} v\=aci nikkik k\=acu\index{cec}{kacu@\textit{k\=acu} pièces de monnaie} irupati\b nayirattumu \dots \b ril oru m\=avaraikkum ivai puravari\index{cec}{vari@\textit{vari} taxe!\textit{puravuvari} officier des impôts} cikara\d na nayakam panta\d nainalluru\d taiy\=a\b n\index{cec}{Pantanai@Pa\d n\d ta\d nainalluru\d taiy\=a\b n} e\b luttu ivai puravari\index{cec}{vari@\textit{vari} taxe!\textit{puravuvari} officier des impôts} cikara\d nattu mukave\d t\d ti\index{cec}{mukavetti@\textit{mukave\d t\d ti} officier qui pose un sceau} ne\b rkuppaiyu\d taiy\=a\b ne\b luttu\index{cec}{Ne\b rkuppaiyu\d taiy\=a\b n} ivai va\.nkattarai-ya\b ne\b luttu\index{cec}{vankattarai@Va\.nkattaraiyya\b n} ivai malaiyappiyar\=aya\b ne\b luttu\index{cec}{Malaiyappir\=ayar} ivai kacciyar\=aya\b n\index{cec}{Kacciyar\=ayar} e\b luttu ivai vayir\=ata-r\=aya\b n\index{cec}{Vayir\=atar\=aya\b n} e\b luttu ivai v\=a\d n\=atar\=aya\b n\index{cec}{Vanata@V\=a\d n\=atar\=aya\b n} e\b luttu ivai k\=a\.nkeyar\=aya\b n\index{cec}{Kankeyar@K\=a\.nkeyar\=ayar} e\b luttu ivai ne\b riyu\d taic-co\b lam\=uventave\d l\=a\b n e\b luttu U ir\=acam\=a\d nikkap\index{cec}{Iracamani@Ir\=acam\=a\d nikka} pallavaraya\b rkuc\index{cec}{Pallavar\=aya\b n} co
	\item llumpa\d ti turokika\d laik\index{cec}{turoki@\textit{tur\=oki} traître} k\=a\d nim\=a\b ri\index{cec}{kani@\textit{k\=a\d ni} droit, propriété} ir\=acar\=acapperuvilai\index{cec}{vilai@\textit{vilai} prix!peruvilai@\textit{peruvilai} prix fixé (?)} vi\b rka ni\d naippi\d t\d ta nilattu\index{cec}{nilam@\textit{nilam} terre} paricaiki\b l\=a\b n virapperum\=a\b nai ir\=a\textbf{j\=a}tir\=a\textbf{ja}va\d lan\=a\d t\d tut tiruv\=alin\=a\d t\d tut\index{cec}{Tiruvalin@Tiruv\=alin\=a\d tu} tiruv\=aliy\=a\b na\index{cec}{Tiruvali@Tiruv\=ali} mummu\d ticco\b lacaturvetima\.nkalattup pi\d t\=akai\index{cec}{pitakai@\textit{pi\d t\=akai} hameau} pa\b na\.nku\d tiyil\index{cec}{panankuti@Pa\b na\.nku\d ti} k\=a\d nim\=a\b ri\b na\index{cec}{kani@\textit{k\=a\d ni} droit, propriété} nilam\index{cec}{nilam@\textit{nilam} terre} pati\b r\b ru veliyum u\d taiy\=ar tirukka\b lumalattut\index{cec}{Tirukkalumalam@Tirukka\b lumalam} tirutto\d nipuramu\d taiy\=a\b nkut tiru[n\=a-mattukk\=a\d ni]y\=aka\index{cec}{kani@\textit{k\=a\d ni} droit, propriété} ir\=acar\=acapperuvilai\index{cec}{vilai@\textit{vilai} prix!peruvilai@\textit{peruvilai} prix fixé (?)} vi\b r\b ratu i\b n\b nilam\index{cec}{nilam@\textit{nilam} terre} e\b l\=avati\b netir\=am\=a\d n\d taikku etir\=am\=a\d n\d tu k\=ar mutal ir\=acar\=acapperuvilaip\index{cec}{vilai@\textit{vilai} prix!peruvilai@\textit{peruvilai} prix fixé (?)} piram\=a\b nap\index{cec}{piramanam@\textit{pram\=a\d nam} document} pa\d tiye ikkoyilil\index{cec}{koyil@\textit{k\=oyil} temple} \=atica\d n-\d te\textbf{\'sva}ratevarka\b nmika\d l kaikko\d n\d tu a\b nupavippat\=akap pa\d n\d nu\dots va\.nkattaraiya\b n\index{cec}{vankattarai@Va\.nkattaraiyya\b n} e\b luttu U y\=a\d n\d tu o\b npat\=avatu \b n\=a\d l 3 100 5 10l ippa\d ti ni\b naippi\b npa\d ti U tirupuva\b nac-cakkiravartti
	\item ko\b neri\b nmaiko\b n[\d t\=a\b n ir\=a\textbf{j\=a}tir\=a\textbf{ja}]va\d la\b n\=a\d t\d tut tirukka\b lumalattu\index{cec}{Tirukkalumalam@Tirukka\b lumalam} u\d taiy\=ar tirutto\d ni-puramu\d taiy\=ar\index{cec}{Tiruttoni@Tirutt\=o\d nipuramu\d taiya n\=aya\b n\=ar, \'Siva} koyil\index{cec}{koyil@\textit{k\=oyil} temple} tevarka\b nmikku cim\=a\textbf{he\'sva}rak ka\d nk\=a\d ni\index{cec}{kani@\textit{k\=a\d ni} droit, propriété}\index{cec}{srimahesvara@\textit{\'sr\=\i mahe\'svara} dévot, surveillant} ceyvarka\d lukkum ittevarkku tevat\=a\b na\index{cec}{tevatanam@\textit{tevat\=a\b nam} propriété divine} i\b raiyiliy\=a\b na\index{cec}{iraiyili@\textit{i\b raiyili} non imposable} nilattu\index{cec}{nilam@\textit{nilam} terre} i\b n\b n\=a\d t\d tu vikkiramaco\b la\b n\index{cec}{Vikkiramaco\b lamarut\=ur} marut\=ur nilam\index{cec}{nilam@\textit{nilam} terre} jeyattu\.nkama\.nkalattu nilam\index{cec}{nilam@\textit{nilam} terre} veliyum virutar\=ayapaya\.nkarava\d lan\=a\d t\d tuk\index{cec}{Virutar\=ayapaya\.nkarava\d lan\=a\d tu} ku-\b rukkaiy\=a\b na\index{cec}{Ku\b rkkai} vikkiramaco\b lacaruppetima\.nkalattu\index{cec}{caturvedimangalam@Caturvedima\.ngalam!Vikramac\=o\b laccaturvedima\.ngalam} nilam\index{cec}{nilam@\textit{nilam} terre} n\=alemukk\=ale m\=u\b n\b ru m\=a-vum ra\dots pa pir\=ama\d nak k\=a\d ni\index{cec}{kani@\textit{k\=a\d ni} droit, propriété} \dots r\=akama\.nkalattu nilam\index{cec}{nilam@\textit{nilam} terre} o\b n\b remukk\=ale orum\=a-muntirikaikki\b lmukk\=ale mu\b n\b ru m\=a a\b la\dots y\=a\b na p\=a\d n\d tiya\b naive\b nko\d n\d taco\b laccarup-petima\.nkalattup\index{cec}{caturvedimangalam@Caturvedima\.ngalam!P\=a\d n\d diya\b naive\b nko\d n\d tac\=o\b laccaturvedima\.ngalam} pirinta ir\=acar\=acanalluril nilam\index{cec}{nilam@\textit{nilam} terre} araiyai oru \dots to
	\item \d t\d tiy\=a\b na ka\d n\d taram\=a\d nikkaccaturvetima\.nkalattu nilam\index{cec}{nilam@\textit{nilam} terre} oru veliyum tirupuva\b na \dots \d tum \=a\b r\b rur\=a\b na ir\=acan\=ar\=aya\d nacaturvetima\.nkalattu cu\.nkantavirttaco\b la\b nall\=ur \dots \b ru\dots i\b ru\dots\ nilattu\index{cec}{nilam@\textit{nilam} terre} ni\b n\b rum k\=u\d ti\b na nilam\index{cec}{nilam@\textit{nilam} terre} o\b n\b re irum\=avarai araikk\=a\d niyum co-\b lakkulavi\d lakkuma\.nkalatto\d tum ir\=acar\=aca\b n am\dots\ \b n\b ru\.nk\=u\d ti \dots k\=a\d t\d turil payiruk-ki\b rutta\index{cec}{iruttu@\textit{i\b ruttu} payer un impôt} pa\b r\b ril ni\b n\b rum k\=u\d ti\b na nilattu\index{cec}{nilam@\textit{nilam} terre} nilam\index{cec}{nilam@\textit{nilam} terre} e\d t\d tu m\=akk\=a\d ni araikk\=a\d ni mu\b ntiri-kaiyum pu\d l\d lur\=a\b na tirucci\b r\b rampalacaruppetima\.nkalattu\index{cec}{caturvedimangalam@Caturvedima\.ngalam!Tirucci\b r\b rampalaccaturvedima\.ngalam} ni\b n\b rum k\=u\d ti\b na nilattu\index{cec}{nilam@\textit{nilam} terre} nilam\index{cec}{nilam@\textit{nilam} terre} ira\d n\d te orum\=avum pu\dots yi ni\b n\b rum k\=u\d ti\b na \b nilattup\index{cec}{nilam@\textit{nilam} terre} payirukki\b rutta pa\b r\b ril kamuku ni\b n\b ra nilam\index{cec}{nilam@\textit{nilam} terre} araiye k\=a\d ni muntirikaiyum k\=acu\index{cec}{kacu@\textit{k\=acu} pièces de monnaie} p\=ati i\b rutta\index{cec}{iruttu@\textit{i\b ruttu} payer un impôt} pa\b r\b ril nilam\index{cec}{nilam@\textit{nilam} terre} mukk\=ale irum\=a arai araikk\=a\d nikki\b l mukk\=alum kulottu\.nkaco\b la\b n\index{cec}{kulottungacolan@Kulottu\.ngac\=o\b la\b n}
	\item v\=a\~nciy\=uril payirikki\b rutta\index{cec}{iruttu@\textit{i\b ruttu} payer un impôt} pa\b r\b ril ni\b n\b rum k\=u\d ti\b na nilattu\index{cec}{nilam@\textit{nilam} terre} nilam\index{cec}{nilam@\textit{nilam} terre} ira\d n\d te \dots\ \=a\d lu\d tai-ya n\=aya\b n\=arkku \dots rakama\.nkalattut tirun\=amattukk\=a\d niy\=ay m\=a\b ri\b na nilam\index{cec}{nilam@\textit{nilam} terre} pati\b n oru velikkum talaima\dots\ pa\b r\b rukku \dots\ ve\d n\d tum nilattukku\index{cec}{nilam@\textit{nilam} terre} ir\=a\textbf{j\=adi}r\=a\textbf{ja}va\d la-\b n\=a\d t\d tut tiruv\=aliy\=a\b na\index{cec}{Tiruvali@Tiruv\=ali} mummu\d tico\b lacaruppetima\.nkalattu\index{cec}{caturvedimangalam@Caturvedima\.ngalam!Mummu\d tic\=o\b laccaturvedima\.ngalam} pa\b la\b nti\b ram e\b n\b ru o\d t-\d tukko\d n\d ta nilattil\index{cec}{nilam@\textit{nilam} terre} pa\b na\.nku\d tiye\b n\b ru\index{cec}{panankuti@Pa\b na\.nku\d ti} \dots ttu varuki\b ra nilattil\index{cec}{nilam@\textit{nilam} terre} ittevatiru\b n\=amattuk-k\=a\d niy\=ay\index{cec}{kani@\textit{k\=a\d ni} droit, propriété} \=a\d lu\d taiya n\=ayan\=ar tevat\=a\b nam\=a\b na\index{cec}{tevatanam@\textit{tevat\=a\b nam} propriété divine} nilam\index{cec}{nilam@\textit{nilam} terre} iruvatt\=are ira\d n\d tu m\=avum iv-v\=uril \dots \d tu\d taiy\=a\b n pa\b r\b ru nilam\index{cec}{nilam@\textit{nilam} terre} o\b n\b raraiye mu\b n\b ru m\=akk\=a\d ni muntirikaik ki\b laru-m\=avaraiyum \dots t\=aya\b n pa\b r\b ru nilam\index{cec}{nilam@\textit{nilam} terre} pati\b r\b ruveli \dots nallur\=a\b na etirilico\b la\b n\index{cec}{Etirilico\b la} pa\b na\.n-ku\d tiyil\index{cec}{panankuti@Pa\b na\.nku\d ti} malaiyappir\=aya\b n\index{cec}{Malaiyappir\=ayar} [pa]\b r\b ru nilam\index{cec}{nilam@\textit{nilam} terre}
 	\item \dots\ mu\b n\b ru m\=akk\=a\d ni mu\b ntirikaik ki\b la\b rum\=avaraiyum m\=a\b rit tevat\=a\b nami\b raiyili-y\=ay\index{cec}{tevatanam@\textit{tevat\=a\b nam} propriété divine}\index{cec}{iraiyili@\textit{i\b raiyili} non imposable} ni\b rkavum ka\d ta[vat\=aka]c colli ippa\d ti ka\d nakkilum i\d t\d tuk ko\d l\d lak ka\d tavat\=aka varikk\=u\b ru\index{cec}{vari@\textit{vari} taxe} ceyv\=arka\d lukku co\b n\b nom i\b n\b nilam\index{cec}{nilam@\textit{nilam} terre} itteva\b rkut tevat\=a\b na\index{cec}{tevatanam@\textit{tevat\=a\b nam} propriété divine} tirumantira olai\index{cec}{olai@\textit{olai} ôle!tirumantiravolai@\textit{tirumantirav\=olai} officier-scribe} ne\b riyu\d taicco\b laventave\d l\=a\b n ivai villavar\=aya\b ne\b luttu\index{cec}{Villavar\=aya\b n} ivai amarako\b n\index{cec}{Amarako\b n} e\b luttu ivai mu\dots\ e\b luttu \dots n\=aya\b n e\b luttu ivai pirutika\.nkar\=aya\b n\index{cec}{Pirutika\.nkar\=aya\b n} e\b luttu ivai ila\.nkecuv\=a\b n\index{cec}{Ila\.nkecuv\=a\b n} e\b luttu ivai k\=a\.nkaiyar\=aya\b n\index{cec}{Kankai@Ka\.nkaiyar\=aya\b n} e\b luttu y\=a\d n\d tu pattu \b n\=a\d l irupatte\b lu U
\end{enumerate}

\subsection*{CEC 8.3 R\'esum\'e}
\subsubsection*{8.3.1 Premi\`ere partie}
Le texte date du 317\up{e} jour de l'ann\'ee suivant la septi\`eme ann\'ee [de r\`egne] de R\=ajar\=ajadeva, empereur des trois mondes.
Les terres\index{gnl}{terre} de Tillaipperum\=a\b n\index{cec}{Tillaipperum\=a\b n} \textit{ki\b l\=a\b n} de Marutai, de Vikkira \dots\ Virapperum\=a\b n et de Co\b la\b n un propriétaire [terrien] de Perumur qui ont trahi, ainsi que celles de leur famille et celles de ceux qui sont inclus dans la trahison au nom de l'esclavage, m\^eme s'ils n'ont rien fait, ont \'et\'e confisqu\'ees. Un document, selon l'ordre\index{gnl}{ordre royal} royal, des terres\index{gnl}{terre} vendues dans les villages \dots\ fut sign\'e par K\=a\.nkeyar\=ayar\index{cec}{Kankeyar@K\=a\.nkeyar\=ayar}, V\=a\d n\=atar\=ayar\index{cec}{Vanata@V\=a\d n\=atar\=aya\b n}, Kacciyar\=ayar\index{cec}{Kacciyar\=ayar}, Malaiyappir\=ayar\index{cec}{Malaiyappir\=ayar}, le \textit{puravari\index{cec}{vari@\textit{vari} taxe!\textit{puravuvari} officier des impôts} cikara\d na n\=ayakam} un propriétaire [terrien] de Panta\d nainall\=ur\index{cec}{Pantanai@Pa\d n\d ta\d nainalluru\d taiy\=a\b n}, le \textit{mukave\d t\d ti}\index{cec}{mukavetti@\textit{mukave\d t\d ti} officier qui pose un sceau} un propriétaire [terrien] de Ne\b rkuppai\index{cec}{Ne\b rkuppaiyu\d taiy\=a\b n} et par le \textit{tirumantira olai}\index{cec}{olai@\textit{olai} ôle!tirumantiravolai@\textit{tirumantirav\=olai} officier-scribe} Ne\b riyu\d taicco\b lamuventave\d l\=a\b n\index{cec}{Neriyutai@Ne\b riy\=u\d taicco\b lamuventave\d l\=a\b n}. La terre\index{gnl}{terre} de dix \textit{v\=eli} confisqu\'ee \`a Virapperum\=a\b n \textit{ki\b l\=a\b n} de Paricai dans Pa\b na\.nku\d ti\index{cec}{panankuti@Pa\b na\.nku\d ti}, hameau de Tiruv\=ali\index{cec}{Tiruvali@Tiruv\=ali} alias Mummu\d tico\b laccaruppetima\.n-kalam\index{cec}{caturvedimangalam@Caturvedima\.ngalam!Mummu\d tic\=o\b laccaturvedima\.ngalam}, dans le Tiruv\=alin\=a\d tu\index{cec}{Tiruvalin@Tiruv\=alin\=a\d tu} du R\=aj\=atir\=ajava\d lan\=a\d tu\index{cec}{Rajadhirajavala@R\=aj\=adhir\=ajava\d lan\=a\d tu}, a été vendue aux ench\`eres\footnote{\textit{R\=ajar\=ajaperuvilai}, litt\'eralement \og grand prix R\=ajar\=aja\fg, semble \^etre le terme d\'esignant la vente aux ench\`eres appel\'ee R\=ajar\=aja, cf. \textsc{Subbarayalu} (2003), s. v. \textit{peruvilai}.} aux autorit\'es \=Atica\d n\d de\'svara\index{cec}{adicandesvara@\=Adica\d n\d de\'svaradeva} du temple\index{gnl}{temple} du Seigneur propri\'etaire du Tirutt\=o\d nipuram\index{gnl}{Tonipuram@T\=o\d nipuram!Tirutt\=o\d nipuram} de Tirukka\b lumalam\index{cec}{Tirukkalumalam@Tirukka\b lumalam}. Cette vente a été l\'egalis\'ee par le \textit{puravavi cikara\d na n\=ayakam} un propriétaire [terrien] de Panta\d nainall\=ur\index{cec}{Pantanai@Pa\d n\d ta\d nainalluru\d taiy\=a\b n}, le \textit{mukave\d t\d ti}\index{cec}{mukavetti@\textit{mukave\d t\d ti} officier qui pose un sceau} un propriétaire [terrien] de Ne\b rkuppai\index{cec}{Ne\b rkuppaiyu\d taiy\=a\b n}, Va\.nkattaraiya\b n\index{cec}{vankattarai@Va\.nkattaraiyya\b n}, Malaiyappiyar\=aya\b n\index{cec}{Malaiyappir\=ayar}, Kacciyar\=aya\b n\index{cec}{Kacciyar\=ayar}, Vayir\=atar\=aya\b n\index{cec}{Vayir\=atar\=aya\b n}, V\=a\d n\=atar\=aya\b n\index{cec}{Vanata@V\=a\d n\=atar\=aya\b n}, Ka\.nkaiyar\=aya\b n\index{cec}{Kankai@Ka\.nkaiyar\=aya\b n} et Ne\b riyu\d taico\b lamuventave\d l\=a\b n\index{cec}{Neriyutai@Ne\b riy\=u\d taicco\b lamuventave\d l\=a\b n}.

Un second paragraphe, marqu\'e par le signe de ponctuation U, mentionne la vente et en pr\'ecise le prix (plus de 20000 \textit{k\=acu}\index{cec}{kacu@\textit{k\=acu} pièces de monnaie}, la fin manque). Le passage est sign\'e par le \textit{puravavi cikara\d na n\=ayakam} un propriétaire [terrien] de Panta\d nainall\=ur\index{cec}{Pantanai@Pa\d n\d ta\d nainalluru\d taiy\=a\b n}, le \textit{mukave\d t\d ti}\index{cec}{mukavetti@\textit{mukave\d t\d ti} officier qui pose un sceau} un propriétaire [terrien] de Ne\b rkuppai\index{cec}{Ne\b rkuppaiyu\d taiy\=a\b n}, Va\.nkattaraiya\b n\index{cec}{vankattarai@Va\.nkattaraiyya\b n}, Malaiyappiyar\=aya\b n\index{cec}{Malaiyappir\=ayar}, Kacciyar\=aya\b n\index{cec}{Kacciyar\=ayar}, Vayir\=atar\=aya\b n\index{cec}{Vayir\=atar\=aya\b n}, V\=a\d n\=atar\=aya\b n\index{cec}{Vanata@V\=a\d n\=atar\=aya\b n}, Ka\.nkaiyar\=aya\b n\index{cec}{Kankai@Ka\.nkaiyar\=aya\b n} et Ne\b riyu-\d taico\b lamuventave\d l\=a\b n\index{cec}{Neriyutai@Ne\b riy\=u\d taicco\b lamuventave\d l\=a\b n}\footnote{Les signataires figurent dans le m\^eme ordre\index{gnl}{ordre royal} que pr\'ec\'edemment (l.~3).}.

Un troisi\`eme paragraphe, annonc\'e par U, r\'ecapitule la transaction. Selon ce qui a \'et\'e dit \`a Ir\=acam\=a\d nikkap Pallavaraya\b n\index{cec}{Pallavar\=aya\b n}\footnote{Cette formulation rendrait compte d'un ordre\index{gnl}{ordre royal} royal re\c cu par Ir\=acam\=a\d nikkap Pallavaraya\b n\index{cec}{Pallavar\=aya\b n} qui est charg\'e de l'appliquer sur le terrain (informations communiqu\'ees par G. \textsc{Vijayavenugopal}). Sur le mode d'\'emission, d'ex\'ecution d'un ordre\index{gnl}{ordre royal} royal et ses diff\'erentes \'etapes jusqu'\`a la gravure, cf. \textsc{Nilakanta Sastri} (*2000 [1955]: 468-469); \textsc{Heitzman} (*2001 [1997]: 156-158); \textsc{Veluthat} (1993: 139) et \textsc{Ali} (2000: 172-174).}, les terres\index{gnl}{terre} des tra\^itres ont \'et\'e confisqu\'ees. Une terre\index{gnl}{terre} de dix \textit{v\=eli} confisqu\'ee \`a Virapperum\=a\b n ki\b l\=a\b n de Paricai dans Pa\b na\.nku\d ti\index{cec}{panankuti@Pa\b na\.nku\d ti}, hameau de Tiruv\=ali\index{cec}{Tiruvali@Tiruv\=ali} alias Mummu\d tico\b laccaruppetima\.nkalam\index{cec}{caturvedimangalam@Caturvedima\.ngalam!Mummu\d tic\=o\b laccaturvedima\.ngalam}, dans le Tiruv\=alin\=a\d tu\index{cec}{Tiruvalin@Tiruv\=alin\=a\d tu} du R\=aj\=atir\=ajava\d lan\=a\d tu\index{cec}{Rajadhirajavala@R\=aj\=adhir\=ajava\d lan\=a\d tu}, selon l'ordre\index{gnl}{ordre royal} royal, et vendue aux ench\`eres en tant que \textit{tirun\=amattukk\=a\d ni} du Seigneur propri\'etaire du Tirutt\=o\d nipuram de Tirukka\b lumalam\index{cec}{Tirukkalumalam@Tirukka\b lumalam}. Cette terre\index{gnl}{terre}, \`a compter de la mousson de l'ann\'ee suivant celle qui suit la septi\`eme (9\up{e} ann\'ee), selon le document de vente aux ench\`eres, prise entre les mains des autorit\'es \=Adica\d n\d de\'svara\index{cec}{adicandesvara@\=Adica\d n\d de\'svaradeva} du temple\index{gnl}{temple}, doit bénéficier (au temple\index{gnl}{temple}). Cet ordre\index{gnl}{ordre royal} date du 350\up{e} jour de la neuvi\`eme ann\'ee.

\subsubsection*{8.3.2 Seconde partie}

Une nouvelle inscription, marqu\'ee par U et le titre royal (\textit{tirupuva\b naccakkira-vartti} ko\b neri\b nmaiko\b n\d t\=a\b n), commence \`a la fin de la l.~5. Elle s'adresse aux autorit\'es et \`a ceux qui font la surveillance \textit{sr\=\i mahe\'svara}\index{cec}{srimahesvara@\textit{\'sr\=\i mahe\'svara} dévot, surveillant} du temple\index{gnl}{temple} du Seigneur propri\'etaire de Tirutt\=o\d nipuram\index{gnl}{Tonipuram@T\=o\d nipuram!Tirutt\=o\d nipuram}. Elle \'enum\`ere les terres\index{gnl}{terre} transform\'ees en \textit{devad\=ana} non imposables de la divinit\'e\footnote{ Sur les diff\'erents statuts terriens (\textit{devad\=ana}, \textit{tirun\=amattukk\=a\d ni}); cf. \textsc{Nilakanta Sastri} (*2000 [1955]: 576-582).}. Parmi ces derni\`eres (l.~8) il y a un \textit{devad\=ana} d'\=A\d lu\d taiya N\=aya\b n\=ar. La l.~9 r\'ecapitule le tout: ces terres\index{gnl}{terre} sont faites \textit{devad\=ana} non imposables et ce changement a \'et\'e signal\'e \`a ceux qui d\'efinissent les taxes. Les signataires sont le \textit{tirumantira olai}\index{cec}{olai@\textit{olai} ôle!tirumantiravolai@\textit{tirumantirav\=olai} officier-scribe} Ne\b riyu\d taicco\b lamuventave\d l\=a\b n\index{cec}{Neriyutai@Ne\b riy\=u\d taicco\b lamuventave\d l\=a\b n}, Villavar\=aya\b n, Amarako\b n\index{cec}{Amarako\b n}, \dots\ Pirutika\.nkar\=aya\b n, Ila\.nkecuva\b n, K\=a\.nkaiyar\=aya\b n\index{cec}{Kankai@Ka\.nkaiyar\=aya\b n}. Ceci date du vingt-septi\`eme jour de la dixi\`eme ann\'ee.\\

\section*{CEC 9}
\subsection*{CEC 9.1 Remarques}

L'inscription a \'et\'e r\'epertori\'ee comme la suite de ARE 1918 393 (CEC 8) sur la transcription de l'ASI. Or, elle se trouve sur la face nord du mur d'enceinte, en-dessous de CEC 11. Elle ne peut \^etre ni la suite directe de CEC 8 par son emplacement ni celle de CEC 11 par son contenu.
Elle comporte sept lignes sur une longueur de plus de treize m\`etres. Le d\'ebut manque. Le texte que nous pr\'esentons est principalement fond\'e sur l'examen de la transcription. Malgr\'e les couches de peinture certains passages \'etaient clairement lisibles \textit{in situ} en 2005 et en 2006.

Le texte se compose de quatre parties ponctu\'ees par U. Les trois premi\`eres forment un ensemble et traitent du changement de statut de terres\index{gnl}{terre} en \textit{devad\=ana}, sur ordre\index{gnl}{ordre royal} royal, \`a partir d'une dixi\`eme ann\'ee, tout comme dans la seconde partie de CEC 8. Ces trois premières parties sont de syntaxe identique. Elles d\'ebutent, comme CEC 8 l.~4-5, par l'expression \textit{collumpa\d ti} pr\'ec\'ed\'ee d'un nom attitr\'e au datif. La derni\`ere partie (fin de la l.~3-7) est une inscription ind\'ependante qui date de la dixi\`eme ann\'ee de r\`egne de \og Tiripuva\b naccakkaravattika\d l\index{cec}{Tribhuvanacakravarti} \'Sr\=\i\ Ir\=ajar\=ajateva\b r\fg, et dont les commanditaires sont les autorit\'es du temple\index{gnl}{temple}.

Les datations et les ressemblances avec CEC 8 permettent de dater CEC 9 de la dixi\`eme ann\'ee de r\`egne de R\=ajar\=aja III\index{gnl}{Rajaraja III@R\=ajar\=aja III}, soit de \textbf{1226}.

\subsection*{CEC 9.2 Texte}
\begin{enumerate}
	\item lumpa\d ti \=a\d lu\d taiya n\=aya\b n\=a\b rku virutur\=ayapaya\.nkarava\d lan\=a\d t\d tu\index{cec}{Virutar\=ayapaya\.nkarava\d lan\=a\d tu} n\=akama\.nkalattu i\d taiyur\=a\b na \textbf{ja}ya\.nko\d n\d taco\b lanall\=urilum\index{cec}{Jayankontacolanallur@Jaya\.nko\d n\d taco\b lanall\=ur} mata\d t\d taiy\=a\b na te\b nc\=atta\.nku\d tiyilum tiru-n\=amattuk k\=a\d niy\=a\b na\index{cec}{kani@\textit{k\=a\d ni} droit, propriété} \b nilam\index{cec}{nilam@\textit{nilam} terre} u\d taiy\=ar tirutto\d nipuramu\d taiy\=a\b rkum\index{cec}{Tiruttoni@Tirutt\=o\d nipuramu\d taiya n\=aya\b n\=ar, \'Siva} u\d taiy\=ar tiruk-ko\d tikk\=a u\d taiy\=a\b rkkum tevat\=a\b namum\index{cec}{tevatanam@\textit{tevat\=a\b nam} propriété divine} ti\b rappum\=akaiy\=alai innilam\index{cec}{nilam@\textit{nilam} terre} ivar m\=a\b ri tevat\=a\b nam\=akavum\index{cec}{tevatanam@\textit{tevat\=a\b nam} propriété divine} \dots\ k\=a\d ni m\=a\b ri\b na\index{cec}{kani@\textit{k\=a\d ni} droit, propriété} nilattukkum\index{cec}{nilam@\textit{nilam} terre} u\d taiy\=ar tirutto\d nipuramu\d tai-y\=a\b rkku\index{cec}{Tiruttoni@Tirutt\=o\d nipuramu\d taiya n\=aya\b n\=ar, \'Siva} mel\=a\b r\b r\=ur\=a\b na \textbf{k\d satri}yacik\=ama\d nicaruppetima\.nkalattup\index{cec}{caturvedimangalam@Caturvedima\.ngalam!K\d satriyac\=\i k\=ama\d niccaturvedima\.ngalam} pirinta vikkira-maco\b lamarut\=uril\index{cec}{Vikkiramaco\b lamarut\=ur} tevat\=a\b nam\=a\b na\index{cec}{tevatanam@\textit{tevat\=a\b nam} propriété divine} nilam\=a\b ri\index{cec}{nilam@\textit{nilam} terre} \=a\d lu\d taiya n\=aya\b n\=arkku \textbf{de}vat\=a\b nam\=aka i\d t\d ta nilam\index{cec}{nilam@\textit{nilam} terre} aivelikkum talaim\=a\b ru tiruv\=aliy\=a\b na\index{cec}{Tiruvali@Tiruv\=ali} mummu\d tico\b laccaruppetima\.nka-lattup\index{cec}{caturvedimangalam@Caturvedima\.ngalam!Mummu\d tic\=o\b laccaturvedima\.ngalam} pa\b la\.nti\b rappil tevat\=a\b nam\=a\b na\index{cec}{tevatanam@\textit{tevat\=a\b nam} propriété divine} nilattu\index{cec}{nilam@\textit{nilam} terre} u\d taiy\=ar tirutto\d nipuramu\d taiy\=ar\index{cec}{Tiruttoni@Tirutt\=o\d nipuramu\d taiya n\=aya\b n\=ar, \'Siva} tirun\=amattuk k\=a
	\item \d niy\=a\b na nilam\index{cec}{nilam@\textit{nilam} terre} irupatt\=a\b re ira\d n\d tum\=avum patt\=avatu mutal tevat\=a\b nattu\index{cec}{tevatanam@\textit{tevat\=a\b nam} propriété divine} m\=a\b ri i\b n\b n\=aya\b n\=a\b rku tevat\=anam\=aka\index{cec}{tevatanam@\textit{tevat\=a\b nam} propriété divine} i\d t\d tu \textbf{pras\=a}ta\~nceytaru\d li\b na ceyyumpa\d ti va\b ntattu ceyyumpa\d tip pa\d tiye i\b n\b nilam\index{cec}{nilam@\textit{nilam} terre} irupatt\=a\b re ira\d n\d tu m\=avum tevat\=a\b nam\=aka\index{cec}{tevatanam@\textit{tevat\=a\b nam} propriété divine} vi\d t\d tuk ku\d tukkap pa\d n\d nuvate ippa\d ticcolluvatu U cetikular\=ayarkkuc collumpa\d ti u\d taiy\=ar tirutto\d nipuramu\d taiy\=a\b rku\index{cec}{Tiruttoni@Tirutt\=o\d nipuramu\d taiya n\=aya\b n\=ar, \'Siva} viritar\=ayapaya\.nkarava\d lan\=a\d t\d til\index{cec}{Virutar\=ayapaya\.nkarava\d lan\=a\d tu} tevat\=a\b nam\=a\b na\index{cec}{tevatanam@\textit{tevat\=a\b nam} propriété divine} \b nilattu\index{cec}{nilam@\textit{nilam} terre} cervill\=ata nilam\=a\b ri\index{cec}{nilam@\textit{nilam} terre} i\b n\b nilattukkut\index{cec}{nilam@\textit{nilam} terre} talaim\=a\b ru i\d tuki\b ra nilattukku\d tal\=aka\index{cec}{nilam@\textit{nilam} terre} tiruv\=ali-y\=a\b na\index{cec}{Tiruvali@Tiruv\=ali} mummu\d tico\b laccaruppetima\.nkalattut\index{cec}{caturvedimangalam@Caturvedima\.ngalam!Mummu\d tic\=o\b laccaturvedima\.ngalam} ti\b rappil i\b n\b n\=aya\b n\=ar tirun\=amattuk k\=a\d niy\=a\b na\index{cec}{kani@\textit{k\=a\d ni} droit, propriété} nilam\index{cec}{nilam@\textit{nilam} terre} pati\b n o\b n\b raraiye mu\b n\b ru m\=akk\=a\d ni muntirikaic ci\b n\b namum patt\=a-vatu mutal tevat\=a\b nam\=aka\index{cec}{tevatanam@\textit{tevat\=a\b nam} propriété divine} i\d t\d tu \textbf{pras\=a}ta\~nceytaru\d li\b na tirumuka
	\item po\b nattu i\b n\b nilam\index{cec}{nilam@\textit{nilam} terre} tevat\=a\b nam\=aka\index{cec}{tevatanam@\textit{tevat\=a\b nam} propriété divine} vi\d t\d tuk ku\d tukkap pa\d n\d nuvate ippa\d ti colluvate UU ka\b nakar\=ayarkkuc collumpa\d ti u\d taiy\=ar tirutto\d nipuramu\d taiy\=a\b rku\index{cec}{Tiruttoni@Tirutt\=o\d nipuramu\d taiya n\=aya\b n\=ar, \'Siva} virutar\=aya-paya\.nkarava\d lan\=a\d t\d tut\index{cec}{Virutar\=ayapaya\.nkarava\d lan\=a\d tu} tevat\=a\b nam\=a\b na\index{cec}{tevatanam@\textit{tevat\=a\b nam} propriété divine} \=urka\d lil cervall\=amai m\=a\b ri\b na \=urka\d lukkut talaim\=a\b ru r\=a\textbf{j\=a}tir\=ayava\d la\b n\=a\d t\d tu va\d tak\=avirinall\=ur\=a\b na etirilico\b la\b nma\d naku\d tiyil\index{cec}{Etirilico\b la} ti-\b rappu nilam\index{cec}{nilam@\textit{nilam} terre} patte mu\b n\b ru m\=akk\=a\d ni muntirikaik k\=\i \b la\b ru (m\=ava)raiyum patt\=avatu mutal tevat\=a\b nam\=aka\index{cec}{tevatanam@\textit{tevat\=a\b nam} propriété divine} i\d t\d tu \textbf{pras\=a}ta\~nceytaru\d li\b na tirucakam\footnote{La transcription lit \textit{tirumukam} mais nous avons lu \textit{in situ} \textit{tirucakam}, dont le sens est \`a d\'echiffrer.} po\b nattu tirumukap pa\d tiye i\b n\b nilam\index{cec}{nilam@\textit{nilam} terre} tevat\=a\b nam\=aka\index{cec}{tevatanam@\textit{tevat\=a\b nam} propriété divine} vi\d t\d tuk ku\d tukkap pa\d n\d nuvate ippa\d ti colluvatu U tiripuva\b naccakkaravattika\d l\index{cec}{Tribhuvanacakravarti} \textbf{\'sr\=\i} ir\=a\textbf{ja}r\=a\textbf{ja}teva\b rku y\=a\d n\d tu patt\=avatu ir\=a(\textbf{j\=a})\textbf{dhi}r\=a\textbf{ja}va\d la\b n\=a\d t\d tut tirukka\b lumalan\=a\d t\d tut\index{cec}{Tirukka\b lumalan\=a\d tu} tirukka\b lumalattu\index{cec}{Tirukkalumalam@Tirukka\b lumalam} u\d taiy\=ar tirut-to\d nipuramu\d taiy\=ar\index{cec}{Tiruttoni@Tirutt\=o\d nipuramu\d taiya n\=aya\b n\=ar, \'Siva} ko
	\item yil \=atica\d n\d te\textbf{\'sva}rar\index{cec}{adicandesvara@\=Adica\d n\d de\'svaradeva} tiruvaru\d l\=al i\b n\b n\=aya\b n\=ar koyil\index{cec}{koyil@\textit{k\=oyil} temple} \textbf{\'sr\=\i}m\=a\textbf{he\'sva}rak ka\d nk\=a\d ni\index{cec}{kani@\textit{k\=a\d ni} droit, propriété}\index{cec}{srimahesvara@\textit{\'sr\=\i mahe\'svara} dévot, surveillant} ceyv\=ar-ka\d lum c\=\i k\=ariyam\index{cec}{srikariyamcey@\textit{\'sr\=\i k\=ariyam cey} employé du temple} ceyv\=a\b num tevaka\b nmi koyil\index{cec}{koyil@\textit{k\=oyil} temple} ka\d nakka\b num\index{cec}{kanakku@\textit{ka\d nakku} comptable} ivva\b naivom i\b n\b n\=a-ya\b n\=ar tevat\=a\b nam\index{cec}{tevatanam@\textit{tevat\=a\b nam} propriété divine} va\d tak\=avirinall\=ur\=a\b na etirilico\b la\b nma\d nakku\d tik ku\d timakka\d lukku i\b rukka niccayittuk k\=a\d niyi\d t\d tuk\index{cec}{kani@\textit{k\=a\d ni} droit, propriété} ku\d tutta paric\=avatu i\b n\b n\=aya\b n\=a\b rku viritar\=ayapa-ya\.nkarava\d lan\=a\d t\d tu\index{cec}{Virutar\=ayapaya\.nkarava\d lan\=a\d tu} tevat\=a\b nam\=a\b na\index{cec}{tevatanam@\textit{tevat\=a\b nam} propriété divine} nilattuc\index{cec}{nilam@\textit{nilam} terre} cervall\=amaiyil m\=a\b ri pa\b r\b rukku okka tevat\=a\b nami\d tuki\b ra\index{cec}{tevatanam@\textit{tevat\=a\b nam} propriété divine} nilattukku\index{cec}{nilam@\textit{nilam} terre} o\b npat\=avatu n\=a\d l mu\b n\b n\=u\b r\b ru aimpatin\=a\b ru i\d t\d ta ni-\d naippi\b npa\d ti malaiyappiyar\=ayaraip\index{cec}{Malaiyappir\=ayar} pa\b r\b ru m\=a\b ri ivv\=uril tevat\=a\b nami\d t\d ta\index{cec}{tevatanam@\textit{tevat\=a\b nam} propriété divine} nilam\index{cec}{nilam@\textit{nilam} terre} 10 3M Q ki\b l Z M1/2 i\b n\b nilam\index{cec}{nilam@\textit{nilam} terre} patte mu\b n\b ru m\=akk\=a\d ni mu\b ntirikaik ki\b la\b ru m\=avarai-yum patt\=avatu k\=ar mutal
	\item ka\d tamaiyirukkumi\d tattu\index{cec}{katamai@\textit{ka\d tamai} taxe foncière} mu\b npu ivv\=uril i\b n\b n\=aya\b n\=a\b rku tevat\=a\b nam\=a\b na\index{cec}{tevatanam@\textit{tevat\=a\b nam} propriété divine} nilam\index{cec}{nilam@\textit{nilam} terre} \=a\b ra-rai velikkum ka\d tamai\index{cec}{katamai@\textit{ka\d tamai} taxe foncière} ko\d l\d lum pa\d tikku ivarka\d lukku ku\d tutta ka\d taippa\d tip pa\d tiye veli o\b n\b rukku n\=u\b r\b ru irupati\b n kalam\=aka\index{cec}{kalam@\textit{kalam} unité de mesure du paddy} vanta nellu irukkavum\index{cec}{iruttu@\textit{i\b ruttu} payer un impôt} irukkumi\d tattu n\=a\d t\d tukku i\d t\d ta ni\d naippum akamum pe\b ravum o\d t\d tuppa\d ti nel kalattuk ku\b ru\d ni n\=a\b n\=a\b li\index{cec}{nali@\textit{n\=a\b li} unité de mesure de graine}y\=aka vanta nellu taravi\d tup\=uritiy\=aka nikkavum nikki ni\b n\b ra nellukku talaiyakappa\d ti k\=acu\index{cec}{kacu@\textit{k\=acu} pièces de monnaie} i\d tavum ku\d timaip pottakap pa\d ti veli o\b n\b rukku irukalam\=aka\index{cec}{kalam@\textit{kalam} unité de mesure du paddy} va\b nta nellut taravi\d tup\=uri\d t\d ta\b ne k\=u\d t\d tikkai vilaippa\d ti\index{cec}{vilai@\textit{vilai} prix} k\=acu\index{cec}{kacu@\textit{k\=acu} pièces de monnaie} i\d tavu(ma)rai n\=a\b li\index{cec}{nali@\textit{n\=a\b li} unité de mesure de graine} koyi\b rpe\b ru i\b rukkak\index{cec}{iruttu@\textit{i\b ruttu} payer un impôt} ka\d tavarka\d l\=akavum ma\b r\b ru o\b n\b rum ka\d tavatall\=atat\=akavum i\b n-\b nilam\index{cec}{nilam@\textit{nilam} terre} 10 3M Q ki\b l Z M1/2rayum mu\b npu ivv\=uril tevat\=a\b nam\=a\b na\index{cec}{tevatanam@\textit{tevat\=a\b nam} propriété divine} nilam\index{cec}{nilam@\textit{nilam} terre} 6 1/2 veliyum
	\item (\=aka) nilam\index{cec}{nilam@\textit{nilam} terre} 16 1/2 3M Q ki\b l Z M1/2kkum ippa\d tiye ca\d ntir\=atittava\b ra irukkak ka\d tavarka\d l\=akak kalve\d t\d tik ku\d tuttom u\d taiy\=ar tirutto\d nipuramu\d taiy\=ar\index{cec}{Tiruttoni@Tirutt\=o\d nipuramu\d taiya n\=aya\b n\=ar, \'Siva} koyil\index{cec}{koyil@\textit{k\=oyil} temple} \=atica\d n\d te\textbf{\'sva}rateva\b rtiruvaru\d l\=al\index{cec}{adicandesvara@\=Adica\d n\d de\'svaradeva} in\b n\=aya\b n\=ar koyil\index{cec}{koyil@\textit{k\=oyil} temple} \textbf{\'sr\=\i}m\=a\textbf{he\'sva}rak ka\d nk\=a\d ni\index{cec}{kani@\textit{k\=a\d ni} droit, propriété}\index{cec}{srimahesvara@\textit{\'sr\=\i mahe\'svara} dévot, surveillant} cey-v\=arka\d lum cik\=ariyam\index{cec}{srikariyamcey@\textit{\'sr\=\i k\=ariyam cey} employé du temple} ceyv\=a\b num tevaka\b nmi koyi\b rka\d nakka\b num\index{cec}{kanakku@\textit{ka\d nakku} comptable} ivva\d naivom ip-pa\d tikku ivai koyi\b rka\d nakku vir\=a\d namu\d taiy\=a\b n\index{cec}{viranam@Vir\=a\d namu\d taiy\=a\b n} e\b luttu ippa\d tikku ivai koyi\b rka\d nak-kut tiruni\b n\b ra \=uru\d taiy\=a\b n\index{cec}{tiruninra@Tiruni\b n\b rav\=uru\d taiy\=a\b n} e\b luttu ippa\d tikku ivai koyil\index{cec}{koyil@\textit{k\=oyil} temple} ka\d nakku\index{cec}{kanakku@\textit{ka\d nakku} comptable} pu\.nk\=uru\d taiy\=a\b n\index{cec}{punkur@Pu\.nk\=uru\d taiy\=a\b n} e\b luttu ippa\d tikku ivai koyil\index{cec}{koyil@\textit{k\=oyil} temple} ka\d nakku\index{cec}{kanakku@\textit{ka\d nakku} comptable} t\=al\=uru\d taiy\=a\b n\index{cec}{talur@T\=al\=uru\d taiy\=a\b n} e\b luttu ippa\d tikku ivai kanmi n\=a\b rpatte\d n\d n\=ayirapa\d t\d tan\index{cec}{narpatten@N\=a\b rpatte\d n\d n\=ayirapa\d t\d tan} ka\d nakku\index{cec}{kanakku@\textit{ka\d nakku} comptable} po\b rkoyilpa\d t\d ta\b n\index{cec}{Porkoyil@Po\b rkoyilpa\d t\d ta\b n} e\b luttu ippa\d tikku ivai tevaka\b n-mi mu\d tiva\b la\.nkuco\b lapa\d t\d ta\b n\index{cec}{mutivalan@Mu\d tiva\b la\.nkuco\b lapa\d t\d ta\b n} e\b luttu
	\item (ippa\d ti)kku ivai ci(m\=a)\textbf{he\'sva}ra ka\d nka\d nik ka\d nakku\index{cec}{kanakku@\textit{ka\d nakku} comptable} ka\d niccaipp\=akkamu\d taiy\=a\b n\index{cec}{kaniccai@Ka\d niccaipp\=akkamu\d taiy\=a\b n} e\b luttu U
\end{enumerate}

\subsection*{CEC 9.3 R\'esum\'e}
\subsubsection*{9.3.1 Premi\`ere partie}

Cette partie, tr\`es obscure, enregistre trois changements de statut d'une terre.
Le premier est confus sur la question du b\'en\'eficiaire: \=A\d lu\d taiya N\=aya\b n\=ar ou le Seigneur propri\'etaire de Tirutt\=o\d nipuram\index{gnl}{Tonipuram@T\=o\d nipuram} re\c coit, selon l'ordre\index{gnl}{ordre royal} royal\footnote{Sur le champ lexical des ordre\index{gnl}{ordre royal}s royaux
% (CEC 8 \textit{tiruvaymolintarulinamaiyil}, \textit{ninaippinpati} et \textit{ninaipitta}; CEC 9 \textit{prasantanceytarulina ceyyumpati}, \textit{prasantanceytarulina tirumuka} (?) \textit{ponattu}, \textit{prasantanceytarulina tirucakam ponattu tirumukapati}, \textit{kataippati}, \textit{talaiyakappati})
 voir \textsc{Veluthat} (1993: 74).} \og venu\fg, en tant que \textit{devad\=ana} une terre\index{gnl}{terre} de vingt-six \textit{v\=eli} et deux \textit{m\=a} situ\'ee dans Tiruv\=ali\index{cec}{Tiruvalin@Tiruv\=alin\=a\d tu}. D'autres terres\index{gnl}{terre} et une autre divinit\'e, U\d taiy\=ar Tirukko\d tikk\=a U\d taiy\=ar, sont mentionn\'ees mais leur fonction dans la transaction reste incompr\'ehensible.

Le deuxi\`eme changement est effectu\'e au profit du Seigneur propri\'etaire [terrien] de Tirutt\=o\d nipuram\index{gnl}{Tonipuram@T\=o\d nipuram} selon ce qui a \'et\'e dit \`a Cetikular\=ayar. Un \'echange a \'et\'e op\'er\'e entre les terres\index{gnl}{terre} de Virutar\=ayapaya\.nkarava\d lan\=a\d tu\index{cec}{Virutar\=ayapaya\.nkarava\d lan\=a\d tu} et le \textit{ti\b rappu} de Tiruv\=ali\index{cec}{Tiruvali@Tiruv\=ali}.

Enfin, le Seigneur propri\'etaire [terrien] de Tirutt\=o\d nipuram\index{gnl}{Tonipuram@T\=o\d nipuram} obtient une terre\index{gnl}{terre} \textit{devad\=ana} selon ce qui a \'et\'e dit \`a Ka\b nakar\=ayar. Une lecture litt\'erale du passage enregistrant cette transaction serait: \og en \'echange des villages qui ont \'et\'e \textit{cervill\=ata nilam\=ari}\index{cec}{nilam@\textit{nilam} terre}\footnote{Litt\'eralement \og confisqu\'es sans attache\fg\ mais le sens demeure myst\'erieux. Il en est de m\^eme pour les expressions similaires \textit{cervill\=amai m\=ari\b na} ou \textit{cervall\=amaiyil m\=ari}.} dans les villages \textit{devad\=ana} de Virutar\=ayapaya\.nkarava\d lan\=a\d tu\index{cec}{Virutar\=ayapaya\.nkarava\d lan\=a\d tu} est donn\'ee en tant que \textit{devad\=ana} \`a partir de la dixi\`eme une terre\index{gnl}{terre} \textit{ti\b rappu} de dix \textit{v\=eli} trois \textit{m\=akk\=a\d ni muntirikai k\=\i l m\=avarai} dans Va\d tak\=avirinall\=ur alias Etirilico\b la\b nma\d naku\d ti\index{cec}{Etirilico\b la}\fg.

\subsubsection*{9.3.2 Seconde partie}

La dixi\`eme ann\'ee [de r\`egne] de R\=ajar\=ajadeva, empereur des trois mondes,
par la gr\^ace d'\=Adica\d n\d de\'svara\index{cec}{adicandesvara@\=Adica\d n\d de\'svaradeva} du temple\index{gnl}{temple} du Seigneur propri\'etaire de Tirutt\=o\d nipuram\index{gnl}{Tonipuram@T\=o\d nipuram!Tirutt\=o\d nipuram}, \`a Tirukka\b lumalam\index{cec}{Tirukkalumalam@Tirukka\b lumalam} dans le Tirukka\b lumalan\=a\d tu\index{cec}{Tirukka\b lumalan\=a\d tu} du R\=aj\=adhir\=ajava\d lan\=a\d tu\index{cec}{Rajadhirajavala@R\=aj\=adhir\=ajava\d lan\=a\d tu}, les employ\'es du temple\index{gnl}{temple} (ceux qui font la surveillance \textit{\'sr\=\i mahe\'svara}\index{cec}{srimahesvara@\textit{\'sr\=\i mahe\'svara} dévot, surveillant}, celui qui fait \textit{\'sr\=\i k\=ariyam}\index{cec}{srikariyamcey@\textit{\'sr\=\i k\=ariyam cey} employé du temple}, et le \textit{devakarm\=\i} comptable du temple\index{gnl}{temple}) pr\'ecisent aux m\'etayers du \textit{devad\=ana} de Va\d tak\=avirinall\=ur alias Etirilico\b la\b nma\d naku\d ti comment les taxes doivent \^etre pay\'ees.
Les comptes se r\'ef\`erent \`a un ordre\index{gnl}{ordre royal} royal du trois cent cinquante sixi\`eme jour de la neuvi\`eme ann\'ee. La phrase finale atteste que cette taxation sur une terre\index{gnl}{terre} de plus de seize \textit{v\=eli} est valable pour l'\'eternit\'e et qu'elle fut grav\'ee sur pierre. Ont sign\'e, par la gr\^ace d'\=Adica\d n\d de\'svaradeva\index{cec}{adicandesvara@\=Adica\d n\d de\'svaradeva} du temple\index{gnl}{temple} du Seigneur propri\'etaire de Tirutt\=o\d nipuram\index{gnl}{Tonipuram@T\=o\d nipuram!Tirutt\=o\d nipuram}, les surveillants \textit{\'sr\=\i mahe\'svara}\index{cec}{srimahesvara@\textit{\'sr\=\i mahe\'svara} dévot, surveillant} du temple\index{gnl}{temple} de ce Seigneur, celui qui fait \textit{\'sr\=\i k\=ariyam}\index{cec}{srikariyamcey@\textit{\'sr\=\i k\=ariyam cey} employé du temple}, le \textit{devakarm\=\i} comptable du temple\index{gnl}{temple}, ainsi que le comptable du temple\index{gnl}{temple} un propriétaire [terrien] de Vir\=a\d nam, le comptable du temple\index{gnl}{temple} un propriétaire [terrien] de Tiruni\b n\b ra(v)\=ur, le comptable du temple\index{gnl}{temple} un propriétaire [terrien] de Pu\.nk\=ur, le comptable du temple\index{gnl}{temple} un propriétaire [terrien] de T\=al\=ur, le \textit{kanmi} N\=a\b rpatte\d n\d n\=ayirapa\d t\d tan comptable Po\b rkoyilpa\d t\d ta\b n, le \textit{tevaka\b nmi} Mu\d tiva\b la\.nkuco\b lapa\d t\d ta\b n et le surveillant \textit{\'sr\=\i mahe\'svara}\index{cec}{srimahesvara@\textit{\'sr\=\i mahe\'svara} dévot, surveillant} comptable un propriétaire [terrien] de Ka\d niccaipp\=akkam.\\

\section*{CEC 10}
\subsection*{CEC 10.1 Remarques}

L'inscription, relev\'ee dans l'ARE 1918 390, est grav\'ee sur le mur nord de la premi\`ere enceinte. Elle date de la dix-huiti\`eme ann\'ee de Tribhuvanacakravartin R\=ajar\=ajadeva. Les donn\'ees astronomiques ont permis \`a L'ARE 1918, appendix E, et ensuite \`a \textsc{Mahalingam} (1992: 551, Tj. 2417) d'identifier le roi\index{gnl}{roi} comme R\=ajar\=aja III\index{gnl}{Rajaraja III@R\=ajar\=aja III} et de proposer la date suivante: \textbf{mercredi 11 janvier 1234}\footnote{Cette date est confirm\'ee par le programme informatique \og Pancanga\fg\ mis en place par MM. \textsc{Yano} et \textsc{Fushimi} et disponible sur \ttfamily{$<$www.kyoto-su.ac.jp/$\sim$yanom/pancanga$>$}.}.

L'\'epigraphe se compose de huit lignes qui couvrent trois m\`etres soixante-dix. Elle est d\'et\'erior\'ee par endroit et recouverte de peinture. Seule la transcription de l'ASI a \'et\'e examin\'ee pour notre \'edition du texte.

L'inscription enregistre la donation par un brahmane\index{gnl}{brahmane} de N\=al\=ur\index{cec}{Nalur@N\=al\=ur} d'au moins cinq terres\index{gnl}{terre} pour \'etablir un jardin \`a fleurs pour \'Siva\index{gnl}{Siva@\'Siva}.

\subsection*{CEC 10.2 Texte}
\begin{enumerate}
	\item \textbf{svasti \'sr\=\i} tiripuva\b naccakkaravattika\d l\index{cec}{Tribhuvanacakravarti} \textbf{\'sr\=\i r\=ajar\=ajade}va\b rku y\=a\d n\d tu 10 8 vatu makara n\=aya\b r\b ru p\=urva pak\d sattu da\'samiyum 			puta\b n ki\b lamaiyum pe\b r\b ra u\dots\ \textbf{j\=adhir\=aja}va\d lan\=a\d t\d tu tirukka\b lumalan\=a\d t\d tu\index{cec}{Tirukka\b lumalan\=a\d tu} \dots\ caturvetima\.nkalattukki\b l pi\d t\=akai\index{cec}{pitakai@\textit{pi\d t\=akai} hameau} pa\d nama\.n\dots\
	\item tiru\~n\=a\b nacampantavatikku\index{cec}{tirunana@Tiru\~n\=a\b nacampanta\b n}\index{cec}{vati@\textit{vati}} ki\b lakku etirilico\b lav\=aykk\=alukku\index{cec}{Etirilico\b la}\index{cec}{vaykkal@\textit{v\=aykk\=al} canal} va\d takku 6 C 1 catirattu N AAAA itil te\b nme\b rka\d taiya om\=am puliy\=u\b rp\=a\b rpati p\=a\dots lai ko\d n\d ta N 1A ivvatikkuk\index{cec}{vati@\textit{vati}} ki\b lakku etirilico\b lav\=aykk\=alukku\index{cec}{Etirilico\b la}\index{cec}{vaykkal@\textit{v\=aykk\=al} canal} va\d takku ikka\d n\d n\=a\b r\b ru 4 catirattu ki\b lakka \dots\
	\item paca\d lait tirucci\b r\b rampalamu\d taiy\=a\b n\index{cec}{tiruccirrampalamutaiyar@Tirucci\b r\b rampalamu\d taiy\=ar} pa\d t\d ta\b n u\d l\d li\d t\d t\=ar pakkal\index{cec}{pakkal@\textit{pakkal} auprès de} vilai ko\d n\d ta\index{cec}{vilai@\textit{vilai} prix!vilai-kol@\textit{vilai-ko\d l} acheter} N AAAm \=al\=alacuntaravatikkuk\index{cec}{alala@\=Al\=alacuntara}\index{cec}{vati@\textit{vati}} ki\b lakku aca\~ncala\b n\dots kku va\d takku N 4 catirattu me\b rka-\d taiya N 4A nikkik ki\b lakka\d taiya kauvu\b naccappu pa\d t\d ta\b n pakkal\index{cec}{pakkal@\textit{pakkal} auprès de} vilai ko\d n\d tu\index{cec}{vilai@\textit{vilai} prix!vilai-kol@\textit{vilai-ko\d l} acheter} \dots\
	\item ita\b n ki\b lakku N 6A ki\b lakka\d taiya iva\b n pakkal\index{cec}{pakkal@\textit{pakkal} auprès de} vilai ko\d n\d ta N AAm ivvatikkuk\index{cec}{vati@\textit{vati}} ki\b lakku ivv\=aykk\=alukku\index{cec}{vaykkal@\textit{v\=aykk\=al} canal} va\d takku 2ntu\d n\d tattu N\dots\ me\b rka\d taiya N 1A nikki ita\b n ki\b lakka\d taiya pr\=ant\=ur p\=atapatip\=aka pa\d t\d ta\b n u\d l\d li\d t\d t\=ar pakkal\index{cec}{pakkal@\textit{pakkal} auprès de} vilaiko\d n\d ta\index{cec}{vilai@\textit{vilai} prix!vilai-kol@\textit{vilai-ko\d l} acheter} N\dots\
	\item N veliyum u\d taiy\=ar tirutto\d nipuramu\d taiya\index{cec}{Tiruttoni@Tirutt\=o\d nipuramu\d taiya n\=aya\b n\=ar, \'Siva} n\=aya\b n\=a\b rkut tirunantava\b nappu\b ra-m\=akka\index{cec}{puram@\textit{pu\b ram} terre de donation!\textit{nantava\b nappu\b ram} terre donnée pour créer un jardin à fleurs} ku\d tutte\b n n\=al\=ur\index{cec}{Nalur@N\=al\=ur} m\=atevapa\d t\d ta\b ne\b n\index{cec}{Mateva@M\=atevapa\d t\d ta\b n} to\d t\d ticeykku ki\b lakku o\d tampokikku me-\b rku campantaperum\=a\b ne\b n\b ru per k\=uvap pa\d t\d ta tirunantava\b nam ku\b li 2 100 5 10 ikku\b li iru \dots\
	\item aimpatum e\b n cant\=a\b natt\=aril e\b n ka\b nu\d sa\b n tillain\=ayakarum\index{cec}{Tillain\=ayakar} m\=a\d nikkakk\=uttarum\index{cec}{manikka@M\=a\d nikkakk\=uttar} intat tirunantava\b nam ceytu u\d taiy\=ar tirutto\d nipuramu\d taiyan\=aya\b n\=ar\index{cec}{Tiruttoni@Tirutt\=o\d nipuramu\d taiya n\=aya\b n\=ar, \'Siva} tirupp\=u-ma\d n\d tapatte tiruppa\d l\d litt\=amam\index{cec}{tiruppallitamam@\textit{tiruppa\d l\d litt\=amam} guirlande du coucher} pa\d nim\=a\b ravum ivarka\d lukku \textbf{j\=\i }va\b n\=a\textbf{\'se\d sa}m\=akak-ku\d tutte \dots\
	\item mikutike \dots m ik\=a\d t\d tuppallum ko\d n\d tu ittirunantava\b nam ceyyum tillain\=ayaka\b r-kum\index{cec}{Tillain\=ayakar} m\=a\d nikkakk\=utta\b rkum\index{cec}{manikka@M\=a\d nikkakk\=uttar} \dots\ te\b n ivaka\d lukkup pi\b npu cim\=ake\textbf{\'sva}rare\index{cec}{srimahesvara@\textit{\'sr\=\i mahe\'svara} dévot, surveillant} tirutto\d n\d tu \dots\
	\item n\=al\=ur\index{cec}{Nalur@N\=al\=ur} \dots \b ne\b n
\end{enumerate}

\subsection*{CEC 10.3 R\'esum\'e}
\og Que la prosp\'erit\'e soit\fg! En la 18\up{e} ann\'ee [de r\`egne] de \'Sr\=\i r\=ajar\=ajadeva, empereur des trois mondes, le mois de \textit{Makara}, le dixi\`eme jour de la quinzaine claire, mercredi, \dots\ des terres\index{gnl}{terre}, au moins cinq (pour un total d'une \textit{v\=eli} l.~5), acquises aupr\`es de diff\'erentes personnes, dans le hameau Pa\d nama\.n\dots\ \`a l'est du \dots caturvetima\.nkalam
dans le Tirukka\b lumalan\=a\d tu\index{cec}{Tirukka\b lumalan\=a\d tu} du R\=aj\=adhir\=ajava\d lan\=a\d tu\index{cec}{Rajadhirajavala@R\=aj\=adhir\=ajava\d lan\=a\d tu}, ont \'et\'e donn\'ees au Seigneur propri\'etaire de Tirutt\=o\d nipuram\index{gnl}{Tonipuram@T\=o\d nipuram!Tirutt\=o\d nipuram} par le donateur M\=atevapa\d t\d ta\b n\index{cec}{Mateva@M\=atevapa\d t\d ta\b n} de N\=al\=ur\index{cec}{Nalur@N\=al\=ur}\footnote{Nous n'avons trouv\'e aucune information sur ce donateur. Mais il est int\'eressant de constater que les terres\index{gnl}{terre} acquises par ce brahmane\index{gnl}{brahmane} appartenaient \`a d'autres brahmane\index{gnl}{brahmane}s de ce hameau de Tirukka\b lumalan\=a\d tu\index{cec}{Tirukka\b lumalan\=a\d tu}.} pour faire un jardin.

De plus, ce dernier donne une terre\index{gnl}{terre} de 250 \textit{ku\b li}, nomm\'ee Campantaperum\=a\b n et situ\'ee \`a l'est de la terre\index{gnl}{terre} \textit{to\d t\d ticey} et \`a l'ouest du petit canal (\textit{o\d tampoki}), \`a Tillain\=ayakar\index{cec}{Tillain\=ayakar} et M\=a\d nikkakk\=uttar\index{cec}{manikka@M\=a\d nikkakk\=uttar}\footnote{Leur parent\'e est peut-\^etre signifi\'ee par le terme \textit{ka\b nu\d sa\b n} dont le sens est obscur.}, parmi sa descendance, en tant que terre\index{gnl}{terre} pour vivre. Ces derniers sont tenus \`a cultiver le jardin et \`a fournir en guirlandes le \textit{ma\d n\d tapam} de fleurs du N\=aya\b n\=ar propri\'etaire de Tirutt\=o\d nipuram\index{gnl}{Tonipuram@T\=o\d nipuram!Tirutt\=o\d nipuram}. La fin du texte précise qu'apr\`es eux, apr\`es leur mort (?), ce service\index{gnl}{service} au temple\index{gnl}{temple} devra \^etre pris en charge par les \textit{\'sr\=\i mahe\'svara}\index{cec}{srimahesvara@\textit{\'sr\=\i mahe\'svara} dévot, surveillant}.

\section*{CEC 11}
\subsection*{CEC 11.1 Remarques}

L'\'epigraphe, relev\'ee dans l'ARE 1918 389, est situ\'ee sur le mur nord de la premi\`ere enceinte, au-dessus de CEC 9. Elle date de la vingt-quatri\`eme ann\'ee de Tribhuvanacakravartin R\=ajar\=ajadeva. \textsc{Mahalingam} (1992: 551, Tj. 2418) identifie ce roi\index{gnl}{roi} comme R\=ajar\=aja III\index{gnl}{Rajaraja III@R\=ajar\=aja III} et date le texte de \textbf{1240}.

Le texte de l'inscription --- comportant quatre lignes sur cinq m\`etres trente --- que nous proposons est \'etablie sur le seul examen de la transcription de l'ASI.

Le texte enregistre un don\index{gnl}{don} de terre\index{gnl}{terre} par un natif d'\=A\b na\.nk\=ur afin de faire un jardin \`a fleurs pour \'Siva\index{gnl}{Siva@\'Siva}.

\subsection*{CEC 11.2 Texte}
\begin{enumerate}
	\item \textbf{svasti \'sr\=\i} \textbf{tribhu}va\b na\textbf{\'scakra}vattika\d l \textbf{\'sr\=\i r\=ajar\=ajade}varkku y\=a\d n\d tu 2 10 4 kumpa n\=aya\b r\b ru p\=urva pak\d sattu pa\~ncamiyum ti\.nka\d l ki\b lamaiyum pe\b r\b ra a\dots ttu n\=a\d l \textbf{brahmade\'sa}m\index{cec}{brahmadeya@\textit{brahmadeya}} tirukka\b lumalattu\index{cec}{Tirukkalumalam@Tirukka\b lumalam} kavu\d niya\b n\index{cec}{kavuniyan@\textit{kavu\d niya\b n gotra}} civatavanav\=a\textbf{sa}\index{cec}{Civatavanav\=asa} tiruva\textbf{gni-\'sva}ramu\d taiy\=anum\index{cec}{Tiruvagni\'svaramu\d taiy\=a\b n} iva\b n tampi tiruna\d t\d tapperum\=anum\index{cec}{tirunatta@Tiruna\d t\d tapperum\=an} iva\b n p\=aryai \=a\d n\d tana\.n-kaicc\=anip\index{cec}{antanankai@\=A\d n\d tana\.n-kaicc\=ani}
	\item pakkal\index{cec}{pakkal@\textit{pakkal} auprès de} na\d tuviln\=a\d t\d tu\index{cec}{natuvil@Na\d tuviln\=a\d tu} \=a\b n\=a\.nk\=ur\index{cec}{anankur@\=A\b n\=a\.nk\=ur} vayiranall\=u\b l\=a\b n\index{cec}{Vayiranall\=u\b l\=a\b n} araiya\b n\index{cec}{Araiya\b n} pu\b r\b ri\d ta\.nko\d n\d t\=a\b n in\b rai n\=a\d lil k\=acu\index{cec}{kacu@\textit{k\=acu} pièces de monnaie} 2 1000kku vilaiko\d n\d tu\index{cec}{vilai@\textit{vilai} prix!vilai-kol@\textit{vilai-ko\d l} acheter} u\d taiy\=ar tirutto\d nipuramu\d taiy\=arkku\index{cec}{Tiruttoni@Tirutt\=o\d nipuramu\d taiya n\=aya\b n\=ar, \'Siva} tiruppa\d l\d lit-t\=amat\index{cec}{tiruppallitamam@\textit{tiruppa\d l\d litt\=amam} guirlande du coucher} tirunantava\b nam\=aka tirun\=amattu tirivi\d t\d tuppukunta piram\=a\d nappa\d ti\index{cec}{piramanam@\textit{pram\=a\d nam} document} cuttamalivatikku\index{cec}{Cuttamali}\index{cec}{vati@\textit{vati}} ki\b lakku ir\=aje
	\item ndiraco\b lav\=aykk\=alukku\index{cec}{vaykkal@\textit{v\=aykk\=al} canal} va\d takku ira\d n\d t\=a\.nka\d n\d n\=a\b r\b ru mutal catirattu te\b rka\d tainta \=a\b ru m\=avil k\=araik kollaiyil te\b nki\b lakka\d taiya turavil p\=ati u\d tpa\d tak ko\d n\d tu vi\d t\d ta kollai ku\b li 2 100 6 10 ikku\b li i\b run\=u\b r\b ru a\b rupattum tirunantava\b nappu\b ram\=aka\index{cec}{puram@\textit{pu\b ram} terre de donation!\textit{nantava\b nappu\b ram} terre donnée pour créer un jardin à fleur} vi\d t\d tamaikku \=a\b n\=a\.nk\=u\b ru
	\item \d taiy\=a\b n\index{cec}{anankur@\=A\b n\=a\.nk\=ur} araiya\b n\index{cec}{Araiya\b n} pu\b r\b ri\d ta\.nko\d n\d t\=a\b n e\b luttu U
\end{enumerate}

\subsection*{CEC 11.3 Traduction}
Que la prosp\'erit\'e soit! En la 24\up{e} ann\'ee [de r\`egne] de \'Sr\=\i r\=ajar\=ajadeva, empereur des trois mondes, le mois de \textit{Kumpa}, le cinqui\`eme jour de la quinzaine claire, lundi, dans [le \textit{nak\d satra}] \dots \footnote{Les donn\'ees astronomiques ne permettent pas de v\'erifier la date exacte. En effet, il semblerait qu'il y ait une erreur d'apr\`es l'ARE 1918, appendix E. Toutefois avec le programme \og Pancanga \fg\ nous pouvons sugg\'erer le lundi 30 janvier 1240.}

Aupr\`es du \textit{kavu\d niya\b n}\index{cec}{kavuniyan@\textit{kavu\d niya\b n gotra}} Civatavanav\=asa Tiruvagni\'svaramu\d taiy\=a\b n\index{cec}{Tiruvagni\'svaramu\d taiy\=a\b n} du \textit{brahmadeya}\index{cec}{brahmadeya@\textit{brahmadeya}} Tirukka\b lumalam\index{cec}{Tirukkalumalam@Tirukka\b lumalam}, de son fr\`ere cadet Tiruna\d t\d tapperum\=an et de son \'epouse \=A\d n\d tana\.n-kaicc\=ani\footnote{La terre\index{gnl}{terre} a \'et\'e acquise aupr\`es de brahmane\index{gnl}{brahmane}s qui appartiennent au \textit{gotra\index{gnl}{gotra@\textit{gotra}}} Kavu\d niya\b n\index{cec}{kavuniyan@\textit{kavu\d niya\b n gotra}} (cf. \textsc{Orr} 2004: n. 7 sur ce \textit{gotra\index{gnl}{gotra@\textit{gotra}}}). Tiruvagn\=\i\'svaram est tr\`es probablement la localit\'e, le temple\index{gnl}{temple} ou le \textit{li\.nga}\index{gnl}{linga@\textit{li\.nga}}, situ\'e \`a quelques centaines de m\`etres au nord-est de C\=\i k\=a\b li\index{gnl}{Cikali@C\=\i k\=a\b li}.}, Pu\b r\b ri\d ta\.nko\d n\d t\=a\b n Vayiranall\=u\b l\=a\b n Araiya\b n\index{cec}{Araiya\b n} d'\=A\b n\=a\.nk\=ur\index{cec}{anankur@\=A\b n\=a\.nk\=ur} dans le Na\d tuviln\=a\d tu a achet\'e, aujourd'hui, pour 2000 \textit{k\=acu}\index{cec}{kacu@\textit{k\=acu} pièces de monnaie}, et a donn\'e [au total] un verger de 260 \textit{ku\b li} en tant que jardin pour [faire] des guirlandes pour le Seigneur propri\'etaire de Tirutt\=o\d nipuram\index{gnl}{Tonipuram@T\=o\d nipuram!Tirutt\=o\d nipuram}, selon le document\footnote{Ce document est qualifi\'e de \textit{tirun\=amattu tirivu i\d t\d tup pukunta} mais sa signification demeure inconnue.}.
[Le jardin est situ\'e] \`a l'est de la \textit{vati}\index{cec}{vati@\textit{vati}} Cuttamali\index{cec}{Cuttamali}, au nord du canal Ir\=ajendiraco\b la, dans le premier carr\'e du deuxi\`eme canalicule, et inclut la moiti\'e du puits [qui se trouve] au sud-est dans le verger \textit{k\=arai} des six \textit{m\=a} au sud.
 Pour le don\index{gnl}{don} de cette terre\index{gnl}{terre} de jardin de deux cent soixante \textit{ku\b li} a sign\'e Pu\b r\b ri\d ta\.nko\d n\d t\=a\b n Araiya\b n\index{cec}{Araiya\b n} un propriétaire [terrien] d'\=A\b n\=a\.nk\=ur\index{cec}{anankur@\=A\b n\=a\.nk\=ur}\footnote{\`A notre connaissance\index{gnl}{connaissance}, cet individu ne figure pas dans d'autres \'epigraphes. Il est un propri\'etaire terrien poss\'edant le titre d'Araiya\b n\index{cec}{Araiya\b n}. Cependant, il est curieux de remarquer que ce donateur est le seul signataire de l'acte. Cela sugg\`ere-t-il qu'il jouissait d'une quelconque autorit\'e au niveau du temple\index{gnl}{temple} ou de la localit\'e\string?}.

\section*{CEC 12}
\subsection*{CEC 12.1 Remarques}

L'inscription, relev\'ee dans l'ARE 1918 391, est situ\'ee sur le mur sud de l'enceinte principale. Elle date du r\`egne de \og Chakravartin Peru\~nji\.ngad\=eva\fg\ que \textsc{Mahalingam} (1992: 552, Tj. 2425) identifie comme roi\index{gnl}{roi} pallava le \textit{K\=a\d tavar} K\=opperu\~n-ci\.nka II\index{gnl}{Kopperuncinka II@K\=opperu\~nci\.nka II} en proposant la \textbf{date approximative de 1243}\footnote{K\=opperu\~nci\.nka, qui se revendique de la dynastie \textit{pallava}\index{gnl}{pallava@\textit{pallava}}, a affront\'e et emprisonn\'e R\=ajar\=aja III\index{gnl}{Rajaraja III@R\=ajar\=aja III}. Il semble qu'il y ait eu deux chef\index{gnl}{chef}s de ce nom mais les informations dont nous diposons à leur sujet sont limit\'ees. Cf. \textsc{Nilakanta Sastri} (*1998 [1975]: 7, 213-214) et \textsc{Younger} (1995: 142-143) pour un bref compte rendu de leurs activit\'es \`a Citamparam\index{gnl}{Citamparam}.
% Voir aussi \textsc{Srinivasan} \&\ \textsc{Reiniche} (1990: 29-30).
}. Elle a \'et\'e publi\'ee dans SII 12 253 avec le commentaire suivant:
\og In this damaged inscription the regnal year is lost. Some of the inscribed slabs are also missing. It seems to record the gift of a garden, free of taxes, in \=Akk\=ur\index{gnl}{Akkur@\=Akk\=ur}, to the \textit{Pa\d dimatt\=ar} of the temple\index{gnl}{temple} of Mah\=a\'s\=asta\b n Peruv\=embu\d daiy\=ar by (the authorities) of the temple\index{gnl}{temple} of Tirutt\=o\d nipuram\index{gnl}{Tonipuram@T\=o\d nipuram!Tirutt\=o\d nipuram}u\d daiy\=ar\fg.

Cette inscription contient dix-neuf lignes sur trois m\`etres. La derni\`ere ligne manque dans la publication et elle est illisible sur la pierre. Nous n'avons pas retrouv\'e la transcription \`a Mysore.

Trois des employ\'es du temple\index{gnl}{temple} qui figurent parmi les signataires apparaissent aussi dans CEC 9 l.~6. Ainsi, nous pensons que ce texte date de la premi\`ere moiti\'e du \textsc{xiii}\up{e} si\`ecle.

\subsection*{CEC 12.2 Texte}
\begin{enumerate}
	\item \dots ravattika\d l \textbf{\'sr\=\i}\dots peru\~nci\.nkateva\b rkku\footnote{\dots ravattika\d l \textbf{\'sr\=\i}\dots peru\~nci\.nkateva\b rkku: en gras dans la publication.} y\=a\d n\d tu \dots [n\=aya\b r\b ru] \textbf{p\=urvapak\d sa}ttu \textbf{pratha}[maiyum] ca\b nikki\b lamaiyum pe\b r\b ra [p\=uca]ttu\b n\=a\d l \textbf{\'sr\=\i}r\=a\textbf{j\=adhi}r\=a\textbf{ja}va\d lan\=a\d t-\d tut
	\item \dots \d tay\=ar \dots\footnote{SII 12 253 note: \og The gap may be filled up with the letters \textit{tirutt\=o\d ni}\fg.}puramu\d taiy\=ar ko[yilk]ka\d nakka\index{cec}{kanakku@\textit{ka\d nakku} comptable} \dots\ tirukka\d talukku e\b lu\textbf{nta}ru\d lic ceyta amaitta\b n\=ar\=aya
	\item \dots kat\dots\ c\=atta\b n peruvempu\d taiya \dots \b rkku icaivuti\d t\d tuk ku\d tutta paric\=avatu [\danda*] u\d taiy\=ar tiru
	\item \dots y\=a\dots\ tiruv\=aliy\=a\b na\index{cec}{Tiruvali@Tiruv\=ali} etirilico\b laccatu\textbf{rvve}[tima\.n]kala\index{cec}{caturvedimangalam@Caturvedima\.ngalam!Etirilic\=o\b laccaturvedima\.ngalam} \dots\ pi\b rinta e\d nveli \=akk\=uril \=atama\.n[ka]\footnote{SII 12 253 note: \og The letter \textit{ka} is engraved below the line\fg. L'\textit{ak\d sara} \textit{ka} est en \'ecriture tamoule dans la publication.}la virac[o]\b la\b n\=ar
	\item \dots tta\dots vempu\d taiy\=ar tirun\=amattukk\=a\d niy\=akak\index{cec}{kani@\textit{k\=a\d ni} droit, propriété} ko\d n\d ta a\dots \b n\b ru perk\=uvappa\d t\d ta \b nilam\index{cec}{nilam@\textit{nilam} terre} pottakappa\d ti N AAAA i\b n
	\item \dots\ araikk\=a\d nikkum tev\=at\=a\b nakka\d tamaiyum\index{cec}{katamai@\textit{ka\d tamai} taxe foncière} puravu\dots\ i\b ru\dots maiyum ceyt\=al pe-\b ruvato\b n\b rumillai e\b n\b rum
	\item \dots n\b n\=a\dots\ vi[\d lai]nilattile \=atalum \dots y\=ar e\b luntaru\d li iru\dots taiva\dots \b r\b ra i\b raiyiliy\=akac\index{cec}{iraiyili@\textit{i\b raiyili} non imposable} cilanilam\index{cec}{nilam@\textit{nilam} terre} pe\b rave\d num ye\b n\b rum i\b nta
	\item {[r\=aj\=adhir\=ajava]\d la\b n\=a\d t\d tu\index{cec}{Rajadhirajavala@R\=aj\=adhir\=ajava\d lan\=a\d tu}\footnote{Conjecture personnelle.} tirukka\b luma[la]\index{cec}{Tirukkalumalam@Tirukka\b lumalam}\dots\ ippa\d ti \dots vane\b n\b rum aru\d liccekai \dots ya mu\b rku\b ritta ira\d n\d tu v[e]li \dots\ pottakappa\d ti nilam\index{cec}{nilam@\textit{nilam} terre} e\d t\d tum\=akk\=a}
	\item \dots ra\d natte\dots rukka\b lumalattu\index{cec}{Tirukkalumalam@Tirukka\b lumalam} tirun\=amattukk\=a\d niyil\index{cec}{kani@\textit{k\=a\d ni} droit, propriété} iva\b rkku k\=a\d niyum\index{cec}{kani@\textit{k\=a\d ni} droit, propriété} i\b raiyili-yu(m)m\=aka\index{cec}{iraiyili@\textit{i\b raiyili} non imposable} [vi\d t\d ta] cuttamallivatikku\index{cec}{Cuttamali}\index{cec}{vati@\textit{vati}} ki\b lakkum r\=a\textbf{jendra}co\b lav\=aykk\=a\index{cec}{vaykkal@\textit{v\=aykk\=al} canal}
	\item \dots\ c\=a\textbf{sta}\b nperu[vempu\d taiy\=a]r\footnote{Conjecture personnelle.}[tiru][k*]koyil\index{cec}{koyil@\textit{k\=oyil} temple} vi\d l\=akakka\d n\d tattu yintap peru-vempuu\d taiy\=ar e\b luntaru[\d liyi]runta koyillum\index{cec}{koyil@\textit{k\=oyil} temple} tirumu\b r\b rattukkum pe\dots kkum yi\b n
	\item {[ta]paric\=avatu \dots m\=aka vi\d t\d ta \=ur N ku\b li 2 100 yikku\b li yirun\=u\b r\b r\=\i \b n\=al N AAAAA yinilam\index{cec}{nilam@\textit{nilam} terre} k\=a\d ni\index{cec}{kani@\textit{k\=a\d ni} droit, propriété} araikk\=a\d nikki\b lk\=alum yi\b nta \dots\ c\=atta\b n peruve \dots }
	\item tu ivar e\b luntaru\d li\dots m\=atto[\d t]\d tam\=aka ku\d tuttom [\danda*] in\b nilattu\index{cec}{nilam@\textit{nilam} terre} me\b nokki\b na maramum ki\b lnokki\b na ki\d na\b rum ma\b r\b rum eppe\b rpa\d t\d ta urimaika\d lum a
	\item kappa\d tak\dots vut\=akavum [\danda*] i\b nilattukku\index{cec}{nilam@\textit{nilam} terre} [o\d taiyil\=e]\dots\ payi\b rkko[\d l\d lakka\d ta]vat\=a-kavum [\danda*] N k\=a\d ni araikk\=a\d nik ki\b lkk\=alum cantir\=atittavaraiyum yivar \dots
	\item yi\b raiyiliyum\index{cec}{iraiyili@\textit{i\b raiyili} non imposable} \dots cammatittu icaivuti\d t\d tuk ku\d tuttom ma\textbf{h\=a}c\=a[\textbf{sta}]\b n peruvem-pu\d taiy\=ar koyilp\index{cec}{koyil@\textit{k\=oyil} temple} pa\d timatt\=a\b rku\footnote{\textit{pa\d timatt\=a\b rku}: en gras dans la publication.} u\d taiy\=ar tirutto\d nipuramu\d taiy\=ar\index{cec}{Tiruttoni@Tirutt\=o\d nipuramu\d taiya n\=aya\b n\=ar, \'Siva} tirukkoyi\dots
	\item tom [\danda*] ippa\d ti \dots kka\d nakku\index{cec}{kanakku@\textit{ka\d nakku} comptable} ka\d np\=uru\d taiy\=a\b n\index{cec}{kanpur@Ka\d np\=uru\d taiy\=a\b n} cim\=a\textbf{he\'sva}rappiriya\b n\index{cec}{Cim\=ahe\'svarappiriya\b n} e\b luttu [\danda*] ippa\d tikku ivai koyil\index{cec}{koyil@\textit{k\=oyil} temple} ka\d nakku\index{cec}{kanakku@\textit{ka\d nakku} comptable} vir\=a\d namu\d taiy\=a\b n\index{cec}{viranam@Vir\=a\d namu\d taiy\=a\b n} tiruvekampappiriya\b n e\b luttu [\danda*] ippa\d tikku i\dots
	\item ka\b nmi po\b r[koyilpa\d t\d ta\b n\index{cec}{porkoyil@Po\b rkoyilpa\d t\d ta\b n} e]\b luttu\footnote{Conjecture personnelle car le nom de cet employ\'e du temple\index{gnl}{temple} se trouve dans la seconde partie de CEC 9. l.~6.} [\danda*] ippa\d tikku ivai koyil(k)\index{cec}{koyil@\textit{k\=oyil} temple} ka\d nakku\index{cec}{kanakku@\textit{ka\d nakku} comptable} vir\=a\d na-mu\d taiy\=a\b n tirutto\d nipuramu\d taiy\=an\index{cec}{Tonipuramutaiyan@T\=o\d nipuramu\d taiy\=a\b n} e\b luttu [\danda*] ippa\d tikku ivai vir\=a\d namu\d taiy\=a\b n pa\b naitta\b lumpa\b n\index{cec}{panaittal@Pa\b naitta\b lumpa\b n} e\dots
	\item \d tikku ivai ka\dots\ e\b luttu [\danda*] ippa\d tikku ivai koyilk\index{cec}{koyil@\textit{k\=oyil} temple} ka\d nakku\index{cec}{kanakku@\textit{ka\d nakku} comptable} tiruni\b n\b rav\=uru\d taiy\=a\b n\index{cec}{Tiruni\b n\b rav\=uru\d taiy\=a\b n} e\b luttu [\danda*] ippa\d tikku ivai tevarka\b nmi mu\d tiva\b la\.nkuco\b lapa\d t\d ta\b n\index{cec}{mutivalan@Mu\d tiva\b la\.nkuco\b lapa\d t\d ta\b n} e\b luttu [\danda*] ippa\d ti \dots
	\item varka\b nmi til\dots \b n e\b luttu [\danda*] i \dots va cim\=a\textbf{he\'sva}raka\d nk\=a\d ni\index{cec}{kani@\textit{k\=a\d ni} droit, propriété}\index{cec}{srimahesvara@\textit{\'sr\=\i mahe\'svara} dévot, surveillant} koyil\index{cec}{koyil@\textit{k\=oyil} temple} ka\d nakku\index{cec}{kanakku@\textit{ka\d nakku} comptable} m\=am-p\=atta\b laiy\=a\b n e\b luttu [\danda*] ippa\d tikku ivai \textbf{\'sr\=\i}k\=ari\index{cec}{srikariyamcey@\textit{\'sr\=\i k\=ariyam cey} employé du temple}
\end{enumerate}

\subsection*{CEC 12.3 R\'esum\'e}

Le texte enregistre un don\index{gnl}{don} de terres\index{gnl}{terre} des employ\'es du temple\index{gnl}{temple} de Tirutt\=o\d nipuram\index{gnl}{Tonipuram@T\=o\d nipuram!Tirutt\=o\d nipuram} aux \textit{pa\d timatt\=ar} d'un temple\index{gnl}{temple} de C\=atta\b n pour que ce dernier parte en procession\index{gnl}{procession} jusqu'\`a la mer\index{gnl}{mer}. Ces terres\index{gnl}{terre} se trouvent dans Tirukka\b lumalam\index{cec}{Tirukkalumalam@Tirukka\b lumalam} et \`a \=Akk\=ur\index{gnl}{Akkur@\=Akk\=ur}, et auraient \'et\'e donn\'ees car elles n'\'etaient pas rentables pour le temple\index{gnl}{temple} de Tirutt\=o\d nipuram\index{gnl}{Tonipuram@T\=o\d nipuram!Tirutt\=o\d nipuram}.

Les employ\'es signataires sont le comptable du temple\index{gnl}{temple} Cim\=ahe\'svarappiriya\b n un propriétaire [terrien] de Ka\d np\=ur, le comptable du temple\index{gnl}{temple} Tiruvekampappiriya\b n un propriétaire [terrien] de Vir\=a\d nam, le \textit{ka\b nmi} Po\b rkoyilpa\d t\d ta\b n, le comptable du temple\index{gnl}{temple} Tirutt\=o\d nipuram\index{gnl}{Tonipuram@T\=o\d nipuram!Tirutt\=o\d nipuram}u\d taiy\=an un propriétaire [terrien] de Vir\=a\d nam, le comptable du temple\index{gnl}{temple} Pa\b naitta\b lumpa\b n un propriétaire [terrien] de Vir\=a\d nam, \dots , le comptable du temple\index{gnl}{temple} un propriétaire [terrien] de Tiruni\b n\b rav\=ur, le \textit{devakarm\=\i} Mu\d tiva\b la\.nkuco\b la-pa\d t\d ta\b n, \dots, le comptable du temple\index{gnl}{temple} et surveillant \textit{\'sr\=\i mahe\'svara}\index{cec}{srimahesvara@\textit{\'sr\=\i mahe\'svara} dévot, surveillant} M\=amp\=atta\b laiy\=a\b n, \dots

\section*{CEC 13}
\subsection*{CEC 13.1 Remarques}

L'inscription, relev\'ee dans ARE 1918 394, date de la dix-neuvi\`eme ann\'ee de r\`egne de \og Sakalabhuvanachakravartin K\=opperu\~nji\.ngad\=eva\fg\ que \textsc{Mahalingam} (1992: 552, Tj. 2423) identifie comme K\=opperu\~nci\.nka II\index{gnl}{Kopperuncinka II@K\=opperu\~nci\.nka II}. Elle se situe sur le mur est de la premi\`ere enceinte, au sud du pavillon d'entr\'ee. Elle a été publi\'ee dans SII 12 210 qui la date, selon les informations astronomiques, du \textbf{mercredi 24 janvier 1263}.

Le texte publi\'e est lacunaire et la conclusion en est cach\'ee par la construction du pavillon d'entr\'ee. Il se compose de onze lignes qui s'\'etendent sur environ deux m\`etres cinquante.

L'inscription enregistre une donation de terres\index{gnl}{terre} pour nourrir \'Siva\index{gnl}{Siva@\'Siva} par un certain Tevarka\d lteva\b n un propriétaire [terrien] de K\=u\d tal\=ur dans le Jaya\.nko\d n\d tac\=o\b lava\d lan\=a\d tu\index{cec}{Jayankontacolavalanatu@Jaya\.nko\d n\d tac\=o\b lava\d lan\=a\d tu}.

\subsection*{CEC 13.2 Texte}
\begin{enumerate}
	\item \textbf{svasti \'sr\=\i} \textbf{sa}kalapuva\b naccakkaravattika\d l\index{cec}{Sakalabhuvanacakravarti} \textbf{\'sr\=\i}kopperu\~nci\.nkatevarkku y\=a\d n\d tu 10 9 t\=avatu makaran\=aya\b r\b ru \textbf{p\=urvvapak\d sa}ttu catu\textbf{r\d da\'si}yum \textbf{budha}[\b n] \dots [\d tu] tiruk\dots
	\item \d tu ve\d n[\d naiy\=ur]n\=a\d t\d tu olaiy\=ama\.nkalattu k\=a\d ni\index{cec}{kani@\textit{k\=a\d ni} droit, propriété} u\d taiya \textbf{ja}ya\.nko\d n\d taco\b lava\d lan\=a\d t\d tu\index{cec}{Jayankontacolavalanatu@Jaya\.nko\d n\d tac\=o\b lava\d lan\=a\d tu} k\=u\d tal\=uru\d taiy\=a\b n tevarka\d l teva\b n\footnote{Le titre royal, l'ann\'ee de r\`egne et le nom du donateur sont en gras dans la publication.} \dots tanta [ni]\dots
	\item pallavarayar\index{cec}{Pallavar\=aya\b n} pakkal\index{cec}{pakkal@\textit{pakkal} auprès de} vilaiko\d n\d ta\index{cec}{vilai@\textit{vilai} prix!vilai-kol@\textit{vilai-ko\d l} acheter} k\=acil ve\d nku\b laya\b n e\b n\b ru peru\d taiya nilattukku\index{cec}{nilam@\textit{nilam} terre} ki\b lp\=a\b rkellai c\=u\d lai a\d tiv\=ay[k]\dots varampuk \dots
	\item inta \=a\d tko\d n\d tavilliye\b n nilattukku\index{cec}{nilam@\textit{nilam} terre} te\b rkkum inn\=a\b nkellaiu\d l na\d tuvupa\d t\d ta nila[m*]\index{cec}{nilam@\textit{nilam} terre} 2A itil ki\b lakka\d taiya ta\d ti o\b n\dots p\=a\d lacca \dots
	\item \.nka vi\d t\d tu paka\b l\=apara\d na\b n e\b n\b ru peru\d taiya nilattukku\index{cec}{nilam@\textit{nilam} terre} ki\b lp\=a\b rkellai per\=urki\b lava\b n periyatevar nilattukku\index{cec}{nilam@\textit{nilam} terre} me\b rkkum i\dots \d takkum \dots
	\item pper\=a\b r\b rukkum matur\=antakappe\b r\=a\b r\b rukkut te\b rku[m] i\b n\b n\=a\b nkellaiyu\d l na\d tuvupa\d t-\d ta N 1AAm te\b rkilppuka\b la\dots \b ne\b n\b ru pe\dots
	\item lukku va\d takkum melp\=a\b rkellai u\d tci\b ruv\=aykk\=alukkuk\index{cec}{vaykkal@\textit{v\=aykk\=al} canal} ki\b lakkum va\d tap\=a\b rkellai c\=u-ral\=ur ki\b lava\b n v\=aykk\=alukkut\index{cec}{vaykkal@\textit{v\=aykk\=al} canal} (t)e\dots
	\item \=ur vi\b lukk\=a\d t\d tuppa\d ti pottakam e\b r\b rivanta nilam\index{cec}{nilam@\textit{nilam} terre} inn\=ayan\=a\b rku tiruppari\dots\ amu-tuceytaru\d la tirun\=amattukk\=a\d ni\index{cec}{kani@\textit{k\=a\d ni} droit, propriété} \=aka ku\d tutte(ta)\b n [k\=u]\dots
	\item \b laiy\=uru\d taiy\=a\b n tirucci\b r\b rampalamu\d taiy\=a\b n\index{cec}{tiruccirrampalamutaiyar@Tirucci\b r\b rampalamu\d taiy\=ar} \=a\d tko\d n\d tan\=ayaka\b ne[\b n*]\index{cec}{atkonta@\=A\d tko\d n\d tan\=ayaka\b n} e\b lu[tte\b n\b rum] innilam\index{cec}{nilam@\textit{nilam} terre} intak k\=u\d tal\=uru\d taiy\=a\b n\index{cec}{Kutalur@K\=u\d tal\=uru\d taiy\=a\b n} \=a\d tko\d n\d tavil \dots
	\item aka\d la\.nkap\index{cec}{Aka\d la\.nka} pallavaraya\b ne \b ne\b n\b rum\index{cec}{Pallavar\=aya\b n} ippa\d ti a\b rive\b n [ku]\b laiy\=uru\d taiy\=a\b n\index{cec}{Kulaiyur@Ku\b laiy\=uru\d taiy\=a\b n} puli[y\=ur] ut-tamaco\b la\b ne \b ne\b n\b rum ippa\d ti a\b rive\b n ku\b laiy\=uru(\d t)ai\dots
	\item \b n\=atap pallavaraiya\b ne \b ne\b n\b rum\index{cec}{Pallavar\=aya\b n} ippa\d ti a\b rive\b n pa\~ncava\b nam\=ateviy\=a\b na\index{cec}{panca@Pa\~ncava\b nam\=atevi} kulottu\.nka-co\b laccaruppetima\.nkalattu\index{cec}{caturvedimangalam@Caturvedima\.ngalam!Kulottu\.nkac\=o\b laccaturvedima\.ngalam} [\=a]curi ma\textbf{h\=adevabha\d t\d ta}\b n\index{cec}{Mah\=adevabha\d t\d ta}e \dots
\end{enumerate}

\subsection*{CEC 13.3 R\'esum\'e}
\og Que la prosp\'erit\'e soit\fg! En la 19\up{e} ann\'ee [de r\`egne] de \'Sr\=\i kopperu\~nci\.nkatevar, empereur de tous les mondes, le mois de \textit{Makara}, le quatorzi\`eme jour de la quinzaine claire, jeudi; Tevarka\d lteva\b n, un propriétaire [terrien] de K\=u\d tal\=ur dans Jaya-ko\d n\d taco\b lava\d lan\=a\d tu\footnote{\`A notre connaissance\index{gnl}{connaissance}, aucune concordance de ce nom n'a \'et\'e trouv\'ee en dehors de CEC 14.}, qui poss\`ede une propri\'et\'e à Olaima\.nkalam dans le Ve\d n\d nai-y\=urn\=a\d tu, donne des terres\index{gnl}{terre}, acquises aupr\`es d'autres, en tant que \textit{tirun\=amattukk\=a\d ni} au seigneur pour qu'il mange.

Une pr\'esentation originale des signataires s'organise autour de la formule: \textit{ippa\d ti a\b rive\b n} X\textit{e\b n} \textit{e\b n\b rum} qui peut signifier \og moi, X, reconnais ceci\fg. Ont ainsi sign\'e \=A\d tko\d n\d tan\=ayaka\b n, un propriétaire [terrien] de Tirucci\b r\b rampalam\index{cec}{tiruccirrampalamutaiyar@Tirucci\b r\b rampalamu\d taiy\=ar} \dots\ un propriétaire [terrien] de K\=u\d tal\=ur \dots\ Aka\d la\.nkap Pallavaraya\b n\index{cec}{Pallavar\=aya\b n}, Puliy\=ur Uttamaco\b la\b n un propriétaire [terrien] de Ku\b laiy\=ur, \dots\ un propriétaire [terrien] de Ku\b laiy\=ur \dots n\=atap Pallavaraiya\b n\index{cec}{Pallavar\=aya\b n}, \=Acuri Mah\=adevabha\d t\d ta\b n de Pa\~ncava\b nam\=atevi\index{cec}{panca@Pa\~ncava\b nam\=atevi}  alias Kulot-tu\.nkaco\b laccaruppetima\.nkalam\index{cec}{caturvedimangalam@Caturvedima\.ngalam!Kulottu\.ngac\=o\b laccaturvedima\.ngalam}\footnote{Il est int\'eressant de constater qu'au moins trois des signataires sont associ\'es \`a Ku\b laiy\=ur et que l'un d'eux est originaire de Pa\~ncava\b nam\=atevi\index{cec}{panca@Pa\~ncava\b nam\=atevi}  alias Kulottu\.nkaco\b laccaruppetima\.nkalam\index{cec}{caturvedimangalam@Caturvedima\.ngalam!Kulottu\.ngac\=o\b laccaturvedima\.ngalam} qui se trouve dans Ve\d n\d naiy\=urn\=a\d tu (ARE 1918 528, 529 et 538).}.

\section*{CEC 14}
\subsection*{CEC 14.1 Remarques}

L'\'epigraphe, relev\'ee dans ARE 1918 395, date de la dix-neuvi\`eme ann\'ee de r\`egne de \og Sakalabhuvanachakravartin K\=opperu\~nji\.ngad\=eva\fg\ que \textsc{Mahalingam} (1992: 552, Tj. 2423) identifie comme K\=opperu\~nci\.nka II\index{gnl}{Kopperuncinka II@K\=opperu\~nci\.nka II}. Elle se situe sur le mur est de la premi\`ere enceinte, au sud du pavillon d'entr\'ee. Elle a été publi\'ee dans SII 12 211 qui donne le r\'esum\'e suivant: \og It is connected with the previous inscription and registers a grant of land as tirun\=amattukk\=a\d ni in \=Olaiy\=ama\.ngalam situated in Ve\d n\d naiy\=urn\=a\d du, a subdivision of R\=aj\=adhir\=ajava\d lan\=a\d du\index{cec}{Rajadhirajavala@R\=aj\=adhir\=ajava\d lan\=a\d tu}, by a certain \'Si\.ng\=arava\d lamu\d dikavitt\=a\b n. \=Olaiy\=ama\.ngalam may be identified with the village \=Oli-y\=amputt\=ur in the Shiyali taluk of the Tanjore district\fg.

L'inscription est situ\'ee en-dessous de CEC 13. Elle est tr\`es lacunaire et se compose de neuf lignes qui s'\'etendent sur environ deux m\`etres cinquante.

Le texte enregistre une donation de terres\index{gnl}{terre} par un certain I\d lanteva\b n Po\b n\b nampa-lakk\=uttar Ci\.nk\=arava\d lamu\d tikavitt\=a\b n. Le donateur de CEC 13 Tevarka\d lteva\b n un propriétaire [terrien] de K\=u\d tal\=ur est mentionn\'e, ainsi que la terre\index{gnl}{terre} qu'il a donn\'ee dans l'Olaima\.nkalam du Ve\d n\d naiy\=urn\=a\d tu.

La subordination de CEC 14 \`a CEC 13 (emplacement, datation et contenu) permet de dater le texte de \textbf{1263}.

\subsection*{CEC 14.2 Texte}
\begin{enumerate}
	\item \textbf{svasti \'sr\=\i}\ cakalapuva\b naccakkaravattika\d l\index{cec}{Sakalabhuvanacakravarti} \textbf{\'sr\=\i} kopperu\~nci\.nkateva\b rku y\=a\d n\d tu 10 9[\=avatu tai]m\=atan tiyati pati(\b n)\b ne\b n\b ri\b n\=al uyyakko\d n\d t\=arva\d lan\=a\d t\d tu\index{cec}{uyyakkontarvala@Uyyakko\d n\d t\=arva\d lan\=a\d tu} ampar\dots
	\item i\d lanteva\b n\index{cec}{ilantevan@I\d lanteva\b n} po\b n\b nampalakk\=uttar\index{cec}{ponnampala@Po\b n\b nampalakk\=uttar} ci\.nk\=arava\d lamu\d tikavitt\=a\b ne\b n\index{cec}{cinkara@Ci\.nk\=arava\d lamu\d tikavitt\=a\b n} [\danda*] [e\b n] per[\=al] \textbf{ja}-ya\.nko\d n\d taco\b lava\d lan\=a\d t\d tut\index{cec}{Jayankontacolavalanatu@Jaya\.nko\d n\d tac\=o\b lava\d lan\=a\d tu} tiruva\b la\b nt\=urn\=a\d t\d tuk\index{cec}{Tiruva\b la\b nt\=urn\=a\d tu} k\=u\d tal\=uru\d tai[y\=a\b n]\index{cec}{kutalur@K\=u\d tal\=uru\d taiy\=a\b n} \dots
	\item \d l\=al n\=a\b n ku\d tutta nilam\=avatu\index{cec}{nilam@\textit{nilam} terre} [\danda*] i\b n\b n\=aya\b n\=ar tevat\=a\b nam\index{cec}{tevatanam@\textit{tevat\=a\b nam} propriété divine} \textbf{r\=aj\=adhir\=aja}va\d lan\=a\d t\d tu\index{cec}{Rajadhirajavala@R\=aj\=adhir\=ajava\d lan\=a\d tu} ve\d n\d naiy\=urn\=a\d t\d tu\index{cec}{vennaiyur@Ve\d n\d naiy\=urn\=a\d tu} olaiy\=ama\.nkalattu\index{cec}{Olaiy\=ama\.nkalam} e\.nka\d l mutaliy\=ar\index{cec}{mutaliyar@\textit{mutaliy\=ar} chef (de monastère)} pakkal\index{cec}{pakkal@\textit{pakkal} auprès de} ku\d tu[tta] \dots
 	\item \dots\ p\=ayki\b ra u\d tci\b ruv\=aykk\=alukkuk\index{cec}{vaykkal@\textit{v\=aykk\=al} canal} ki\b lakkum ku\~ncaramallavi\d l\=akattukkut te\b rkum i\b n\b n\=a\b nkellaiyu\d l na\d tuvupa\d t\d ta mikutikku\b raivu u\d l\d la\d ta\.nkat ta[\d ti] u\dots
 	\item \dots nalliyai mutuka\d n\d n\=akakko\d n\d tu ampar aruvantai araya\b n \dots vv\=a\b lv\=a\dots
	\item u\d tpa\d tat tirun\=amattikk\=a\d niy\=akak\index{cec}{kani@\textit{k\=a\d ni} droit, propriété} ku\d tukkak k\=u\d tal\=uru\d taiy\=a\b n tevarka\d lteva \dots
 	\item \dots \d tattirun\=amattukk\=a\d niy\=akak\index{cec}{kani@\textit{k\=a\d ni} droit, propriété} ku\d tuttamaikku ivai k\=u\d tal\=uru\d taiy\=a\b n tevarka \dots
	\item ippa\d ti a\b rive\b n [perum]peru\d taiy\=a\b n \dots\ \=a[\d n\d ta]pi\d l\d laiye \b ne\b n\b rum ippa\d ti \dots
 	\item \dots \textbf{\'sva}ramu\d taiy\=ar koyi\b r tevarka\b nmi p\=arattuv\=aci a\b lakiyaco\b lappirama\dots
\end{enumerate}

\subsection*{CEC 14.3 R\'esum\'e}
\og Que la prosp\'erit\'e soit\fg! En la 19\up{e} ann\'ee [de r\`egne] de \'Sr\=\i kopperu\~ncinkatevar, empereur de tous les mondes, le dixi\`eme jour du mois de \textit{Tai}. Le donateur I\d lanteva\b n Po\b n\b nampalakk\=uttar\index{cec}{ponnampala@Po\b n\b nampalakk\=uttar} Ci\.nkarava\d lamu\d tikavitt\=an\footnote{Ce donateur, sauf erreur, n'est pas mentionn\'e ailleurs.} donne une terre\index{gnl}{terre} qui est li\'ee à un propri\'etaire [terrien] de K\=u\d tal\=ur du Va\b la\b nt\=urn\=a\d tu dans le Jayako\d n\d taco\b lava\d lan\=a\d tu. Il semble que la transaction s'effectue en pr\'esence du \textit{mutali} du donateur dans l'Olaima\.nkalam du Ve\d n\d naiy\=urn\=a\d tu (l.~3). Il est question aussi de tuteurs \textit{mutukka\d n} qui agissent pour autrui (l.~5).

Les signataires apparaissent sur le m\^eme mod\`ele que CEC 13. Sont pr\'esents, entres autres, Tevarka\d lteva\b n un propriétaire [terrien] de K\=u\d tal\=ur, \dots\ un propriétaire [terrien] de Perumper\dots\ \=A\d n\d tapi\d l\d lai, \dots\ P\=arattuv\=aci A\b lakiyaco\b lappirama\b n \textit{devakarm\=\i} du temple\index{gnl}{temple} du seigneur\dots


\section*{C. Galeries int\'erieures}
\section*{CEC 15}
\subsection*{CEC 15.1 Remarques}

L'inscription a été relev\'ee dans ARE 1918 371 qui la date du seizi\`eme jour du mois de K\=arttikai de l'ann\'ee cyclique Rudhirodg\=ari. Le rapport place l'\'epigraphe sur le mur nord de la galerie autour du temple\index{gnl}{temple} principal, et pr\'ecise que l'ann\'ee 1300 donn\'ee (l. 1) est une erreur pour 1306, datant ainsi le texte de \textbf{1384}.

L'inscription n'a pas \'et\'e identifi\'ee et localis\'ee \textit{in situ} \`a cause des \'epaisses couches de peinture. Cependant, selon la transcription de Mysore, le texte daterait du quatorzi\`eme jour du mois de K\=arttikai.

\subsection*{CEC 15.2 Texte}
\begin{enumerate}
	\item 1000 3 100 utiro\b rk\=ari varu\textbf{\d sa}m k\=attikai m\=atam pati\b n\=a\d l\=an tiyati tiru\~n\=a\b nacam-panta\b n [pa\d t\d tam] kamuku vaitt\=arai ki\b l[ppayir]kku na \dots ttu[kku]\d tutta pa\d tikku \dots ta\b ruvaittu ve\d t \dots kamavum a\b ruttu ku\d tu[kka] ka\d tavat\=aka[vu]m ka[\b rppit]te\b n \dots \.nkal\=aka ki\b lpayir\=a \dots ra\d lavum
	\item kamuku palam\=a\b n\=al o\b n\b ru patt\=a[yu]m pe\b ra ka\d tav\=akavum te\b namaram pal\=a ivai [10.] i\d lamariy\=ati pe\b ra ka\d tav\=akavum ku\d tiv\=aram o\b n\b ru p\=ata \dots kaipa\b r\b ru pala upa \dots kapar\=ayu\d n\d t\=akil na \dots \b n\=a \dots pi\b n [ci]p\=atame nayi\b na\b n \=a\d lu\d taiyapi\d l\d laiy\=ar\index{cec}{Alutaiyapillaiyar@\=A\d lu\d taiyapi\d l\d laiy\=ar} cip\=atame \dots tala.e\d t\d t\=a
	\item yu\d n\d t\=akil cipatta \dots \d lpa\d t\d tatu pa\d takav\=akavum \=a\d lu\d taiyapi\d l\d laiy\=ar p\=atame k\=avalai
\end{enumerate}

\subsection*{CEC 15.3 R\'esum\'e}
En l'ann\'ee Utiro\b rk\=ari 1300, le quatorzi\`eme jour du mois de K\=arttikai, un certain Tiru\~n\=a\b nacampanta\b n\index{cec}{tirunana@Tiru\~n\=a\b nacampanta\b n} pa\d t\d ta\b n ordonne \`a ceux qui ont plant\'e les ar\'equiers d'offrir des noix d'arec. Il est aussi question de cocotiers et de jacquiers.
Le texte continue sur une impr\'ecation et la donation est plac\'e sous la protection des pieds d'\=A\d lu\d taiyapi\d l\d laiy\=ar\index{cec}{Alutaiyapillaiyar@\=A\d lu\d taiyapi\d l\d laiy\=ar}.


\section*{CEC 16}
\subsection*{CEC 16.1 Remarques}
Selon l'ARE 1918 370 l'inscription se trouverait sur le soubassement nord de la galerie int\'erieure.
Elle daterait, en \`ere \'Saka, de 1313, de l'ann\'ee Praj\=apati, du mois de Makara. Le rapport propose ainsi la date du \textbf{vendredi 29 d\'ecembre 1391}.

Nous n'avons ni identifi\'e ni localis\'e le texte \textit{in situ}.

\subsection*{CEC 16.2 Texte}
\begin{enumerate}
	\item \textbf{svasti \'sr\=\i} cak\=a\textbf{pta}m \=ayirattu mu\b n\b n\=u\b r\b ru orupattu mu\b n\b ril me[l cell\=ani\b n\b ra \textbf{pra[j\=a}pa]ti varu\textbf{\d sa}m makaran\=aya\b r\b ru \textbf{p\=urvabhak\d sa}ttu \textbf{tri}tikaiyum ve\d l\d likki\b la-maiyum pe\b r\b ra [ca][t]aiyattu [n\=a\d l] \dots\ to\d nipuramu\d taiy\=arum\index{cec}{Tiruttoni@Tirutt\=o\d nipuramu\d taiya n\=aya\b n\=ar, \'Siva} n\=aya\b n\=ar \=a\d lu\d taiya [pi\d l]\d laiy\=ana \dots ru\d li\b napa\d ti \dots\ \=atica\d n\d te\textbf{\'sva}ra\b n\index{cec}{adicandesvara@\=Adica\d n\d de\'svaradeva} aru\d l\=al tiruma\~nca\b na \dots tay\=ar \dots ttu \dots tti \dots
	\item tiru\~n\=a\b nacampanta\b n\index{cec}{tirunana@Tiru\~n\=a\b nacampanta\b n} tiruma\d tam\=aka\index{cec}{matam@\textit{ma\d tam} monastère} va\d takkil ma\d tam\=aka\index{cec}{matam@\textit{ma\d tam} monastère} na\d takka ku\d tutta va\d ta-karai ir\=ac\=atir\=aca va\d lan\=a\d t\d tu\index{cec}{Rajadhirajavala@R\=aj\=adhir\=ajava\d lan\=a\d tu} tirukka\b lumalan\=a\d t\d tu\index{cec}{Tirukka\b lumalan\=a\d tu} tirukka\b lumalattu\index{cec}{Tirukkalumalam@Tirukka\b lumalam} ca\b n\b natikku\index{cec}{cannati@\textit{ca\b n\b nati} temple} te\b rku \dots ruvitikku me\b rkkum va\d takkum civak\=amacuntaripperu \dots [\b lu]\.n[kai]kku ka \dots\ n\=a\b nkellaikku u\d lppa\d t\d ta ma\b nai kol pati\b n\=alum ma\d tamum\index{cec}{matam@\textit{ma\d tam} monastère} ma\d tappu \dots kka-\d taiyum\=akavum i \dots ma\d tattu
	\item kku ma\d tappu\b ram\=aka\index{cec}{puram@\textit{pu\b ram} terre de donation!\textit{ma\d tappu\b ram} terre donnée pour le monastère} ku\d tutta ki\b lakku va\d tak\=al um\=apatinall\=uril ku\d tutta nilam\index{cec}{nilam@\textit{nilam} terre} n\=a\b rpa[tum]\=avum inta ma\d tamum\index{cec}{matam@\textit{ma\d tam} monastère} tamakku paramparam\=aka tanta a\d lavukku nammu \dots \d lavum \=atica\d n\d tecura\b n\index{cec}{adicandesvara@\=Adica\d n\d de\'svaradeva} aru\d l\=al tirutto\d nipurappiriya\b n ma\d tattu\index{cec}{matam@\textit{ma\d tam} monastère}
	\item va\d lattu v\=a\b lvitt\=a\b rku p\=ut\=a\b nam\=akavum\index{cec}{putanam@\textit{p\=ut\=a\b nam} terre de donation} um\=apatinall\=uril vi\d t\d ta N 8 M Nlam e\d t\d tu-m\=avukkum ivai tirutto\d nipurappiriya\b n e\b luttu inta nilam\index{cec}{nilam@\textit{nilam} terre} 4 10 8 m\=avum n\=aya\b n\=a \dots\ pi\d l\d laiy\=arkku tiruma\~nca\b nattukku c\=a[ttu]ppa\d tikku \dots \d lavum cella[mi\d la] \dots\ ku\d tuttamaikku ivai [tiru]ma\~n[ca\b nama]\b lakiy\=a\b n e\b luttu U
\end{enumerate}

\subsection*{CEC 16.3 R\'esum\'e}
Le texte date de l'ann\'ee [cyclique] Praj\=apati qui restait sans aller plus haut que mille trois cent treize, le mois de Makara [Pu\d sya, Tai], le troisi\`eme jour de la quinzaine claire, vendredi, le jour du [\textit{nak\d satra}] Catayam.
Il est question du bain du Seigneur de Tirutt\=o\d nipuram\index{gnl}{Tonipuram@T\=o\d nipuram} et du Seigneur \=A\d lu\d taiyapi\d l\d laiy\=ar\index{cec}{Alutaiyapillaiyar@\=A\d lu\d taiyapi\d l\d laiy\=ar} en invoquant la gr\^ace d'\=Atica\d n\d te\'svara\b n\index{cec}{adicandesvara@\=Adica\d n\d de\'svaradeva}.
Une terre\index{gnl}{terre} de quarante-huit \textit{m\=a} a \'et\'e donn\'ee au \textit{n\=aya\b n\=ar}\index{gnl}{nayanmar@\textit{n\=aya\b nm\=ar}!\textit{n\=aya\b n\=ar}} \=A\d lu\d taiyapi\d l\d laiy\=ar\index{cec}{Alutaiyapillaiyar@\=A\d lu\d taiyapi\d l\d laiy\=ar} pour que soit offert le bain sacr\'e. Ceci a été sign\'e par Tiruma\~nca-\b nama\b lakiy\=a\b n.
Ce lot de quarante-huit \textit{m\=a} se compose d'une terre\index{gnl}{terre} de quarante \textit{m\=a} correspondant \`a un jardin de monast\`ere\index{gnl}{monastère} (\textit{ma\d tam}) \`a Um\=apatinall\=ur (l. 3), et d'une terre\index{gnl}{terre} de huit \textit{m\=a}, toujours \`a Um\=apatinall\=ur, donn\'ee par un certain Tirutto\d nipurap-piriya\b n \og \`a Celui qui fait grandir et vivre le \textit{ma\d tam}\fg\index{cec}{matam@\textit{ma\d tam} monastère}. Cette transaction avait \'et\'e sign\'ee par Tirutto\d nipurappiriya\b n (l. 4).

\section*{CEC 17}
\subsection*{CEC 17.1 Remarques}
L'inscription a \'et\'e relev\'ee dans l'ARE 1918 373 et localis\'ee sur le soubassement ouest de la galerie int\'erieure. Le texte est très précis quant à la date mais la date exacte n'a pu \^etre reconstitu\'ee par le rapport. L'ann\'ee peut \^etre \textbf{1393 ou 1394}.
Notre \'edition repose sur la transcription de l'ASI.
Le texte enregistre un don\index{gnl}{don} de terre\index{gnl}{terre} fait au b\'en\'efice de diff\'erents membres du temple\index{gnl}{temple} et du monast\`ere\index{gnl}{monastère}.

\subsection*{CEC 17.2 Texte}
\begin{enumerate}
	\item \textbf{svasti \'sr\=\i ma}\b n ma\textbf{h\=a}ma\d n\d tali\textbf{\'svaran p\=urva dak\d si}\d na uttara \textbf{sa}muttir\=atipati \textbf{\'sr\=\i }vira\textbf{hariharar\=a}ya\b n kumara\b n \textbf{\'sr\=\i }viraviruppa\d na u\d taiy\=a\b rku \textbf{p\textsubring{r}thivir\=ajyam} cell\=ani\b n\b ra \textbf{\'sak\=apta}m 1000 3 100 10 5 \b n mela \textbf{\'sr\=\i mukhavar\d sa}m \textbf{m\=argga\'sira} \textbf{\'suddha} \textbf{panca}miyum \textbf{\'su} ... k\=attikaim\=atam 2 10 5 \dots\ co\b lama\d n\d talattu k\=ave-ri[kkum] ko\d l\d li\d tattukkum na\d tuppa\d t\d ta pa\b r\b ru r\=aj\=adhir\=acava\d lan\=a\d t\d tu\index{cec}{Rajadhirajavala@R\=aj\=adhir\=ajava\d lan\=a\d tu} ki\b lakku va-\d tak\=ala\b nu \dots ttillaivi\d ta\.nkanall\=ur\index{cec}{Tillaivi\d ta\.nkanall\=ur} a\d taippi\b npa\d ti\index{cec}{ataippu@\textit{a\d taippu} limite}
	\item N 6 10 Vkku N 2 100 4 10 M n\=aya\b n\=ar \=a\d lu\d taiyapi\d l\d laiy\=a\b rku N 10 Vkku N 4 10 M \textbf{bh\=aradv\=ajigotra\index{gnl}{gotra@\textit{gotra}}}ttu \textbf{\=apastambha} \textbf{s\=utra}ttu kamukai aru\d na\textbf{giri\'siva}\b rku\index{cec}{Aru\d nagiri\'siva} ivv\=ur nattamu\d tpa\d ta ma\d tappu\b ram\index{cec}{puram@\textit{pu\b ram} terre de donation!\textit{ma\d tappu\b ram} terre donnée pour le monastère} N 2 10 Vkku N 8 10 M inta \textbf{gotra\index{gnl}{gotra@\textit{gotra}}}ttu inta \textbf{sutra}ttu \textbf{r\=aman\=adhabha\d t\d ta}\b rku\index{cec}{rama@R\=aman\=adhabha\d t\d ta\b n} \textbf{bh\=ud\=a}na\index{cec}{putanam@\textit{p\=ut\=a\b nam} terre de donation!\textit{bh\=ud\=ana}} ta\d n\d ti \dots \d ni\textbf{\'sra}\b rku N 5 Vkku N 2 10 M cik\=a\b li tirune\b rim\=a\d likai ma\d tattu\index{cec}{matam@\textit{ma\d tam} monastère} mutaliy\=a\b rku\index{cec}{mutaliyar@\textit{mutaliy\=ar} chef (de monastère)} ma\d tappu\b ram\index{cec}{puram@\textit{pu\b ram} terre de donation!\textit{ma\d tappu\b ram} terre donnée pour le monastère} N 2 10 M \textbf{p\=ujak\=ala\'sv\=ami} tiru\~n\=anacampanta\index{cec}{tirunana@Tiru\~n\=a\b nacampanta\b n} pa\d n\d tita\b rku p\=ut\=a\b nata\d n\d ti\b raiyil\index{cec}{putanam@\textit{p\=ut\=a\b nam} terre de donation}\index{cec}{iraiyili@\textit{i\b raiyili} non imposable}
	\item kku N 2 10 M tevaka\b nmi \textbf{k\=a\'sya}pa\b n\index{cec}{kasyapa@K\=a\'syapa\b n} k\=a\b lika\b rpakapa\d t\d ta\b rku\index{cec}{kali@K\=a\b lika\b rpakapa\d t\d ta\b n} p\=ut\=a\b nata\d n\d ti\b raiyilikku\index{cec}{putanam@\textit{p\=ut\=a\b nam} terre de donation} N 2 10 M \=aka N 6 10 V ma\b nul N 2 100 4 10 M i\b n\b nilam\index{cec}{nilam@\textit{nilam} terre} irun\=u\b r\b ru n\=a\b rpatu m\=avum \textbf{sarvad\=a\b nya}m\=aka \textbf{candr\=aditya}varaiyum na\d tattikko\d l\d la itukku virota\~nceyt\=aru\d n\index{cec}{virotam@\textit{vir\=otam} opposition, trahison} \dots\ cempilum ve\d t\d tikko\d l\d la ippa\d tikku \textbf{dharmma}c\=ata\b nappa\d t\d taiya ku\d tuttamaikku
\end{enumerate}

\subsection*{CEC 17.3 R\'esum\'e}
Le texte est dat\'e en ann\'ee \textit{\'saka} de 1315 en cours pendant le r\`egne (\textit{p\textsubring{r}thivir\=a-jyam}) du seigneur V\=\i raviruppa\d na, fils de V\=\i rahariharar\=aya\b n qui est le chef\index{gnl}{chef} de la mer\index{gnl}{mer} \`a l'est, au sud et au nord, grand chef\index{gnl}{chef} de la province (\textit{mah\=ama\d n\d tale\'svara}). Nous sommes en l'ann\'ee [cyclique] \'Sr\=\i mukha, le cinqui\`eme jour clair du [\textit{nak\d satra}] M\=arga\'sira, le 25\up{\`eme} jour du mois de K\=arttikai.
Le don\index{gnl}{don} est localis\'e entre le Ko\d l\d li\d tam et la K\=averi du Co\b lama\d n\d talam, dans la r\'egion (\textit{pa\b r\b ru}) de R\=aj\=adhir\=acava\d lan\=a\d tu\index{cec}{Rajadhirajavala@R\=aj\=adhir\=ajava\d lan\=a\d tu}. Il s'agit d'une terre\index{gnl}{terre} de 60 \textit{veli}, soit de 240 \textit{m\=a}, selon les limites de Tillaivi\d ta\.nkanall\=ur.
Cette terre\index{gnl}{terre} est partag\'ee ainsi:
\begin{enumerate}
\item une terre\index{gnl}{terre} de 10 \textit{veli} soit 40 \textit{m\=a} pour le \textit{n\=ayan\=ar} \=A\d lu\d taiyapi\d l\d laiy\=ar\index{cec}{Alutaiyapillaiyar@\=A\d lu\d taiyapi\d l\d laiy\=ar},
\item une terre\index{gnl}{terre} de 20 \textit{veli} soit 80 \textit{m\=a} de jardin du \textit{ma\d tam}\index{cec}{matam@\textit{ma\d tam} monastère} incluant le hameau (\textit{nattam}) de ce village pour Aru\d nagiri\'siva de Kamukai de l'\'ecole (\textit{s\=utra}) \textit{\=apastamba} et de la lign\'ee (\textit{gotra\index{gnl}{gotra@\textit{gotra}}}) Bh\=aradv\=aji,
\item une terre\index{gnl}{terre} de don\index{gnl}{don} (sans) taxe [de 10 \textit{veli} soit 40 \textit{m\=a}] pour le \textit{bha\d t\d tar} R\=aman\=adha de la m\^eme lign\'ee et de la m\^eme \'ecole,
\item une terre\index{gnl}{terre} de 5 \textit{veli} soit 20 \textit{m\=a} pour \dots,
\item une terre\index{gnl}{terre} de 20 \textit{m\=a} de jardin du \textit{ma\d tam}\index{cec}{matam@\textit{ma\d tam} monastère} pour le chef\index{gnl}{chef} (\textit{mutaliy\=ar}) du monast\`ere\index{gnl}{monastère} (\textit{ma\d tam}) Tirune\b rim\=alikai de C\=\i k\=a\b li\index{gnl}{Cikali@C\=\i k\=a\b li},
\item une terre\index{gnl}{terre} de 20 \textit{m\=a} comme terre\index{gnl}{terre} de don\index{gnl}{don} sans taxe pour l'officiant des service\index{gnl}{service}s (P\=ujjak\=ala\'sv\=ami) \textit{pa\d n\d titar} Tiru\~n\=a\b nacampantar\index{cec}{tirunana@Tiru\~n\=a\b nacampanta\b n},
\item et une terre\index{gnl}{terre} de 20 \textit{m\=a} comme terre\index{gnl}{terre} de don\index{gnl}{don} sans taxe pour le surveillant (\textit{tevaka\b nmi}) \textit{bha\d t\d tar} K\=a\b lika\b rpaka K\=a\'syapa\b n\index{cec}{cikali@C\=\i k\=a\b li}\index{cec}{kasyapa@K\=a\'syapa\b n}.
\end{enumerate}
La r\'ecapitulation du don\index{gnl}{don} se trouve l. 3: une terre\index{gnl}{terre} de 60 \textit{veli} soit 240 \textit{m\=a}, terre\index{gnl}{terre} exempt\'ee (\textit{sarvad\=a\b nyam}) de toute taxe, doit \^etre mise en fonction tant que durent la lune et le soleil. L'impr\'ecation est suivie de l'ordre\index{gnl}{ordre royal} de graver ce texte sur cuivre et de la signature d'un certain Dharmmac\=ata\b nappa\d t\d ta\b n.

\section*{CEC 18}
\subsection*{CEC 18.1 Remarques}
L'inscription a \'et\'e relev\'ee dans l'ARE 1918 400 et localis\'ee sur le soubassement sud de la galerie int\'erieure. Elle a \'et\'e dat\'ee du \textbf{mercredi 6 mars 1398}.
L'\'edition qui suit reproduit la transcription de l'ASI.

\subsection*{CEC 18.2 Texte}
\begin{enumerate}
	\item \dots\ viruppa\b na u\d taiy\=ar \textbf{p\textsubring{r}thivi}r\=ac(ciyam pa\d n\d niyaru\d l\=ani\b n\b ra) \textbf{\'sak\=apda}m (1000 3 100) 10 9 \b nmel coll\=ani\b n\b ra \textbf{\=\i \'svara}varu\textbf{\d sa}m pa\.nku\b ni m\=atam 10 1 U cik\=ariyam\index{cec}{srikariyamcey@\textit{\'sr\=\i k\=ariyam cey} employé du temple} cu\b namp\=akkattu k\=aciyapa\b n\index{cec}{kasyapa@K\=a\'syapa\b n} (t)e(y)van\=ayakar peril nam pa\d t\d t\=arkku ilakkaikku \dots kakku nellu kalam\index{cec}{kalam@\textit{kalam} unité de mesure du paddy} \=a\d n\d tu o\b n\b rukku pu\d tavai mutalukku pa\d nam muppatukkum koyilil\index{cec}{koyil@\textit{k\=oyil} temple} nilam\index{cec}{nilam@\textit{nilam} terre} cerkkum mariy\=ati nellukku\b ru\d nikkum pu\d tavai mutal \dots na\d taikko-l\=al
 	\item \dots N \dots\ 10 2 innilam\index{cec}{nilam@\textit{nilam} terre} pa\b n\b nira\d n\d tu m\=avukkum \dots m nellu \dots l po\b n mutalum up\=atikku ko\d l\d lum nellum tiruppa\d nippa\d na vicam m\=atam pa\b lavari\index{cec}{vari@\textit{vari} taxe} putuvariyum ma\b r\b rum epperppa\d t\d ta a\b naittu up\=atiyum i\b n\b nilattukkukko\d l\d lum\index{cec}{nilam@\textit{nilam} terre} i\b raiyilik\=a\d nikkai-yum\index{cec}{iraiyili@\textit{i\b raiyili} non imposable} u\d tpa\d ta tanta a\d lavukku i\b n\b nilam\index{cec}{nilam@\textit{nilam} terre} pa\b n\b nira\d n\d tu m\=avum kamuku to\b luntu tarumapuce\.nka\b lun\=\i r u\d tpa\d tatt\=am ve\d n\d tum payir ceytu ko\d n\d tu iv \dots
	\item lika\d tintaco\b lap piram\=ar\=ayar e\b luttu ivai ne \dots k\=amukan\=ayakapa\d t\d tar e\b luttu ivai cimu \dots n\=ayar\=a\d n e\b luttu ituvum ivar maka\b num cik\=ariyum\index{cec}{srikariyamcey@\textit{\'sr\=\i k\=ariyam cey} employé du temple} k\=acipan\index{cec}{kasyapa@K\=a\'syapa\b n} mutaliy\=ark-ku\index{cec}{mutaliyar@\textit{mutaliy\=ar} chef (de monastère)} ilakkaikku ivv\=uril u\b lava\b n\=al cetta N aka\d ta \dots\ v\=ara\d nappiriya\b n e\b luttu
 	\item \dots yar cey\=amal ki\d taikkaiyi\b n\=al itu tamakku tiruv\=aymo\b lintaru\d lina \=a\d tciy\=aka tan-taru\d lina a(\d la)vukkuntilara\b lamarak\=aka \dots ttu tiruma\d n\d tapamum ceyvikkavum inta tiruma\d n\d tapattile \dots\ aru\d la ippa\d tikku \textbf{candr\=aditta}\b r u\d l\d la a\d lavum na\d takka \dots mu.ku.ivai po\b n\b nampalakk\=utta\b n\index{cec}{ponnampala@Po\b n\b nampalakk\=uttar} e\b luttu
 	\item \dots\ pa\d t\d ta\b n e\b luttu ivai \textbf{\'sr\=\i m\=ahe\'svara}muta\index{cec}{srimahesvara@\textit{\'sr\=\i mahe\'svara} dévot, surveillant} \dots tu ivai koyil\index{cec}{koyil@\textit{k\=oyil} temple} (ka\d n)ka\d nakku\index{cec}{kanakku@\textit{ka\d nakku} comptable} tiru-vatiyapa\d t\d ta \dots\ e\b luttu
\end{enumerate}

\subsection*{CEC 18.3 R\'esum\'e}
Le texte est dat\'e de \og quand r\'egnait Viruppa\b na U\d taiy\=ar \dots, en l'ann\'ee \textit{\'saka} 1319, de l'ann\'ee [cyclique] \=I\'svara, le 11\up{e} [jour] du mois de Pa\.nku\b ni\fg.
Cette inscription semble traiter des diff\'erents composants du salaire (\textit{ilakkai}) des officiants: riz\index{gnl}{riz} non d\'ecortiqu\'e (\textit{nellu}), v\^etement (\textit{pu\d tavai}) et argent (\textit{pa\d nam}).
Le texte pr\'ecise l. 4 qu'il faut faire faire un \textit{ma\d n\d dapa} sur une terre\index{gnl}{terre} li\'ee \`a un ordre\index{gnl}{ordre royal} (\textit{tiruv\=aymo\b lintaru\d lina \=a\d tciy\=aka tantaru\d lina}).
Les signataires semblent être des officier\index{gnl}{officier}s du temple\index{gnl}{temple} (\textit{cik\=ariyam}\index{cec}{srikariyamcey@\textit{\'sr\=\i k\=ariyam cey} employé du temple}, \textit{pa\d t\d tar}, \textit{ka\d nakku}\index{cec}{kanakku@\textit{ka\d nakku} comptable}, etc.).

\section*{CEC 19}
\subsection*{CEC 19.1 Remarques}
Le texte, relev\'e dans ARE 1918 372, date d'un jeudi de la quinzaine sombre du mois de Tul\=a (\=A\'svina) de l'ann\'ee Siddh\=arti. Il se trouve sur le soubassement de la galerie ouest. L'\'edition se base seulement sur l'examen de la transcription de l'ASI qui ne reproduit pas la quatri\`eme ligne.

\subsection*{CEC 19.2 Texte}
\begin{enumerate}
	\item \textbf{svasti \'sr\=\i matu svasti \'sr\=\i\ sitt\=arddhi} varu\textbf{\d sa}m tul\=an\=aya\b r\b ru \textbf{aparabhak\d sa}t-tu viy\=a\b lak\dots\ na\d l tirukka\b lumalam\index{cec}{Tirukkalumalam@Tirukka\b lumalam} u\d l\d l\=ur pa\b r\b ra\d tai e\.nka\d l [k\=a\d ni]\index{cec}{kani@\textit{k\=a\d ni} droit, propriété} \=a\d tciyil pa\b r\b ra\d tai ku\d timakka\d l per\=al u\d l\d lamutalukku e\.nka\d l per\=al koyil\index{cec}{koyil@\textit{k\=oyil} temple} k\=avamu[\b r]ikko\d l\d luki\b ra inn\=a-ya\b n\=ar periyan\=aya\b n\=arkku pa\d ti \dots\ n\=ayan\=ar \=a\d lu\d tai \dots kk\=al u\d l\d la mutal ku\d tutta a\d lavukku pa\b r\b ra\d taikku\d timakka\d l \dots taravu e\b lutikko\d l\d la tirukka\b lumalam\index{cec}{Tirukkalumalam@Tirukka\b lumalam} u\d l\d l\=uril tirun\=amattukk\=a\d ni\index{cec}{kani@\textit{k\=a\d ni} droit, propriété} nikki e\.nka\d l k\=a\d niy\=a\b na\index{cec}{kani@\textit{k\=a\d ni} droit, propriété} nilattil\index{cec}{nilam@\textit{nilam} terre}
	\item ku\d ti ni\.nk\=ata tevat\=a\b nam\=aka\index{cec}{tevatanam@\textit{tevat\=a\b nam} propriété divine} cerkka N 100 10 M itil tirukku\d lan\=aya\b nm\=ar e\b r\dots 10 M nikka N 100 m ko\d ta\.nku\d tiyil tirun\=amattukk\=a\d ni\index{cec}{kani@\textit{k\=a\d ni} droit, propriété} nikki e\.nka\d l k\=a\d niy\=a\b na\index{cec}{kani@\textit{k\=a\d ni} droit, propriété} nilam\index{cec}{nilam@\textit{nilam} terre} muppatu muppatu m\=a \=aka nilam\index{cec}{nilam@\textit{nilam} terre} n\=u\b r\b ru muppatu \dots l kulottu\.nkaco\b la\textbf{brah-mar\=a}yarkku\index{cec}{kulottungacolabra@Kulottu\.ngac\=o\b labrahmar\=ayar} ka\d n\d tu ilakkaikku vi\d t\d ta N 2 10 M kki N .yirkku tiru \dots \d t\d tukku ko\d l\d lumariy\=atiyil m\=att\=al irukalanellum \dots \d ta \dots yakko\d l\d la inta nilam\index{cec}{nilam@\textit{nilam} terre} u\b lanta melum e\.nka\d l k\=a\d niyil\index{cec}{kani@\textit{k\=a\d ni} droit, propriété} ku\d tini\.nk\=attevat\=a\b nam\=aka\index{cec}{tevatanam@\textit{tevat\=a\b nam} propriété divine} m\=aka certtukko\d l\d lum
	\item nilattukkum\index{cec}{nilam@\textit{nilam} terre} immariy\=ati ko\d li\b n u\d taiy\=ar tirutto\d nipura\index{cec}{Tiruttoni@Tirutt\=o\d nipuramu\d taiya n\=aya\b n\=ar, \'Siva} \dots\ \=a\d lu\d taiyapi\d l\d laiy\=ar koyilukkum\index{cec}{koyil@\textit{k\=oyil} temple} mutalum e\b lutitaravukkum e\b luttu i\d t\d tuppota ippa\d tikku cantir\=atitta \dots yum na\d takka itukku virota\~nco\b n\b n\=aru\d n\d t\=atil\index{cec}{virotam@\textit{vir\=otam} opposition, trahison} n\=aya\b n\=a\b r \=a\d lu\d taiyapi\d l\d laiy\=ar tiruv\=a \dots tuve\d t\d tikku\d tuttom ivai m\=a\d tala\b n\index{cec}{matalan@M\=a\d tala\b n} kalika\d tintaco\b la\textbf{brahm\=ar\=a}ya\b n\index{cec}{Kalika\d tintaco\b labrahm\=ar\=aya\b n} \dots m\=a\d tala\b n\index{cec}{matalan@M\=a\d tala\b n} kulottu\.nkaco\b la\textbf{brahm\=ar\=a}va\b n\index{cec}{kulottungacolabra@Kulottu\.ngac\=o\b labrahmar\=ayar} e\b luttu ivai m\=a\d tala\b n\index{cec}{matalan@M\=a\d tala\b n} cempiya\b n\index{cec}{Cempiya\b n}\textbf{brahm\=ar\=a}ya\b n e\b luttu ivai m\=a\d tala\b n\index{cec}{matalan@M\=a\d tala\b n} ka\b naka\textbf{sabh\=apatibha\d t\d ta}\b n\index{cec}{kanaka@Ka\b nakasabh\=apatibha\d t\d ta\b n} e\b luttu ivai \textbf{bh\=radv\=aji} (?) karik\=alaco\b la\textbf{brahma} \dots natiy\=aka vi\b nota\textbf{brahm\=ar\=a}yar\index{cec}{vinota@Vi\b notabrahm\=ar\=ayar} \dots
\end{enumerate}

\subsection*{CEC 19.3 R\'esum\'e}

L'\'epigraphe enregistre un don\index{gnl}{don} de terre\index{gnl}{terre} pour le temple\index{gnl}{temple} du Seigneur de Tirutt\=o\d nipuram\index{gnl}{Tonipuram@T\=o\d nipuram} et celui d'\=A\d lu\d taiyapi\d l\d laiy\=ar\index{cec}{Alutaiyapillaiyar@\=A\d lu\d taiyapi\d l\d laiy\=ar}. Il s'ach\`eve sur une impr\'ecation et la signature de plusieurs brahmane\index{gnl}{brahmane}s (M\=a\d tala\b n\index{cec}{matalan@M\=a\d tala\b n} Kalika\d tintaco\b labrahmar\=aya\b n, M\=a\d tala\b n Kulottu\.nkaco\b labrahmar\=aya\b n\index{cec}{kulottungacolabra@Kulottu\.ngac\=o\b labrahmar\=ayar}, M\=a\d tala\b n Cempiya\b nbrahmar\=aya\b n\index{cec}{Cempiya\b n}, M\=a\d tala\b n Ka\b naka-sabh\=apatibha\d t\d ta\b n\index{cec}{kanaka@Ka\b nakasabh\=apatibha\d t\d ta\b n}, Bh\=attm\=aji Karik\=alaco\b labrahmar\=aya\b n, \dots\ Vi\b notabrahm\=ar\=ayar).

\section*{CEC 20}
\subsection*{CEC 20.1 Remarques}
L'inscription a \'et\'e relev\'ee dans l'ARE 1918 396. Elle se trouve sur le mur int\'erieur de droite du pavillon d'entr\'ee du temple\index{gnl}{temple} principal de \'Siva\index{gnl}{Siva@\'Siva}. Elle se compose de vingt-deux lignes qui courent sur deux m\`etres. Les donn\'ees astronomiques permettent de dater le texte du \textbf{mercredi 29 octobre 1488}.
L'\'edition propos\'ee est bas\'ee sur l'examen de la transcription de l'ASI, des clich\'es de \textsc{G. Ravindran} de l'EFEO et de la lecture \textit{in situ} avec \textsc{G. Vijayavenugopal}. Nous remercions \textsc{A. Griffiths} pour la lecture de la formule de protection en vers sanskrit (l. 19-20).

\subsection*{CEC 20.2 Texte}
\begin{enumerate}
	\item \textbf{svasti \'sr\=\i\ \'sak\=apptam} 1000 4 100 10 .... cell\=a ni\b n\b ra kilaka \textbf{sa}\.nvat
	\item \textbf{sara}ttu tul\=an\=aya\b r\b ru ki\textbf{\d s\d napak\d sa}ttu puta[\b n]v\=aramum pe\b r\b ra [makarana\textbf{k\d sa}t-tirattu]
	\item \textbf{\'sr\=\i man} \textbf{pa\d t\d tukka\d t\d t\=ari} pa\textbf{\'sa}paya\.nkara matiy\=atama\b n\b nar ma\d na[m\=a\b na] k\=a\~nc\=\i-pura
	\item va\textbf{r\=adhi\'sva}ra\index{cec}{Kanci@K\=a\~nc\=\i puravar\=adhi\'svara} koneri\textbf{devamah\=ar\=a\'s\=a} c\=\i k\=a\b li\index{cec}{cikali@C\=\i k\=a\b li} koyil\index{cec}{koyil@\textit{k\=oyil} temple} t\=anatt\=a\b rku ta\.nka\d l
	\item \d l koyil\index{cec}{koyil@\textit{k\=oyil} temple} c\=\i rmai\index{cec}{cirmai@\textit{c\=\i rmai} région sous le contrôle des N\=ayaka} \textbf{sarva}mum \textbf{m\=annya}m\index{cec}{manyam@\textit{m\=anyam} non imposable} \=aka na\d takkumpa\d tiyum ka\b rpittu
	\item ira\d n\d t\=a\b r\b ruppa\b r\b ru c\=\i rmaikku\index{cec}{cirmai@\textit{c\=\i rmai} région sous le contrôle des N\=ayaka} kurukular\=aya i\d t\d ta mu\b riyilum k\=u\d t\=a
	\item k\=u\d t\=amal mu\b ripi\b rinta ni\b n\b rayam \=akaiy\=ale ammariyy\=atile pa\d t\d ta\d taic\=att\=ama kal
	\item ve\textbf{\d t\d ti}vari\index{cec}{vari@\textit{vari} taxe} pa\b raiy\=avari u\d l\d li\d t\d ta vakaiyum \textbf{sarvam\=annyam}\index{cec}{manyam@\textit{m\=anyam} non imposable} \=aka na\d takkum pa\d ti-yum ta\.nka\d l
	\item ta\.nka\d l koyil\index{cec}{koyil@\textit{k\=oyil} temple} cirmaiy\=a\b na\index{cec}{cirmai@\textit{c\=\i rmai} région sous le contrôle des N\=ayaka} va\d tak\=averinall\=ur k\=\i \b lna\b rko\b r\b ra\.nku\d ti tirun\=avukkaraca
	\item nall\=ur vi\d tunilam\index{cec}{nilam@\textit{nilam} terre} k\=\i \b r\=anall\=ur vi\d tunilam\index{cec}{nilam@\textit{nilam} terre} u\d l\d li\d t\d tavakaiyile n\=a\b rpattu iruveli ni
	\item lam pa\d n\d naiy\=akappatintu na\d tantu vantatu e\b n\b ra pa\d tiy\=ale anta \=urka\d l mu\b npu pole
	\item le koyil\index{cec}{koyil@\textit{k\=oyil} temple} pa\d n\d t\=arattile\index{cec}{pantaram@\textit{pa\d n\d t\=aram} trésorerie du temple} ka\d n\d tuko\d l\d lum pa\d tiyum mu\b n\b ne ka\b rpittu inta nilattuk\index{cec}{nilam@\textit{nilam} terre}
	\item ku a\d t\d tava\d nai pu\b ravarikku\index{cec}{vari@\textit{vari} taxe} ka\d nakk\=al u\d n\d t\=ana\b n\=a\b na kurukula ir\=aya\b n peril kutta
	\item kaippa\d ti mutalile celavi\d t\d tu \textbf{sarva}mum m\=a\textbf{nnya}m \=aka na\d takkumpa\d ti ka\b rpitta a\d lavukku im
	\item mariy\=atayile inta vakaikku u\d n\d t\=\i r \=a\b na po\b n koyil\index{cec}{koyil@\textit{k\=oyil} temple} pa\d n\d t\=arattile\index{cec}{pantaram@\textit{pa\d n\d t\=aram} trésorerie du temple} ka\d n\d tu ko
	\item \d n\d tu p\=ucai tiruppa\d ni t\=a\b lva\b ra na\d tatti o\b n\b rukkum a\~nc\=amal irukkavum ....y\=a\b ruvo-tara?
	\item \b lai yovalac\=acantaic\=ativariyi\b n vari\index{cec}{vari@\textit{vari} taxe} \textbf{sarva}mum \textbf{m\=anny}am\=aka\index{cec}{manyam@\textit{m\=anyam} non imposable} kalluve\d t\d tu vittu p\=ucai
	\item tiruppa\d ni t\=a\b lvu a\b ra na\d tattavum U ca\b ruvamum m\=a\b niniyam\=aka\index{cec}{manyam@\textit{m\=anyam} non imposable} kalluve\d t\d ti n\=a\d t-\d tikko\d n\d tu p\=u
	\item cai tiruppa\d ni na\d tatti o\b n\b rukkum a\~nc\=amal irukkavum U ... \textbf{d\=anap\=alanayo
	\item rmmadhye d\=an\=at \'sreyonup\=alanam||d\=an\=at svargamav\=apnoti p\=alan\=a-daccyutam padam|\'subhama
	\item stu} iru\.nko\d lap\=a\d n\d tin\=a\d tu va\.nk\=aramu\d taiy\=ar \textbf{pu\d spa}va\b napperum\=a\d l kurukular\=ayar per\=ale ca\textbf{rvam\=a}
	\item \textbf{nnya}m \=aka nirupam varukaiyil avar pa\d n\d nuvitta \textbf{dharmmam} ||
\end{enumerate}

\subsection*{CEC 20.3 R\'esum\'e}
Le texte est dat\'e de l'ann\'ee \textit{\'saka} courante 1410, ann\'ee [cyclique] Kilaka, du mois de Tul\=a, dans la quinzaine sombre, mercredi, dans le \textit{nak\d satra} Makara.
Sur l'ordre\index{gnl}{ordre royal} de
% \'Sr\=\i man pa\d t\d tukka\d t\d t\=ari pa\'sapaya\.nkara matiy\=atama\b n\b nar ma\d na[m\=a\b na] K\=a\~nc\=\i puravar\=adhi\'svara
Koneridevamah\=ar\=a\'s\=a, les taxes de quarante-deux \textit{veli} de terres\index{gnl}{terre} dans plusieurs villages doivent r\'e-int\'egrer la trésorerie\index{gnl}{tresorerie@trésorerie} du temple\index{gnl}{temple} de C\=\i k\=a\b li\index{gnl}{Cikali@C\=\i k\=a\b li}, comme auparavant, afin d'assurer les culte\index{gnl}{culte}s du temple\index{gnl}{temple}.
La protection en vers sanskrit (l.19-20) : \og parmi le don\index{gnl}{don} et la protection, la protection est meilleure que le don\index{gnl}{don}, par le don\index{gnl}{don} on obtient le ciel [et] par la protection le séjour éternel (la lib\'eration ?) ; que la prosp\'erit\'e soit !\fg

\section*{CEC 21}
\subsection*{CEC 21.1 Remarques}
L'inscription a \'et\'e relev\'ee dans l'ARE 1918 397. Elle se trouve sur le mur int\'erieur de gauche du pavillon d'entr\'ee du temple\index{gnl}{temple} principal de \'Siva\index{gnl}{Siva@\'Siva}. Elle se compose de vingt lignes qui courent sur deux m\`etres trente. La fin des lignes 9 \`a 18 se situe sur le c\^ot\'e du mur est. Les donn\'ees astronomiques permettent de dater le texte du \textbf{vendredi 11 avril 1511}.
L'\'edition propos\'ee est bas\'ee sur l'examen de la transcription, des clich\'es de \textsc{G. Ravindran} de l'EFEO et de la lecture \textit{in situ} avec \textsc{G. Vijayavenugopal}.

\subsection*{CEC 21.2 Texte}
\begin{enumerate}
	\item \textbf{svasti \'sr\=\i man mah\=ama\d n\d dal\=\i \'svaran hariharar\=aya vip\=a\d tan}
	\item p\=a\textbf{\d sai}kkuttappuvar\=ayar ka\d n\d ta\b n ka\d n\d tan\=a\d tu ko\d n\d du
	\item ko\d n\d tan\=a\d tu ko\textbf{\d t\=ak\=an} \textbf{p\=urvadak\d sa\d na pac\'sima}
	\item uttara\textbf{catusamudr\=adipati} ci\b rvirappi\b rat\=apaki\b ru\d t\d ta\d na
	\item teva maka ir\=ac\=api\b riti ir\=acciyam pa\d n\d ni
	\item aru\d l\=ani\b n\b ra cak\=attam 1000 4 100 3 10 3 mel cell\=ani\b n\b ra pi\b rac\=apatica\.nva
	\item cata[ra]ttu me\textbf{\d sa}n\=aya\b r\b ru p\=u\b ruvapa\textbf{k\d sa}ttu ti\b rutiyaiyum pe\b r\b ra cukki\b rav\=ara-mum acuvati
	\item na\textbf{k\d satra}ttu n\=a\d l ceya\.nko\d n\d taco\b lama\d n\d talam\=a\b na\index{cec}{Jayankontacolamandala@Jaya\.nko\d n\d tac\=o\b lama\d n\d dalam} to\d n\d taima\d n\d talattu
	\item k\=ali[y\=ur ko\d t\d ta]ttu.liva\b na....\=ur....maru\dots nta\b raco\b lacatuvetima|\d n\d talattu
	\item \=ur ka\d nakku\index{cec}{kanakku@\textit{ka\d nakku} comptable} m\=aca\b nappiriyar va. unayi\b n\=ur puttira\b n kom\=a\d na\b rku co\b lama\d n\d talat|-tu ir\=ac\=a
	\item tir\=acava\d lan\=a\d t\d tu\index{cec}{Rajadhirajavala@R\=aj\=adhir\=ajava\d lan\=a\d tu} tirukka\b lumalattu\index{cec}{Tirukkalumalam@Tirukka\b lumalam} c\=\i k\=alimutaliy\=ar\index{cec}{cikali@C\=\i k\=a\b li}\index{cec}{mutaliyar@\textit{mutaliy\=ar} chef (de monastère)} t\=a\b natt\=arum\index{cec}{tanattar@\textit{t\=a\b natt\=ar} employé du temple} nilamum\index{cec}{nilam@\textit{nilam} terre} ma\d nai-yum vi|laippira
	\item m\=a\d nam pa\d n\d n\=\i kku\d tuttapa\d ti ma\d nai o\b n\b rukku .llai \=avatu tiruk|ka\b lumalattil\index{cec}{Tirukkalumalam@Tirukka\b lumalam}
	\item ir\=ac\=akka\d l tampir\=a\b n\index{cec}{iracakkal@Ir\=ac\=akka\d l tampir\=a\b n} tiruviti teru meltu\d n\d tattil te\b nci[\b r\=a]kile me\b rka\d tai|ya ki\b lmel
	\item kolamu\d taiy\=a\b n......kku ma\d nai o\b n\b ru nila\index{cec}{nilam@\textit{nilam} terre} velikku tiruka\b lumalattil\index{cec}{Tirukkalumalam@Tirukka\b lumalam} nila\index{cec}{nilam@\textit{nilam} terre}|m k\=alaca\.n-ka\b na
	\item pa\d t\d til a\b rama\b na \dots\ o\b n\b ru te \=aka N 1V ma\d nai 1 inta nilam\index{cec}{nilam@\textit{nilam} terre} velikkum ma\d nai o\b n\b ruk|kum vilai\index{cec}{vilai@\textit{vilai} prix} \=aka
	\item ni\b n\b raipa\d ti pa\d n\d ni\b na po\b n pattu intappo\b n pattukku ma\d nai o\b n\b rum nilam\index{cec}{nilam@\textit{nilam} terre} veliyum vilai\index{cec}{vilai@\textit{vilai} prix} \=akavu|m inta ma\d nai
	\item o\b n\b rum nilam\index{cec}{nilam@\textit{nilam} terre} veliyum n\=aya\b n\=ar periyan\=aya\b n\=arkkum n\=aya\b n\=ar.....r\=a\d n\b rkum tam|-mu\d taiya ta\b n
	\item .................\d l.lakkaikku.....caittapa\d tiy\=a|(pa)lapu\b ravari\index{cec}{vari@\textit{vari} taxe}
	\item m\=a\b niyam\=aka\index{cec}{manyam@\textit{m\=anyam} non imposable} [inta ta\b nmam canti]r\=atittavaraiyum na\d takkakka\d tava............ma-tattu
	\item nilamuma\d naiyum\index{cec}{nilam@\textit{nilam} terre} vilaip piram\=a\d nam\index{cec}{vilai@\textit{vilai} prix!vilaippiramanam@\textit{vilaippiram\=a\d nam} document d'achat} pa\d n\d nikku\d tutta [......\d narkku k\=a\b limuta]liy\=ar\index{cec}{cikali@C\=\i k\=a\b li} t\=a\b n
	\item t\=a\b natt\=arom\index{cec}{tanattar@\textit{t\=a\b natt\=ar} employé du temple} v\=atuceytav\=ara\d nappira[mar e\b luttu]
\end{enumerate}

\subsection*{CEC 21.3 R\'esum\'e}

L'inscription semble enregistrer la mise en place par le chef\index{gnl}{chef} du monast\`ere\index{gnl}{monastère} (\textit{c\=\i k\=a\b limutaliy\=ar}), en m\'etayage, de terres\index{gnl}{terre} et des habitats appartenant au temple\index{gnl}{temple} pour la somme de dix pièce\index{gnl}{piece@pièce}s d'or par an. Un des signataires est V\=atuceytav\=ara\d nap-piramar.


\section*{CEC 22}
\subsection*{CEC 22.1 Remarques}
L'inscription a \'et\'e relev\'ee dans l'ARE 1918 399 et localis\'ee sur les c\^ot\'es droit et gauche du pavillon d'entr\'ee principal. Le texte est grav\'e plus exactement sur le soubassement est du pavillon d'entr\'ee du temple\index{gnl}{temple} de \'Siva\index{gnl}{Siva@\'Siva}, au nord, \`a l'int\'erieur.
Le donateur est un certain R\=amappan\=ayaka, fils de K\=o\d tal Va\'sava\d nan\=ayaka. Il s'agit du m\^eme donateur que dans la SII 23 271 de Tiruvi\d taimarut\=ur\index{gnl}{Tiruviraimarutur@Tiruvi\d taimarut\=ur} datant de \textbf{1535}\footnote{Cf. \textsc{Karashima} (2002: 160).}.

\subsection*{CEC 22.2 Texte}
\begin{enumerate}
	\item vikkiramavaru\d satti\b rkuccellum vi\textbf{\d su} Am cittirai M1l \dots\ mutali n\=a\d l m\=ayecurar-mutaliy\=ar\index{cec}{mutaliyar@\textit{mutaliy\=ar} chef (de monastère)} t\=a\b natt\=arum\index{cec}{tanattar@\textit{t\=a\b natt\=ar} employé du temple} \dots
	\item \d tecurappe\b ruvilaiy\=aka\index{cec}{vilai@\textit{vilai} prix!peruvilai@\textit{peruvilai} vente aux enchères} ko\d tal vacava\d nan\=ayakkar maka\b n ir\=amappan\=ayakkar ku-\d tutta k\=a\d niy\=a\d tcippa\d t\d tayam\index{cec}{kani@\textit{k\=a\d ni} droit, propriété} tevum tiruvumu\d tai \dots
	\item pa\d nippe\d n\d ta\d t\d ti pakav\=a\b n ma\d naikku te\b rku ka\d n\d n\=arku\b lali ma\b naikku va\d takku cetir\=appicca\b n ma\b naikku \dots ma\b nai \dots
	\item vilaiy\=aka\index{cec}{vilai@\textit{vilai} prix} ni\b n\b raiyam pa\d n\d ni\b na N k\textbf{\textsubring{r}\d s\d na}r\=aya\b n cantiyil amutuceyki\b ra amutil k\=a\d niy\=a\d tcikk\=arar\index{cec}{kani@\textit{k\=a\d ni} droit, propriété} a\d taippu\index{cec}{ataippu@\textit{a\d taippu} limite} pi\b rac\=ata
	\item amutu @ appam 1 va\d tai 1 a\d taikk\=ayamutu\index{cec}{ataikkay@\textit{a\d taikk\=ay} noix d'arec} 1 ilaiyamutu\index{cec}{ilaiyamutu@\textit{ilaiyamutu} feuille de bétel} 2 \=aka intavakai vilaiy\=aka\index{cec}{vilai@\textit{vilai} prix} ni\b n\b rayam pa\d n\d ni\b na
	\item appam o\b n\b rum va\d tai o\b n\b rum a\d taikk\=ayamu\index{cec}{ataikkay@\textit{a\d taikk\=ay} noix d'arec} \dots\ ilaiyamutu\index{cec}{ilaiyamutu@\textit{ilaiyamutu} feuille de bétel} ira\d n\d tum aramakku vilaikku\b ravi\b r \dots
	\item ko\d l\d lavum inta \dots yum amutupa\d niy\=aramum a\d takk\=ayilaiyamutum\index{cec}{ilaiyamutu@\textit{ilaiyamutu} feuille de bétel} cantir\=atitta-varaiyum a\b nupavip
 	\item \dots
	\item ivai \dots k\=uttamutali e\b luttu inta ni\b n\b rayattil cantir\=atittavaraiyum a\b nupavittuk-ko\d l\d lavum inta ni
 	\item \dots vaikalika\d tintaco\b la\textbf{brahm\=a}r\=aya\b n e\b luttu
\end{enumerate}

\subsection*{CEC 22.3 R\'esum\'e}
Le jour \dots\ du mois de Cittirai de l'ann\'ee Vikkirama, Ir\=amappan\=ayakkar, fils de Ko\d tal Vacava\d nan\=ayakkar donne une terre\index{gnl}{terre} pour offrir, avec la nourriture, une cr\^epe (\textit{appam}), un beignet de lentille (\textit{va\d tai}), une noix d'arec et deux feuilles de b\'etel lors de la c\'er\'emonie (\textit{canti}) \'etablie au nom du roi\index{gnl}{roi} K\textsubring{r}\d s\d nar\=aya\b n.
Cette d\'ecision doit \^etre ex\'ecut\'ee tant que [durent] la lune et le soleil. Un certain Kalika\d tintaco\b labrahm\=ar\=a-ya\b n a pos\'e sa signature.


\section*{CEC 23}
\subsection*{CEC 23.1 Remarques}

L'inscription a \'et\'e relev\'ee dans l'ARE 1918 398. Elle se trouve sous l'image\index{gnl}{image} d'Atik\=arananti, sur le mur sud-est du pavillon d'entr\'ee, \`a l'int\'erieur de l'enceinte, au-dessus du soubassement et du niveau de piliers. Elle se compose de sept lignes, tr\`es endommag\'ees. Les donn\'ees astronomiques permettent de dater le texte du \textbf{lundi 28 ao\^ut 1598}.
L'\'edition propos\'ee est bas\'ee sur l'examen de la transcription de l'ASI, des clich\'es de \textsc{G. Ravindran} de l'EFEO et de la lecture \textit{in situ} avec \textsc{G. Vijayavenugopal}. Nous remercions \textsc{Y. Subbarayalu} d'avoir lu et comment\'e notre \'edition.

\subsection*{CEC 23.2 Texte}
\begin{enumerate}
	\item \dots\ \textbf{mah\=amama\d n\d da}lecura\textbf{n hariharar\=aya} vip\=a\d ta\textbf{n} p\=a\textbf{\d sai}kkuttappu(va)r\=a-yar ka\d n\d tan ka\d n\d ta\b n\=a\d tu ko\d n\d tu ko\d n\d ta\b n\=a\d tu ko\d t\=at\=a\b n p\=u\b ruvate\textbf{k\d sa}\d na paccima uttiracatu\textbf{sa}mu \dots ti \textbf{\'sr\=\i}ve\.nka\d ta\textbf{devasya}r\=ayar piritivi
 	\item \dots ll\=a ni\b n\b ra cak\=aptam 1000 5 100 2 10 cell\=a ni\b n\b ra vi\d lum\textbf{bi} \textbf{sa}\.nva\b r\textbf{sa}rattu \textbf{si\.nha}n\=aya\b r\b ru p\=uruvapa\textbf{k\d sa}ttu \textbf{sapta}miyum \textbf{so}mav\=aramum pe\b r\b ra anur\=at\=a na\textbf{k\d satra}ttu n\=a\d l \dots miy\=a\b r \textbf{\=apaduddh\=ara\d na}\b rku\index{cec}{apaduddh@\=Apaduddh\=ara\d na\b r}
 	\item \dots\ tiruppa\d ni velaikk\=ararkku \textbf{s}v\=amiy\=ar \=apatu\textbf{ddh\=a}ra\d na vira\textbf{pras\=a}tiy\=ana r\=aca-ri\textbf{\d si} vi\d t\d tale\textbf{\'sva}racco\b lako\b n\=ar \textbf{dhamma}m\=aka \textbf{s}v\=amiy\=ar \=apa\textbf{du} \dots \b rku ma\textbf{h\=a-bhi\d se}kam na\d t\d takkave
 	\item \dots \d tu\textbf{\d sa}kattirumeni \textbf{go}miya \textbf{a}pi\d ta \textbf{sna}panam pa\d n\d ninapa\d tiy\=ale \textbf{bhut\=a}\b nattukku \b n\=aya\b n\=ar tirutto\d nipuramu\d taiya\b n\=aya\b n\=ar\index{cec}{Tiruttoni@Tirutt\=o\d nipuramu\d taiya n\=aya\b n\=ar, \'Siva} tiru\b n\=amattukk\=a\d niy\=ana\index{cec}{kani@\textit{k\=a\d ni} droit, propriété} ci \dots\ n\=aya\b n\=ar koyil\index{cec}{koyil@\textit{k\=oyil} temple} ka
 	\item \dots
	\item ca\textbf{ndr\=arkka}m\=aka ca\textbf{ndr\=adi}tittavaraiyum \textbf{sarva}m\=a\b niyam\=aka\index{cec}{manyam@\textit{m\=anyam} non imposable} \textbf{putrapautra-pa}ramparaiy\=aka a\b nupavittu ko\d l\d lakka\d tavar\=akavum inta \textbf{prati\d s\d tai}kku \=aka piramappuran\=ayaka \dots
\end{enumerate}

\subsection*{CEC 23.3 R\'esum\'e}
Pendant le r\`egne de Ve\.nka\d tadevasyar\=ayar, l'ann\'ee \textit{\'saka} courante 1520, l'ann\'ee [cyclique] Vi\d lampa, le mois de Si\.nha, le septi\`eme jour de la quinzaine claire, lundi (\textit{soma}), dans le \textit{nak\d satra} Anur\=at\=a, pour le m\'erite de Vi\d t\d tale\'svaracco\b lako\b n\=ar, chef\index{gnl}{chef} de monast\`ere\index{gnl}{monastère} ? (\textit{r\=ajar\d si}), on installe l'image\index{gnl}{image} du Seigneur \=Apaduddh\=ara\d nar et on lui donne une terre\index{gnl}{terre} non imposable afin d'effectuer pour lui le grand ondoiement (\textit{mah\=abhi\d sekam}).


\section*{CEC 24}

L'inscription a \'et\'e relev\'ee dans l'ARE 1918 401. Elle se trouverait sur le soubassement sud de la galerie int\'erieure. La transcription a disparu \`a Mysore. Le rapport fait \'etat d'un texte mentionnant les titres (\textit{biru\d da}) de Vi\d t\d thaladevamah\=ar\=aja qui retrace la g\'en\'ealogie de Vi\d t\d thala depuis des rois mythiques, en passant par les Ch\=alukyas de l'ouest, tout en louant les conqu\^etes de ses anc\^etres.

\section{Temple d'\=A\d lu\d taiyapi\d l\d laiy\=ar}

\begin{figure}[h!]
  \centering
  \includegraphics[height=7cm]{docthese/photoCIIKAALI/TNCsud.jpg}
  \caption{Face sud de la chapelle de Campantar, vue de l'intérieur de l'enceinte, C\=\i k\=a\b li (cliché U. \textsc{Veluppillai}, 2006).}
  \end{figure}

\section*{A. Temple}

Les textes de cette partie ont \'et\'e \'edit\'es sur la base de l'examen des transcriptions de l'ASI et surtout, de photographies (G. \textsc{Ravindran}, EFEO) et de la lecture \textit{in situ} avec G. \textsc{Vijayavenugopal}.

\section*{CEC 25}
\subsection*{CEC 25.1 Remarques}

L'\'epigraphe a été relev\'ee dans ARE 1918 380 et localis\'ee sur le mur sud du temple\index{gnl}{temple} de Campantar\index{gnl}{Campantar}. Elle date de la troisi\`eme ann\'ee de r\`egne de Tribhuvanacakravartin Kulottu\.ngac\=o\b ladeva. ARE 1918, appendix E, pr\'ecise, compte tenu des informations astronomiques, qu'elle date du \textbf{lundi 19 ao\^ut 1135}. Ainsi, le roi\index{gnl}{roi} est identifi\'e comme Kulottu\.nga II\index{gnl}{Kulottu\.nga II}. Il s'agit vraisemblablement de l'inscription la plus ancienne, encore en place, sur le site de C\=\i k\=a\b li\index{gnl}{Cikali@C\=\i k\=a\b li}.

L'inscription se situe sur la face sud de la chapelle et plus exactement sur le soubassement, \`a l'est de la fen\^etre \`a claire-voie. Le lapicide s'est appliqu\'e \`a graver sur les diff\'erents composants du soubassement de mani\`ere continue: les deux premi\`eres lignes figurent sur les deux premi\`eres tables entrecoup\'ees par un \'el\'ement saillant d\'ecoratif. Les lignes 3 \`a 6 couvrent un espace rentrant. Puis la ligne 7 appara\^it sur la table en-dessous. Les lignes 8 \`a 10 se trouvent sur la face sup\'erieure horizontale d'un \'el\'ement saillant et enfin, la derni\`ere ligne vient sur la face m\'ediane verticale de ce m\^eme \'el\'ement.

\begin{figure}[h!]
  \centering
  \includegraphics[width=10cm]{docthese/Toniyappartemple299.jpg}
  \caption{CEC 25 (cliché G. \textsc{Ravindran}/EFEO, 2005).}
  \end{figure}

Le texte enregistre la mise en place d'une terre\index{gnl}{terre} pour nourrir Campantar\index{gnl}{Campantar} avec du riz\index{gnl}{riz} au lait\index{gnl}{lait} par l'assemblée\index{gnl}{assemblée} de Tirukka\b lumalam\index{cec}{Tirukkalumalam@Tirukka\b lumalam}.

\subsection*{CEC 25.2 Texte}
\begin{enumerate}
	\item $\pm$\footnote{Ce symbole reproduit un signe compos\'e d'un trait vertical de la hauteur des graph\`emes qui est coup\'e horizontalement par cinq traits plus petits. Sa valeur et sa signification restent inconnues.} \textbf{tribhu}va\b naccak\textbf{kra}va[tti]ka\d l [\textbf{\'sr\=\i}]kulottu\.nkaco\b lateva\b rkku\index{cec}{kulottungacoladeva@Kulottu\.ngac\=o\b ladeva} y\=a\d n\d tu 3 vatu \textbf{si}\.n\textbf{ha} n\=aya\b r\b ru aparapa\textbf{k\d sa}ttu navamiyum
	\item ti\.nka\d tki\b lamaiyum pe\b r\b ra [pur\=a\d ta]ttu n\=a\d l ir\=a\textbf{j\=a}tir\=a\textbf{ja}va\d lan\=a\d t\d tu tiruka\b lumala-n\=a\d t\d tu\index{cec}{Tirukka\b lumalan\=a\d tu} \textbf{brahmadeya}m\index{cec}{brahmadeya@\textit{brahmadeya}} tirukka\b lumalattu\index{cec}{Tirukkalumalam@Tirukka\b lumalam} \textbf{sa}paikkuccamainta
	\item peru\.nku\b rip perumakka\d lil (parattu)v\=aci teva\b n kum\=aranum\index{cec}{kumaran@Kum\=ara\b n} m\=a\d tala\b n\index{cec}{matalan@M\=a\d tala\b n} civateva\b n\index{cec}{Civateva\b n} tirutto\d nipuramu\d taiy\=a\b num\index{cec}{Tonipuramutaiyan@T\=o\d nipuramu\d taiy\=a\b n} p\=al\=a\'sr\=\i
	\item ya\b n teva\b n va\d tuka\b num\index{cec}{vatukan@Va\d tuka\b n} [pira]\d laiyavi\d ta\.nka\b n\index{cec}{Piralaiya@Pira\d laiyavi\d ta\.nka\b n} c\=\i tara\b num v\=acciya\b n\index{cec}{Vacciyan@V\=acciya\b n} n\=ata\b n kum\=ara-num\index{cec}{kumaran@Kum\=ara\b n} v\=acciya\b n\index{cec}{Vacciyan@V\=acciya\b n} [n\=a]ta\b n tirutto\d nipuramu\d taiy\=a\index{cec}{Tonipuramutaiyan@T\=o\d nipuramu\d taiy\=a\b n}
	\item \b num v\=acciya\b n\index{cec}{Vacciyan@V\=acciya\b n} civateva\b n\index{cec}{Civateva\b n} ci..va\b num p\=arattuv\=aci k\=atil ve\d nku\b laiya\b n\index{cec}{venkulaiyan@Ve\d nku\b laiya\b n} tirutto\d ni-puramu\d taiy\=a\b num\index{cec}{Tonipuramutaiyan@T\=o\d nipuramu\d taiy\=a\b n} v\=acciya\b n\index{cec}{Vacciyan@V\=acciya\b n} c\=attakumara\b n
	\item {[\dots \d taiy\=a\b n civateva\b num\index{cec}{Civateva\b n} u\d l\d li\d t\d taperu\.n \dots ]\footnote{Cette ligne est actuellement recouverte de ciment. La lecture de la transcription est adopt\'ee.}}
	\item m \=a\d lu\d taiyapi\d l\d laiy\=a\b rkkut tirupp\=a\b rponakamamutu ceytaru\d la na\b r\d tpu\b naik\=av\=a\b n ut[tama]co\b lanall\=uril\index{cec}{Uttamaco\b lanall\=ur} \=urkki\b li\b raiyi
	\item .\textbf{sa}ntir\=atittaval cellavi\d t\d ta N 2 M in[nila]m\index{cec}{nilam@\textit{nilam} terre} i\b nN i[ra*]\d n\d tu m\=a[vum kaik]ko\d n\d tu i\b lavupa\d t\d ta .... innilattukku\index{cec}{nilam@\textit{nilam} terre} talaim\=a\b ru \=a\d lu
	\item \d taiyapi\d l\d laiy\=ar tevarka\b nmika\d l\d lukku vi\b r\b rukku[\d tut]ta nila\index{cec}{nilam@\textit{nilam} terre}vilaiy\=ava\d nam\index{cec}{vilai@\textit{vilai} prix!vilaiyavanam@\textit{vilaiy\=ava\d nam} document de vente} [e\.nka]-\d lukku \textbf{sa}paip potu[v\=aykki\d ta]\b nta na\b rpu\b naik\=av\=a\b n uttamaco\b la
	\item .....[ca]\d n\d tecuravatikku ki\b lakku ......vatikku te\b rkku te\b r................... innilam\index{cec}{nilam@\textit{nilam} terre} i-ra\d n\d tu ... m\=avum tirupp\=a\b rponaka
	\item pu\b ram\=aka \textbf{sa}ntir\=atittavarai
\end{enumerate}

\subsection*{CEC 25.3 Traduction}
En la 3\up{e} ann\'ee [de r\`egne] de \'Sr\=\i kulottu\.ngac\=o\b ladeva\index{cec}{kulottungacoladeva@Kulottu\.ngac\=o\b ladeva}, empereur des trois mondes, le mois de \textit{Si\.nha}, le neuvi\`eme jour de la quinzaine sombre, lundi, dans le [\textit{nak\d satra}] \textit{Pur\=a\d tam}; les membres parmi la grande assemblée\index{gnl}{assemblée}\footnote{La \textit{sabh\=a} est l'assemblée\index{gnl}{assemblée} attach\'ee aux villages de type \textit{brahmadeya}\index{cec}{brahmadeya@\textit{brahmadeya}}. Compos\'ee de membres brahmane\index{gnl}{brahmane}s, comme ici, elle g\`ere au niveau local\index{gnl}{local} les affaires du village. Sur cette organisation, sa composition, son fonctionnement et son r\^ole administratif voir \textsc{Nilakanta Sastri} (*2000 [1955]: 492-503), \textsc{Karashima} (*2001a [1966]), \textsc{Subbarayalu} (*2001h [1982]: 91-92), \textsc{Veluthat} (1993: 203-207) et, bien s\^ur, les tr\'es c\'el\`ebres inscriptions d'Uttaram\=er\=ur (SII 6 283 et 284). Pour des expos\'es particuliers, cf. \textsc{Minakshi} (1938: 124-125) qui pr\'esente la \textit{sabh\=a} sous les Pallava\index{gnl}{Pallava} et \textsc{Veluthat} (1985) qui expose ses racines dans les textes dharma\'s\=astriques.} de Tirukka\b lumalam\index{cec}{Tirukkalumalam@Tirukka\b lumalam}, \textit{brahmadeya}\index{cec}{brahmadeya@\textit{brahmadeya}} de Tirukka\b lumalan\=a\d tu dans le R\=aj\=adhir\=ajava\d lan\=a\d tu\index{cec}{Rajadhirajavala@R\=aj\=adhir\=ajava\d lan\=a\d tu} --- [comprenant] Parattuv\=aci Teva\b n Kum\=aran\index{cec}{kumaran@Kum\=ara\b n}, M\=a\d tala\b n\index{cec}{matalan@M\=a\d tala\b n} Civateva\b n\index{cec}{Civateva\b n} Tirutt\=o\d nipuram\index{gnl}{Tonipuram@T\=o\d nipuram!Tirutt\=o\d nipuram}u\d taiy\=a\b n, P\=al\=a\'sr\=\i ya\b n Teva\b n Va\d tuka\b n, Pira\d laiyavi\d ta\.nka\d l\index{cec}{Piralaiya@Pira\d laiyavi\d ta\.nka\b n} C\=\i tara\b n, V\=acciya\b n\index{cec}{Vacciyan@V\=acciya\b n} N\=ata\b n Kum\=aran\index{cec}{kumaran@Kum\=ara\b n}, V\=acciya\b n\index{cec}{Vacciyan@V\=acciya\b n} N\=ata\b n Tirutt\=o\d nipuram\index{gnl}{Tonipuram@T\=o\d nipuram!Tirutt\=o\d nipuram}u\d taiy\=a\b n, V\=acciya\b n\index{cec}{Vacciyan@V\=acciya\b n} Civateva\b n\index{cec}{Civateva\b n} Ci..va\b n, P\=arattuv\=aci Katilce\d n-ku\b laiya\b n Tirutt\=o\d nipuram\index{gnl}{Tonipuram@T\=o\d nipuram!Tirutt\=o\d nipuram}u\d taiy\=a\b n, V\=acciya\b n\index{cec}{Vacciyan@V\=acciya\b n} C\=attakum\=a\b na \dots --- pour nourrir en riz\index{gnl}{riz} au lait\index{gnl}{lait} \=A\d lu\d taiyapi\d l\d laiy\=ar, prennent en main une terre\index{gnl}{terre} de 2 \textit{m\=a} \textit{\=urkki\b li\b raiyili}\index{cec}{iraiyili@\textit{i\b raiyili} non imposable}\footnote{Ce terme renvoie aux terres\index{gnl}{terre} non imposables qui sont \`a la charge de l'assemblée\index{gnl}{assemblée} villageoise \textit{\=ur} (\textsc{Subramaniam} 1957 et \textsc{Subbarayalu} 2003).} \`a Na\b rpu\b naik\=av\=a\b n Uttamaco\b lanall\=ur  et la donnent tant que durent lune et soleil aux \textit{devakarm\=\i} du [temple\index{gnl}{temple}] d'\=A\d lu\d taipi\d l\d laiy\=ar selon le document de vente.

La terre\index{gnl}{terre} qui \'etait commune \`a notre assemblée\index{gnl}{assemblée}, dans Na\b rpu\b naik\=av\=a\b n Uttamaco\b la-nall\=ur, \`a l'est de la \textit{vati}\index{cec}{vati@\textit{vati}} Ca\d n\d tecura\index{cec}{candesvara@Ca\d n\d de\'svara} et au sud du canal \dots, cette terre\index{gnl}{terre} de 2 \textit{m\=a} est faite terre\index{gnl}{terre} pour riz\index{gnl}{riz} au lait\index{gnl}{lait} tant que durent lune et soleil.

\section*{CEC 26}
\subsection*{CEC 26.1 Remarques}

L'\'epigraphe a été relev\'ee dans ARE 1918 381 et localis\'ee sur le mur sud du temple\index{gnl}{temple} de Campantar\index{gnl}{Campantar}. Elle date de la quatri\`eme ann\'ee de r\`egne de Tribhuvana-cakravartin Kulottu\.ngac\=o\b ladeva que \textsc{Mahalingam} (1992: 550, Tj. 2414) n'identifie pas. Ce dernier reprend le r\'esum\'e de l'ARE: \og Gift of land for setting up image\index{gnl}{image}s (\string?) and restoring those that had been already set up and had suffered damage\fg.

Il n'est pas question d'image\index{gnl}{image}s. En effet, le texte enregistre un don\index{gnl}{don} de terre\index{gnl}{terre} pour r\'e-ouvrir le \textit{Tirukkaiko\d t\d ti} de la chapelle qui conservait les \textit{Tirumu\b rai}\index{gnl}{Tirumurai@\textit{Tirumu\b rai}}, pour remplacer ceux qui sont endommag\'es et pour en installer de nouveaux. Malgr\'e son caract\`ere exceptionnel, ce texte n'a connu qu'un rayonnement\index{gnl}{rayonnement} limit\'e dans la litt\'erature secondaire sans doute \`a cause du r\'esum\'e erron\'e de l'ARE\footnote{\textsc{Ve\d l\d laiv\=ara\d nam} (*1994 [1962 et 1969]: 52-53) donne une version du texte et \textsc{Swamy} (1972: 107) s'en sert dans sa d\'emonstration sans mentionner les r\'ef\'erences.}.

L'inscription se compose de trente-huit lignes qui s'\'etendent sur le mur sud, entre deux pilastres, \`a l'est de la fen\^etre \`a claire-voie, au-dessus de CEC 25. Des restaurations ont eu lieu depuis 1918. Les pierres de la chapelle furent ciment\'ees et les lignes \`a proximit\'e de ces espaces sont donc illisibles. Ainsi, la majorit\'e des conjectures propos\'ees pour ce texte suivent la lecture de la transcription de l'ASI faite en 1918.

L'emplacement de ce texte sur le mur sud, juste au-dessus de CEC 25, ainsi que la proximit\'e des dates de r\`egne et la concordance d'un des membres de l'assemblée\index{gnl}{assemblée} (M\=a\d tila\b n Civateva\b n\index{cec}{Civateva\b n} Tirutt\=o\d nipuram\index{gnl}{Tonipuram@T\=o\d nipuram!Tirutt\=o\d nipuram}u\d taiy\=a\b n) permettent d'arguer que l'inscription date de la quatri\`eme ann\'ee de r\`egne de Kulottu\.nga II\index{gnl}{Kulottu\.nga II}, soit de \textbf{1136}.

\subsection*{CEC 26.2 Texte}
\begin{enumerate}
	\item {[\textbf{tribhu}]va\b naccak\textbf{kra}va[ttika\d l] kulottu\.nka}
	\item {[co]\b lateva\b rku\index{cec}{kulottungacoladeva@Kulottu\.ngac\=o\b ladeva} y\=a\d n\d tu 4 vatu ir\=a\textbf{j\=a}}
	\item {(\textbf{dhi})r\=a\textbf{ja}va\d lan\=a[\d t\d tu tiru]kka\b lumalan\=a\d t\d tu\index{cec}{Tirukka\b lumalan\=a\d tu} \textbf{bra}}
	\item {(\textbf{h})\textbf{made}cam\index{cec}{brahmadeya@\textit{brahmadeya}} tirukka\b lumala[m]\index{cec}{Tirukkalumalam@Tirukka\b lumalam} ka\b rka\d ta[ka]n\=aya\b ru}
	\item {(mu)tal kir\=amak\=ariya\~ncey[ki]\b rak\=u\d t\d ta[y\=ava]rumkku}
	\item {[tirum\=a\d likai \=a\d lu\d taiyapi\d l\d laiy\=ar tiru]m\=a\d likai tami\b l viraka[r]}
	\item {[ka]\d n\d tu ikkoyi\b rtirukkaiko\d t\d tiyil e\b luntaru\d li irukki\b ra tiru[mu\b raika\d l tirukk\=a]}
	\item ppu nikki a\b livu\d l\d li\b na e\b luntaru\d livittum ma\b r\b rum putit\=aka e\b luntaru\d li[vikku]
	\item {[m tirumu\b raika\d l\dots]}
	\item .. i\b raiyiliy\=aka\index{cec}{iraiyili@\textit{i\b raiyili} non imposable} i\d t\d ta nilam\index{cec}{nilam@\textit{nilam} terre} ivv\=ur ca\d n\d te\textbf{\'sva}ravatikkuk\index{cec}{candesvara@Ca\d n\d de\'svara}\index{cec}{vati@\textit{vati}} ki\b lakku ni\b n\b r\=a
	\item nv\=aykk\=alukku\index{cec}{vaykkal@\textit{v\=aykk\=al} canal} va[\d ta]kku muta\b rka\d n\d n\=a\b r\b ru ira\d n\d t\=a\~ncati[rat]tu ki\b l i\b raiy\=a
	\item \.nku\d t\d ti p\=al p\=aratt\=uv\=aci \textbf{\'sr\=\i}k\=a\b lin\=a\d tu\d taiy\=a\b n\index{cec}{cikali@C\=\i k\=a\b li} tiruve\d nk\=a\d tu\d taiy\=a\b n\index{cec}{Tiruvenkatu@Tiruve\d nk\=a\d tu\d taiy\=a\b n} nila\index{cec}{nilam@\textit{nilam} terre}
	\item ..vi\d lai N 2M½ ½K Q i\b n\b nilam\index{cec}{nilam@\textit{nilam} terre} irum\=avarai araik[k\=a]\d ni mu[nti]rikai
	\item {[\dots\ i\b n\b nila\~nc\=u\b l\textbf{nta} kulaiyum ti\d talum ku\d lamum ki\b lkka\d tai\textbf{nta} co\b la\dots ]}
	\item ..me\b rka\d taiya ku\d ti[yiru]pputti\d tar \b nilamum\index{cec}{nilam@\textit{nilam} terre} cuttamalivatikku\index{cec}{Cuttamali}\index{cec}{vati@\textit{vati}} me\b rku \b ni\b n\b r\=a
	\item {[\b nv\=aykk\=alu]kku\index{cec}{vaykkal@\textit{v\=aykk\=al} canal} va\d takku [muta]\b rka\d n\d n\=a\b r\b ru 3n catirattu p\=al\=a\textbf{\'sr\=\i}yan m\=ate}
	\item {[va\b n...]\b nilattu\index{cec}{nilam@\textit{nilam} terre} va\d tame[\b r]ka\d taiyap pu\b nceynilattu\index{cec}{nilam@\textit{nilam} terre} u\d taiya punceynilam\index{cec}{nilam@\textit{nilam} terre}}
	\item {[\dots\ u\d lpa\d tak kaikko\d n\d tu ca\textbf{nti}r\=atittavarai k\=acu\index{cec}{kacu@\textit{k\=acu} pièces de monnaie} ko\d l\d l\=a i\b rai]}
	\item yiliy\=akavum cilvari\index{cec}{vari@\textit{vari} taxe} peruvari ve\d t\d ti [mu]\d t\d taiy\=a\d l ko\d l\d l\=atom\=aka[vum co]
	\item \b n\b nom i\b n\b nila\.nkaikko\d n\d tu a[\b nu]pavittu tirumu\b rai tirukk\=appu\b nikki ippa\d tiye
	\item tirum\=a\d likaiyile kallilu\~ncempilum ve\d t\d tikko\d lka pa\d niy\=al u\b rka\d nakkup\index{cec}{kanakku@\textit{ka\d nakku} comptable} pa
	\item {[....piri]ya\b n e\b luttu \textbf{sabhai}y\=aril\index{cec}{sabha@\textit{sabh\=a} assemblée} [e\b lu]tti\d t\d t\=ar p\=arattuv\=aci tevaka\b n ce\b n\=a[pa]nampi m\=a}
	\item {[\d ti]la\b n civateva\b n\index{cec}{Civateva\b n} tirutto\d nipuramu[\d tai]y\=a\b n\index{cec}{Tonipuramutaiyan@T\=o\d nipuramu\d taiy\=a\b n} p\=arattuv\=aci teva\b n tillain\=a[ya]ka\b n\index{cec}{Tillain\=ayakar} v[\=a]}
	\item {[\textbf{jya}n kalika\d tint\=a\b n \dots]}
	\item maikku pa\d t\d tappiriya\b n e\b luttu v\=a\textbf{jya}n kum\=ara\b n\index{cec}{kumaran@Kum\=ara\b n} ku..\b n\=a\b na tiruve\d nk\=a\d tu\d taiy\=a\b n\=a\b na\index{cec}{Tiruvenkatu@Tiruve\d nk\=a\d tu\d taiy\=a\b n} tiru[m]
	\item \=a\d likai \b nampi\index{cec}{Nampi} p\=al\=a\textbf{\'sr\=\i}ya[n*] tirucci\b r\b rampalamu[\d taiy]\=a\b n\index{cec}{tiruccirrampalamutaiyar@Tirucci\b r\b rampalamu\d taiy\=ar} tirucci\b r\b rampalamu\d tai-y\=a\b n\footnote{Ce nom seul au milieu d'une liste mentionnant \textit{gotra\index{gnl}{gotra@\textit{gotra}}} et nom du père\index{gnl}{pere@père} est incompr\'ehensible. Est-ce le lapicide qui l'a r\'ep\'et\'e machinalement?} \textbf{sai}[\textbf{j\~n\=a}]
	\item ta\b namaikkup pa\d t\d tappiriya\b n v\=a\textbf{jya}n arukka\b n [\=a]\d lu\d taiy\=a\b n p\=al\=a\textbf{\'sr\=\i}ya\b n kumara\b n n\=\i laka
	\item \d n\d ta\b n p\=al\=a\textbf{\'sr\=\i}ya\b n pira\d layavi\d ta\.nka\b n\index{cec}{Piralaiya@Pira\d laiyavi\d ta\.nka\b n} tirut[to]\d nipuramu\d taiy\=a\b n v\=a\textbf{jya}\b n teva\b n tirut
	\item to\d nipuramu\d taiy\=a\b n\index{cec}{Tonipuramutaiyan@T\=o\d nipuramu\d taiy\=a\b n} \textbf{saij\~n\=a}ta\b namaikku pa\d t\d ta[p]piriya\b n (c\=ap\=anatice\b nta\b n) vira-pattara\b n [\textbf{s}]
	\item {(\textbf{aij\~n\=a}ta\b namaikkum p\=arattuv\=aci [m\=a]teva\b n tiruva\textbf{gni\'sva}ramu\d taiy\=a\b n \textbf{saij\~n\=a}-ta\b namaikku)}
	\item pa\d t\d tappuriya\b n pira\d layavi\d ta\.nka\b n\index{cec}{Piralaiya@Pira\d laiyavi\d ta\.nka\b n} teva[\b n m]\=a\d tila\b n civateva\b n\index{cec}{Civateva\b n} p\=alentiramavuli \=a\textbf{stri}ya\b n [c\=u]riyat
	\item {[e]va\b nkolam\=aki\b ni\b n\b r\=a\b n tirucci\b r\b rampalamu\d taiy\=a\b n\index{cec}{tiruccirrampalamutaiyar@Tirucci\b r\b rampalamu\d taiy\=ar} saij\~n\=ata\b namaikku \b n\=a\b n\=u\b r\b ru }
	\item {[va]ppiriya\b n v\=acciya\b n\index{cec}{Vacciyan@V\=acciya\b n} kumara\b n pori......n\=ata\b n kumara\b n p\=arat\=aya\b n matic\=uta\b n tiruv}
	\item e\d nk\=a\d tu\d taiy\=a\b n\index{cec}{Tiruvenkatu@Tiruve\d nk\=a\d tu\d taiy\=a\b n} p\=arattuv\=aci m\=ateva\b n pira\d layavi\d ta\.nka\b n\index{cec}{Piralaiya@Pira\d laiyavi\d ta\.nka\b n} m\=a\d tala\b n\index{cec}{matalan@M\=a\d tala\b n} civateva\b n\index{cec}{Civateva\b n} tiruva-\b nanti....
	\item ...nampi\index{cec}{Nampi} \textbf{saij\~n\=a}ta\b namaikku a\textbf{\d s\d ta}mutti n\=a\b rpatte\d n\d n\=ayira\b nampi\index{cec}{Nampi} v\=acciya\b n\index{cec}{Vacciyan@V\=acciya\b n} nak-ka\b n \textbf{sai}
	\item {[\textbf{j\~n\=a}ta\b namaikku acuku\d taiy\=a\b n tiruvaiy\=a\b ru\d taiy\=a\b n k\=a\.n\dots\footnote{Ces trois derni\`eres lignes ne se trouvent pas sur le mur mais sur la partie sup\'erieure horizontale d'un \'el\'ement saillant du soubassement, enti\`erement tapiss\'ee de mortier qui rend la lecture impossible.}
	\item v\=urav\=ap\=a \dots\ i\textbf{\d sabha}teva\b n kavu\d niya\b n\index{cec}{kavuniyan@\textit{kavu\d niya\b n gotra}} teva\b n tirucci\b r\b rampalamu\d taiy
	\item \=a\b n\index{cec}{tiruccirrampalamutaiyar@Tirucci\b r\b rampalamu\d taiy\=ar} \dots\ tirucci\b r\b rampalamu\d taiy\=a\b n\tdanda |]}
\end{enumerate}

\subsection*{CEC 26.3 Traduction}
(1-6) En la 4\up{e} ann\'ee [de r\`egne] de Kulottu\.ngac\=o\b ladeva\index{cec}{kulottungacoladeva@Kulottu\.ngac\=o\b ladeva}, empereur des trois mondes;
\`a partir du mois de \textit{Karka\d taka}, [un acte est adress\'e] \`a tous ceux du \textit{k\=u\d t\d tam}\footnote{Ce terme d\'esignerait le corps ex\'ecutif d'une \textit{sabh\=a}, \textsc{Nilakanta Sastri} (*2000 [1955]: 498-501).} qui s'occupe des affaires villageoises \`a Tirukka\b lumalam\index{cec}{Tirukkalumalam@Tirukka\b lumalam}, \textit{brahmadeya}\index{cec}{brahmadeya@\textit{brahmadeya}} de Tirukka\b lumalan\=a\d tu\index{cec}{Tirukka\b lumalan\=a\d tu} dans le R\=aj\=adhir\=ajava\d lan\=a\d tu\index{cec}{Rajadhirajavala@R\=aj\=adhir\=ajava\d lan\=a\d tu}. [Cet acte] a été vu par l'expert en tamoul du palais\footnote{\textit{Tirum\=a\d likai} renvoie \'evidemment au temple\index{gnl}{temple} de Campantar\index{gnl}{Campantar}.} d'\=A\d lu\d taiyapi\d l\d lai.

(7-17) [voici] les terres\index{gnl}{terre} donn\'ees comme non imposables pour ouvrir\footnote{Litt\'eralement \og ayant retir\'e la protection sacr\'ee\fg, cette expression d\'enote, pour nous, l'ouverture des portes. En effet, \textit{tiruk\=appu} prend le sens de porte\index{gnl}{porte} en contexte \'epigraphique d'après \textsc{Subbarayalu} (2003, s.v.).} les \textit{Tirumu\b rai}\index{gnl}{Tirumurai@\textit{Tirumu\b rai}} qui \'etaient install\'es dans le \textit{Tirukkaiko\d t\d ti}\footnote{\textit{Tirukkaiko\d t\d ti} est la pièce\index{gnl}{piece@pièce} dans le temple\index{gnl}{temple} o\`u \'etaient r\'ecit\'es les hymne\index{gnl}{hymne}s (cf. ARE 1908 203, 414, 454 et ARE 1909, paragraphe 51 ainsi que dans notre thèse 1.3 et 4.1.2).
%ARE 1908 203, Mantrapurisvara temple at Kovilur, Tanjore district, 22nd year of ?, gift of land for the eulogists in the Tirukkaikko??i of the temple, by order of Sokkanaya? alias Rajagambira So?iyavaraiya?.-	ARE 1908 414, Vi?inatar at Vi?imi?alai, Nannilam taluk, Ja?avarma? Sundara Pa??ya II ?-9, 1285, gift of land for the recital of the Tirumu?ai hymns in the Tirukkaikko??i which was constructed for that purpose in the temple during the time of the king Narasi?hadeva.-	ARE 1908 454, detached stones lying in the Umamahesvarasvamin temple in Tirukkaravacal, Nagapattinam taluk, Tanjore district, RIII-28, 1244 : gift of land for feeding the persons who recite the Tirumu?ai in the tirukkaikko??i of the temple at Tirukka?ayil by the residents of Muvur a village in Puliyurna?u, a sd of Arumo?idevava?ana?u.-	SII IV 225, Citamparam\index{gnl}{Citamparam}-	ARE 1918 10, Punyanathasvami temple, Tiruvi?aivayal, Rajendra III, réf à tirupa??iya?ai wherein the sacred hymns were consecrated and worshipped.-
} de ce temple\index{gnl}{temple}, pour r\'einstaller ceux qui \'etaient ab\^im\'es, puis pour en installer de nouveaux:\\
- [La terre\index{gnl}{terre} qui  se trouve] \`a l'est de la \textit{vati}\index{cec}{vati@\textit{vati}} Ca\d n\d te\'svara\index{cec}{candesvara@Ca\d n\d de\'svara} de ce village, au nord du canal Ni\b n\b r\=a\b n, le deuxi\`eme carr\'e du premier canalicule, pr\`es de Ki\b li\b raiy\=a\.nku\d t\d ti. [Elle a \'et\'e achet\'ee] \`a P\=aratt\=uv\=aci \'Sr\=\i k\=a\b lin\=a\d tu\d taiy\=a\b n\index{cec}{cikali@C\=\i k\=a\b li} Tiruve\d nk\=a\d tu\d taiy\=a\b n\index{cec}{Tiruvenkatu@Tiruve\d nk\=a\d tu\d taiy\=a\b n}. Cette terre\index{gnl}{terre} de 2,5 m\=a et \textit{araikk\=a\d ni muntirikai} inclut les rivages, les terres\index{gnl}{terre} \textit{ti\d tal} et les \'etangs, ainsi que la terre\index{gnl}{terre} r\'esidentielle qui est \`a l'ouest.\\
- [La terre\index{gnl}{terre} qui  se trouve] \`a l'ouest de la \textit{vati}\index{cec}{vati@\textit{vati}} Cuttamali\index{cec}{Cuttamali}, au nord du canal Ni\b n\b r\=a\b n, 3\up{e} carr\'e du premier canalicule. [Elle a été acquise aupr\`es] de P\=al\=a\'sr\=\i yan M\=ateva\b n\index{cec}{Matev@M\=ateva\b n} \dots\ ainsi que la terre\index{gnl}{terre} s\`eche de la terre\index{gnl}{terre} s\`eche (\textit{pu\b nceynilattu\index{cec}{nilam@\textit{nilam} terre} u\d taiya punceynilam\index{cec}{nilam@\textit{nilam} terre}}) au nord-ouest.

(18-22) Ayant pris en main cette terre\index{gnl}{terre} exempt\'ee et invendable tant que durent lune et soleil, nous avons dit que nous ne pr\'el\`everons pas les taxes \textit{mu\d t\d taiy\=a\d l}, \textit{ve\d t\d ti}, \textit{cilvari} et \textit{peruvari}\footnote{Sur une interpr\'etation de ces taxes qui portent en partie sur l'irrigation (\textit{mu\d t\d taiy\=a\d l} et \textit{ve\d t\d ti}); cf. \textsc{Heitzman} (*2001 [1997]: 162-163) et en particulier les notes ~27 et 28 p. 177, \textsc{Veluthat} (1993: 147) et \textsc{Karashima} (*2001c [1972]) pour les fr\'equences et occurrences de ces termes dans les inscriptions.}. Cette terre\index{gnl}{terre} prise en main et jouie, et la fermeture de \textit{Tirumu\b rai}\index{gnl}{Tirumurai@\textit{Tirumu\b rai}} retir\'ee, que l'on grave ainsi sur la pierre du palais et sur cuivre.

(22-38) [A été sign\'e] par le service\index{gnl}{service} le comptable du village \dots\ piriya\b n. Ceux de l'assemblée\index{gnl}{assemblée}\footnote{Il s'agit probablement de l'assemblée\index{gnl}{assemblée} de CEC 25 qui pr\'esente ses membres de la m\^eme fa\c con. De plus, un des noms de brahmane\index{gnl}{brahmane}, M\=a\d tila\b n Civateva\b n\index{cec}{Civateva\b n} Tirutt\=o\d nipuram\index{gnl}{Tonipuram@T\=o\d nipuram!Tirutt\=o\d nipuram}u\d taiy\=a\b n (l.~23), concorde avec CEC 25 l.~3.} qui ont pos\'e leur signature  sont: P\=arattuv\=aci Tevaka\b n Ce\b n\=apa Nampi\index{cec}{Nampi},
M\=a\d tila\b n Civateva\b n\index{cec}{Civateva\b n} Tirutt\=o\d nipuram\index{gnl}{Tonipuram@T\=o\d nipuram!Tirutt\=o\d nipuram}u\d taiy\=a\b n,
P\=arattuv\=aci Teva\b n Tillain\=aya-ka\b n\index{cec}{Tillain\=ayakar},
V\=ajyan Kalika\d tint\=a\b n \dots,
V\=ajyan Kum\=ara\b n\index{cec}{kumaran@Kum\=ara\b n} Ku..\b n\=a\b n,
Tiruve\d nk\=a\d tu\d taiy\=a\b n\index{cec}{Tiruvenkatu@Tiruve\d nk\=a\d tu\d taiy\=a\b n} alias P\=al\=a\'sr\=\i ya Tirucci\b r\b rampalamu\d taiy\=a\b n\index{cec}{tiruccirrampalamutaiyar@Tirucci\b r\b rampalamu\d taiy\=ar} officiant du palais,
Tirucci\b r\b rampalamu\d taiy\=a\b n\index{cec}{tiruccirrampalamutaiyar@Tirucci\b r\b rampalamu\d taiy\=ar},
\textit{Saij\~n\=ata\b namaikku}\footnote{L'expression \textit{saij\~n\=ata\b namaikku}, sauf erreur, n'est pas attest\'ee. Son sens reste obscur m\^eme si nous pouvons supposer que ce terme d\'erive du sk. \textit{sa\d mj\~n\=a} signifiant \og accord, entente\fg\ et qu'il souligne l'approbation de la transaction par les signataires. Mais dans ce cas pourquoi n'est-il pas syst\'ematique pour tous les membres\string?} Pa\d t\d tappiriya\b n V\=ajyan Arukka\b n \=A\d lu\d taiy\=a\b n,
P\=al\=a\'sr\=\i ya\b n Kuma-ra\b n N\=\i laka\d n\d ta\b n,
P\=al\=a\'sr\=\i ya\b n Pira\d layavi\d ta\.nka\b n\index{cec}{Piralaiya@Pira\d laiyavi\d ta\.nka\b n} Tirutt\=o\d nipuram\index{gnl}{Tonipuram@T\=o\d nipuram!Tirutt\=o\d nipuram}u\d taiy\=a\b n,
V\=ajya\b n Te-va\b n Tirutt\=o\d nipuram\index{gnl}{Tonipuram@T\=o\d nipuram!Tirutt\=o\d nipuram}u\d taiy\=a\b n,
\textit{Saij\~n\=ata\b namaikku} Pa\d t\d tappiriya\b n C\=ap\=anatice\b nta\b n Vi-rapattira\b n,
\textit{Saij\~n\=ata\b namaikku} P\=arattuv\=aci M\=ateva\b n\index{cec}{Matev@M\=ateva\b n} Tiruvagni\'svaramu\d taiy\=a\b n\index{cec}{Tiruvagni\'svaramu\d taiy\=a\b n},
\textit{Sai-j\~n\=ata\b namaikku} Pa\d t\d tappuriya\b n Pira\d layavi\d ta\.nka\b n\index{cec}{Piralaiya@Pira\d laiyavi\d ta\.nka\b n} Teva\b n,
M\=a\d tila\b n Civateva\b n\index{cec}{Civateva\b n} P\=alen-tiramavuli,
\=Astriya\b n C\=uriyateva\b nkolam\=aki\b ni\b n\b r\=a\b n Tirucci\b r\b rampalamu\d taiy\=a\b n\index{cec}{tiruccirrampalamutaiyar@Tirucci\b r\b rampalamu\d taiy\=ar},
[ainsi que] \textit{Saij\~n\=ata\b namaikku} \b N\=a\b n\=u\b r\b ruvappiriya\b n,
V\=acciya\b n\index{cec}{Vacciyan@V\=acciya\b n} Kumara\b n Pori\dots\ n\=ata\b n Ku-mara\b n,
P\=arat\=aya\b n Matic\=uta\b n Tiruve\d nk\=a\d tu\d taiy\=a\b n\index{cec}{Tiruvenkatu@Tiruve\d nk\=a\d tu\d taiy\=a\b n},
P\=arattuv\=aci M\=ateva\b n\index{cec}{Matev@M\=ateva\b n} Pira\d laya-vi\d ta\.nka\b n\index{cec}{Piralaiya@Pira\d laiyavi\d ta\.nka\b n},
M\=a\d tala\b n\index{cec}{matalan@M\=a\d tala\b n} Civateva\b n\index{cec}{Civateva\b n} Tiruva\b nanti\dots\ nampi\index{cec}{Nampi},
\textit{Saij\~n\=ata\b namaikku} A\d sa\dots\ N\=a\b r-patte\d n\d n\=ayirampi,
V\=acciya\b n\index{cec}{Vacciyan@V\=acciya\b n} Nakka\b n,
\textit{Saij\~n\=ata\b namaikku} Acuku\d taiy\=a\b n Tiruvaiy\=a-\b ru\d taiy\=a\b n K\=a\.n \dots\ v\=urav\=ap\=a\dots\ i\d sabhateva\b n,
Kavu\d niya\b n\index{cec}{kavuniyan@\textit{kavu\d niya\b n gotra}} Teva\b n Tirucci\b r\b rampalamu-\d taiy\=a\b n\index{cec}{tiruccirrampalamutaiyar@Tirucci\b r\b rampalamu\d taiy\=ar},
\dots\ Tirucci\b r\b rampalamu\d taiy\=a\b n\index{cec}{tiruccirrampalamutaiyar@Tirucci\b r\b rampalamu\d taiy\=ar}.


\section*{CEC 27}
\subsection*{CEC 27.1 Remarques}

L'\'epigraphe a été relev\'ee dans l'ARE 1918 374 et localis\'ee sur le mur nord du temple\index{gnl}{temple} de Campantar\index{gnl}{Campantar}. Elle date de la dixi\`eme ann\'ee de r\`egne de Tribhuvanacakravartin Kulottu\.ngac\=o\b ladeva\index{cec}{kulottungacoladeva@Kulottu\.ngac\=o\b ladeva}. Les donn\'ees astronomiques permettent \`a L'ARE, appendix E, de dater le texte du \textbf{mercredi 27 janvier 1143}, sous le r\`egne de Kulottu\.nga II\index{gnl}{Kulottu\.nga II}.

L'inscription se situe sur le soubassement de la face nord de la chapelle. Elle s'\'etend sur trois m\`etres deux et comporte quatorze lignes qui se r\'epartissent sur deux surfaces. Les six premi\`eres sont sur l'ultime composant du soubassement, sur cinq pierres align\'ees, et les huit derni\`eres sont sur des dalles pos\'ees au sol, dans l'espace, l\'eg\`erement creux, du canal d'\'evacuation. Cette partie, enduite de mortier, ne peut pas \^etre lue enti\`erement \textit{in situ}.

Le texte mentionne la donation d'une \og terre\index{gnl}{terre} de cuisine\fg\ pour offrir du riz\index{gnl}{riz} au lait\index{gnl}{lait} \`a Campantar\index{gnl}{Campantar} par la \textit{mulaparu\d sai} de Talaicca\.nk\=a\d tu\index{cec}{Talaiccankatu@Talaicca\.nk\=a\d tu} dans l'\=Akk\=urn\=atu du Jaya\.nko\d n\d taco\b lava\d lan\=a\d tu\index{cec}{Jayankontacolavalanatu@Jaya\.nko\d n\d tac\=o\b lava\d lan\=a\d tu}.

\subsection*{CEC 27.2 Texte}
\begin{enumerate}
	\item \textbf{svasti \'sr\=\i\ tribhuvanaccakrava}rttika\d l \textbf{\'sr\=\i kulo}ttu\.nka\textbf{co\d la}tevarkku\index{cec}{kulottungacoladeva@Kulottu\.ngac\=o\b ladeva} y\=a\d n\d tu pa-tt\=avatu m\=acitti\.nka\d l munn\=a\d lpakkam o\b npatu ki\b lamai puta\b n n\=a\d l \textbf{m\textsubring{r}ga\'s\=\i r\d sa}m i\b nn\=a\d l mi\textbf{thu}nam\=aka na\b n pakal \=ava\d nattu\index{cec}{avanam@\textit{\=ava\d nam} document}
	\item ke\d t\d tukka\b lanta\footnote{La ligature \textit{nta} est constitu\'ee de la superposition des graph\`emes \textit{na} et \textit{ta}.} poru\d l nila\index{cec}{nilam@\textit{nilam} terre}vilaiy\=ava\d nam\index{cec}{vilai@\textit{vilai} prix!vilaiyavanam@\textit{vilaiy\=ava\d nam} document de vente} \textbf{ja}ya\.nko\d n\d taco\b lava\d lan\=a\d t\d tu\index{cec}{Jayankontacolavalanatu@Jaya\.nko\d n\d tac\=o\b lava\d lan\=a\d tu} \=akk\=urn\=a\d t-\d tut talaicca\.nk\=a\d t\d tu\index{cec}{Talaiccankatu@Talaicca\.nk\=a\d tu} mula\textbf{paru\d sai}yom ivv\=ur mummu\d tico\b la\b n perampalatte k\=u\d t-\d ta\.nku\b raiva\b rak k\=u\d ti irun
	\item tu \textbf{r\=aj\=adhir\=aja}va\d lan\=a\d t\d tut\index{cec}{Rajadhirajavala@R\=aj\=adhir\=ajava\d lan\=a\d tu} tirukka\b lumalan\=a\d t\d tu\index{cec}{Tirukka\b lumalan\=a\d tu} \textbf{brahmadeya}m\index{cec}{brahmadeya@\textit{brahmadeya}} tiruka\b lumalat-tu\index{cec}{Tirukkalumalam@Tirukka\b lumalam} \=a\d lu\d taiyapi\d l\d laiy\=arkkut tirupp\=a\b rponakam amutuceytaru\d lukaikku tiruma\d taip-pa\d l\d lippu\b ram\=aka\index{cec}{puram@\textit{pu\b ram} terre de donation!\textit{ma\d tappa\d l\d lippu\b ram} terre donnée pour la cuisine} n\=a\.nka\d l vi\b r\b ruk ku\d tutta nilam\=ava\index{cec}{nilam@\textit{nilam} terre}
	\item {[tu].va.\b n\=urp pi\d t\=akai\index{cec}{pitakai@\textit{pi\d t\=akai} hameau} co\b lap\=a\d n\d tiyanall\=uril antarivatikku\index{cec}{vati@\textit{vati}} ki\b lakku \textbf{r\=ajendra}co\b la-v\=aykk\=alukku\index{cec}{vaykkal@\textit{v\=aykk\=al} canal} va\d takku mu\b n\b r\=a\.nka\d n\d n\=a\b r\b ru ira\d n\d t\=a\~ncatirattu kavuciya\b n \=\i c\=a\b na[ka]-laiyan u\d l\d li\d t\d t\=ar per\=al ka\d n\d n\=a\b ru en\b ru per k\=u}
	\item vappa\d t\d ta nilattukkum\index{cec}{nilam@\textit{nilam} terre} pu\b nceykkum ku\d lattukkum ti\d talukkum ellaiy\=avatu mel-p\=arkkellai n\=al\=a\~ncatirattukku ki\b lakkum va\d tap\=arkkellai \=atittateva\b nv\=aykk\=aluk\index{cec}{atittat@\=Atittateva\b n}\index{cec}{vaykkal@\textit{v\=aykk\=al} canal}-kut te\b rkum ki\b lp\=arkkellai muta\b rcatirattukku me
	\item {[\b rkum] te\b n [p\=a]rkkellaik ka\d n\d n\=a\b r\b rukku va\d takkum ivvicaitta perun\=a\b nkellaiyu\d l akappa\d t\d ta vi\d laini[la]m [ku\d lamu]m ti\d talum ku\b li patin\=ayirattirun\=u\b r\b raimpatu itil te\b nki\b lakka\d taiyakk\=amakk\=a\d nit\footnote{La moiti\'e inf\'erieure de cette ligne est recouverte de ciment.}}
	\item {[tirucci\b r\b rampalamu\d taiy\=a\b n\index{cec}{tiruccirrampalamutaiyar@Tirucci\b r\b rampalamu\d taiy\=ar} para\d na u\d l\d li\d t\d t\=ar per\=al ki\d tanta vi\d lai nilamum\index{cec}{nilam@\textit{nilam} terre} [ku\d la-mum*] ti\d talum ku\b li irun\=u\b r\b raimpatu ikku\b li irun\=u\b r\b raimpatum nikki n\=a\.nka\d l vi\b r\b rukku\d tutta vi\d lainilamum\index{cec}{nilam@\textit{nilam} terre} [ku\d lamum*] ti\d talum \dots}
	\item ikku\b li pati\b n\=ayiratti\b nul nilam\index{cec}{nilam@\textit{nilam} terre} ayveli innilam\index{cec}{nilam@\textit{nilam} terre} ayveliyum vi\b r\b ruk ku\d tuttuk ko\d lva-t\=aka emmillicainta vi\d laipporu\d l a\b n\b r\=a\d tu na\b n k\=acu\index{cec}{kacu@\textit{k\=acu} pièces de monnaie} \=ayiram ikk\=acu \=ayiramum innilam\index{cec}{nilam@\textit{nilam} terre} canti\textbf{r\=a}tittavarai k\=acu ko\d l\d l\=a i\b raiyili]\index{cec}{iraiyili@\textit{i\b raiyili} non imposable}
	\item {[y\=aka i\b raiyi\b li]ccik ku\d tuttuk ko\d n\d ta k\=acu\index{cec}{kacu@\textit{k\=acu} pièces de monnaie} \=ayiram [ikk\=acu \=ayiramum innilattukku\index{cec}{nilam@\textit{nilam} terre} cennir ve\d t\d ti\index{cec}{cennirvetti@\textit{ce\b n\b n\=\i rve\d t\d ti} taxe sur l'irrigation} mu\d t\d taiy\=a\d lum ku\d timaika\d lukku canti\textbf{r\=a}tittavarai tavi[r*]ntu ku\d tuttuk ko\d n\d ta k\=acu n\=u\b ru ikk\=acu n\=u\b rum \=akak k\=a]}
	\item cu ira\d n\d t\=ayarattorun\=u\b ru ikk\=acu\index{cec}{kacu@\textit{k\=acu} pièces de monnaie} ira\d n\d t\=ayirattu [oru \b n\=u\b rum kaikko\d n\d tu innilam\index{cec}{nilam@\textit{nilam} terre} \textbf{sa}pai vilaiy\=aka\index{cec}{vilai@\textit{vilai} prix!sabhaivilai@\textit{sabhaivilai} vente aux enchères} vi\b r\b ru k\=acu\index{cec}{kacu@\textit{k\=acu} pièces de monnaie} ko\d l\d l\=a i\b raiyyili ceytu ce\b n\b n\=\i r ve\d t\d ti\index{cec}{cennirvetti@\textit{ce\b n\b n\=\i rve\d t\d ti} taxe sur l'irrigation} mu\d t\d taiy\=a\d l ku\d timai-yum ta[vi*]rntu innilam\index{cec}{nilam@\textit{nilam} terre} ayve]
	\item liyum cempilum kallilum ve\d t\d tik ko\d l\d lak ka\d tav\=aka\d l\=aka vi\b r\b ruk ku\d tuttom piramatecam\index{cec}{brahmadeya@\textit{brahmadeya}} tirukka\b lumalattu\index{cec}{Tirukkalumalam@Tirukka\b lumalam} \=a\d lu\d taiyapi\d l\d laiy\=arku \textbf{sabhai}vi\d lai[y\=aka\index{cec}{sabha@\textit{sabh\=a} assemblée} vi\b r\b rukku\d tuttom mulaparu\textbf{\d sai}yom \dots]
	\item i\d taiy\=ana muv\=ayirava\b n e\b luttu mulaparu\textbf{\d sai}y\=alil e\b lutti\d t\d t\=ar tirucci\b r\b rampalamu-\d taiy\=an\index{cec}{tiruccirrampalamutaiyar@Tirucci\b r\b rampalamu\d taiy\=ar} ka\d n\d nan t\=a\b li [u\b maicc\=a\d na\.nkiran k\=uttan c\=att\=a\d ni kum\=aranniya\b n\index{cec}{kumaran@Kum\=ara\b n} kavuciya\b n kalaiya\b n \dots ntaminiya\b n tirucci\b r\b rampalamu\d taiya\index{cec}{tiruccirrampalamutaiyar@Tirucci\b r\b rampalamu\d taiy\=ar} \dots]
	\item {[\dots\ m\=a\d tila\b n\=ata\b n va\d tuka\b n u\b maicc\=a\d na\b n nilaka\d n\d ta\b n \dots tta\b n kavuciya\b n nila\b n cuppirama\d n\d niya\b n\index{cec}{Cuppirama\d nyam} kavuciya\b n nila\b n m\=a\d niya\b n koma\d ta\b n c\=av\=anc\=atta\b n p\=al\=aci \dots\ t\=aya\b n cantiracekara\b n kum\=ara\b n\index{cec}{kumaran@Kum\=ara\b n} c\=atta\b n m\=ala\b n kum\=ara\b n\index{cec}{kumaran@Kum\=ara\b n}}
 	\item \dots mpa\d l cantiracekara\b n\=a\b na ka\b nakavarat\=a\b n citurvetika\dots ya\b n c\=att\=a\d nicaturvati makkumaru k\=a\d ta\b n c\=att\=a\d ni[ca*]turveti teva\b n ka\d n\d napa\d t\d ta\b n to\b liya\b n nila\b n\=a\b na \dots]
\end{enumerate}

\subsection*{CEC 27.3 Traduction}
(1-3) Que la prosp\'erit\'e soit! En la dixi\`eme ann\'ee [de r\`egne] de \'Sr\=\i kulottu\.ngac\=o\b la-deva\index{cec}{kulottungacoladeva@Kulottu\.ngac\=o\b ladeva}, empereur des trois mondes, le mois lunaire de \textit{M\=aci}, le neuvi\`eme jour de la quinzaine claire, jeudi, dans le [\textit{nak\d satra}] \textit{M\textsubring{r}ga\'s\=\i r\d sam}, \textit{i\b nn\=a\d l mithunam\=aka}\footnote{Mot \`a mot \og en ce jour en tant que \textit{mithuna}\fg.}, de jour; nous les [membres] la \textit{m\=ulaparu\d sai}\footnote{Ce terme, d\'eriv\'e du sk. \textit{pari\d sad}, d\'esignerait une des organisations de villages \textit{brahmadeya}\index{cec}{brahmadeya@\textit{brahmadeya}} souvent charg\'ee des affaires administratives du temple\index{gnl}{temple} et, dont les membres brahmane\index{gnl}{brahmane}s sont choisis parmi la \textit{sabh\=a}; cf. \textsc{Veluthat} (1985: 76) et \textsc{Subbarayalu} (*2001h [1982]: 91).} de Talaicca\.nk\=a\d tu\index{cec}{Talaiccankatu@Talaicca\.nk\=a\d tu} dans l'\=Akk\=urn\=a\d tu\footnote{Cette division territoriale se trouve \`a une vingtaine de kilomètres au sud de Tirukka\b lumalan\=a\d tu\index{cec}{Tirukka\b lumalan\=a\d tu}; cf. \textsc{Subbarayalu} (1973, carte 10).} du Jaya\.nko\d n\d taco\b lava\d lan\=a\d tu\index{cec}{Jayankontacolavalanatu@Jaya\.nko\d n\d tac\=o\b lava\d lan\=a\d tu}, réunis avec le \textit{quorum} (littéralement \og réunis sans manque\fg) et assis dans la grande salle Mummu\d tico\b la\b n de ce village\footnote{L'action se d\'eroule vraisemblablement \`a Talaicca\.nk\=a\d tu\index{cec}{Talaiccankatu@Talaicca\.nk\=a\d tu} car au moins deux inscriptions (ARE 1925 187 et 181) du temple\index{gnl}{temple} de Tiruna\b npa\d l\d li\index{gnl}{Nanipalli@Na\b nipa\d l\d li!Tiruna\b npa\d l\d li} (Pu\~ncai), situ\'e \`a Talaicca\.nk\=a\d tu\index{cec}{Talaiccankatu@Talaicca\.nk\=a\d tu}, mentionnent que les membres de cette \textit{m\=ulaparu\d sai} se r\'eunissent en assemblée\index{gnl}{assemblée} pl\'eni\`ere dans cette salle (ces textes, non publi\'es, ont été g\'en\'ereusement communiqu\'es par Charlotte \textsc{Schmid}).}, apr\`es \'ecoute et consultation du document, [nous pr\'esentons] le document du prix de la vente d'une terre\index{gnl}{terre}.
[Voici] la terre\index{gnl}{terre} que nous avons vendue comme terre\index{gnl}{terre} de cuisine pour nourrir avec du riz\index{gnl}{riz} au lait\index{gnl}{lait} \=A\d lu\d taiyapi\d l\d laiy\=ar\index{cec}{Alutaiyapillaiyar@\=A\d lu\d taiyapi\d l\d laiy\=ar} de Tirukka\b lumalam\index{cec}{Tirukkalumalam@Tirukka\b lumalam}, \textit{brahmadeya}\index{cec}{brahmadeya@\textit{brahmadeya}} du Tirukka\b lumalan\=a\d tu\index{cec}{Tirukka\b lumalan\=a\d tu} dans le R\=aj\=adhir\=ajava\d lan\=a\d tu\index{cec}{Rajadhirajavala@R\=aj\=adhir\=ajava\d lan\=a\d tu}:

(4-7) / Dans Co\b lap\=a\d n\d tiyanall\=ur, hameau de \dots, [la terre\index{gnl}{terre}] du deuxi\`eme carr\'e du troisi\`eme canalicule, \`a l'est de la \textit{vati}\index{cec}{vati@\textit{vati}} Antari, au nord du canal R\=ajendraco\b la, [est la terre\index{gnl}{terre} vendue qui est] appel\'ee \textit{ka\d n\d n\=a\b ru} par Kavuciya\b n \=Ic\=a\b nakalaiyan et d'autres. Les limites pour la terre\index{gnl}{terre} [irrigu\'ee?]  \textit{nilam}\index{cec}{nilam@\textit{nilam} terre}, la terre\index{gnl}{terre} s\`eche, le point d'eau\index{gnl}{eau} et pour la terre\index{gnl}{terre} \textit{ti\d tal} sont [les suivantes]: la limite ouest est l'est du quatri\`eme carr\'e, la limite nord est le sud du canal \=Atittateva\b n\index{cec}{atittat@\=Atittateva\b n}, la limite est est l'ouest du premier carr\'e et la limite sud est le nord du canalicule. La terre\index{gnl}{terre}, le point d'eau\index{gnl}{eau} et la terre\index{gnl}{terre} \textit{ti\d tal}, inclus \`a l'int\'erieur de ces quatre grandes limites, [font] dix mille deux cent cinquante \textit{ku\b li}.
De ceci, [il faut] d\'eduire au sud-est, la terre\index{gnl}{terre}, [le point d'eau\index{gnl}{eau}*] et la terre\index{gnl}{terre} \textit{ti\d tal} au nom de Tirucci\b r\b rampalamu\d taiy\=a\b n\index{cec}{tiruccirrampalamutaiyar@Tirucci\b r\b rampalamu\d taiy\=ar} Para\d na et autres, de deux cent cinquante \textit{ku\b li}.

(7-10) Ayant d\'eduit ces deux cent cinquante \textit{ku\b li}, la terre\index{gnl}{terre}, le point d'eau\index{gnl}{eau} et la terre\index{gnl}{terre} \textit{ti\d tal} que nous vendons [fait] dix mille \textit{ku\b li} soit cinq \textit{v\=eli}\footnote{Cette \'equivalence confirme la d\'efinition du \textit{TL} que 100 \textit{ku\b li} = 1 \textit{m\=a} = 1/20 \textit{v\=eli} soit 1 \textit{v\=eli} = 2000 \textit{ku\b li}.}.
Le prix convenu pour vendre cette terre\index{gnl}{terre} de 5 \textit{v\=eli} est de mille \textit{k\=acu}\index{cec}{kacu@\textit{k\=acu} pièces de monnaie} \`a cours l\'egal,
 [plus] les mille \textit{k\=acu}\index{cec}{kacu@\textit{k\=acu} pièces de monnaie} pour que cette terre\index{gnl}{terre} soit faite non imposable et invendable tant que durent lune et soleil,
[plus] les cent \textit{k\=acu} donn\'es pour exclure [les taxes] \textit{cennir ve\d t\d ti}\index{cec}{cennirvetti@\textit{ce\b n\b n\=\i rve\d t\d ti} taxe sur l'irrigation}, \textit{mu\d t\d taiy\=a\d l} et \textit{ku\d timai} tant que durent lune et soleil, soit [au final] une somme de deux mille cent  \textit{k\=acu}\index{cec}{kacu@\textit{k\=acu} pièces de monnaie}\footnote{Cette transaction montre que l'achat d\'efinitif des exemptions peut s'effectuer au niveau local\index{gnl}{local} et est contr\^ol\'e par l'assemblée\index{gnl}{assemblée} villageoise. Les taxes touchent l'irrigation et l'habitation.}.

(10-11) Nous avons pris en main cette somme de deux mille cent et vendu cette terre\index{gnl}{terre} au prix [\'etabli] par l'assemblée\index{gnl}{assemblée}. [Puis], nous l'avons faite non imposable et invendable et avons exclu les taxes \textit{cennir ve\d t\d ti}, \textit{mu\d t\d taiy\=a\d l} et \textit{ku\d timai}. [Enfin], nous avons vendu cette terre\index{gnl}{terre} de 5 \textit{v\=eli} en gravant sur pierre et cuivre. Nous les [membres] de la \textit{m\=ulaparu\d sai} avons vendu au prix [\'etabli] par l'assemblée\index{gnl}{assemblée} \`a \=A\d lu\d taipi\d l\d laiy\=ar du \textit{brahmadeya}\index{cec}{brahmadeya@\textit{brahmadeya}} Tirukka\b lumalam\index{cec}{Tirukkalumalam@Tirukka\b lumalam}.

(11-14) \dots\ signature de \dots\ i\d taiy\=ana Muv\=ayirava\b n. Ont sign\'e parmi les [membres] de la \textit{mulaparu\d sai}:
Tirucci\b r\b rampalamu\d taiy\=an\index{cec}{tiruccirrampalamutaiyar@Tirucci\b r\b rampalamu\d taiy\=ar} Ka\d n\d nan T\=a\b li,
U\b laicc\=a\d na\.n Kiran K\=uttan\footnote{Ce membre de la \textit{m\=ulaparu\d sai} de Talaicca\.nk\=a\d tu\index{cec}{Talaiccankatu@Talaicca\.nk\=a\d tu} appara\^it dans une transaction du temple\index{gnl}{temple} de Tiruna\b nipa\d l\d li\index{gnl}{Nanipalli@Na\b nipa\d l\d li!Tiruna\b nipa\d l\d li} (ARE 1925 181 l.~14) datant de 1138, sous le r\`egne de Vikramac\=o\b la\index{gnl}{Vikramacola@Vikramac\=o\b la}.},
C\=att\=a\d ni Kum\=aranniya,
Kavuciya\b n Kalaiya\b n \dots ntaminiya\b n, Tirucci\b r\b ram-palamu\d taiya\dots\
M\=a\d tila \b N\=ata\b n Va\d tuka\b n,
U\b laicc\=a\d na\b n Nilaka\d n\d ta\b n \dots tta\b n,
Kavuciya\b n Nila\b n Cuppirama\d n\d niya\b n\index{cec}{Cuppirama\d nyam},
Kavuciya\b n Nila\b n M\=a\d niya\b n,
Koma\d ta\b n C\=av\=an C\=atta\b n,
P\=a-l\=aci \dots\ t\=aya\b n Cantiracekara\b n Kum\=ara\b n\index{cec}{kumaran@Kum\=ara\b n},
C\=atta\b n M\=ala\b n Kum\=ara\b n\index{cec}{kumaran@Kum\=ara\b n} \dots mpa\d l, Cantira-cekara\b n\=a\b nan Ka\b nakavarat\=a\b n Citurvetika\dots ya\b n,
C\=att\=a\d nicaturvati Makkumaru K\=a-\d ta\b n,
C\=att\=a\d ni[ca*]turveti Teva\b n Ka\d n\d napa\d t\d ta\b n,
To\b liya\b n Nila\b n\=a\b na \dots
%\footnote{gotra\index{gnl}{gotra@\textit{gotra}}? U\b laicc\=a\d na\.n (8 ref ds Concordance + 2 \`a Citamp SII 12 149 l.~6 et 171 l.~9) et pb Koma\d ta\b n/koma\d ti, C\=att\=a\d nicaturvati que Charlotte lit c\=atakar\d ni dans ARE 1925 181 par ex.XXXXX}

\section*{CEC 28}
\subsection*{CEC 28.1 Remarques}

L'\'epigraphe se trouve sur le mur nord de la chapelle de Campantar\index{gnl}{Campantar} et a été relev\'ee dans ARE 1918 378 qui la date de la dix-septi\`eme ann\'ee de r\`egne de Tribhuvanacakravartin Kulottu\.ngac\=o\b ladeva\index{cec}{kulottungacoladeva@Kulottu\.ngac\=o\b ladeva}. \textsc{Mahalingam} (1992, 551, Tj. 2416) pose l'hypothèse de la date de 1195, sous Kulottu\.nga III\index{gnl}{Kulottu\.nga III}.

L'inscription est grav\'ee sur la face nord, \`a l'ouest de CEC 29, sur la troisi\`eme portion du mur en partant de l'est. Encadr\'ees par deux pilastres, ses trente lignes couvrent les trois-quarts du mur.

L'examen de la pierre montre que l'ann\'ee de r\`egne n'est pas dix-sept mais dix. Aussi, l.~20, il est clairement \'ecrit en toutes lettres que l'ann\'ee en cours est la dixi\`eme (\textit{ivv\=a\d n\d tu patt\=avatu}). De plus, CEC 28 ressemble fortement \`a CEC 29 au niveau de la structure: le texte, sans invocation\index{gnl}{invocation}, enregistre un \'ecrit d'une assemblée\index{gnl}{assemblée} (\textit{perumakka\d l e\b luttu}) qui a \'et\'e vu (\textit{ka\d n\d tu}) par les employ\'es de la chapelle de Campantar\index{gnl}{Campantar}. R\=aj\=adhir\=ajava\d lan\=a\d tu\index{cec}{Rajadhirajavala@R\=aj\=adhir\=ajava\d lan\=a\d tu} y est pr\'esent\'e comme un \textit{devad\=ana} d'U\d taiy\=ar Tirucci\b r\b rampalamu\d taiy\=ar\index{cec}{tiruccirrampalamutaiyar@Tirucci\b r\b rampalamu\d taiy\=ar}. Viennent ensuite la transaction, la r\'ecapitulation et la liste des signataires. D'un point de vue pal\'eographique, ces deux textes ont des graph\`emes de taille et de style identiques avec un m\^eme interligne. Nous supposons ainsi que ces inscriptions jumelles sont contemporaines. CEC 29 datant du r\`egne de Kulottu\.nga II\index{gnl}{Kulottu\.nga II} (voir les remarques pour CEC 29), il est probable que CEC 28 date de sa dixi\`eme ann\'ee de r\`egne, soit de \textbf{1143}.

L'inscription enregistre une donation \`a Campantar\index{gnl}{Campantar} par l'assemblée\index{gnl}{assemblée} de Kulottu\.nkaco\b laccaruppetima\.nkalam\index{cec}{caturvedimangalam@Caturvedima\.ngalam!Kulottu\.ngac\=o\b laccaturvedima\.ngalam}: une terre\index{gnl}{terre} pour diverses offrandes et un jardin \`a fleurs.

\subsection*{CEC 28.2 Texte}
\begin{enumerate}
	\item tiripuva\b naccakkaravattika\d l\index{cec}{Tribhuvanacakravarti} \textbf{\'sr\=\i}kulottu\.nkaco\b la\textbf{de}
	\item {[va\b r]ku\index{cec}{kulottungacoladeva@Kulottu\.ngac\=o\b ladeva} 10\=A\footnote{La boucle de l'abr\'eviation pour ann\'ee, tr\`es fleurie ici, vient compl\`etement encercler l'\textit{ak\d sara} \textit{ya} qui a la valeur de 10. Ce que l'ARE a cru \^etre un 7 appartient en fait \`a l'abr\'eviation \=A.} u\d taiy\=ar tirucci\b r\b ram[pa]lamu\d taiy\=ar\index{cec}{Tirucirrampalamu@Tirucci\b ra\b rampalamu\d taiy\=ar} te}
	\item {[vat\=a\b nam r\=aj\=adhir\=ajava\d lan\=a\d t\d tu\index{cec}{Rajadhirajavala@R\=aj\=adhir\=ajava\d lan\=a\d tu}]\footnote{Cette conjecture personnelle est fond\'ee sur CEC 29 l.~3} \dots}
	\item y\=ur\b n\=a\d t\d tu (k)olottu\.nkaco\b laccaru[ppe]tima\.nkalat\index{cec}{caturvedimangalam@Caturvedima\.ngalam!Kulottu\.ngac\=o\b laccaturvedima\.ngalam}
	\item {[tu]\index{cec}{caturvedimangalam@Caturvedima\.ngalam!Kulottu\.ngac\=o\b laccaturvedima\.ngalam} peru\.nku[\b ri] ma\textbf{h\=asabhai}p\index{cec}{sabha@\textit{sabh\=a} assemblée} peruma[k]ka\d l e\b luttu i}
	\item {[\b n]\b n\=a\d t\d tut [tiru]kka\b lumala\b n\=a\d t\d tut\index{cec}{Tirukka\b lumalan\=a\d tu} tirukka\b lumalattu\index{cec}{Tirukkalumalam@Tirukka\b lumalam} \=a}
	\item {[\d lu\d taiyapi\d l\d laiy\=ar koyilil\index{cec}{koyil@\textit{k\=oyil} temple} \textbf{\'sr\=\i}m\=a\textbf{he\'sva}rakka\d nk\=a]\index{cec}{srimahesvara@\textit{\'sr\=\i mahe\'svara} dévot, surveillant}\footnote{Conjecture personnelle \'etablie \`a partir de CEC 29 l.~6-7.}}
	\item \d ni caiv\=a[ru]ka\d lukkum\footnote{La lecture conjectur\'ee propos\'ee par la transcription \textit{caiv\=a[c\=ari]ka\d lukkum} est s\'eduisante mais elle n'est pas possible car un seul \textit{ak\d sara} manque sur la pierre.} tevaka\b nmiyum ka[\d na]kka\b nu\.nka\d n\d tu nam
	\item mur pi\d t\=akai\index{cec}{pitakai@\textit{pi\d t\=akai} hameau} tiru[v\=ur] akkaraiya\b nall\=ur vira[\b n\=a]r\=aya\d navatikkuk\index{cec}{vati@\textit{vati}} ki\b lakku
	\item mulaparu\textbf{\d sa}va[tikku va\d takku] 2 C 2ccatirattup pa\d l\d la[v\=ay N AAA i\.nke ku\d la-mu\.n]
	\item ti\d talum N 6m 3 C 1 catirattukellai [N] Am i\.nke pa\d l\d lav\=ay [A] A 2 ca
	\item tirattu a[ma]\b naka A nikki N A 6m 4 [C] 1 catirattu te\b rka\d taiya va[ramu\b rai-pattum]
	\item Am 2 catirat[tu pa]\d l\d lav\=ay N Am \=aka(p pa)\d l\d lav\=ay N 10 1 Am\=a apa.\=ala.
	\item .\b lum v\=ay[ma]\b raip\=a\b lum N A 4 [\=aka] N 10 1 A i\b n\=amavirivu..rula(m\=a)
	\item \dots\ po\b rp\=akanall\=ur matur\=antakavatikku\index{cec}{vati@\textit{vati}}\dots
	\item Akku ma\d takku 6 C \dots\ i\.nke ko
	\item llai N Am \=aka N AAAA N 1V \=a[ka] N 6 V innilam\index{cec}{nilam@\textit{nilam} terre} a\b ruveli
	\item yum mu\b npu cu\.nkantavirttaru\d li\b na kolottu\.n[ka]co\b la\textbf{de}va\b rkku patine\d t\d t\=ava
	\item tumutal pi\d l\d laiy\=ar i\b raiyiliy\=ay\index{cec}{iraiyili@\textit{i\b raiyili} non imposable} pi\b npu [n\=a]\.nka\d l kaikko\d n\d tu a\b nu
	\item paviy\=a[t]e ki\d tanta nilam\index{cec}{nilam@\textit{nilam} terre} ivv\=a\d n\d tu patt\=avatu[va]raiyum payi\b rceyy\=ate p\=a\b l\=a
	\item ......yil innilam\index{cec}{nilam@\textit{nilam} terre} ivv\=a\d n\d tu mutal [i\b raiyili]\index{cec}{iraiyili@\textit{i\b raiyili} non imposable} \dots
	\item \d n\d tu payi\b r caiyyal\=ay vi\d lainilam\index{cec}{nilam@\textit{nilam} terre} [pa]yi\b rcaiytum kollaittirunantavana\~n
	\item ceytum \=a\d lu\d taiyapi\d l\d laiy\=ar tirum\=a\d likaikku tiruppa\d tim\=a\b r\b ruc celuttukaik
	\item ku k\=acu\index{cec}{kacu@\textit{k\=acu} pièces de monnaie} ko\d l\d l\=a i\b raiyiliy\=aka\index{cec}{iraiyili@\textit{i\b raiyili} non imposable} vi\d t\d tom [i]nnila\.nk\=acu\index{cec}{nilam@\textit{nilam} terre}\index{cec}{kacu@\textit{k\=acu} pièces de monnaie} ko\d l\d l\=a i\b raiyiliy\=aka\index{cec}{iraiyili@\textit{i\b raiyili} non imposable}
	\item vi\d t\d tamaikku cempilu\.nkallilum [v]e\d t\d tik ko\d n\d tu tirumeni kalliy\=a\d na ti
	\item rumeniy\=akat tiruppa\d tim\=a\b r\b ruc celu[t]tap pa\d n\d nuka pa\d niy\=al \=urka\d nak
	\item {[ku ala\.nk\=arappiriya\b n e\b luttu \textbf{sabhai}y\=aral\index{cec}{sabha@\textit{sabh\=a} assemblée} e\b luttu ...teyva\b n\=ayakapa]}
	\item \d t\d ta\b n tiruve\d n[k\=a]\d tupa\d t\d ta\b n tirurucci\b r\b rampalanam[pi\index{cec}{Tirucirrampalanampi@Tirucci\b ra\b rampalanampi} k]\=akka\d n\d tur tirucci\b r\b ram
	\item palanampi ca\.nkarapa[\d t\d ta]\b n v\=acciya\b n\index{cec}{Vacciyan@V\=acciya\b n} tirucci\b r\b rampalamu[\d tai]y\=a\b n\index{cec}{tiruccirrampalamutaiyar@Tirucci\b r\b rampalamu\d taiy\=ar} i\d taiy\=a\b r\b ru
	\item kka\d tan\=ar\=aya\d napa[\d t\d ta]\b n\tdanda |
\end{enumerate}

\subsection*{CEC 28.3 R\'esum\'e}
(1-8) Le texte date de la 10\up{e} ann\'ee de r\`egne de \'Sr\=\i kulottu\.ngac\=o\b ladeva\index{cec}{kulottungacoladeva@Kulottu\.ngac\=o\b ladeva}, empereur des trois mondes. Il enregistre un acte des membres de la grande assemblée\index{gnl}{assemblée} de Kulottu\.nkaco\b laccaruppetima\.nkalam dans \dots\ du R\=aj\=adhir\=ajava\d lan\=a\d tu\index{cec}{Rajadhirajavala@R\=aj\=adhir\=ajava\d lan\=a\d tu} qui est un \textit{devad\=ana} du Seigneur propri\'etaire de Tirucci\b r\b rampalam, [acte] qui a \'et\'e vu par les surveillants \textit{\'sr\=\i mahe\'svara}\index{cec}{srimahesvara@\textit{\'sr\=\i mahe\'svara} dévot, surveillant}, le \textit{devakarm\=\i} et le comptable du temple\index{gnl}{temple} d'\=A\d lu\d taiyapi\d l\d laiy\=ar\index{cec}{Alutaiyapillaiyar@\=A\d lu\d taiyapi\d l\d laiy\=ar} \`a Tirukka\b lumalam\index{cec}{Tirukkalumalam@Tirukka\b lumalam} dans le Tirukka\b lumalan\=a\d tu\index{cec}{Tirukka\b lumalan\=a\d tu} dans ce pays.

(8-21) Une terre\index{gnl}{terre} de six \textit{v\=eli} au total est offerte. Cette terre\index{gnl}{terre}, acquise par l'assemblée\index{gnl}{assemblée} la 18\up{e} ann\'ee de r\`egne de Kulottu\.ngac\=o\b la\index{cec}{kulottungacola@Kulottu\.ngac\=o\b la} qui a an\'eanti les douanes (Kulottu\.nga I\index{gnl}{Kulottu\.nga I}), est rest\'ee sans culture jusqu'\`a la dixi\`eme ann\'ee en cours.

(24-26) L'assemblée\index{gnl}{assemblée} donne cette terre\index{gnl}{terre} (qui doit \^etre cultiv\'ee et dont le verger doit former un jardin \`a fleurs) comme non imposable et invendable pour les diverses offrandes (\textit{tirupa\d tim\=a\b r\b ru}) destin\'ees au palais d'\=A\d lu\d taiyapi\d l\d laiy\=ar\index{cec}{Alutaiyapillaiyar@\=A\d lu\d taiyapi\d l\d laiy\=ar}. Elle ordonne que ce don\index{gnl}{don} exempt\'e et invendable soit grav\'e sur cuivre et pierre et que soient faites les offrandes pour \textit{tirumeni kalliy\=a\d na tirumeniy\=aka}\footnote{Litt\'eralement \og pour que le corps sacr\'e devienne le corps sacr\'e de mariage\index{gnl}{mariage}\fg. Selon G. \textsc{Vijayavenugopal}, il s'agit d'une formule indiquant que le but de la donation est la guérison\index{gnl}{guerison@guérison} du roi\index{gnl}{roi}.}.

(26-30) Les signataires sont le comptable du village Ala\.nk\=arappiriya\b n et des membres de l'assemblée\index{gnl}{assemblée} : \dots\ Teyvan\=ayakapa\d t\d ta\b n, Tiruve\d nk\=a\d tupa\d t\d ta\b n\index{cec}{Tiruvenkatup@Tiruve\d nk\=a\d tupa\d t\d ta\b n}, Tirurucci\b r-\b rampala Nampi\index{cec}{Nampi}, K\=akka\d n\d tur Tirucci\b r\b rampala Nampi\index{cec}{Nampi}, Ca\.nkarapa\d t\d ta\b n et V\=acciya\b n\index{cec}{Vacciyan@V\=acciya\b n} Ti-rucci\b r\b rampalamu\d taiy\=a\b n\index{cec}{tiruccirrampalamutaiyar@Tirucci\b r\b rampalamu\d taiy\=ar} I\d taiy\=a\b r\b rukka\d tan\=ar\=aya\d napa\d t\d ta\b n\footnote{Il est possible de sp\'eculer sur l'identit\'e\index{gnl}{identit\'e} de cette assemblée\index{gnl}{assemblée} qui n'est pas celle de CEC 29 (aucun membre ne concorde). Les informations certaines permettent de dire qu'elle se trouve \`a Kulottu\.nkaco\b laccaruppetima\.nkalam\index{cec}{caturvedimangalam@Caturvedima\.ngalam!Kulottu\.ngac\=o\b laccaturvedima\.ngalam} dans un \textit{n\=a\d tu} dont la terminaison est en -\textit{y\=ur}, et ce tr\`es probablement dans le R\=aj\=adhir\=ajava\d lan\=a\d tu\index{cec}{Rajadhirajavala@R\=aj\=adhir\=ajava\d lan\=a\d tu}. Or, il existe une assemblée\index{gnl}{assemblée} \`a Pa\~ncava\b nam\=atevi\index{cec}{panca@Pa\~ncava\b nam\=atevi}  alias Kulottu\.nkaco\b laccaruppetima\.nkalam dans le Ve\d n\d naiy\=urn\=a\d tu du R\=aj\=adhir\=ajava\d lan\=a\d tu\index{cec}{Rajadhirajavala@R\=aj\=adhir\=ajava\d lan\=a\d tu} (ARE 1918 528 et 538 \`a \=Acc\=a\d lpuram\index{gnl}{Accalpuram@\=Acc\=a\d lpuram}, localit\'e situ\'ee \`a environ dix kilomètres au nord-est de C\=\i k\=a\b li\index{gnl}{Cikali@C\=\i k\=a\b li}). Ainsi, nous supposons que l'assemblée\index{gnl}{assemblée} de CEC 28 est celle de Pa\~ncava\b nam\=atevi\index{cec}{panca@Pa\~ncava\b nam\=atevi}  alias Kulottu\.nkaco\b laccaruppetima\.nkalam dans le Ve\d n\d naiy\=urn\=a\d tu.}.

\section*{CEC 29}
\subsection*{CEC 29.1 Remarques}

L'\'inscription a été relev\'ee dans l'ARE 1918 377 et localis\'ee sur le mur nord du temple\index{gnl}{temple} de Campantar\index{gnl}{Campantar}. Elle date de la douzi\`eme ann\'ee de r\`egne de Tribhuvanacakravartin Kulottu\.ngac\=o\b ladeva\index{cec}{kulottungacoladeva@Kulottu\.ngac\=o\b ladeva} que \textsc{Mahalingam} (1992: 550, Tj. 2415) identifie sans en être certain comme Kulottu\.nga III\index{gnl}{Kulottu\.nga III} en proposant la date de 1190. Les r\'esum\'es publiés mentionnent juste un \'echange de terres\index{gnl}{terre}.

Situ\'ee sur la face nord, l'\'epigraphe comporte vingt-neuf lignes qui couvrent les trois-quarts de la deuxi\`eme portion du mur d\'elimit\'e par des pilastres en partant de l'est. Il se trouve au-dessus de CEC 30 et \`a l'est de CEC 28. Les espaces entres les pierres ont \'et\'e renforc\'es avec du ciment et ce faisant, les l.~4, 8, 16 et 22 sont illisibles \textit{in situ}. Elles ont \'et\'e partiellement reconstitu\'ees avec la transcription de l'ASI.

L'emplacement de CEC 29 par rapport \`a CEC 30 permet de d\'efendre que le roi\index{gnl}{roi} est Kulottu\.nga II\index{gnl}{Kulottu\.nga II}. En effet, CEC 30 d\'ebute exactement sous CEC 29. Aucun \'el\'ement architectural ne vient les distinguer. Un simple interligne, l\'eg\`erement plus grand que le corps du texte, signale le passage d'une inscription \`a l'autre. De plus, CEC 30 s'\'etend sur le dernier quart du mur et d\'eborde sur le soubassement. Les \'epigraphes occupent en g\'en\'eral un corps de b\^atiment homog\`ene quand elles ont de la place. Ainsi, il est certain que CEC 30 a été grav\'ee apr\`es CEC 29 et qu'elle lui est donc post\'erieure\footnote{Consid\'erant, pour la chapelle de Campantar\index{gnl}{Campantar}, le style architectural dit \og \textit{c\=o\b la} tardif\fg\ et les donn\'ees \'epigraphiques correspondant \`a cette p\'eriode, la possibilit\'e que ces inscriptions soient des copies d'anciennes qui auraient disparu au moment d'une \'eventuelle reconstruction du b\^atiment est très faible à notre avis.}. Les donn\'ees astronomiques de CEC 30 ont permis \`a \textsc{Mahalingam} de dater l'inscription d'un lundi du mois d'avril 1158. Cette information est v\'erifi\'ee et compl\'et\'ee avec le programme informatique \og Pancanga\fg: CEC 30 date du lundi 21 avril 1158 sous le r\`egne de R\=ajar\=aja II\index{gnl}{Rajaraja II@R\=ajar\=aja II}\footnote{L'ann\'ee de r\`egne (12), le mois (\textit{Me\d sa}, \textit{Caitra}), la quinzaine lunaire (sombre) et le \textit{nak\d satra} (\textit{Uttir\=a\d tam}, \textit{Uttara}-\textit{\=A\d s\=a\d dha}) et le jour de la semaine (\textit{ti\.nka\d l}, lundi) correspondent \`a cette datation. Sur \og Pancanga\fg, toutes ces donn\'ees ne co\"incident jamais pour R\=ajar\=aja III\index{gnl}{Rajaraja III@R\=ajar\=aja III}.}. Et ce faisant, CEC 29 date de la douzi\`eme ann\'ee de r\`egne de Kulottu\.nga II\index{gnl}{Kulottu\.nga II}, \textbf{1145}.

L'inscription enregistre la compensation d'une terre\index{gnl}{terre} donn\'ee par une autre terre donnée par l'assemblée\index{gnl}{assemblée} de Tiruv\=alin\=a\d tu\index{cec}{Tiruvalin@Tiruv\=alin\=a\d tu} au temple\index{gnl}{temple} de Campantar\index{gnl}{Campantar}. La nouvelle terre\index{gnl}{terre} donn\'ee est destin\'ee, comme l'ancienne, \`a nourrir Campantar\index{gnl}{Campantar} en riz\index{gnl}{riz} au lait\index{gnl}{lait}.

\subsection*{CEC 29.2 Texte}
\begin{enumerate}
	\item {[ti]ripuva\b naccakkiravarttika\d l \textbf{\'sr\=\i}kulottu\.nkaco\b lateva[\b r]}\index{cec}{kulottungacoladeva@Kulottu\.ngac\=o\b ladeva}
	\item ku y\=a\d n\d tu pa\b n\b nira\d n\d t\=a[vatu] u\d taiy\=ar tirucci\b r\b rampa[la]
	\item mu\d taiy\=ar\index{cec}{Tirucirrampalamu@Tirucci\b ra\b rampalamu\d taiy\=ar} tevat\=anam\index{cec}{tevatanam@\textit{tevat\=a\b nam} propriété divine} ir\=a\textbf{j\=adhi}r\=a\textbf{ja}va\d lan\=a\d t\d tut tiruv\=ali[n\=a]
	\item {[\d t\d tu\index{cec}{Tiruvalin@Tiruv\=alin\=a\d tu} mummu\d tico\b laccaruppetima\.nkalattup\index{cec}{caturvedimangalam@Caturvedima\.ngalam!Mummu\d tic\=o\b laccaturvedima\.ngalam} peru\.nku\b ri pe]\footnote{SII 5 986 \`a Tiruve\.nk\=a\d tu\index{gnl}{Venkatu@Ve\.nk\=a\d tu!Tiruve\.nk\=a\d tu}, CEC 8, 9 et 39 se r\'ef\`erent \`a l'assemblée\index{gnl}{assemblée} de Tiruv\=ali\index{cec}{Tiruvali@Tiruv\=ali} alias Mummu\d tico\b laccaturvetima\.nkalam dans le Tiruv\=alin\=a\d tu du R\=aj\=adhir\=ajava\d lan\=a\d tu\index{cec}{Rajadhirajavala@R\=aj\=adhir\=ajava\d lan\=a\d tu}. Alors que ARE 1918 403-405, CEC 12 et 34 mentionnent l'assemblée\index{gnl}{assemblée} de Tiruv\=ali\index{cec}{Tiruvali@Tiruv\=ali} alias Etirilico\b lacaturvetima\.nkalam\index{cec}{caturvedimangalam@Caturvedima\.ngalam!Etirilic\=o\b laccaturvedima\.ngalam} dans le Tiruv\=alin\=a\d tu. Nous pensons qu'il s'agit de la m\^eme assemblée\index{gnl}{assemblée} qui aurait connu un changement de nom. Par ailleurs, un des membres de l'assemblée dans CEC 28, un certain Va\.nkippu\b rattu M\=atevapa\d t\d ta\b n\index{cec}{Mateva@M\=atevapa\d t\d ta\b n}, figure dans SII 5 986 l.~23. Il est donc probable qu'il faille conjecturer l.~4 \textit{mummu\d tico\b laccaturvetima\.nkalam}.}}
	\item rumak[ka]\d l e\b luttu i\b n\b n\=a\d t\d tu \textbf{brahmadeya}m\index{cec}{brahmadeya@\textit{brahmadeya}} tirukka\b lumalattu\index{cec}{Tirukkalumalam@Tirukka\b lumalam}
	\item \=a\d lu\d taiyapi\d l\d laiy\=ar koyilil\index{cec}{koyil@\textit{k\=oyil} temple} tevaka\b nmika\d lum \textbf{\'sr\=\i}m\=a
	\item \textbf{he\'sva}rakka\d nk\=a\d ni\index{cec}{kani@\textit{k\=a\d ni} droit, propriété}\index{cec}{srimahesvara@\textit{\'sr\=\i mahe\'svara} dévot, surveillant} ceyvarka\d lum ka\d n\d tu \=a\d lu\d taiyapi\d l\d lai
	\item {[\dots\ nammurp pi\d t\=akai\index{cec}{pitakai@\textit{pi\d t\=akai} hameau} \dots]}
	\item ...ka\d t\d ta\d laiyil vi\d t\d tu a\b nupavittu varuki\b ra nilam\index{cec}{nilam@\textit{nilam} terre} n\=alu m\=avum akka\d t[\d ta]
	\item \d laiyil kulottu\.nkaco\b laccaruppetima\.nkalattup\index{cec}{caturvedimangalam@Caturvedima\.ngalam!Kulottu\.ngac\=o\b laccaturvedima\.ngalam} pi\d t\=akaiy\=a\index{cec}{pitakai@\textit{pi\d t\=akai} hameau}
	\item y \=urkki\b li\b raiyili\index{cec}{iraiyili@\textit{i\b raiyili} non imposable} u\d lppa\d tap pi\b ri\textbf{nta}maiyil i\b n\b nilattukkut\index{cec}{nilam@\textit{nilam} terre} talai
	\item m\=a\b ru namm\=urp pi\d t\=akai\index{cec}{pitakai@\textit{pi\d t\=akai} hameau} kulam\=a\d nikkanall\=urk ka\d t\d ta\d laiyil ti[ru]pp\=a\b r
	\item ponakam\=aka y\=a\d n\d tu \=a\b r\=avatu mutal vi\d t\d ta ir\=a\textbf{ja}ir\=a\textbf{ja}vatikkuk\index{cec}{vati@\textit{vati}} ki\b lakku mu
	\item mmu\d tico\b lav\=aykk\=alukku\index{cec}{vaykkal@\textit{v\=aykk\=al} canal} va\d takku 3 C 3 tu\d n\d tattu 4 C 1 tu\d n\d tattum ki
	\item \b lakka\d taiya kulaiyum ti\d talum te\.nk\=a tu\d n\d tattu\footnote{Le lapicide a ajout\'e ce mot dans l'interligne au-dessus de \textit{nilaiyum}.} nilaiyum akappa\d ta N 4 [M] i\b n\b nila
	\item {[n\=alu m\=avukku \dots viyunta i\b rai iyi \dots\ ta\b n\=urile e\b r\b ri\d nta i\b rukka i\b n\b ni]}
	\item la n\=alu m\=avum k\=acu\index{cec}{kacu@\textit{k\=acu} pièces de monnaie} [ko]\d l\d l\=a i\b raiyiliy\=akak\index{cec}{iraiyili@\textit{i\b raiyili} non imposable} kaikko\d ntu payircce
	\item ytu tirupp\=arpona[ka]p pu\b ram\=aka\index{cec}{puram@\textit{pu\b ram} terre de donation!\textit{p\=arponakappu\b ram} terre donnée pour offrir du riz au lait} amutuceytaru\d lap pa\d n\d nuka pa\b niy\=a
	\item l \=urkka\d nakkut\index{cec}{kanakku@\textit{ka\d nakku} comptable} tirunilaka\d n\d tan tirukka\b lipp\=alai u\d taiy\=an e\b lutti\d t\d tatu
	\item \textbf{sa}paiy\=aril e\b lutti\d t\d t\=ar va\.nkippu\b rattu k\=uriyatevapa\d t\d ta\b n tiricci\b r\b rampalamu\d t
	\item {[ai]y\=a\b n\index{cec}{tiruccirrampalamutaiyar@Tirucci\b r\b rampalamu\d taiy\=ar} tillain\=ayaka\b n\index{cec}{Tillain\=ayakar} ka\textbf{sya}pa\b n tiruve\d nk\=a\d tu\d taiy\=a\b n\index{cec}{Tiruvenkatu@Tiruve\d nk\=a\d tu\d taiy\=a\b n} [m\=a]teva\b n ir\=ay\=ur [ti]ricci\b r\b ra}
	\item {[mpalamu\d taiy\=a\b npa\d t\d ta\b n \dots k\=ur cuppirama\d niyapa\d t\d ta\b n\index{cec}{Cuppirama\d nyam} t\=aca \dots a\d nain]}
	\item ta[ko]\d laripa\d t\d ta\b n mu\b rica\d t\d tukkum\=arapa\d t\d ta\b n \textbf{\'sr\=\i}v\=asudeva\b n tiricci\b r\b rampalamu-\d taiy\=a\b n\index{cec}{tiruccirrampalamutaiyar@Tirucci\b r\b rampalamu\d taiy\=ar} mu\d tu
 	\item ..[vi]\b n\=ayakapa\d t\d ta\b n va\.nkippu\b rattu m\=atevapa\d t\d ta\b n\index{cec}{Mateva@M\=atevapa\d t\d ta\b n} kuromiya tiruve\d nk\=a\d tupa\d t-\d ta\b n\b n\=a\index{cec}{Tiruvenkatup@Tiruve\d nk\=a\d tupa\d t\d ta\b n}
	\item ..\textbf{dak\d si}\d n\=amuttipa\d t\d ta\b n u\b ruppu\d t\d tur nampi\index{cec}{Nampi} ve\d l\d l\=ur tiruve\d nk\=a\d tupa\d t\d tan\index{cec}{Tiruvenkatup@Tiruve\d nk\=a\d tupa\d t\d ta\b n} va\.nkip-pu\b rat
	\item tu [m\=a]tevapa\d t\d ta\b n cuppirama\d n\d niyapa\d t\d ta\b n\index{cec}{Cuppirama\d nyam} c\=ant\=urar nampi\index{cec}{Nampi} pir\=akaiccantira-cekarapa\d t\d ta\b n kavi
	\item {[\d niya\b n] cunt\=atto\d tu\d taiy\=a\b n kaviniya\b n [arumo\b liteva\b n tiricci\b r\b rampala]mu\d taiy\=a\b n\index{cec}{tiruccirrampalamutaiyar@Tirucci\b r\b rampalamu\d taiy\=ar} pa\d t\d ta}
	\item {[\b n ta]\~ncapoca\b n [ti]rucci\b r\b rampalamu\d taiy\=a\b n\index{cec}{tiruccirrampalamutaiyar@Tirucci\b r\b rampalamu\d taiy\=ar} vakuntuci tiruve\d nk\=a\d tupa\d t\d ta\b n\index{cec}{Tiruvenkatup@Tiruve\d nk\=a\d tupa\d t\d ta\b n} p\=arat-tuv\=aci}
	\item tiruve\d nk\=a\d tu\d taiy\=a\b n\index{cec}{Tiruvenkatu@Tiruve\d nk\=a\d tu\d taiy\=a\b n} tirucci\b r\b rampalamu\d taiy\=a\b n\index{cec}{tiruccirrampalamutaiyar@Tirucci\b r\b rampalamu\d taiy\=ar} k\=akka\d n\d tu[r*] \textbf{dak\d si}\d n\=amuttipa\d t-\d ta\b n\ddanda |
\end{enumerate}

\subsection*{CEC 29.3 Traduction}
(1-7) En la douzi\`eme ann\'ee de [r\`egne] de \'Sr\=\i kulottu\.ngac\=o\b ladeva\index{cec}{kulottungacoladeva@Kulottu\.ngac\=o\b ladeva}, empereur des trois mondes, [cet acte a été] \'ecrit par les membres de la grande assemblée\index{gnl}{assemblée} de Mummu\d tico\b laccaruppetima\.nkalam\index{cec}{caturvedimangalam@Caturvedima\.ngalam!Mummu\d tic\=o\b laccaturvedima\.ngalam} de Tiruv\=alin\=a\d tu\index{cec}{Tiruvalin@Tiruv\=alin\=a\d tu} dans le R\=aj\=adhir\=ajava\d lan\=a\d tu\index{cec}{Rajadhirajavala@R\=aj\=adhir\=ajava\d lan\=a\d tu}, [qui est un] \textit{devad\=ana} du Seigneur propri\'etaire de Tirucci\b r\b rampalam\index{cec}{tiruccirrampalamutaiyar@Tirucci\b r\b rampalamu\d taiy\=ar}. [Et il a été] vu par les intendants du temple\index{gnl}{temple} d'\=A\d lu\d taiyapi\d l\d laiy\=ar\index{cec}{Alutaiyapillaiyar@\=A\d lu\d taiyapi\d l\d laiy\=ar} de Tirukka\b lumalam\index{cec}{Tirukkalumalam@Tirukka\b lumalam}, \textit{brahmadeya}\index{cec}{brahmadeya@\textit{brahmadeya}} de ce pays, et par ceux qui assurent la surveillance \textit{\'sr\=\i mahe\'svara}\index{cec}{srimahesvara@\textit{\'sr\=\i mahe\'svara} dévot, surveillant}.

(7-16) \`A cause de la s\'eparation (\string?) en tant que hameau de Kulottu\.nkaco\b lac-caruppetima\.nkalam\index{cec}{caturvedimangalam@Caturvedima\.ngalam!Kulottu\.ngac\=o\b laccaturvedima\.ngalam} incluant \textit{\=urkki\b li\b raiyili}\index{cec}{iraiyili@\textit{i\b raiyili} non imposable} [o\`u se trouvait] la terre\index{gnl}{terre} de quatre \textit{m\=a} jouie actuellement et laiss\'ee sous le \textit{ka\d t\d ta\d lai}
%\footnote{XXXXX\textit{ka\d t\d ta\d lai} + dans CEC 33 + \textsc{Gros} 1970, 12 + \textsc{Younger} 1995, 148-149 + \textsc{Reiniche} }
 \dots\ \=A\d lu\d taiyapi\d l\d laiy\=ar\index{cec}{Alutaiyapillaiyar@\=A\d lu\d taiyapi\d l\d laiy\=ar}; en \'echange de cette terre\index{gnl}{terre}, [l'assemblée\index{gnl}{assemblée} donne] la terre\index{gnl}{terre} --- laiss\'ee depuis la sixi\`eme ann\'ee [de r\`egne] en tant que terre\index{gnl}{terre} pour riz\index{gnl}{riz} au lait\index{gnl}{lait} dans le \textit{ka\d t\d ta\d lai} de Kulam\=a\d nikkanall\=ur, hameau de notre village, [et qui se trouve] \`a l'est de la \textit{vati}\index{cec}{vati@\textit{vati}} Ir\=ajaIr\=aja, au nord du canal Mummu\d tico\b la, la 3\up{e} portion du 3\up{e} canalicule et la 1\up{\`ere} portion du 4\up{e} canalicule, incluant le \textit{kulai} \`a l'est, la terre\index{gnl}{terre} \textit{ti\d tal} et le \textit{nilai} de la portion sud --- de 4 \textit{m\=a}\footnote{Une terre\index{gnl}{terre} de quatre \textit{m\=a} qui \'etait donn\'ee pour offrir du riz\index{gnl}{riz} au lait\index{gnl}{lait} à Campantar\index{gnl}{Campantar} a vraisemblablement chang\'e de statut. Le donateur, une assemblée\index{gnl}{assemblée} de Tiruv\=alin\=a\d tu\index{cec}{Tiruvalin@Tiruv\=alin\=a\d tu}, d\'ecide de donner une autre terre\index{gnl}{terre} de quatre \textit{m\=a} en compensation pour continuer l'offrande.}.

(16-19) Ayant pris en main cette terre\index{gnl}{terre} de quatre \textit{m\=a} non imposable et invendable, l'ayant cultiv\'ee, que l'on en fasse une terre\index{gnl}{terre} [destin\'ee] \`a nourrir en riz\index{gnl}{riz} au lait\index{gnl}{lait}.

(20-29) Par le service\index{gnl}{service}, a sign\'e le comptable du village Tirunilaka\d n\d tan un propriétaire [terrien] de Tirukka\b lipp\=alai. Ceux qui ont sign\'e parmi les membres de l'assemblée\index{gnl}{assemblée}:
Va\.nkippu\b rattu K\=uriyatevapa\d t\d ta\b n,
Tirucci\b r\b rampalamu\d taiy\=a\b n\index{cec}{tiruccirrampalamutaiyar@Tirucci\b r\b rampalamu\d taiy\=ar} Tillain\=aya-ka\b n\index{cec}{Tillain\=ayakar},
K\=asyapa\b n\index{cec}{kasyapa@K\=a\'syapa\b n} Tiruve\d nk\=a\d tu\d taiy\=a\b n\index{cec}{Tiruvenkatu@Tiruve\d nk\=a\d tu\d taiy\=a\b n} M\=ateva\b n\index{cec}{Matev@M\=ateva\b n},
Ir\=ay\=ur Tiricci\b r\b rampalamu\d taiy\=a\b n\index{cec}{tiruccirrampalamutaiyar@Tirucci\b r\b rampalamu\d taiy\=ar}pa\d t-\d ta\b n,
\dots k\=ur Cuppirama\d niyapa\d t\d ta\b n\index{cec}{Cuppirama\d nyam},
T\=aca\dots\ a\d nainta[k]o\d laripa\d t\d ta\b n,
Mu\b rica\d t\d tukku-m\=arapa\d t\d ta\b n \'Sr\=\i v\=as\=udeva\b n,
Tiricci\b r\b rampalamu\d taiy\=a\b n\index{cec}{tiruccirrampalamutaiyar@Tirucci\b r\b rampalamu\d taiy\=ar} Mu\d tu ..vi\b n\=ayakapa\d t\d ta\b n,
Va\.n-kippu\b rattu M\=atevapa\d t\d ta\b n\index{cec}{Mateva@M\=atevapa\d t\d ta\b n},
Kuromiya Tiruve\d nk\=a\d tupa\d t\d ta\b n\index{cec}{Tiruvenkatup@Tiruve\d nk\=a\d tupa\d t\d ta\b n} \b N\=a..dak\d si\d n\=amurttipa\d t\d ta\b n,
U\b ruppa\d t\d tur Nampi\index{cec}{Nampi},
Ve\d l\d l\=ur Tiruve\d nk\=a\d tupa\d t\d tan\index{cec}{Tiruvenkatup@Tiruve\d nk\=a\d tupa\d t\d ta\b n},
Va\.nkippu\b rattu M\=atevapa\d t\d ta\b n\index{cec}{Mateva@M\=atevapa\d t\d ta\b n} Cup-piramma\d n\d niyapa\d t\d ta\b n\index{cec}{Cuppirama\d nyam},
C\=ant\=urar Nampi\index{cec}{Nampi} Pir\=akaiccantiracekarapa\d t\d ta\b n,
K\=avi\d niya\b n Cu-nt\=atto\d tu\d taiy\=a\b n,
Kaviniya\b n Arumo\b liteva\b n Tiricci\b r\b rampalamu\d taiy\=a\b n\index{cec}{tiruccirrampalamutaiyar@Tirucci\b r\b rampalamu\d taiy\=ar}pa\d t\d ta\b n,
Ta\~nca-poca\b n Tirucci\b r\b rampalamu\d taiy\=a\b n\index{cec}{tiruccirrampalamutaiyar@Tirucci\b r\b rampalamu\d taiy\=ar},
Vakuntuci Tiruve\d nk\=a\d tupa\d t\d ta\b n\index{cec}{Tiruvenkatup@Tiruve\d nk\=a\d tupa\d t\d ta\b n},
P\=arattuv\=aci Ti-ruve\d nk\=a\d tu\d taiy\=a\b n\index{cec}{Tiruvenkatu@Tiruve\d nk\=a\d tu\d taiy\=a\b n} Tirucci\b r\b rampalamu\d taiy\=a\b n\index{cec}{tiruccirrampalamutaiyar@Tirucci\b r\b rampalamu\d taiy\=ar},
K\=akka\d n\d tur Dak\d si\d n\=amuttipa\d t\d ta\b n\footnote{Les noms ont \'et\'e s\'epar\'es par rapport aux \textit{gotra\index{gnl}{gotra@\textit{gotra}}} et aux origine\index{gnl}{origine}s g\'eographiques. Va\.nkippu\b ram, U\b ruppa\d t\d tur et Ir\=ay\=ur semblent \^etre des toponymes fortement associ\'es \`a des brahmane\index{gnl}{brahmane}s \textit{pa\d t\d tar}, membres des \textit{sabh\=a} (SII 5 986, 7 1025, 17 586, 3 78). Il est aussi notable que cinq d'entre eux sont li\'es \`a Tiruve\.nk\=at\d u.}.


\section*{CEC 30}
\subsection*{CEC 30.1 Remarques}

L'\'epigraphe, relev\'ee dans l'ARE 1918 375 et localis\'ee sur le mur nord de la chapelle de Campantar\index{gnl}{Campantar}, date de la douzi\`eme ann\'ee de r\`egne de Tribhuvanacakravartin R\=ajar\=ajadeva. Nous avons montr\'e dans les remarques de CEC 29 que ce texte date du \textbf{lundi 21 avril 1158} sous R\=ajar\=aja II\index{gnl}{Rajaraja II@R\=ajar\=aja II}.

L'inscription comporte trente-trois lignes qui se trouvent en-dessous de CEC 29 r\'eparties sur le mur (1-12) et le soubassement (13-33).

Le texte enregistre une donation de terre\index{gnl}{terre} de cuisine pour nourrir Ma\.nkaiyarkka-raci N\=acciy\=ar\index{gnl}{Mankaiyarkkaraci@Ma\.nkaiyarkkaraci} qui est install\'ee dans le temple\index{gnl}{temple} de Campantar\index{gnl}{Campantar}. Le donateur est le \textit{parikkirakam} de V\=\i rac\=o\b lanall\=ur\index{cec}{Viracolanallur@V\=\i raco\b lanall\=ur} dans le Tirukka\b lumalan\=a\d tu\index{cec}{Tirukka\b lumalan\=a\d tu}.

\subsection*{CEC 30.2 Texte}
\begin{enumerate}
	\item tiripuvanaccakkaravatti[ka]\d l\index{cec}{Tribhuvanacakravarti} \textbf{\'sr\=\i}r\=a\textbf{ja}r\=a\textbf{jade}va\b rku y\=a\d n\d tu pa\b n\b nira\d n\d t\=avatu \textbf{me-\d sa}\b n\=aya\b r\b ru aparapa\textbf{k\d sa}tu
	\item ti\.nka\d l ki\b lamaiyum [u]ttir\=a\d tamum pe\b r\b ra a\b n\b ru r\=a\textbf{j\=adha}r\=a\textbf{ja}va\d la\b n\=a\d t\d tut tirukka-\b lumala\b n\=a\d t\d tu viraco\b la
	\item {[\b nal]l\=ur parikkirakattu [mu]tala\d taippu\index{cec}{ataippu@\textit{a\d taippu} limite} cuv\=ami \textbf{santo\d sa}p pallavaraya\b num\index{cec}{Pallavar\=aya\b n} pa-r\=a\textbf{kra}maco\b lappa[lla]varaya\b num}
	\item c[e]mpiya\b n pallavaraya\b num\index{cec}{Pallavar\=aya\b n} cempiya\b n\index{cec}{Cempiya\b n} vi\b lupparaya\b num\index{cec}{vilupparayan@Vi\b lupparaya\b n} kulottu\.nkaco\b lap\index{cec}{kulottungacola@Kulottu\.ngac\=o\b la} palla-[va]raya\b num\index{cec}{Pallavar\=aya\b n} ta
	\item {[....pallavaraya\b num\index{cec}{Pallavar\=aya\b n} atik\=ara\b n\=ayakap] pallavaraiya\b num\index{cec}{Pallavar\=aya\b n} ka\.nkaiko\d n\d taco\b lap palla-varaya\b nu\index{cec}{Pallavar\=aya\b n}}
	\item m ir\=aca\b n\=ar\=aca\d nap pallavaraya\b num\index{cec}{Pallavar\=aya\b n} m\=a\b na\dots
	\item co\b lako\b n pallavaraya\b num\index{cec}{Pallavar\=aya\b n} ir\=a\textbf{ja}r\=a\textbf{ja}p pallavaraya\b num\index{cec}{Pallavar\=aya\b n} malaiyar\=aya\b num \=al\=alacu-ntarap pallavaraya
	\item \b num\index{cec}{Pallavar\=aya\b n} cik\=a\b lip\index{cec}{cikali@C\=\i k\=a\b li} pallavaraya\b num\index{cec}{Pallavar\=aya\b n} vikkiramaco\b lap\index{cec}{Vikkiramaco\b la} pallavaraya\b num\index{cec}{Pallavar\=aya\b n} ka\d tampar\=aya\b nu-m ca\d npayar\=aya\b num ka\.nkam\=a\b nu\footnote{Ce nom figure \`a la fin de la l.~8 et non au d\'ebut de la suivante comme l'indique la transcription.}
	\item m vila\.nk\=amo\b lip pallavaraya\b num\index{cec}{Pallavar\=aya\b n} v\=anava\b n pallavaraya\b num\index{cec}{Pallavar\=aya\b n} cuttamali\index{cec}{Cuttamali} vi\b luppara-ya\b num\index{cec}{vilupparayan@Vi\b lupparaya\b n} kolla
	\item ttaraiya\b num u\d l\d li\d t\d taparikkirakattom inn\=a\d t\d tu tirukka\b lumalattu\index{cec}{Tirukkalumalam@Tirukka\b lumalam} \=a\d lu\d taiyapi\d l\d lai-y\=ar koyili\index{cec}{koyil@\textit{k\=oyil} temple}
	\item l e\b luntaru\d liyirukkum ma\.n[ka*]ya\b rkaraici \b n\=acciy\=a\b rku (amu)tucetaru\d lukaikku tiruma\d taippa\d l\d lippu\b ram\=aka\index{cec}{puram@\textit{pu\b ram} terre de donation!\textit{ma\d tappa\d l\d lippu\b ram} terre donnée pour la cuisine} \b n\=a\.nka\d l
	\item vi\b r\b rukku\d tutta nilam\=avatu\index{cec}{nilam@\textit{nilam} terre} i\b n\b n\=a\d t\d tu \textbf{brahmadaya}m\index{cec}{brahmadeya@\textit{brahmadeya}} tirukka\b lumalattu\index{cec}{Tirukkalumalam@Tirukka\b lumalam} vira(co-\b la)[nall\=ur]........ e\.nka
	\item {[\d l nilattu\index{cec}{nilam@\textit{nilam} terre} cuttamalivatikku\index{cec}{Cuttamali}\index{cec}{vati@\textit{vati}} me\b rku ir\=acentiraco\b lav\=aykk\=alukkutte\b rku\index{cec}{vaykkal@\textit{v\=aykk\=al} canal} ira\d n\d t\=a
	\item \.nka\d n\d n\=a\b r\b ru muta\b rcatirattu \=a\d lu\d taiyapi\d l\d laiy\=ar tiruna\b ntava\b nattukku me\b rku \dots]}\footnote{Les l.~13 et 14 ne sont pas accessibles et donc elles n'ont pu \^etre v\'erifi\'ees \textit{in situ}. La lecture de la transcription de l'ASI est suivie.}
	\item ...tavatikkuk\index{cec}{vati@\textit{vati}} ki\b lakku p\=arattu[v\=a]ci tiv\=akara\b n.\textbf{\'sva} kollaivatikkut\index{cec}{vati@\textit{vati}} te\b rku p\=arat-tu[v\=aci] ...vatti\textbf{\'sva}
	\item ..y\=a\b na kollai\b nilattukku\index{cec}{nilam@\textit{nilam} terre} va\dots\footnote{La pierre de cette partie du soubassement est bris\'ee en son milieu.}\b nkellaiyu\d l \b na\d tuvupa\d t\d ta kollai N 1/2 2 M H
	\item i\b n\b nilam\index{cec}{nilam@\textit{nilam} terre} araiye ira\d n\d tu m\=a mukk\=a\d niyum mikutik ku\b raivu u\d l\d la\d ta\.nka vi\b r\b ruk ku\d tuttuk ko\b l
	\item vat\=a\b na emmilicai\b nta vilaipporu\d l\index{cec}{vilai@\textit{vilai} prix} a\b n\b r\=a\d tu \b na\b rk\=acu\index{cec}{kacu@\textit{k\=acu} pièces de monnaie} 6 10 4 ikk\=acu a\b rupattu\b n\=alum i\b n\b nilam\index{cec}{nilam@\textit{nilam} terre} k\=acu
	\item ko\d l\d l\=a i\b r\=aiyiliy\=aka i\b raiyi\b liccik ko\d n\d ta k\=acu 8 10 6 \=akak k\=acu 100 5 10 ikk\=acu n\=u\b r\b raimpatu
	\item m [\=a]va\d nakace\textbf{nti}ye k\=a\d ta...kkaccila va\b rakko\d n\d tu vi\b r\b ru vilaiy\=ava\d nam\index{cec}{vilai@\textit{vilai} prix!vilaiyavanam@\textit{vilaiy\=ava\d nam} document de vente} cai.. ittom ma
	\item \.nkaya\b rkaraci \b n\=acciy\=a\b rku viraco\b la \dots tom i\b n\b nilattu\index{cec}{nilam@\textit{nilam} terre} men
	\item \b nokki\b na maramum ki\b l\b n\=okkina ki\d na\b rum ma\b naiyum ma\b naippa\d tappaiyum u\d l-pa\d ta i\b n\b nilam\index{cec}{nilam@\textit{nilam} terre} araiye
	\item ira\d ntu m\=a mukk\=a\d niyum k\=acu\index{cec}{kacu@\textit{k\=acu} pièces de monnaie} ko\d l\d l\=a i\b raiyiliy\=aka\index{cec}{iraiyili@\textit{i\b raiyili} non imposable} vi\b r\b rukku\d tuttu ikk\=acu \b n\=u\b r\b raim-patum ko
	\item \d n\d tu i\b n\b nilam\index{cec}{nilam@\textit{nilam} terre} cempilum kallilum ve\d t\d tikko\d l\d lakka\d tavat\=aka vi\b r\b rukku\d tuttom v\=\i ra-co\b la\b na
	\item ll\=ur parikkirakattom ivarka\d l colla ippi\b ram\=a\d nam e\b luti\b n\=a\b n ka\b lumalamu\d taiy\=a\b n tirukkolakk\=a
	\item vu\d taiy\=a\b n pira\d laiyavi\d ta\.nkan\index{cec}{Piralaiya@Pira\d laiyavi\d ta\.nka\b n} e\b luttu parikkirakatt\=aril e\b lutti\d t\d t\=ar cempiya\b n\index{cec}{Cempiya\b n} palla-varaiya\b n\index{cec}{Pallavar\=aya\b n} cempiya\b n\index{cec}{Cempiya\b n}
	\item vi\b lupparaiya\b n\index{cec}{vilupparayan@Vi\b lupparaya\b n} kulottu\.nkaco\b lap\index{cec}{kulottungacola@Kulottu\.ngac\=o\b la} pallavaraiya\b n\index{cec}{Pallavar\=aya\b n} tevarka\d l[\b n\=aya]ka\b n pukali\index{cec}{Pukali}\b n\=a\d t\d tu ve-\d l\=a\b n pira\d laiya
	\item vi\d ta\.nkap pallavaraiya\b n\index{cec}{Pallavar\=aya\b n} po\b rkoyi\b rco\b lap pallavaraiya\b n\index{cec}{Pallavar\=aya\b n} ..ticaiv\=ara\d nap pallavaraiya-\b n\index{cec}{Pallavar\=aya\b n} i
	\item r\=aca\b n\=ar\=aca\d nap pallavaraiya\b n\index{cec}{Pallavar\=aya\b n} ir\=a\textbf{j\=adha}r\=a\textbf{ja}p pallavaraiya\b n\index{cec}{Pallavar\=aya\b n} tiru\b n\=a\b nacampantap\index{cec}{tirunana@Tiru\~n\=a\b nacampanta\b n} pallavaraiya\b n\index{cec}{Pallavar\=aya\b n} cu
	\item v\=ami\textbf{santo\d sa}p pallavaraiya\b n\index{cec}{Pallavar\=aya\b n} atik\=aran\=ayakap pallavaraiya\b n\index{cec}{Pallavar\=aya\b n} ka\.nkaiko\d n\d taco\b la
	\item pallavaraiya\b n\index{cec}{Pallavar\=aya\b n} ka\.nkaim\=a\b n atikaim\=a\b n \dots va\b lupparaiya\b n cuttamali\index{cec}{Cuttamali} vi\b luppa]
	\item raiya\b n\index{cec}{vilupparayan@Vi\b lupparaya\b n} ka\d tak\=av\=a\d nap pallavaraiya\b n\index{cec}{Pallavar\=aya\b n} (vikkiramaco\b lap\index{cec}{Vikkiramaco\b la} pallavaraiya\b n\index{cec}{Pallavar\=aya\b n} a\b lay\=a\d lu\d taiy\=a)
	\item \b n tiruv\=a\b ru\d taiya\b n pi\d l\d laive\d l\=a\b n\danda |
\end{enumerate}

\subsection*{CEC 30.3 R\'esum\'e}
(1-12) La douzi\`eme ann\'ee de r\`egne de \'Sr\=\i r\=ajar\=ajadeva, empereur des trois mondes, le mois de \textit{Me\d sa}, la quinzaine sombre, lundi et le jour de l'obtention du [\textit{nak\d satra}] \textit{Uttir\=a\d tam}; les membres du \textit{parikkirakam}\footnote{Le \textit{parikkirakam} semble \^etre un groupe assurant la garde d'une localit\'e. Il se compose d'officier\index{gnl}{officier}s, ici, majoritairement titr\'e de -\textit{pallavarayar}\index{cec}{Pallavar\=aya\b n} (informations communiqu\'ees par G. \textsc{Vijayavenugopal}).} de Viraco\b lanall\=ur dans le Tirukka\b lumalan\=a\d tu\index{cec}{Tirukka\b lumalan\=a\d tu} du R\=aj\=adhir\=ajava\d lan\=a\d tu\index{cec}{Rajadhirajavala@R\=aj\=adhir\=ajava\d lan\=a\d tu} --- comprenant Cuv\=ami Santo\d sap Pallavaraya\b n\index{cec}{Pallavar\=aya\b n} du premier rang\footnote{\textit{mutala\d taippu}?}, Par\=akramaco\b la Pallavaraya\b n\index{cec}{Pallavar\=aya\b n}, Cempiya\b n\index{cec}{Cempiya\b n} Pallava-raya\b n\index{cec}{Pallavar\=aya\b n}, Cempiya\b n\index{cec}{Cempiya\b n} Vi\b lupparaya\b n\index{cec}{vilupparayan@Vi\b lupparaya\b n}, Kulottu\.nkaco\b lap\index{cec}{kulottungacola@Kulottu\.ngac\=o\b la} Pallavaraya\b n\index{cec}{Pallavar\=aya\b n}, Ta \dots varaya\b n, Ati-k\=ara\b n\=ayaka Pallavaraiya\b n\index{cec}{Pallavar\=aya\b n}, Ka\.nkaiko\d n\d tac\=o\b la Pallavaraya\b n\index{cec}{Pallavar\=aya\b n}, Ir\=aca\b n\=ar\=aca\d na Pallava-raya\b n\index{cec}{Pallavar\=aya\b n} \dots\ Co\b lako\b n Pallavaraya\b n\index{cec}{Pallavar\=aya\b n}, Ir\=ajar\=aja Pallavaraya\b n\index{cec}{Pallavar\=aya\b n}, Malaiyar\=aya\b n, \=Al\=alacuntara\index{cec}{alala@\=Al\=alacuntara} Pallavaraya\b n\index{cec}{Pallavar\=aya\b n}, Cik\=a\b li\index{cec}{cikali@C\=\i k\=a\b li} Pallavaraya\b n\index{cec}{Pallavar\=aya\b n}, Vikkiramaco\b la\index{cec}{Vikkiramaco\b la} Pallavaraya\b n\index{cec}{Pallavar\=aya\b n}, Ka\d tampar\=aya\b n, Ca\d npaya-r\=aya\b n, Ka\.nkam\=a\b n, Vila\.nk\=amo\b li Pallavaraya\b n\index{cec}{Pallavar\=aya\b n}, V\=anava\b n Pallavaraya\b n\index{cec}{Pallavar\=aya\b n}, Cuttamali\index{cec}{Cuttamali} Vi\b lupparaya\b n\index{cec}{vilupparayan@Vi\b lupparaya\b n} et Kollattaraiya\b n --- vendent la terre\index{gnl}{terre} suivante en tant que terre\index{gnl}{terre} de cuisine pour nourrir la Dame Ma\.nkaya\b rkaraci qui est \'erig\'ee dans le temple\index{gnl}{temple} d'\=A\d lu\d taiyapi\d l\d laiy\=ar\index{cec}{Alutaiyapillaiyar@\=A\d lu\d taiyapi\d l\d laiy\=ar} \`a Tirukka\b lumalam\index{cec}{Tirukkalumalam@Tirukka\b lumalam}:


(12-21) la terre\index{gnl}{terre} de Viraco\b lanall\=ur \`a Tirukka\b lumalam\index{cec}{Tirukkalumalam@Tirukka\b lumalam}, \textit{brahmadeya}\index{cec}{brahmadeya@\textit{brahmadeya}} de ce Pays, qui se trouve \`a l'ouest de la \textit{vati}\index{cec}{vati@\textit{vati}} Cuttamali\index{cec}{Cuttamali} et au sud du canal d'Ir\=acentiraco\b la, dans le troisi\`eme carr\'e du deuxi\`eme canalicule, \`a l'ouest du jardin d'\=A\d lu\d taiyapi\d l\d laiy\=ar\index{cec}{Alutaiyapillaiyar@\=A\d lu\d taiyapi\d l\d laiy\=ar}.
Cette terre\index{gnl}{terre} d'un demi \textit{v\=eli}, deux \textit{m\=a} et \textit{mukk\=a\d ni} est vendue au prix convenu pour 64 \textit{k\=acu}\index{cec}{kacu@\textit{k\=acu} pièces de monnaie} ayant cours l\'egal, s'y ajoutent 86 \textit{k\=acu} pour que cette terre\index{gnl}{terre} soit faite non imposable et invendable, soit au total, pour 150 \textit{k\=acu}.

(21-25) La r\'ecapitulation indique que les membres du \textit{parikkirakam} de Viraco\b la-nall\=ur ont vendu cette terre\index{gnl}{terre} d'une demi \textit{v\=eli}, deux \textit{m\=a} et \textit{mukk\=a\d ni} (incluant les arbres qui regardent en haut, les puits qui regardent en bas, les maisons et leur extension) comme non imposable et invendable pour cent cinquante \textit{k\=acu}\index{cec}{kacu@\textit{k\=acu} pièces de monnaie} pour la Dame Ma\.nkaya\b rkaraci. Puis, ils ordonnent que cette vente soit grav\'ee sur pierre et cuivre.

(25-33) \`A la dict\'ee des membres du \textit{parikkirakam} de Viraco\b lanall\=ur, Pira\d laiyavi-\d ta\.nkan\index{cec}{Piralaiya@Pira\d laiyavi\d ta\.nka\b n} un propriétaire [terrien] de Ka\b lumalam et de Tirukkolak\=a a \'ecrit ce document. Les membres qui ont pos\'e leurs signatures sont: Cempiya\b n\index{cec}{Cempiya\b n} Pallavaraiya\b n\index{cec}{Pallavar\=aya\b n},
Cempiya\b n\index{cec}{Cempiya\b n} Vi\b lupparaiya\b n\index{cec}{vilupparayan@Vi\b lupparaya\b n},
Kulottu\.nkaco\b lap\index{cec}{kulottungacola@Kulottu\.ngac\=o\b la} Pallavaraiya\b n\index{cec}{Pallavar\=aya\b n},
Tevarka\d l\b n\=ayaka\b n Ve\d l\=a\b n de Pukalin\=a\d tu,
Pira\d laiyavi\d ta\.nka Pallavaraiya\b n\index{cec}{Pallavar\=aya\b n},
Po\b rkoyi\b rco\b la Pallavaraiya\b n\index{cec}{Pallavar\=aya\b n},
Va\dots ti-caiv\=ara\d na Pallavaraiya\b n,
Ir\=aca\b n\=ar\=aca\d na Pallavaraiya\b n,
Ir\=aj\=adhar\=aja Pallavaraiya\b n,
Tiru\b n\=a\b nacampanta\index{cec}{tirunana@Tiru\~n\=a\b nacampanta\b n} Pallavaraiya\b n,
Cuv\=ami Santo\d sa Pallavaraiya\b n,
Atik\=aran\=aya-ka Pallavaraiya\b n,
Ka\.nkaiko\d n\d taco\b la Pallavaraiya\b n\index{cec}{Pallavar\=aya\b n},
Ka\.nkaim\=a\b n,
Atikaim\=a\b n,
\dots va\b lup-paraiya\b n,
Cuttamali\index{cec}{Cuttamali} Vi\b lupparaiya\b n\index{cec}{vilupparayan@Vi\b lupparaya\b n},
Ka\d tak\=av\=a\d na Pallavaraiya\b n,
Vikkiramaco\b la\index{cec}{Vikkiramaco\b la} Pal-lavaraiya\b n
et [enfin], A\b lay\=a\d lu\d taiy\=a\b n Tiruv\=a\b ru\d taiya\b n Pi\d l\d laive\d l\=a\b n.

\section*{CEC 31}
\subsection*{CEC 31.1 Remarques}

L'\'epigraphe a \'et\'e l'objet de nombreuses publications. L'ARE 1896 123 la rel\`eve et la localise sur le mur nord de la chapelle de Campantar\index{gnl}{Campantar}. Puis, l'ARE 1918 379 la situe sur le mur sud. Cette localisation est reprise par \textsc{Mahalingam} (1992, 549, Tj. 2407) qui semble ignorer le relev\'e de 1896. Enfin, elle a \'et\'e publi\'ee dans SII 5 988 qui la place sur le mur nord de la chapelle selon ARE 1896. CEC 31 se trouve sur le soubassement de la face sud, en-dessous de CEC 25 et 26. Elle comporte huit lignes qui s'\'etendent sur deux pierres d'un m\`etre quarante.

L'inscription date de la onzi\`eme ann\'ee de r\`egne de Tribhuvanacakravartin R\=aj\=adhir\=ajadeva que \textsc{Mahalingam} identifie comme R\=aj\=adhir\=aja II en proposant la date de \textbf{1174}.

Au niveau pal\'eographique, les \textit{pu\d l\d li} sont marqu\'es et la diphtongue \textit{ai} est not\'ee par deux \textit{kompu} bien distincts.

Le texte enregistre le don\index{gnl}{don} d'une terre\index{gnl}{terre} par un certain \=A\d tko\d n\d tan\=ayaka\b n Tiru-na\d t\d tapperum\=a\d l \textit{ki\b l\=a\b n} de Ve\d nma\d ni \`a Campantar\index{gnl}{Campantar} pour le nourrir quotidiennement, les jours fastes et les jours de f\^ete\index{gnl}{fete@fête} annuelle.

\subsection*{CEC 31.2 Texte}
\begin{enumerate}
	\item tiripuva\b naccakkaravattika\d l\index{cec}{Tribhuvanacakravarti} \textbf{\'sr\=\i}ir\=a\textbf{j\=a}tir\=a\textbf{ja}teva\b rkku y\=a\d n\d tu 10 1 ve\d nma\d ni ki\b l\=a\b n \=a\d tko\d n\d ta
	\item n\=ayaka\b n tiruna\d t\d tapperum\=a\d l \=a\d lu\d taiyapi[\d l]\d laiy\=arkku ka\b riyamut\=aka amutu cetaru-\d la n\=a\d l o\b n\b ru[k]
	\item ku alacantip paya\b ru n\=a\b liy\=aka\index{cec}{nali@\textit{n\=a\b li} unité de mesure de graine} amutucetaru\d lavum ci[\b ra]ppu tirun\=a\d lka\d lum \=a\d t\d tait tirun\=a\d lka\d lukkum (a)
	\item mutucetaru\d lukaikku[m] \=a\d t\d taikku paya\b ru n\=a\b rka(la)[m]\=aka\footnote{La conjecture du texte publi\'e ne respecte pas le nombre d'\textit{ak\d sara} manquant. Il manque deux graph\`emes dont le premier d\'ebute clairement par une petite boucle, comme pour la semi-voyelle \textit{la}.} cantir\=atittava\b ra cella ir\=a\textbf{j\=a}tar\=a\textbf{ja}va\d lan\=a(\d t)
	\item \d tut tirukka\b lumalan\=a\d t\d tu\index{cec}{Tirukka\b lumalan\=a\d tu} \textbf{brahmadeya}m\index{cec}{brahmadeya@\textit{brahmadeya}} tiruka\b lumalat[tu\index{cec}{Tirukkalumalam@Tirukka\b lumalam} cu]tamalivatikku\index{cec}{vati@\textit{vati}} ki\b lakku ce\b n\b n\=atav\=akk\=alukkut te\b rku[m]
	\item ra\d n\d t\=a\.nka\d n\d n\=a\b r\b ru mu\b n\b r\=a\~n[ca*]tirattu k\=aciyapa\b n\index{cec}{kasyapa@K\=a\'syapa\b n} utaiyativ\=akara\b n tillain\=ayaka\b n\index{cec}{Tillain\=ayakar} \=a\b na p\=an\=ayakana[m]
	\item pi pakkal\index{cec}{pakkal@\textit{pakkal} auprès de} ko\d n\d ta ma\d takku N H A K M innilamukk[\=a]\d niyaraikk\=a\d nik ki\b l orum\=a-vum \=ur m\=avintav\=ayi\b rut[tu]\footnote{La lecture de la publication \textit{m\=avintavar pirittu} est erron\'ee. Cependant, le sens de la lecture propos\'ee ici reste obscur.}
	\item {[i\b rai]mikiti ko\d n\d tu celuttuvat\=aka ko\d n\d tu [vi\d t\d tatu]\ddanda U}
\end{enumerate}

\subsection*{CEC 31.3 Traduction}
En la 11\up{e} ann\'ee [de r\`egne] de R\=aj\=adhir\=ajadeva, empereur des trois mondes, \=A\d tko\d n\d tan\=ayaka\b n Tiruna\d t\d taperum\=a\d l \textit{ki\b l\=a\b n} de Ve\d nma\d ni [donne une terre\index{gnl}{terre}] pour nourrir \=A\d lu\d taiyapi\d l\d laiy\=ar\index{cec}{Alutaiyapillaiyar@\=A\d lu\d taiyapi\d l\d laiy\=ar} en mets. Qu'on le nourrisse quotidiennement d'un \textit{n\=a\b li}\index{cec}{nali@\textit{n\=a\b li} unité de mesure de graine}\footnote{Les termes \textit{n\=a\b li} et \textit{kalam}\index{cec}{kalam@\textit{kalam} unité de mesure du paddy} renvoient \`a des unit\'es de mesure de graines. Un \textit{kalam} \'equivalait dans la r\'egion de Ta\~nc\=av\=ur\index{gnl}{Tancavur@Ta\~nc\=av\=ur} \`a quatre-vingt-seize \textit{n\=a\b li}\index{cec}{nali@\textit{n\=a\b li} unité de mesure de graine} (\textsc{Appadorai} *1990 [1936]: 407).} d'\textit{alacantippaya\b ru}\footnote{Le sens d'\textit{alacantippaya\b ru} est inconnu. Nous supposons que c'est une esp\`ece de lentille.}. Et, pour les jours fastes et de festival annuel [sont donn\'es] pour le nourrir quatre \textit{kalam}\index{cec}{kalam@\textit{kalam} unité de mesure du paddy} de \textit{paya\b ru} par an tant que durent lune et soleil. [La terre\index{gnl}{terre} donn\'ee se trouve] \`a Tirukka\b lumalam\index{cec}{Tirukkalumalam@Tirukka\b lumalam}, \textit{brahmadeya}\index{cec}{brahmadeya@\textit{brahmadeya}} de Tirukka\b lumalan\=a\d tu\index{cec}{Tirukka\b lumalan\=a\d tu} dans l'Ir\=ajar\=ajava\d lan\=a\d tu,
 \`a l'ouest de la \textit{vati}\index{cec}{vati@\textit{vati}} Cutamali, au sud du canal Ce\b n\b n\=ata, dans le troisi\`eme carr\'e du deuxi\`eme canalicule. [Cette terre\index{gnl}{terre}] de \textit{mukk\=a\d ni araikk\=a\d ni ki\b l} et un \textit{m\=a}, achet\'ee aupr\`es de K\=aciyapa\b n\index{cec}{kasyapa@K\=a\'syapa\b n} Utaiyativ\=akara\b n Tillain\=ayaka\b n\index{cec}{Tillain\=ayakar} alias P\=an\=ayaka Nampi\index{cec}{Nampi}, a été donn\'ee pour offrir [des mets] avec ce qui reste une fois les taxes pay\'ees.

\section*{CEC 32}
\subsection*{CEC 32.1 Remarques}

L'\'epigraphe a été relev\'ee dans l'ARE 1918 376 et localis\'ee sur le mur nord du temple\index{gnl}{temple} de Campantar\index{gnl}{Campantar}. Elle date de la sixi\`eme ann\'ee de r\`egne de Tribhuvanacakravartin V\=\i rar\=aj\=endradeva. \textsc{Mahalingam} (1992: 549, Tj. 2409) identifie ce roi\index{gnl}{roi} comme Kulottu\.nga III\index{gnl}{Kulottu\.nga III} et donne, selon les donn\'ees astronomiques pr\'ecises, la date du jeudi 8 mars 1184. Cependant, cette date serait inexacte selon le programme \og Pancanga\fg\ qui pour ces m\^emes informations propose la date du \textbf{jeudi 1 mars 1184}.

L'inscription se trouve sur le soubassement de la face nord de la chapelle, \`a l'ouest du bec d'\'evacuation. Il se compose d'au moins quatorze lignes. Les sept premi\`eres s'\'etendent sur deux pierres align\'ees (un m\`etre), puis \`a leur c\^ot\'e, un peu plus en avant, les sept autres lignes sont gravées sur deux autres pierres (un m\`etre cinquante). L'épigraphe est inachev\'ee et son \'ecriture est peu d\'elicate.

Le texte enregistre une \'echange de terre\index{gnl}{terre} entre les employ\'es du temple\index{gnl}{temple} de Campantar\index{gnl}{Campantar} et un certain Ticaivi\d la\.nkuco\b la Vi\b lupparaiya\b n\index{cec}{vilupparayan@Vi\b lupparaya\b n}.

\subsection*{CEC 32.2 Texte}
\begin{enumerate}
	\item tiripuva\b nacakkaravattika\d l\index{cec}{Tribhuvanacakravarti} \textbf{\'sr\=\i}virar\=acentira[te]va\b rku\index{cec}{Virarajendre@V\=\i rar\=ajendra} y\=a\d n\d tu \=a\b r\=a
	\item vatu m\=\i \b na n\=aya\b r\b ru apa\textbf{rapak\d sa}ttu ti[vi]tiyaiyum\footnote{La transcription de l'ASI conjecture \og ti[ru]tiyaiyum\fg. Or, selon les autres donn\'ees astronomiques c'est le deuxi\`eme jour de la quinzaine sombre qui est dans le \textit{nak\d satra} Cittirai. Cette correction avait d\'ej\`a \'et\'e faite par \textsc{Mahalingam}.} vi
	\item y\=a\b lak ki\b lamaiyum pe\b r\b ra citti[r]ai n\=a\d l ir\=a\textbf{j\=adhi}
	\item r\=a\textbf{ja}va\d lan\=a\d t\d tu tirukka\b lumalan[\=a]\d t\d tu\index{cec}{Tirukka\b lumalan\=a\d tu} \textbf{baimadaya}m
	\item tiruka\b lumalattu\index{cec}{Tirukkalumalam@Tirukka\b lumalam} \=a\d lu\d taiyapi\d l\d laiy\=ar t[e]vaka\b nmika\d lom
	\item kulottu\.nkaco\b lava\d lan\=a\d t\d tu\index{cec}{kulottungacolavala@Kulottu\.ngac\=o\b lava\d lan\=a\d tu} tiruna[\b r]aiy\=urn\=a\d t\d tu ve\d lur
	\item ve\d l\=ur ki\b lava\b n utaiya\~nceyt\=a\b n [c]ent\=amaraikka\d n\d na
	\item \b n\=a\b na ticaivi\d la\.nkuco\b la vi\b lupparaiya\b nukku\index{cec}{vilupparayan@Vi\b lupparaya\b n} nilap\index{cec}{nilam@\textit{nilam} terre} parivatta\d nai pa\d n\d ni\b na pa\d tiy\=ava-tu
	\item ittevar tevata\b nam\index{cec}{tevatanam@\textit{tevat\=a\b nam} propriété divine} tirun\=a\b nacampantama\.nkalattu\index{cec}{tirunanacampantam@Tiru\~n\=a\b nacampantama\.ngalam} tevar k\=a\d niy\=ay\index{cec}{kani@\textit{k\=a\d ni} droit, propriété} a\d taimuta[l ko]\d l\d lum
	\item nilattu\index{cec}{nilam@\textit{nilam} terre} cuttamalivatikku\index{cec}{Cuttamali} kil\=akku ir\=a\textbf{je}ntiraco\b lav\=aykk\=alukku\index{cec}{vaykkal@\textit{v\=aykk\=al} canal} va\d takku 2 C 1 catirattu na\d tuvi[\b rpu]\b ram
	\item pa\d l\d lam N AAAAAAA ita\b n me\b rku ta\d tar N AAA 10 4vum iva\b nukku Pkku Pm anta[r\=a]
	\item yam i\b rukkak\index{cec}{iruttu@\textit{i\b ruttu} payer un impôt} k\=a\d niy\=aka\index{cec}{kani@\textit{k\=a\d ni} droit, propriété} ku\d tuttu itukkut talai[m\=a]\b ru ve\d lk\=a\b niy\=ay Pkku nellum anta[r\=ayam]
	\item i\b rukki\b ra nilattu\index{cec}{nilam@\textit{nilam} terre} ivatikku\index{cec}{vati@\textit{vati}} ki\b lakku ivv\=akk\=alukku va\=akku 2 C 2 catirattu te\b rkil po\b naceva..
	\item vi\d lai N AAA ta\d tar N AAAA N A i\b n\b nilam\index{cec}{nilam@\textit{nilam} terre} ira\d n\d tu m\=avam tevark\=a\d nip\index{cec}{kani@\textit{k\=a\d ni} droit, propriété} pa\b r\b r\=a\b na nila\index{cec}{nilam@\textit{nilam} terre}
\end{enumerate}

\subsection*{CEC 32.3 R\'esum\'e}
Le texte date de la sixi\`eme ann\'ee de r\`egne de \'Sr\=\i v\=\i rar\=ajendradeva, empereur des trois mondes, le mois de \textit{M\=\i \b na}, le deuxi\`eme jour de la quinzaine sombre, jeudi, dans le \textit{nak\d satra} Cittirai.
Il pr\'esente la mani\`ere dont a \'et\'e fait l'\'echange de terre\index{gnl}{terre} entre les \textit{tevaka\b nmika\d l} du temple\index{gnl}{temple} d'\=A\d lu\d taiyapi\d l\d laiy\=ar\index{cec}{Alutaiyapillaiyar@\=A\d lu\d taiyapi\d l\d laiy\=ar} de Tirukka\b lumalam\index{cec}{Tirukkalumalam@Tirukka\b lumalam}, \textit{brahmadeyam}\index{cec}{brahmadeya@\textit{brahmadeya}} de Tirukka\b lumalan\=a\d tu\index{cec}{Tirukka\b lumalan\=a\d tu} et Utaya\~nceyt\=an\index{cec}{utaiyanceytan@Utaiya\~nceyt\=a\b n} Cent\=amaraikka\d n\d na\b n alias Ticaivi\d la\.nkuco\b la Vi\b lupparaiya\b n\index{cec}{vilupparayan@Vi\b lupparaya\b n} \textit{ki\b lava\b n} de Ve\d l\=ur \`a Ve\d lur, dans le Tiruna\b raiy\=urn\=a\d tu du Kulottu\.nkaco\b lava\d lan\=a\d tu\index{cec}{kulottungacolavala@Kulottu\.ngac\=o\b lava\d lan\=a\d tu}\footnote{Sur la localisation de ce \textit{n\=a\d tu}; cf. \textsc{Subbarayalu} (1973, carte 7) qui cependant, ne mentionne pas Ve\d l\=ur.}. Suit la description des terres\index{gnl}{terre} et des montants de la taxe en paddy.

\section*{B. Ma\d n\d dapa}
\section*{CEC 33}
\subsection*{CEC 33.1 Remarques}

L'\'epigraphe a été relev\'ee dans l'ARE 1918 382 et localis\'ee sur le mur nord du \textit{ma\d n\d dapa} devant la chapelle de Campantar\index{gnl}{Campantar}. Il n'y a pas de datation.

L'inscription contient trois lignes qui s'\'etendent sur tout le long d'un \'el\'ement saillant du soubassement sur vingt-quatre m\`etres et quatre-vingt centimètres, au-dessus de CEC 34.

Le texte enregistre les copies, \textit{u\d lvari}\index{cec}{vari@\textit{vari} taxe} (\textsc{Veluthat} 1993: 139 et 142), sur pierre, des documents concernant les propri\'et\'es \textit{tirun\=amattukk\=a\d ni} appartenant au \textit{brahmadeya}\index{cec}{brahmadeya@\textit{brahmadeya}} de Tirukka\b lumalam\index{cec}{Tirukkalumalam@Tirukka\b lumalam}.

Elle abonde en abr\'eviations de mesures de terrain. Ces derni\`eres sont not\'ees, sans distinction de forme et de nombre, par \og @\fg. La sortie d'un bec d'\'evacuation interrompt le cours du texte. Elle est marqu\'ee par \og $\Psi$\fg. Le texte pr\'esent\'e ci-dessous est bas\'e sur l'examen de la transcription de l'ASI, de clich\'es (G. \textsc{Ravindran}, EFEO) et de notre lecture \textit{in situ}.

\subsection*{CEC 33.2 Texte}
\begin{enumerate}
	\item \textbf{svasti \'sr\=\i} \textbf{brahma}tecam\index{cec}{brahmadeya@\textit{brahmadeya}} tirukka\b lumalattut\index{cec}{Tirukkalumalam@Tirukka\b lumalam} tirun\=amattuk k\=a\d nikku\index{cec}{kani@\textit{k\=a\d ni} droit, propriété} \textbf{pram\=a}-\d nap\index{cec}{piramanam@\textit{pram\=a\d nam} document} pa\d ti ta\d ti u\d lvarikkuk\index{cec}{vari@\textit{vari} taxe} kalve\d t\d tu vi\d lainilam\index{cec}{nilam@\textit{nilam} terre} [brahm\=a\d nappa]\b r\b ru talaicca\.nk\=a\d t-\d tuvatikku\index{cec}{Talaiccankatu@Talaicca\.nk\=a\d tu}\index{cec}{vati@\textit{vati}} ki\b lakkut tillaivi\d ta\.nkav\=aykk\=alukkut\index{cec}{Tillaivitankan@Tillaivi\d ta\.nka\b n}\index{cec}{vaykkal@\textit{v\=aykk\=al} canal} te\b rku 1 C 1 catirattu N @ itil ti\d tarum kulai(yu)m N @ nikki N @ 2 C 1 catirattu N @ ivvatikku\index{cec}{vati@\textit{vati}} ki\b lakku \textbf{ja}nan\=ayav\=ayk\=alukkut\index{cec}{vaykkal@\textit{v\=aykk\=al} canal} te\b rku 2 C 1 catirattu N @ i\.nke N @\dots\ [cuttamaliva]tikku\index{cec}{Cuttamali} me\b rku tillaivi\d ta\.nkav\=aykk\=alukkut\index{cec}{Tillaivitankan@Tillaivi\d ta\.nka\b n}\index{cec}{vaykkal@\textit{v\=aykk\=al} canal} te\b rku 1 C 3 catirattu N @ ivvatikku\index{cec}{vati@\textit{vati}} me\b rku \textbf{ja}nan\=ayav\=ayk\=alu[kkut\index{cec}{vaykkal@\textit{v\=aykk\=al} canal} te\b rku 1] C 2 catirattu N @ itil ku\d lamum kulaiyum N @ nikki N @ e\d tuttap\=atavatikku\index{cec}{vati@\textit{vati}} me\b rku r\=a\textbf{jendra}co\b lav\=ayk\=alukku\index{cec}{vaykkal@\textit{v\=aykk\=al} canal} te\b rku 1 C 1 catirattu N @ ita\b n ki\b lakku N @ 2 C 1 catirattu N @ ni\b n\b r\=a\b nav\=ayk\=a-lukku\index{cec}{vaykkal@\textit{v\=aykk\=al} canal} va\d takku 1\dots @ itil ti\d tar N @ $\Psi$ nikki N @ cuttamalivatikku\index{cec}{Cuttamali}\index{cec}{vati@\textit{vati}} ki\b lakku ni\b n\b r\=a\b nv\=ayk\=alukku\index{cec}{vaykkal@\textit{v\=aykk\=al} canal} va\d takku 2 C 1 catirattu N @ 2 C 2ntu\d n\d tattu N @\dots\ \=a\b rp\=avaiyil parivattitta \dots \b la\b n pa\b r\b ril po\b n\b nampalanampikku\index{cec}{ponnampalan@Po\b n\b nampalanampi} ve\d l\d l\=a\b n pa\b r\b ril parivattitta N @ \=aka N @ nikki N @ ve\d l\d l\=a\b n pa\b r\b ru talaica\.nk\=a\d t\d tuvatikku\index{cec}{Talaiccankatu@Talaicca\.nk\=a\d tu}\index{cec}{vati@\textit{vati}} ki\b lakku \textbf{ja}nan\=ayav\=ayk\=alukku\index{cec}{vaykkal@\textit{v\=aykk\=al} canal} te\b rku 1 C 2 catirattu N @ 3 C 2 catirattu N @ e\d tuttap\=a-tavatikku\index{cec}{vati@\textit{vati}} me\b rku r\=a\textbf{jendra}co\b lav\=ayk\=alukku\index{cec}{vaykkal@\textit{v\=aykk\=al} canal} te\b rku 2 C muta\b r(tu\d n\d tat)tu N @ itil ku\d la N @ nikki N @ ni\b n\b r\=anv\=ayk\=alukku\index{cec}{vaykkal@\textit{v\=aykk\=al} canal} va\d takku 1 C 2 catirattu N @ 3 tu\d n\d tattu N @ itil ti\d tar N @ nikki N @ ka\d t\d ta\d lai e\d tuttap\=atavatikku\index{cec}{vati@\textit{vati}} me\b rku
	\item nin\b r\=anv\=aykk\=alukku\index{cec}{vaykkal@\textit{v\=aykk\=al} canal} va\d takku 2 C 3 catirattu N @ itil kulamum kulaiyum N @ nikki N @ i\.nke N @ i\.nke N @ itil ku\d lamum ti\d tarum N @ nikki N @ ivvatikku\index{cec}{vati@\textit{vati}} me\b rku tillaiy\=a\d liv\=ayk[k\=alukku]\index{cec}{Tillaiy\=a\d li}\index{cec}{vaykkal@\textit{v\=aykk\=al} canal} va\d takku 1 C 1 catirattu N @ e\.nka\d t\d ta\d lai cuttamalivatikku\index{cec}{Cuttamali}\index{cec}{vati@\textit{vati}} ki\b lakku nin\b r\=anv\=ayk\=aluk[ku\index{cec}{vaykkal@\textit{v\=aykk\=al} canal} va]\d takku 2 C 2 tu\d n\d tattu N \dots\ [tillaiy\=a\d li]v\=ayk\=alukku\index{cec}{vaykkal@\textit{v\=aykk\=al} canal} va\d takku 1 C 1 catirattu N @ itil ti\d tar N @ nikki N @ i\.nke N @ itil pu\b ncey N @ nikki N @ 2 tu\d n\d tattu N @ [ka]\d t\d ta\d lai cuttamalivatikku\index{cec}{Cuttamali}\index{cec}{vati@\textit{vati}} me\b rku r\=a\textbf{jendra}co\b lav\=ayk\=alukku\index{cec}{vaykkal@\textit{v\=aykk\=al} canal} te\b rku 2 C 2 catirattu v\=acciyan\index{cec}{Vacciyan@V\=acciya\b n} araiyatevan pukalivi\d ta\.nkan u\d l\d li\d t\d t\=ar pakkal\index{cec}{pakkal@\textit{pakkal} auprès de} ko\d n\d ta \dots\ itil \=a\d lu\d tai$\Psi$yapi\d l\d laiy\=ar tirun\=amattu ni\.nkal\=aka nikkina N @ nikki te\b rka\d taiya pu\~nce N @ itil inn\=ayan\=ar tevat\=a\b nam\index{cec}{tevatanam@\textit{tevat\=a\b nam} propriété divine} vi\d lai[nilam]\index{cec}{nilam@\textit{nilam} terre} nikki puncey N @ tillaivi\d ta\.nkav\=ayk\=alukku\index{cec}{Tillaivitankan@Tillaivi\d ta\.nka\b n}\index{cec}{vaykkal@\textit{v\=aykk\=al} canal} te\b rku 1 C 2 catirattu vatakka\d taiya kulottu\.nkaco\b lap\index{cec}{kulottungacola@Kulottu\.ngac\=o\b la} Piramam\=ar\=ayar pakkal\index{cec}{pakkal@\textit{pakkal} auprès de} vilai ko\d n\d ta\index{cec}{vilai@\textit{vilai} prix!vilai-kol@\textit{vilai-ko\d l} acheter} pu\b ncey N @ 3 catirattu va\d takka\d taiya c\=av\=antiy\=arka\d l u\d l\d li\d t\d t\=ar pa(kkal) vilai ko\d n\d ta pu\b ncey N \dots\ 2 C 1 catirattu puncey N @ 2ntu\d n\d tattu c\=av\=antiuyyani\b n\b r\=a-\d tuv\=a\b n pu\b ncey N @ \textbf{je}nan\=a[yav\=ayk\=alu]kku\index{cec}{vaykkal@\textit{v\=aykk\=al} canal} te\b rku 1 C 1 tu\d n\d tattu ca\d n\d te\'svara\index{cec}{candesvara@Ca\d n\d de\'svara} nampi\index{cec}{Nampi} u\d l\d li\d t\d t\=ar pakkal\index{cec}{pakkal@\textit{pakkal} auprès de} ko\d n\d ta pu\b ncey N @ 2ntu\d n\d tattu k\=a\d te\b r\b ruk kollai N @ e\d tuttap\=atavatikku\index{cec}{vati@\textit{vati}} me\b rku tillaivi\d ta\.nkav\=ayk\=alukku\index{cec}{Tillaivitankan@Tillaivi\d ta\.nka\b n}\index{cec}{vaykkal@\textit{v\=aykk\=al} canal} te\b rku 2 C 1 tu\d n\d tattu N @ ka\d t\d ta\d lai e\d tuttap\=atavati\index{cec}{vati@\textit{vati}}
	\item kku me\b rku r\=a\textbf{jendra}co\b lav\=ayk\=alukku\index{cec}{vaykkal@\textit{v\=aykk\=al} canal} te\b rku 1 C 1 catirattu te\b rka\d tainta \dots\ ki\b lakka\d taiya N @ i\.nku vi\d t\d tu va\d tame\b rka\d taiya \textbf{ga}\d nan\=aya\b n tiruva\textbf{gni\'sva}ramu-\d taiy\=a\b n pakkal\index{cec}{pakkal@\textit{pakkal} auprès de} vilai ko\d n\d ta\index{cec}{vilai@\textit{vilai} prix!vilai-kol@\textit{vilai-ko\d l} acheter} N @ i\.nke kolam\=akini\b n\b r\=a\b n u\d l\d li\d t\d t\=ar pakkal\index{cec}{pakkal@\textit{pakkal} auprès de} vilai ko\d n\d ta N @ \=aka N @ i\.nke va\d takka\d tainta N @ puncey N @ ivvatikku\index{cec}{vati@\textit{vati}} me\b rku ivv\=ayk\=alukku\index{cec}{vaykkal@\textit{v\=aykk\=al} canal} va\d takku 1 C 1 catirattu te\b rka[\d t]ai[ya] N @vil ku\d n\d tila\b n tirutto\d ni-puramu\d taiy\=a\b n\index{cec}{Tonipuramutaiyan@T\=o\d nipuramu\d taiy\=a\b n} N @ ulokac\=u\d l\=ama\d nivatikku\index{cec}{vati@\textit{vati}} me\b rku ni\b n\b r\=a\b nv\=ayk\=alukku\index{cec}{vaykkal@\textit{v\=aykk\=al} canal} va\d takku 1 C 2ntu\d n\d tattu pu\b ncey N @ ka\d t\d ta\d lai e\d tuttap\=atavatikku\index{cec}{vati@\textit{vati}} me\b rku ni\b n\b r\=a\b nv\=ayk\=a-lukku\index{cec}{vaykkal@\textit{v\=aykk\=al} canal} va\d takku 1 C 2 catirattu pu\b ncey N @ 3ntu\d n\d tattu pu\b ncey N @ cuttamalivatikku\index{cec}{Cuttamali}\index{cec}{vati@\textit{vati}} me\b rku r\=a\textbf{jendra}co\b lav\=ayk\=alukku\index{cec}{vaykkal@\textit{v\=aykk\=al} canal} va\d takku 1 C 3 catirattu tillaikk\=ut-t\=a\d n\d t\=ar\index{cec}{Tillaikk\=utt\=a\d n\d t\=ar} pakkal\index{cec}{pakkal@\textit{pakkal} auprès de} vilai ko\d n\d ta\index{cec}{vilai@\textit{vilai} prix!vilai-kol@\textit{vilai-ko\d l} acheter} N @ eka\d t\d ta\d lai cuttamalivatikku\index{cec}{Cuttamali}\index{cec}{vati@\textit{vati}} [ki\b lakku]\dots ttu pu\b ncey N @ $\Psi$
\end{enumerate}

\section*{CEC 34}
\subsection*{CEC 34.1 Remarques}

L'\'epigraphe a été relev\'ee dans l'ARE 1918 383 et localis\'ee sur le mur nord du \textit{ma\d n\d dapa} devant la chapelle de Campantar\index{gnl}{Campantar}. Elle date de la vingt-septi\`eme ann\'ee de r\`egne de Tribhuvanacakravartin Kulottu\.ngac\=o\b ladeva\index{cec}{kulottungacoladeva@Kulottu\.ngac\=o\b ladeva}, \og who was pleased to take Madura, \=I\b lam\index{gnl}{Srilanka!Ilam@\=I\b lam} (Ceylon), Karuv\=ur and the crowned head of the P\=a\d n\d dya\fg. \textsc{Mahalingam} (1992: 547, Tj. 2401) identifie ce roi\index{gnl}{roi} comme Kulottu\.nga I\index{gnl}{Kulottu\.nga I} et date le texte de 1097.

La conqu\^ete de Karuv\=ur figurant dans la tr\`es courte introduction du roi\index{gnl}{roi} est cependant attribu\'ee g\'en\'eralement \`a Kulottu\.nga III\index{gnl}{Kulottu\.nga III} (\textsc{Nilakanta Sastri} *2000 [1955]: 377 et 391). De plus, le texte mentionne plusieurs dates: vingt-septi\`eme, vingt-huiti\`eme, vingt-neuvi\`eme et trente-et-uni\`eme ann\'ee de règne de Kulottu\.nga puis la troisi\`eme année de règne de R\=ajar\=aja. Il semble avoir \'et\'e grav\'e en une fois. Nous supposons donc que c'est sous R\=ajar\=aja qu'il a été inscrit. La datation sous R\=ajar\=aja appara\^it \`a la fin et fournit des informations astronomiques compl\`etes qui permettent d'obtenir la date du \textbf{mercredi 13 f\'evrier 1219} gr\^ace au programme \og Pancanga\fg. Ainsi, CEC 34 date, selon nous, du r\`egne de R\=ajar\=aja III\index{gnl}{Rajaraja III@R\=ajar\=aja III} et relate des acquisitions de terres\index{gnl}{terre} effectu\'ees sous Kulottu\.nga III\index{gnl}{Kulottu\.nga III}.

L'inscription se situe sur un \'el\'ement saillant du soubassement de la face nord du \textit{ma\d n\d dapa}, sous CEC 33. Elle comporte quatre lignes qui couvrent vingt-quatre m\`etres et quatre-vingt centimètres. Elle est inachev\'ee et abonde en abr\'eviations de mesures de terrain. Ces derni\`eres sont not\'ees, sans distinction de forme et de nombre, par \og @\fg. Une pierre, au moins, manque et perturbe la continuit\'e du texte. Cet espace est marqu\'e par \og $\Psi$\fg. Le texte pr\'esent\'e ci-dessous est bas\'e sur l'examen de la transcription de l'ASI, de clich\'es personnels et de notre lecture \textit{in situ}.

Structur\'ee selon les dates données, l'\'epigraphe enregistre une liste de terres\index{gnl}{terre} \`a Tirumullaiv\=ayil, hameau de Tiruv\=ali\index{cec}{Tiruvali@Tiruv\=ali} alias Etirilico\b lacaturvedima\.nkalam\index{cec}{caturvedimangalam@Caturvedima\.ngalam!Etirilic\=o\b laccaturvedima\.ngalam}, acquises en tant que \textit{tirun\=amattukk\=a\d ni}\index{cec}{kani@\textit{k\=a\d ni} droit, propriété} pour \=A\d lu\d taiyapi\d l\d laiy\=ar\index{cec}{Alutaiyapillaiyar@\=A\d lu\d taiyapi\d l\d laiy\=ar}.



\subsection*{CEC 34.2 Texte}
\begin{enumerate}
	\item \textbf{svasti \'sr\=\i}\ U \textbf{tribhu}va\b naccakrava\textbf{tti}ka\d l maturaiyumi\b lamu\.nkaruv\=urum\index{cec}{Maturai} p\=a\d n\d ti-ya\b n mu\d tit talaiyum ko\d n\d taru\d li\b na \textbf{\'sr\=\i}kulottu\.nkaco\b la\textbf{de}va\b rku\index{cec}{kulottungacoladeva@Kulottu\.ngac\=o\b ladeva} y[\=a]\d n\d tu 2 10 7 vatu \textbf{r\=aj\=adhar\=a}[\textbf{ja}]va\d lan\=a\d t\d tut tirukka\b lumalattu\index{cec}{Tirukkalumalam@Tirukka\b lumalam} \=a\d lu\d taiyapi\d l\d laiy[\=a]\b rku i\b n\b n\=a\d t\d tut tiruv\=aliy\=a\b na\index{cec}{Tiruvali@Tiruv\=ali} etirilico\b laccatu\textbf{rve}tima\.nkalattup\index{cec}{caturvedimangalam@Caturvedima\.ngalam!Etirilic\=o\b laccaturvedima\.ngalam} pi\d t\=akai\index{cec}{pitakai@\textit{pi\d t\=akai} hameau} tirumullaiv\=ayil ka\d t\d ta\d lai-yil palarpakkalum\index{cec}{pakkal@\textit{pakkal} auprès de} tirun\=amattukk\=a\d niy\=aka\index{cec}{kani@\textit{k\=a\d ni} droit, propriété} vilai ko\d n\d ta\index{cec}{vilai@\textit{vilai} prix!vilai-kol@\textit{vilai-ko\d l} acheter} nila\.nka\d lukku \textbf{pram}[\textbf{\=a}]-\d nap pa\d ti \textbf{v\textsubring{r}\'sci}ka n\=aya\b r\b ru \textbf{p\=urvapak\d sa}ttu \textbf{pra}[\textbf{tha}maiyum \textbf{budha}\b nki\b lamai]-yum pe\b r\b ra mulattu n\=a\d l mullaiv\=ayilu\d taiy\=a\b n (ci\.nkap)pir\=a\b n periy\=a..\b n pakkal\index{cec}{pakkal@\textit{pakkal} auprès de} \textbf{pautra}m\=a\d nikkavati[k]ku\index{cec}{vati@\textit{vati}} ki\b lakku ka\d napuratevav\=ayk\=alukku\index{cec}{vaykkal@\textit{v\=aykk\=al} canal} va[\d ta]kku mu\b n-\b r\=a\.nka\d n\d n\=a\b r\b ru 4 catirattu ki\b laikka\d taiya N 1@kku k\=acu\index{cec}{kacu@\textit{k\=acu} pièces de monnaie} 1000 3 100\d latu mullaiv\=ayilu\d taiy\=a\b n c\=\i r\=amateva\b n \=avimutti\textbf{\'sva}ramu\d taiy\=a\b n pakkal\index{cec}{pakkal@\textit{pakkal} auprès de} ivvatikkuk\index{cec}{vati@\textit{vati}} ki\b lak-ku iv[v]\=ayk\=alukku va\d takku 2 C 4 catirattu N 1@ i\.nke N @ \=aka N @kku k\=acu\index{cec}{kacu@\textit{k\=acu} pièces de monnaie} 1000 3 100 ta\b nu n\=aya\b r\b ru aparapa\textbf{k\d sa}ttu [pa\~nca]miyum n\=aya\b r\b ruk[ki]\b lamaiyum pe\b r\b ra makattu n\=a\d l ma\b n\b na\b n ve\d nk\=a\d ta\b n pakkal\index{cec}{pakkal@\textit{pakkal} auprès de} \textbf{pautra}m\=a\d nikkavatikkuk\index{cec}{vati@\textit{vati}} ki\b lakku ka\d napuratevav\=ayk\=alkku\index{cec}{vaykkal@\textit{v\=aykk\=al} canal} va\d takku 3 C 4 catirattu N @kku k\=a[cu 2] 100 5 10\d latu mullaiv\=ayalu\d taiy\=a\b nayyampukk\=a\b n \textbf{\textsubring{r}\d sa}pate[va]\b num iva\b n tampi civatava\b np pe-rumaka\b num pakkal\index{cec}{pakkal@\textit{pakkal} auprès de} kalika\b n\b rivatikkuk\index{cec}{vati@\textit{vati}} ki\b lakku ivv[\=a]yk\=alukku va\d takku 3 C 1 catirattu N 6 @vil me\b rka\d taiya N 1@kku k\=acu\index{cec}{kacu@\textit{k\=acu} pièces de monnaie} $\Psi$ \b ru pu\textbf{rvva}pa\textbf{k\d sa}ttu \textbf{da\'samiyu}m ti\.nka\d l ki\b lamaiyum pe\b r\b ra \textbf{m\textsubring{r}ga\'s\=\i \d sa}ttu \b n\=a\d l ka\d tala\b npe\b r\b r\=a\b n\=a\b na k\=a\b likka\b r-pakan\=a\d t\=a\b lv\=a\b n pakkal\index{cec}{pakkal@\textit{pakkal} auprès de} \textbf{pautra}m\=a\d nikkavatikkuk\index{cec}{vati@\textit{vati}} ki\b lakku ka\d napurav\=ayk\=alukku\index{cec}{vaykkal@\textit{v\=aykk\=al} canal} va\d takku 2 C 4 catirattu te\b rka\d taiya k\=averiy\=a\b r\b ra\.nkarai c\=u\b lnta i\d tattu N 2 100 k\=acu\index{cec}{kacu@\textit{k\=acu} pièces de monnaie} 2 100 mi\b na n\=aya\b r\b ru p\=u\textbf{rva}pa\textbf{k\d sa}ttu a[\textbf{\d s\d ta}]miyum ti\.nka\d l ki\b lamaiyum pe\b r\b ra \textbf{m\textsubring{r}ga\'si\d sa}ttu ma\b n\b na\b n ka\d tal\=a\b n\=a\b na
	\item po\b n\b nampalak[k\=u]ttan\=a\d t\=a\b lv\=a\b num\index{cec}{ponnampala@Po\b n\b nampalakk\=uttar} ivan tampi ma[ruta]\b n\=a\b na vetava\b na\b n\=a\d t\=a\b lv\=a-\b num ivan tampi tirukka\b lipp\=alai u\d taiy\=a[\b nu]m iva\b n tampi a\b laka\b num pakkal\index{cec}{pakkal@\textit{pakkal} auprès de} ko\d n\d ta ikka\d n\d n\=a\b r\b ru iccatirattu N 1@ k\=acu\index{cec}{kacu@\textit{k\=acu} pièces de monnaie} 8 100 \textbf{\textsubring{r}\d sa}pa \b n\=aya\b r\b ru aparapa\textbf{k\d sa}ttu catu\textbf{rddhi}yum k\=a\textbf{tti}kai n\=a\d l ce\d t\d tapo[ca]kovintapa\d t\d tan pakkal\index{cec}{pakkal@\textit{pakkal} auprès de} k\=a\b rkiy\=a\b nvatikku\index{cec}{vati@\textit{vati}} ki\b lakku mummu\d ti[c]o\b lav\=ayk\=alukku\index{cec}{vaykkal@\textit{v\=aykk\=al} canal} va\d takku 2 C 2 catirattu ka\b liyu\.nka\b likkop-pum N 1@ 3 catirattu N 2@ 3 C 2 [catirattu N 2@ iccati]rattu N 3@ 4 C 2 catirattu N 2@ 3 catirattu N 1@ 4 catirattu N [1 @] 3 catirattu N [1@] 6 C 2 catirattu N 1@ 7 C 2 catirattu N 1@ 8 C 2 catirattu N 1@ \=aka N 16@ pauttiram\=a\d nikkavatikku\index{cec}{vati@\textit{vati}} ki\b lakku ivv\=ay[k]k\=alukku va\d takku 5 C 1 catirattu N 3@ 2 catirattu N 1/2@ 6 C 1 catirattu N 1@ 2 catirattu N 3@ 7 C 2 catirattu N 4 1/2@ \=aka N 12@ \=aka N 2 10 8@ \=aka k\=acu\index{cec}{kacu@\textit{k\=acu} pièces de monnaie} 7 100 U 2 10 8[vatu] ka\b rka\d taka n\=aya\b r\b ru p\=u\textbf{rvvapak\d sa}ttu \textbf{da\'sa}miyum ti\.nka\d l ki\b lamaiyum pe\b r\b ra vi\textbf{\'s\=a}kattu n\=a\d l kommaip\=akkamu\d taiy\=a\b n k\=akku\b n\=ayakka\b n vitivi\d ta\.nka\b n pakkal\index{cec}{pakkal@\textit{pakkal} auprès de} k\=a\b rkiy\=a\b n vatikku\index{cec}{vati@\textit{vati}} ki\b lakku ka\d napuratevav\=ayk\=alukku\index{cec}{vaykkal@\textit{v\=aykk\=al} canal} va\d takku 7 C 4 catirattu ki\b laikka\d taiya N 1@ ivvatikku\index{cec}{vati@\textit{vati}} ki\b lakku tillaiy\=a\d liv\=aykk\=alukku\index{cec}{Tillaiy\=a\d li}\index{cec}{vaykkal@\textit{v\=aykk\=al} canal} va\d takku 2 C 3 catirattu me\b rka\d taiya N 1@ nikki ita\b nki\b lakku N 2 1/2@ 3 C 3 catirattu N 6@ itil me\b rka\d taiya N 2@ ki\b lakka\d taiya N 2@ \=aka N 8@kku $\Psi$ ya\b r\b ru aparapa\textbf{k\d sa}ttu catu\textbf{rda\'si}yum \textbf{budha}n ki\b lamaiyum pe\b r\b ra u\textbf{tti}r\=a\d tattu n\=a\d l ki\b laku\d taiy\=a\b n \=a\d n\d ta nampi\index{cec}{Nampi} uyyakko\d n-\d t\=a\b n\index{cec}{uyyakkontan@Uyyakko\d n\d t\=a\b n} pakkal\index{cec}{pakkal@\textit{pakkal} auprès de} \textbf{gau}tava\b n tirukka\b lipp\=alai u\d taiy\=a\b n kecava.\b n matur\=antaka \textbf{brahmam\=a}r\=ayarum \textbf{gau}tava\b n tirukka\b lipp\=alai u\d taiy\=a\b n \=a\d n\d t\=ar\=a\b navillava\b n \textbf{brahmam\=a}r\=ayarum \textbf{gau}tava\b n tirukka\b lipp\=alai u\d taiy\=a\b n tirucci\b r\b rampalanampiyum\index{cec}{Tirucirrampalanampi@Tirucci\b ra\b rampalanampi} u\d l\d li\d t\d t\=ar pakkal\index{cec}{pakkal@\textit{pakkal} auprès de} \textbf{s\=a}ka\d naitevati a\b nantapa\d t\d ta\b n peril \textbf{\=anya}n\=amakira\d nattu \textbf{d\=ana-pram\=a\d na}m\index{cec}{piramanam@\textit{pram\=a\d nam} document}
	\item e\b lutivittuk ko\d n\d tu e\b n\b nut\=ay n\=a\b n a\b nupavittu varuki\b ra Nm\=ay n\=a\b n vi\b r\b ru ku\d tutta Nm ivvatikku\index{cec}{vati@\textit{vati}} ki\b lakku ivv\=ayk\=alukku\index{cec}{vaykkal@\textit{v\=aykk\=al} canal} va\d takku 3 C 4 catirattu N 6@m 4 C 4 catirattu N [6]@ 5 C 3 catirattu me\b rka\d taya N 4@m 4 catirattu me\b rka\d taya N 3@ 6 C 4 catirattu N 6@ 7 C 4 catirattu me\b rka\d taiya N 2@ 8 [C 1] catirattu ki\b lakka\d taya N 4@ 4 catirattu me\b rka\d taya N 3@ \=aka N 3 10 4@kku k\=a(cu) 5 1000 [U*] 2 10 9vatu \textbf{si\d mha} n\=aya\b r\b ru aparapa\textbf{k\d sa}ttu \textbf{saptami}yum ca\b nik ki\b lamaiyum pe\b r\b ra ro\textbf{\'sa}\d ni [n\=a\d l to.......] \textbf{\'sr\=\i k\textsubring{r}\d s\d na}pa\d t\d ta\b n pakkal\index{cec}{pakkal@\textit{pakkal} auprès de} mullaiv\=aylu\d taiy\=a\b n cir\=amateva\b n utaya\~nceyt\=a\b n\=ana\index{cec}{utaiyanceytan@Utaiya\~nceyt\=a\b n} tiru\~n\=a\b nacampanta\index{cec}{tirunana@Tiru\~n\=a\b nacampanta\b n} a\d lakaikko\b n-peril vilai ko\d n\d ta\index{cec}{vilai@\textit{vilai} prix!vilai-kol@\textit{vilai-ko\d l} acheter} ivvatikku\index{cec}{vati@\textit{vati}} ki\b lakku ivv\=ayk\=alukku\index{cec}{vaykkal@\textit{v\=aykk\=al} canal} va\d takku 2 C 3 catirattu me\b rka\d taiya N 3@ itil me\b rka\d taiya N [1@]kku (k\=acu\index{cec}{kacu@\textit{k\=acu} pièces de monnaie} 2 100) [\textbf{mina}\b n\=aya\b r\b ru a]parapa\textbf{k\d sa}ttu \textbf{tri}tiyaiyum viy\=a\b la[k] ki\b lamaiyum pe\b r\b ra makattu n\=a\d l \textbf{s\=a}ka\d nai-tevati a\b nantapa\d t\d ta\b n pakkal\index{cec}{pakkal@\textit{pakkal} auprès de} ivvatikku\index{cec}{vati@\textit{vati}} ki\b lakku ivv\=ayk\=alukku\index{cec}{vaykkal@\textit{v\=aykk\=al} canal} va\d takku 3 C 3 catirattu me\b rka\d taiya \=a\d lu\d taiyapi\d l\d laiy\=ar tirunantava\b nam N 2@ nikki ita\b n ki\b lakku ve\d l\d l\=a\b n pa\b r\b ru N 1@ itil va\d takka\d taiya kollai N 5@kku k\=acu\index{cec}{kacu@\textit{k\=acu} pièces de monnaie} 100 ta[\b nu] n\=aya\b r\b ru aparapa\textbf{k\d sa}ttu \textbf{trayoda\'siyu}m \textbf{budhan} ki\b lamaiyum pe\b r\b ra \textbf{vi\'s\=a}kattu n\=a\d l \textbf{\'sr\=\i }kuntavacceri umiy\=ur v\=amapai\d t\d ta\b num iva\b n tampi ke\textbf{\'sa}pa\d t\d ta\b num umiy\=ur t\=amot\=apa\d t\d ta\b n maka\b n \=a\d lappi\b rant\=a\b npa\d t\d ta\b num pakkal\index{cec}{pakkal@\textit{pakkal} auprès de} ivvatikku\index{cec}{vati@\textit{vati}} ki\b lakku ka\d na-puratevav\=ayk\=alukku\index{cec}{vaykkal@\textit{v\=aykk\=al} canal} va\d takku 7 C 4 catira $\Psi$ me\b rka\d taiya N 3@ nikki ita\b nki\b lakku e\.nka\d l pit\=akka\d l ap\=avattu e\.nka\d lut\=ay vi\b r\b ru ku\d tutta N @kku k\=acu\index{cec}{kacu@\textit{k\=acu} pièces de monnaie} 100 2 10 \textbf{\textsubring{r}\d sapa} n\=aya\b r\b ru p\=u\textbf{rva}pa\textbf{k\d sa}ttu \textbf{pa\~ncada\'si}(yum ti\.n)[ka\d l ki\b lamaiyu]m pe\b r\b ra mulattu n\=a\d l ivvatikku\index{cec}{vati@\textit{vati}} ki\b lakku tillaiy\=a[\d li]v\=ayk\=alukku\index{cec}{Tillaiy\=a\d li}\index{cec}{vaykkal@\textit{v\=aykk\=al} canal} va\d takku 1 C 4 catirattu N6@kku k\=acu\index{cec}{kacu@\textit{k\=acu} pièces de monnaie}\footnote{Un blanc pouvant contenir environ cinq \textit{ak\d sara} pr\'ec\`ede le chiffre 30.} 3 10 lu\footnote{La ligne s'arr\^ete brutalement alors qu'il reste de la place pour, approximativement, une soixantaine de graph\`emes.}
	\item iva\b n maka\b n tiruve\d nk\=a\d tuteva\b num\index{cec}{Tiruvenkatut@Tiruve\d nk\=a\d tuteva\b n} ivantampi \=ata\b n\=uru\d taiy\=a\b n tillaiventa\b n\index{cec}{Tillaiventa\b n} ta\b ni-n\=ayaka\d nai mutuka\d n\d n\=akkako\d n\d tu pau\textbf{tra}m\=a\d nikkavatikku\index{cec}{vati@\textit{vati}} ki\b lakku ka\d napurate-vav\=ayk\=alukku\index{cec}{vaykkal@\textit{v\=aykk\=al} canal} va\d takku 1 C 4 catirattu me\b rka\d taiya u\d taiy\=ar tirumullaiv\=ayilu-\d taiy\=ar \=u\b rki\b l i\b raiyilum\index{cec}{iraiyili@\textit{i\b raiyili} non imposable} vilaiko\d n\d tu\index{cec}{vilai@\textit{vilai} prix!vilai-kol@\textit{vilai-ko\d l} acheter} a\b nupa[vi]ttu varuki\b ra N 1@kku k\=acu\index{cec}{kacu@\textit{k\=acu} pièces de monnaie} 1000 5 100 \textbf{si}[\textbf{\d m}]\textbf{ha} n\=aya\b r\b ru aparapa\textbf{k\d sa}ttu \textbf{ek\=ada\'siyu}m ti\.nka\d l ki\b lamaiyum pe\b r\b ra pu\d narp\=ucattu n\=a\d l peru\.nkoma\.nkala\.nki\b l\=a\b n [n\=ar\=aya\d nateva\b n\=a\b na]\index{cec}{narayanan@N\=ar\=aya\d nateva\b n}\footnote{Conjecture \'etablie \`a partir du vendeur de la 31\up{e} ann\'ee.} [ku]lottu\.n-kaco\b lap pallavaraiya\b n\index{cec}{Pallavar\=aya\b n} pakkal\index{cec}{pakkal@\textit{pakkal} auprès de} ivvatikku\index{cec}{vati@\textit{vati}} ki\b lakku mummu\d tico\b lav\=aykk\=alukku\index{cec}{vaykkal@\textit{v\=aykk\=al} canal} va-\d takku 8 C 2 catirattu ki\b laikka\d taiya N 2@ vil te\b rka\d taiya N 1@m 8 C 4 catirattu N 6@vil te\b rka\d taiya N 4@m \=a[ka] N 5@kku k\=acu\index{cec}{kacu@\textit{k\=acu} pièces de monnaie} 1000 kan\b ni n\=aya\b r\b ru [p\=u\textbf{rva}]pa\textbf{k\d sa}ttu ........ [ki]\b lamaiyum p[e]\b r\b ra avi\d t\d tattu n\=a\d l ilakkamu\d taiy\=a\b n am-palava\b n pakkal\index{cec}{pakkal@\textit{pakkal} auprès de} tiru\~n\=a\b nacampanta\index{cec}{tirunana@Tiru\~n\=a\b nacampanta\b n} a\d lakaiyko\b n peril ko\d n\d ta k\=a\b rkiy\=a\b nvatikkuk\index{cec}{vati@\textit{vati}} ki\b lakku tillaiy\=a\d liv\=ayk\=alukku\index{cec}{Tillaiy\=a\d li}\index{cec}{vaykkal@\textit{v\=aykk\=al} canal} [va]\d takku 2 C 4 catirattu ki\b lakka\d taiya N 6@kku k\=acu\index{cec}{kacu@\textit{k\=acu} pièces de monnaie} 1000 7 100 U 3 10 1vatu ka\b n\b ni \b n\=aya\b r\b ru p\=u\textbf{rva}pa\textbf{k\d sa}ttu \textbf{ek\=ada\'siyu}m \textbf{\'sa}nik ki\b lamaiyum pe\b r\b ra avi\d t\d tattu \b n\=a\d l peru\.nkoma\.nkala\.nki\b l\=a\b n \b n\=ar\=aya\d nateva-\b n\=a\b na\index{cec}{narayanan@N\=ar\=aya\d nateva\b n} kulottu\.nkaco\b lap\index{cec}{kulottungacola@Kulottu\.ngac\=o\b la} pallavaraiya\b n\index{cec}{Pallavar\=aya\b n} pakkal\index{cec}{pakkal@\textit{pakkal} auprès de} u\d taiy\=ar tirumullaiv\=ayi[lu]\d taiy\=ar koyilil\index{cec}{koyil@\textit{k\=oyil} temple} iva\b n vilai ko\d n\d ta\index{cec}{vilai@\textit{vilai} prix!vilai-kol@\textit{vilai-ko\d l} acheter} nilam\=ayk\index{cec}{nilam@\textit{nilam} terre} ko\d n\d ta pau\textbf{tra}m\=a\d nikkavatikku\index{cec}{vati@\textit{vati}} ki\b lakku ka\d na-puratevav\=aykk\=alukku\index{cec}{vaykkal@\textit{v\=aykk\=al} canal} va\d takku 4 C 4 catirattu N $\Psi$ 1000 U \textbf{tribhu}va\b nacca-\textbf{kra}vattika\d l \textbf{\'sr\=\i}r\=a\textbf{ja}r\=a\textbf{jadeva}\b rku y\=a\d n\d tu 3vatu kumpa \b n\=aya\b r\b ru aparapa\textbf{k\d sa}ttu \textbf{dv\=ada\'siyu}m puta\b n ki\b lamaiyum pe\b r\b ra uttir\=a\d tattu n\=a\d l [matur\=anta]kaccerik ku\d n\d t\=ur v\=\i \b r\b rirunt\=a\b npa\d t\d ta
\end{enumerate}

\subsection*{CEC 34.3 Traduction}
Que la prosp\'erit\'e soit! En la 27\up{e} ann\'ee [de r\`egne] de \'Sr\=\i kulottu\.ngadeva\index{cec}{kulottungacoladeva@Kulottu\.ngac\=o\b ladeva} qui a conquis la t\^ete couronn\'ee du P\=a\d n\d dya, Karuv\=ur, I\b lam et Maturai\index{cec}{Maturai}, empereur des trois mondes.

Selon les documents [voici] les terres\index{gnl}{terre} qui ont \'et\'e achet\'ees comme \textit{tirun\=amattuk-k\=a\d ni}\index{cec}{kani@\textit{k\=a\d ni} droit, propriété} pour \=A\d lu\d taiyapi\d l\d laiy\=ar\index{cec}{Alutaiyapillaiyar@\=A\d lu\d taiyapi\d l\d laiy\=ar} de Tirukka\b lumalam\index{cec}{Tirukkalumalam@Tirukka\b lumalam} dans le R\=aj\=adhir\=ajava\d lan\=a\d tu\index{cec}{Rajadhirajavala@R\=aj\=adhir\=ajava\d lan\=a\d tu}, aupr\`es de plusieurs [personnes] dans le \textit{ka\d t\d ta\d lai} de Tirumullaiv\=ayil, hameau de Tiruv\=ali\index{cec}{Tiruvali@Tiruv\=ali} alias Etirico\b laccaturvvetima\.nkalam de ce Pays:
\begin{enumerate}
\item Le mois de \textit{V\textsubring{r}\'scika}, le premier jour de la quinzaine claire, mercredi, dans le [\textit{nak\d satra}] \textit{M\=ula}:\\
- [la terre\index{gnl}{terre} de] @ [achet\'ee] aupr\`es de Ci\.nkappir\=a\b n un propriétaire [terrien] de Mullaiv\=ayil\index{gnl}{Mullaivayil@Mullaiv\=ayil} pour 1300 \textit{k\=acu}\index{cec}{kacu@\textit{k\=acu} pièces de monnaie}  \`a l'est de la \textit{vati}\index{cec}{vati@\textit{vati}} Pautram\=a\d nikkam, au nord du canal Ka\d napuratevar, dans le 4\up{e} carr\'e du troisi\`eme canalicule, terre\index{gnl}{terre} qui atteint l'est.\\
- [la terre\index{gnl}{terre} de] @ [achet\'ee] aupr\`es d'\=Avimutti\'svaram C\=\i r\=amateva\b n un propriétaire [terrien] de Mullaiv\=ayil\index{gnl}{Mullaivayil@Mullaiv\=ayil} pour 1300 \textit{k\=acu}\index{cec}{kacu@\textit{k\=acu} pièces de monnaie} \`a l'est de cette \textit{vati}\index{cec}{vati@\textit{vati}}, au nord de ce canal, dans le 4\up{e} carr\'e du 2\up{e} canalicule.

\item Le mois de \textit{Ta\b nu}, le cinqui\`eme jour de la quinzaine sombre, dimanche, dans le [\textit{nak\d satra}] \textit{Makam}:\\
- [la terre\index{gnl}{terre} de] @ [achet\'ee] aupr\`es de Ma\b n\b na\b n Ve\d nk\=a\d ta\b n pour 250 \textit{k\=acu}\index{cec}{kacu@\textit{k\=acu} pièces de monnaie} \`a l'est de la \textit{vati} \index{cec}{vati@\textit{vati}}Pautram\=a\d nikkam, au nord du canal Ka\d napuratevar, dans le 4\up{e} carr\'e du 3\up{e} canalicule.\\
- [la terre\index{gnl}{terre} de] @ [achet\'ee] aupr\`es de \d R\d sapateva\b n Pa\d tampukk\=a\b n un propriétaire [terrien] de Mullaiv\=ayil\index{gnl}{Mullaivayil@Mullaiv\=ayil} et de son fr\`ere cadet Civatava\b npperum\=a\b n, pour \dots\ \textit{k\=acu}\index{cec}{kacu@\textit{k\=acu} pièces de monnaie}, \`a l'est de la \textit{vati}\index{cec}{vati@\textit{vati}} Kalika\b n\b ri et au nord de ce canal, dans le 1\up{er} carr\'e du 3\up{e} canalicule, terre\index{gnl}{terre} qui atteint l'ouest.

\item \dots\ le dixi\`eme jour de la quinzaine claire, lundi, dans le [\textit{nak\d satra}] \textit{M\textsubring{r}ga\'sir\d sa} [la terre\index{gnl}{terre} achet\'ee] aupr\`es de K\=a\b likka\b rpakan\=a\d t\=a\b lv\=a\b n alias Ka\d tala\b npe\b r\b r\=a\b n, pour 200 \textit{k\=acu}\index{cec}{kacu@\textit{k\=acu} pièces de monnaie}, \`a l'est de la \textit{vati}\index{cec}{vati@\textit{vati}} Pautram\=a\d nikkam, au nord du canal Ka\d napuratevar, dans le 4\up{e} carr\'e du 2\up{e} canalicule, terre\index{gnl}{terre} qui atteint le sud et qu'entoure le rivi\`ere K\=averi.

\item Le mois de \textit{Mi\b na}, le huiti\`eme jour de la quinzaine claire, lundi, dans le [\textit{nak\d satra}] \textit{M\textsubring{r}ga\'si\d sa} [la terre\index{gnl}{terre} de] @ [achet\'ee] aupr\`es de Ma\b n\b na\b n Ka\d tal\=a\b n alias Po\b n\b nampalakk\=uttan\index{cec}{ponnampala@Po\b n\b nampalakk\=uttar} N\=a\d t\=a\b lv\=a\b n, de son fr\`ere cadet Maruta\b n alias Vetava\b nan N\=a\d t\=a\b lv\=a\b n, de son fr\`ere cadet Tirukka\b lipp\=alai U\d taiy\=a\b n et de son fr\`ere cadet A\b laka\b n, pour 800 \textit{k\=acu}\index{cec}{kacu@\textit{k\=acu} pièces de monnaie}, dans ce carr\'e de ce canalicule.

\item Le mois de \textit{\d R\d sapa}, le quatri\`eme jour de la quinzaine sombre, dans le [\textit{nak\d satra}] \textit{K\=attikai} [la terre\index{gnl}{terre} achet\'ee] aupr\`es de Ce\d t\d tapocakovintapa\d t\d tan, \`a l'est de la \textit{vati}\index{cec}{vati@\textit{vati}} K\=a\b rkiy\=a\b n, au nord du canal Mummu\d tico\b la --- une terre\index{gnl}{terre} de @ dans le 2\up{e} carr\'e du 2\up{e} canalicule, et le Ka\b liyu\.nka\b likkoppu,
dans le 3\up{e} carr\'e une terre\index{gnl}{terre} de @,
dans le 2\up{e} carr\'e du 3\up{e} canalicule une terre\index{gnl}{terre} de @,
dans ce m\^eme carr\'e une terre\index{gnl}{terre} de @,
dans le 2\up{e} carr\'e du 4\up{e} canalicule une terre\index{gnl}{terre} de @,
dans le 3\up{e} carr\'e une terre\index{gnl}{terre} de @,
dans le 4\up{e} carr\'e, dans le 3\up{e} carr\'e du 1\up{er} canalicule une terre\index{gnl}{terre} de @,
dans le 2\up{e} carr\'e du 6\up{e} canalicule une terre\index{gnl}{terre} de @,
dans le 2\up{e} carr\'e du 7\up{e} canalicule une terre\index{gnl}{terre} de @,
dans le 2\up{e} carr\'e du 8\up{e} canalicule une terre\index{gnl}{terre} de @ --- soit [au total] une terre\index{gnl}{terre} de 16@; [plus]
\`a l'est de la \textit{vati}\index{cec}{vati@\textit{vati}} Ve\d lattiram\=a\d nikkam, au nord de ce canal --- dans le 1\up{e} carr\'e du 5\up{e} canalicule une terre\index{gnl}{terre} de @,
dans le 2\up{e} carr\'e une terre\index{gnl}{terre} de @,
dans le 1\up{er} carr\'e du 6\up{e} canalicule une terre\index{gnl}{terre} de @,
dans le 2\up{e} carr\'e une terre\index{gnl}{terre} de @,
dans le 2\up{e} carr\'e du 7\up{e} canalicule une terre\index{gnl}{terre} de @ ---
 soit [au total] une terre\index{gnl}{terre} de 12@, soit [au total final] une terre\index{gnl}{terre} de 28@ pour 728 \textit{k\=acu}\index{cec}{kacu@\textit{k\=acu} pièces de monnaie}.

\item Le mois de \textit{Ka\b rka\d taka}, le dixi\`eme jour de la quinzaine claire, lundi, dans le [\textit{nak\d satra}] \textit{Vi\'s\=aka}, [les terres\index{gnl}{terre} de] @ [achet\'ees] aupr\`es de Vitivi\d ta\.nka\b n K\=akku-\b nuyakka\b n un propriétaire [terrien] de Kompaip\=akkam, \`a l'est de la \textit{vati}\index{cec}{vati@\textit{vati}} K\=a\b rki-y\=a\b n, au nord du canal Ka\d napurateva\b n, dans le 4\up{e} carr\'e du 7\up{e} canalicule, terre\index{gnl}{terre} qui atteint l'est ; \`a l'est de cette \textit{vati}\index{cec}{vati@\textit{vati}}, au nord du canal Tillaiy\=a\d li, dans le 3\up{e} carr\'e du 2\up{e} canalicule, ayant d\'eduit cette terre\index{gnl}{terre} de @ qui atteint l'ouest ; \`a l'est de ceci la terre\index{gnl}{terre} de @ ; dans le 3\up{e} carr\'e du 3\up{e} canalicule une terre\index{gnl}{terre} de @, de ceci la terre\index{gnl}{terre} de @ qui atteint l'ouest et la terre\index{gnl}{terre} de @ qui atteint l'est, soit [au total] une terre\index{gnl}{terre} de @ \dots

\item \dots\ le quatorzi\`eme jour de la quinzaine sombre, mercredi, dans le [\textit{nak\d satra}] \textit{Uttir\=a\d tam},
Ki\b laku\d taiy\=a\b n \=A\d n\d ta Nampi Uyyakko\d n\d t\=a\b n\index{cec}{uyyakkontan@Uyyakko\d n\d t\=a\b n} ---
aupr\`es [des t\'emoins?] Gautava\b n \dots\ Kecava\b n un propriétaire [terrien] de Tirukka\b lipp\=alai,
Mavar\=an-taka Brahmamar\=ayar,
Gautava\b n \=A\d n\d t\=ar\=a\b navillava\b n Brahmam\=ar\=ayar un propriétaire [terrien] de Tirukka\b lipp\=alai et
Gautava\b n Tirucci\b r\b rampala Nampi\index{cec}{Nampi} un propriétaire [terrien] de Tirukka\b lipp\=alai ---
ayant \'ecrit le document de don\index{gnl}{don} \textit{\=anyan\=amakira\d anam} au nom de Srava\d naiteva\b n A\b nantapa\d t\d ta\b n,
la terre\index{gnl}{terre} dont je jouis \'etant de @, [la terre\index{gnl}{terre} que] je vends [est]: \`a l'est de cette \textit{vati}\index{cec}{vati@\textit{vati}}, au nord de ce canal, dans le 4\up{e} carr\'e du 3\up{e} canalicule une terre\index{gnl}{terre} de 6@; dans le 4\up{e} carr\'e du 4\up{e} canalicule une terre\index{gnl}{terre} de 6@; dans le 3\up{e} carr\'e du 5\up{e} canalicule une terre\index{gnl}{terre} de 4@; dans le 4\up{e} carr\'e une terre\index{gnl}{terre} de 3@ qui atteint l'ouest, dans le 4\up{e} carr\'e du 6\up{e} canalicule une terre\index{gnl}{terre} de 6@; dans le 4\up{e} carr\'e du 7\up{e} canalicule une terre\index{gnl}{terre} de @ qui atteint l'ouest; dans le 1\up{er} carr\'e du 8\up{e} canalicule une terre\index{gnl}{terre} de 4@ qui atteint l'est et dans le 4\up{e} carr\'e une terre\index{gnl}{terre} de 3@ qui atteint l'ouest, soit [au total] une terre\index{gnl}{terre} de 34@ pour 5000 \textit{k\=acu}\index{cec}{kacu@\textit{k\=acu} pièces de monnaie}.

\item Le mois de \textit{Si\d mha}, le septi\`eme jour de la quinzaine sombre, samedi, dans le [\textit{nak\d satra}] \textit{Ro\'sa\d ni}, aupr\`es de \'Sr\=\i k\textsubring{r}\d s\d napa\d t\d ta\b n \dots , la [terre\index{gnl}{terre}] achet\'ee au nom de Cir\=amateva\b n Utaya\~nceyt\=a\b n\index{cec}{utaiyanceytan@Utaiya\~nceyt\=a\b n} alias Tiru\~n\=a\b nacampanta\index{cec}{tirunana@Tiru\~n\=a\b nacampanta\b n} A\d lakaikko\b n un propriétaire [terrien] de Mullaiv\=ayil\index{gnl}{Mullaivayil@Mullaiv\=ayil}, \`a l'est de cette \textit{vati}\index{cec}{vati@\textit{vati}}, au nord de ce canal, dans le 3\up{e} carr\'e du 2\up{e} canalicule une terre\index{gnl}{terre} de 3@ qui atteint l'ouest, de ceci pour une terre\index{gnl}{terre} de 1@ qui atteint l'ouest pour [200] \textit{k\=acu}\index{cec}{kacu@\textit{k\=acu} pièces de monnaie}.

\item Le mois de \textit{Mina}, le troisi\`eme jour de la quinzaine sombre, jeudi, dans le [\textit{nak\d satra}] \textit{Makam}, [la terre\index{gnl}{terre} achet\'ee] pour 100 \textit{k\=acu}\index{cec}{kacu@\textit{k\=acu} pièces de monnaie} aupr\`es de S\=aka\b naitevati A\b nantapa\d t\d ta\b n, \`a l'est de cette \textit{vati}\index{cec}{vati@\textit{vati}}, au nord de ce canal, dans le 3\up{e} carr\'e du 3\up{e} canalicule, ayant d\'eduit la terre\index{gnl}{terre} de 1@ du jardin d'\=A\d lu\d taiyapi\d l\d laiy\=ar\index{cec}{Alutaiyapillaiyar@\=A\d lu\d taiyapi\d l\d laiy\=ar} qui atteint l'ouest, \`a l'est de ceci dans le \textit{pa\b r\b ru} des Ve\d l\d l\=a\b n, le verger de 5@ qui atteint le nord.

\item Le mois de \textit{Ta\b nu}, le treizi\`eme jour de la quinzaine sombre, mercredi, dans le [\textit{nak\d satra}] \textit{Vi\'s\=aka}, [la terre\index{gnl}{terre} achet\'ee] pour 120 \textit{k\=acu}\index{cec}{kacu@\textit{k\=acu} pièces de monnaie} aupr\`es de \'Sr\=\i kuvacceri Umiy\=ur V\=amapai\d t\d ta\b n, de son fr\`ere cadet Ke\'sapa\d t\d ta\b n et d'\=A\d lappi\b rant\=a\b npa\d t-\d ta\b n, fils de T\=amot\=apa\d t\d ta\b n d'Umiy\=ur, \`a l'est de cette \textit{vati}\index{cec}{vati@\textit{vati}}, au nord du canal de Ka\d napuratevar, dans le 2\up{e} carr\'e du 7\up{e} canalicule \dots\ ayant d\'eduit la terre\index{gnl}{terre} de 3@ qui atteint l'ouest, \`a l'est de ceci la terre\index{gnl}{terre} de @ [qui est la terre\index{gnl}{terre}] laiss\'ee par nos anc\^etres \textit{ap\=avattu} pour nous.

\item Le mois de \textit{\d R\d sabha}, le quinzi\`eme jour de la quinzaine claire, lundi, dans le [\textit{nak\d satra}] de \textit{Mulam}, \`a l'est de cette \textit{vati}\index{cec}{vati@\textit{vati}}, au nord du canal Tillaiy\=a\b li, dans le 4\up{e} carr\'e du 1\up{er} canalicule, la terre\index{gnl}{terre} de 6@ pour \dots\ \textit{k\=acu}\index{cec}{kacu@\textit{k\=acu} pièces de monnaie}

\item \dots\ son fils Tiruve\d nk\=a\d tuteva\b n\index{cec}{Tiruvenkatut@Tiruve\d nk\=a\d tuteva\b n}, son fr\`ere cadet Tillaiventa\b n un propriétaire [terrien] d'\=Ata\b n\=ur, avec pour gardien Ta\b nin\=ayaka\b n, \`a l'est de la \textit{vati}\index{cec}{vati@\textit{vati}} Pautram\=a\d nikkam, au nord du canal Ka\d napuratevar, dans le 4\up{e} carr\'e du 1\up{er} canalicule, la terre\index{gnl}{terre} qui atteint l'ouest, achet\'ee non imposable et jouie du village d'U\d taiy\=ar seigneur de Tirumullaiv\=ayil de 1@ pour 1500 \textit{k\=acu}\index{cec}{kacu@\textit{k\=acu} pièces de monnaie}.

\item Le mois de \textit{Si\d mha}, le onzi\`eme jour de la quinzaine sombre, lundi, dans le [\textit{nak\d satra}] \textit{Pu\b narp\=ucam}, [la terre\index{gnl}{terre} achet\'ee] aupr\`es de N\=ar\=aya\d nateva\b n\index{cec}{narayanan@N\=ar\=aya\d nateva\b n} \textit{ki\b l\=a\b n} de Peru\.nkoma\.nkalam alias Kulottu\.nkaco\b lap\index{cec}{kulottungacola@Kulottu\.ngac\=o\b la} Pallavaraiya\b n\index{cec}{Pallavar\=aya\b n}, \`a l'est de cette \textit{vati}\index{cec}{vati@\textit{vati}}, au nord du canal Mummu\d tico\b la, dans le 2\up{e} carr\'e du 8\up{e} canalicule, une terre\index{gnl}{terre} de 1@ qui atteint le sud de la terre\index{gnl}{terre} de 2@ qui atteint l'est, [puis] dans le 4\up{e} carr\'e du 8\up{e} canalicule une terre\index{gnl}{terre} de 4@ qui atteint le sud de la terre\index{gnl}{terre} de 6@, soit [au total] une terre\index{gnl}{terre} de 5@ pour 1000 \textit{k\=acu}\index{cec}{kacu@\textit{k\=acu} pièces de monnaie}.

\item Le mois de \textit{Ka\b n\b ni}, \dots\ quinzaine claire, dans le [\textit{nak\d satra}] d'\textit{Avi\d t\d tam}, [la terre\index{gnl}{terre} achet\'ee] \`a 1730 \textit{k\=acu}\index{cec}{kacu@\textit{k\=acu} pièces de monnaie} aupr\`es de Ilakkamu\d taiy\=a\b n Ampalaca\b n pour la terre\index{gnl}{terre} de 6@ qui est au nom de Tiru\~n\=a\b nacampanta\index{cec}{tirunana@Tiru\~n\=a\b nacampanta\b n} A\d lakaikko\b n, \`a l'est de la \textit{vati}\index{cec}{vati@\textit{vati}} K\=akkiy\=a\b n, au nord du canal Tillaiy\=a\b li, dans le 4\up{e} carr\'e du 2\up{e} canalicule.

\item Le mois de \textit{Ka\b n\b ni}, le onzi\`eme jour de la quinzaine claire, samedi, dans le [\textit{nak\d satra}] d'\textit{Avi\d t\d tam}, [la terre\index{gnl}{terre} achet\'ee] \`a 1000 (\textit{k\=acu}\index{cec}{kacu@\textit{k\=acu} pièces de monnaie})  aupr\`es de N\=ar\=aya\d nateva\b n\index{cec}{narayanan@N\=ar\=aya\d nateva\b n} \textit{ki\b l\=a\b n} de Peru\.nkoma\.nkalam alias Kulottu\.nkaco\b lap\index{cec}{kulottungacola@Kulottu\.ngac\=o\b la} Pallavaraiya\b n\index{cec}{Pallavar\=aya\b n}, [terre\index{gnl}{terre}] qu'il a achet\'ee dans le temple\index{gnl}{temple} du seigneur de U\d taiy\=ar Tirumullaiv\=ayil, \`a l'est de la \textit{vati}\index{cec}{vati@\textit{vati}} Pautram\=a\d nikkam, au nord du canal Ka\d napuratevar, dans le 4\up{e} carr\'e du 4\up{e} canalicule \dots

\item En la 3\up{e} ann\'ee [de r\`egne] de \'Sr\=\i r\=ajar\=ajadeva, empereur des trois mondes, le mois de \textit{Kumpa}, le douzi\`eme jour de la quinzaine sombre, mercredi, dans le [\textit{nak\d satra}] d'\textit{Uttir\=a\d tam}, V\=\i \b r\b rirunt\=a\b npa\d t\d ta\b n de Matur\=antakakeccerikku\d n\d t\=ur.
\end{enumerate}

\section*{C. Enceinte}
\section*{CEC 35}
\subsection*{CEC 35.1 Remarques}

L'\'epigraphe a été relev\'ee dans l'ARE 1918 388 et localis\'ee sur le mur sud de l'enceinte de la chapelle de Campantar\index{gnl}{Campantar}. Elle date de la deuxi\`eme ann\'ee de r\`egne de Tribhuvanacakravartin R\=ajar\=ajadeva que \textsc{Mahalingam} (1992: 551, Tj. 2420) sugg\`ere d'identifier comme R\=ajar\=aja III\index{gnl}{Rajaraja III@R\=ajar\=aja III} en proposant la date de 1218. Il s'agit en r\'ealit\'e de la troisi\`eme année de règne de R\=ajar\=aja III. Ainsi CEC 35 semble dater de \textbf{1219}.

L'inscription se trouve sur deux pierres align\'ees sur la face sud de l'enceinte de la chapelle de Campantar\index{gnl}{Campantar}. Ces pierres ont subi un d\'eplacement lors de la reconstruction de l'enceinte car la lecture devrait se faire de haut en bas. La fin manque. L'\'edition que nous en pr\'esentons est fond\'ee sur l'examen de la transcription de l'ASI, de nos clich\'es et de la lecture \textit{in situ}.

Le texte enregistre un don\index{gnl}{don} d'argent pour r\'eparer l'enceinte du temple\index{gnl}{temple} de Campantar\index{gnl}{Campantar} par un certain \=Aramp\=u\d n\d t\=a\b n.

\subsection*{CEC 35.2 Texte}
\begin{enumerate}
	\item \textbf{svasti \'sr\=\i}\ tiripuva\b naccakkaravattika\d l\index{cec}{Tribhuvanacakravarti}
	\item \textbf{\'sr\=\i}ir\=a\textbf{ja}r\=a\textbf{ja}tevarkku y\=a\d n\d tu 3 \=avatu
	\item \d l 2 100 10 9l ir\=a\textbf{j\=adhi}r\=a\textbf{ja}va\d la\b n\=a\d t\d tut tirukka\b luma
	\item {[la]\b n\=a\d t\d tut\index{cec}{Tirukka\b lumalan\=a\d tu} tirukka\b lumalattu\index{cec}{Tirukkalumalam@Tirukka\b lumalam} \=a\d lu\d tiyapi\d l[\d l]ai}
	\item y\=ar tirukkoyil\index{cec}{koyil@\textit{k\=oyil} temple} mutal \textbf{pr\=a}k\=arattu tirumati\b n ti[ru]
	\item pa\d nikku ka\.nkaiko\d n\d taco\b lapurattu ka\.n[k]ai
	\item ko\d n\d taco\b la\b n tirumati\d lukku\d l\d la va\d tak\=u
	\item \b r(i)l uttamaco\b lapperu\b nteruvil v\=a\d na
	\item m\=a\d likai u\d taiy\=a\b n vempa\b n vaiciy\=ar
	\item maka\b n \=aramp\=u\d n\d t\=a\b n ittiruppa\d nikku \dots
	\item \~nccal\=akai accu irun\=u\b ru iva\b n akamu\d taiy\=a\index{cec}{akamutaiyal@\textit{akamu\d taiy\=a\d l} épouse}
	\item ........... tiruppa\d nikku\d tal\=aka ku \dots
	\item i\d t\d tum\=a\b ri po\b n a\b ru ka\b la\~ncum iva\b na\d tipira\d la \dots
	\item mayai vi\b r\b ra e\d t\d tu m\=a\b ri po\b n 8 3 ka\b la\~ncum
\end{enumerate}

\subsection*{CEC 35.3 R\'esum\'e}
Le texte date du 219\up{e} jour de la 3\up{e} ann\'ee du r\`egne d'\'Sr\=\i r\=ajar\=ajatevar, empereur des trois mondes. Le donateur est \=Aramp\=u\d n\d t\=a\b n, fils de Vempa\b n Vaiciy\=ar, un propriétaire [terrien] de V\=a\d nam\=a\d likai de la grande rue Uttamaco\b la dans Va\d takk\=ur \`a l'int\'erieur de l'enceinte de Ka\.nkaiko\d n\d taco\b l\=a\b n, \`a Ka\.nkaiko\d n\d taco\b lapuram\footnote{Malgr\'e la pr\'ecision de l'adresse du donateur, ce dernier n'est pas encore identifi\'e.}. Il offre de l'argent, et apparemment de l'or aussi (l.~13), pour financer la r\'eparation du mur de la premi\`ere enceinte du temple\index{gnl}{temple} d'\=A\d lu\d taiyapi\d l\d laiy\=ar\index{cec}{Alutaiyapillaiyar@\=A\d lu\d taiyapi\d l\d laiy\=ar} \`a Tirukka\b lumalam\index{cec}{Tirukkalumalam@Tirukka\b lumalam}. Le texte mentionne aussi l'\'epouse du donateur (l.~11).

\section*{CEC 36}
\subsection*{CEC 36.1 Remarques}

L'\'epigraphe, relev\'ee dans l'ARE 1918 387 et localis\'ee sur le mur est de l'enceinte de la chapelle de Campantar\index{gnl}{Campantar}, est fragmentaire. La date est lacunaire et seules figurent les informations astronomiques avec une troisi\`eme ann\'ee de r\`egne sans roi\index{gnl}{roi}. L'ARE propose le r\'esum\'e suivant: \og Stones out of order. Seems to register a gift of land for the teachers who gave instruction in tiruvi\'sai (music)\fg.

Le texte pr\'esent\'e est bas\'e sur l'unique examen de la transcription. Nous ne l'avons pas retrouv\'e \textit{in situ} \`a l'endroit indiqu\'e par l'ARE.

\subsection*{CEC 36.2 Texte}
\begin{enumerate}
\item \dots \d n\d tu mu[\b n*]\b r\=avatu kum\textbf{bha} n\=aya\b r\b ru aparapa\textbf{k\d sa}ttu \textbf{sapta}miyum ti\.nka\d l ki\b lamaiyum pe\b r\b ra anilattu\index{cec}{nilam@\textit{nilam} terre} n\=a\d l u\d taiy\=ar tirucci[ram]palamu\d taiy\=ar tevat\=a\b nam\index{cec}{tevatanam@\textit{tevat\=a\b nam} propriété divine} \textbf{r\=ajar\=aja}va\d lan\=a\d t\d tut tirukka\b lumalan\=a\d t\d tu\index{cec}{Tirukka\b lumalan\=a\d tu} akara \dots
\item narmuk kir\=amak\=ariya\~nceyki\b ra k\=u\d t\d ta perumakka\d l ka\d n\d tu tirukka\b lumalattu\index{cec}{Tirukkalumalam@Tirukka\b lumalam} \=a\d lu-\d taiyapi\d l\d laiy\=ar tiru\dots tar\=ama\dots \d n\d na \=ur\=akay\=ale ivv\=urile tiruvicai ka\b rpikkum \=aci-riyarka\d lukkum pa\dots
\item \dots mur ta\b naccai pirakarattuk ka\d tampantai \=anavivatikku\index{cec}{vati@\textit{vati}} ki\b lakku civap\=ataceka-rav\=akk\=alukku va\d takku 5 C 4 catirattu tilu(m*) 6 C
\item \dots\ k\=a\d ni muntirikaiyum nattama\b nai ira\d n\d ti\b n\=al nilam\index{cec}{nilam@\textit{nilam} terre} muntirikai ki\b laraiyum \=aka nilam\index{cec}{nilam@\textit{nilam} terre} ira\d n\d tu m\=a mukk\=a\d ni araikk\=a\d nikki\dots yila varimikiti\index{cec}{vari@\textit{vari} taxe} iva[r*]ka\d lukku \textbf{ji}va\b nattukku potukutillai e\b n\b rum innilam\index{cec}{nilam@\textit{nilam} terre} i\b raiyili\index{cec}{iraiyili@\textit{i\b raiyili} non imposable}
\item r vantu collukaiy\=ale innilam\index{cec}{nilam@\textit{nilam} terre} ira\d n\d tu m\=a mukk\=a\d ni araikk\=a\d nikki\b laraiyi\b nul tarap-pa\d ti ma\d takku nilam\index{cec}{nilam@\textit{nilam} terre} araikk\=a\dots ntirikai ki\b l mukkale araikk\=a\d nikki\b l orum\=avukku o\d t\d tuppa\d ti nellu n\=a\b rpat
\item \dots\ cantir\=atittavaraiyum i\b raiyiliy\=aka\index{cec}{iraiyili@\textit{i\b raiyili} non imposable} \dots\ nilam\index{cec}{nilam@\textit{nilam} terre} ira\d n\d tu m\=a mu[kk\=a]\d ni araikk\=a\d nik-ki\b larai \dots\ ma\d takku nilam\index{cec}{nilam@\textit{nilam} terre} araikk\=a\d ni muntirikaikki\b l mukk\=ale araikk\=a
\item \dots\ nellu na\dots \b npa\dots l k\=u\d ta \dots\ innilatt\=alu\d n\d t\=a\b na antar\=ayum ku\d timaiyum \dots\ cantir\=atitta
\item \dots \b niyokam\index{cec}{niyoka@\textit{niyokam} ordre} e\b lutikku\d tukka ippa\d niy\=al \dots\ m\=a\textbf{he\'sva}rappiya\b n e\b luttu U ir\=ay\=ur con\b nav\=ara\b riv\=a\b n\textbf{bha\d t\d tasya}\index{cec}{Con\b nav\=ara\b riv\=a\b nbha\d t\d ta} ippa\d tikku i[vai]
\item \dots\ ippa\d tikku tiruna\d t\d tam\=a\d ti\d tay\=an\textbf{bha\d t\d tasya}\index{cec}{Tiruna\d t\d tam\=a\d ti\d tay\=anbha\d t\d ta} ippa\d tikku ivai uloka\d taiy\=an\textbf{bha-\d t\d tasya}\index{cec}{Uloka\d taiy\=a\b nbha\d t\d ta} ippa\d tikku ivai kurava \dots
\item \dots\ ivai \textbf{sa}t\=aciva \dots\ [i]pa\d tikku ivai ti(ru)na\d t\dots\ \textbf{bha\d t\d tasya} \=a\d lu\d taiy\=an\textbf{bha\d t\d ta-sya}\index{cec}{alutaiyan@\=A\d lu\d taiy\=a\b nbha\d t\d ta} ca\.nkara\textbf{bha\d t\d tasya}\index{cec}{Ca\.nkarabha\d t\d ta} civaloka\dots
\end{enumerate}


\subsection*{CEC 36.3 R\'esum\'e}
Le texte enregistre le don\index{gnl}{don} d'une terre\index{gnl}{terre} par les membres du \textit{ku\d t\d tam} qui g\`erent les affaires du village\footnote{Ce groupe appara\^it dans CEC 26 mais les membres brahmane\index{gnl}{brahmane}s sont diff\'erents.} \`a Tirukka\b lumalam\index{cec}{Tirukkalumalam@Tirukka\b lumalam} dans le R\=ajar\=ajava\d lan\=a\d tu [qui est un] \textit{devad\=ana} du Seigneur propri\'etaire de Tirucci\b r\b rampalam. Ce don\index{gnl}{don} est destin\'e aux enseignants de musique\index{gnl}{musique} dans ce village et li\'e au temple\index{gnl}{temple} d'\=A\d lu\d taiyapi\d l\d laiy\=ar\index{cec}{Alutaiyapillaiyar@\=A\d lu\d taiyapi\d l\d laiy\=ar} de Tirukka\b lumalam\index{cec}{Tirukkalumalam@Tirukka\b lumalam}. La terre\index{gnl}{terre}, situ\'ee \`a l'est de la \textit{vati}\index{cec}{vati@\textit{vati}} Ka\d tampantai \=Anavi et au nord du canal Civap\=atacekkara, est donn\'ee non imposable, pour assurer leur subsistance, tant que durent lune et soleil.

Les signataires sont M\=ahe\'svarappiya\b n,
Ir\=ay\=ur Co\b n\b nav\=ara\b riv\=a\b nbha\d t\d ta,
Tiruna\d t-\d tam\=a\d ti\d tay\=anbha\d t\d ta,
Uloka\d taiy\=anbha\d t\d ta, Kurava\dots, Sat\=aciva \dots,
Tiruna\d t \dots bha\d t\d ta,
\=A\d lu\d taiy\=anbha\d t\d ta,
Ca\.nkarabha\d t\d ta,
Civaloka\dots

\section*{CEC 37}
\subsection*{CEC 37.1 Remarques}

L'\'epigraphe a été relev\'ee dans l'ARE 1918 386 et localis\'ee sur le mur de droit du pavillon d'entr\'ee de la chapelle de Campantar\index{gnl}{Campantar}. Il n'y a aucune datation.

Cette inscription de douze\index{gnl}{douze} lignes se trouve sur le mur face au nord dans l'entr\'ee du pavillon. Elle est \'edit\'ee sur la base de l'examen de la transcription de l'ASI, de clich\'es (G. \textsc{Ravindran}, EFEO) et de la lecture \textit{in situ}.

Le texte enregistre la donation d'une terre\index{gnl}{terre} pour assurer les travaux dans le temple\index{gnl}{temple} de Campantar\index{gnl}{Campantar}.

\subsection*{CEC 37.2 Texte}
\begin{enumerate}
\item \textbf{svasti \'sr\=\i}\ n\=aya\b n\=ar \=a\d lu\d taiyapi\d l\d laiy\=ar tiru
\item kkoyilukku\index{cec}{koyil@\textit{k\=oyil} temple} tiruppa\d nikku tiru\~n\=a\b nacampan
\item ta\b n\index{cec}{tirunana@Tiru\~n\=a\b nacampanta\b n} kamuku tirunantava\d namum inn\=aya\b n\=ar tiru
\item kkoyilukku\index{cec}{koyil@\textit{k\=oyil} temple} amutu pa\d ti p\=akkum ilai a
\item mutum\index{cec}{ilaiyamutu@\textit{ilaiyamutu} feuille de bétel} pokki nikki ni\b n\b ra mutalum tiruv\=a
\item kk\=uril \=urkki\b li\b raiyiliy\=a\b na\index{cec}{iraiyili@\textit{i\b raiyili} non imposable} nilat\index{cec}{nilam@\textit{nilam} terre}
\item tu mutal\=a\b na mutalum ittirukkoyilukku\index{cec}{koyil@\textit{k\=oyil} temple}
\item ttiruppa\d nikkup pokki ko\d l\d lavum inta
\item mutalil ittiruppa\d ni o\b liya ve\b rucilava\b littal
\item ceyt\=aru\d n\d t\=akil campantapperum\=a
\item \d l tiruva\d tikkup pi\b laitt\=arka\d l\=akavum
\item civat turokika\d l\index{cec}{turoki@\textit{tur\=oki} traître} [pa\d t\d tatu pa\d takka\d tavarka\d l]
\end{enumerate}

\subsection*{CEC 37.3 Traduction}
Que la prosp\'erit\'e soit! [Ceci est un don\index{gnl}{don}] pour [assurer les d\'epenses] des travaux du temple\index{gnl}{temple} du Seigneur \=A\d lu\d taiyapi\d l\d laiy\=ar\index{cec}{Alutaiyapillaiyar@\=A\d lu\d taiyapi\d l\d laiy\=ar}. Que
le capital qui reste --- ayant retir\'e le verger d'ar\'equier [nomm\'e] Tiru\~n\=a\b nacampanta\b n\index{cec}{tirunana@Tiru\~n\=a\b nacampanta\b n} et l'offrande de nourriture en noix d'arec et feuille de b\'etel pour le temple\index{gnl}{temple} de ce Seigneur --- plus le capital de la terre\index{gnl}{terre} \textit{\=urkki\b li\b raiyili}\index{cec}{iraiyili@\textit{i\b raiyili} non imposable} \`a Tiruv\=akk\=ur soient utilis\'es pour les travaux du temple\index{gnl}{temple}.

S'il est des gens qui utilisent autrement ce capital, d\'etruisant [ainsi] les d\'epenses pour ces travaux, ils deviendront ceux qui ont failli aux pieds du seigneur Campantar\index{gnl}{Campantar} et obtiendront le statut de tra\^itres de \'Siva\index{gnl}{Siva@\'Siva} (\textit{civat turokika\d l}).

%\newpage
%\part{Fragments}
%\thispagestyle{empty}
%\newpage
\section{Fragments}
\section*{A. Fragments relev\'es}
\section*{CEC 38}
\subsection*{CEC 38.1 Remarques}

Ce fragment se trouve sur une dalle au sol sur le chemin de circumambulation du temple\index{gnl}{temple} de \'Siva\index{gnl}{Siva@\'Siva}. Il a été relev\'e dans l'ARE 1918 368. Il date de la deuxi\`eme ann\'ee de r\`egne de Tribhuvanacakravartin R\=ajar\=ajadeva que \textsc{Mahalingam} (1992: 551, Tj. 2419) identifie avec doute \`a R\=ajar\=aja III\index{gnl}{Rajaraja III@R\=ajar\=aja III} en proposant la date de \textbf{1218}.

L'\'epigraphe n'a pas \'et\'e retrouv\'ee. Le texte \'edit\'e ci-dessous repose sur l'unique examen de la transcription de l'ASI.

\subsection*{CEC 38.2 Texte}
\begin{enumerate}
\item \textbf{svasti \'sr\=\i}\ tiripuva\b naccakkaravattika\index{cec}{Tribhuvanacakravarti} \textbf{\'sr\=\i}ir\=a\textbf{ja}
\item ku y\=a\d n\d tu 2 ir\=a\textbf{j\=adhi}r\=a\textbf{ja}va\d lan\=a\d t\d tu tirukka\b lumala\index{cec}{Tirukkalumalam@Tirukka\b lumalam}
\item \textbf{brahmade\'sa}m\index{cec}{brahmadeya@\textit{brahmadeya}} tirukka\b lumalattuk\index{cec}{Tirukkalumalam@Tirukka\b lumalam} ki\b lpi\d t\=akai\index{cec}{pitakai@\textit{pi\d t\=akai} hameau} anupa
\item \dots r u\d taiy\=ar r\=a\textbf{ja}r\=a\textbf{je\'sva}ramu\d taiya\b n\=ayan\=ar te
\item kulottu\.nkaco\b latevarkku\index{cec}{kulottungacoladeva@Kulottu\.ngac\=o\b ladeva} y\=a\d n\d tu 3 10 5 pari\b na
\item cimpiyattarayanum civ\=alaiya tevar pi\d l\d laikarai
\item lukku u\d l irukkum nallu\b l\=a\b n pir\=an\=a\d n\d t\=ar[k]ku kai
\item \b r pi\d l\d laika\d lal ku\d lakku\d taiy\=a\b n ampalava\b n\index{cec}{Ampalava\b n} u\d l\d li
\item ku\d taiy\=a\b n kovanum p\=a\d tu \=a\d n\d t\=ar akaratevar ko
\item \d tuva\b ntu maturai\index{cec}{Maturai} civatavanav\=a\textbf{si}pa\d t\d tanu\d l\d li\d t
\item lai m\=a\d tila\b n tirutto\d nipuramu\d taiy\=a\b n\index{cec}{Tonipuramutaiyan@T\=o\d nipuramu\d taiy\=a\b n} civa
\item r mutuka\d n pa\d t\d tu piram\=a\d nappa\d ti\index{cec}{piramanam@\textit{pram\=a\d nam} document} talaicca\.nk\=a\d t\d tu\index{cec}{Talaiccankatu@Talaicca\.nk\=a\d tu}
\end{enumerate}

\subsection*{CEC 38.3 R\'esum\'e}

Le texte date de la 2\up{e} ann\'ee de r\`egne de R\=ajar\=ajadeva, empereur des trois mondes. Il fait r\'ef\'erence \`a la 35\up{e} ann\'ee de Kulottu\.ngac\=o\b ladeva\index{cec}{kulottungacoladeva@Kulottu\.ngac\=o\b ladeva} et \`a U\d taiy\=a\b r R\=ajar\=a-je\'svaramu\d taiya N\=ayan\=ar qui est le nom du \textit{li\.nga}\index{gnl}{linga@\textit{li\.nga}} du grand temple\index{gnl}{temple} de Ta\~nc\=av\=ur\index{gnl}{Tancavur@Ta\~nc\=av\=ur}.

Figurent ensuite une liste de noms: Cimpiyattarayan, Civ\=alaiyatevar Pi\d l\d lai\dots,
Nallu\b l\=a\b n Pir\=an\=a\d n\d t\=ar,
Ku\d lakku\d taiy\=a\b n Ampalava\b n\index{cec}{Ampalava\b n},
Kovan \dots,
Pa\d tu \=A\d n\d t\=ar Akaratevar \dots,
 Civatavanav\=asipa\d t\d tar de Maturai\index{cec}{Maturai} \dots,
M\=a\d tila\b n Tirutt\=o\d nipuramu\d taiy\=a\b n.
Il est question d'un document effectu\'e par un tuteur et de Talaicca\.nk\=a\d tu\index{cec}{Talaiccankatu@Talaicca\.nk\=a\d tu}.

\section*{CEC 39}

\subsection*{CEC 39.1 Remarques}
Les fragments CEC 39 \`a 46 se trouvent sur des dalles au sol sur le chemin de circumambulation du temple\index{gnl}{temple} de \'Siva\index{gnl}{Siva@\'Siva} et ont \'et\'e relev\'es dans l'ARE 1918 369. Les textes reprennent la lecture des transcriptions de l'ASI.

\subsection*{CEC 39.2 Texte}
\begin{enumerate}
\item yan\=a\b rku i\b n\b n\=a\d t\d tu tiruv\=aliy\=a\b na\index{cec}{Tiruvali@Tiruv\=ali} mumu\d tico\b la
\item cempa\.nku\d ti ku\d lakku\d tiy\=ana kulottu\.nkaco\b la\index{cec}{kulottungacola@Kulottu\.ngac\=o\b la}
\item i\b ruttu\index{cec}{iruttu@\textit{i\b ruttu} payer un impôt} varuki\b ra nilattukku\index{cec}{nilam@\textit{nilam} terre} ivv\=uril t\=am pe\b r\=a
\item tu varuki\b ra nilattile\index{cec}{nilam@\textit{nilam} terre} mutal ko\d n\d tu i \dots
\item tirun\=amattukk\=a\d niyum\=aka\index{cec}{kani@\textit{k\=a\d ni} droit, propriété} \=uravar p\=u\textbf{jai}kkum tiru
\item kku ki\b lakku tiru\~n\=a\b nacampantav\=aykk\=alukku\index{cec}{tirunana@Tiru\~n\=a\b nacampanta\b n}\index{cec}{vaykkal@\textit{v\=aykk\=al} canal}
\item catirattu me\b rka\d taiya vi\d t\d ta nilam\index{cec}{nilam@\textit{nilam} terre} mu\b n\b ru m\=a
\item \b r\b riliy\=a\b na \=urppa\d ti nilam\index{cec}{nilam@\textit{nilam} terre} orum\=a U
\end{enumerate}


\section*{CEC 40}
\begin{enumerate}
\item (n)\=aya\b r\b ru aparapa\textbf{k\d sa}tattu \textbf{tri}ti
\item y\=a\b na kulottu\.nkaco\b lanall\=ur\index{cec}{kulottungacolanallur@Kulottu\.ngac\=o\b lanall\=ur}
\item pakkalum\index{cec}{pakkal@\textit{pakkal} auprès de} iva\b n m\=at\=avi\b n pakkalum\index{cec}{pakkal@\textit{pakkal} auprès de}
\item lum i\b ruttu\index{cec}{iruttu@\textit{i\b ruttu} payer un impôt} i\b rai mikuti ko\d n\d tu tiru
\item ve\b nav\=av\=uru\d taiy\=a\b n ka\d n\d navi.t.ai
\end{enumerate}


\section*{CEC 41}
\begin{enumerate}
\item tu tiruvekampamu\d taiyan\=ayana\index{cec}{Tiruvekampamu\d taiy\=a\b n}
\item \d tal\=aka ivv\=ur pi\d t\=akai\index{cec}{pitakai@\textit{pi\d t\=akai} hameau} c\=attama\.nka
\item m\=a mukk\=a\d ni araikk\=a\d niyum n\=a
\item m \=aka \=urpa\d ti nilam\index{cec}{nilam@\textit{nilam} terre} irupattu
\item rile e\b r\b ri tarami\d t\d tu ko\d n\d tu i
\item \d ti nilam\index{cec}{nilam@\textit{nilam} terre} n\=al m\=a. ivv\=uril
\item \d n\=a\b r\b ru ira\d n\d t\=a\~ncatirattu u\d l
\item \b na\d l\=a \dots \b na\textbf{ja}kkum tiruma
\end{enumerate}

\section*{CEC 42}
\begin{enumerate}
\item ku\d tal\=aka tirun\=amattukk\=a\d niy\=aka\index{cec}{kani@\textit{k\=a\d ni} droit, propriété} ko\d n\d ta
\item lattu ve\b ru pi\b ri\b nta o\d nveli \=akk\=uril va\d tamatu
\item pakkal\index{cec}{pakkal@\textit{pakkal} auprès de}ko\d n\d ta ma\d nali o\b n\b ru perk\=uvappa\d t\d ta nila\index{cec}{nilam@\textit{nilam} terre}
\item palav\=aykk\=alukkum\index{cec}{vaykkal@\textit{v\=aykk\=al} canal} p\=uta\b n\=uru\d taiy\=a\b n nila\index{cec}{nilam@\textit{nilam} terre}
\item \b ra v\=aya\b n ku\d la\.nkaraikkum inna
\item va \dots laikku va\d takkum me
\item \dots\ tevat\=anattukkuk\index{cec}{tevatanam@\textit{tevat\=a\b nam} propriété divine} ki
\item \dots na\d tuvupa\d t\d ta viri\d ni
\end{enumerate}

\section*{CEC 43}
\begin{enumerate}
\item itteva\b rku o\b npat\=avatu n\=a\d lil
\item deviy\=a\b na kulottu\.nkaco\b lacca
\item \b rkku p\=u\textbf{jai}kkum tiruppa\d nikkumu\d ta
\item ya i\b raiyili\index{cec}{iraiyili@\textit{i\b raiyili} non imposable} ceytu vi\d t\d ta nilattu\index{cec}{nilam@\textit{nilam} terre}
\item le mutal ko\d n\d tu \=urki\b l i\b na
\item kku ki\b lakku mulaparu\textbf{\d sa}v\=aykk\=aluk\index{cec}{vaykkal@\textit{v\=aykk\=al} canal}
\item \=a\b r\=a\.nka\d n\d n\=a\b r\b rira\d n\d t\=an tu\d n\d tattu
\end{enumerate}


\section*{CEC 44}
\begin{enumerate}
\item \b lacaturvetima[\.nkalattu] ve\d l\d l\=a\b n pa\b r\b ril
\item kkumu\d tal\=aka iv\=ur pi\d t\=akai\index{cec}{pitakai@\textit{pi\d t\=akai} hameau} tirukkuru
\item \b n\b ru i\b rukka niccayitta nilattu\index{cec}{nilam@\textit{nilam} terre} \=urk
\item ippa\d ti nellukku ivv\=uril taramili
\item l ko\d n\d tu n\=a\.nk\=urvatikku\index{cec}{nankur@N\=a\.nk\=ur}\index{cec}{vati@\textit{vati}} ki\b lakku ka\d n
\item \d n\=a\b r\b rira\d n\d t\=a\~ncatirattu ki\b lakka\d taiya
\item \b lakka\d taiyavum vi\d t\d ta \=urpa\d tinilam\index{cec}{nilam@\textit{nilam} terre} n\=alum\=a
\end{enumerate}


\section*{CEC 45}
\begin{enumerate}
\item pa\b r\b ra
\item runilai
\item \d li\d l\=al
\item ye ota
\item n\=amat
\end{enumerate}


\section*{CEC 46}
\begin{enumerate}
\item v\=acal te\d n\d ta
\item p\=a\d na pi\d l\d lai
\item malaimeyik\=avil
\end{enumerate}

\section*{CEC 47}
\subsection*{CEC 47.1 Remarques}

L'\'epigraphe a été relev\'ee dans l'ARE 1918 367 \`a partir d'une dalle du temple\index{gnl}{temple} de \'Siva\index{gnl}{Siva@\'Siva}. Elle date du r\`egne de Kopperu\~nci\.nkateva que \textsc{Mahalingam} (1992: 552, Tj. 2424) identifie comme K\=opperu\~nci\.nka II\index{gnl}{Kopperuncinka II@K\=opperu\~nci\.nka II} en sp\'eculant la \textbf{date approximative de 1243}. Elle a \'et\'e publi\'ee dans SII 12 252.

Nous n'avons pas retrouv\'e l'inscription. Le texte pr\'esent\'e ci-dessous reprend la publication.

CEC 47 enregistre un don\index{gnl}{don} de terre\index{gnl}{terre} pour r\'eciter les hymne\index{gnl}{hymne}s (\textit{tiruppatiyam}\index{gnl}{tiruppatiyam@\textit{tiruppatiyam}}) dans le temple\index{gnl}{temple}, nous supposons, d'[\=A\d lu\d taiya]pi\d l\d laiy\=ar

\subsection*{CEC 47.2 Texte}
\begin{enumerate}
\item lapuva\b naccakravattika\d l \textbf{\'sr\=\i}kopperu\~nci\.nkateva
\item \.nka\d lukkum u\d tal\=aka na\d tuviln\=a\d t\=a\b na ir\=a[ja*]r\=a\textbf{ja}
\item \d l\d laiy\=ar koyi[li]l tiruppatiyam vi\d n\d nappa\~n[c]e\index{cec}{vinnappamcey@\textit{vi\d n\d nappamcey} chanter}
\item \b n\b nila[m o]\b n\b re o\b npatu m\=avum t[e]vat\=a\b na k\=a
\item \dots \b lappal[la]varaiya\b ne
\end{enumerate}

\section*{CEC 48}
\subsection*{CEC 48.1 Remarques}
L'\'epigraphe a été relev\'ee dans l'ARE 1918 384 et localis\'ee sur le mur nord du \textit{ma\d n\d dapa} devant la chapelle de Campantar\index{gnl}{Campantar}.

L'inscription se trouve sur une pierre (neuf lignes sur quatre-vingt-dix centim\`etres) du soubassement de la partie carr\'ee saillante de la face sud du \textit{ma\d n\d dapa} de la chapelle de Campantar\index{gnl}{Campantar}. L'\'edition se base sur la transcription de l'ASI, nos clich\'es et la lecture \textit{in situ}.

Le texte enregistre un don\index{gnl}{don} pour nourrir \=A\d lu\d taiyapi\d l\d laiy\=ar\index{cec}{Alutaiyapillaiyar@\=A\d lu\d taiyapi\d l\d laiy\=ar}. Il est question de douze\index{gnl}{douze} \textit{kalam} de riz\index{gnl}{riz} d\'ecortiqu\'e non cuit. Les \textit{\'sr\=\i mahe\'svara}\index{cec}{srimahesvara@\textit{\'sr\=\i mahe\'svara} dévot, surveillant} et les \textit{t\=a\b natt\=ar}\index{cec}{tanattar@\textit{t\=a\b natt\=ar} employé du temple} du temple\index{gnl}{temple}  sont pr\'esents.

\subsection*{CEC 48.2 Texte}
\begin{enumerate}
\item \=a\d lu\d taiyapi\d l\d laiy\=ar pac\=anat\index{cec}{pacanam@\textit{pac\=anam} moisson} tiruppu putiyutu tirupp\=av\=a
\item \d tay\=aka amutuceytaru\d li \textbf{\'sr\=\i}m\=a\textbf{he\'sva}rkkum t\=a\b natt\=arkkum\index{cec}{tanattar@\textit{t\=a\b natt\=ar} employé du temple} [li]
\item rumu\b nappukkam\=aka \=ayiram ma\d takkil i\d t\d tu \textbf{pras\=a}tikkapo\b nakap
\item pa\b la arici pa\b n\b niru kalattukkum itukku ve\d n\d tu\.nka\b riyamu[tu]
\item {[vi]\~nca\b nattukkum tevai cevv\=arkku ve\d n\d tuva\b na vayi[\b r\b ru]}
\item ko\d n\d tu celutta tirun\=amattuk\=a\d niy\=aka\index{cec}{kani@\textit{k\=a\d ni} droit, propriété} kolliku\b rumpu\d tai[ya]
\item {[\=a]\d n\d t\=ar celvamalku pakal polum pero\d liy\=aka k\=acu\index{cec}{kacu@\textit{k\=acu} pièces de monnaie} tantu [n]}
\item {[e] c[e]mpiyan pa\d nama\.nkalattu pi\b rinta vikkiramaco\b lakkollai\index{cec}{Vikkiramaco\b lakkollai}}
\item {[araiyil i]\b raimikuti [ko\d n\d tu ippa\d ti ca\.ntir\=atittaval]}
\end{enumerate}




\section*{CEC 49}
\subsection*{CEC 49.1 Remarques}
L'inscription a été relev\'ee dans l'ARE 1918 385 et localis\'ee sur trois piliers dans le \textit{ma\d n\d dapa} de la chapelle de Campantar\index{gnl}{Campantar}. Ces piliers, en r\'e-emploi dans la chapelle de la d\'esse, servent aujourd'hui \`a maintenir une cloche.

Le texte mentionne deux noms: \=A\b ra\b n\=ur I\b naiccayappa\b n et \=Ak\=aravallava\b n.

\subsection*{CEC 49.2 Texte}
\begin{enumerate}
\item \=a\b ra\b n\=u\b r
\item i\d laicca
\item yappa\b n
\end{enumerate}

\begin{enumerate}
\item \=ak\=arava
\item llavan
\end{enumerate}

\begin{enumerate}
\item \=ak\=arava
\item llavan
\end{enumerate}

\section*{B. Fragments d\'ecouverts}
\section*{CEC 50}
\subsection*{CEC 50.1 Remarques}
Ce fragment se trouve sur une dalle dans la cour du temple\index{gnl}{temple} de \'Siva\index{gnl}{Siva@\'Siva}, c\^ot\'e sud. Son \'edition est faite \`a partir de clich\'es (G. \textsc{Ravindran}, EFEO) et de la lecture \textit{in situ}.

\subsection*{CEC 50.2 Texte}
\begin{enumerate}
\item \textbf{svasti \'sr\=\i}\ v[\=\i ]rar\=a\dots
\item tirukka\b lumala\b n\=a\d tu\index{cec}{Tirukka\b lumalan\=a\d tu} viraco\b la..
\item vi\d ta\.nkanal\=ur ir\=a\textbf{ji}curamu\d tai
\item n\=a\.nkal vi\d t\d ta nilam\index{cec}{nilam@\textit{nilam} terre} ivv\=uril ti
\item cav\=ayk\=a\b rku te\b rku muta\b rka\d n\d n\=a\b r\b ru muta\b r
\item .....yum cantir\=atittavarai cel
\end{enumerate}

\section*{CEC 51}
\subsection*{CEC 51.1 Remarques}
Ce fragment se trouve sur la face sud du \textit{ma\d n\d dapa} de la chapelle de Campantar\index{gnl}{Campantar}. Son \'edition est \'etablie \`a partir de clich\'es (G. \textsc{Ravindran}, EFEO) et de la lecture \textit{in situ}.

\subsection*{CEC 51.2 Texte}
\begin{enumerate}
\item \b rukkum i\b raiyi\b ruttu\index{cec}{iruttu@\textit{i\b ruttu} payer un impôt} mikuti
\item \b n pakkal\index{cec}{pakkal@\textit{pakkal} auprès de} vilaiko\d n\d ta\index{cec}{vilai@\textit{vilai} prix!vilai-kol@\textit{vilai-ko\d l} acheter} nilattu\index{cec}{nilam@\textit{nilam} terre} \dots
\item k[o]\d n\d tu vi\d t\d ta tirukka\b lumalan\=a\d t\d tu\index{cec}{Tirukka\b lumalan\=a\d tu} p\=ati(ra)kku\d tiy
\item viku\b rump\=uril ti\d tar cey nilam\index{cec}{nilam@\textit{nilam} terre} arai innila\index{cec}{nilam@\textit{nilam} terre}
\end{enumerate}

\section*{CEC 52}
\subsection*{CEC 52.1 Remarques}
Ce fragment se trouve sur la face nord du \textit{ma\d n\d dapa} de la chapelle de Campantar\index{gnl}{Campantar}. Son \'edition se fait \`a partir de clich\'es et de la lecture \textit{in situ}.

\subsection*{CEC 52.2 Texte}
\begin{enumerate}
\item \dots\ itil ti
\item \dots\ 3 catirattu N A
\item yk\=alukku va\d takku 1 C 1 catira
\end{enumerate}

\section*{CEC 53}
\subsection*{CEC 53.1 Remarques}
Ce fragment se trouve sous la fen\^etre \`a claire-voie sur le mur sud de la chapelle de Campantar\index{gnl}{Campantar}. Notre \'edition est établie \`a partir de clich\'es et de la lecture \textit{in situ}.

\subsection*{CEC 53.2 Texte}
\begin{enumerate}
\item (ti)tukka\b lumalattu\index{cec}{Tirukkalumalam@Tirukka\b lumalam} \=ur m\=aviyanti
\item kiyi\b ruttu\index{cec}{iruttu@\textit{i\b ruttu} payer un impôt} i\b raimikuti ko\d n
\item \dots
\item \d taiy\=a\b n \=a\d tko\dots
\item .yiti..\.nka....ti
\item va\d laka\d ta.....kap\=ara
\item \dots\ pakkal\index{cec}{pakkal@\textit{pakkal} auprès de}
\item ko\d n\d tu vi\d t\d ta \=a\b r\=a\.nka\d t\d ta\d lai e\d tut
\item {[tap\=a]tavatikku\index{cec}{vati@\textit{vati}} me\b rku ni\b n\b r\=a\b nv\=ayk\=alu}\index{cec}{vaykkal@\textit{v\=aykk\=al} canal}
\item kellai \b ni....ta N A\ddanda |
\end{enumerate}

\section*{CEC 54}
\subsection*{CEC 54.1 Remarques}
Ce fragment se trouve sur le \textit{gopura}\index{gnl}{gopura@\textit{gopura}} nord.
\subsection*{CEC 54.2 Texte}
\begin{enumerate}
\item k\=averit
\item \b lan\=u(ya\d t\d ta)
\item m iruk..
\item irun\=a\b liy\index{cec}{nali@\textit{n\=a\b li} unité de mesure de graine}
\end{enumerate}

\section*{CEC 55}
\subsection*{CEC 55.1 Remarques}
Ce fragment se trouve sur le \textit{gopura}\index{gnl}{gopura@\textit{gopura}} nord.

\subsection*{CEC 55.2 Texte}
\begin{enumerate}
\item .kevatu
\item kkukama iv\=u
\item .r\=a.camupaya
\item ku \textbf{\'sivabr\=a}
\item y\=anupaya\ddanda |
\end{enumerate}


\chapter{L'histoire du site}

Le \'Siva\index{gnl}{Siva@\'Siva} ou le \textit{li\.nga}\index{gnl}{linga@\textit{li\.nga}} de C\=\i k\=a\b li\index{gnl}{Cikali@C\=\i k\=a\b li} \'etait nomm\'e U\d taiy\=ar Tirutt\=o\d nipuram\index{gnl}{Tonipuram@T\=o\d nipuram} U\d taiy\=ar, d\'efinissant ainsi le site par rapport au mythe\index{gnl}{mythe} fondateur du déluge\index{gnl}{deluge@déluge}. Il se trouve au lieu-dit de Ka\b lumalam\index{gnl}{Kalumalam@Ka\b lumalam} qui est un \textit{brahmadeya}\index{cec}{brahmadeya@\textit{brahmadeya}} (voir note de CEC 1) du pays de Ka\b lumalam\index{gnl}{Kalumalam@Ka\b lumalam} (Ka\b lumalan\=a\d tu\index{cec}{Tirukka\b lumalan\=a\d tu}) dans la division r\'egionale du R\=aj\=adhir\=ajava\d lan\=a\d tu\index{cec}{Rajadhirajavala@R\=aj\=adhir\=ajava\d lan\=a\d tu}. Le pays de Ka\b lumalam\index{gnl}{Kalumalam@Ka\b lumalam} inclut dans ses terres\index{gnl}{terre} K\=olakk\=a\index{gnl}{Kolakka@K\=olakk\=a} (ARE 1918 410) et Agn\=\i \'svaram (\'edition des textes épigraphiques en pr\'eparation) qui se situent \`a environ un kilom\`etre du temple\index{gnl}{temple} actuel de C\=\i k\=a\b li\index{gnl}{Cikali@C\=\i k\=a\b li} au nord-ouest et au nord-est, respectivement. \textsc{Subbarayalu} (1973: carte 10), d\'elimite ce territoire en y incluant d'autres sites comme Talai\~n\=ayi\b ru. La grande division du R\=aj\=adhir\=ajava\d lan\=a\d tu\index{cec}{Rajadhirajavala@R\=aj\=adhir\=ajava\d lan\=a\d tu} qui figure dans le CEC n'aurait \'et\'e d\'efinie qu'en 1080 (\textsc{Subbarayalu} 1973: 64). Elle longe la c\^ote et est travers\'ee en son milieu par le Ko\d l\d li\d tam qui se jette dans la mer\index{gnl}{mer}. Elle englobe au nord Citamparam\index{gnl}{Citamparam} et au sud Mayil\=a\d tutu\b rai\index{gnl}{Mayilatuturai@Mayil\=a\d tutu\b rai}.
C\=\i k\=a\b li\index{gnl}{Cikali@C\=\i k\=a\b li} est situ\'e au centre de cette zone fertile. Les toponymes mentionn\'es dans le CEC font partie de cette zone ou la jouxtent, soulignant ainsi un rayonnement\index{gnl}{rayonnement} g\'eographique (d\'e)-limit\'e.

Pour pr\'esenter l'histoire du site, \'etudions ses pierres et ses hommes \`a travers le CEC, puis sa long\'evit\'e.


\section{La formation du complexe}

Les trente-six inscriptions datables du temple\index{gnl}{temple} de C\=\i k\=a\b li\index{gnl}{Cikali@C\=\i k\=a\b li}, de la premi\`ere moiti\'e du \textsc{xii}\up{e} si\`ecle \`a l'extr\^eme fin du \textsc{xvi}\up{e}, sont le reflet d'une histoire active marqu\'ee par la succession de diff\'erentes dynasties pr\'esentes dans le delta de la K\=av\=eri\index{gnl}{Kaveri@K\=av\=eri} de 1135 \`a 1598. Ces empreintes laiss\'ees sur les pierres du temple\index{gnl}{temple} sous les C\=o\b la\index{gnl}{Cola@C\=o\b la} (de Kulottu\.nga II\index{gnl}{Kulottu\.nga II} \`a R\=ajar\=aja III\index{gnl}{Rajaraja III@R\=ajar\=aja III}), les P\=a\d n\d dya (M\=a\b ravarman Vikrama P\=a\d n\d dya IV), les K\=a\d tavar tardifs se revendiquant Pallava\index{gnl}{Pallava} (Kopperu\~nsi\d mhadeva II) et les Vijayanagara (Viruppa\d n\d na, K\textsubring{r}\d s\d nadeva, Ve\.nka\d tadeva) ainsi que sous leurs subordonn\'es N\=ayaka permettent de comprendre quelque peu l'arch\'eologie du site et de reconnaître diff\'erentes strates de formation du complexe.

L'emplacement des inscriptions conservées laisse supposer que la chapelle de Campantar\index{gnl}{Campantar} est le b\^atiment le plus ancien avec le temple\index{gnl}{temple} principal de \'Siva\index{gnl}{Siva@\'Siva}. Il n'y a aucune inscription aujourd'hui sur les murs du temple\index{gnl}{temple} principal de \'Siva\index{gnl}{Siva@\'Siva}. L'architecture semble appartenir \`a celle de la période dite \og \textit{c\=o\b la}\index{gnl}{cola@\textit{c\=o\b la}} tardive\fg\ (Cf. \textsc{Balasubramaniam} 1979). Le temple n'apparaît pas dans les listes des premiers temples \textit{c\=o\b la}\index{gnl}{cola@\textit{c\=o\b la}} recensés par \textsc{Hoekveld-Meier} (1981) et par \textsc{Meister \&\ Dhaky} (1983) par exemple. Ces deux monuments abritant les \textit{cella} de Campantar\index{gnl}{Campantar} et du \textit{li\.nga}\index{gnl}{linga@\textit{li\.nga}} dateraient de la fin du \textsc{xi}\up{e} et du d\'ebut du \textsc{xii}\up{e} si\`ecle. Les \textit{ma\d n\d dapa} construits devant ces deux cella seraient l\'eg\`erement post\'erieurs. Les inscriptions du \textit{ma\d n\d dapa} de \'Siva\index{gnl}{Siva@\'Siva} datent de 1184 (CEC 1) \`a 1339 (CEC 6) et celles du \textit{ma\d n\d dapa} de Campantar\index{gnl}{Campantar} de 1219 (CEC 33 et 34)\footnote{CEC 30, datant de 1158, enregistre une donation de terre\index{gnl}{terre} pour nourrir l'image\index{gnl}{image} de la dévot\index{gnl}{devot(e)@dévot(e)}e Ma\.nkaiya\b rkkaraci\index{gnl}{Mankaiyarkkaraci@Ma\.nkaiyarkkaraci}, reine\index{gnl}{reine} \textit{p\=a\d n\d dya}\index{gnl}{pandya@\textit{p\=a\d n\d dya}} qui, selon le \textit{Periyapur\=a\d nam}\index{gnl}{Periyapuranam@\textit{Periyapur\=a\d nam}}, fit appel \`a Campantar\index{gnl}{Campantar} pour convertir du ja\"inisme\index{gnl}{jainisme@ja\"inisme} au shiva\"isme\index{gnl}{shivaisme@shiva\"isme} son \'epoux. La statue d'une figure f\'eminine, appel\'ee J\~n\=an\=ambik\=a, est pr\'esente aujourd'hui dans la chapelle de Campantar\index{gnl}{Campantar}. Elle est abrit\'ee, plus exactement, dans la \textit{cella}, ouverte au sud, sur le c\^ot\'e nord du \textit{ma\d n\d dapa}. S'il s'agit de la m\^eme image\index{gnl}{image} et si son emplacement n'a pas \'et\'e modifi\'e, le \textit{ma\d n\d dapa} de Campantar\index{gnl}{Campantar} est ant\'erieur \`a 1158.}. Les murs d'enceinte des temple\index{gnl}{temple}s de \'Siva\index{gnl}{Siva@\'Siva} et de Campantar\index{gnl}{Campantar} ont sans doute probablement \'et\'e mis en place au d\'ebut du \textsc{xiii}\up{e} si\`ecle parce que nous relevons des \'epigraphes allant de 1224 (CEC 7) \`a 1263 (CEC 13) chez \'Siva\index{gnl}{Siva@\'Siva} et de 1218 (CEC 35) chez Campantar\index{gnl}{Campantar}. Suivant toujours ce m\^eme raisonnement, les galeries int\'erieures du temple\index{gnl}{temple} de \'Siva\index{gnl}{Siva@\'Siva} sont datables du \textsc{xiv}\up{e} si\`ecle et le pavillon d'entr\'ee du \textsc{xv}\up{e} si\`ecle. Le bassin est mentionn\'e dans CEC 6 qui date de 1339.
Seule la chapelle de la d\'eesse\index{gnl}{deesse@déesse} ne poss\`ede aucune inscription. Son style architectural rappelle celui des chapelles de la d\'eesse\index{gnl}{deesse@déesse} \`a Vaitt\=\i \'svarakk\=oyil et \`a Tiruve\.nk\=a\d tu\index{gnl}{Venkatu@Ve\.nk\=a\d tu!Tiruve\.nk\=a\d tu} datant du \textsc{xvii}\up{e} si\`ecle (ARE 1918 521 et ARE 1918 420). L'image\index{gnl}{image} d'un personnage masculin dans la galerie de la chapelle renforce l'hypoth\`ese de cette datation. Une inscription moderne au-dessus de ce personnage l'identifie \`a Ku\d t\d tiy\=api\d l\d lai. Un individu du m\^eme nom est mentionné dans une inscription de Ma\d n\d nipa\d l\d lam, datant de 1595 (ARE 1927 160), qui enregistre une donation de \textit{ma\d n\d dapa}, pavillon d'entr\'ee et de bassin par Ci\b n\b n\=ayi, celle du palais de Ku\d t\d tiy\=api\d l\d lai, pour le m\'erite de V\=\i r\=ayi\footnote{\dots\ \textit{ku\d t\d tiy\=api\d l\d lai arama\b naiy\=ar ci\b n\b na \=ayi v\=\i r\=ayi pu\d n\d niyam\=aka \dots} (l. 5-8).}. Nous supposons donc que ce Ku\d t\d tiy\=api\d l\d lai, figure importante de la r\'egion de Talai\~n\=ayi\b ru, \`a proximit\'e de C\=\i k\=a\b li\index{gnl}{Cikali@C\=\i k\=a\b li}, a \oe uvr\'e dans la fondation\index{gnl}{fondation} de la chapelle de la d\'eesse\index{gnl}{deesse@déesse} \`a C\=\i k\=a\b li\index{gnl}{Cikali@C\=\i k\=a\b li} entre le \textsc{xvi}\up{e} et le \textsc{xvii}\up{e} si\`ecle.
\begin{figure}[!h]
  \centering
 \includegraphics[height=6cm]{docthese/photoCIIKAALI/chapdeesse5.JPG}
  \caption{Ku\d t\d tiy\=api\d l\d lai, galerie ouest de la chapelle de la déesse, C\=\i k\=a\b li (cliché U. \textsc{Veluppillai}, 2006).}
\end{figure}

Le pr\'esence de fragments d'inscription sur les dalles du chemin de circumambulation du temple\index{gnl}{temple} de \'Siva\index{gnl}{Siva@\'Siva} et dans les murs du pavillon d'entr\'ee nord et des derniers murs d'enceinte (en partant du centre) t\'emoigne des divers travaux effectu\'es dans le temple\index{gnl}{temple} ces deux derniers si\`ecles comme en atteste la brochure du \textit{C\=\i k\=a\b li\index{gnl}{Cikali@C\=\i k\=a\b li} talavaral\=a\b ru} de l'année 2000 (p. 28-31).
\begin{figure}[!h]
  \centering
 \includegraphics[height=6cm]{docthese/photoCIIKAALI/sivatpl29.JPG}
  \caption{Espace entre le corps principal et le b\^atiment à étages dans le temple de \'Siva, C\=\i k\=a\b li (cliché U. \textsc{Veluppillai}, 2006).}
\end{figure}

\section{Les acteurs}

C\=\i k\=a\b li\index{gnl}{Cikali@C\=\i k\=a\b li} est le lieu de naissance\index{gnl}{naissance} de Campantar\index{gnl}{Campantar} qui lui aurait d\'edi\'e soixante-sept hymne\index{gnl}{hymne}s du \textit{T\=ev\=aram}\index{gnl}{Tevaram@\textit{T\=ev\=aram}}. Campantar\index{gnl}{Campantar} est devenu l'enfant\index{gnl}{enfant} prodige\index{gnl}{prodige} hautement c\'el\'ebr\'e dans le \textit{Periyapur\=a\d nam} au \textsc{xii}\up{e} si\`ecle. La litt\'erature religieuse t\'emoigne donc de l'existence du temple\index{gnl}{temple} de C\=\i k\=a\b li\index{gnl}{Cikali@C\=\i k\=a\b li} avant le \textsc{xii}\up{e} si\`ecle. Cependant, aucune donn\'ee historique disponible (\'epigraphique ou arch\'eologique) n'atteste la pr\'esence du temple\index{gnl}{temple} de C\=\i k\=a\b li\index{gnl}{Cikali@C\=\i k\=a\b li} avant la premi\`ere moiti\'e du \textsc{xii}\up{e} si\`ecle. Nous ne pouvons que penser que la fixation par \'ecrit de la légende\index{gnl}{legende@légende} de Campantar\index{gnl}{Campantar} a marqu\'e un tournant dans l'histoire du temple\index{gnl}{temple} qui conna\^it d\`es lors une certaine \og renaissance\fg. Par exemple, l'assemblée\index{gnl}{assemblée} de Tirukka\b lumalam\index{gnl}{Kalumalam@Ka\b lumalam} ordonne de r\'eparer et d'entretenir les \^ole\index{gnl}{ole@\^ole}s du \textit{Tirumu\b rai}\index{gnl}{Tirumurai@\textit{Tirumu\b rai}} enferm\'ees dans le \textit{tirukkaikko\d t\d ti} de la chapelle de Campantar\index{gnl}{Campantar} en 1136 (CEC 26).

Cette \og renaissance\fg\ tardive, au \textsc{xii}\up{e} si\`ecle, explique en partie l'absence de donation royale alors qu'\`a vingt kilom\`etres au nord, la famille royale
\textit{c\=o\b la}\index{gnl}{cola@\textit{c\=o\b la}} couvrent d'or le temple\index{gnl}{temple} de Citamparam\index{gnl}{Citamparam}! Qui sont les acteurs qui font l'histoire du site de C\=\i k\=a\b li\index{gnl}{Cikali@C\=\i k\=a\b li} du \textsc{xii}\up{e} au \textsc{xvi}\up{e} si\`ecle?

Les rois ne font pas de donation à C\=\i k\=a\b li. Ils y interviennent trois fois, indirectement. Au \textsc{xiii}\up{e} si\`ecle, R\=ajar\=aja III\index{gnl}{Rajaraja III@R\=ajar\=aja III} envoie l'ordre\index{gnl}{ordre royal} de mettre en vente les terres\index{gnl}{terre} de traîtres et on grave son éloge\index{gnl}{eloge@éloge} royal (CEC 7 et CEC 8). Au \textsc{xiv}\up{e} si\`ecle, un individu du pays \textit{p\=a\d n\d dya}\index{gnl}{pandya@\textit{p\=a\d n\d dya}}, d'une relative importance semble-t-il, installe les image\index{gnl}{image}s du roi\index{gnl}{roi} M\=a\b ravarman Vikrama P\=a\d n\d dya IV et de son \'epouse et met en place un culte\index{gnl}{culte} \`a leur nom (CEC 6). Enfin, au \textsc{xv}\up{e} si\`ecle, un certain K\=on\=eridevamah\=ar\=aja qui est li\'e \`a K\=a\~ncipuram\index{gnl}{Kancipuram@K\=a\~ncipuram}, mais que nous n'avons pas pu identifier, donne l'ordre\index{gnl}{ordre royal} de reverser, comme auparavant, les taxes des villages dans la trésorerie\index{gnl}{tresorerie@trésorerie} du temple\index{gnl}{temple} (CEC 20).

Les notables et les officier\index{gnl}{officier}s royaux sont les principaux donateurs du temple\index{gnl}{temple} de \'Siva\index{gnl}{Siva@\'Siva}. Certains ont pu \^etre identifi\'es gr\^ace \`a leur activit\'e soutenue comme Karu\d n\=akarat\=eva\b n alias V\=a\d n\=atir\=aya\b n dans CEC 1. Beaucoup d'agents administratifs posent leur signature, l\'egalisent les transactions qu'ils int\`egrent ainsi dans les affaires du royaume.

Dans la chapelle de Campantar\index{gnl}{Campantar}, les donateurs sont essentiellement des assemblées\index{gnl}{assemblée} villageoises comme celle de Ka\b lumalam\index{gnl}{Kalumalam@Ka\b lumalam} (CEC 25 et 26), de Talaicca\.nk\=a\d tu\index{gnl}{Talaiccankatu@Talaicca\.nk\=a\d tu} (CEC 27), de Tiruv\=alin\=a\d tu\index{cec}{Tiruvalin@Tiruv\=alin\=a\d tu} (CEC 29) ou un autre groupe, celui des gardes, \textit{parikirakam}, de Viraco\b lanall\=ur dans le pays Ka\b lumalam\index{gnl}{Kalumalam@Ka\b lumalam} (CEC 30).

Jusqu'au \textsc{xiv}\up{e} si\`ecle, les inscriptions du temple\index{gnl}{temple} de \'Siva\index{gnl}{Siva@\'Siva} ne mentionne pas la chapelle de Campantar\index{gnl}{Campantar} et inversement comme s'ils \'etaient deux entit\'es distinctes avec un fonctionnement s\'epar\'e. Le temple\index{gnl}{temple} de \'Siva\index{gnl}{Siva@\'Siva} vit de la g\'en\'erosit\'e de particuliers de haut rang alors que la chapelle de Campantar\index{gnl}{Campantar} est soutenue massivement par des assemblées\index{gnl}{assemblée} villageoises brahmane\index{gnl}{brahmane}s de la région. \`A partir de la fin du \textsc{xiv}\up{e} si\`ecle, dans les inscriptions \textit{vijayanagara}, \'Siva\index{gnl}{Siva@\'Siva}-T\=o\d nipuramu\d taiy\=ar est li\'e \`a Campantar\index{gnl}{Campantar}-\=A\d lu\d taiyapi\d l\d lai. Le temple\index{gnl}{temple} et la chapelle ont une administration commune. Les donateurs sont principalement des brahmane\index{gnl}{brahmane}s qui travaillent ou qui vivent pr\`es du temple\index{gnl}{temple}. Les inscriptions mentionnent leurs salaires, leurs terres\index{gnl}{terre}, etc. CEC 17 illustre parfaitement ce propos: un lopin de soixante \textit{veli} est partag\'ee en sept au b\'en\'efice d'\=A\d lu\d taiyapi\d l\d lai, d'un renon\c cant initi\'e appel\'e Aru\d nagiri\'siva, d'un officiant, d'un chef\index{gnl}{chef} de monast\`ere\index{gnl}{monastère}, d'un officiant des \textit{p\=uj\=a}\index{gnl}{puja@\textit{p\=uj\=a}} et d'un surveillant. Nous constatons, par ailleurs, que la chapelle de Campantar\index{gnl}{Campantar} n'a plus d'inscription apr\`es le \textsc{xiii}\up{e} si\`ecle. Nous supposons qu'elle est devenue \`a partir de cette \'epoque subordonn\'ee au temple\index{gnl}{temple} de \'Siva\index{gnl}{Siva@\'Siva}, comme aujourd'hui. CEC 17 enregistre ainsi une donation pour Campantar\index{gnl}{Campantar} mais le texte est grav\'e dans le temple\index{gnl}{temple} de \'Siva\index{gnl}{Siva@\'Siva}. L'inscription de CEC 15 est gravée sur le temple de \'Siva mais la donation qu'elle enregistre est placée sous la protection de Campantar.

Le \textsc{xiii}\up{e} si\`ecle est aussi la p\'eriode o\`u se d\'eveloppe un monast\`ere\index{gnl}{monastère} pr\`es du temple\index{gnl}{temple} de \'Siva\index{gnl}{Siva@\'Siva}. Il se nomme Tirumu\b raitt\=ev\=araccelva\b n\index{gnl}{Tirumuraittevaraccelvan@Tirumu\b raitt\=ev\=araccelva\b n} et se trouve au nord du temple\index{gnl}{temple} de T\=o\d nipuram\index{gnl}{Tonipuram@T\=o\d nipuram} (SII 8 205 et ARE 1918 10). Il n'est pas express\'ement nomm\'e dans le CEC mais de nombreux textes évoquent des chefs\index{gnl}{chef} (\textit{mutali}), des terres\index{gnl}{terre} et des jardins de monast\`ere\index{gnl}{monastère}. Selon les informations du bureau du \textit{devasth\=anam}, un monast\`ere\index{gnl}{monastère}, situ\'e face au pavillon d'entr\'ee nord, \'etait encore en service\index{gnl}{service} il y a une cinquantaine d'ann\'ees lorsque le bureau \'etait dirigé par un disciple du monast\`ere\index{gnl}{monastère} de Tarumapuram sur place. Aujourd'hui, il ne reste que des ruines.

Ainsi, pour r\'esumer tr\`es sch\'ematiquement les donn\'ees \'epigraphiques, nous pouvons dire qu'entre le \textsc{xii}\up{e} et le \textsc{xiii}\up{e} si\`ecle deux groupes de donateurs se distinguent \`a C\=\i k\=a\b li\index{gnl}{Cikali@C\=\i k\=a\b li}: de hauts fonctionnaires donnent au temple\index{gnl}{temple} de \'Siva\index{gnl}{Siva@\'Siva} et des assemblées\index{gnl}{assemblée} villageoises brahmane\index{gnl}{brahmane}s d\'eveloppent la chapelle de Campantar\index{gnl}{Campantar}. \`A partir des \textsc{xiii}\up{e}-\textsc{xiv}\up{e} si\`ecles, ces deux corps de b\^atiment sont unis et jouissent d'une m\^eme administration. Les donateurs sont dès lors des brahmane\index{gnl}{brahmane}s locaux\index{gnl}{local}, li\'es plus ou moins au temple\index{gnl}{temple}, qui imposent ou renforcent leur fonction \og eccl\'esiastique\fg\ avec le partenariat d'un monast\`ere\index{gnl}{monastère}. Le rayonnement\index{gnl}{rayonnement} du temple\index{gnl}{temple} de C\=\i k\=a\b li\index{gnl}{Cikali@C\=\i k\=a\b li} semble se restreindre. Parall\`element, le temple\index{gnl}{temple} de Citamparam\index{gnl}{Citamparam} marque son h\'eg\'emonie sur un territoire de plus en plus vaste. La grande division du R\=aj\=adhir\=ajava\d lan\=a\d tu\index{cec}{Rajadhirajavala@R\=aj\=adhir\=ajava\d lan\=a\d tu} devient un \textit{devad\=anam} du Seigneur de Tirucci\b r\b rampalam (CEC 28 et 29)\footnote{La pr\'esentation de la division territoriale du R\=aj\=adhir\=ajava\d lan\=a\d tu\index{cec}{Rajadhirajavala@R\=aj\=adhir\=ajava\d lan\=a\d tu} comme une terre\index{gnl}{terre} de la divinit\'e de Citamparam\index{gnl}{Citamparam} appara\^it aussi dans les inscriptions de Talai\~n\=ayi\b ru (ARE 1927 142 l.~6-8) et d'\=Acc\=a\d lpuram (ARE 1918 527 l.~1) dès la seconde moitié du \textsc{xii}\up{e} siècle.}. Nous ne poss\'edons pas assez d'\'el\'ements dans le CEC pour \'etablir un lien certain entre les destin\'ees de ces deux temple\index{gnl}{temple}s.
Nous pouvons simplement affirmer qu'\`a la fin du \textsc{xvi}\up{e} si\`ecle C\=\i k\=a\b li\index{gnl}{Cikali@C\=\i k\=a\b li} se revitalise, en quelque sorte, avec la venue dans son enceinte d'une divinit\'e du pays ka\d n\d na\d da. L'installation\index{gnl}{installation d'une image} d'\=Apaduddh\=ara\d na, \og Celui qui tire [les hommes] de la d\'etresse\fg, est un tournant (CEC 23) dans l'histoire du site. Ce dieu\index{gnl}{dieu} est une forme de Bhairava\index{gnl}{Bhairava}. En tamoul, il est nomm\'e Ca\d t\d tain\=atar\index{gnl}{Cattainatar@Ca\d t\d tain\=atar}, \og le Seigneur \`a la chemise\fg. Car il porte comme chemise la peau de Vi\d s\d nu\index{gnl}{Visnu@Vi\d s\d nu}. C\=\i k\=a\b li\index{gnl}{Cikali@C\=\i k\=a\b li} semble reprendre son souffle.

\section{La vie actuelle du temple}

Le temple\index{gnl}{temple} de C\=\i k\=a\b li\index{gnl}{Cikali@C\=\i k\=a\b li} est aujourd'hui sous le patronage du monast\`ere\index{gnl}{monastère} de Tarumapuram.
Ce monast\`ere\index{gnl}{monastère} non brahmane\index{gnl}{brahmane}, situ\'e dans le delta de la K\=av\=eri\index{gnl}{Kaveri@K\=av\=eri}, se trouve dans un hameau du m\^eme nom, \`a l'est de la localit\'e de M\=ayavaram (ou Mayila\d tutu\b rai), \`a une quarantaine de kilom\`etres au sud de Citamparam\index{gnl}{Citamparam}.
Fond\'e au \textsc{xvi}\up{e} si\`ecle, il est un des hauts-lieux du \'Saiva Siddh\=anta\index{gnl}{Saiva@\'Saiva Siddh\=anta}.
Le chef\index{gnl}{chef} religieux actuel (\textit{ma\d t\=atipati}) du monast\`ere\index{gnl}{monastère} de Tarumapuram est nomm\'e Ca\d nmuka T\=ecika \~N\=a\b nacampanta Param\=ac\=ariya Cuv\=amika\d l. Il est le vingt-sixi\`eme chef religieux depuis le fondateur Kuru\~n\=a\b nacampantat\=ecikar. Il est \`a la t\^ete d'une institution religieuse dont le pouvoir \'economique et politique d\'epasse largement l'espace de son implantation. D\`es sa fondation\index{gnl}{fondation}, à la manière d'un temple\index{gnl}{temple}, ce monast\`ere\index{gnl}{monastère} a re\c cu des dotations de terre\index{gnl}{terre} dans diff\'erentes parties du Pays Tamoul\index{gnl}{Pays Tamoul}. Il appara\^it au \textsc{xix}\up{e} si\`ecle comme une puissance locale\index{gnl}{local} qui influence l'\'economie et la politique de sa r\'egion. Il vit encore aujourd'hui des revenus de ses biens mais d'une mani\`ere bien moins faste qu'il y a un si\`ecle (Cf. \textsc{Reiniche} 1985 et \textsc{Whashbrook} 1975).
Ce monast\`ere\index{gnl}{monastère} régit aujourd'hui vingt-six temple\index{gnl}{temple}s, ou plus exactement vingt-et-un, car il jouit seulement de droits\index{gnl}{droits!kattalai@\textit{ka\d t\d talai}} (\textit{ka\d t\d talai}) dans cinq temple\index{gnl}{temple}s. Il d\'el\`egue \`a quelques-uns de ses disciples renon\c cants (\textit{tampir\=a\b n}) l'administration de certains temple\index{gnl}{temple}s. \`A C\=\i k\=a\b li\index{gnl}{Cikali@C\=\i k\=a\b li} c'est le bureau du \textit{devasth\=anam} qui sert d'interm\'ediaire. Une des activit\'es principales du monast\`ere\index{gnl}{monastère} est l'enseignement. Celui-ci est dispens\'e au monast\`ere m\^eme ou dans des \'ecoles qu'il finance. On y enseigne le sanskrit, le tamoul, la musique\index{gnl}{musique}, le chant, etc. A l'heure actuelle le monast\`ere\index{gnl}{monastère} vient d'achever la contruction d'un gigantesque \'etablissement scolaire de lettres, \og Art College\fg, \`a Tarumapuram. Ainsi, cette institution religieuse qui a été un magnat de la finance est aussi un patron culturel qui \'etablit des \'ecoles, des biblioth\`eques et une maison d'\'edition. Gr\^ace \`a cette derni\`ere le monast\`ere\index{gnl}{monastère} a diffus\'e une litt\'erature religieuse sectaire\index{gnl}{sectaire} riche et diverse dont le \textit{talapur\=a\d nam}\index{gnl}{Purana@\textit{Pur\=a\d na}!\textit{talapur\=a\d nam}} de C\=\i k\=a\b li\index{gnl}{Cikali@C\=\i k\=a\b li}.

 \begin{figure}[!h]
  \centering
  \includegraphics[width=11cm]{docthese/plan2.jpg}
  \caption{Plan approximatif du temple principal de \'Siva.}
  \end{figure}

Le temple\index{gnl}{temple} est nomm\'e aujourd'hui d'après la divinit\'e principale, le \textit{li\.nga}\index{gnl}{linga@\textit{li\.nga}} Brahm\=a-pure\'svara, un des douze\index{gnl}{douze} toponymes traditionnels du site. Il est ouvert à l'Est (fig. 6.1). Une allée délimitant deux jardins (l'un au nord avec l'étable [1] \textit{pacuma\d tam} et l'autre au sud \textit{tirunantava\b nam}) mène au premier pavillon d'entrée qui marque l'entrée réelle du temple. Un portique hypostyle conduit au sanctuaire principal de \'Siva [A]. Il est flanqué au sud d'un petit bâtiment [2] inséré entre huit colonnes, abritant le char de procession. Derrière, dans l'angle sud-est, se trouvent le bureau du \textit{devasthana} [3], le \textit{ma\d n\d dapa} \og balan\c coire d'or\fg\ \textit{po\b n\=u\~njal} [4] qui en dehors des périodes festives sert de lieu de stockage et la cuisine [5] où sont préparées les offrandes de nourriture (\textit{naivedya}). Au nord du portique, un autre ma\d n\d apa \textit{\=u\~njal} [6] est destiné à recevoir les images mobiles (\textit{utsavam\=urti}), lors des cérémonies de la balan\c coire qui font partie des grandes fêtes annuelles. Pendant les autres périodes de l'année ce \textit{ma\d n\d dapa}, plus grand que le précédent, tel un grenier conserve les riz non décortiqués récoltés sur les terres du temple. Au centre du portique sont alignés dans l'allée, entre le pavillon d'entrée principal et l'entrée du temple de \'Siva, dans l'axe de la porte, une représentation de Ga\d e\'sa, l'autel (\textit{balip\=\i \d tha}), V\textsubring{r}\d sabha faisant face au sanctuaire de \'Siva, le mât à étendard (\textit{dhvaja}) et un tronc.
\noindent
L'accès au sanctuaire de \'Siva [A] s'effectue par une entrée flanquée à l'extérieur de deux images de Ga\d ne\'sa [A1] qui se présentent comme les gardiens. Le temple de \'Siva est encadré d'une enceinte dont les murs intérieurs sont longés de galeries à piliers, surélevées par un soubassement. Le corps principal est au centre de ce dispositif.
Monté sur un soubassement, il est constitué d'une première salle hypostyle [A2] d'où saillit au nord la cella de \'Siva dansant [A3] en bronze, ouverte face au sud. La seconde [A4] qui possède des piliers moins larges et plus espacés, est flanquée le long de son mur nord de la chambre à coucher (\textit{pa\d l\d liya\b rai}) [A5]. Dans l'angle sud-est sont placés les tambours frappés pour prévenir le dieu et alerter les dévots au début des cultes. Dans l'allée centrale de cette salle, dans l'axe de la porte de la cella, sont alignés un autel [A6], un disque solaire (\textit{s\=uryacakra}) [A7] et V\textsubring{r}\d sabha [A8] qui fait face à \'Siva. Deux creusets à feu (\textit{ku\d n\d da}) [A9] sont disposés de chaque côté de cet axe. Ensuite, se trouve un premier vestibule [A10] gardé par deux gardiens de la porte (\textit{dv\=arap\=ala}) qui encadrent l'entrée dans la pièce où se tiennent les dévots au moment des cérémonies. Au-delà, se trouvent un second vestibule [A11] où sont autorisés les dévots privilégiés et enfin, la cella principale (\textit{garbhag\textsubring{r}ha}) [A12] abritant le \textit{li\.nga} Brahm\=apure\'svara recouvert d'une multitude de guirlandes, de vêtements et entouré d'un cercle de flamme placé sur le mur du fond. Contre le mur nord du premier vestibule, un espace est aménagé pour ranger et stocker des images mobiles [A13]. Depuis ce même vestibule, une ouverture au sud permet d'accéder à une petite chapelle dédiée à Campantar [A14].
A ce corps principal est accolé à l'ouest, un bâtiment à étages [A15] dont l'accès s'effectue par un escalier qui se trouve à l'angle nord-ouest. Cette structure comporte un rez-de-chaussée massif, un premier niveau abritant T\=o\d niyappar et Periyan\=acciy\=ar (noms du couple figurant dans le CEC 2), et un second où loge Ca\d t\d tain\=atar.
Au premier étage se trouve la cella de \'Siva Tô\d niyappar assis sur un radeau avec P\=arvat\=\i\ Periyan\=acciy\=ar. T\=o\d niyappar, en stuc et de grande taille, a quatre bras non-armés. Il est honoré au niveau supérieur par deux dévots debout les mains jointes (\textit{a\~njali}). P\=arvat\=\i, plus petite, a deux bras. Deux porteuses de chasse-mouches sont situées au-dessus d'elle. Un déambulatoire autour de la cella permet la circumambulation par la droite. Sur les murs extérieurs de la cella est peint le mythe de T\=o\d niyappar.
Au second étage est placée la chapelle de Ca\d t\d tain\=atar dont la légende est peinte à la suite du mythe de T\=o\d niyappar. La statue de Ca\d t\d tain\=atar en bois est placée en hauteur, face au sud, dans une petite cella où seul l'officiant peut pénétrer par une petite porte située à l'est. Elle le représente avec deux bras et vêtu d'un manteau : la main droite fait le geste de l'enseignement (\textit{cinmudr\=a}) et la gauche s'appuie sur une massue.
Ainsi, le corps central du sanctuaire de \'Siva comporte quatre manifestations de ce dernier (Brahm\=apure\'svara, T\=o\b niyappar accompagné de la déesse, Ca\d t\d tain\=atar et \'Siva dansant) et une de Campantar.

Autour de la cella du \textit{li\.nga}, dans les niches des fa\c cades extérieures, sont placées diverses images dans le sens de la circumambulation par la droite : au sud Agastya, Ga\d ne\'sa et Ga\.ng\=avisarjanam\=urti, à l'ouest Li\.ngodbhavam\=urti, et au nord Brahm\=a, Bhik\d s\=a\d tana, Durg\=a et K\=alabhairava.
Au sud, Agastya [A16] debout, ventripotent, barbu, tient une cruche (\textit{kama\d n\d dalu}) de sa main gauche et de la droite fait le geste d'absence de crainte (\textit{abhayamudr\=a}). Vighne\'svara [A17] à tête d'éléphant, à quatre bras, est debout sur un piédestal lotiforme : ses deux mains supérieures portent le croc (\textit{a\.nku\'sa}) et le lasso (\textit{p\=a\'sa}), sa main droite principale tient l'une de ses défenses et sa gauche une pâtisserie (\textit{m\=odaka}) convoitée par la trompe. Ga\.ng\=avisarjanam\=urti \og celui qui laisse couler la Ga\.ng\=a\fg\ [A18] est debout accompagné de P\=arvat\=\i. Sa main droite supérieure tient en l'air une mèche de cheveux sur laquelle est assise Ga\.ng\=a les mains jointes. Sa main droite principale posée sur le menton de P\=arvat\=\i\ qui détourne son visage présente une scène de bouderie dans laquelle \'Siva, portant Ga\.ng\=a dans sa chevelure, tente d'amadouer P\=arvat\=\i\ jalouse. Dans une chapelle qui jouxte le corps central, placée à l'angle sud-ouest, ouverte au sud, est honoré \'Siva Dak\d si\d n\=am\=urti [A19].
Sur la fa\c cade extérieure ouest de la cella, difficilement accessible à cause de la proximité du temple à étages, mais pourtant couvert de pâte de santal, se trouve Li\.ngodbhavam\=urti [A20], \'Siva sortant de la colonne de feu qu'est le \textit{li\.nga}, debout, la main droite principale fait le geste d'absence de crainte.
Sur la fa\c cade nord, Brahm\=a [A21] est debout sur un piédestal lotiforme : trois de ses visages sont visibles, ses mains supérieures portent le rosaire et la cruche, sa main droite principale fait le geste d'absence de crainte, et l'autre est posée sur la base de sa cuisse. Ensuite, Bhik\d s\=a\d tana [A22] debout en position de marche (le talon droit légèrement relevé suggère ce mouvement) est nu. Il est échevelé. Ses mains supérieures portent le tambour-sablier (\textit{\d damaru}) à droite et un bâton (?) à gauche. Son bras droit est tendu vers une antilope et l'autre porte à hauteur de la taille la coupe crânienne (\textit{kap\=ala}). Sous cette main, se tient debout un gnome (\textit{bh\=uta}) portant sur la tête un bol. Durg\=a [A23] est debout sur une tête de buffle. Ses mains supérieures portent le disque (\textit{cakra}) et la conque (\textit{\'sa\.nkha}). Sa main droite principale fait le geste d'absence de crainte et sa gauche appuie sur la base de sa cuisse. Elle est en culte : elle est encadrée de lampes, recouverte de vermillon, de guirlandes de fleurs et d'un sari. Et enfin, K\=alabhairava [A24], debout, possède quatre bras. Ses mains supérieures portent le \textit{\d damaru} et le lasso, et les autres le trident et la calotte crânienne. Il est nu, orné d'un serpent pour ceinture et d'une longue guirlande d'os. Dans ses cheveux dressés, il porte le serpent, une tête humaine et le croissant de lune. Il a des crocs.

La galerie des murs intérieurs de l'enceinte du temple de \'Siva est élevée par un soubassement. Dans le sens de la circumambulation par la droite elle renferme
un lieu de stockage des véhicules de procession (\textit{yantra}) [A25],
une série des soixante-trois \textit{n\=ayanm\=ar} [A26] (leur identification se fait principalement par des inscriptions modernes),
une série des sept mères (\textit{saptam\=at\textsubring{r}k\=a}) [A27]\footnote{Elles sont assises la jambe gauche pliée et posée sur le siège. Elles ont chacune quatre bras, la main droite principale fait le geste d'absence de crainte (\textit{abhayamudr\=a}) et celle de gauche le geste de l'invitation à la faveur (\textit{ah\=ayavaradamudr\=a}). Brahm\=a\d n\=\i\ dont trois têtes sont visibles porte le rosaire et la cruche. M\=ahe\'svar\=\i\ porte la hachette et l'antilope. Celle qui est nommée Kaum\=ar\=\i, très endommagée, est difficilement identifiable. Vai\d s\d nav\=\i\ porte le disque et la conque. V\=ar\=ah\=\i\ a une tête de sanglier. Indr\=a\d n\=\i\ porte une arme tranchante (\textit{\d ta\.nka}) et le foudre (\textit{vajra}). C\=amu\d n\d d\=a, les cheveux échevelés, tient le \textit{\d damaru} et un serpent dressé.},
et une autre série de dévots shiva\"ites [A28]\footnote{Ces dévots shiva\"ites sont identifiés par les inscriptions modernes dans cet ordre: Nampi \=A\d n\d t\=ar Nampi, C\=ekki\b l\=ar, les quatre \og maîtres de la religion shivaïte\fg\ (\textit{camay\=ac\=ariyar}) que sont M\=a\d nikkav\=acakar, Cuntarar, Appar et Campantar, les quatre \og maîtres de la lignée shiva\"ite\fg\ (\textit{cant\=anakuruvar}) que sont Meyka\d n\d tacivam, Aru\d lnanticivam, Ma\b rai\~n\=a\b nacampantar et Um\=apaticivam, ainsi que Pa\d t\d tinattuppi\d l\d lai et Aru\d nakirin\=atar.}.
Ensuite, entre quatre piliers se trouve une chapelle [A29]. Une fresque murale représentant deux gardiens de porte, un autel au centre, un \textit{li\.nga} et une image de Ga\d ne\'sa à qui fait face sa monture le rat dans l'angle sud-ouest, la composent.
Se succèdent, dans la galerie ouest, de plus petites chapelles, ouvertes à l'est, dédiées à Vigne\'svara [A30] en pierre faisant face à un rat,
à Som\=askanda [A31] en bronze,
à un groupe de \textit{li\.nga} et d'autels [A32],
à Muttuca\d t\d tain\=atar [A33] en pierre, semblable à celui de la cella,
à Malaikkum\=arar [A34] à quatre bras, accompagné de ses deux femmes, devant un paon et un autel,
à Gajalak\d sm\=\i\ [A35], à quatre bras, arrosée par des éléphants.
 Dans la galerie nord, se trouve l'emplacement de la vente des tickets de cérémonie (\textit{p\=uj\=a}\index{gnl}{puja@\textit{p\=uj\=a}}) [A36], une autre image de Muttuca\d t\d tain\=atar [A37] et, le reste de la galerie est vide d'image : seules des strophes du \textit{T\=ev\=aram} sont inscrites sur les murs.
	Et enfin, la galerie nord-est renferme dans l'ordre :
un \textit{li\.nga} non-nommé [A38],
une série des neuf astres (\textit{navagraha}) [A39],
un \textit{li\.nga} nommé Dharmapur\=\i \'svara [A40],
un \textit{li\.nga} nommé Sarvabhuvane\'svara [A41] face à V\textsubring{r}\d sabha,
trois \textit{li\.nga} sans piédestal réunis sous l'appellation de Yugali\.nga [A42],
un \textit{li\.nga} Par\=a\'sara) [A43],
deux images de Bhairava [A44], à quatre bras, accompagnés de leur chien,
une image de S\=urya [A45],
et un \textit{li\.nga} nommé Skala [A46], non-identifié.
Dans la cour intérieure, au nord, se trouvent deux arbres du site (tam. \textit{talamaram}, skt. \textit{sthalav\textsubring{r}\d sa}) : bambous [A47] et une variété de jasmin couleur corail (\textit{pava\b lamalli}) [A48]. Une chapelle [A49] sans nom abrite un \textit{li\.nga}. Et, la chapelle de Ca\d n\d de\'svara [A50] est ouverte au sud, l'accès se fait par des marches à l'est. De là, un passage permet l'entrée dans le corps principal et mène le dévot devant les \textit{dv\=arap\=ala} de la seconde salle hypostyle.
\noindent
Au niveau du pavillon d'entrée sud du temple de C\=\i k\=ali la chapelle des huit Bhairava [7] s'élève sur un soubassement. Dans l'angle sud-ouest un abri [8], à l'abandon aujourd'hui, accueillait l'éléphant du temple. Le long du mur ouest de la deuxième enceinte se succèdent plusieurs petites chapelles.
Ga\d ne\'sa [9] en pierre fait face à un rat.
B\=ala Ga\d napati [10] est placé dans une niche du pavillon d'entrée.
Skanda [11] est assis sur un paon, il a quatre bras.
Un \textit{li\.nga} nommé K\=a\b lipur\=\i \'svara [12] est accompagné de Ga\d ne\'sa.
K\=a\b li Ka\d nan\=atar\footnote{Ka\d nan\=atar est un des soixante-trois dévots shiva\"ites. Né à C\=\i k\=ali, dans une famille brahmane, il consacre sa vie à honorer \'Siva. Il enseigne aux autres dévots les différents services quotidiens qu'il est possible de rendre à \'Siva dans le temple : la confection de guirlandes, le désherbage, etc.} [13] est un homme en \textit{a\~njali} vêtu d'un pagne, portant un rosaire au cou et un autre qui noue son chignon.
Et un autel abrite Ka\d nan\=atar [14].
\noindent
Dans l'angle nord-ouest de la deuxième enceinte est établie la chapelle de la déesse Tirunilain\=ayaki [C], ouverte à l'est, reliée au bassin [D] (dans l'angle nord-est) par un portique hypostyle. Dans l'allée centrale de ce portique, dans l'axe de la porte, sont placés un autel, un mât à étendard et V\textsubring{r}\d sabha. L'entrée est gardée par deux gardiennes. Elles ont chacune quatre bras, des yeux globuleux et un visage terrible. Le corps principal de la chapelle est formé d'une salle hypostyle qui donne dans la salle fermée où se tiennent les dévots lors des cérémonies et est gardée par deux images de \textit{dv\=ara\'sakti} et de Ga\d ne\'sa de chaque côté. La cella proprement dite abrite la déesse, Tirunilain\=ayaki \og la Dame du site\fg, qui se tient debout sur un piédestal lotiforme. Elle a quatre bras. Les galeries des murs intérieurs de l'enceinte formées sur le même modèle que le temple de \'Siva sont vides à l'exception des angles sud-ouest et nord-ouest qui re\c coivent respectivement une chapelle de deux images de Ga\d ne\'sa et une chapelle de Skanda. Ce dernier est debout face à un paon et a quatre bras.
Les niches des fa\c cades extérieures de la cella re\c coivent les représentations indifférenciées de cinq déesses dont l'identification s'effectue par leurs noms. Dans le sens de la circumambulation par la droite se trouvent : \'Sy\=amal\=a, \og la Noire\fg, et Icch\=a\'sakti au sud, J\~n\=ana\'sakti à l'ouest, Kriy\=a\'sakti et Durg\=a au nord.
%Les cinq niches des murs extérieurs (de l'avant-corps et du corps d'édifice) de la chapelle de la déesse à Darasuram auraient aussi accueilli cinq formes de la déesse . Mais il n'en reste aujourd'hui que trois : Icch\=a©akti, Jñ\=ana©akti et Kriy\=a©akti sur les murs du corps d'édifice. Elles sont debout, ont quatre bras et portent des attributs identiques. Ces déesses qui seraient des émanations de la divinité principale, appartiennent à une série de cinq ©akti : \=adi-Par\=a-Icch\=a-Jñ\=ana-Kriy\=a. Si les deux images manquantes de l'avant-corps, à Darasuram, étaient celles des deux ©akti supérieures (\=adi et Par\=a), comment expliquer alors qu'à Cîk\=ali au même emplacement (sur les murs de l'avant-corps) se trouvent deux images postérieures identifiées comme terribles ? F.L'Hernault conclut que ces niches, habitées par des ©akti semblables à l'origine, re\c coivent postérieurement les images iconographiquement différentes des m\=at?k\=a qui disparaissent des temples de \'Siva pour réintégrer partiellement les chapelles de la déesse. Mais, à Cîk\=ali, les m\=at?k\=a restent dans le temple de \'Siva et les cinq images des niches, bien qu'incluant nommément deux représentations de Durg\=a, sont identiques.
Près de la fa\c cade nord se trouve une petite chapelle consacrée à Ca\d n\d dike\'svar\=\i. Il y a un puits dans l'angle nord-est.
A l'extérieur de la chapelle de la déesse, au nord, entre le portique et le pavillon d'entrée ouest, une chapelle abrite une manifestation de Skanda, Ma\d n\d dapakum\=ara \og le Prince du \textit{ma\d n\d dapa}\fg\ [15], flanqué de ses deux femmes, à qui fait face un paon. Dans l'angle sud-est du portique hypostyle s'est développée une petite chapelle [16] dédiée à la déesse. Elle se tient debout, déhanchée, portant un lotus et devant elle se trouvent un autel et V\textsubring{r}\d sabha.
Le bassin Brahm\=at\=\i rtha, dans l'angle nord-est, est accessible par des escaliers de chaque côté. L'entrée principale sud est marquée par une arcade.

Et enfin, la chapelle de Campantar [B] dans la troisième enceinte, ouverte à l'est, possède les mêmes caractéristiques que le temple de \'Siva : un corps principal au centre d'un espace encadré de galeries à soubassement dont la partie sud-est re\c coit les douze peintures murales illustrant les douze mythes fondateurs du temple. Le c\^oté du mur nord du vestibule suivant la salle hypostyle abrite sur toute sa longueur une bibliothèque qui est aujourd'hui fermée à tous.

A l'extérieur du temple de C\=\i k\=a\b li, accolés aux murs extérieurs de la troisième enceinte se trouvent trois petites chapelles : la première, sur le mur est, est un simple abri recevant un trident [17], la seconde au sud est dédiée à Ga\d ne\'sa [18] et la dernière à Skanda [19] (fig. 6.1).

Le temple\index{gnl}{temple} employait en janvier 2007 cinquante-sept personnes dont six officiants principaux (\textit{kurukka\d l}), quatre aides (\textit{pir\=ama\d napi\d l\d lai}), deux surveillants (\textit{meykk\=aval} qui d\'etiennent les cl\'es du temple\index{gnl}{temple}), sept gardiens, six femme\index{gnl}{femme}s de m\'enage, un ma\c con, deux vendeurs de tickets de culte\index{gnl}{culte} (il faut acheter un ticket pour ordonner une \textit{p\=uj\=a}\index{gnl}{puja@\textit{p\=uj\=a}} personnelle ou particuli\`ere), six musiciens et neuf officier\index{gnl}{officier}s de bureau\footnote{\`A l'exception d'un gardien qui est pay\'e par le gouvernement, les autres gardiens sont pay\'es par le temple\index{gnl}{temple}. Les salaires mensuels s'\'etendent de cent quinze \`a trois mille quarante-cinq roupies indiennes. Un partie du salaire est aussi versée en nature, du riz\index{gnl}{riz} non d\'ecortiqu\'e ou cuit, provenant des terres\index{gnl}{terre} du temple\index{gnl}{temple}.}. Pr\`es de mille acres de terres\index{gnl}{terre} fertiles (\textit{na\b ncey}), exploit\'ees sous m\'etayage, appartiendraient au temple\index{gnl}{temple}. De plus, le temple\index{gnl}{temple} contr\^ole dix-huit templions sem\'es un peu partout dans la ville de C\=\i k\=a\b li\index{gnl}{Cikali@C\=\i k\=a\b li} et desservis par les officiants du temple\index{gnl}{temple}.
%Ainsi, ce temple se particularise par la pr\'esence de deux formes singuli\`eres de \'Siva (T\=o\d niyappar et Ca\d t\d tain\=atar\index{gnl}{Cattainatar@Ca\d t\d tain\=atar}) et par sa relation profonde avec le \textit{T\=ev\=aram} que l'on voit \`a travers la chapelle de Tiru\~n\=a\b nacampantar, demi-dieu\index{gnl}{dieu} \`a qui l'on voue un culte\index{gnl}{culte} \`a part enti\`ere.
%\section{Les rites}

 \begin{figure}[!h]
  \centering
  \includegraphics[width=6cm]{docthese/beskar.jpg}
  \caption{Un officier du bureau scelle le temple principal de \'Siva chaque soir, C\=\i k\=a\b li (cliché U. \textsc{Veluppillai}, 2005).}
  \end{figure}

Le \textit{li\.nga}\index{gnl}{linga@\textit{li\.nga}} b\'en\'eficie de six \textit{p\=uj\=a}\index{gnl}{puja@\textit{p\=uj\=a}} quotidiennes et compl\`etes (ondoiement, ornement, offrandes de nourriture, lumi\`ere et fumigation). La d\'eesse\index{gnl}{deesse@déesse} est honor\'ee apr\`es chaque \textit{p\=uj\=a}\index{gnl}{puja@\textit{p\=uj\=a}} effectu\'ee au \textit{li\.nga}\index{gnl}{linga@\textit{li\.nga}}. Mais T\=o\d niyappar\index{gnl}{Toniyappar@T\=o\d niyappar} et Ca\d t\d tain\=atar\index{gnl}{Cattainatar@Ca\d t\d tain\=atar} ne re\c coivent que quatre \textit{p\=uj\=a}\index{gnl}{puja@\textit{p\=uj\=a}} non compl\`etes. En effet, l'absence de dispositif aux \'etages pour \'evacuer les liquides ne permet pas leur ondoiement. De plus, T\=o\d niyappar\index{gnl}{Toniyappar@T\=o\d niyappar} et sa par\`edre sont en pl\^atre. Tiru\~n\=a\b nacampantar re\c coit deux petites \textit{p\=uj\=a}\index{gnl}{puja@\textit{p\=uj\=a}} quotidiennes.
Une \textit{p\=uj\=a}\index{gnl}{puja@\textit{p\=uj\=a}} hebdomadaire est offerte \`a Ca\d t\d tain\=atar\index{gnl}{Cattainatar@Ca\d t\d tain\=atar} et aux huit Bhairava\index{gnl}{Bhairava} tous les vendredis \`a partir de 22 heures. L'image\index{gnl}{image} de Ca\d t\d tain\=atar\index{gnl}{Cattainatar@Ca\d t\d tain\=atar} est ce jour-l\`a recouverte d'une p\^ate \`a base de civette.

La grande f\^ete\index{gnl}{fete@fête} annuelle du temple\index{gnl}{temple}, \textit{brahmotsava}, a lieu en \textit{cittirai} (avril-mai). Nous avons assist\'e \`a celle de 2004, du 24 avril au 14 mai. La veille au soir de la grande f\^ete\index{gnl}{fete@fête}, on commence par demander la permission \`a Ga\d ne\'sa\index{gnl}{Ganesa@Ga\d ne\'sa}, on sort les cinq image\index{gnl}{image}s de procession\index{gnl}{procession} (\textit{pa\~ncam\=urti}), on effectue la c\'er\'emonie purificatrice du site (\textit{v\=astu\'s\=anti}) puis on collecte de la terre\index{gnl}{terre} pour la germination (\textit{m\textsubring{r}tsa\d mgraha\d na}) et enfin, on proc\`ede au rite\index{gnl}{rite} de la germination des pousses (\textit{a\.nkur\=arpa\d na}).
Le premier jour, au matin, on hisse le drapeau (\textit{dhvaj\=aroha\d na}) et le soir, la premi\`ere procession\index{gnl}{procession} sort dans les rues du char.
Le deuxi\`eme jour est tr\`es populaire: le matin, on reproduit l'\'episode du lait\index{gnl}{lait}. Selon la légende\index{gnl}{legende@légende} Campantar\index{gnl}{Campantar} re\c coit le lait\index{gnl}{lait} de la d\'eesse\index{gnl}{deesse@déesse} \`a l'\^age de trois ans au bord du bassin du temple\index{gnl}{temple} et commence \`a chanter en l'honneur de \'Siva\index{gnl}{Siva@\'Siva} le premier hymne\index{gnl}{hymne} du \textit{T\=ev\=aram}. Le soir, le palanquin\index{gnl}{palanquin} de Campantar\index{gnl}{Campantar} part pour le temple\index{gnl}{temple} de K\=olakk\=a\index{gnl}{Kolakka@K\=olakk\=a}, \`a une dizaine de kilom\`etres, afin de recevoir des cymbale\index{gnl}{cymbale}s, comme dans la légende\index{gnl}{legende@légende}, et rentre \`a C\=\i k\=a\b li\index{gnl}{Cikali@C\=\i k\=a\b li} \`a l'aube. Ensuite les \textit{pa\~ncam\=urti} partent en procession\index{gnl}{procession}.
Le troisi\`eme jour est consacr\'e \`a l'\'episode de la victoire de Campantar\index{gnl}{Campantar} sur les ja\"in\index{gnl}{jain@ja\"in}s. Un jeu théâtral reproduit la conversion du roi\index{gnl}{roi} \textit{p\=a\d n\d dya}\index{gnl}{pandya@\textit{p\=a\d n\d dya}}\footnote{\`A plusieurs reprises, les dévots et les officiants du temple\index{gnl}{temple} ont appelé ce moment le \textit{ka\b luv\=e\b r\b ral} \og empalement\index{gnl}{empaler}\fg\ comme s'il y avait eu, dans un pass\'e r\'evolu mais m\'emorable, une repr\'esentation de cette condamnation par empalement\index{gnl}{empaler} des ja\"in\index{gnl}{jain@ja\"in}s décrite dans le \textit{Periyapur\=a\d nam}\index{gnl}{Periyapuranam@\textit{Periyapur\=a\d nam}}.}.
La procession\index{gnl}{procession} a lieu la nuit.
Le quatri\`eme jour, la procession\index{gnl}{procession} part sur des montures en argent.
Le cinqui\`eme jour, la procession\index{gnl}{procession} est effectu\'ee sur un \textit{capram}, gigantesque temple\index{gnl}{temple} en toile mont\'e sur des roulettes.
Le sixi\`eme jour c'est la procession\index{gnl}{procession} du \textit{Tirumu\b rai}\index{gnl}{Tirumurai@\textit{Tirumu\b rai}}. Le soir a lieu le mariage\index{gnl}{mariage} de la d\'eesse\index{gnl}{deesse@déesse} et de \'Siva\index{gnl}{Siva@\'Siva} suivi d'une procession\index{gnl}{procession} dans les rues.
Le septi\`eme jour est consacr\'e \`a la procession\index{gnl}{procession} de Bhik\d s\=a\d tana et des autres dieux sur des montures en bois.
Le huiti\`eme jour la procession\index{gnl}{procession} des image\index{gnl}{image}s se fait sur le grand char.
Le neuvi\`eme jour, les image\index{gnl}{image}s partent en simple procession\index{gnl}{procession}.
Le dixi\`eme jour est vou\'e \`a la procession\index{gnl}{procession} de \'Siva\index{gnl}{Siva@\'Siva} dansant\footnote{Elle n'a pas eu lieu \`a cause de la pluie en 2004.}. Ensuite, la c\'er\'emonie du bain final (\textit{t\=\i rtha}) a lieu dans le bassin et enfin, la descente du drapeau (\textit{avaroha\d na}) clôt la fête\index{gnl}{fete@fête} \og normative\fg.
Le onzi\`eme jour, la procession\index{gnl}{procession} se fait sans musiciens.
Le douzi\`eme jour, \'Siva\index{gnl}{Siva@\'Siva} et la d\'eesse\index{gnl}{deesse@déesse} avancent, face-\`a-face, dans la procession\index{gnl}{procession}.
Le treizi\`eme jour est la f\^ete\index{gnl}{fete@fête} des radeau\index{gnl}{radeau}x sur le bassin du temple\index{gnl}{temple}.
Le quatorzi\`eme jour est consacr\'e \`a la procession\index{gnl}{procession} de Ca\d n\d de\'svara avec les autres image\index{gnl}{image}s mobiles qui sont, ensuite, rang\'ees.
Enfin, le vingt-et-uni\`eme jour on prom\`ene Ca\d t\d tain\=atar\index{gnl}{Cattainatar@Ca\d t\d tain\=atar} autour de la premi\`ere enceinte.
\begin{figure}[!h]
  \begin{minipage}[c]{0.45\textwidth}
  \centering
  \includegraphics[height=5cm]{docthese/photoCIIKAALI/Copiedediscours.jpg}
  \caption{Discours inaugural du chef du monastère de Tarumapuram avant le don du lait dans la chapelle de Campantar, C\=\i k\=a\b li (cliché E. \textsc{Francis}, 2004).}
  \end{minipage}\hspace{1cm}
  \begin{minipage}[c]{0.45\textwidth}
  \centering
  \includegraphics[height=5cm]{docthese/photoCIIKAALI/dondulait.jpeg}
  \caption{Représentation sur palanquin du don du lait devant l'arche du bassin, deuxième jour de la grande f\^ete de \textit{Cittirai}, C\=\i k\=a\b li (cliché E. \textsc{Francis}, 2004).}
  \end{minipage}
\end{figure}
\begin{figure}[!h]
  \begin{minipage}[c]{0.45\textwidth}
  \centering
  \includegraphics[height=5cm]{docthese/photoCIIKAALI/festivalsuite017}
  \caption{L'\textit{\=Otuv\=ar} de C\=\i k\=a\b li pendant la procession du deuxième jour de la grande f\^ete, C\=\i k\=a\b li (cliché U. \textsc{Veluppillai}, 2005).}
  \end{minipage}\hspace{1cm}
  \begin{minipage}[c]{0.45\textwidth}
  \centering
  \includegraphics[height=5cm]{docthese/photoCIIKAALI/festivalsuite063}
  \caption{Départ en procession du \textit{Tirumu\b rai} avec l'\textit{\=Otuv\=ar} et ses élèves de Tarumapuram, C\=\i k\=a\b li (cliché U. \textsc{Veluppillai}, 2005).}
  \end{minipage}
\end{figure}

Ainsi, la f\^ete\index{gnl}{fete@fête} principale est marqu\'ee par deux points culminants qui sont l'\'episode du lait\index{gnl}{lait} et la procession\index{gnl}{procession} intra-muros de Ca\d t\d tain\=atar\index{gnl}{Cattainatar@Ca\d t\d tain\=atar}. Le chef\index{gnl}{chef} du monast\`ere\index{gnl}{monastère} de Tarumapuram est présent lors de ces deux journées qui rassemblent des milliers de dévot\index{gnl}{devot(e)@dévot(e)}s.
L'ann\'ee est scand\'ee par d'autres f\^etes\index{gnl}{fete@fête} : \textit{\=a\d tip\=u\b ram} d\'edi\'ee \`a la d\'eesse\index{gnl}{deesse@déesse} pendant dix jours au mois d'\textit{\=a\d ti} (juillet-ao\^ut); \textit{navar\=atri} c\'el\`ebrant la victoire des d\'eesse\index{gnl}{deesse@déesse}s sur les d\'emons sur dix jours en \textit{pura\d t\d t\=aci} (septembre-octobre)\footnote{Cf. \textsc{Fuller} \&\ \textsc{Logan} (1985) pour une étude de cette fête au temple Maturai.}; \textit{skanda\'sa\d s\d ti} reproduit la victoire de Skanda\index{gnl}{Skanda} pendant six jours en \textit{aippaci} (octobre-novembre)\footnote{Cf. \textsc{Clothey} (1969) pour une étude de cette fête.}; la f\^ete\index{gnl}{fete@fête} du mois de \textit{m\=arka\b li} (décembre-janvier), appel\'ee aussi la f\^ete\index{gnl}{fete@fête} de la balan\c coire (\textit{\=u\~ncal utsavam}), dure dix jours\footnote{Voir \textsc{L'Hernault} et \textsc{Reiniche} 1999 pour une \'etude des rites\index{gnl}{rite} et f\^etes\index{gnl}{fete@fête} du temple\index{gnl}{temple} de Tiruva\d n\d n\=amalai\index{gnl}{Tiruvannamalai@Tiruva\d n\d n\=amalai}. Nous pr\'eparons un travail d\'etaill\'e sur le calendrier festif du temple\index{gnl}{temple} de C\=\i k\=a\b li\index{gnl}{Cikali@C\=\i k\=a\b li}.}.

\`A C\=\i k\=a\b li\index{gnl}{Cikali@C\=\i k\=a\b li}, quatre groupes de dévot\index{gnl}{devot(e)@dévot(e)}s, form\'es en comit\'es, sont charg\'es d'organiser des c\'er\'emonies particuli\`eres en l'honneur des soixante-trois \textit{n\=aya\b nm\=ar}\index{gnl}{nayanmar@\textit{n\=aya\b nm\=ar}} et autres ma\^itres religieux le jour du \textit{nak\d satra} de leur mort, de Skanda\index{gnl}{Skanda} le jour du \textit{nak\d satra} de \textit{k\=\i rttikai}, de \'Siva\index{gnl}{Siva@\'Siva} lors des \textit{prado\d sa} bimensuels et des huit Bhairava\index{gnl}{Bhairava} le vendredi. Le culte\index{gnl}{culte} des Bhairava\index{gnl}{Bhairava} est actuellement tr\`es en vogue avec la figure centrale de Ca\d t\d tain\=atar\index{gnl}{Cattainatar@Ca\d t\d tain\=atar}.


\begin{center}
*
\end{center}

L'étude du CEC met en évidence l'importance de la localité dans le rayonnement\index{gnl}{rayonnement} du temple de C\=\i k\=a\b li à date ancienne. Par ailleurs, nous constatons qu'il y avait très probablement deux temples à T\=o\d nipuram: le temple de \'Siva-T\=o\d nipuramu\d taiy\=ar et le temple de Campantar-\=A\d lu\d taiyapi\d l\d laiy\=ar avaient une administration et un \og public\fg\ distints. Les officiers du royaume donnaient au temple de \'Siva et les assemblées villageoises brahmanes donnaient au temple de Campantar. Le patronage de ces temples, qui s'uniront à partir du \textsc{xiii}\up{e} siècle, est donc local. Les rois et leurs familles ne se sont pas intéressés à ce site qui, pourtant, est au c\oe ur de la \textit{bhakti} shivaïte tamoule. Le rayonnement\index{gnl}{rayonnement} géographique de ce temple bien que limité est perenne car depuis les premiers témoignages matériels sur ce site (\textsc{xii}\up{e} siècle) jusqu'à aujourd'hui C\=\i k\=a\b li est un temple en activité.

\chapter*{Le nouveau héros ou Conclusion}
\addcontentsline{toc}{chapter}{Le nouveau héros ou Conclusion}


Avec notre regard \og archéologique\fg\ sur le temple de C\=\i k\=a\b li et sur le poète Campantar nous avons revisité quelque peu la tradition des textes de \textit{bhakti} shivaïte tamouls anciens. L'\'etude des hymne\index{gnl}{hymne}s attribu\'es \`a Campantar\index{gnl}{Campantar} nous a permis de souligner leurs particularit\'es structurales et de soulever l'hypoth\`ese d'interpolations\index{gnl}{interpolation} concernant les poème\index{gnl}{poeme@poème}s \`a exercices rh\'etoriques, certains envois\index{gnl}{envoi} et nombre des strophes contenant des allusions biographiques\index{gnl}{biographie!biographique}. L'étude de la légende de Campantar à travers les textes hagiographiques, l'iconographie et l'épigraphie nous a permis de poser l'hypothèse que la \og Légende dorée\fg\ de l'enfant Campantar se développe à partir du \textsc{x}\up{e} siècle seulement. Nous avons aussi suggéré que les douze toponymes liés au site de C\=\i k\=a\b li résulteraient d'un \og bricolage\fg\ opéré au moment d'une compilation des hymnes du \textit{T\=ev\=aram} au \textsc{xii}\up{e} siècle.

Notre travail inédit sur ce site, analysant des textes littéraires et des textes épigraphiques, vient préciser l'histoire du shivaïsme au Pays Tamoul. Les textes littéraires magnifient ce site et son poète Campantar qui prennent une importance démesurée alors que les textes épigraphiques du site présentent un rayonnement\index{gnl}{rayonnement} restreint qui fait écho à une \textit{bhakti} locale, responsable, sans doute d'un certain essoufflement aux \textsc{xiv}\up{e}-\textsc{xv}\up{e} siècles. \`A la fin du \textsc{xvi}\up{e} siècle, le site semble se raviver avec l'installation\index{gnl}{installation d'une image} d'une nouvelle figure, Ca\d t\d tain\=atar\index{gnl}{Cattainatar@Ca\d t\d tain\=atar}.\\


Ca\d t\d tain\=atar\index{gnl}{Cattainatar@Ca\d t\d tain\=atar} est repr\'esent\'e debout, avec deux bras: la main droite tient une massue (\textit{gad\=a}) pointant le sol et la main gauche fait le geste\index{gnl}{geste} de l'enseignement (\textit{vy\=akhy\=anamudr\=a})\footnote{Notre description est fond\'ee sur l'observation \textit{in situ} de quelques image\index{gnl}{image}s de Ca\d t\d tain\=atar\index{gnl}{Cattainatar@Ca\d t\d tain\=atar} dans le Pays Tamoul\index{gnl}{Pays Tamoul} et sur l'examen de clich\'es appartenant \`a la phototh\`eque IFP/EFEO de Putucc\=eri\index{gnl}{Putucc\=eri}. Nous avons visit\'e de nombreux sites dans le sud du Pays Tamoul\index{gnl}{Pays Tamoul} \`a la recherche de Ca\d t\d tain\=atar\index{gnl}{Cattainatar@Ca\d t\d tain\=atar}: les temple\index{gnl}{temple}s de T\=ayum\=a\b nacuv\=ami et de Ka\b rku\d ti \`a Tirucci\index{gnl}{Tirucci dt.} qui appartiennent au monast\`ere\index{gnl}{monastère} de Tarumapuram, les temple\index{gnl}{temple}s de Ca\d t\d taiyappa\b n et de N\=akaikk\=ar\=o\d nam \`a N\=akapa\d t\d ti\b nam qui offrent une variante int\'eressante de Ca\d t\d tain\=atar\index{gnl}{Cattainatar@Ca\d t\d tain\=atar} \`a dix bras et enfin, les temple\index{gnl}{temple}s de Ka\.nkaiko\d n\d t\=a\b n, de Tiruppu\d taimarut\=ur et de Nellaiyappar \`a Tirunelv\=eli.}. Sa chevelure compos\'ee de m\`eches torsad\'ees forme un halo. Son sexe est visible. Et, Ca\d t\d tain\=atar poss\`ede des crocs. Enfin, il porte une longue chemise (\textit{ca\d t\d tai}), attribut fondamental qui lui donne son nom tamoul. Nous constatons parfois quelques variations. \`A Amp\=al (Na\b n\b nilam taluk), la chemise ne descend pas jusqu'aux chevilles. Le geste\index{gnl}{geste} de l'enseignement peut \^etre remplac\'e par le geste\index{gnl}{geste} de l'absence de crainte ou encore par un cr\^ane. Sa chevelure et ses crocs font de lui une divinit\'e terrible.

\begin{figure}[!h]
  \begin{minipage}[c]{0.45\textwidth}
  \centering
  \includegraphics[height=7cm]{docthese/cattaipourthese/acalpuram05}
  \caption{Ca\d t\d tain\=atar, galerie intérieure sud-est, temple de \'Sivalokaty\=age\'sa à \=Acc\=a\d lpuram (cliché U. \textsc{Veluppillai}, 2005).}
  \end{minipage}\hspace{1cm}
  \begin{minipage}[c]{0.45\textwidth}
  \centering
  \includegraphics[height=7cm]{docthese/cattaipourthese/Tiruvenkatu05}
  \caption{Ca\d t\d tain\=atar, mur sud, temple de Ve\.nk\=a\d tar à Tiruve\.nk\=a\d tu (cliché U. \textsc{Veluppillai}, 2005).}
  \end{minipage}
\end{figure}

Ca\d t\d tain\=atar\index{gnl}{Cattainatar@Ca\d t\d tain\=atar} est en effet un Bhairava\index{gnl}{Bhairava}, la forme que \'Siva\index{gnl}{Siva@\'Siva} prend pour d\'ecapiter la cinqui\`eme t\^ete de Brahm\=a. Parce qu'il a commis un brahmanicide \'Siva\index{gnl}{Siva@\'Siva} est maudit et doit endurer douze\index{gnl}{douze} ann\'ees d'errance le cr\^ane de Brahm\=a coll\'e dans sa paume. Au Pays Tamoul\index{gnl}{Pays Tamoul}, la forme de Bhairava\index{gnl}{Bhairava} est de plus en plus repr\'esent\'ee \`a partir de l'\'epoque \textit{c\=o\b la}. Ensuite, elle a \'et\'e associ\'ee aux gardiens de territoire (\textit{k\d setrap\=ala}) et fix\'ee dans l'angle nord-est des temple\index{gnl}{temple}s (Cf. \textsc{Adic\'eam} 1965a et 1965b). S'il existe plusieurs vari\'et\'es de Bhairava\index{gnl}{Bhairava} \`a quatre bras, celles poss\'edant une massue et deux bras, comme Ca\d t\d tain\=atar\index{gnl}{Cattainatar@Ca\d t\d tain\=atar}, sont rares. \textsc{Ladrech} (2002) propose une hypoth\`ese convaincante sur la formation iconographique de Ca\d t\d tain\=atar\index{gnl}{Cattainatar@Ca\d t\d tain\=atar}\footnote{\textsc{Ladrech} (2002: 185).}:

\scriptsize
\begin{quote}
A rather unusual iconographic type in Indian sculpture, met with in Andhra Pradesh and Tamil Nadu, shows the god Bhairava\index{gnl}{Bhairava} furnished with a big club held downwards. This attribute is more specifically associated with another form of \'Siva, Lakul\=\i \'sa, considered by some to be an \textit{avat\=ara} of \'Siva and regarded as a divine \textit{guru} by \'Saivites like P\=a\'supatas and K\=al\=amukhas. In Andhra Pradeh, where we find the earliest known image\index{gnl}{image}s of Bhairava with the club, we can notice some iconographic confusion between Bhairava and Lakul\=\i \'sa. In Tamil Nadu --- where we hardly meet any Lakul\=\i \'sa sculpture ---, image\index{gnl}{image}s of this club-handed Bhairava were carved from the Cola period onwards. A new iconographic form, called Ca\d t\d tain\=atar\index{gnl}{Cattainatar@Ca\d t\d tain\=atar}, was then conceived in the Tamil land. Holding the club in one hand and displaying the teaching gesture with the other, it shows Bhairava as a god who, at one and the same time, punishes and teaches, who --- just as Lakul\=\i \'sa who holds his club to preach the \'Saivite faith --- is the gardian of \textit{\'sivadharma} and the divine \textit{guru} showing men the path to salvation.
\end{quote}
\normalsize

Les textes d\'ecrivant Ca\d t\d tain\=atar\index{gnl}{Cattainatar@Ca\d t\d tain\=atar}, ou un Bhairava\index{gnl}{Bhairava} debout \`a deux bras tenant une massue, sont soit d\'epourvus de datation fiable soit relativement tardifs. Pour les textes sanskrits, nous ne nous r\'ef\'erons qu'\`a ceux donn\'es dans \textsc{Ladrech} (2002). Il y est clair que les textes, \textit{dhy\=ana\'sloka} ou \textit{stotra}, qui d\'ecrivent pr\'ecis\'ement Ca\d t\d tain\=atar\index{gnl}{Cattainatar@Ca\d t\d tain\=atar} sont tr\`es tardifs. Cependant, deux textes, du \textsc{xii-xiv}\up{e} si\`ecle, \'evoquent deux Bhairava\index{gnl}{Bhairava} qui ressemblent \`a notre Ca\d t\d tain\=atar\index{gnl}{Cattainatar@Ca\d t\d tain\=atar}.
Dans l'\textit{\=I\'s\=ana\'sivagurudeva-paddhati}, Va\d tukabhairava est identifi\'e comme un \textit{k\d setrap\=ala} \`a deux bras tenant une massue et un cr\^ane. Il est un enfant\index{gnl}{enfant} de huit ans.
Dans l'\textit{Uttarak\=ara\d n\=agama}, \=Apaduddh\=ara\d nabhairava poss\`ede deux bras: une main fait le geste\index{gnl}{geste} d'absence de crainte et l'autre tient un b\^aton. Il est petit et porte sur le dos la peau de Vi\d s\d nu\index{gnl}{Visnu@Vi\d s\d nu}. La ressemblance et la confusion iconographique entre Ca\d t\d tain\=atar\index{gnl}{Cattainatar@Ca\d t\d tain\=atar}, Va\d tukabhairava et \=Apaduddh\=ara\d nabhairava sont r\'esolues dans les textes tamouls dans lesquels tous ces noms sont attribu\'es \`a Ca\d t\d tain\=atar\index{gnl}{Cattainatar@Ca\d t\d tain\=atar} qui voit sa légende\index{gnl}{legende@légende} fix\'ee.

Dans le \textit{C\=\i k\=a\b littalapur\=a\d nam}\index{gnl}{Cikalittalapuranam@\textit{C\=\i k\=a\b littalapur\=a\d nam}}, compos\'e au milieu du \textsc{xviii}\up{e} si\`ecle par le poète\index{gnl}{poete@poète} Aru\d n\=acalakkavir\=ayar\index{gnl}{Arunacalakkavirayar@Aru\d n\=acalakkavir\=ayar} (1712-1779), la légende\index{gnl}{legende@légende} de Ca\d t\d tain\=atar\index{gnl}{Cattainatar@Ca\d t\d tain\=atar} est cont\'ee dans le chapitre 25 intitul\'e \og Va\d tukan\=ata\fg, contenant quarante strophes. Ce chapitre narre comment \'Siva\index{gnl}{Siva@\'Siva} en vint \`a porter la peau de Vi\d s\d nu\index{gnl}{Visnu@Vi\d s\d nu}. V\=amana, un des dix avatars de Vi\d s\d nu\index{gnl}{Visnu@Vi\d s\d nu}, prend la forme d'un nain brahmane\index{gnl}{brahmane} pour sauver l'univers de l'\textit{asura} Bali. V\=amana demande \`a Bali une terre\index{gnl}{terre} mesurant ses trois pas. Bali lui accorde ce don\index{gnl}{don}. Et, V\=amana, en faisant ses trois pas, grandit d\'emesur\'ement, conqu\'erant ainsi tous les mondes et d\'etruit Bali. Selon le \textit{pur\=a\d nam} du site, apr\`es la victoire, V\=amana demeure grand, arrogant et effrayant. Les dieux font appel à \'Siva\index{gnl}{Siva@\'Siva} pour appaiser la peur que leur inspire V\=amana. \'Siva\index{gnl}{Siva@\'Siva} prend la forme du jeune Va\d tuka, va \`a la rencontre de V\=amana, le frappe \`a la poitrine, le tue, fait une chemise de sa peau et une massue \`a partir de ses os. Plus tard, par compassion pour Lak\d sm\=\i\index{gnl}{Lak\d sm\=\i}, il redonne la vie à Vi\d s\d nu.
Ce texte du \textsc{xviii}\up{e} si\`ecle vient expliquer l'iconographie inhabituelle de Ca\d t\d tain\=atar\index{gnl}{Cattainatar@Ca\d t\d tain\=atar}.

\`A la diff\'erence de \textsc{Ladrech} (2002) qui sugg\`ere que Ca\d t\d tain\=atar\index{gnl}{Cattainatar@Ca\d t\d tain\=atar} fait son apparition sous Kulottu\.nga II\index{gnl}{Kulottu\.nga II} --- \'epoque \textit{c\=o\b la}\index{gnl}{cola@\textit{c\=o\b la}} durant laquelle de nombreux chercheurs pensent que le shivaïsme aurait men\'e un combat virulent contre le vishnouisme --- nous posons l'hypoth\`ese que cette forme ne vient au Pays Tamoul\index{gnl}{Pays Tamoul}, plus pr\'ecis\'ement \`a C\=\i k\=a\b li\index{gnl}{Cikali@C\=\i k\=a\b li}, qu'\`a l'extr\^eme fin du \textsc{xvi}\up{e} si\`ecle.

\begin{figure}[!h]
  \begin{minipage}[c]{0.45\textwidth}
  \centering
  \includegraphics[height=7cm]{docthese/cattaipourthese/gangaikontan,tirunelveli,nov06}
  \caption{Ca\d t\d tain\=atar, pilier face sud, temple de Kail\=asan\=atha à Ka\.nkaiko\d n\d t\=a\b n (cliché U. \textsc{Veluppillai}, 2006).}
  \end{minipage}\hspace{1cm}
  \begin{minipage}[c]{0.45\textwidth}
  \centering
  \includegraphics[height=7cm]{docthese/cattaipourthese/ciikaalipiliers,06sept79}
  \caption{Ca\d t\d tain\=atar, pilier d'entrée est, face sud, temple de C\=\i k\=a\b li (cliché U. \textsc{Veluppillai}, 2006).}
  \end{minipage}
\end{figure}

Ca\d t\d tain\=atar\index{gnl}{Cattainatar@Ca\d t\d tain\=atar}, ou plut\^ot \=Apaduddh\=ara\d nar, est mentionn\'e pour la premi\`ere fois \`a C\=\i k\=a\b li\index{gnl}{Cikali@C\=\i k\=a\b li} dans CEC 23, inscription datant de 1598 et enregistrant une donation pour effectuer son installation\index{gnl}{installation d'une image} et son grand ondoiement (\textit{mah\=abhi\d seka}). Nous pensons qu'il s'agit ici de la premi\`ere repr\'esentation datable de cette divinit\'e. Le texte pr\'ecise que cette donation est faite pour le m\'erite du R\=aja\textsubring{r}\d si Vi\d t\d tale\'svaracc\=o\b lak\=o\b n\=ar. Nous pouvons simplement établir un rapprochement entre cet individu, de par son nom, et le temple\index{gnl}{temple} de Vi\d t\d taladeva\index{gnl}{vittaladeva@Vi\d t\d taladeva} construit au \textsc{xv}\up{e} si\`ecle \`a Hampi. Notons aussi qu'un certain Vi\d t\d taladeva mah\=ar\=aja a été gouverneur provincial de l'empire \textit{vijayanagara} (\textit{mah\=ama\d n\d dale\'svara}) au \textsc{xvi}\up{e} si\`ecle. L'image\index{gnl}{image} install\'ee \`a C\=\i k\=a\b li\index{gnl}{Cikali@C\=\i k\=a\b li} est probablement celle qui se trouve actuellement dans la galerie int\'erieure ouest [A33]. Le dieu\index{gnl}{dieu} poss\`ede deux bras: une main tient une massue et l'autre fait le geste\index{gnl}{geste} de l'enseignement. Il porte une longue chemise. Sa chevelure forme un halo. Ses crocs et son sexe sont visibles.

Aujourd'hui, Ca\d t\d tain\=atar\index{gnl}{Cattainatar@Ca\d t\d tain\=atar} est une divinit\'e omnipr\'esente \`a C\=\i k\=a\b li\index{gnl}{Cikali@C\=\i k\=a\b li}. Le bureau administratif porte son nom: \og C\=\i k\=a\b li\index{gnl}{Cikali@C\=\i k\=a\b li} \'Sr\=\i ca\d t\d tain\=atasv\=ami devasth\=anam\fg. Ses repr\'esentations sont multiples. Une image\index{gnl}{image} en pierre se trouve comme nous l'avons dit plus haut dans la galerie ouest, une image\index{gnl}{image} métallique de procession\index{gnl}{procession} dans la galerie nord [A37], une image\index{gnl}{image} en bois au second \'etage, une image\index{gnl}{image} peinte dans la chapelle des huit Bhairava\index{gnl}{Bhairava} et enfin de nombreuses sculptures occupent la base des piliers de la salle hypostyle \`a l'entr\'ee principale est.
Ca\d t\d tain\=atar\index{gnl}{Cattainatar@Ca\d t\d tain\=atar} b\'en\'eficie de deux c\'er\'emonies principales dans le complexe. La premi\`ere, \textit{\'sukrav\=aram}, telle que prescrite ou d\'ecrite dans le \textit{talapur\=a\d nam}\index{gnl}{Purana@\textit{Pur\=a\d na}!\textit{talapur\=a\d nam}}, a lieu les soirs de vendredi. Elle d\'ebute \`a 22 heures. La \textit{p\=uj\=a}\index{gnl}{puja@\textit{p\=uj\=a}}, longue et sophistiqu\'ee, comprend l'ondoiement, l'ornementation, l'offrande de nourriture et de lumi\`ere sur l'autel de lotus situ\'e dans la galerie sud [A29], sous le regard de l'image\index{gnl}{image} en bois du second \'etage. Pendant la c\'er\'emonie \`a l'autel l'\textit{\=otuv\=ar}\index{gnl}{otuvar@\textit{\=otuv\=ar}} du temple\index{gnl}{temple} chante l'\textit{\=Apaduddh\=ara\d na m\=alai}, un poème\index{gnl}{poeme@poème} en tamoul de trente quatrains compos\'e par le dixi\`eme chef\index{gnl}{chef} du monast\`ere\index{gnl}{monastère} (1715-1770) de Tarumapuram au \textsc{xviii}\up{e} si\`ecle. Ensuite, les offrandes de nourriture et de lumi\`ere sont pr\'esent\'ees \`a l'image\index{gnl}{image} de procession\index{gnl}{procession} (qui a re\c cu l'ondoiement et l'ornement le matin). Enfin, l'image\index{gnl}{image} en bois est honor\'ee avec le chant de l'\textit{\=Apaduddh\=ara\d na m\=alai}. Le non-ondoiement de cette image\index{gnl}{image} s'explique par son mat\'eriau, le bois, et par l'absence de syst\`eme d'\'evacuation des liquides. L'ornementation consiste \`a appliquer de la civette sur le bois. L'offrande de lumi\`ere est pr\'ec\'ed\'ee par celle de nourriture qui est particuli\`ere; elle se compose de graines de tapioca au lait\index{gnl}{lait} (\textit{p\=ay\=acam}) et de beignet de lentilles (\textit{va\d tai}) qui symbolisent, selon l'officiant, le sang et la chair. Enfin, vers 1 ou 2 heures du matin, une fois \'Siva\index{gnl}{Siva@\'Siva} et P\=arvat\=\i\index{gnl}{Parvati@P\=arvat\=\i}\ couchés et la premi\`ere enceinte fermée, l'image\index{gnl}{image} peinte re\c coit nourriture et lumi\`ere dans la chapelle des huit Bhairava\index{gnl}{Bhairava} qui est la chambre \`a coucher de la divinit\'e.
La seconde c\'er\'emonie a lieu, depuis plus d'une cinquantaine d'ann\'ees maintenant, le dernier jour de la grande f\^ete\index{gnl}{fete@fête} annuelle, un vendredi. Ce jour est uniquement patronn\'e par les brahmane\index{gnl}{brahmane}s. Ils sont seuls habilit\'es \`a porter l'image\index{gnl}{image} de procession\index{gnl}{procession} autour de la premi\`ere enceinte. L'image\index{gnl}{image} de procession\index{gnl}{procession} de Ca\d t\d tain\=atar\index{gnl}{Cattainatar@Ca\d t\d tain\=atar} est dite trop puissante pour \^etre contr\^ol\'ee hors du temple\index{gnl}{temple}. Le matin l'\textit{utsavam\=urti} est apport\'ee \`a la chapelle des huit Bhairava\index{gnl}{Bhairava} o\`u elle est honor\'ee. Le soir, elle revient dans la galerie nord et le \textit{\'sukrav\=aram} d\'ebute.
\begin{figure}[!h]
 \begin{minipage}[c]{0.45\textwidth}
  \centering
  \includegraphics[height=8cm]{docthese/cattaipourthese/cikaliutsava05}
  \caption{Ca\d t\d tain\=atar, image de procession, temple de C\=\i k\=a\b li (cliché U. \textsc{Veluppillai}, 2005).}
  \end{minipage}\hspace{1cm}
  \begin{minipage}[c]{0.45\textwidth}
  \centering
 \includegraphics[height=8cm]{docthese/cattaipourthese/ampalutsava05}
  \caption{Ca\d t\d tain\=atar, image de procession, temple d'Amp\=al (cliché U. \textsc{Veluppillai}, 2005).}
  \end{minipage}
\end{figure}
\begin{figure}[!h]
 \begin{minipage}[c]{0.45\textwidth}
  \centering
  \includegraphics[height=5cm]{docthese/photoCIIKAALI/festivalsuite194}
  \caption{Ca\d t\d tain\=atar en procession dans le temple de \'Siva, C\=\i k\=a\b li (cliché U. \textsc{Veluppillai}, 2005).}
  \end{minipage}\hspace{1cm}
  \begin{minipage}[c]{0.45\textwidth}
  \centering
 \includegraphics[height=5cm]{docthese/cattaipourthese/plafond,Cattainatatpl,nagapattinam,nov06}
  \caption{Ca\d t\d tain\=atar à dix bras, peinture de plafond, temple de Ca\d t\d taiyapp\=a à N\=akapa\d t\d ti\b nam (cliché U. \textsc{Veluppillai}, 2006).}
  \end{minipage}
\end{figure}

La pr\'esence et la place actuelle de Ca\d t\d tain\=atar\index{gnl}{Cattainatar@Ca\d t\d tain\=atar} \`a C\=\i k\=a\b li\index{gnl}{Cikali@C\=\i k\=a\b li} r\'esultent, nous semble-t-il, du patronage du monast\`ere\index{gnl}{monastère} de Tarumapuram qui a pris, au fil du temps, cette divinité sous sa tutelle. Le corps religieux de Tarumapuram a commandit\'e le \textit{talapur\=a\d nam}\index{gnl}{Purana@\textit{Pur\=a\d na}!\textit{talapur\=a\d nam}} et l'un de ses chef\index{gnl}{chef}s a compos\'e l'\textit{\=Apaduddh\=ara\d na m\=alai}. Nous trouvons des sculptures de Ca\d t\d tain\=atar\index{gnl}{Cattainatar@Ca\d t\d tain\=atar} dans la plupart des temple\index{gnl}{temple}s appartenant au monast\`ere\index{gnl}{monastère} ou \`a proximit\'e de C\=\i k\=a\b li\index{gnl}{Cikali@C\=\i k\=a\b li}. Le monast\`ere\index{gnl}{monastère} de Tarumapuram a grandement contribu\'e \`a l'essor de cette divinit\'e au Pays Tamoul\index{gnl}{Pays Tamoul} en \'elaborant et r\'epandant son mythe\index{gnl}{mythe}. La relation de Ca\d t\d tain\=atar avec un ordre monastique\index{gnl}{ordre monastique} existait-elle d\'ej\`a \`a la fin du \textsc{xvi}\up{e} si\`ecle lorsqu'il a été install\'e à C\=\i k\=a\b li pour le m\'erite d'un R\=aja\textsubring{r}\d si, un chef\index{gnl}{chef} de monast\`ere\index{gnl}{monastère}?

Ca\d t\d tain\=atar\index{gnl}{Cattainatar@Ca\d t\d tain\=atar}, le nouveau héros\index{gnl}{heros@héros} de C\=\i k\=a\b li\index{gnl}{Cikali@C\=\i k\=a\b li}, \'eclipse, aujourd'hui, Campantar\index{gnl}{Campantar}.

\backmatter
\newpage
\chapter*{Bibliographie}
\addcontentsline{toc}{chapter}{Bibliographie}

\small

\def\author#1{#1}
\def\entry#1#2{\begin{list}{}{\labelwidth=5em\leftmargin=6em
 \topsep=0pt\parsep=0pt\itemsep=0pt} \item[#1\hfil] #2\end{list}}
%\raggedright

\def\refer#1{\leftskip=6em\parindent=-\leftskip {\small #1}\par\leftskip=0mm\parindent=6mm}

\def\author#1{{\small \noindent #1}}
\def\entry#1#2{\begin{list}{}{
 \labelwidth=4em
 \leftmargin=6em
 \topsep=0pt
 \parsep=0pt
 \itemsep=0pt}
 \item[{\small #1}\hfil] {\small #2\par}
\end{list}}


\subsection*{Abréviations bibliographiques}

\noindent
AA, Arts Asiatiques\\
%AION, Annali@@\\
BEFEO, Bulletin de l'\'Ecole fran\c caise d'Extr\^eme-Orient\\
BEI, Bulletin d'\'Etudes Indiennes\\
BSOAS, Bulletin of the School of Oriental and African Studies\\
EFEO, \'Ecole fran\c caise d'Extr\^eme-Orient\\
EHESS, École des Hautes Études en Sciences Sociales\\
HR, History of religion\\
IESHR, The Indian Economic and Social History Review\\
IFI, Institut fran\c cais d'Indologie\\
IFP, Institut fran\c cais de Pondichéry\\
%IJDL, International Journal of Dravidian Linguistics\\
JA, Journal Asiatique\\
JAS, Journal of Asian Studies\\
JAOS, Journal of the American Oriental Society\\
JESHO, Journal of the Economic and Social History of the Orient\\
OUP, Oxford University Press\\
PDI, Publications du Département d'Indologie\\
PEFEO, Publications de l'\'Ecole fran\c caise d'Extr\^eme-Orient\\
PIFI, Publications de l'Institut fran\c cais d'Indologie\\
PUF, Presse Universitaire de France\\
QJMS, Quarterly Journal of the Mythic Society\\
%WZKS, Wiener Zeitschrift f\"ur die Kunde s\"ud-undostasiens\\
%ZDMG Zeitschrift der Deutschen Morgenländischen Gesellschaft\\

\subsection*{Sources primaires}


\author{\textit{C\=\i k\=a\b littalapur\=a\d nam} d'\textsc{Aru\d n\=acalakkavir\=ayar}}
\entry{*1937} {[1887] C\=\i k\=a\b li\index{gnl}{Cikali@C\=\i k\=a\b li}: C\=\i k\=a\b li\index{gnl}{Cikali@C\=\i k\=a\b li} Kumara\b n Accukk\=u\d tu.}

\author{\textit{Cikalittala varal\=a\b ru}}
\entry{} {Tarumaiyatinam, patippu 1980, 1998 \&\ 2000.}

\author{\textit{Cilappatik\=aram} d'\textsc{I\d la\.nk\=ova\d tika\d l}}
\entry{*2001} {[1892] \textit{cilappatik\=aram}, éd. par U. V\=e. C\=amin\=ataiyar, Ce\b n\b nai: U. V\=e. C\=amin\=a-taiyar n\=ul nilaiyam.}
\entry{1989} {\textit{Cilappatikaram}, traduit du tamoul par R. S. \textsc{Pillai}, Thanjavur: Tamil University Press.}
\entry{1990} {\textit{Le roman de l'anneau}, traduit du tamoul par Alain \textsc{Daniélou} et R. S. \textsc{Desikan}, Paris: Gallimard, Unesco.}

\author{\textsc{K\=araikk\=alammaiy\=ar}}
\entry{1982} {\textit{Chants dévotionnels tamouls}, éd. et traduction par \textsc{Karavelane}, introduction par J. \textsc{Filliozat}, postface et index-glossaire par F. \textsc{Gros}, \textit{PIFI} 1, Pondichéry: IFI.}
\entry{1993} {\textit{The Hymns of Kaaraikkaal Ammaiyaar}, traduction par T. N. \textsc{Ramachandran}, Dharmapuram: International Institute of Saiva Siddhanta Research.}

%\author{\textsc{Kalidasa}}
%\entry{1967} {Meghaduta (Le Nuage Messager), traduit et annoté par R. H. Assier de Pompignan, Les Belles Lettres, Paris.}
%\entry{1980} {La naissance\index{gnl}{naissance} de Kumara (Kumarasa?bhava), traduit du sanskrit par Tubini B., Gallimard, Paris, réimp.}
\author{\textit{M\=uvarul\=a}}
\entry{*1992} {[1946] \textit{m\=uvarul\=a}, éd. et commentaire par U. V\=e. C\=amin\=ataiyar, Ce\b n\b nai: U. V\=e. C\=amin\=ataiyar n\=ul nilayam.}

\author{\textit{N\=al\=ayirativviyappirapantam}}
\entry{*2000} {[1973] \textit{nalayirativviyappirapantam}, éd. par \textsc{Vittuv\=a\b n Ki. V\=e\.nka\d tac\=ami Re\d t\d tiy\=ar}, Ce\b n\b nai: Tiruv\=e\.nka\d tatt\=a\b n tiruma\b n\b ram.}
\entry{*2002} {[1993] \textit{n\=al\=ayirativyappirapantam}, commentaire par \textsc{Es. Jekatra\d tcaka\b n}, Ce\b n\b nai: \=A\b lv\=arka\d l \=ayvu maiyam.}

\author{\textit{Parip\=a\d tal}}
\entry{1968} {\textit{Le Parip\=a\d tal, Texte tamoul}, introduction, traduction et notes par F. \textsc{Gros}, \textit{PIFI} 35, Pondich\'ery: IFI.}

\author{\textit{Periyapur\=a\d nam}\index{gnl}{Periyapuranam@\textit{Periyapur\=a\d nam}} de \textsc{C\=ekki\b l\=ar\index{gnl}{Cekkilar@C\=ekki\b l\=ar}}}
%\entry{2002} {Periyapur\=a\d nam e?a va?a?kum Tirutto??ar Pura?am (mulam), Kacittiruma?am, Tiruppa?anta?, réimp.}
%\entry{1985} {Periya Pura?am, Condensed English Version by Vanmikanathan G., Sri Ramakrishna Math, Madras.}
\entry{1990} {\textit{St. Sekkizhar's Periya Puranam}, traduction par T. N. \textsc{Ramachandran}, Tamil University Publication 121, vol. \textsc{i}, Thanjavur: Tamil University.}
\entry{1995} {\textit{St. Sekkizhar's Periya Puranam}, traduction par T. N. \textsc{Ramachandran}, Tamil University Publication 121-1, vol. \textsc{ii}, Thanjavur: Tamil University.}
\entry{*1975} {[1937] \textit{periya pur\=a\d nam e\b n\b num tirutto\d n\d tar pur\=a\d nam, tirumalaic carukkam - tillaiv\=a\b lanta\d nar carukkam}, éd. et commentaire par \textsc{Ci. K\=e. Cuppirama\d niya Mutaliy\=ar}, vol. \textsc{i}, v. 1-550, K\=oyamputt\=ur: K\=ovait tami\b lc ca\.nkam.}
\entry{*2001} {[1940] \textit{periya pur\=a\d nam e\b n\b num tirutto\d n\d tar pur\=a\d nam, ilaimalinta carukkam - mummaiy\=alulak\=a\d n\d tacarukkam}, éd. et commentaire par \textsc{Ci. K\=e. Cuppirama-\d niya Mutaliy\=ar}, vol. \textsc{ii}, v. 551-1265, K\=oyamputt\=ur: C\=ekki\b l\=ar\index{gnl}{Cekkilar@C\=ekki\b l\=ar} nilaiyam.}
\entry{*1997} {[1943] \textit{periya pur\=a\d nam e\b n\b num tirutto\d n\d tar pur\=a\d nam, tiruni\b n\b ra carukkam}, éd. et commentaire par \textsc{Ci. K\=e. Cuppirama\d niya Mutaliy\=ar}, vol. \textsc{iii-1}, v. 1266-1694, K\=oyamputt\=ur: C\=ekki\b l\=ar\index{gnl}{Cekkilar@C\=ekki\b l\=ar} nilaiyam.}
\entry{*1980} {[1946] \textit{tirutto\d n\d tar pur\=a\d nam e\b n\b num periya pur\=a\d nam}, éd. et commentaire par \textsc{Ci. K\=e. Cuppirama\d niya Mutaliy\=ar}, vol. \textsc{iii-2}, v. 1695-1898, K\=oyamputt\=ur: K\=ovait tami\b lc ca\.nkam.}
\entry{*1971} {[1949] \textit{tirutto\d n\d tar pur\=a\d nam e\b n\b num periya pur\=a\d nam}, éd. et commentaire par \textsc{Ci. K\=e. Cuppirama\d niya Mutaliy\=ar}, vol. \textsc{iv}, v. 1899-2530, K\=oyamputt\=ur: K\=ovait tami\b lc ca\.nkam.}
\entry{*1973} {[1950] \textit{tirutto\d n\d tar pur\=a\d nam e\b n\b num periya pur\=a\d nam, vampa\b r\=avariva\d n\d tuc carukkam}, éd. et commentaire par \textsc{Ci. K\=e. Cuppirama\d niya Mutaliy\=ar}, vol. \textsc{v}, v. 2531-3154, K\=oyamputt\=ur: K\=ovait tami\b lc ca\.nkam.}
\entry{*1969} {[1953] \textit{tirutto\d n\d tar pur\=a\d nam e\b n\b num periya pur\=a\d nam, vampa\b r\=avariva\d n\d tuc carukkam - v\=arko\d n\d ta va\b namulaiy\=a\d l carukkam}, éd. et commentaire par \textsc{Ci. K\=e. Cuppirama\d niya Mutaliy\=ar}, vol. \textsc{vi}, v. 3155-3747, K\=oyamputt\=ur: K\=ovait tami\b lc ca\.nkam.}
\entry{*1992} {[1954] \textit{tirutto\d n\d tar pur\=a\d nam e\b n\b num periya pur\=a\d nam, v\=arko\d n\d ta va\b namulaiy\=a\d l carukkam - mutal ve\d l\d l\=a\b naic carukkam mu\b lutum}, éd. et commentaire par \textsc{Ci. K\=e. Cuppirama\d niya Mutaliy\=ar}, vol. \textsc{vii}, v. 3748-4281, K\=oyamputt\=ur: K\=ovait tami\b lc ca\.nkam.}
\entry{2002} {\textit{c\=ekki\b l\=ar\index{gnl}{Cekkilar@C\=ekki\b l\=ar} cuv\=amika\d l aruliya periyapur\=a\d nam e\b na va\b la\.nkum tirutto\d n\d tar pur\=a\d nam}, Tiruppa\b nant\=a\d l: K\=acittiruma\d tam.}
\entry{2006} {\textit{The History of The Holy Servants of the Lord Siva. A translation of the Periya Pur\=a\d nam of C\=ekki\b l\=ar\index{gnl}{Cekkilar@C\=ekki\b l\=ar}}, traduction par Alastair \textsc{McGlashan}, Victoria, B. C. : Trafford Publishing.}

\author{\textit{Pu\b ran\=a\b n\=u\b ru}}
\entry{*1993} {[1936], U. V\=e. C\=amin\=ataiyar éd., Ce\b n\b nai: U. V\=e. C\=amin\=ataiyar n\=ul nilaiyam.}
\entry{*2002} {[1999] \textit{The Pu\b ran\=a\b n\=u\b ru, Four Hundred Songs of War and Wisdom, An Anthology of Poems from Classical Tamil}, éd. et traduction par \textsc{Hart} George L. \& \textsc{Heifetz} Hank, New Delhi: Penguin Books (première publication: Columbia University Press, 1999).}

\author{\textit{T\=ev\=aram}}
\entry{1984-85} {\textit{T\=ev\=aram hymne\index{gnl}{hymne}s \'siva\"ites du Pays Tamoul\index{gnl}{Pays Tamoul}}, 2 vols., \'edition par T. V. \textsc{Gopal Aiyar}, introduction par F. \textsc{Gros}, \textit{PIFI} 68.1-2, Pondich\'ery: IFI.}
%\entry{1995} {T\=ev\=aram, mutal tirumu\b rai, Tiruv\=ava\d tutu\b rai\index{gnl}{Avatuturai@\=Ava\d tutu\b rai!Tiruv\=ava\d tutu\b rai} \=ati\d nam.}
\entry{2007} {\textit{Digital T\=ev\=aram}, éd. électronique par J.-L.\textsc{Chevillard} et S. A. S. \textsc{Sarma} de la traduction anglaise du \textit{T\=ev\=aram} par V. M. \textsc{Subramanya Aiyar}, \textit{Collection Indologie} 103, Pondichéry: IFP/EFEO.}

\author{\textit{Tirumu\b rai}}
\entry{*1997} {[1953] \textit{tiru\~n\=a\b nacampanta cuv\=amika\d l arulicceyta t\=ev\=arat tiruppatika\.nka\d l, mutal tirumu\b rai}, vol. \textsc{i}, éd. par \'Sr\=\i la\'sr\=\i\ \textsc{K\=aciv\=aci Muttukkum\=arac\=ami tampir\=a\b n Cuv\=amika\d l}, note explicative par \textsc{Ca. Ta\d n\d tap\=a\d ni T\=ecikar} et commentaire par \textsc{Vi. C\=a. Kuruc\=ami T\=ecikar}, Mayil\=a\d tutu\b rai: \~N\=a\b nacam-pantam patippakam, Tarumai \=at\=\i \b nam.}
\entry{*1997} {[1954] \textit{iru\~n\=a\b nacampanta cuv\=amika\d l arulicceyta t\=ev\=arat tiruppatika\.nka\d l, i-ra\d n\d t\=am tirumu\b rai}, vol. \textsc{ii}, éd. par Vittuv\=a\b n \'Sr\=\i mat \textsc{Mau\b na Mak\=ali\.nka Tampir\=a\b n Cuv\=amika\d l}, note explicative par \textsc{Cu. M\=a\d nikkav\=acaka Mutaliy\=ar} et commentaire par \textsc{Vi. C\=a. Kuruc\=ami T\=ecikar}, Mayil\=a\d tutu\b rai: \~N\=a\b nacampantam patippakam, Tarumai \=at\=\i \b nam.}
\entry{*1997} {[1955] \textit{tiru\~n\=a\b nacampanta cuv\=amika\d l arulicceyta t\=ev\=arat tiruppatika\.nka\d l, m\=u-\b n\b r\=am tirumu\b rai}, vol. \textsc{iii}, éd. par \'Sr\=\i mat \textsc{Po\b n\b nampala Tampir\=a\b n Cuv\=ami-ka\d l}, note explicative par \textsc{A. Kantac\=amippi\d l\d lai} et commentaire par \textsc{K\=o-mati C\=uriyam\=urtti}, Mayil\=a\d tutu\b rai: \~N\=a\b nacampantam patippakam, Tarumai \=at\=\i \b nam.}
\entry{*1997} {[1957] \textit{tirun\=avukkaracu cuv\=amika\d l arulicceyta t\=ev\=arat tiruppatika\.nka\d l, n\=a\b n-k\=am tirumu\b rai}, vol. \textsc{iv}, éd. par Vittuv\=a\b n \'Sr\=\i mat \textsc{Ir\=amali\.nka Tampir\=a\b n Cuv\=amika\d l}, note explicative par \textsc{Cu. M\=a\d nikkav\=acaka Mutaliy\=ar} et commentaire par \textsc{Ti. V\=e. K\=op\=alayyar}, Mayil\=a\d tutu\b rai: \~N\=a\b nacampantam patippakam, Tarumai \=at\=\i \b nam.}
\entry{*1997} {[1961] \textit{tirun\=avukkaracu cuv\=amika\d l arulicceyta t\=ev\=arat tiruppatika\.nka\d l, aint\=am tirumu\b rai}, vol. \textsc{v}, éd. par \'Sr\=\i mat \textsc{Kum\=arac\=ami Tampir\=a\b n Cuv\=amika\d l}, note explicative par \textsc{Ka. Vaccirav\=el Mutaliy\=ar} et \textsc{Vi. C\=a. Kuruc\=ami T\=ecikar} et commentaire par \textsc{Co. Ci\.nk\=arav\=ela\b n}, Mayil\=a\d tutu\b rai: \~N\=a\b na-campantam patippakam, Tarumai \=at\=\i \b nam.}
\entry{*1997} {[1963] \textit{tirun\=avukkaracu cuv\=amika\d l arulicceyta t\=ev\=arat tiruppatika\.nka\d l, \=a\b r\=am tirumu\b rai}, vol. \textsc{vi}, éd. par \'Sr\=\i mat \textsc{Kantac\=ami Tampir\=a\b n Cuv\=amika\d l}, note explicative par \textsc{Ci. Aru\d naiva\d tiv\=el Mutaliy\=ar} et commentaire par \textsc{Ti. V\=e. K\=op\=alayyar}, Mayil\=a\d tutu\b rai: \~N\=a\b nacampantam patippakam, Tarumai \=at\=\i \b nam.}
\entry{*1997} {[1963] \textit{cuntaram\=urtti cuv\=amika\d l arulicceyta t\=ev\=arat tiruppatika\.nka\d l, \=e\b l\=am tirumu\b rai}, vol. \textsc{vii}, éd. par \'Sr\=\i mat \textsc{Tirun\=avukkaracut Tampir\=a\b n Cuv\=a-mika\d l}, note explicative et commentaire par \textsc{Ci. Aru\d naiva\d tiv\=elu Mutaliy\=ar}, Mayil\=a\d tutu\b rai: \~N\=a\b nacampantam patippakam, Tarumai \=at\=\i \b nam.}
\entry{*1997} {[1966] \textit{m\=a\d nikkav\=acaka cuv\=amika\d l arulicceyta tiruv\=acakam, e\d t\d t\=am tirumu\b rai}, vol. \textsc{viii-1}, éd. par \'Sr\=\i mat \textsc{Cuntaram\=urtti Tampir\=a\b n Cuv\=amika\d l}, note explicative par \textsc{Ci. Aru\d naiva\d tiv\=elu Mutaliy\=ar} et commentaire par \textsc{Em. Civaccantira\b n}, Mayil\=a\d tutu\b rai: \~N\=a\b nacampantam patippakam, Tarumai \=at\=\i \b nam.}
\entry{*1997} {[1977] \textit{m\=a\d nikkav\=acaka cuv\=amika\d l arulicceyta tirukk\=ovaiy\=ar e\b na va\b la\.nkum Tirucci\b r\b rampalakk\=ovaiy\=ar, e\d t\d t\=am tirumu\b rai}, vol. \textsc{viii-2}, éd. par \'Sr\=\i mat \textsc{Cuntaram\=urtti Tampir\=a\b n Cuv\=amika\d l}, Mayil\=a\d tutu\b rai: \~N\=a\b nacampantam patippakam, Tarumai \=at\=\i \b nam.}
\entry{*1997} {[1969] \textit{tirum\=a\d likaitt\=evar mutaliya o\b npati\b nmar arulicceyta tiruvicaipp\=a tiruppall\=a\d n\d tu, o\b npat\=am tirumu\b rai}, vol. \textsc{ix}, éd. par \'Sr\=\i mat \textsc{Tiru\~n\=a\b nacampanta Tampir\=a\b n Cuv\=amika\d l}, note explicative par \textsc{Ci. Aru\d naiva\d tiv\=elu Mutaliy\=ar} et commentaire par \textsc{Ti. V\=e. K\=op\=alayyar}, Mayil\=a\d tutu\b rai: \~N\=a\b na-campantam patippakam, Tarumai \=at\=\i \b nam.}
\entry{*1997} {[1974] \textit{tirum\=ulat\=eva n\=aya\b n\=ar arulicceyta tirumantiram\=alaiy\=akiya tirumantiram, patt\=am tirumu\b rai}, vol. \textsc{x-1}, éd. par Vittuv\=a\b n \'Sr\=\i mat \textsc{Mau\b na Mak\=ali\.n-ka Tampir\=a\b n Cuv\=amika\d l}, note explicative et commentaire par \textsc{Ci. Aru-\d naiva\d tiv\=elu Mutaliy\=ar}, Mayil\=a\d tutu\b rai: \~N\=a\b nacampantam patippakam, Tarumai \=at\=\i \b nam.}
\entry{*1997} {[1984] \textit{tirum\=ulat\=eva n\=aya\b n\=ar arulicceyta tirumantiram\=alaiy\=akiya tirumantiram, patt\=am tirumu\b rai}, vol. \textsc{x-2}, éd. par Vittuv\=a\b n \'Sr\=\i mat \textsc{Kantac\=ami Tampir\=a\b n Cuv\=amika\d l}, note explicative et commentaire par \textsc{Ci. Aru\d nai-va\d tiv\=elu Mutaliy\=ar}, Mayil\=a\d tutu\b rai: \~N\=a\b nacampantam patippakam, Tarumai \=at\=\i \b nam.}
\entry{*1997} {[1995] \textit{tirum\=ulat\=eva n\=aya\b n\=ar arulicceyta tirumantiram\=alaiy\=akiya tirumantiram, patt\=am tirumu\b rai}, vol. \textsc{x-3}, éd. par \'Sr\=\i mat \textsc{Tiru\~n\=a\b nacampanta Tampir\=a\b n Cuv\=amika\d l}, note explicative et commentaire par \textsc{Ci. Aru\d nai-va\d tiv\=elu Mutaliy\=ar}, Mayil\=a\d tutu\b rai: \~N\=a\b nacampantam patippakam, Tarumai \=at\=\i \b nam.}
\entry{1995} {\textit{tiruv\=alav\=ayu\d taiy\=ar mutaliya pa\b n\b niruvar aruliya pati\b no\b n\b r\=am tirumu\b rai}, vol. \textsc{xi}, éd. par Vittuv\=a\b n \'Sr\=\i mat \textsc{Cuv\=amin\=ata Tampir\=a\b n Cuv\=amika\d l}, note explicative et commentaire par \textsc{Ci. Aru\d naiva\d tiv\=elu Mutaliy\=ar}, Mayil\=a-\d tutu\b rai: \~N\=a\b nacampantam patippakam, Tarumai \=at\=\i \b nam.}
\entry{2002} {\textit{pati\b no\b n\b r\=am tirumu\b rai}, Tiruppa\b nant\=a\d l: K\=acittiruma\d tam.}

\author{\textit{Tirumuruk\=a\b r\b rupa\d tai}}
\entry{1973} {\textit{Un texte de la religion Kaum\=ara, le Tirumuruk\=arrupa\d tai}, traduction par J. \textsc{Filliozat}, \textit{PIFI} 49, Pondichéry: IFI.}

\author{\textit{Tirupp\=avai} d'\textsc{\=A\d n\d t\=a\d l}}
\entry{1972} {\textit{Un texte tamoul de dévotion vishnouite\index{gnl}{vishnouite}, le Tirupp\=avai d'\=A\d n\d t\=a\d l}, traduction par J. \textsc{Filliozat}, \textit{PIFI} 45, Pondich\'ery: IFI.}

\author{\textit{Tiruv\=acakam}}
\entry{2001} {\textit{Tiruvaachakam}, traduction par T. N. \textsc{Ramachandran}, Ce\b n\b nai: International Institute of Tamil Studies.}

%\author{\textsc{Sankaracarya}}
%\entry{1987} {Saundaryalahari, with text, and transliteration, translation and notes based on Lak?midhara's commentary by Swami Tapasyananda, Sri Ramakrishna Math, Madras, 1987.}

%\author{\textit{Svacchandatantram}}
%\entry{1927} {with commentary by Kshemaraja, vol. IV, Kashmir Series of Texts and Studies n. XLVIII, Bombay.}
\subsection*{Sources épigraphiques}

\author{\textit{Annual Reports on (South) Indian Epigraphy}}
\entry{1885-1997} {New Delhi: Archaeological Survey of India.}

\author{\textit{\=Ava\d nam}}
\entry{1993-2006} {Journal of the Tamil Nadu Archaeological Society, Tanjavur: Tamil University.}

\author{\textit{Epigraphia Indica}}
\entry{1892-1992} {42 vols., Calcutta/New Delhi: Archaeological Survey of India.}

\author{\textit{Inscriptions (texts) of the Pudukottai State}}
\entry{*2002} {[1929] Chennai: Government Museum}

\author{\textit{(L')épigraphie de Vijayanagar du début à 1377}}
\entry{\textsc{voir}} {\textsc{Filliozat} V.}

\author{\textit{Ka\b n\b niy\=akumarik kalve\d t\d tuka\d l}}
\entry{1979} {éd. par \textsc{Ir\=a. N\=akac\=ami}, vol. 4 et 5, Ce\b n\b nai: tami\b ln\=a\d tu aracu tolporu\d l \=ayvuttu\b rai.}

\author{\textit{Meykk\=\i rttika\d l}}
\entry{\textsc{voir}} {\textsc{Cuppirama\d niyam, Pu.}}

\author{\textsc{Nagaswamy}, R.}
\entry{2005} {\textit{u\.nka\d l \=ur kalve\d t\d tut tu\d naiva\b n, Path way to the Antiquity of your Village}, Chennai: Tamil Arts Academy.}

\author{\textit{Na\b n\b nilam kalve\d t\d tuka\d l}}
\entry{1979-1980} {éd. par \=A. \textsc{Patm\=avati}, sous la direction de \textsc{Ir\=a. N\=akac\=ami}, 3 vols., Ce\b n\b nai: tami\b ln\=a\d tu aracu tolporu\d l \=ayvuttu\b rai.}

\author{\textit{Perumukkal kalve\d t\d tuka\d l}}
\entry{1998} {éd. par \textsc{K\=acin\=ata\b n} et \textsc{Ara. Vacantakaly\=a\d ni}, Ce\b n\b nai: tami\b ln\=a\d tu aracu tolporu\d l \=ayvuttu\b rai.}

\author{\textit{Putucc\=eri m\=anilakkalve\d t\d tukka\d l, Pondicherry Inscriptions, Part I}}
\entry{2006} {\textit{Introduction and Texts with Notes}, compilation par \textsc{Bahour S.
Kuppusamy}, éd. par \textsc{G. Vijayavenugopal}, préface par Leslie C. \textsc{Orr}, Collection Indologie n°83.1, Pondichéry: IFP/EFEO.}

\author{\textit{Putucc\=eri m\=anilakkalve\d t\d tukka\d l, Pondicherry Inscriptions, Part II}}
\entry{2010} {\textit{Translation, appendices, glossary and phrases}, compilation par \textsc{Bahour S. Kuppusamy}, éd. par \textsc{G. Vijayavenugopal}, préface par Emmanuel \textsc{Francis} et Charlotte \textsc{Schmid}, Collection Indologie n°83.2, Pondichéry: IFP/EFEO.}

\author{\textit{South Indian Inscriptions}}
\entry{1890-2001} {27 vols., Madras/New Delhi: Archaeological Survey of India.}

\author{\textit{South Indian Temple Inscriptions}}
\entry{\textsc{voir}} {\textsc{Subramaniam}, T. N.}

\author{\textit{T\=amaraipp\=akkam kalve\d t\d tuka\d l}}
\entry{1999} {éd. par \textsc{Cu. Ir\=acak\=op\=al}, \textsc{\=A. Patm\=avati} et \textsc{Ara. Vacantakaly\=a\d ni}, sous la direction de \textsc{Ku. T\=am\=otara\b n}, Ce\b n\b nai: tami\b ln\=a\d tu aracu tolporu\d l \=ayvuttu\b rai.}

\author{\textit{Tami\b lk kalve\d t\d tuc collakarati. Glossary of Tamil Inscriptions}}
\entry{\textsc{voir}} {\textsc{Subbarayalu}, Y.}

\author{\textit{Tarumapuri kalve\d t\d tuka\d l}}
\entry{1975} {éd. par \textsc{Ir\=a. N\=akac\=ami}, vol. 1, Ce\b n\b nai: tami\b ln\=a\d tu aracu tolporu\d l \=ayvuttu\b rai.}

\author{\textit{Tiruttu\b raipp\=u\d n\d tik kalve\d t\d tuka\d l}}
\entry{1978} {éd. par \textsc{Ir\=a. N\=akac\=ami}, Ce\b n\b nai: tami\b ln\=a\d tu aracu tolporu\d l \=ayvuttu\b rai.}

\author{\textit{Tiruvala\~ncu\b li kalve\d t\d tuka\d l}}
\entry{2001} {éd. par \textsc{\=A. Patm\=avati}, sous la direction de \textsc{A. Aptulmaj\=\i t}, Ce\b n\b nai: tami\b ln\=a\d tu aracu tolliyal tu\b rai.}

\author{\textit{Tiruv\=\i \b limi\b lalaik kalve\d t\d tuka\d l}}
\entry{1994} {éd. par \textsc{\=A. Patm\=avati}, sous la direction de \textsc{K\=acin\=ata\b n}, Ce\b n\b nai: tami\b ln\=a\d tu aracu tolporu\d l \=ayvuttu\b rai.}

\author{\textit{(A) Topographical list of Inscriptions in the Tamil Nadu and Kerala states}}
\entry{\textsc{voir}} {\textsc{Mahalingam T. V.}}

\section*{Sources secondaires}

\author{\textsc{Adic\'eam}, Emmanuel}
\entry{1966} {\textit{La géographie de l'irrigation dans le Tamilnad}. Paris: EFEO.}

\author{\textsc{Adic\'eam}, Marguerite E.}
\entry{1965a} {\og Les image\index{gnl}{image}s de \'Siva dans l'Inde du Sud: \textsc{ii}. Bhairava \fg,\textit{AA} 11-2, p. 23-44.}
\entry{1965b} {\og Les image\index{gnl}{image}s de \'Siva dans l'Inde du Sud: \textsc{iii} et \textsc{iv}. Bhik\d s\=a\d tanam\=urti et Ka\.nk\=alam\=urti\fg, \textit{AA} 12, p. 83-112.}
\entry{1966} {\og Les image\index{gnl}{image}s de \'Siva dans l'Inde du Sud: \textsc{v}. Harihara\fg, \textit{AA} 13, p. 83-98.}
\entry{1968} {\og Les image\index{gnl}{image}s de \'Siva dans l'Inde du Sud: \textsc{vi}. Ardhan\=ar\=\i \'svara\fg, \textit{AA} 17, p. 143-172.}
\entry{1969} {\og Les image\index{gnl}{image}s de \'Siva dans l'Inde du Sud: \textsc{vii}. V\textsubring{r}\d sav\=ahanam\=urti\fg, \textit{AA} 19, p. 85-106.}
\entry{1970} {\og Les image\index{gnl}{image}s de \'Siva dans l'Inde du Sud: \textsc{viii}, \textsc{ix} et \textsc{x}. Kevala-, Um\=asahita- et \=Ali\.nga-Candra\'sekharam\=urti\fg, \textit{AA} 21, p. 41-70.}
\entry{1971} {\og Les image\index{gnl}{image}s de \'Siva dans l'Inde du Sud: \textsc{xi}. P\=a\'supatam\=urti\fg, \textit{AA} 24, p. 23-50.}
\entry{1973} {\og Les image\index{gnl}{image}s de \'Siva dans l'Inde du Sud: \textsc{xii}, \textsc{xiii} et \textsc{xiv}. Sukh\=asana-, Um\=asahitasukh\=asana-, Um\=amahe\'svaram\=urti\fg, \textit{AA} 28, p. 63-101.}
\entry{1976} {\og Les image\index{gnl}{image}s de \'Siva dans l'Inde du Sud: \textsc{xv}. Ga\.ng\=adharam\=urti\fg, \textit{AA} 32, p. 99-138.}

\author{\textsc{Ali}, Daud}
\entry{2000} {\og Royal Eulogy as World History, Rethinking Copper-plate Inscriptions in C\=o\b la India\fg, dans R.~Inden, J.~Walters et D.~Ali (\'ed.), \textit{Querying the Medieval, Texts and the History of Practices in South Asia}. New York: OUP, p. 165-235.}
\entry{2007} {\og The service\index{gnl}{service} of the Chola court: a study of the term \textit{ve\d lam} in Tamil inscriptions\fg, \textit{BSOAS} 70.3, p. 487-509.}

\author{\textsc{Appadorai}, A.}
\entry{*1990} {[1936] \textit{Economic Conditions in Southern India, 1000-1500 A.D.}, 2 vols. Madras: University of Madras.}

\author{\textsc{Appadurai}, Arjun}
\entry{1978} {\og Kings, Sects and Temples in South India, 1350-1700 A.D.\fg, dans B. Stein (\'ed.), \textit{South Indian Temples}, New Delhi: Vikas Publishing House, p. 47-73 [r\'eimp. de \textit{IESHR} 14.1, p. 47-73].}
%\entry{1981} {\textit{Worship and Conflict Under Colonial Rule: A South Indian Case}, Cambridge: Cambridge University Press.}
%\entry{1983} {\og The Puzzling Status of Brahman Temple Priests in Hindu India\fg, \textit{South Asian Anthropologist}, p. 43-52.}

\author{\textsc{Aravamuthan}, T. G.}
\entry{1934-35} {\og The Authors of the Holy Canon of Tamil Saivism\fg, \textit{QJMS} 25, p. 143-160.}

\author{\textsc{Arunachalam}, M.}
\entry{1985} {\textit{The Saiva Saints}, Mayuram : Gandhi Vidyalayam.}

\author{\textsc{Assayag}, J.}
\entry{2001} {\textit{L'Inde, désir de nation}, Paris: Odile Jacob.}

\author{\textsc{Bakker}, Hans}
%\entry{@@@}, \og At the Right Side of the Teacher\fg, @@@, @@@.
\entry{2004} {\'ed., \textit{Origin and Growth of the Pur\=a\d nic Text Corpus}, Delhi: Motilal Banarsidass Publishers.}
%\entry{2004a} {\og The Structure of the V\=ar\=a\d nas\=\i m\=ah\=atmya in Skandapur\=a\d na 26-31\fg, dans Hans \textsc{Bakker} \'ed., \textit{Origin and Growth of the Pur\=a\d nic Text Corpus}, Delhi: Motilal Banarsidass Publishers, p. 1-16.}

\author{\textsc{Balambal}, V.}
\entry{1998} {\textit{Studies in Ch\=o\b la History}, Delhi: Kalinga Publications.}

\author{\textsc{Balasubrahmanyam}, S. R.}
\entry{1979} {\textit{Later Chola temples: Kulottunga I to Rajendra III (A.D. 1070-1280)}, avec la collaboration de B. \textsc{Natarajan}, B. \textsc{Venkataraman} et B. \textsc{Ramachandran}, Madras: Mudgala Trust.}

%\author{\textsc{Balbir}, Nalini}
%\entry{2001} {\og Hospitalit\'e obligatoire et hospitalit\'e pervetie dans les r\'ecits de l'Inde: asc\`etes, brahmane\index{gnl}{brahmane}s et autres figures d'invit\'es\fg, A. Montandon (\'etudes r\'eunies par), dans \textit{L'hospitalit\'e dans les contes}, @@@: Presses Universitaires Blaise Pascal, p. 373-394.}

\author{\textsc{Banerjea}, Jitendra Nath}
\entry{*2002} {[1956] \textit{The Development of Hindu Iconography}, New Delhi: Munshiram Manoharlal Publishers.}

\author{\textsc{Barnoud-Sethupathy}, Elisabeth}
\entry{1994} {\textit{Le chant du T\=ev\=aram dans les temples du Pays Tamoul\index{gnl}{Pays Tamoul}. Au confluent de la bhakti \'sivaïte et de la musique tamoule}, thèse de doctorat sous la direction de F. \textsc{Gros}, Université Sornonne Nouvelle - Paris 3.}

%\author{\textsc{B\"artels} A.}
%\entry{1994} {Guide des plantes tropicales, Ulmer E., Paris.}

%\author{\textsc{Beck} B.E.F.}
%\entry{1981} {\og The Goddess and the Demon, a local\index{gnl}{local} South Indian Festival and its wider context\fg, in BIARDEAU M. (éd.) Autour de la déesse hindoue, Puru\d s\=artha 5, EHESS, Paris, p. 83-136.}

\author{\textsc{Bhatt}, N. R.}
\entry{2000} {\textit{La religion de \'Siva d'après les sources sanskrites}, Paris: Agamat.}

%\author{\textsc{Bhattacharya} G.}
%\entry{1977} {\og Nandin and V??abha\fg, ZDMG, Franz Steiner, Stuttgart, p. 1545-67.}

%\author{\textsc{Biardeau}, Madeleine}
%\entry{1967-68} {\og brahmane\index{gnl}{brahmane}s combattants dans un mythe\index{gnl}{mythe} du Sud de l'Inde\fg, \textit{Adyar Library Bulletin, Dr.V. Raghavan Felicitation Volume} 31-32, p. 519-530.}

%\author{\textsc{Biardeau} M.}
%\entry{1990} {\og Brahmans and Meat-Eating Gods\fg, dans \textit{Criminal Gods and Demon Devotees}, A. Hiltebeitel \'ed., Delhi: Manohar, p. 19-33.}

%\author{\textsc{Bisschop}, Peter}
%\entry{@@@} {\og The P\=a\'supata Observance (\textit{Atharvavedapari\'si\d s\d ta} 40)\fg}
%\entry{2004} {\og \'Siva's \=Ayatanas in the Various Recensions of Skandapur\=a\d na 167\fg, dans \textit{Origin and Growth of the Pur\=a\d nic Text Corpus}, Hans \textsc{Bakker} \'ed., Delhi: Motilal Banarsidass Publishers, p. 65-78.}

%\author{\textsc{Boisselier}, Jean}
%\entry{1959} {\og La statuaire pr\'eangkorienne et Pierre Dupont\fg, \textit{AA} 6-1, p. 59-69.}

\author{\textsc{Bolle}, Kees W.}
\entry{1969} {\og Speaking of a place\fg, dans \textit{Myths and Symbols, studies in honor of Mircea Eliade}, J.M. \textsc{Kitagawa} et C.H. \textsc{Long} \'ed., Chicago: The University of Chicago Press, p. 127-139.}

%\author{\textsc{Brunner}, H\'el\`ene}
%\entry{1964} {\og Les cat\'egories sociales v\'ediques dans le \'Siva\"isme du Sud\fg, \textit{JA} 252, p. 451-472.}

%\author{\textsc{Brunner} H.}
%\entry{1967} {Analyse du Suprabhedagama, Journal Asiatique 255, p. 31-60.}

\author{\textsc{Buitenen} J. A. B.}
\entry{1971} {\textit{The Mah\=abh\=arata. Book 1, The book of the beginning}, vol. 1, traduction et éd., Chicago : University of Chicago Press.}
\entry{1975} {\textit{The Mah\=abh\=arata. Book 2, The book of the Assembly Hall. Book 3, The Book of the forest}, vol. 2, traduction et éd., Chicago : University of Chicago Press.}


\author{\textsc{Ce\.nkalvar\=aya Pi\d l\d lai}, Va. Cu.}
\entry{1950} {\textit{t\=ev\=ara o\d line\b ri}, vol. 2, Madras: The South India Saiva Siddhanta Works Publishing Society.}

%\author{\textsc{Census of India 1961}}
%\entry{1971} {vol. IX, Tamil Nadu, K.Chockalingam.}


\author{\textsc{Champakalakshmi}, R.}
%\entry{1972} {\og Vaisnava Concepts in Early Tamil Nadu\fg, \textit{Journal of Indian History}, vol. L, part III, p. 723-754.}
\entry{1978} {\og Religious conflict in the Tamil Country: a re-appraisal of epigraphic evidence\fg, \textit{Journal of the Epigraphical Society of India} 5, p. 69-81.}
\entry{1981} {\og Peasant State and Society in Medieval South India\fg, \textit{IESHR} 18 n. 3-4, p. 411-27.}
\entry{*2004} {[2001] \og Reappraisal of a Brahmanical Institution: The Brahmad\=eya and its Ramifications in Early Medieval South India\fg\ dans \textit{Structure and society in early South India : essays in honour of Noboru Karashima}, Kenneth R. \textsc{Hall} éd.; New Delhi: OUP.}
\entry{2006} {\og Bhakti and Tamil Tradition\fg dans \textit{Tattvabodha. Essays from the Lecture Series of the National Mission for Manuscripts}, S. \textsc{Gopalakrishnan} éd., New Delhi: National Mission for Manuscripts et Munshiram Manoharlal Publishers.}

\author{\textsc{Chevillard}, Jean-Luc}
%\entry{1996} {Le commentaire de Ce?avaraiyar sur le Collatikaram du Tolkappiyam, vol. I, Publications du Département d'Indologie 84.1, Pondichéry.}
\entry{2000} {\og Le \textit{T\=ev\=aram} au XX\up{e} si\`ecle\fg, \textit{BEFEO} 87-2, p. 729-740.}
%\entry{2003} {\og R\'eflexions contrastives sur l'ordre des mots en tamoul et en fran\c cais\fg, Communication pour la 2\up{\`eme} journ\'ee de traduction tamoul-fran\c cais-tamoul, Universit\'e Paris 8, Vincennes Saint-Denis.}
\entry{2007} {\textsc{voir} \textit{Digital T\=ev\=aram}}


\author{\textsc{Clothey}, F.}
\entry{1969} {\og Skanda-\d sa\d s\d ti : a festival in Tamil India\fg, \textit{HR} 8, p. 236-259.}

%\author{\textsc{Coomaraswamy} A.}
%\entry{1983} {\textit{La danse de Siva, quatorze essais sur l'Inde}, Paris: l'Harmattan.}


\author{\textsc{Colas-Chauhan}, Usha}
\entry{2002} {\og Um\=apati on Pr\=am\=a\d nya. An annoted translation\fg, \textit{Journal of Indian Philosophy} 30, p. 305-338.}

\author{\textsc{Cort}, John E.}
%\entry{1995} {\og Genres of Jain History\fg, \textit{@@@} 23, p. 469-506.}
\entry{*1999} {[1998] \'ed., \textit{Open Boundaries, Jain Communities and Cultures in Indian History}, (premi\`ere publication: State of University of New York Press, 1998), Delhi: Sri Satguru Publications.}

\author{\textsc{Couture}, André}
\entry{1991} {\textit{L'enfance de Krishna}, Paris: Les éditions du Cerf/Les Presses de l'Université Laval.}

\author{\textsc{Cox}, Whitney}
\entry{2005} {\og The Transfiguration of Ti\d n\d na\b n the Archer\fg, \textit{Indo-Iranian Journal} 48, p. 223-252.}
\entry{2006a} {\textit{Making a Tantra in Medieval South India: The Mah\=arthamañjar\=\i\ and the Textual Culture of C\=o\b la Cidambaram}, thèse de doctorat soutenue à l'Université de Chicago.}
\entry{2006b} {\og From \=Ava\d nam to Pur\=a\d nam\fg, dans \textit{Dimensions of South Asian Religion}, T. H. \textsc{Barrett} (éd.), SOAS Working Papers in the Study of Religions. The School of Oriental and African Studies, p. 5-34.}

\author{\textsc{Cuppirama\d niyam}, Pu.}
\entry{1983} {\textit{Meykkirttika\d l}, Ce\b n\b nai: International Institute of Tamil Studies.}

\author{\textsc{Cutler}, Norman}
\entry{1979} {\og The Nature of Tamil Devotion\fg, dans \textit{Aryan and Non-Aryan in India}, M.M. \textsc{Deshpande} and P.E. \textsc{Hook} \'ed., Ann Arbor: The University of Michigan, p. 11-33.}
\entry{1983} {\og Tamil Religion: Melting Pot or Battleground?\fg, Review article, \textit{HR} 22-4, p. 381-391.}
\entry{1984} {\og The Devotee's Experience of the Sacred Tamil Hymns\fg, \textit{HR} 24-2, p. 91-112.}
\entry{1987} {\textit{Songs of Experience, The Poetics of Tamil Devotion}, Bloomington: Indiana University Press.}
\entry{2004} {\og Three Moments in the Genealogy of Tamil Literary Culture\fg, dans \textit{Literary Cultures in History, Reconstructions from South Asia}, S. \textsc{Pollock} \'ed., New Delhi: Oxford University Press, (premi\`ere publication: University of California, 2003), p. 271-322.}

\author{\textsc{Dagens}, Bruno}
\entry{1979} {\textit{Le florilège de la doctrine \'sivaïte. \'Saiv\=agamaparibh\=a\d s\=ama\~njar\=\i}, édition critique, traduction et notes, PIFI 60, Pondichéry: IFI.}

\author{\textsc{Daniélou}, Alain}
\entry{\textsc{voir}} {\textit{Cilappatik\=aram}}

%\author{\textsc{Das} R. K.}
%\entry{1964} {Temples of Tamilnad, Bombay.}

\author{\textsc{Davis}, Richard H.}
\entry{*1999} {[1998] \og The Story of the Disappearing Jains: Retelling the \'Saiva-Jain Encounter in Medieval South India\fg, dans \textit{Open Boundaries, Jain Communities and Cultures in Indian History}, J.E. \textsc{Cort} \'ed., Delhi: Sri Satguru Publications, (première publication: State University of New York, 1998), p. 213-224.}
\entry{*2000} {[1991] \textit{Worshiping \'Siva in Medieval India. Ritual in an Oscillating Universe}, Delhi: Motilal Banarsidass Publishers.}

\author{\textsc{Dehejia}, Vidya}
\entry{1987} {\og Sambandar: a Child-Saint of South India\fg, \textit{South Asian Studies} 3, p. 53-61.}
\entry{*2002} {[1988] \textit{Slaves of the Lord, The path of the Tamil Saints}, New Delhi: Munshiram Manoharlal Publishers.}
\entry{2002} {\textit{The Sensuous and the Sacred}, New York, Ahmedabad: American Federation of Arts, Mapin Publishing.}

\author{\textsc{Desayar}, M.}
\entry{2004} {\og Temples as Courts of Justice in Medieval Tamil Nadu\fg, dans \textit{Sr\=\i\ Pu\d sp\=añ-jali, Recent Researches in Prehistory, Protohistory, Art, Architecture, Numismatics, Iconography and Epigraphy (Dr. C.R. Srinivasan Commemoration Volume)}, K.V. \textsc{Ramesh} \'ed., Delhi: Bharatiya Kala Prakashan, p. 351-358.}

\author{\textsc{Dessigane}, R. \&\ \textsc{Pattabiramin}, P. Z.}
\entry{1967} {\textit{La légende\index{gnl}{legende@légende} de Skanda: selon le Kandapura\d nam tamoul et l'iconographie}, \textit{PIFI} 31, Pondichéry: IFI.}

\author{\textsc{Dessigane}, R., \textsc{Pattabiramin}, P. Z. \&\ \textsc{Filliozat}, Jean}
\entry{1960} {\textit{La légende\index{gnl}{legende@légende} des jeux de \c Civa \`a Maturai\index{gnl}{Maturai} d'après les textes et les peintures}, \textit{PIFI} 19, Pondich\'ery: IFI.}
\entry{1964} {\textit{Les légende\index{gnl}{legende@légende}s \c civaïtes de K\=a\~ncipuram, analyse de textes et iconographie}, \textit{PIFI} 27, Pondich\'ery: IFI.}

\author{\textsc{Devakunjari}, D.}
\entry{1979} {\textit{Madurai through the ages : from the earliest times to 1801 A.D.}, Madras : Society for Archaeological, Historical, and Epigraphical Research.}


%\author{\textsc{Dhavamony}, M.}
%\entry{1971} {\textit{Love of God, According to Saiva Siddhanta}, Oxford University Press.}

\author{\textsc{Dubyanskiy}, A.}
\entry{2005} {\og Messenger-poems in Tamil poetry\fg, \textit{Cracow Indological Studies} 7, p. 259-274.}

\author{\textsc{Ebeling}, Sacha}
\entry{2010} {\textit{Colonizing the Realm of Words. The Transformation of Tamil Literature in Nineteenth-Century South}, Albany: State University of New York Press.}

\author{\textsc{Eck}, Diana L.}
\entry{1981} {\og India's \textit{T\=\i rthas}: \og Crossings\fg\ in Sacred Geography\fg, \textit{HR} 20-4, p. 323-344.}

\author{\textsc{Filliozat}, Jean}
\entry{1961} {\og Les image\index{gnl}{image}s de Siva dans l'Inde du Sud: \textsc{i}. L'image\index{gnl}{image} de l'origine du Li\.nga\fg, \textit{Arts Asiatiques} 8, p. 43-56.}
\entry{1968} {\og Archaeology and Tamil Studies\fg, dans \textit{Proceedings of the Second International Conference Seminar of Tamil Studies} 1, R.E. \textsc{Asher} \'ed., p. 3-11.}
\entry{1972} {\textsc{voir} \textit{Tirupp\=avai}}
\entry{1973} {\textsc{voir} \textit{Tirumuruk\=a\b r\b rupa\d tai}}

\author{\textsc{Filliozat}, Pierre-Sylvin}
\entry{1994} {\textit{Dictionnaire des littératures de l'Inde}, Paris: PUF.}

\author{\textsc{Filliozat}, Vasundhara}
\entry{1973} {\textit{L'épigraphie de Vijayanagar du début à 1377}, \textit{PEFEO} 91, Paris: EFEO.}

\author{\textsc{Francis}, Emmanuel}
\entry{2009} {\textit{Le discours royal. Monuments et inscriptions pallava (IVème-IXème
siècles)}, thèse de doctorat en langues et lettres soutenue le 10 juin
2009, Louvain-la-Neuve : Institut Orientaliste, Université catholique
de Louvain.}

\author{\textsc{Francis}, Emmanuel \&\ \textsc{Schmid}, Charlotte}
\entry{2010} {\textsc{voir} \textit{Putucc\=eri m\=anilakkalve\d t\d tukka\d l, Pondicherry Inscriptions, Part II}}

\author{\textsc{Fuller}, Christopher J.}
\entry{1984} {Servants of the Goddess, Cambridge: Cambridge University Press.}
\entry{1985} {\og The king, the law and the priests in a South Indian Temple\fg, dans \textit{L'espace du temple, espaces, itin\'eraires, m\'ediations}, J.-C. \textsc{Galey} (\'etudes r\'eunies par), \textit{Puru\d s\=artha} 8, p. 149-175.}

\author{\textsc{Fuller}, Christopher J. \&\ \textsc{Logan}, P.}
\entry{1985} {\og The Navar\=atri Festival in Madurai\fg, \textit{BSOAS} 48, p. 79-105.}

\author{\textsc{Gillet}, Val\'erie}
\entry{2007} {\og Entre démon et dévot: la figure de R\=ava\d na dans les représentations \textit{pallava}\fg, \textit{AA} 62, p. 29-45.}
\entry{2010} {\textit{La création d'une iconographie \'sivaïte narrative. Incarnations du dieu\index{gnl}{dieu} dans les temples pallava construits}, \textit{Collection Indologie} 113, Pondichéry : IFP/ EFEO.}

%\author{\textsc{Giteau}, Madeleine}
%\entry{1967-68} {\og Two Tenth Century Bas-reliefs Depicting the \textit{R\=ava\d n\=anugraham\=urti}\fg, \textit{Adyar Library Bulletin, Dr.V. Raghavan Felicitation Volume} 31-32, p. 593-599.}
\author{\textsc{Goodall}, Dominic}
\entry{2004} {\textit{The Par\=akhyatantra. A Scripture of the \'Saiva Siddh\=anta}, édition critique et traduction annotée, \textit{Collection Indologie} 98, Pondichéry: IFP/EFEO.}

\author{\textsc{Gopal Iyer}, T. V.}
\entry{1984-85} {\textsc{voir} \textit{T\=ev\=aram}}
\entry{1991} {\textit{T\=ev\=aram, \'etudes et glossaire tamouls}, vol. \textsc{iii}, \textit{PDI} 68.3, Pondich\'ery: IFP.}

%\author{\textsc{Goudriaan} T. \&\ \textsc{Gupta} S.}
%\entry{1981} {Hindu Tantric and Sakta Literature, in A History of Indian Literature, Otto Harrassowitz, Wiesbaden, vol. II/2.}

\author{\textsc{Granoff}, Ph.}
%\entry{1986} {\og The miracle\index{gnl}{miracle} of a hagiography without miracle\index{gnl}{miracle}s: some comments on the jain lives of the Pratyekabuddha Karaka\d n\d da\fg, \textit{Journal of Indian Philosophy} 14, p. 389-403.}
\entry{1997} {\og Heaven on Earth: Temples and Temple Cities of Medieval India\fg, dans \textit{India and Beyond: Aspects of Literature, Meaning, Ritual and Thought, Essays in Honour of Frits Staal}, Dick van der \textsc{Meij} (ed.), New York: Kegan Paul International, p. 170-93.}
\entry{2004} {\og Saving the Saviour: \'Siva and the Vai\d s\d nava Avat\=aras in the Early Skandapur\=a\d na\fg, dans \textit{Origin and Growth of the Pur\=a\d nic Text Corpus}, Hans \textsc{Bakker} \'ed., Delhi: Motilal Banarsidass Publishers, p. 111-138.}

\author{\textsc{Gros}, Fran\c cois}
\entry{1968} {\textsc{voir} \textit{Parip\=a\d tal}}
\entry{1972} {\textsc{voir K\=araikk\=alammaiy\=ar}}
%\entry{1982} {\og tradition\index{gnl}{tradition} tamoule et mythologie hindoue (notes critiques)\fg, Revue d'Histoire des Religions CIC, p. 74-5.}
\entry{1983} {\og La litt\'erature du Sangam et son public\fg, dans \textit{Inde et litt\'erature}, M.-C. \textsc{Porcher} (\'etudes r\'eunies par), \textit{Puru\d s\=artha} 7, p. 77-107.}
\entry{1984} {\textsc{voir} \textit{T\=ev\=aram}}
\entry{2001} {\og In\'epuisable \textit{Periya Pur\=a\d nam}: Sur deux listes et soixante-douze\index{gnl}{douze} mani\`eres de servir\fg, dans \textit{Constructions hagiographique\index{gnl}{hagiographie!hagiographique}s dans le monde indien. Entre mythe\index{gnl}{mythe} et histoire}, sous la responsabilit\'e de Fran\c coise \textsc{Mallison}, Paris: Editions Champion, p. 19-60.}

\author{\textsc{Gros}, Fran\c cois \&\ \textsc{Nagaswamy}, R.}
\entry{1970} {\textit{Uttaram\=er\=ur, légende\index{gnl}{legende@légende}s, Histoire, Monuments}. Avec le \textit{Pa\~ncavaradak\d setra m\=ah\=atmya} édité par K. \textsc{Srinivasacharya}, \textit{PIFI} 39, Pondichéry: IFI}

\author{\textsc{Guilmoto}, Christophe, \textsc{Reiniche}, Marie-Louise \&\ \textsc{Pichard}, Pierre}
\entry{1990} {\textit{Tiruvannamalai: un lieu saint \'siva\"ite du sud de l'Inde}, vol. 5, \textit{La ville}, \textit{PEFEO} 156.5, Paris: EFEO.}

\author{\textsc{Hall}, K.}
\entry{1981} {\og Peasant State and Society in Chola Times: a view from Tiruvidaimarudur\fg, \textit{IESHR} 18 n. 3-4, p. 411-427.}

%\author{\textsc{Hanneder}, J\"urgen.}
%\entry{2002} {\og The Blue Lotus, Oriental Research between Philology, Botany and Poetics?\fg, \textit{Zeitschrift der Deutschen Morgenl\"andischen Gesellschaft} 152-2, p. 295-308.}

%\author{\textsc{Hara\index{gnl}{Hara}}, Minoru}
%\entry{1992} {\og P\=a\'supata Studies (1)\fg, dans T. Goudriaan (\'ed.), \textit{Ritual and Speculation in Early Tantrism, Studies in Honor of Andr\'e Padoux}, Albany: State University of New York Press, p. 209-226.}
%\entry{1994} {\og P\=a\'supata Studies II\fg, \textit{WZKS} 38, p. 323-335.}

\author{\textsc{Hardy}, Friedhelm W.}
\entry{*2001} {[1983] \textit{Viraha-Bhakti}, New Delhi: OUP.}
\entry{1992} {\og The \'Sr\=\i vai\d s\d nava Hagiography of Parak\=ala\fg, dans \textit{The Indian Narrative, Perspertives and Patterns}, C. \textsc{Shackle} et R. \textsc{Snell} (éd.), Wiesbaden: Otto Harrassowitz, p. 81-116.}

\author{\textsc{Harimoto}, Kengo}
\entry{2006} {\og The Date of \'Sa\.nkara: Between the C\=a\d lukyas and the R\=a\d s\d trak\=u\d tas\fg, \textit{Journal of Indological Studies} 18 (anciennement \textit{Studies in the History of Indian Thought}), p. 85-111.}

\author{\textsc{Harman}, W.}
\entry{1987} {\og Two Versions of a Tamil Text and the Contexts in Which They Were Written\fg, \textit{Journal of the Institute of Asian Studies} 5. 1, p. 1-18.}

\author{\textsc{Hart, III}, George L.}
\entry{1979} {\og The Nature of Tamil Devotion\fg, dans \textit{Aryan and Non-aryan in India}, M. \textsc{Deshpande} et P.E. \textsc{Hook} (\'ed.), Ann Arbor: The University of Michigan, p. 11-33.}
\entry{1980} {\og The Little Devotee: C\=ekkil\=ar's story of Cirutto\d n\d tar\fg, dans \textit{Sanskrit and Indian Studies, Essays in Honour of Daniel H.H. Ingalls}, M. \textsc{Nagatomi} et al. (\'ed.), Dordrecht: R. Reidel Publishing Company, p. 217-236.}

\author{\textsc{Hart, III}, George L. \&\ \textsc{Heifetz}, Hank}
\entry{*2002} {\textsc{voir} \textit{Pu\b ran\=a\b n\=u\b ru}}

\author{\textsc{Hawley}, John Stratton}
\entry{1988} {\og Author and Authority in the Bhakti Poetry of North India\fg, \textit{JAS} 47-2, p. 269-290.}

\author{\textsc{Heitzman}, James}
\entry{*2001} {[1997] \textit{Gifts of power. Lordship in an Early Indian State}. New Delhi: OUP.}

\author{\textsc{Heitzman}, James \&\ \textsc{Rajagopal} S.}
\entry{1985} {\og Temple Landholding and Village Geography in the C\=o\b la Period: Reconstructions Through Inscriptions\fg, \textit{Tamil Civilization} 3-2 \&\ 3, p. 6-31.}
\entry{1987} {\og Temple Urbanism in Medieval South India\fg, \textit{JAS} 46-4, p. 791-826.}
\entry{1987} {\og State formation in South India, 850-1280\fg, \textit{IESHR} 24. 1, p. 35-61.}

\author{\textsc{Hiltebeitel}, Alf}
%\entry{1978} {\og The Indus Valley "Proto-\'Siva", Reexamined through Reflections on the Goddess, the Buffalo, and the Symbolism of \textit{v\=ahanas}\fg, \textit{Anthropos} 73-5 \&\ 6, p. 767-797.}
\entry{*1990} {[1989] \textit{Criminal Gods and Demon Devotees},(éd.), New Delhi: Manohar (première publication: Albany: State University of New York Press, 1989).}

%\author{\textsc{Heras} H.}
%\entry{1927} {The Aravidu Dynasty of Vijayanagara, B.G. Paul \&\ Co. Publishers, Madras.}

%\author{\textsc{Hira Lal}, R.B.}
%\entry{1927} {\og The Golaki Matha\fg, \textit{JBORS} 13, p. 137-144.}

\author{\textsc{Hoekveld-Meier}, G.}
\entry{1981} {\textit{Koyils in the Colama\d n\d dalam, typology and development of early Cola temples}, Amsterdam: Krips Repro.}

\author{\textsc{Hudson}, D. D.}
\entry{*1990} {[1989] \og Violent and Fanatical Devotion Among the Naya?ars: A Study in the Periya Pur\=a\d nam of C\=ekki\b l\=ar\index{gnl}{Cekkilar@C\=ekki\b l\=ar}\fg, dans \textit{Criminal Gods and Demon Devotees}, \textsc{Hiltebeitel} A. (éd.), New Delhi: Manohar, p. 373-404.}

\author{\textsc{Ir\=acam\=a\d nikka\b n\=ar}, M\=a.}
\entry{\textsc{voir}} {\textsc{Rajamanickam}, M.}
\entry{*1996} {[1968] \textit{C\=ekki\b l\=ar\index{gnl}{Cekkilar@C\=ekki\b l\=ar}}, Ce\b n\b nai: Ma\b ru patippu.}

%\author{\textsc{Jamison} S. W.}
%\entry{1996} {Sacrificed Wife / Sacrificer's Wife : Women, Ritual, and Hospitality in Ancient India, Oxford University Press.}
\author{\textsc{Jha}, D. N.}
\entry{*1977} {[1974] \og Temples as Landed Magnates in Early Medieval South India\fg, dans \textit{Indian Society: Historical Probings (in memory of D.D. Kosambi)}, \textsc{Sharma} R.S. (éd.), New Delhi: People's Publishing House, p. 202-216.}

\author{\textsc{Kaimal}, Padma}
\entry{2003} {\og A Man's World? Gender, Family, and Architectural Patronage in Medieval India\fg, \textit{Archives of Asian Arts} 2002-2003, p. 26-53.}
\entry{1996} {\og Early C\=o\b la Kings and Early C\=o\b la Temples: Art and the evolution of kingships\fg, \textit{Artibus Asiae}, vol. LVI 1-2, p. 33-66.}

\author{\textsc{Kandaswamy Pillai}, N.}
\entry{1967-1970} {Index des mots de la littérature tamoule ancienne, sous la direction de, 3 vols., \textit{PIFI} 37.1-2-3, Pondichéry: IFI.}

%\author{\textsc{Kanakasabhai Pillai}, V.}
%\entry{1889} {\og Tamil Historical Texts\fg, \textit{The Indian Antiquary} 18, Bombay, p. 258-265.}

%\author{\textsc{Kandiah} A.}
%\entry{1974} {A Critical Study of Early Tamil Saiva Bhakti Literature, with special reference to Tevaram, Ph.D. dissertation, University of London.}

\author{\textsc{Karashima}, Noboru}
\entry{} {Nayaka's rule in the region of North and South Arcot Districts in South India during the sixteenth century, unpublished, 50p.}
\entry{1996} {\og South Indian Temple Inscriptions: a new approach to their study\fg, \textit{South Asia} 19-1, p. 1-12.}
\entry{*2001a} {[1966] \og All\=ur and \=I\'s\=anamangalam: Two South Indian Villages of Chola Times\fg, IESHR III-2, p. 150-162. Réimpression: \textit{History and Society in South India. The Cholas to Vijayanagar}, New Delhi: OUP.}
\entry{*2001c} {[1972] \og Revenue Terms in Chola Inscriptions\fg\ (co-authored by B. Sitaraman), Journal of Asian and African studies 5, ILCAA, Tokyo. Réimpression: \textit{History and Society in South India. The Cholas to Vijayanagar}, New Delhi: OUP.}
\entry{2002} {\textit{A Concordance of N\=ayakas. The Vijayanagar Inscriptions in South India}. New Delhi: OUP.}
\entry{*2004} {[2001] \og Whispering of Inscriptions\fg\ dans \textit{Structure and society in early South India: essays in honour of Noboru Karashima}, éd. par Kenneth R. \textsc{Hall}, New Delhi: OUP, p. 44-58.}

\author{\textsc{Karashima}, N., \textsc{Subbarayalu}, Y. \&\ \textsc{Matsui}, T.}
\entry{1978} {\textit{A Concordance of the Names in the C\=o\b la Inscriptions}, 3 vols., Madurai: Sarvodaya Ilakkiya Pannai.}

\author{\textsc{Karavelane}}
\entry{1982} {\textsc{voir} \textsc{K\=araikk\=alammaiy\=ar}}

\author{\textsc{Kingsbury}, F. \&\ \textsc{Phillips}, G.E.}
\entry{*2000} {[1921] Hymns of the Tamil Saivite Saints, OUP.}

\author{\textsc{Kramrisch}, S.}
\entry{1988} {\textit{The Presence of \'Siva}, Delhi: Motilal Banarsidass.}

\author{\textsc{Krishnaswami}, A.}
\entry{1964} {\textit{The Tamil Country under Vijayanagar}, Annamalai University.}

\author{\textsc{Ladrech}, Karine}
\entry{2002} {\og Bhairava \`a la massue\fg, \textit{BEI} 20-1, p. 163-192.}
\entry{2010} {\textit{Le crâne et le glaive. Représentations de Bhairava en Inde du Sud (VIIIe-XIIIe siècles)}, Collection Indologie 112, Pondichéry : IFP/EFEO.}

\author{\textsc{Lef\`evre}, Vincent}
\entry{2001} {\og L'enfant\index{gnl}{enfant}-mod\`ele dans la sculpture d'Inde du Sud des Pallava\index{gnl}{Pallava} \`a Vijayanagar\fg, dans \textit{Les \^ages de la vie dans le monde indien}, C. \textsc{Chojnacki} (\'ed.), Paris: Diffusion De Boccard, p. 217-231.}
%\entry{2006} {\textit{Commanditaires et artistes en Inde du Sud. Des Pallava\index{gnl}{Pallava} aux N\=ayak (\textsc{vi}\up{e}-\textsc{xviii}\up{e} siècle)}, Paris: Presses Sorbonne Nouvelle.}

\author{\textsc{Lehmann} Thomas \&\ \textsc{Malten} Thomas}
\entry{1993} {\textit{A Word Index for Ca\.nkam Literature}, Madras: Institute of Asian Studies.}

\author{\textsc{L'Hernault}, Fran\c coise}
\entry{1978} {\textit{L'iconographie de Subrahma\d nya au Tamilnad}, \textit{PIFI} 59, Pondichéry: IFI.}
\entry{1987} {\textit{Darasuram. Epigraphical study, étude architecturale, étude iconographique}, vol. 1 : Texte, vol. 2: Planches, avec des collaborations de P.R. \textsc{Srinivasan} et de J. \textsc{Dumar\c cay}, Paris: EFEO.}
%\entry{2002} {\og Le \textit{gopura}, pavillon d'entr\'ee des temples sud-indiens\fg, \textit{AION} 62, p. 115-123.}
%\entry{1998} {\og La personnification du Meurtre de brahmane\index{gnl}{brahmane}\fg, \textit{AION} 58, p. 365-372.}

\author{\textsc{L'Hernault}, Fran\c coise, \textsc{Pichard}, Pierre \&\ \textsc{Deloche}, Jean}
\entry{1990} {\textit{Tiruvannamalai: un lieu saint \'siva\"ite du sud de l'Inde}. Vol. 2, \textit{Archéologie du site}, \textit{PEFEO} 156.2, Paris: EFEO.}

\author{\textsc{L'Hernault}, Fran\c coise \&\ \textsc{Reiniche}, Marie-Louise}
\entry{1999} {\textit{Tiruvannamalai: un lieu saint \'siva\"ite du sud de l'Inde}. Vol. 3, \textit{Rites et fêtes}, \textit{PEFEO} 156.3, Paris: EFEO.}

\author{\textsc{McGlashan}, Alastair Robin}
\entry{2006} {\textsc{voir} \textit{Periyapur\=a\d nam}\index{gnl}{Periyapuranam@\textit{Periyapur\=a\d nam}}}

%\author{\textsc{Mahabharata (le)}}
%\entry{1985-1986} {2 vols., Paris, Garnier Flammarion.}

\author{\textsc{Mahalingam}, T. V.}
%\entry{1967} {South Indian Polity, rev. ed., University of Madras.}

%\entry{1985} {A Topographical List of Inscriptions in the Tamil Nadu and Kerala States, Vol. 1 North Arcot District; New Delhi: Indian Council of Historical Research.}
\entry{1988} {\textit{A Topographical List of Inscriptions in the Tamil Nadu and Kerala States}, Vol. 2 South Arcot District; New Delhi: Indian Council of Historical Research.}
\entry{1989} {\textit{A Topographical List of Inscriptions in the Tamil Nadu and Kerala States}, Vol. 3 Chingleput District; New Delhi: Indian Council of Historical Research.}
%\entry{1989} {A Topographical List of Inscriptions in the Tamil Nadu and Kerala States, Vol. 4 Coimbatore and Dharmapuri Districts; New Delhi: Indian Council of Historical Research.}
%\entry{1989} {A Topographical List of Inscriptions in the Tamil Nadu and Kerala States, Vol. 5 Kanyakumari, Madras and Madurai Districts; New Delhi: Indian Council of Historical Research.}
\entry{1991a} {\textit{A Topographical List of Inscriptions in the Tamil Nadu and Kerala States}, Vol. 6 Nilgiris District, Pudukkottai District, Ramanathapuram District, Salem District; New Delhi: Indian Council of Historical Research.}
\entry{1991b} {\textit{A Topographical List of Inscriptions in the Tamil Nadu and Kerala States}, Vol. 8 Tiruchchirappalli District; New Delhi: Indian Council of Historical Research.}
\entry{1992} {\textit{A Topographical List of Inscriptions in the Tamil Nadu and Kerala States}, Vol. 7 Thanjavur District; New Delhi: Indian Council of Historical Research.}

%\entry{1995} {A Topographical List of Inscriptions in the Tamil Nadu and Kerala States, Vol. 9 Tirunelveli District; New Delhi: Indian Council of Historical Research.}

\author{\textsc{Mallison}, F.}
\entry{2001} {\textit{Constructions hagiographique\index{gnl}{hagiographie!hagiographique}s dans le monde indien. Entre mythe\index{gnl}{mythe} et histoire}, (sous la responsabilit\'e de), Paris: Editions Champion.}

\author{\textsc{Marr}, John R.}
\entry{1979} {\og The \textit{Periya Pur\=a\d nam} Frieze at T\=ar\=acuram: Episodes in the Lives of the Tamil \'Saiva Saints\fg, \textit{BSOAS} 42, p. 268-289.}
\entry{1992} {The Folly of Righteousness: Episodes from the \textit{Periya Pur\=a\d nam}, dans \textit{The Indian Narrative. Perspertives and Patterns}, C. \textsc{Shackle} et R. \textsc{Snell} (éd.), Wiesbaden: Otto Harrassowitz, p. 117-135.}

%\author{\textsc{Masilamani-Meyer} E.}
%\entry{1990} {\og the Changing Face of Kattavarayan\fg, in Hiltebeitel A. (ed.), Criminal Gods and Demon Devotees, Manohar, p. 69-103.}

\author{\textsc{Meister}, M. W. \&\ \textsc{Dhaky}, M.A.}
\entry{1983} {\textit{Encyclopaedia of Indian Temple Architecture, Lower Dravi\d dadesa}, (éd.), vol. I. part 1, OUP.}

\author{\textsc{Minakshi}, C.}
\entry{1938} {\textit{Administration and social life under the Pallavas}, Madras: University of Madras.}

\author{\textsc{Monius}, Anne E.}
\entry{2004} {\og Love, Violence, and the Aesthetics of Disgust: \'Saivas and Jains in Medieval South India\fg, \textit{Journal of Indian Philosophy} 32, p. 113-172.}

\author{\textsc{Nagaswamy}, R.}
%\entry{1965} {\og South Indian Temple---as an Employer\fg, \textit{IESHR} @@@, p. 367-372.}
\entry{1968} {\og The Origin and Evolution of the Tamil, Vatteluttu and Grantha Scripts\fg, dans R.E. Asher (\'ed.), Proceedings of the Second International Conference Seminar of Tamil Studies 2, p. 410-415.}
\entry{1989} {\textit{\'Siva bhakti}, New Delhi : Navrang.}
\entry{2005} {\textsc{voir} \textit{u\.nka\d l \=ur kalve\d t\d tut tu\d naiva\b n, Pathway to the Antiquity of your village}}

\author{\textsc{Nandi}, R. N.}
\entry{*1977} {[1974] \og Origin and Nature of Saivite Monasticism: The Case of Kalamukhas\fg, dans \textit{Indian Society: Historical Probings (in memory of D.D. Kosambi)}, R.S. \textsc{Sharma} (éd.), New Delhi: People's Publishing House, p. 190-201.}

\author{\textsc{Niklas}, U.}
\entry{1988} {\og Introduction to Tamil Prosody\fg, \textit{BEFEO} 77, p. 165-227.}

\author{\textsc{Nilakanta Sastri}, K. A.}
\entry{1932} {\og The Economy of a South Indian Temple in the Cola Period\fg, dans \textit{The Malaviya Commemoration Volume}, Allahabad, p. 305-319.}
\entry{*2000} {[1955] \textit{The Colas}, University of Madras.}
\entry{*1998} {[1975] \textit{A history of South India}, 4\up{e} éd., Delhi: Oxford India Paperbacks.}

\author{\textsc{Nambi Arooran}, K.}
\entry{1977} {\textit{Glimpses of Tamil Culture based on Periyapuranam}, Madurai: Koodal Publishers.}

\author{\textsc{Oddie}, G. A.}
\entry{1981} {\og The character, role and significance of non-brahman saivite mutts in Tanjore district, in the nineteenth century\fg, Paper of the Seventh European Conference on Modern South Aszian Studies, 7-11 july 1981.}

%\author{\textsc{Olivelle}, P.}
%\entry{1998} {\og Hair and Society: Social Significance of Hair in South Asian Traditions\fg, dans A. Hiltebeitel et B.D. Miller (\'eds.), \textit{Hair}, Albany: State University of New York Press, p. 11-49.}

\author{\textsc{Orr}, Leslie C.}
\entry{*2004} {[2001] \og Women in the Temple, the Palace, and the Family: The Construction of Women's Identities in Pre-Colonial Tamiln\=a\d du\fg\ dans \textit{Structure and society in early South India : essays in honour of Noboru Karashima}, edited by Kenneth R. \textsc{Hall}, New Delhi: OUP.}
\entry{2004} {\og Temple Life at Chidambaram in the Chola Period: an Epigraphical Study\fg, dans \textit{\'Sr\=\i\ Pu\d sp\=añjali (Recent Researches in Prehistory, Protohistory, Art, Architecture, Numismatics, Iconography and Epigraphy)}, Dr. C.R. Srinivasan Commemoration Volume, K.V. \textsc{Ramesh} (éd.), Delhi: Bharatiya Kala Prakashan, p. 227-241.}
\entry{2005} {\og Poets, Philosophers and Saints: Canon and Icon in Medieval Tamil Saivism\fg, paper presented at the Madison South Asia conference (Oct 6-9, 2005).}
%\entry{2006} {\og The Sacred Landscape of Tamil Saivism: Constructing Connections and Plotting Place\fg, paper presented at the Annual conference on South Asia, Madison, October 21, 2006.}
\entry{2007} {\og Singing Saintly Songs: Tamil Hymns in the Medieval South Indian Temple\fg, paper presented at the AAS Annual Meeting, 25 March 2007.}
\entry{2008} {\og Tamil temple traditions: transmission, reclamation, and transformation\fg, paper presented at the Fourth Annual Tamil Chair Conference, UC Berkeley, April 2008.}
\entry{2009} {\og The Sacred Landscape of Tamil Saivism: Constructing Connections and Plotting Place\fg, paper presented at the Bhakti workshop, Pondichery, 28 August 2009.}
\entry{2006} {\textsc{voir} \textit{Putucc\=eri m\=anilakkalve\d t\d tukka\d l, Pondicherry Inscriptions, Part I}}
%\author{\textsc{Padoux} A.}
%\entry{1990} {(études rassemblées par), L'image\index{gnl}{image} divine, culte\index{gnl}{culte} et méditation dans l'hindouisme, CNRS, Paris.}

%\author{\textsc{Parpola}, Asko}
%\entry{2002} {\og @@@ and S\=\i t\=a: on the Historical Background of the Sanskrit Epics\fg, \textit{JAOS, Indic and Iranian Studies in Honor of Stanley Insler on his Sixty-fifth Birthday} 122-2, p. 361-373.}

\author{\textsc{Peterson}, Indira}
\entry{1982} {\og Singing of a Place: Pilgrimage\index{gnl}{image} as Metaphor and Motif in the T\=ev\=aram Songs of the Tamil \'Saivite Saints\fg, \textit{JAOS} 102-1, p. 69-89.}
\entry{1983} {\og Lives of the Wandering Singers: Pilgrimage\index{gnl}{image} and Poetry in Tamil \'Saivite Hagiography\fg, \textit{HR} 22-4, p. 338-360.}
\entry{*1991} {[1989] \textit{Poems to Siva, The Hymns of the Tamil Saints}, Delhi: Motilal Banarsidass.}
\entry{1994} {\og Tamil \'Saiva Hagiography, The narrative of the holy servants (of \'Siva) and the hagiographical project in Tamil \'Saivism\fg, dans \textit{According to Tradition, Hagiographical writing in India}, W.M. \textsc{Callewaert} et R. \textsc{Snell} (\'ed.), Wiesbaden: Harrassowitz Verlag, p. 191-228.}
\entry{*1999} {[1998] \og Srama\d nas Against the Tamil Way\fg, dans \textit{Open Boundaries, Jain Communities and Cultures in Indian History}, J.E. \textsc{Cort} \'ed., Delhi: Sri Satguru Publications, p. 163-85.}

\author{\textsc{Pichard}, Pierre}
\entry{1994} {\textit{Vingt ans après Tanjavur, Gangaikondacholapuram}, avec des collaborations de Fran\c coise \textsc{L'Hernault}, Fran\c coise \textsc{Boudignon} et L. \textsc{Thyagarajan} , 2 tomes, \textit{Mémoires archéologiques} 20, Paris: EFEO.}
\entry{1995} {\textit{Tanjavur B\textsubring{r}hadisvara : an architectural study}, New Delhi : Indira Gandhi National Centre for Arts, Pondichéry: EFEO.}

\author{\textsc{Pillai}, S. V.}
\entry{1956} {History of Tamil language and literature, Madras: New Century Book House.}

\author{\textsc{Prentiss}, Karen Pechilis}
\entry{1996} {\og A Tamil Lineage for Saiva Siddhanta Philosophy\fg, \textit{HR} 35.3, p. 231-257.}
\entry{1999} {\textit{The Embodiment of Bhakti}, New York: OUP.}
\entry{2001a} {\og On the making of a canon: Historicity and experience in the Tamil \'Siva-\textit{bhakti} canon\fg, \textit{International Journal of Hindu Studies} 5.1, p. 1-26.}
\entry{2001b} {\og Translation of the \textit{Tirumu\b raika\d n\d tapur\=a\d nam}; attributed to Um\=apati Civ\=a-c\=ariyar\fg, \textit{International Journal of Hindu Studies} 5.1, p. 27-44.}
\entry{2005} {\og The Story of Nandanar: Contesting the Order of Things\fg, dans \textit{Untouchable Saints, an Indian Phenemenon}, E. \textsc{Zelliot} et R. \textsc{Mokashi-Punekar} \'ed., New Delhi: Manohar, p. 95-107.}
\entry{2006} {\og The Story of the Classical Tamil Woman Saint, K\=araikk\=al Ammaiy\=ar: A Translation of Her Story from C\=ekki\b l\=ar\index{gnl}{Cekkilar@C\=ekki\b l\=ar}'s \textit{Periya Pur\=a\d nam}\fg, \textit{International Journal of Hindu Studies} 10, p. 173-186.}

\author{\textsc{Raghavan}, V.}
\entry{1960} {\og Tamil Versions of the Pur\=a\d nas\fg, \textit{Pur\=a\d na} 2/1-2, p. 225-242.}

\author{\textsc{Rajagopal}, S.}
\entry{2001} {\textit{Kaveri. Studies in Epigraphy, Archaeology and History (Professor Y. Subbarayalu Felicitation Volume)}, Chennai: Panpattu Veliyiittakam, .}

%\author{\textsc{Rajam}, V.S.}
%\entry{1986} {\og \textit{A\d na\.nku}: a Notion Semantically Reduced to Signify Female Sacred Power\fg, \textit{JAOS} 106-2, p. 257-272.}

\author{\textsc{Rajamanickam}, M.}
\entry{\textsc{voir}} {\textsc{Ir\=acam\=a\d nikka\b n\=ar} M\=a.}
\entry{1964} {\textit{The development of Saivism in South India (AD. 300-1300)}, Tarumapuram\index{gnl}{Tarumapuram}: Dharmapuram Adeenam.}

\author{\textsc{Ramachandran}, T. N.}
\entry{1990-1995} {\textsc{voir} \textit{Periyapur\=a\d nam}\index{gnl}{Periyapuranam@\textit{Periyapur\=a\d nam}}}
\entry{1993} {\textsc{voir} \textsc{K\=arraikk\=alammaiy\=ar}}
\entry{2001} {\textsc{voir} \textit{Tiruv\=acakam}}

%\author{\textsc{Randhawa} M. S.}
%\entry{1983} {Flowering trees, National Book Trust, New Delhi.}

\author{\textsc{Rangaswamy}, D.}
\entry{*1990} {[1958] \textit{The Religion and Philosophy of Tevaram, with special reference to Nampi Aruvar (Sundarar)}, Madras: University of Madras.}

\author{\textsc{Rao}, T. A. G.}
\entry{*1997} {[1914] \textit{Elements of Hindu Iconography}, 4 vols., Delhi: Motilal Banarsidass.}

\author{\textsc{Rao}, V. N.}
\entry{1990} {\textit{Siva's Warriors, The Basava Pur\=a\d na of P\=alkuriki Soman\=atha}, Princeton University Press.}

\author{\textsc{Reiniche}, Marie-Louise}
%\entry{1979} {\og Les Dieux et les hommes. Etudes des culte\index{gnl}{culte}s d'un village du Tirunelveli, Inde du Sud\fg, coll. "Cahiers de l'Homme", nouvelle série XVIII, Paris/La Haye: Mouton, p. 202-17.}
\entry{1985} {\og Le temple dans sa localit\'e: quatre exemples au Tamilnad\fg, dans \textit{L'espace du temple, espaces, itin\'eraires, m\'ediations}, J.-C. \textsc{Galey} (\'etudes r\'eunies par), \textit{Puru\d s\=artha} 8, p. 75-119.}
\entry{1989} {\textit{Tiruvannamalai: un lieu saint \'siva\"ite du sud de l'Inde}. Vol. 4, \textit{La configuration sociologique du temple hindou}, \textit{PEFEO} 156.4, Paris: EFEO.}

%\entry{1999} {\og Point d'orgue, Fran\c coise L'Hernault (1937-1999)\fg, \textit{BEFEO} 86, p. 21-28.}
\author{\textsc{Renou}, L. \& \textsc{Filliozat}, J.}
\entry{*1985} {[1947] \textit{L'Inde Classique, manuel des études indiennes}, T. \textsc{i}, Paris: Maisonneuve.}
\entry{*1996} {[1953] \textit{L'Inde Classique, manuel des études indiennes}, T. \textsc{ii}, \textit{Réimpressions de l'EFEO}, Paris: EFEO.}

\author{\textsc{Sanderson}, Alexis}
\entry{2003-2004} {\og The \'Saiva Religion among the Khmers (Part I)\fg, \textit{BEFEO} 90-91, p. 349-364.}

%\entry{1988} {\og \'Saivism and the Tantric Traditions\fg, dans S. Sutherland, L. Houlden, P. Clarke et F. Hardy (\'eds.), \textit{The World's Religions}, London: Routledge, p. 660-704.}

\author{\textsc{Salomon}, Richard.}
\entry{1998} {\textit{Indian epigraphy: a guide to the study of inscriptions in Sanskrit, Prakrit, and the other Indo-Aryan languages}, New York ; Delhi : Oxford University Press ; Munshiram manoharlal.}

\author{\textsc{Schmid}, Charlotte}
\entry{2002} {\og Aventures divines de K\textsubring{r}\d s\d na : la l\=\i l\=a et les traditions narratives des temples c\=o\b la\fg, \textit{AA} 57, p. 33-49.}
%\entry{2003-2004} {\og Le sanglier qui danse : étude iconographie d'un avatara de Vi??u\fg, dans Du corps humain, au carrefour de plusieurs savoirs en Inde, Mélanges offerts à Arion Rosu à l'occasion de son 80e anniversaire, Studia Asiatica IV (2003) - V (2004), p. 579-610.}
%\entry{2003-2004} {\og A propos des premières image\index{gnl}{image}s de la Tueuse de buffle : déesses et krishnaïsme ancien\fg, BEFEO 90-91, p. 7-67.}
\entry{2005} {\og Au seuil du monde divin: reflets et passages du dieu\index{gnl}{dieu} d'\=Alantu\b rai à Pu\d l\d lama\.nkai\fg, \textit{BEFEO} 92, p. 39-157.}
\entry{à paraitre (a)} {\og \textit{Bhakti} in its infancy: the Skanda-Muruka\b n of the Kail\=asan\=atha of K\=a\~ncipuram\fg.}
\entry{à paraitre (b)} {\og The edifice of Bhakti, an \og archaeological\fg\ reading of the \textit{T\=ev\=aram} and \textit{Periyapur\=a\d nam}\fg.}

%\author{\textsc{Schmiedchen}, Annette}
%\entry{?} {\og Epigraphical Evidence for the History of Atharvavedic Brahmins\fg, @@@.}

%\author{\textsc{Schwindler} G.}
%\entry{1981} {\og Sculpture in medieval South India ca. 9-11th centuries AD: Some old ideas and some new directions\fg, in Kaladarsana, Williams J. G. (ed.), New Delhi: Oxford and IBH, p. 91-98.}

\author{\textsc{Sethuraman}, N.}
\entry{1978} {\textit{The Imperial Pandyas. Mathematics Reconstructs the Chronology}, Kumbakonam: Raman \&\ Raman.}
\entry{1980} {\textit{Medieval Pandyas (A.D. 1000-1200)}, Kumbakonam: Raman \&\ Raman.}

\author{\textsc{Shackle}, C. et \textsc{Snell}, R.}
\entry{1992} {\textit{The Indian Narrative. Perspertives and Patterns}, Wiesbaden: Otto Harrassowitz.}

\author{\textsc{Shulman}, D. D.}
\entry{1980} {\textit{Tamil Temple Myths, Sacrifice and Divine Marriage in the South Indian Saiva Tradition}, Princeton University Press.}
\entry{1990} {\textit{Songs of the Harsh Devotee, the T\=ev\=aram of Cuntaram\=urttin\=aya\b n\=ar}, University of Pennsylvania.}
%\entry{1990} {\og Outcaste, Guardian, and Trickster: Notes on the Myth of Kattavarayan\fg, in Hiltebeitel A. (ed.), Criminal Gods and Demon Devotees, Manohar, p. 35-67.}
\entry{1993} {\textit{The Hungry God, Hindu Tales of Filicide and Devotion}, The University of Chicago Press.}
\entry{*2001} {[1993] \og From Author to Non-Author in Tamil Literary Legend\fg, dans \textit{The Wisdom of Poets. Studies in Tamil, Telugu and Sanskrit}, New Delhi: OUP, p. 103-128 (première publication dans \textit{Journal of Asian Studies} (Tiruvanmiyur), 10, 25 (1993), p. 1-23.}
\entry{2004} {\og Notes on Tillaikkalampakam\fg, dans \textit{South-Indian Horizons. Felicitation Volume for Fran\c cois Gros on the occasion of his 70th birthday}, éd. par Jean-Luc \textsc{Chevillard} et Eva \textsc{Wilden}, \textit{PDI} 94, Pondichéry: IFP/EFEO, p. 157-176.}

\author{\textsc{Singaravelu}, S.}
\entry{1968} {\og \textit{Theevaaram} Verses in Pallava-Chola-Grantha Script\fg, dans R.E. \textsc{Asher} (\'ed.), Proceedings of the Second International Conference Seminar of Tamil Studies 2, p. 70-78.}

\author{\textsc{Sircar}, D. C.}
\entry{1966} {\textit{Indian Epigraphical Glossary}, Delhi: Motilal Banarsidass.}

%\author{\textsc{Sivaprakasam}, C.K.}
%\entry{?} {\og Golaki School of Saivism under the Imperial C\=o\b las\fg, Tamil Civilization 3/2-3, p. 145-149.}

%\author{\textsc{Sivapur\=a\d na}}
%\entry{2000} {J. L. Shastri (ed.), translated by a board of scholars, 4 vols., Motilal Banarsidass, Delhi, réimp.}

\author{\textsc{Sivaramamurti}, C.}
%\entry{1955} {Royal Conquests and Cultural Migrations in South India and the Deccan, Calcutta.}
\entry{*1994} {[1974] \textit{Nataraja in art, thought and literature}, New Delhi: Publications Division, Ministry of Information and Broadcasting, Government of India.}

\author{\textsc{Smith}, D.}
\entry{1998} {\textit{The Dance of Siva, religion, art and poetry in South India}, New Delhi: Cambridge University Press.}

\author{\textsc{Sohnen}, Renate}
\entry{1995} {\og On the Concept and Presentation of "yamaka" in Early Indian Poetic Theory\fg, Bulletin of the School of Oriental and African Studies, University of London, vol. 58, no. 3, p. 495-520.}

\author{\textsc{Somasundaram}, P. S.}
\entry{1986} {\textit{Tirujñanasambandhar, philosophy and religion}, Madras: Vani Pathippakam.}

\author{\textsc{Soundra}, P.}
\entry{1979} {\textit{A study of St. Tirugnana Campantar\index{gnl}{Campantar}}, Annamalai University.}


\author{\textsc{Spencer}, George W.}
\entry{1968} {\og Temple Money-Lending and Livestock Redistribution in Early Tanjore\fg, \textit{IESHR} 5-3, p. 277-293.}
\entry{1969} {\og Religious Networks and Royal Influence in Eleventh Century South India\fg, \textit{JESHO} 12-1, p. 42-56.}
\entry{1970} {\og The Sacred Geography of the Tamil Shaivite Hymns\fg, \textit{Numen} 17, p. 232-244.}
\entry{1987} {\textit{Temples, Kings and Peasants : Perceptions of South India's Past} (éd.), Madras: New Era Publicatons.}

\author{\textsc{Srinivasan}, C. R.}
\entry{1979} {\textit{Kanchipuram through the ages}. Delhi: Agam Kala Prakashan.}

\author{\textsc{Srinivasan}, P. R.}
\entry{*1994} {[1963] \textit{Bronze of South India}, Madras: Government Museums Madras, Government of Tamil Nadu.}

\author{\textsc{Srinivasan}, P. R. \&\ \textsc{Reiniche}, Marie-Louise}
\entry{1990} {\textit{Tiruvannamalai: a \'Saiva sacred complex of South India}. 2 vols., \textit{PIFP} 75, Pondichéry : IFP.}

\author{\textsc{Stein}, Burton}
\entry{1960} {\og The Economic Function of a Medieval South Indian Temple\fg, \textit{JAS} 19-2, p. 163-176.}
\entry{1967-68} {\og Brahman and Peasant in Early South Indian History\fg, \textit{Adyar Library Bulletin, Dr.V. Raghavan Felicitation Volume} 31-32, p. 229-269.}
\entry{1968} {\og Social Mobility and Medieval South Indian Hindu Sects\fg, dans J. Silverberg (\'ed.), \textit{Social Mobility in The Caste System in India, an Interdisciplinary Symposium}, The Hague: Mouton, p. 78-94.}
%\entry{XXXXX} {\og Temples in Tamil Country, 1300-1750 A.D.\fg, \textit{IESHR} 14-1, p.11-45.}
\entry{1980} {\textit{Peasant state and society in medieval South India}, Delhi: OUP.}
\entry{*1997} {[1975]\og The State and the Agrarian Order in Medieval South India: A Historiographical Critique\fg, in STEIN B. (éd.), Essays on South India, Munshiram Manoharlal (1st edition: The University Press of Hawaii, 1975).}
\entry{1978} {(ed.), South Indian Temples, Vikas Publishing House Pvt Ltd, New Delhi.}



\author{\textsc{Subbarayalu}, Y.}
\entry{1973} {\textit{Political Geography of Chola Country}, Madras: State Departement of Archaeology, Government of Tamilnadu.}
\entry{*2001a} {[1983] \og A Side Light on Cola Officialdom\fg\ dans \textit{Srinidhih: Perspectives in Indian Archaeology, Art and Culture: K.R. Srinivasan Festschrift}, Madras: New Era Publication. Réimpression: \textsc{Rajagopal} S. (\'ed.), \textit{Kaveri. Studies in Epigraphy, Archaeology and History}, Chennai: Panpattu Veliyiittakam, p. 17-21.}
\entry{*2001b} {[?] \og Social Change and the Valangai and Idangai divisions\fg\ dans \textsc{Rajagopal} S. (\'ed.), \textit{Kaveri. Studies in Epigraphy, Archaeology and History}, Chennai: Panpattu Veliyiittakam, p. 22-30.}
\entry{*2001c} {[?] \og Land Measurement in Medieval Tamil Nadu\fg\ dans \textsc{Rajagopal} S. (\'ed.), \textit{Kaveri. Studies in Epigraphy, Archaeology and History}, Chennai: Panpattu Veliyiittakam, p. 31-40.}
\entry{*2001d} {\og The Sale Deeds in Cola inscriptions\fg\ dans \textsc{Rajagopal} S. (\'ed.), \textit{Kaveri. Studies in Epigraphy, Archaeology and History}, Chennai: Panpattu Veliyiittakam, p. 41-52.}
\entry{*2001e} {[1977] \og Classification of land and Assessment of Land tax\fg\ dans \textsc{Rajagopal} S. (\'ed.), \textit{Kaveri. Studies in Epigraphy, Archaeology and History}, Chennai: Panpattu Veliyiittakam, p. 53-59. Premi\`ere publication: \textit{Indian History Congress, Proceedings of 38th Session}, Bhuvanesvar, 1977.}
\entry{*2001f} {[1984] \og Kudimai: An aspect of the Cola Revenue System\fg\ dans \textsc{Rajagopal} S. (\'ed.), \textit{Kaveri. Studies in Epigraphy, Archaeology and History}, Chennai: Panpattu Veliyiittakam, p. 60-64. Premi\`ere publication: \textit{Indian History Congress, Proceedings of 45th Session}, Annamalai University, 1984.}
\entry{*2001g} {[?] \og Quantifying Cola land revenue Assessment\fg\ dans \textsc{Rajagopal} S. (\'ed.), \textit{Kaveri. Studies in Epigraphy, Archaeology and History}, Chennai: Panpattu Veliyiittakam, p. 65-81.}
\entry{*2001h} {[1982] \og The Cola State\fg\ dans \textsc{Rajagopal} S. (\'ed.), \textit{Kaveri. Studies in Epigraphy, Archaeology and History}, Chennai: Panpattu Veliyiittakam, p. 82-143. Premi\`ere publication: \textit{Studies in History}, vol. IV, n. 2, 1982.}
\entry{2002-3} {\textit{Tami\b lk kalve\d t\d tuc collakarati. Glossary of Tamil Inscriptions}, Ce\b n\b nai: Santi Sadhana.}

\author{\textsc{Subramaniam}, T. N.}
\entry{1957} {\textit{South Indian Temple Inscriptions}, Madras: Government Oriental Manuscripts Library.}

\author{\textsc{Subramanya Aiyar}, \textsc{Chevillard} \&\ \textsc{Sarma}}
\entry{2007} {\textsc{voir} \textit{Digital T\=ev\=aram}}

\author{\textsc{Subramoniam}, V. I.}
\entry{1962} {\textit{Index of Puranaanuuru}, Trivandrum: Department of Tamil, University of Kerala.}

%\author{\textsc{Suppirama\d niyam, P\=u.}}
%\entry{1983} {Meykk\=\i rttika\d l, Madras: International Institute of Tamil Studies.}

\author{\textsc{Swamy}, B. G. L.}
\entry{1972} {\og The Four Saivite Samayacaryas of The Tamil Country in Epigraphy\fg, \textit{Journal of Indian History} vol. L part I, p. 95-128.}
\entry{1975a} {\og The Golaki school of Saivism in the Tamil Country\fg, \textit{Journal of Indian History} LIII, Trivandrum, p. 167-209.}
\entry{1975b} {\og The date of the T\=ev\=aram trio: an analysis and re-appraisal\fg, \textit{Bulletin of the Institute of traditional Cultures}, January-June, p. 119-179.}

\author{\textsc{Talbot}, Cynthia}
\entry{1991} {\og Temples, Donors and Gifts: Patterns of Patronage in Thirteenth-Century South India\fg, \textit{JAS} 50-2, p. 308-340.}
%\entry{?} {Golaki Matha Inscriptions from Andhra Pradesh: A Study of a Saiva Monastic Lineage.}

\author{\textsc{Tamil Lexicon}}
\entry{*1982} {7 vols., réimp., University of Madras.}

%\author{\textsc{Thirunaranan} B. M.}
%\entry{1940} {\og The Traditional Limits and Subdivisions of the Tamil Region\fg, in K. V. Rangaswami Aiyangar Commemoration Volume, G. S. Press, Madras.}

\author{\textsc{Tieken}, Herman}
\entry{2001} {\textit{K\=avya in South India, Old Tamil Ca\.nkam Poetry}, Groningen: Egbert Forsten.}
\entry{2004} {\og The Nature of the Language of Ca\.nkam Poetry\fg, dans \textit{South-Indian Horizons. Felicitation Volume for Fran\c cois Gros on the occasion of his 70th birthday}, éd. par Jean-Luc \textsc{Chevillard} et Eva \textsc{Wilden}, \textit{PDI} 94, Pondichéry: IFP/EFEO, p.365-387.}
%\author{\textsc{Tiruva\d l\d luvar}}
%\entry{1992} {Le Livre de l'Amour, traduit du tamoul, présenté et annoté par F. Gros, Gallimard/Unesco, Paris.}

\author{\textsc{T\"orzs\"ok}, Judit}
\entry{2004} {\og \'Siva le fou et ses d\'evots tamouls dans le \textit{T\=ev\=aram}\fg, dans \textit{South-Indian Horizons. Felicitation Volume for Fran\c cois Gros on the occasion of his 70th birthday}, éd. par Jean-Luc \textsc{Chevillard} et Eva \textsc{Wilden}, \textit{PDI} 94, Pondichéry: IFP/EFEO, p. 3-28.}
%\entry{2004} {\og Three Chapters of \textit{\'Saiva} Material Added to the Earliest Known Recension of the Skandapur\=a\d na\fg, dans \textit{Origin and Growth of the Pur\=a\d nic Text Corpus}, Hans Bakker \'ed., Delhi: Motilal Banarsidass Publishers, p. 17-39.}

%\author{\textsc{Valmiki}}
%\entry{1998} {Ramaya?a, with Sanskrit text and English Translation, 2 vols., Gita Press, Gorakhpur.}

%\author{\textsc{Vamadeva}, Chandraleka}
%\entry{1995} {\textit{The concept of va\b n\b na\b npu (violent love) in Tamil Saivism, with special reference to the Periyapur\=a\d nam}, Suède: Uppsala University.}

\author{\textsc{Ve\d l\d laiv\=ara\d na\b n}, Ka.}
\entry{*1994} {[1962 et 1969] \textit{pa\b n\b niru tirumu\b rai varal\=a\b ru}, 2 vols., a\d n\d n\=amalai palkalaik-ka\b lakam.}

\author{\textsc{Veluppillai}, Uthaya}
\entry{2003a} {\textit{Le T\=ev\=aram à C\=\i k\=a\b li: Hymnes dévotionnels tamouls célébrant un lieu saint \'sivaïte}, D.E.A. sous la direction de Bruno \textsc{Dagens}, Université de la Sorbonne Nouvelle - Paris 3.}
\entry{2003b} {\og Au service\index{gnl}{service} des serviteurs: l'hospitalité dans le Periya Pur\=a\d nam\fg, \textit{BEI} 21.1, p. 99-130.}
\entry{2013} {\og Offrande d'ambroisie, note sur le terme \textit{amutu} dans le \textit{Periyapur\=a\d nam}\fg, \textit{BEI} 28-29, p. 379-384.}

\author{\textsc{Veluthat}, Kesavan}
\entry{1979} {\og The temple-base of the bhakti movement in South India\fg, dans Proceedings of the Indian History Congress, Waltair, p. 185-194.}
\entry{1985} {\og The Sabha and Parisad in Early Medieval South India: Correlation of Epigraphic and Dharmasastraic Evidences\fg, \textit{Tamil Civilisation} 3 2-3, p. 75-82.}
\entry{1993} {\textit{The Political Structure of Early Medieval South India}, New Delhi: Orient Longman.}

\author{\textsc{Vettam}, Mani}
\entry{*2002} {[1975] \textit{Pur\=a\d nic Encyclopaedia. A Comprehensive Work with Special Reference to the Epic and Pur\=a\d nic Literature}, Delhi: Motilal Banarsidass Publishers.}

\author{\textsc{Vijayavenugopal}, G.}
\entry{1999} {\og The rise and fall of a mahasabha of Tiruna\d l\d l\=a\b ru. A case study\fg, \textit{Journal of the Epigraphical Society of India} 25, p. 51-53.}
%\entry{1999} {\og Jainism in Tamilnadu as gleanedfrom Pallava Inscriptions\fg, PILC Journal of Dravidic Studies 9. 2, p. 189-195.}
\entry{2000} {\og From Hagiology to History: references from Tiruna\d l\d l\=a\b ru inscriptions\fg, \textit{Journal of the Epigraphical Society of India} 26, p. 188-194.}
\entry{2006-2010} {\textsc{voir} \textit{putucc\=eri m\=anilak kalve\d t\d tukka\d l. Pondicherry Inscriptions.}}


\author{\textsc{Whashbrook}, David}
\entry{1975} {\og Political Change in a Stable Society: Tanjore District 1880 to 1920\fg, dans \textit{South India: Politival Institutions and Political Change 1880-1940}, David \textsc{Whashbrook} et Christopher \textsc{Baker} éd., Delhi: The Macmillan Company of India Limited, p. 20-68.}

\author{\textsc{Wilden}, Eva}
\entry{2002} {\og Towards an Internal Chronology of Old Tamil Ca\.nkam literature or How to Trace the Laws of a Poetic Universe\fg, \textit{Wiener Zeitschrift f\"ur die Kunde s\"ud-undostasiens} 46, p. 105-133.}

%\author{\textsc{Wilson} H.H.}
%\entry{1828} {Descriptive Catalogue of the Oriental Manuscripts Collected by the Late Lieutenant-Colonel Colin Mackenzie, 3vols., Calcutta.}

\author{\textsc{Yocum}, Glenn E.}
\entry{1982} {\textit{Hymns to the dancing \'Siva: A Study of M\=a\d nikkav\=acakar\index{gnl}{Manikkavacakar@M\=a\d nikkav\=acakar}'s Tiruv\=acakam}. New Delhi: Heritage.}

\author{\textsc{Younger}, Paul}
\entry{1995} {\textit{The Home of Dancing \'Siva\b n: The Traditions of the Hindu Temple in Citamparam\index{gnl}{Citamparam}}. New York: OUP.}

\author{\textsc{Zvelebil}, Kamil V.}
\entry{1973} {\textit{The smile of Murukan on Tamil literature of South India}, Leiden: E.J. Brill.}
\entry{1974} {\textit{Tamil Literature}, dans \textit{A History of Indian Literature}, vol.X/1, Wiesbaden: Harrassowitz.}
\entry{1975} {\textit{Tamil Literature}, Leiden: E. J. Brill.}
\entry{1977} {\og The Beginnings of \textit{bhakti} in South India\fg, \textit{Temenos} 13, p. 223-257.}
%\entry{1986} {\og The Term 'Tami\b l'\fg, \textit{IJDL} 15-1, p. 1-10.}
\entry{1992} {\og Tamil sthalapur\=a\d nas\fg, Archív Orientální 60/2, p. 128-133.}
\entry{1995} {\textit{Lexicon of Tamil Literature}, Leiden: E. J. Brill.}
%\entry{1997} {\og Les id\'ees-piliers de la tradition linguistique tamoule\fg, \textit{JA} 285-1, p. 281-300.}

\newpage
\normalsize
\addcontentsline{toc}{chapter}{Annexes}
\newpage
\addcontentsline{toc}{section}{Liste des tableaux}
\listoftables
\newpage
\addcontentsline{toc}{section}{Table des figures}
\listoffigures
\newpage
\chapter*{CD du Corpus \'Epigraphique de C\=\i k\=a\b li}

\vspace*{1cm}

\begin{center}
(voir pochette à la fin de la thèse)
\end{center}

\addcontentsline{toc}{section}{CD du Corpus \'Epigraphique de C\=\i k\=a\b li}
\newpage
\addcontentsline{toc}{chapter}{Index}
\footnotesize
\printindex{gnl}{Index général}

\printindex{cec}{Index du Corpus \'Epigraphique de C\=\i k\=a\b li}

\newpage
\scriptsize
\thispagestyle{empty}

\begin{center}
\footnotesize\textbf{{\textsc{C\=\i k\=a\b li: hymnes, héros, histoire.}\\
Rayonnement d'un lieu saint shiva\"ite au Pays Tamoul}}
\end{center}

\textbf{
\begin{flushleft}
{Résumé}
\end{flushleft}
}

C\=\i k\=a\b li est le site le plus c\'el\'ebr\'e dans le \textit{T\=ev\=aram}, corpus de poèmes de la \textit{bhakti} shivaïte composés en tamoul dans la seconde moitié du premier millénaire: soixante-et-onze hymnes lui sont dédi\'es. Lieu de naissance de Campantar, un des trois auteurs du \textit{T\=ev\=aram}, C\=\i k\=a\b li aurait \'et\'e chant\'e, selon la tradition, sous douze toponymes diff\'erents.

Notre travail de type monographique porte sur l'histoire religieuse du site de C\=\i k\=a\b li qui n'a jamais été étudié alors qu'il représente un haut lieu de la tradition des textes de \textit{bhakti} shivaïte tamoule. Nos sources sont constituées de trois corpus textuels appartenant à trois genres différents de diverses périodes qui permettent de rendre compte du rayonnement continu de ce site:
le corpus du \textit{T\=ev\=aram} sur C\=\i k\=a\b li (partie \textsc{i}), généralement daté des \textsc{vii}\up{e}-\textsc{ix}\up{e} siècles,
le corpus des hagiographies sur Campantar (partie \textsc{ii}) attribuées à des poètes des \textsc{xi}\up{e}-\textsc{xii}\up{e} siècles,
et le corpus des inscriptions du temple de C\=\i k\=a\b li (partie \textsc{iii}) qui forme une documentation inédite du \textsc{xii}\up{e} au \textsc{xvi}\up{e} siècle.

\`A travers une approche \og archéologique\fg\ de ces sources qui permettent de reconstituer, de manière générale, l'histoire du site de C\=\i k\=a\b li, nous proposons une étude historique des textes du \textit{T\=ev\=aram} sur C\=\i k\=a\b li, nous retraçons l'histoire de la légende de l'enfant Campantar et nous éditons le corpus épigraphique de ce temple au rayonnement local.

\noindent
\textbf{Mots-clés:} C\=\i k\=a\b li, Campantar, Pays Tamoul, \textit{T\=ev\=aram}, temple, histoire.

\vspace*{0.5cm}

\begin{center}
\footnotesize\textbf{{\textsc{C\=\i k\=a\b li: hymns, heroes, history.}\\
Spread of a Shaiva sacred place in Tamilnad}}
\end{center}

\textbf{
\begin{flushleft}
{Abstract}
\end{flushleft}
}

C\=\i k\=a\b li is the most celebrated temple in the \textit{T\=ev\=aram}, a corpus of Shaiva \textit{bhakti} poems composed in Tamil in the second half of the first millennium: 71 hymns are dedicated to it. The birth place of Campantar, one of the three authors of the \textit{T\=ev\=aram}, C\=\i k\=a\b li has been praised, according to tradition, under 12 names.

Our monographic study deals with the religious history of the C\=\i k\=a\b li temple which has never been studied althought it is a highly traditional place for Tamil \textit{bhakti} texts. Our sources are three corpuses of different genres and periods which highlight the continuous spread of this site:
the \textit{T\=ev\=aram} corpus on C\=\i k\=a\b li (part \textsc{i}), which can be dated in the \textsc{vii}\up{th}-\textsc{ix}\up{th} centuries,
the hagiographical corpus on Campantar (part \textsc{ii}) attributed to poets of the \textsc{xi}\up{th}-\textsc{xii}\up{th} centuries,
and the unpublished epigraphical corpus of the C\=\i k\=a\b li temple (part \textsc{iii}) from the \textsc{xii}\up{th} to the \textsc{xvi}\up{th} century.

On the basis of our archaeological approach of these sources, we reconstruct the history of the C\=\i k\=a\b li temple. Further, we propose a historical study of the \textit{T\=ev\=aram} on C\=\i k\=a\b li, we investigate the history of the child Campantar's legend and we edit the epigraphical corpus of this localy spread site.

\noindent
\textbf{Keywords:} C\=\i k\=a\b li, Campantar, Tamilnad, \textit{T\=ev\=aram}, temple, history.

\vspace*{0.5cm}

\begin{flushleft}
{\textsc{\textbf{Université Sorbonne Nouvelle - Paris 3}}}

{ED 268 Langages et langues: description, théorisation, transmission}

{UFR Langues, littératures, cultures et sociétés étrangères}

{UMR 7528 Mondes iranien et indien}

Service des Doctorats. Centre Censier. 13, rue de Santeuil 75231 Paris Cedex 05
\end{flushleft}

\end{document}
